%
%
%    LaTeX  Macro File  /usr2/jcm/tex/macros.tex
%
%
\tracingonline=0 % shorter error messages (on screen)

%\long\def\comment#1{} % mulit-line comments
%\newcommand{\note}[1]{\fbox{#1}}

% font changes  (as function calls, scribe style)

\renewcommand{\i}[1]{{\it #1\/}}       % italics with space correction
%\newcommand{\emex}[1]{\/{\em #1}}   % emphasis in example, exercise, theorem, etc.
\renewcommand{\c}[1]{{\sc #1}}       % small caps (eliminates \c as cedilla)
%\newcommand{\r}[1]{{\rm #1}}            % for roman font in math mode
%\newcommand{\s}[1]{{\scr #1}}          % script (for use with tatex) 
\newcommand{\s}[1]{{\cal #1}}
\renewcommand{\b}[1]{{\bf #1}}           % bold face
\newcommand{\calg}[1]{{\cal #1}}          % caligraphic 

%\newcommand{\q}[1]{``#1''}    % matching quotes for in-line quotation

% numbered environments

%\newcounter{partcounter}
%\setcounter{partcounter}{0}
%\renewcommand{\part}[1]{\newpage \addtocounter{partcounter}{1}
%\noindent{\Large \bf Part \Roman{partcounter}. #1 } \\[1ex]}

\newtheorem{thm}{Theorem}[section]
\newtheorem{theorem}[thm]{Theorem}
\newtheorem{lemma}[thm]{Lemma}
\newtheorem{cor}[thm]{Corollary}
\newtheorem{corollary}[thm]{Corollary}
\newtheorem{claim}[thm]{Claim}
\newtheorem{prop}[thm]{Proposition}
\newtheorem{conj}[thm]{Conjecture}
\newtheorem{definition}[thm]{Definition}
\newtheorem{exercise}[thm]{Exercise}
\newtheorem{example}[thm]{Example}
\newtheorem{remark}[thm]{Remark}
\newtheorem{open}[thm]{Open Problem}

%\newcommand{\proof}{\\{\bf Proof.}\ }
\newenvironment{proof}{{\bf Proof. }}{\thmbox}

% axiom and inference rule (centered, in math mode,  with name at left)

\newcommand{\axiom}[2]
{\[ \hbox to \columnwidth
    { \rlap{$#2$} \hfil {$ #1 $} \hfil }
\]}

\newcommand{\infrule}[4]
{\[ \hbox to \columnwidth %\textwidth
    { \rlap{$#4$} \hfil $
      \frac {\strut\displaystyle #1 } {\strut\displaystyle #2 } \; \rlap{$#3$} \hfil $
      \hfil }
\]}
\newcommand{\infruletw}[4]
{\[ \hbox to \textwidth
    { \rlap{$#4$} \hfil $
      \frac {\strut\displaystyle #1 } {\strut\displaystyle #2 } \; \rlap{$#3$} \hfil $
      \hfil }
\]}

%\newcommand{\infrule}[4]
%{\[ \hbox to \columnwidth
%    { \rlap{$#4$} \hfil $
%      {{\displaystyle\strut #1}\over{\displaystyle\strut #2}}\quad\makebox[0pt][l]{\it #3} $
%      \hfil }
%\]}

% axiom and inference rule macros for use in tables, etc.
% presumed in math mode #1=axiom, #2=side condition
\newcommand{\Axiom}[2]{
{\displaystyle\strut #1}\qquad\makebox[0pt][l]{\it #2}
}
% presumed in math mode #1=top, #2=bottom, #3=side condition
\newcommand{\Infrule}[3]{
{{\displaystyle\strut #1}\over{\displaystyle\strut #2}}\;\mbox{\scriptsize$\bf #3$}
}

% sequence of #1's, numbered up to #2  (e.g., seq{x}{n} for x1, ..., xn   )

\newcommand{\seq}[2]{#1_{1} \ldots #1_{#2}}  
%\newcommand{\ie}{{\it i.e.}}
%\newcommand{\eg}{{\it e.g.}}
%\newcommand{\cf}{{\it c.f.\,}}

% common symbols

\newcommand{\union}{\cup}
\newcommand{\intersect}{\cap}
\newcommand{\subs}{\subseteq}
\newcommand{\el}{\in}
\newcommand{\nel}{\not\in}
\newcommand{\ns}{\emptyset}
\newcommand{\compose}{\circ}
\newcommand{\set}[2]{ \{\, #1 \,\mid\, #2 \,\}  } % set macro
\newcommand{\infinity}{\infty}
\newcommand{\pair}[1]{\langle #1 \rangle}
\newcommand{\tuple}[1]{\langle #1 \rangle}


\newcommand{\fa}{\forall}
\newcommand{\te}{\exists}
\newcommand{\imp}{\supset}
\renewcommand{\implies}{\supset}
\newcommand{\ts}{\vdash}
\newcommand{\dts}{\models}

\newcommand{\aro}{\mathord\rightarrow} % see pages 154-155 of TeX manual
\newcommand{\paro}{\rightharpoonup} 
\newcommand{\karo}{\mathop\Rightarrow} % see pages 154-155 of TeX manual
%\newcommand{\cross}{\times}
%\newcommand{\dlb}{\lbrack\!\lbrack}
%\newcommand{\drb}{\rbrack\!\rbrack}
\newcommand{\mean}[1]{\lbrack\!\lbrack #1 \rbrack\!\rbrack}
\newcommand{\lam}{\lambda}
\newcommand{\subst}[2]{{}[#1/#2]}
\renewcommand{\dot}{\mathrel{\bullet}}
\newcommand{\Dinf}{D_{\infty}}
\newcommand{\bottom}{\perp}

\mathcode`:="603A  % treat : as punctuation instead of relation in math mode
\mathchardef\colon="303A	% relation colon

\newcommand{\eqdef}{\mathrel{:=}}
\newcommand{\Dom}{\mathop{\rm dom}}
\newcommand{\Pow}{\mathop{\rm Pow}}

\newcommand{\aequiv}{\equiv_\alpha}
\newcommand{\baro}{\buildrel \beta \over \rightarrow}
\newcommand{\earo}{\buildrel \eta \over \rightarrow}
\newcommand{\red}{\rightarrow\!\!\!\!\rightarrow}
\newcommand{\backred}{\leftarrow\!\!\!\!\leftarrow}
\newcommand{\bred}{\buildrel \beta \over \red}
\newcommand{\ered}{\buildrel \eta \over \red}
\newcommand{\conv}{\leftrightarrow}
\newcommand{\beconv}{\buildrel {\beta, \eta} \over\leftrightarrow}

% lambda calculus abbreviations
\newcommand{\letdec}[3]{\b{let\ } #1 = #2 \b{\ in\ } #3}
\newcommand{\letrec}[3]{\b{letrec\ } #1 = #2 \b{\ in\ } #3}

\newcommand{\thmbox}
   {{\ \hfill\hbox{%
      \vrule width1.0ex height1.0ex
   }\parfillskip 0pt }}
\newcommand{\qed}{\thmbox}


% make single spacing

\newcommand{\singlespace}{\renewcommand{\baselinestretch}{1}\@normalsize}
\newcommand{\etal}{{\em et al.}}

% bycase command for definition by cases (pg 49 of LaTeX)

\newcommand{\bycase}[1]
	{\left\{ \begin{array}{ll}  #1  \end{array} \right. }


% make \cite put blanks after the comma (use in alpha style)

%\def\@citex[#1]#2{\if@filesw\immediate\write\@auxout{\string\citation{#2}}\fi
%  \def\@citea{}\@cite{\@for\@citeb:=#2\do
%    {\@citea\def\@citea{,\penalty100\hskip2.5pt plus1.5pt minus.8pt}%
%       \@ifundefined{b@\@citeb}{{\bf ?}\@warning
%       {Citation `\@citeb' on page \thepage \space undefined}}
%\hbox{\csname b@\@citeb\endcsname}}}{#1}}
