% Document Type: LaTeX
% Master File: tutorial.tex
\section{A Brief Tour of \pvs}
\label{system-tutorial}

In this section we introduce the system by developing a theory and
doing a simple proof.  This will introduce the most useful commands
and provide a glimpse into the philosophy behind \pvs.  You will get
the most out of this section if you are sitting in front of a
workstation (or terminal) with \pvs\ installed.  In the following we
assume familiarity with Sun Unix and \gnu.

Start by going to a \unix\ shell window and creating a working
directory (using {\tt mkdir}). Next, connect ({\tt cd}) to that
working directory and start up \pvs\ by typing {\tt pvs}.\footnote{You
may need to include a pathname, depending on where and how \pvs\ is
installed.} This command executes a shell script which runs \gnu,
loads the necessary \pvs\ \emacs\ extensions, and starts the \pvs\
lisp image as a subprocess.\footnote{All the \gnu\ (and X-Windows or
Emacstool) command line flags can be added to the {\tt pvs} command
and passed through as appropriate; the {\tt -q} flag inhibits loading
of the user's {\tt .emacs} initialization file, and should be used if
difficulties are encountered starting \pvs\ or if there appear to be
conflicts in keybindings.  Do {\em not\/} report errors to us unless
they can be reproduced when the {\tt -q} flag is used.} 
After a few moments, you should see the
welcome screen indicating the version of \pvs\ being run, the current
directory, and instructions for getting help.  You may be asked
whether you want to create a new context in the directory; answer {\tt
yes} unless it is the wrong directory or you don't have write
permission there, in which case you should answer {\tt no} and provide
an alternative directory when prompted.

\pvs\ uses \emacs\ as its interface by extending \emacs\ with \pvs\
functions, but all the underlying capabilities of \emacs\ are available.
Thus the user can read mail and news, edit non\pvs\ files, or execute
commands in a shell buffer in the usual way.

In the following, \pvs\ \emacs\ commands are given first in their long
form, followed by an alternative abbreviation and/or key binding in
parentheses.  For example, the command for proving in \pvs\ is given
as \ecmd{prove} (\ecmd{pr}, \key{C-c p}).  This command can be entered
by typing the {\tt Escape} key, then an {\tt x}\footnote{Many
keyboards provide a {\tt Meta} key (hence the {\tt M-} prefix), and
this may be used instead.  On the \sun 3, the {\tt Meta} key is
normally labeled {\tt Left} and on the \sun 4 ({\sc sparc}), it is
labeled $\Diamond$.  The {\tt Meta} key is like the shift key; to use
it simply hold the {\tt Meta} key down while typing another key.}
followed by {\tt prove} (or {\tt pr}) and the {\tt Return} key.
Alternatively, hold the {\tt Control} key down while typing a {\tt c},
then let go and type a {\tt p}.  The {\tt Return} key does not need to
be pressed when giving the key binding form.  In \pvs\ all commands
and abbreviations are preceded by a {\tt M-x}; everything else is a
key-binding.  In later sections we will refer to commands by their
long form name, without the {\tt M-x} prefix.  Some of the commands
prompt for a theory or \pvs\ file name and specify a default; if the
default is the desired theory or file, you can simply type the {\tt
Return} key.  Although the basic keyword commands described here are
preferred by most serious users, \pvs\ commands are also available as
menu selections if you are running under \emacs\ 19.

To begin, type \iecmd{pvs-help} ({\tt C-h p}) for an overview of the
commands available in \pvs\ (type {\tt q} to exit the help buffer).
To exit \pvs, use \iecmd{exit-pvs} (\key{C-x C-c}).

\pvs\ specifications consist of a number of files, each of which
contains one or more theories.  Theories may import other theories;
imported theories must either be part of the prelude (the standard
collection of theories built-in to PVS), or the files containing them
must be in the same directory.\footnote{\pvs\ does support soft links,
thus supporting a limited capability for reusing theories.}
Specification files in \pvs\ all have a {\tt .pvs} extension.  As
specifications are developed, their proofs are kept in files of the
same name with {\tt .prf} extensions.  The specification and proof
files in a given directory constitute a PVS {\em context\/}; \pvs\
maintains the state of a specification between sessions by means of
the {\tt .pvscontext} file.  The {\tt .pvscontext} and {\tt .prf}
files are not meant to be modified by the user.  Other files used or
created by the system will be described as needed.  You may move to a
different context (\ie\ directory) using the \ecmd{change-context}
command, which is analogous to the \unix\ {\tt cd} command.

Now let's develop a small specification:
\begin{pvsex}
sum: THEORY
 BEGIN
  n: VAR nat
  sum(n): RECURSIVE nat =
   (IF n = 0 THEN 0 ELSE n + sum(n - 1) ENDIF)
   MEASURE (LAMBDA n: n)
  closed_form: THEOREM sum(n) = (n * (n + 1))/2
 END sum
\end{pvsex}
%
This is a specification for summation of the first $n$ natural numbers

This simple theory has no parameters and contains three declarations.
The first declares {\tt n} to be a variable of type {\tt nat}, the
built-in type of natural numbers.  The next declaration is a recursive
definition of the function {\tt sum(n)}, whose value is the sum of the
first {\tt n} natural numbers.  Associated with this definition is a
{\em measure\/} function, following the {\tt MEASURE} keyword, which
will be explained below.\footnote{In this case, the measure is the
identity function, which could have been written simply as {\tt MEASURE
n}.} The final declaration is a formula which gives the closed form of
the sum.

\subsection{Creating the Specification}

The {\tt sum} theory may be introduced to the system in a number of
ways, all of which create a file with a {\tt .pvs}
extension,\footnote{The file does not have to be named {\tt sum.pvs}, it
simply needs the {\tt .pvs} extension.} which can be done by
\begin{enumerate}

\item using the {\tt M-x new-pvs-file} command (\ecmd{nf}) to create a new
\pvs\ file, and typing {\tt sum} when prompted.  Then type in the {\tt
sum} specification.

\item Since the file is included on the distribution tape in the {\tt
Examples/tutorial} subdirectory of the main \pvs\ directory, it can be
imported with the {\tt M-x import-pvs-file} command (\ecmd{imf}).  Use
the \ecmd{whereis-pvs} command to find the path of the main \pvs\
directory.

\item Finally, any external means of introducing a file with extension
{\tt .pvs} into the current directory will make it available to the
system; for example, using {\tt vi} to type it in, or {\tt cp} to copy
it from the {\tt Examples/tutorial} subdirectory.

\end{enumerate}
The first two alternatives display the specification in a buffer.
The third option requires an explicit request such as a built-in \gnu\
file command (like {\tt M-x find-file}, {\tt C-x C-f}), or the {\tt M-x
find-pvs-file} ({\tt M-x ff} or {\tt C-c C-f}) command.  The latter is
more useful when there are multiple specification files, as it supports
completion on just the specification files, ignoring other files that
you or the system have created in the directory.

\subsection{Parsing}

Once the {\tt sum} specification is displayed, it can be parsed with the
{\tt M-x parse} ({\tt M-x pa}) command, which creates the internal
abstract representation for the theory described by the specification.
If the system finds an error during parsing, an error window will pop up
with an error message, and the cursor will be placed in the vicinity of
the error.  If you didn't get an error, introduce one (say by
misspelling the {\tt VAR} keyword), then move the cursor somewhere else and
parse the file again (note that the buffer is automatically saved).  Fix
the error and parse once more.  In practice, the parse command is rarely
used, as the system automatically parses the specification when it needs
to.

\subsection{Typechecking}
\index{typecheck|(}

The next step is to typecheck the file by typing \ecmd{typecheck}
(\ecmd{tc}, \key{C-c t}), which checks for semantic errors, such as
undeclared names and ambiguous types.  Typechecking may build new files
or internal structures such as \tccs.  When {\tt sum} has been
typechecked, a message is displayed in the minibuffer indicating that
two \tccs\index{TCCs@\tccs|(} were generated.  These \tccs\ represent
{\em proof obligations\/} that must be discharged before the {\tt sum}
theory can be considered typechecked.  The proofs of the \tccs\
may be postponed indefinitely, though it is a good idea to view them to
see if they are provable.  \tccs\ can be viewed using the \ecmd{show-tccs}
command, the results of which are shown in Figure~\ref{sum-tccs} below.

\pvstheory{sum-tccs}{\tccs\ for Theory {\tt sum}}{sum-tccs}

The first \tcc\ is due to the fact that {\tt sum} takes an argument of
type {\tt nat}, but the type of the argument in the recursive call to
{\tt sum} is integer, since {\tt nat} is not closed under subtraction.
Note that the \tcc\ includes the condition {\tt NOT n = 0}, which holds
in the branch of the {\tt IF-THEN-ELSE} in which the expression
{\tt n - 1} occurs.

The second \tcc\ is needed to ensure that the function {\tt sum} is
total, \ie\ terminates.  \pvs\ does not directly support partial
functions, although its powerful subtyping mechanism allows \pvs\ to
express many operations that are traditionally regarded as partial.  The
measure function is used to show that recursive definitions are total by
requiring the measure to decrease with each recursive call.

These \tccs\ are trivial, and in fact can be discharged automatically
by using the \ecmd{typecheck-prove} (\ecmd{tcp}) command, which attempts
to prove all \tccs\ that have been generated.  (Try it).
\index{TCCs@\tccs|)}\index{typecheck|)}

\subsection{Proving}

We are now ready to try to prove the main theorem.  Place the cursor on
the line containing the {\tt closed\_form} theorem, and type
\ecmd{prove} (\ecmd{pr} or \key{C-c p}).  A new buffer will pop up, the
formula will be displayed, and the cursor will appear at the {\tt Rule?}
prompt, indicating that the user can interact with the prover.  The
commands needed to prove this theorem constitute only a very small
subset of the commands available to the prover; more details can be
found in the prover guide~\cite{PVS:prover}.

First, notice the display (reproduced below), which consists of a
single formula (labeled {\tt \{1\}}) under a dashed line.  This is a
{\em sequent\/}; formulas above the dashed lines are called {\em
antecedents\/} and those below are called {\em succedents\/}.  The
interpretation of a sequent is that the conjunction of the antecedents
implies the disjunction of the succedents.  Either or both of the
antecedents and succedents may be empty.\footnote{An empty antecedent
is equivalent to {\tt true}, and an empty succedent is equivalent to
{\tt false}, so if both are empty the sequent is unprovable.} In our
case, we are trying to prove a single succedent.

The basic objective of the proof is to generate a {\em proof tree\/} in
which all of the leaves are trivially true.  The nodes of the proof tree
are sequents, and while in the prover you will always be looking at an
unproved leaf of the tree.  The {\em current\/} branch of a proof is the
branch leading back to the root from the current sequent.  When a given
branch is complete (\ie\ ends in a true leaf), the prover automatically
moves on to the next unproved branch, or, if there are no more unproven
branches, notifies you that the proof is complete.

Now back to the proof.  We will prove this formula by induction on {\tt
n}.  To do this, type {\tt (induct "n")}.\footnote{\pvs\ expressions are
case-sensitive, and must be put in double quotes when they appear as
arguments in prover commands.} This is not an \emacs\ command, rather it
is typed directly at the prompt, including the parentheses.  This
generates two subgoals; the one displayed is the base case, where {\tt
n} is {\tt 0}.  To see the inductive step, type {\tt (postpone)}, which
postpones the current subgoal and moves on to the next unproved one.
Type {\tt (postpone)} a second time to cycle back to the original
subgoal (labeled {\tt closed\_form.1}).\footnote{Three extremely useful
\emacs\ key sequences to know here are \key{M-p}, \key{M-n}, and
\key{M-s}.  \key{M-p} gets the last input typed to the prover; further
uses of \key{M-p} cycle back in the input history.  \key{M-n} works in
the opposite direction.  To use \key{M-s}, type the beginning of a
command that was previously input, and type \key{M-s}.  This will get
the previous input that matches the partial input; further uses of
\key{M-s} will find earlier matches.  Try these key sequences out; they
are easier to use than to explain.}

To prove the base case, we need to expand the definition of {\tt sum},
which is done by typing {\tt (expand "sum")}.  After expanding the
definition of {\tt sum}, we send the proof to the \pvs\ decision
procedures, which automatically decide certain fragments of
arithmetic, by typing {\tt (assert)}.\footnote{The {\tt
assert} command actually does a lot more than decide arithmetical
formulas, performing three basic tasks:
\begin{itemize}\def\itemsep{0in}
\item it tries to prove the subgoal using the decision procedures.

\item it stores the subgoal information in an underlying database,
allowing automatic use to be made of it later.

\item it simplifies the subgoal, again utilizing the underlying decision
procedures.
\end{itemize}
These arithmetic and equality procedures are the main workhorses to
most \pvs\ proofs.  You should learn to use them effectively in a
proof.} 
This completes the proof of this subgoal, and the system moves on to
the next subgoal, which is the inductive step.

The first thing to do here is to eliminate the {\tt FORALL} quantifier.
This can most easily be done with the {\tt skolem!}\
command\footnote{The exclamation point differentiates this command from
the {\tt skolem} command, where the new constants have to be provided by
the user.}, which provides new constants for the bound variables.  To
invoke this command type {\tt (skolem!)} at the prompt.  The resulting
formula may be simplified by typing {\tt (flatten)}, which will break up
the succedent into a new antecedent and succedent.  The obvious thing to
do now is to expand the definition of {\tt sum} in the succedent.  This
again is done with the {\tt expand} command, but this time we want to
control where it is expanded, as expanding it in the antecedent will not
help.  So we type {\tt (expand "sum" +)}, indicating that we want to
expand {\tt sum} in the succedent.\footnote{We could also have specified
the exact formula number (here {\tt 1}), but including formula numbers
in a proof tends to make it less robust in the face of changes.  There
is more discussion of this in the prover guide~\cite{PVS:prover}.}

The final step is to send the proof to the \pvs\ decision procedures
by typing {\tt (assert)}.  The proof is now complete, the system may
ask whether to save the new proof, and whether to display a brief
printout of the proof.  You should answer {\tt yes} to these questions
just to see how they work.  After responding to these questions, the
buffer from which the \cmd{prove} command was issued is redisplayed if
necessary, and the cursor is placed on the formula that was just
proved.  The entire proof transcript is shown below.  Yours may be
different, depending on your window size and the timings involved.

{\smaller\smaller\smaller\begin{alltt}
     closed_form :  
     
       |-------
     \{1\}   (FORALL (n: nat): sum(n) = (n * (n + 1)) / 2)
     
     Rule? {\bf (induct "n")}
     Inducting on n,
     this yields  2 subgoals: 
     closed_form.1 :  
     
       |-------
     \{1\}   sum(0) = (0 * (0 + 1)) / 2
     
     Rule? {\bf (postpone)}
     Postponing closed_form.1.
     
     closed_form.2 :  
     
       |-------
     \{1\}   (FORALL (j: nat):
              sum(j) = (j * (j + 1)) / 2
                IMPLIES sum(j + 1) = ((j + 1) * (j + 1 + 1)) / 2)
     
     Rule? {\bf (postpone)}
     Postponing closed_form.2.
     
     closed_form.1 :  
     
       |-------
     \{1\}   sum(0) = (0 * (0 + 1)) / 2
     
     Rule? {\bf (expand "sum")}
     (IF 0 = 0 THEN 0 ELSE 0 + sum(0 - 1) ENDIF) 
     simplifies to 0
     Expanding the definition of sum,
     this simplifies to: 
     closed_form.1 :  
     
       |-------
     \{1\}   0 = 0 / 2
     
     Rule? {\bf (assert)}
     Simplifying, rewriting, and recording with decision procedures,
     
     This completes the proof of closed_form.1.
     
     closed_form.2 :  
     
       |-------
     \{1\}   (FORALL (j: nat):
              sum(j) = (j * (j + 1)) / 2
                IMPLIES sum(j + 1) = ((j + 1) * (j + 1 + 1)) / 2)
     
     Rule? {\bf (skolem!)}
     Skolemizing,
     this simplifies to: 
     closed_form.2  |-------
     \{1\}   sum(j!1) = (j!1 * (j!1 + 1)) / 2
             IMPLIES sum(j!1 + 1) = ((j!1 + 1) * (j!1 + 1 + 1)) / 2
     
     Rule? {\bf (flatten)}
     Applying disjunctive simplification to flatten sequent,
     this simplifies to: 
     closed_form.2 :  
     
     \{-1\}   sum(j!1) = (j!1 * (j!1 + 1)) / 2
       |-------
     \{1\}   sum(j!1 + 1) = ((j!1 + 1) * (j!1 + 1 + 1)) / 2
     
     Rule? {\bf (expand "sum" +)}
     (IF j!1 + 1 = 0 THEN 0 ELSE j!1 + 1 + sum(j!1 + 1 - 1) ENDIF) 
     simplifies to 1 + sum(j!1) + j!1
     Expanding the definition of sum,
     this simplifies to: 
     closed_form.2 :  
     
     [-1]   sum(j!1) = (j!1 * (j!1 + 1)) / 2
       |-------
     \{1\}   1 + sum(j!1) + j!1 = (2 + j!1 + (j!1 * j!1 + 2 * j!1)) / 2
     
     Rule? {\bf (assert)}
     Simplifying, rewriting, and recording with decision procedures,
     
     This completes the proof of closed_form.2.
     
     Q.E.D.
     
     
     Run time  = 5.62 secs.
     Real time = 58.95 secs.

\end{alltt}}

Note: The proof presented here is a low-level interactive one chosen
for illustrative purposes.  In practice, trivial theorems such as this
are handled automatically by the higher-level strategies of PVS.  This
particular theorem, for example, is proved automatically by the single
command {\tt (induct-and-simplify "n" :defs T)}.


\subsection{Status}

Now type \iecmd{status-proof-theory} (\ecmd{spt}) and you will see a
buffer which displays the formulas in {\tt sum} (including the \tccs),
along with an indication of their proof status.  This command is
useful to see which formulas and \tccs\ still require proofs.  Another
useful command is \iecmd{status-proofchain} (\ecmd{spc}), which
analyzes a given proof to determine its dependencies.  To use this,
go to the {\tt sum.pvs} buffer, place the cursor on the {\tt
closed\_form} theorem, and enter the command.  A buffer will pop up
indicating whether the proof is complete, and that it depends on the
\tccs\ and the {\tt nat\_induction} axiom.

\subsection[Generating \LaTeX]{Generating \BoldLaTeX}

In order to try out this section, you must have access to \LaTeX\ and a
\TeX\ previewer, such as {\tt vitex} or {\tt dvitool} (for \sunview), or
{\tt xdvi} (for X-windows).  Otherwise this section may be skipped.

Type \iecmd{latex-theory-view} (\ecmd{ltv}).  You will be prompted for
the theory name---type {\tt sum}, or just {\tt Return} if {\tt sum} is
the default.  You will then be prompted for the \TeX\ previewer name.
Either the previewer must be in your path, or the entire pathname must
be given.  This information will only be prompted for once per session,
after that \pvs\ assumes that you want to use the same previewer.

\begin{figure}[ht]
\begin{center}
\begin{boxedminipage}{\textwidth}
{\smaller\smaller% The following substitutions are from the file:
%   /home/owre/pvs.git/pvs-tex.sub
\def\munderscoretimestwofn#1#2{{#1 \times #2}}% How to print function m_times with arity (2)
\def\fmunderscoretimestwofn#1#2{{#1 \times #2}}% How to print function fm_times with arity (2)
\def\sigmaunderscoretimestwofn#1#2{{#1 \times #2}}% How to print function sigma_times with arity (2)
\def\generatedunderscoresubsetunderscorealgebraonefn#1{{{\cal A}(#1)}}% How to print function generated_subset_algebra with arity (1)
\def\generatedunderscoresigmaunderscorealgebraonefn#1{{{\cal S}(#1)}}% How to print function generated_sigma_algebra with arity (1)
\def\aeunderscoredecreasingotheronefn#1{{\pvsid{decreasing?}(#1)~\mbox{\it a.e.}}}% How to print function ae_decreasing? with arity (1)
\def\aeunderscoreincreasingotheronefn#1{{\pvsid{increasing?}(#1)~\mbox{\it a.e.}}}% How to print function ae_increasing? with arity (1)
\def\aeunderscoreconvergenceothertwofn#1#2{{#1 \longrightarrow #2~\mbox{\it a.e.}}}% How to print function ae_convergence? with arity (2)
\def\aeunderscoreeqothertwofn#1#2{{#1 = #2~\mbox{\it a.e.}}}% How to print function ae_eq? with arity (2)
\def\aeunderscoreleothertwofn#1#2{{#1 \leq #2~\mbox{\it a.e.}}}% How to print function ae_le? with arity (2)
\def\aeunderscoreposotheronefn#1{{#1> 0~\mbox{\it a.e.}}}% How to print function ae_pos? with arity (1)
\def\aeunderscorenonnegotheronefn#1{{#1 \geq 0~\mbox{\it a.e.}}}% How to print function ae_nonneg? with arity (1)
\def\aeunderscorezerootheronefn#1{{#1 = 0~\mbox{\it a.e.}}}% How to print function ae_0? with arity (1)
\def\xunderscorelttwofn#1#2{{#1 < #2}}% How to print function x_lt with arity (2)
\def\xunderscoreletwofn#1#2{{#1 \leq #2}}% How to print function x_le with arity (2)
\def\xunderscoreeqtwofn#1#2{{#1 = #2}}% How to print function x_eq with arity (2)
\def\xunderscoretimestwofn#1#2{{#1 \times #2}}% How to print function x_times with arity (2)
\def\xunderscoreaddtwofn#1#2{{#1 + #2}}% How to print function x_add with arity (2)
\def\xunderscorelimitonefn#1{{\pvsid{limit}(#1)}}% How to print function x_limit with arity (1)
\def\xunderscoresumonefn#1{{\sum #1}}% How to print function x_sum with arity (1)
\def\xunderscoresigmathreefn#1#2#3{{\sum_{#1}^{#2} #3}}% How to print function x_sigma with arity (3)
\def\xunderscoresuponefn#1{{\pvsid{sup}(#1)}}% How to print function x_sup with arity (1)
\def\xunderscoreinfonefn#1{{\pvsid{inf}(#1)}}% How to print function x_inf with arity (1)
\def\pointwiseunderscoreconvergesunderscoredowntoothertwofn#1#2{{#1 \searrow #2}}% How to print function pointwise_converges_downto? with arity (2)
\def\pointwiseunderscoreconvergesunderscoreuptoothertwofn#1#2{{#1 \nearrow #2}}% How to print function pointwise_converges_upto? with arity (2)
\def\pointwiseunderscoreconvergenceothertwofn#1#2{{#1 \longrightarrow #2}}% How to print function pointwise_convergence? with arity (2)
\def\convergesunderscoredowntoothertwofn#1#2{{#1 \searrow #2}}% How to print function converges_downto? with arity (2)
\def\convergesunderscoreuptoothertwofn#1#2{{#1 \nearrow #2}}% How to print function converges_upto? with arity (2)
\def\convergenceothertwofn#1#2{{#1 \longrightarrow #2}}% How to print function convergence? with arity (2)
\def\convergencetwofn#1#2{{#1 \longrightarrow #2}}% How to print function convergence with arity (2)
\def\crossunderscoreproducttwofn#1#2{{#1 \times #2}}% How to print function cross_product with arity (2)
\def\conjugateonefn#1{{\overline{#1}}}% How to print function conjugate with arity (1)
\def\cunderscoredivtwofn#1#2{{#1/#2}}% How to print function c_div with arity (2)
\def\cunderscoremultwofn#1#2{{#1\times#2}}% How to print function c_mul with arity (2)
\def\cunderscoresubtwofn#1#2{{#1-#2}}% How to print function c_sub with arity (2)
\def\cunderscorenegonefn#1{{-#1}}% How to print function c_neg with arity (1)
\def\cunderscoreaddtwofn#1#2{{#1+#2}}% How to print function c_add with arity (2)
\def\Imonefn#1{{\Im(#1)}}% How to print function Im with arity (1)
\def\Reonefn#1{{\Re(#1)}}% How to print function Re with arity (1)
\def\Etwofn#1#2{{\mathbb{E}(#1~|~#2)}}% How to print function E with arity (2)
\def\Eonefn#1{{\mathbb{E}(#1)}}% How to print function E with arity (1)
\def\Ptwofn#1#2{{\mathbb{P}(#1~|~#2)}}% How to print function P with arity (2)
\def\Ponefn#1{{\mathbb{P}(#1)}}% How to print function P with arity (1)
\def\xtwofn#1#2{{#1\times#2}}% How to print function x with arity (2)
\def\asttwofn#1#2{{#1\ast#2}}% How to print function ast with arity (2)
\def\minusonefn#1{{{#1}^{-}}}% How to print function minus with arity (1)
\def\plusonefn#1{{{#1}^{+}}}% How to print function plus with arity (1)
\def\astonefn#1{{{#1}^{\ast}}}% How to print function ast with arity (1)
\def\dottwofn#1#2{{#1\bullet#2}}% How to print function dot with arity (2)
\def\integralthreefn#1#2#3{{\int_{#1}^{#2} #3}}% How to print function integral with arity (3)
\def\integraltwofn#1#2{{\int_{#1} #2}}% How to print function integral with arity (2)
\def\integralonefn#1{{\int#1}}% How to print function integral with arity (1)
\def\normonefn#1{{\left||{#1}\right||}}% How to print function norm with arity (1)
\def\phionefn#1{{\pvssubscript{\phi}{#1}}}% How to print function phi with arity (1)
\def\infunderscoreclosedonefn#1{{\left(-\infty,~#1\right]}}% How to print function inf_closed with arity (1)
\def\closedunderscoreinfonefn#1{{\left[#1,~\infty\right)}}% How to print function closed_inf with arity (1)
\def\infunderscoreopenonefn#1{(-\infty,~#1)}% How to print function inf_open with arity (1)
\def\openunderscoreinfonefn#1{(#1,~\infty)}% How to print function open_inf with arity (1)
\def\closedtwofn#1#2{{\left[#1,~#2\right]}}% How to print function closed with arity (2)
\def\opentwofn#1#2{(#1,~#2)}% How to print function open with arity (2)
\def\sigmathreefn#1#2#3{{\sum_{#1}^{#2} #3}}% How to print function sigma with arity (3)
\def\sigmatwofn#1#2{{\sum_{#1} {#2}}}% How to print function sigma with arity (2)
\def\ceilingonefn#1{{\lceil{#1}\rceil}}% How to print function ceiling with arity (1)
\def\flooronefn#1{{\lfloor{#1}\rfloor}}% How to print function floor with arity (1)
\def\absonefn#1{{\left|{#1}\right|}}% How to print function abs with arity (1)
\def\roottwofn#1#2{{\sqrt[#2]{#1}}}% How to print function root with arity (2)
\def\sqrtonefn#1{{\sqrt{#1}}}% How to print function sqrt with arity (1)
\def\sqonefn#1{{\pvssuperscript{#1}{2}}}% How to print function sq with arity (1)
\def\expttwofn#1#2{{\pvssuperscript{#1}{#2}}}% How to print function expt with arity (2)
\def\opcarettwofn#1#2{{\pvssuperscript{#1}{#2}}}% How to print function ^ with arity (2)
\def\indexedunderscoresetsotherIIntersectiononefn#1{{\bigcap #1}}% How to print function indexed_sets.IIntersection with arity (1)
\def\indexedunderscoresetsotherIUniononefn#1{{\bigcup #1}}% How to print function indexed_sets.IUnion with arity (1)
\def\setsotherIntersectiononefn#1{{\bigcap #1}}% How to print function sets.Intersection with arity (1)
\def\setsotherUniononefn#1{{\bigcup #1}}% How to print function sets.Union with arity (1)
\def\setsotherremovetwofn#1#2{{(#2 \setminus \{#1\})}}% How to print function sets.remove with arity (2)
\def\setsotheraddtwofn#1#2{{(#2 \cup \{#1\})}}% How to print function sets.add with arity (2)
\def\setsotherdifferencetwofn#1#2{{(#1 \setminus #2)}}% How to print function sets.difference with arity (2)
\def\setsothercomplementonefn#1{{\overline{#1}}}% How to print function sets.complement with arity (1)
\def\setsotherintersectiontwofn#1#2{{(#1 \cap #2)}}% How to print function sets.intersection with arity (2)
\def\setsotheruniontwofn#1#2{{(#1 \cup #2)}}% How to print function sets.union with arity (2)
\def\setsotherstrictunderscoresubsetothertwofn#1#2{{(#1 \subset #2)}}% How to print function sets.strict_subset? with arity (2)
\def\setsothersubsetothertwofn#1#2{{(#1 \subseteq #2)}}% How to print function sets.subset? with arity (2)
\def\setsothermembertwofn#1#2{{(#1 \in #2)}}% How to print function sets.member with arity (2)
\def\opohtwofn#1#2{{#1\circ#2}}% How to print function O with arity (2)
\def\opdividetwofn#1#2{{\frac{#1}{#2}}}% How to print function / with arity (2)
\def\optimestwofn#1#2{{#1\times#2}}% How to print function * with arity (2)
\def\opdifferenceonefn#1{{-#1}}% How to print function - with arity (1)
\def\opdifferencetwofn#1#2{{#1-#2}}% How to print function - with arity (2)
\def\opplustwofn#1#2{{#1+#2}}% How to print function + with arity (2)
\begin{alltt}
\pvsid{sum}: \pvskey{THEORY}
 \pvskey{BEGIN}

  \(n\): \pvskey{VAR} \(\mathbb{N}\)\vspace*{\pvsdeclspacing}

  \pvsid{sum}\pvsid{(}\(n\)\pvsid{)}: \pvskey{RECURSIVE} \(\mathbb{N}\) \pvskey{=} \pvsid{(}\pvskey{IF} \(n\) \(=\) \(0\) \pvskey{THEN} \(0\) \pvskey{ELSE} \(\opplustwofn{n}{\pvsid{sum}\pvsid{(}\opdifferencetwofn{n}{1}\pvsid{)}}\) \pvskey{ENDIF}\pvsid{)}
     \pvskey{MEASURE} \pvsid{(}\(\lambda\) \(n\): \(n\)\pvsid{)}\vspace*{\pvsdeclspacing}

  \pvsid{closed\_form}: \pvskey{THEOREM} \pvsid{sum}\pvsid{(}\(n\)\pvsid{)} \(=\) \(\opdividetwofn{\pvsid{(}\optimestwofn{n}{\pvsid{(}\opplustwofn{n}{1}\pvsid{)}}\pvsid{)}}{2}\)\vspace*{\pvsdeclspacing}

 \pvskey{END} \pvsid{sum}\end{alltt}
}
\end{boxedminipage}
\end{center}
\caption{Theory {\tt sum}}\label{sum-plain}
\end{figure}

After a few moments the previewer will pop up displaying the {\tt sum}
theory, as shown in Figure~\ref{sum-plain}.  Note that {\tt LAMBDA}
has been translated as $\lambda$.  This and other translations are
built into \pvs; the user may also specify translations for keywords
and identifiers (and override those built-in) by providing a
substitution file, {\tt pvs-tex.sub}, which contains commands to
customize the \LaTeX\ output.  For example, if the substitution file
contains the three lines

{\smaller\smaller\begin{alltt}
    THEORY key 7 \verb|{\large\bf Theory}|
    sum    1   2 \verb|{\sum_{i = 0}^{#1} i}|
\end{alltt}}
the output will look like Figure~\ref{sum-sub}.

\begin{figure}[ht]
\begin{center}
\begin{boxedminipage}{\textwidth}
{\smaller\smaller% The following substitutions are from the file:
%   /home/owre/pvs.git/pvs-tex.sub
\def\munderscoretimestwofn#1#2{{#1 \times #2}}% How to print function m_times with arity (2)
\def\fmunderscoretimestwofn#1#2{{#1 \times #2}}% How to print function fm_times with arity (2)
\def\sigmaunderscoretimestwofn#1#2{{#1 \times #2}}% How to print function sigma_times with arity (2)
\def\generatedunderscoresubsetunderscorealgebraonefn#1{{{\cal A}(#1)}}% How to print function generated_subset_algebra with arity (1)
\def\generatedunderscoresigmaunderscorealgebraonefn#1{{{\cal S}(#1)}}% How to print function generated_sigma_algebra with arity (1)
\def\aeunderscoredecreasingotheronefn#1{{\pvsid{decreasing?}(#1)~\mbox{\it a.e.}}}% How to print function ae_decreasing? with arity (1)
\def\aeunderscoreincreasingotheronefn#1{{\pvsid{increasing?}(#1)~\mbox{\it a.e.}}}% How to print function ae_increasing? with arity (1)
\def\aeunderscoreconvergenceothertwofn#1#2{{#1 \longrightarrow #2~\mbox{\it a.e.}}}% How to print function ae_convergence? with arity (2)
\def\aeunderscoreeqothertwofn#1#2{{#1 = #2~\mbox{\it a.e.}}}% How to print function ae_eq? with arity (2)
\def\aeunderscoreleothertwofn#1#2{{#1 \leq #2~\mbox{\it a.e.}}}% How to print function ae_le? with arity (2)
\def\aeunderscoreposotheronefn#1{{#1> 0~\mbox{\it a.e.}}}% How to print function ae_pos? with arity (1)
\def\aeunderscorenonnegotheronefn#1{{#1 \geq 0~\mbox{\it a.e.}}}% How to print function ae_nonneg? with arity (1)
\def\aeunderscorezerootheronefn#1{{#1 = 0~\mbox{\it a.e.}}}% How to print function ae_0? with arity (1)
\def\xunderscorelttwofn#1#2{{#1 < #2}}% How to print function x_lt with arity (2)
\def\xunderscoreletwofn#1#2{{#1 \leq #2}}% How to print function x_le with arity (2)
\def\xunderscoreeqtwofn#1#2{{#1 = #2}}% How to print function x_eq with arity (2)
\def\xunderscoretimestwofn#1#2{{#1 \times #2}}% How to print function x_times with arity (2)
\def\xunderscoreaddtwofn#1#2{{#1 + #2}}% How to print function x_add with arity (2)
\def\xunderscorelimitonefn#1{{\pvsid{limit}(#1)}}% How to print function x_limit with arity (1)
\def\xunderscoresumonefn#1{{\sum #1}}% How to print function x_sum with arity (1)
\def\xunderscoresigmathreefn#1#2#3{{\sum_{#1}^{#2} #3}}% How to print function x_sigma with arity (3)
\def\xunderscoresuponefn#1{{\pvsid{sup}(#1)}}% How to print function x_sup with arity (1)
\def\xunderscoreinfonefn#1{{\pvsid{inf}(#1)}}% How to print function x_inf with arity (1)
\def\pointwiseunderscoreconvergesunderscoredowntoothertwofn#1#2{{#1 \searrow #2}}% How to print function pointwise_converges_downto? with arity (2)
\def\pointwiseunderscoreconvergesunderscoreuptoothertwofn#1#2{{#1 \nearrow #2}}% How to print function pointwise_converges_upto? with arity (2)
\def\pointwiseunderscoreconvergenceothertwofn#1#2{{#1 \longrightarrow #2}}% How to print function pointwise_convergence? with arity (2)
\def\convergesunderscoredowntoothertwofn#1#2{{#1 \searrow #2}}% How to print function converges_downto? with arity (2)
\def\convergesunderscoreuptoothertwofn#1#2{{#1 \nearrow #2}}% How to print function converges_upto? with arity (2)
\def\convergenceothertwofn#1#2{{#1 \longrightarrow #2}}% How to print function convergence? with arity (2)
\def\convergencetwofn#1#2{{#1 \longrightarrow #2}}% How to print function convergence with arity (2)
\def\crossunderscoreproducttwofn#1#2{{#1 \times #2}}% How to print function cross_product with arity (2)
\def\conjugateonefn#1{{\overline{#1}}}% How to print function conjugate with arity (1)
\def\cunderscoredivtwofn#1#2{{#1/#2}}% How to print function c_div with arity (2)
\def\cunderscoremultwofn#1#2{{#1\times#2}}% How to print function c_mul with arity (2)
\def\cunderscoresubtwofn#1#2{{#1-#2}}% How to print function c_sub with arity (2)
\def\cunderscorenegonefn#1{{-#1}}% How to print function c_neg with arity (1)
\def\cunderscoreaddtwofn#1#2{{#1+#2}}% How to print function c_add with arity (2)
\def\Imonefn#1{{\Im(#1)}}% How to print function Im with arity (1)
\def\Reonefn#1{{\Re(#1)}}% How to print function Re with arity (1)
\def\Etwofn#1#2{{\mathbb{E}(#1~|~#2)}}% How to print function E with arity (2)
\def\Eonefn#1{{\mathbb{E}(#1)}}% How to print function E with arity (1)
\def\Ptwofn#1#2{{\mathbb{P}(#1~|~#2)}}% How to print function P with arity (2)
\def\Ponefn#1{{\mathbb{P}(#1)}}% How to print function P with arity (1)
\def\xtwofn#1#2{{#1\times#2}}% How to print function x with arity (2)
\def\asttwofn#1#2{{#1\ast#2}}% How to print function ast with arity (2)
\def\minusonefn#1{{{#1}^{-}}}% How to print function minus with arity (1)
\def\plusonefn#1{{{#1}^{+}}}% How to print function plus with arity (1)
\def\astonefn#1{{{#1}^{\ast}}}% How to print function ast with arity (1)
\def\dottwofn#1#2{{#1\bullet#2}}% How to print function dot with arity (2)
\def\integralthreefn#1#2#3{{\int_{#1}^{#2} #3}}% How to print function integral with arity (3)
\def\integraltwofn#1#2{{\int_{#1} #2}}% How to print function integral with arity (2)
\def\integralonefn#1{{\int#1}}% How to print function integral with arity (1)
\def\normonefn#1{{\left||{#1}\right||}}% How to print function norm with arity (1)
\def\phionefn#1{{\pvssubscript{\phi}{#1}}}% How to print function phi with arity (1)
\def\infunderscoreclosedonefn#1{{\left(-\infty,~#1\right]}}% How to print function inf_closed with arity (1)
\def\closedunderscoreinfonefn#1{{\left[#1,~\infty\right)}}% How to print function closed_inf with arity (1)
\def\infunderscoreopenonefn#1{(-\infty,~#1)}% How to print function inf_open with arity (1)
\def\openunderscoreinfonefn#1{(#1,~\infty)}% How to print function open_inf with arity (1)
\def\closedtwofn#1#2{{\left[#1,~#2\right]}}% How to print function closed with arity (2)
\def\opentwofn#1#2{(#1,~#2)}% How to print function open with arity (2)
\def\sigmathreefn#1#2#3{{\sum_{#1}^{#2} #3}}% How to print function sigma with arity (3)
\def\sigmatwofn#1#2{{\sum_{#1} {#2}}}% How to print function sigma with arity (2)
\def\ceilingonefn#1{{\lceil{#1}\rceil}}% How to print function ceiling with arity (1)
\def\flooronefn#1{{\lfloor{#1}\rfloor}}% How to print function floor with arity (1)
\def\absonefn#1{{\left|{#1}\right|}}% How to print function abs with arity (1)
\def\roottwofn#1#2{{\sqrt[#2]{#1}}}% How to print function root with arity (2)
\def\sqrtonefn#1{{\sqrt{#1}}}% How to print function sqrt with arity (1)
\def\sqonefn#1{{\pvssuperscript{#1}{2}}}% How to print function sq with arity (1)
\def\expttwofn#1#2{{\pvssuperscript{#1}{#2}}}% How to print function expt with arity (2)
\def\opcarettwofn#1#2{{\pvssuperscript{#1}{#2}}}% How to print function ^ with arity (2)
\def\indexedunderscoresetsotherIIntersectiononefn#1{{\bigcap #1}}% How to print function indexed_sets.IIntersection with arity (1)
\def\indexedunderscoresetsotherIUniononefn#1{{\bigcup #1}}% How to print function indexed_sets.IUnion with arity (1)
\def\setsotherIntersectiononefn#1{{\bigcap #1}}% How to print function sets.Intersection with arity (1)
\def\setsotherUniononefn#1{{\bigcup #1}}% How to print function sets.Union with arity (1)
\def\setsotherremovetwofn#1#2{{(#2 \setminus \{#1\})}}% How to print function sets.remove with arity (2)
\def\setsotheraddtwofn#1#2{{(#2 \cup \{#1\})}}% How to print function sets.add with arity (2)
\def\setsotherdifferencetwofn#1#2{{(#1 \setminus #2)}}% How to print function sets.difference with arity (2)
\def\setsothercomplementonefn#1{{\overline{#1}}}% How to print function sets.complement with arity (1)
\def\setsotherintersectiontwofn#1#2{{(#1 \cap #2)}}% How to print function sets.intersection with arity (2)
\def\setsotheruniontwofn#1#2{{(#1 \cup #2)}}% How to print function sets.union with arity (2)
\def\setsotherstrictunderscoresubsetothertwofn#1#2{{(#1 \subset #2)}}% How to print function sets.strict_subset? with arity (2)
\def\setsothersubsetothertwofn#1#2{{(#1 \subseteq #2)}}% How to print function sets.subset? with arity (2)
\def\setsothermembertwofn#1#2{{(#1 \in #2)}}% How to print function sets.member with arity (2)
\def\opohtwofn#1#2{{#1\circ#2}}% How to print function O with arity (2)
\def\opdividetwofn#1#2{{\frac{#1}{#2}}}% How to print function / with arity (2)
\def\optimestwofn#1#2{{#1\times#2}}% How to print function * with arity (2)
\def\opdifferenceonefn#1{{-#1}}% How to print function - with arity (1)
\def\opdifferencetwofn#1#2{{#1-#2}}% How to print function - with arity (2)
\def\opplustwofn#1#2{{#1+#2}}% How to print function + with arity (2)
% The following substitutions are from the file:
%   /home/owre/pvs.git/doc/wift95/pvs-tex.sub
\def\sumonefn#1{{\sum_{i = 0}^{#1} i}}% How to print function sum with arity (1)
\begin{alltt}
\pvsid{sum}: \({\large\bf Theory}\)
 \pvskey{BEGIN}

  \(n\): \pvskey{VAR} \(\mathbb{N}\)\vspace*{\pvsdeclspacing}

  \(\sumonefn{n}\): \pvskey{RECURSIVE} \(\mathbb{N}\) \pvskey{=} \pvsid{(}\pvskey{IF} \(n\) \(=\) \(0\) \pvskey{THEN} \(0\) \pvskey{ELSE} \(\opplustwofn{n}{\sumonefn{\opdifferencetwofn{n}{1}}}\) \pvskey{ENDIF}\pvsid{)}
     \pvskey{MEASURE} \pvsid{(}\(\lambda\) \(n\): \(n\)\pvsid{)}\vspace*{\pvsdeclspacing}

  \pvsid{closed\_form}: \pvskey{THEOREM} \(\sumonefn{n}\) \(=\) \(\opdividetwofn{\pvsid{(}\optimestwofn{n}{\pvsid{(}\opplustwofn{n}{1}\pvsid{)}}\pvsid{)}}{2}\)\vspace*{\pvsdeclspacing}

 \pvskey{END} \pvsid{sum}\end{alltt}
}
\end{boxedminipage}
\end{center}
\caption{Theory {\tt sum}}\label{sum-sub}
\end{figure}

Finally, using the \iecmd{latex-proof} command, it is possible to
generate a \LaTeX\ file from a proof.  A part of an example is shown
below; details are in the PVS system manual.

\noindent
\begin{boxedminipage}{\linewidth}
\def\sumonefn#1{{\sum_{i = 0}^{#1} i}}

Expanding the definition of sum

{\tt closed\_form.2:}

\vspace*{0.2in}\hspace*{0.2in}
\begin{tabular}{ll}
$\{-1\}$ &\begin{minipage}[t]{6in}{ \begin{program} 
 \sumonefn { j^{\prime} } \mbox{ }=\mbox{ }(\ii j^{\prime} \mbox{ }\times\mbox{ }(\ii j^{\prime} \mbox{ }+\mbox{ } 1)) \mbox{ }/\mbox{ } 2 \\ 
\oo\oo\zi\zi\zi\zi\zo\zo \end{program}}\end{minipage}\\\hline
$\{1\}$ &\begin{minipage}[t]{6in}{ \begin{program} 
(\ii \pvskey{IF\mbox{ }} j^{\prime} \mbox{ }+\mbox{ } 1 \mbox{ }=\mbox{ } 0 \pvskey{\mbox{ }THEN\mbox{ }} 0 \pvskey{\mbox{ }ELSE\mbox{ }} j^{\prime} \mbox{ }+\mbox{ } 1 \mbox{ }+\mbox{ } \sumonefn { j^{\prime} \mbox{ }+\mbox{ } 1 \mbox{ }-\mbox{ } 1 } \pvskey{\mbox{ }ENDIF}) \\
\oo\zi \mbox{ }=\mbox{ }(\ii(\ii j^{\prime} \mbox{ }+\mbox{ } 1) \mbox{ }\times\mbox{ }(\ii j^{\prime} \mbox{ }+\mbox{ } 1 \mbox{ }+\mbox{ } 1)) \mbox{ }/\mbox{ } 2 \\ 
\oo\oo\oo\zi\zi\zi\zo\zo \end{program}}\end{minipage}\\
\end{tabular}
\end{boxedminipage}
