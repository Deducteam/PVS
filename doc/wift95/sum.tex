% The following substitutions are from the file:
%   /homes/helium/pvs/pvs-tex.sub
\def\membertwofn#1#2{{#1 \in #2}}% How to print function member with arity (2)
 \begin{program} 
 \pvsid{sum} :\mbox{ } \pvskey{THEORY} \\
\zi \pvskey{BEGIN} \\
 \\[-0.2in]
 n :\mbox{ } \pvskey{VAR\mbox{ }} \pvsid{nat} \\
 \\[-0.2in]
 \pvsid{sum}(\ii n) :\mbox{ } \pvskey{RECURSIVE\mbox{ }} \pvsid{nat } \mbox{ }= \\
\oo\zi\zi( \pvskey{IF\mbox{ }} n \mbox{ }=\mbox{ } 0 \pvskey{\mbox{ }THEN\mbox{ }} 0 \pvskey{\mbox{ }ELSE\mbox{ }} n \mbox{ }+\mbox{ } \pvsid{sum}(\ii n \mbox{ }-\mbox{ } 1) \pvskey{\mbox{ }ENDIF}) \\
\oo\zi\zi\zi\zo\zo \pvskey{MEASURE\mbox{ }}( \lambda\mbox{ }\ii n :\mbox{ } n) \\
\oo\zo\zo\zo \\[-0.2in]
 \pvsid{closed\_form} :\mbox{ } \pvskey{THEOREM\mbox{ }} \pvsid{sum}(\ii n) \mbox{ }=\mbox{ }( n \mbox{ }\times\mbox{ }( n \mbox{ }+\mbox{ } 1)) \mbox{ }/\mbox{ } 2 \\
\oo\zi\zi\zo\zo \\[-0.2in]
 \pvskey{END\mbox{ }} \pvsid{sum}
 \end{program}
