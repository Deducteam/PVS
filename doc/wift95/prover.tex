
\section{The PVS Proof Checker}

The PVS Proof Checker is also referred to as an interactive theorem
prover.  It is much more automated than the low-level ``proof
editors'' that support some specification notations, but operates
under the user's direction and is therefore more controllable than
purely automatic theorem provers.


\subsection{Introduction}

Just as we execute programs to check if they return the desired result,
we subject high-level functional descriptions of a system to challenges
by demanding proofs of desirable properties.  We call such challenges
{\em putative theorems}.  Here are some simple examples:
\begin{itemize}
  \item If a function that reverses a list has been
correctly specified, then we should be able to prove that we get the
original list by reversing a list twice.

  \item When a train is allowed into a railroad crossing the gates must
be down.  

  \item If the operational semantics is correct, then it should conform
to the denotational semantics.
\end{itemize}
%
The point of these challenges is that the process of proving putative
theorems quickly highlights the gaps, errors, and inadequacies in the
functional description.  In some cases, such a proof could also prove
that the system as described meets its complete specification, in other
words, that it is {\em correct\/}.  But it is the rare proof that
succeeds.  The typical proof attempt fails, but such failure usually
yields valuable insights that can be used to correct oversights in the
specification or formulation of the challenge.  A useful automated proof
assistant must therefore play the role of an intelligent but implacable
skeptic in rejecting any argument that is not entirely watertight.
Furthermore, in rejecting these arguments, such a skeptic must pinpoint
the source of the failure so that the argument can be corrected and the
dialogue resumed.  The PVS proof checker is intended to serve as the
skeptical party in such a dialogue.  The user supplies the steps in the
argument and PVS applies them to the goal of the proof progressively
breaking them into simpler subgoals or to obvious truths or falsehoods.
If all of the subgoals are reduced to obvious truths, the proof attempt
has succeeded.  Otherwise, the proof attempt fails either because the
argument or the conjecture is incorrect.

The central design assumptions in PVS are therefore that
\begin{itemize}
  \item The purpose of an automated proof checker is not merely to prove
theorems but also to provide useful feedback from failed and partial proofs
by serving as a rigorous skeptic.
  \item The most straightforward mechanizable criterion for a rigorous
argument is that of a formal proof.
  \item Automation serves to minimize the tedious aspects of
formal reasoning  while maintaining a high level of accuracy in the
book-keeping and formal manipulations.
  \item Automation should also be used to capture repetitive patterns
of argumentation.
  \item The end product of a proof attempt should be a proof that, with
only a small amount of work, can be made humanly readable so that it can
be subjected to the {\em social process\/} of mathematical scrutiny.
\end{itemize}
In following these design assumptions, the PVS proof checker 
is more automated than a low-level proof checker such as
AUTOMATH~\cite{deBruijn80}, LCF~\cite{LCF}, Nuprl~\cite{Constable86},
Coq~\cite{Coquand-Huet85}, and HOL~\cite{Gordon:HOL}, but provides more
user control over the structure of the proof than highly
automated systems such as Nqthm~\cite{Boyer-Moore79,boyer-moore88} and
Otter~\cite{Otter90}.  We feel that the low-level systems over-emphasize
the formal correctness of proofs at the expense of their cogency, and
the highly automated systems emphasize theorems at the expense of their
proofs.

What is unusual about PVS is the extent to which aspects of the
language, the typechecker, and proof checker are intertwined.  The
typechecker invokes the proof checker in order to discharge proof
obligations that arise from typechecking expressions involving predicate
subtypes or dependent types.  The proof checker also makes heavy use of
the typechecker to ensure that all expressions involved in a proof are
well-typed.  This use of the typechecker can also generate proof
obligations that are either discharged automatically or are presented as
additional subgoals.  Several aspects of the language, particularly the
type system, are built into the proof checker.  These include the
automatic use of type constraints by the decision procedures, the
simplifications given by the abstract datatype axioms, and forms of
beta-reduction and extensionality, Another less unusual aspect of PVS is
the extent to which decision procedures involving equalities and linear
arithmetic inequalities are employed.\footnote{The Ontic
system~\cite{mcallester89} is a proof checker where decision procedures
are ubiquitously used.  Nqthm~\cite{Boyer-Moore79,boyer-moore88},
Eves~\cite{Pase-Saaltink}, and IMPS~\cite{IMPS91} also rely heavily on
the use of decision procedures.} The most direct consequence of this is
that the trivial, obvious, or tedious parts of the proof are often
entirely hidden from the displayed proof so that the user can focus on
the intellectually demanding parts of the proof, and the resulting proof
is also easier to read.

As with much else in PVS, the implementation philosophy of the proof
checker has been guided by the 80-20 rule, namely that 80\% of the
functionality of a nearly perfect system can be built with 20\% of the
effort, and the remaining 20\% of the functionality can take up the
remaining 80\% of the effort.  PVS attempts to provide much of the 80\%
of the functionality that is easily implemented.  Each PVS proof
commands performs the function that, in our experience, is typically
required of it.  To some reasonable extent, the less typical
functionality can be obtained by providing optional arguments to these
proof commands.  In atypical instances, the burden of carrying out some
manipulation falls squarely on the user.  Even in these instances, it is
not too tedious to achieve one's ends with the existing proof commands
in some fairly simple ways.  The reader should let us know if any of our
design decisions are found to be ill-considered.

In order to learn how to use the PVS proof checker, one must first
understand the sequent representation used by PVS to represent proof
goals, the commands used to move around and undo parts of the proof
tree, and the commands used to get help.  One must then understand the
syntax and effects of proof commands used to build proofs.  Many of these
commands are extremely powerful even in their simplest usage.  Several
of these commands can be more carefully directed by supplying them with
one or more optional arguments.  The advanced user will also need to
understand how to define proof strategies that capture repetitive
patterns of proof commands, and commands used for displaying, editing,
and replaying proofs.

Section~\ref{prelims} provides the basic information needed to get
started with the PVS proof checker.  The remaining sections give a
collection of typical examples of how the proof checker is used.
The PVS Proof Checker Reference
Manual~\cite{PVS:prover} contains detailed descriptions of the PVS
proof commands.

\subsection{Preliminaries}\label{prelims}

\paragraph{Sequent Representation of Proof Goals.}
Each goal or subgoal in a PVS proof attempt is a sequent
of the form $\Gamma\vdash\Delta$, where $\Gamma$ is a sequence of
{\em antecedent\/} formulas and $\Delta$ is a 
sequence of {\em consequent\/} formulas.  The actual displayed form of a
PVS sequent is
\begin{center}
\begin{tabular}{ll}
  {\tt \{-1\}} & $A_1$\\
  {\tt \{-2\}} & $A_2$\\
  {\tt [-3]} & $A_3$\\
 & \vdots\\
 \multicolumn{2}{l}{\tt |-------}\\
  {\tt [1]} & $B_1$\\
  {\tt \{2\}} & $B_2$\\
  {\tt \{3\}} & $B_3$\\
 & \vdots
\end{tabular}
\end{center}
where each $A_i$ is an antecedent formula and each $B_i$ is a consequent
formula.  The intuitive reading of such a sequent is as the formula
$$(A_1\wedge A_2 \wedge A_3 \wedge \ldots ) \supset (B_1 \vee B_2 \vee
B_3 \vee \ldots).$$ Note that the antecedent formulas are numbered with
negative integers and the consequent formulas with positive integers.
These numberings are used in directing the PVS proof commands. If a
formula number $n$ appears as {\tt [$n$]} in the sequent, it is an
indication that the formula was unaffected by the proof step that
created the sequent.  It is a good heuristic is to examine the new
formulas (\ie\ the formulas whose number appears as {\tt \{$n$\}}) in
the sequent to formulate the next proof step.


\paragraph{Starting and Quitting Proofs. }

As indicated earlier, the PVS Emacs command {\tt M-x pr} initiates a
proof with the cursor on the formula to be proved.   This brings up the
{\tt *pvs*} buffer with the goal sequent and a {\tt Rule?}\ prompt.
Typing the PVS Emacs command {\tt M-x help-pvs-prover} brings up
help on the prover commands.
To quit out of an existing proof attempt, type {\tt q} or {\tt
quit} at the {\tt Rule?}\ prompt.   You will be asked whether you wish to
save the partial proof.  Remember that if you answer {\tt yes}, the old
proof will be overwritten, and if you answer {\tt no}, you will lose the
partial proof that you have developed up to this point.


\comment{The main steps in initiating a proof are to: 
\begin{enumerate}
  \item Set up a working directory for each specification.
  \item Copy any {\tt .pvs} files you need into this directory.
  \item Start \pvs\ by invoking ``{\tt pvs}'' (provided you have the
relevant pvs directory in your path.
  \item To exit \pvs, type {\tt C-x C-c} in Emacs.
  \item {\tt M-x cc} to change the context to your working directory.
  \item {\tt M-x nf} starts a new pvs file called {\tt \bkt{\em
name}.pvs}, where \bkt{\em name} is your response to the prompt.
  \item {\tt M-x ps} parses a buffer, {\tt M-x tc} typechecks a buffer,
and {\tt M-x tcp} typechecks a buffer with proof attempts on the \tccs.
Each PVS command does 
the parsing and typechecking of the buffer as needed.  {\tt M-x ppe} shows the
expanded and pretty printed version of the buffer with all the \tccs\ and
datatype axioms.
  \item {M-x pr} on a declaration starts a proof attempt on the declaration.
If the formula has been proved, you are asked whether you wish to
continue the proof attempt.  If there is an existing proof, you are
queried if you wish to rerun that proof.
  \item Proof commands are typed in response to the ``{\tt Rule?}'' prompt.
  \item Typing {\tt (help)} prints a brief line describing each of the
proof commands.  Typing {\tt (help {\em command-name})} prints the
information for a specific proof command.
\end{enumerate}
}


Since PVS proof construction is carried out in a Lisp buffer, there is a
small chance that you could find yourself at a Lisp breakpoint with a
`{\tt ->}' prompt.  Typing {\tt (restore)} at this point should almost
always  take you back to the nearest sensible proof goal and a  {\tt
Rule?}\ prompt.   


\paragraph{The Structure of PVS Proofs. }
In the course of a proof, PVS builds up a tree of sequents where
each sequent is a subgoal generated from its parent sequent by a PVS
proof command.  At any point in a proof attempt, the control is at a
leaf sequent of such a proof tree.  At this point a PVS proof command
can either 
\begin{itemize}
  \item  cause control to be transferred to next proof sequent in the
tree ({\tt postpone})
  \item undo a subtree by causing control to move up to some ancestor node
in the proof tree ({\tt undo})
  \item prove the {\em current sequent\/} causing control to move to the
next remaining leaf sequent in the tree
  \item generate subgoals so that control moves to the first of these
subgoals, or
  \item leave the proof tree unchanged while providing some useful status information.
\end{itemize}
A proof is completed when there are no remaining unproved leaf sequents
in the proof tree.  The resulting proof is saved and can be edited and
rerun on the same or a different conjecture.


\subsection{Using the Proof Checker}\label{using}

\subsubsection{Propositional Proof Commands}

Now that we have gotten past the preliminaries, we can look at
examples of some simple interactions with the PVS proof checker.
We start with the following PVS theory named {\tt propositions}
that declares three Boolean constants {\tt A}, {\tt B}, and {\tt C}, and
states a theorem named {\tt prop} asserting that the conjunction of
$( \mbox{\tt A} \supset ( \mbox{\tt B} \supset \mbox{\tt C}))$ and $(
\mbox{\tt A} \supset \mbox{\tt B})$ and $ \mbox{\tt A}$ implies $
\mbox{\tt C}$.

\begin{pvsscript}
propositions : THEORY
  BEGIN

  A, B, C: bool

  prop: THEOREM (A IMPLIES (B IMPLIES C)) AND (A IMPLIES B) AND A
               IMPLIES C

  END propositions
\end{pvsscript}

The proof script displayed below is the result of typing the PVS Emacs
command {\tt M-x pr} on the formula {\tt prop} and typing the inputs
(shown in bold-face) in response to the {\tt Rule?}\ prompt or to other
queries from PVS.  The {\tt (flatten)} command eliminates the
disjunctive connectives in the formula so as to flatten the formula out
into the sequent.  The next proof command {\tt (split)} picks the first
available conjunctive formula, in this case {\tt (A IMPLIES (B IMPLIES
C))}, and generates the three subgoals resulting from the conjunctive
splitting of this formula.  PVS then observes that the first of these
subgoals is trivially true since it has {\tt C} in both the antecedent
and consequent.  The {\tt (split)} command applied to the second subgoal
generates two further subgoals which are both recognized as being
trivially true, as is the remaining subgoal from the earlier {\tt
(split)} command.  The proof has now been successfully completed
generating the {\tt Q.E.D.} message, and  the new proof is automatically
saved.  The system inquires whether the user would like to see an
abbreviated version of the proof which is then printed out following the
{\tt yes} response.  For space reasons, we only display a few lines of
this printout in the script below.  The two timings printed out at the
end provide the machine time and the human time for the
proof attempt,  respectively.  The Emacs command {\tt M-x
show-last-proof} can be used to bring up an abbreviated version
of the most recently completed proof that can be used as a guide
in developing an informal presentation of the proof.  It displays
the sequents at the branch points in the proof and the commentary in
between.  

\begin{pvsscript}
     prop :   

       |-------
     \{1\}   (A IMPLIES (B IMPLIES C)) AND (A IMPLIES B) AND A IMPLIES C

     Rule? {\bf (flatten)}
     Applying disjunctive simplification to flatten sequent,
     this simplifies to: 
     prop :   

     \{-1\}   (A IMPLIES (B IMPLIES C))
     \{-2\}   (A IMPLIES B)
     \{-3\}   A
       |-------
     \{1\}   C

     Rule? {\bf (split)}
     Splitting conjunctions,
     this yields  3 subgoals: 
     prop.1 :   

     \{-1\}   C
     [-2]   (A IMPLIES B)
     [-3]   A
       |-------
     [1]   C

     which is trivially true.

     This completes the proof of prop.1.

     prop.2 :   

     [-1]   (A IMPLIES B)
     [-2]   A
       |-------
     \{1\}   B
     [2]   C

     Rule? {\bf (split)}
     Splitting conjunctions,
     this yields  2 subgoals: 
     prop.2.1 :   

     \{-1\}   B
     [-2]   A
       |-------
     [1]   B
     [2]   C

     which is trivially true.

     This completes the proof of prop.2.1.

     prop.2.2 :   

     [-1]   A
       |-------
     \{1\}   A
     [2]   B
     [3]   C

     which is trivially true.

     This completes the proof of prop.2.2.


     This completes the proof of prop.2.

     prop.3 :   

     [-1]   (A IMPLIES B)
     [-2]   A
       |-------
     \{1\}   A
     [2]   C

     which is trivially true.

     This completes the proof of prop.3.

     Q.E.D.


     Run time  = 0.52 secs.
     Real time = 14.32 secs.
\end{pvsscript}

\paragraph{Summary. }
The PVS Emacs command {\tt M-x pr} is used to invoke the PVS proof
checker.  Proof goals are represented as sequents with the formulas
numbered.  The command {\tt (flatten)} flattens the top-level
disjunctive structure of all of the sequent formulas so that there are
no disjunctive formulas in the resulting subgoal sequent.  (Variations:
{\tt (flatten *)} is the same as {\tt (flatten)}.  {\tt (flatten +)}
flattens only the consequent formulas, and {\tt (flatten -)} the
antecedent formulas.  {\tt (flatten -2 3 4)} flattens formulas numbered
{\tt -2}, {\tt 3}, and {\tt 4} in the goal sequent.)  The command {\tt
(split)} picks the first top-level conjunctive sequent formula and
generates the subgoals that result from splitting this conjunction.
As with {\tt flatten}, {\tt (split *)} is the same as {\tt (split)},
{\tt (split -)} splits the first antecedent conjunction, {\tt (split +)}
the first consequent conjunction, and {\tt (split -3)} splits the
formula numbered {\tt -3}.  

With the same example, we can now attempt to repeat the proof in order
to explore some other commands.  When we now type {\tt M-x pr} at the
formula {\tt prop} in the theory {\tt proposition}, PVS informs us that
the formula has already been proved and asks whether we wish to retry
proving the formula.  If we respond that we do, then PVS inquires
whether the existing proof should be rerun.  If we choose to rerun the
existing proof, the following script is automatically generated.

\begin{pvsscript}
    prop :   

      |-------
    \{1\}   (A IMPLIES (B IMPLIES C)) AND (A IMPLIES B) AND A IMPLIES C

    Rerunning step: (FLATTEN)
    Applying disjunctive simplification to flatten sequent,
    this simplifies to: 
    prop :   

    \{-1\}   (A IMPLIES (B IMPLIES C))
    \{-2\}   (A IMPLIES B)
    \{-3\}   A
      |-------
    \{1\}   C

    Rerunning step: (SPLIT)
    Splitting conjunctions,
    this yields  3 subgoals: 
    prop.1 :   

    \{-1\}   C
    [-2]   (A IMPLIES B)
    [-3]   A
      |-------
    [1]   C

    which is trivially true.

    This completes the proof of prop.1.

       \vdots
\end{pvsscript}

\paragraph{Summary. }  Proofs can be rerun by responding suitably to the
mini-buffer query when {\tt M-x pr} is invoked on a formula that has a
proof or a partial proof.  Another way to rerun the existing proof is to
type {\tt (rerun)} as the first step in a manual proof.

We can retry the same example to explore some further proof commands.
In this version, we choose not to rerun the existing proof.
Typing the inappropriate command {\tt (split)} results in
{\tt No change} to the proof state since there is no top level
conjunctive formula in the sequent.  We then type {\tt (flatten)} which
flattens the formula followed by {\tt (split)} which generates three
subgoals, the first of which is trivially true.  We then type {\tt
(postpone)} at the second subgoal.  This causes the control to shift to
the third subgoal which is also trivially true.  The control now returns
to the second subgoal.  A further {\tt (postpone)} brings us back to the
same subgoal since there are no other pending subgoals.  At this point,
we simply choose to quit the proof by typing {\tt q} at the {\tt Rule?}
prompt.  At the query, we choose to save the partial proof from the
current proof attempt.  

\begin{pvsscript}
     prop :   

       |-------
     \{1\}   (A IMPLIES (B IMPLIES C)) AND (A IMPLIES B) AND A IMPLIES C

     Rule? {\bf (split)}
     No change on: (SPLIT) 
     prop :   

       |-------
     \{1\}   (A IMPLIES (B IMPLIES C)) AND (A IMPLIES B) AND A IMPLIES C

     Rule? {\bf (flatten)}
     Applying disjunctive simplification to flatten sequent,
     this simplifies to: 
     prop :   

     \{-1\}   (A IMPLIES (B IMPLIES C))
     \{-2\}   (A IMPLIES B)
     \{-3\}   A
       |-------
     \{1\}   C

     Rule? {\bf (split)}
     Splitting conjunctions,
     this yields  3 subgoals: 
     prop.1 :   

     \{-1\}   C
     [-2]   (A IMPLIES B)
     [-3]   A
       |-------
     [1]   C

     which is trivially true.

     This completes the proof of prop.1.

     prop.2 :   

     [-1]   (A IMPLIES B)
     [-2]   A
       |-------
     \{1\}   B
     [2]   C

     Rule? {\bf (postpone)}
     Postponing prop.2.

     prop.3 :   

     [-1]   (A IMPLIES B)
     [-2]   A
       |-------
     \{1\}   A
     [2]   C

     which is trivially true.

     This completes the proof of prop.3.

     prop.2 :   

     [-1]   (A IMPLIES B)
     [-2]   A
       |-------
     \{1\}   B
     [2]   C

     Rule? {\bf (postpone)}
     Postponing prop.2.

     prop.2 :   

     [-1]   (A IMPLIES B)
     [-2]   A
       |-------
     \{1\}   B
     [2]   C

     Rule? {\bf q}
     Do you really want to quit?   (Y or N): {\bf y}
     Would you like the partial proof to be saved? 
     (***Old proof will be overwritten.***)
      (Yes or No) {\bf yes}
     Use M-x revert-proof to revert to previous proof.

     Run time  = 0.77 secs.
     Real time = 22.63 secs.
\end{pvsscript}

We can again type {\tt M-x pr} and this time we can rerun the partial
proof that we saved.  
Notice that we are back at the subgoal where we quit the proof
since this is the only unfinished subgoal in the proof.

\paragraph{Summary. }  The command {\tt (postpone)} is used to navigate
cyclically around the unproved subgoals in a proof.  The PVS Emacs
command {\tt M-x siblings} displays all those subgoals that share the
same parent goal as the current subgoal in the proof.  The PVS Emacs
command {\tt M-x ancestry} displays the chain of goals leading back from
the current goal back to the root node of the proof tree.  A {\tt q} or
{\tt quit} can be used to quit out of a proof-in-progress with the
option of saving the partial proof.  If a previous proof is overwritten
as a result, then the PVS Emacs command {\tt M-x revert-proof}
can be used to recover the earlier proof.  
The PVS Emacs command {\tt M-x show-proof} can be used to display a
proof in progress in such a way that parts of it can be edited and
used as input to the {\tt rerun} proof command.  The PVS Emacs command
{\tt M-x edit-proof} with the cursor positioned on a formula in a theory
brings up a buffer containing the proof of the formula displayed as a
tree of commands.  This displayed proof can also be edited and rerun.


\comment{
\begin{pvsscript}
> ;;;
prop :   

  |-------
\{1\}   (A IMPLIES (B IMPLIES C)) AND (A IMPLIES B) AND A IMPLIES C

Rerunning step: (FLATTEN)
Applying disjunctive simplification to flatten sequent,
this simplifies to: 
prop :   

\{-1\}   (A IMPLIES (B IMPLIES C))
\{-2\}   (A IMPLIES B)
\{-3\}   A
  |-------
\{1\}   C

Rerunning step: (SPLIT)
Splitting conjunctions,
this yields  3 subgoals: 
prop.1 :   

\{-1\}   C
[-2]   (A IMPLIES B)
[-3]   A
  |-------
[1]   C

which is trivially true.

This completes the proof of prop.1.

prop.2 :   

[-1]   (A IMPLIES B)
[-2]   A
  |-------
\{1\}   B
[2]   C

Rerunning step: (POSTPONE)
prop.3 :   

[-1]   (A IMPLIES B)
[-2]   A
  |-------
\{1\}   A
[2]   C

which is trivially true.

This completes the proof of prop.3.

prop.2 :   

[-1]   (A IMPLIES B)
[-2]   A
  |-------
\{1\}   B
[2]   C

Rule? {\bf q}
Do you really want to quit?   (Y or N): {\bf y}
Would you like the partial proof to be saved? 
(***Old proof will be overwritten.***)
 (Yes or No) {\bf no}

Run time  = 0.61 secs.
Real time = 137.33 secs.
\end{pvsscript}
}

\subsubsection{Quantifier Proof Commands}

We now consider a simple example involving quantifiers displayed in the
theory {\tt predicate} below.

\begin{pvsscript}
predicate: THEORY
  BEGIN
   T : TYPE
   x, y, z: VAR T
   P, Q : [T -> bool]

   pred_calc: THEOREM
      (FORALL x: P(x) AND Q(x))
      IMPLIES (FORALL x: P(x)) AND (FORALL x: Q(x))

  END predicate
\end{pvsscript}

The proof script for this example starts with the application of {\tt
(flatten)} to the given conjecture followed by the {\tt (split)} command
to break the consequent conjunction.  In the first branch of the proof,
we use the {\tt (skolem)} command to replace the universally quantified
variable {\tt x} in the consequent formula numbered {\tt 1} with the
(Skolem) constant {\tt X}, where {\tt X} is new (\ie\ undeclared) in the
present context.  The next step is to instantiate the universally
quantified variable {\tt x} in the antecedent formula numbered {\tt -1}
with the constant {\tt X} using the {\tt (inst)} command.  The first
branch of the proof is then easily completed by propositional reasoning.
Note that the two quantifier steps, {\tt skolem} and {\tt inst}, only
affect the outermost quantifier of a formula in the sequent.  Also,
universally quantified variables in consequent formulas are replaced by
new constants, whereas antecedent universally quantified variables are
instantiated with terms.  Existentially quantified variables behave
dually.  The second branch of the proof employs minor variants of the
{\tt skolem} and {\tt inst}.  Here the {\tt (skolem!)} command picks
the first ``skolemizable'' sequent formula and replaces the quantified
variables with internally generated constants (containing exclamations).
The {\tt (inst?)} command picks the first instantiable sequent formula
and tries to find an instantiation for the quantified variables by
matching against the rest of the sequent.

\begin{pvsscript}
    pred_calc :   

      |-------
    \{1\}   (FORALL x: P(x) AND Q(x)) IMPLIES (FORALL x: P(x)) AND (FORALL x: Q(x))

    Rule? {\bf (flatten)}
    Applying disjunctive simplification to flatten sequent,
    this simplifies to: 
    pred_calc :   

    \{-1\}   (FORALL x: P(x) AND Q(x))
      |-------
    \{1\}   (FORALL x: P(x)) AND (FORALL x: Q(x))

    Rule? {\bf (split)}
    Splitting conjunctions,
    this yields  2 subgoals: 
    pred_calc.1 :   

    [-1]   (FORALL x: P(x) AND Q(x))
      |-------
    \{1\}   (FORALL x: P(x))

    Rule? {\bf (skolem 1 {\tt "X"})}
    For the top quantifier in 1, we introduce Skolem constants: X
    this simplifies to: 
    pred_calc.1 :   

    [-1]   (FORALL x: P(x) AND Q(x))
      |-------
    \{1\}   P(X)

    Rule? {\bf (inst -1 {\tt "X"})}
    Instantiating the top quantifier in -1 with the terms: 
     X
    this simplifies to: 
    pred_calc.1 :   

    \{-1\}   P(X) AND Q(X)
      |-------
    [1]   P(X)

    Rule? {\bf (prop)}
    By propositional simplification,

    This completes the proof of pred_calc.1.

    pred_calc.2 :   

    [-1]   (FORALL x: P(x) AND Q(x))
      |-------
    \{1\}   (FORALL x: Q(x))

    Rule? {\bf (skolem!)}
    Skolemizing,
    this simplifies to: 
    pred_calc.2 :   

    [-1]   (FORALL x: P(x) AND Q(x))
      |-------
    \{1\}   Q(x!1)

    Rule? {\bf (inst?)}
    Found substitution:
    x gets x!1,
    Instantiating quantified variables,
    this simplifies to: 
    pred_calc.2 :   

    \{-1\}   P(x!1) AND Q(x!1)
      |-------
    [1]   Q(x!1)

    Rule? {\bf (prop)}
    By propositional simplification,

    This completes the proof of pred_calc.2.

    Q.E.D.
\end{pvsscript}

\comment{
Types in PVS are not necessarily nonempty so that one cannot prove
statements like {\tt (FORALL x:\ P(x)) IMPLIES (EXISTS x:\ P(x))} unless
the variable {\tt x} is known to range over a nonempty type.
}

\paragraph{Summary. }
The command {\tt (skolem 1 "X")} is used to introduce a new constant
{\tt X} in place of the universally quantified variable in the formula
numbered {\tt 1}.  {\tt (skolem 1 ("X" "\_" "Z"))} is to be used if
there are three variables bound by the universal quantifier and only the
first and third are to be replaced by constants.  {\tt (skolem + "X")}
carries out the skolemization step for the first consequent universally
quantified formula, and {\tt (skolem - "X")} for the first antecedent
existentially quantified formula.  The variations of the instantiation
command {\tt inst} are similar to those of {\tt skolem}.  The command
forms {\tt (skolem!)}, {\tt (skolem! 1)}, {\tt (skolem! -)}, etc., are
variants of {\tt skolem} where the new constant names are internally
generated.  The command {\tt (inst?)} is a version of {\tt inst} that
tries to find a matching substitution for a chosen quantified formula.
It can also be supplied a partial substitution to disambiguate the
matching process as in {\tt (inst? - :subst ("x" "X"))}.  Both {\tt
inst} and {\tt inst?}\ take an optional {\tt :copy?}\ argument that can
be given as {\tt T} in order to retain a copy of the original quantified
formula in the sequent in case further instances of the formula are
needed, as in {\tt (inst + ("x" "X") :copy? T)}.  The PVS rule
{\tt inst-cp} is a version of the {\tt inst} that automatically
copies the quantified formula, and {\tt inst} is the non-copying
variant.  
Note that optional
arguments to PVS proof commands can be given by order or by keyword.
To find out the order, the keywords, and defaults for each of the proof
commands, use {\tt M-x help-pvs-prover}.

\subsubsection{Decision Procedures}

The equality and linear inequality decision procedures are the
workhorses of almost any nontrivial PVS proof.  The theory {\tt
decisions} displayed below illustrates some of the power of these
decision procedures.  The formulas marked {\tt THEOREM} are those that
can be proved using the decision procedures, and the ones marked
{\tt CONJECTURE} are either true but cannot be proved solely by the
decision procedures (like {\tt badarith1}) or false (like {\tt badarith}
and {\tt badarith2}) and hence unprovable.  The reader should
invoke {\tt M-x pr} on each of the formulas in {\tt decisions} and
type either {\tt (then (skolem!)(ground))} or
{\tt (then* (skolem!)(flatten)(assert))} to the {\tt Rule?}\ prompt
to observe the effects of the decision procedures.  The command
{\tt assert} is used to either record equality or inequality information
into the data-structures used by the decision procedures, or to simplify
propositional or {\tt IF-THEN-ELSE} structures in a formula, or
carry out the automatic rewrites (to be described below).
The command {\tt (ground)} is a combination of {\tt (prop)} and {\tt (assert)}.

\begin{pvsscript}
decisions: THEORY
  BEGIN
   x,y,v: VAR number
   f: [number -> number]

   eq1: THEOREM x = f(x) IMPLIES f(f(f(x))) = x

   g : [number, number -> number]

   eq2: THEOREM x = f(y) IMPLIES g(f(y + 2 - 2), x + 2) = g(x, f(y) + 2)

   arith: THEOREM  %Proved by decision procedures
     x < 2*y AND y < 3*v IMPLIES 3*x < 18*v

   badarith: CONJECTURE %Not proved; statement is false.
     x < 2*y AND y < 3*v IMPLIES 3*x < 17*v

   badarith1: CONJECTURE %Not proved; statement true but non-linear
     x<0 AND y<0 IMPLIES x*y>0

   i, j, k: VAR int
 
   intarith: THEOREM %Proved by decision procedures  
     2*i < 5 AND i > 1 IMPLIES i = 2

   badarith2: CONJECTURE  %Not proved; stmt. true of integers but not reals.
     2*x < 5 AND x > 1 IMPLIES x = 2

   range : THEOREM  %Proved by decision procedures
     i > 0 AND i < 3 IMPLIES i = 1 OR i = 2

  END decisions
\end{pvsscript}

We now consider an example proof that further illustrates the use of
decision procedures.  The theory {\tt stamps} below contains the formula
asserting that any postage requirement of 8~cents or more can be met
solely with 3 and 5~cent stamps, \ie\ is the sum of some multiple of 3
and some multiple of 5.

\begin{pvsscript}
stamps: THEORY
  BEGIN
  
  i, three, five: VAR nat
  
  stamps: LEMMA (FORALL i: (EXISTS three, five: i + 8 = 3 * three + 5 * five))
  
  END stamps
\end{pvsscript}

In abstract terms, the proof proceeds by induction on {\tt i}.  In the
base case, when {\tt i} is {\tt 0}, the left-hand side is 8.  Letting
{\tt m} and {\tt n} both be {\tt 1} fulfills the equality.  In the
induction case, we know that {\tt that i + 8} can be expressed as {\tt
3*M + 5*N} for some {\tt M} and {\tt N} and we need to find {\tt m} and
{\tt n} such that {\tt i + 8 + 1} is {\tt 3*m + 5*n}.  If {\tt N = 0},
then {\tt M} is at least {\tt 3}.  We then let {\tt m} be {\tt  M - 3}
and {\tt n} be {\tt 2}, \ie\ we remove three 3~cent stamps and add two
5~cent stamps to get postage worth {\tt i + 8 + 1}.  If {\tt N > 0},
then we simply remove a 5~cent stamp and add two 3~cent stamps to prove
the induction conclusion.

In the proof script below, the first command {\tt (induct "i")} directs
PVS to use induction on {\tt i}.  PVS deduces from the type {\tt nat} of
{\tt i} that natural number induction is to be used and formulates an
induction predicate based on the formula number 1 in the sequent.  The
command {\tt induct}, like {\tt prop} and {\tt ground}, is a compound
step or a {\em proof strategy\/}.  Two subgoals are generated
corresponding to the base and induction cases.  In the base case, the
{\tt inst} command is used to instantiate {\tt three} with {\tt 1} and {\tt
five} with {\tt 1}.  The decision procedures are invoked to prove the
resulting trivial arithmetic equality.  In the induction case, the {\tt
skolem} command followed by {\tt flatten} results in a sequent
containing the induction hypothesis in its antecedent and the conclusion
in its consequent part.  The witnesses corresponding to the induction
hypothesis are produced by the {\tt skolem!}\ command.  The case-split
according to {\tt five!1 = 0} is created by the {\tt case} command.  In the
first {\tt five!1 = 0} case, we instantiate the existential quantifiers in
the conclusion as required by the abstract proof.  Since the bound
variable {\tt three} has type {\tt nat} (which is a subtype of the {\tt
integer} type consisting of the non-negative integers), the {\tt inst}
command generates a second (type correctness) subgoal demanding proof
that {\tt three!1 - 3} is at least {\tt 0}.  Both subgoals are discharged
through the use of {\tt assert}.  In the case when {\tt five!1 = 0} is
false, note that the assumption of falsity is indicated by the formula
{\tt five!1 = 0} appearing in the consequent part of the goal sequent.  We
now follow an approach that is slightly different from that of the
previous branch; we use {\tt assert} at this point.  This has no visible
effect on the sequent to be proved, but the falsity of {\tt five!1 = 0} is
noted by the decision procedures for use deeper in the proof.  Now note
that the {\tt inst} command instantiating {\tt five} with {\tt five!1 - 1}
does not generate the type correctness subgoal that was generated in the
previous branch since the decision procedures were able to automatically
demonstrate that {\tt five!1 - 1} was non-negative from the known
information. 

\begin{pvsscript}
     stamps :  

       |-------
     \{1\}   (FORALL i: (EXISTS three, five: i + 8 = 3 * three + 5 * five))
     
     Rule? {\bf (induct "i")}
     Inducting on i,
     this yields  2 subgoals: 
     stamps.1 :  
     
       |-------
     \{1\}   (EXISTS (three: nat), (five: nat): 0 + 8 = 3 * three + 5 * five)
     
     Rule? {\bf (inst 1 1 1)}
     Instantiating the top quantifier in 1 with the terms: 
      1, 1,
     this simplifies to: 
     stamps.1 :  
     
       |-------
     \{1\}   0 + 8 = 3 * 1 + 5 * 1
     
     Rule? {\bf (assert)}
     Simplifying, rewriting, and recording with decision procedures,
     
     This completes the proof of stamps.1.
     
     stamps.2 :  
     
       |-------
     \{1\}   (FORALL (j: nat):
              (EXISTS (three: nat), (five: nat): j + 8 = 3 * three + 5 * five)
                IMPLIES
                (EXISTS (three: nat), (five: nat):
                   j + 1 + 8 = 3 * three + 5 * five))
     
     Rule? {\bf (skolem + "JJ")}
     For the top quantifier in +, we introduce Skolem constants: JJ,
     this simplifies to: 
     stamps.2 :  
     
       |-------
     \{1\}   (EXISTS (three: nat), (five: nat): JJ + 8 = 3 * three + 5 * five)
             IMPLIES
             (EXISTS (three: nat), (five: nat):
                JJ + 1 + 8 = 3 * three + 5 * five)
     
     Rule? {\bf (flatten)}
     Applying disjunctive simplification to flatten sequent,
     this simplifies to: 
     stamps.2 :  
     
     \{-1\}   (EXISTS (three: nat), (five: nat): JJ + 8 = 3 * three + 5 * five)
       |-------
     \{1\}   (EXISTS (three: nat), (five: nat): JJ + 1 + 8 = 3 * three + 5 * five)
     
     Rule? {\bf (skolem!)}
     Skolemizing,
     this simplifies to: 
     stamps.2 :  
     
     \{-1\}   JJ + 8 = 3 * three!1 + 5 * five!1
       |-------
     [1]   (EXISTS (three: nat), (five: nat): JJ + 1 + 8 = 3 * three + 5 * five)
     
     Rule? {\bf (case "five!1 = 0")}
     Case splitting on 
        five!1 = 0, 
     this yields  2 subgoals: 
     stamps.2.1 :  
     
     \{-1\}   five!1 = 0
     [-2]   JJ + 8 = 3 * three!1 + 5 * five!1
       |-------
     [1]   (EXISTS (three: nat), (five: nat): JJ + 1 + 8 = 3 * three + 5 * five)
     
     Rule? {\bf (inst + "three!1 - 3" 2)}
     Instantiating the top quantifier in + with the terms: 
      three!1 - 3, 2,
     this yields  2 subgoals: 
     stamps.2.1.1 :  
     
     [-1]   five!1 = 0
     [-2]   JJ + 8 = 3 * three!1 + 5 * five!1
       |-------
     \{1\}   JJ + 1 + 8 = 3 * (three!1 - 3) + 5 * 2
     
     Rule? {\bf (assert)}
     Simplifying, rewriting, and recording with decision procedures,
     
     This completes the proof of stamps.2.1.1.
     
     stamps.2.1.2 (TCC):   
     
     [-1]   five!1 = 0
     [-2]   JJ + 8 = 3 * three!1 + 5 * five!1
       |-------
     \{1\}   three!1 - 3 >= 0
     
     Rule? {\bf (assert)}
     Simplifying, rewriting, and recording with decision procedures,
     
     This completes the proof of stamps.2.1.2.
     
     
     This completes the proof of stamps.2.1.
     
     stamps.2.2 :  
     
     [-1]   JJ + 8 = 3 * three!1 + 5 * five!1
       |-------
     \{1\}   five!1 = 0
     [2]   (EXISTS (three: nat), (five: nat): JJ + 1 + 8 = 3 * three + 5 * five)
     
     Rule? {\bf (assert)}
     Simplifying, rewriting, and recording with decision procedures,
     this simplifies to: 
     stamps.2.2 :  
     
     \{-1\}   8 + JJ = 5 * five!1 + 3 * three!1
       |-------
     [1]   five!1 = 0
     \{2\}   (EXISTS (three: nat), (five: nat): 9 + JJ = 5 * five + 3 * three)
     
     Rule? {\bf (inst + "three!1 + 2" "five!1 - 1")}
     Instantiating the top quantifier in + with the terms: 
      three!1 + 2, five!1 - 1,
     this simplifies to: 
     stamps.2.2 :  
     
     [-1]   8 + JJ = 5 * five!1 + 3 * three!1
       |-------
     [1]   five!1 = 0
     \{2\}   9 + JJ = 5 * (five!1 - 1) + 3 * (three!1 + 2)
     
     Rule? {\bf (assert)}
     Simplifying, rewriting, and recording with decision procedures,
     
     This completes the proof of stamps.2.2.
     
     
     This completes the proof of stamps.2.
     
     Q.E.D.


\end{pvsscript}

\paragraph{Summary. }
PVS proofs make heavy use of decision procedures to simplify tedious
equality and arithmetic reasoning so that the number
of trivial subgoals can be minimized and to keep the sequent formulas 
simple.  The equality decision procedure employs congruence closure
to propagate equality information along the term structure to quickly
decide whether a sequent containing equalities and other propositions is
true.  An antecedent formula $P$ that is not an equality can be treated
as $P \mbox{\tt = TRUE}$, and a consequent formula $P$ as the equality
$P \mbox{\tt = FALSE}$.  The {\tt assert} rule is the most powerful form
in which decision procedures are applied.  It is a combination of
the {\tt record} rule which records sequent formulas in the
data-structures used by the decision procedures, {\tt simplify} which
simplifies branching and propositional structure using the decision
procedures, {\tt beta} which beta-reduces record, tuple,
function-update, {\tt LAMBDA}, and abstract datatype redexes,
and {\tt do-rewrite} which applies the rewrites specified by
{\tt auto-rewrite} and {\tt auto-rewrite-theory}.  

The {\tt (case $\langle formula\rangle^*$)} command used in the above
proof is extremely useful for case-splitting on a formula.  For example,
if there is no straightforward way to simplify a formula $A$ to another
formula $A'$, then one can case-split on $A'$ so that we can use $A'$ on
one branch and prove it from $A$ on the other branch.  The {\tt case}
command can also be used to replace a term $s$ by $s'$ by case-splitting
on $s = s'$ and using the {\tt replace} proof command (which is not
explained here) to carry out the replacement.

\subsubsection{Using Definitions and Lemmas}

For the purpose of this discussion, we use the following very simple
example of a recursive function that halves a given natural number.

\begin{pvsscript}\label{half}
half: THEORY
  BEGIN
  
  i, j, k: VAR nat
  
  half(i): RECURSIVE nat =
      (IF i = 0 THEN 0 ELSIF i = 1 THEN 0 ELSE half(i - 2) + 1 ENDIF)
        MEASURE (LAMBDA i: i)
  
  half_halves: THEOREM half(2 * i) = i
  
  half_half: THEOREM half(2 * half(2 * i)) = i
  
  END half
\end{pvsscript}

We show a segment of the proof of {\tt half\_halves} where the
definition of {\tt half} is expanded.  Notice that the first use
of {\tt expand} brings in an unsimplified expansion of the definition of
{\tt half}.  When we {\tt undo} this proof step and retry the same
{\tt expand} step following an {\tt assert}, not only is the expansion
simplified, but the equality is itself reduced to {\tt TRUE}.  

\begin{pvsscript}
       \vdots
    half_halves.2 :   

    \{-1\}   half(2 * J) = J
      |-------
    \{1\}   half(2 * (J + 1)) = J + 1

    Rule? {\bf (expand {\tt "half"} +)}
    Expanding the definition of half
    this simplifies to: 
    half_halves.2 :   

    [-1]   half(2 * J) = J
      |-------
    \{1\}   (IF 2 * (J + 1) = 0 THEN 0 ELSE half(2 * (J + 1) - 2) + 1 ENDIF) = J + 1

    Rule? {\bf (undo)}
    This will undo the proof to: 
    half_halves.2 :   

    \{-1\}   half(2 * J) = J
      |-------
    \{1\}   half(2 * (J + 1)) = J + 1
    Sure? (Y or N): y
    half_halves.2 :   

    \{-1\}   half(2 * J) = J
      |-------
    \{1\}   half(2 * (J + 1)) = J + 1

    Rule? {\bf (assert)}
    Invoking decision procedures,
    this simplifies to: 
    half_halves.2 :   

    [-1]   half(2 * J) = J
      |-------
    [1]   half(2 * (J + 1)) = J + 1

    Rule? {\bf (expand {\tt "half"} +)}
    Expanding the definition of half
    this simplifies to: 
    half_halves.2 :   

    [-1]   half(2 * J) = J
      |-------
    \{1\}   TRUE

    which is trivially true.
       \vdots
\end{pvsscript}

The {\tt rewrite} command is an alternative to {\tt expand}, though {\tt
rewrite} can be used to rewrite with both formulas and definitions.  In
the script below, the {\tt rewrite} step replaces the second of the
above applications of {\tt expand}.  Notice that {\tt rewrite} behaves
slightly differently from {\tt expand}, but it too is sensitive to the
facts recorded by the decision procedures from a previous {\tt assert}.

\begin{pvsscript}
       \vdots
    half_halves.2 :   

    [-1]   half(2 * J) = J
      |-------
    [1]   half(2 * (J + 1)) = J + 1

    Rule? {\bf (rewrite {\tt "half"} +)}
    Rewriting using half,  
    this simplifies to: 
    half_halves.2 :   

    [-1]   half(2 * J) = J
      |-------
    \{1\}   half(2 * (J + 1) - 2) + 1 = J + 1

    Rule? {\bf (assert)}
    Invoking decision procedures,

    This completes the proof of half_halves.2.
       \vdots
\end{pvsscript}

In summary, {\tt expand} is used to expand definitions, and {\tt
rewrite} is used to rewrite using definitions and formulas.  Both employ
decision procedures for simplification during rewriting.  Decision
procedures are also used to discharge any conditions (arising from a
conditional rewrite rule) and the type-correctness conditions arising
from the lemma instantiation applied by {\tt rewrite}.  The {\tt expand}
step is the preferred way to expand definitions.
%Here is an example of
%the use of rewrite on a formula taken from the proof of {\tt half\_half}.
%
%\begin{pvsscript}
%       \vdots
%    half_half :   
%
%      |-------
%    \{1\}   half(2 * half(2 * i!1)) = i!1
%
%    Rule? {\bf (rewrite {\tt "half\_halves"})}
%    Rewriting using half_halves,  
%    Q.E.D.
%\end{pvsscript}

\paragraph{Other Commands.}
We have described some typical commands, but have not mentioned many
others.  A partial account of some of those we've omitted is given
below; a complete, annotated list of prover commands can be found
in The PVS Prover Checker Reference Manual~\cite{PVS:prover}.
The {\tt lemma} command is used to bring in an instance of a
lemma as an antecedent sequent formula.  The {\tt extensionality}
proof command is similarly used to bring in the extensionality scheme
given a suitable type expression, \ie\ a function, record, or tuple
type or an abstract datatype.  The {\tt beta} rule is used to carry
out beta-reduction of redexes including those involving {\tt
LAMBDA}-abstraction, record access, tuple access, function updates,
and datatype expressions.  The command {\tt delete} can be used to
drop irrelevant sequent formulas; {\tt hide} is a more conservative
form of {\tt delete} where the formula can be restored using the {\tt
reveal} command.  The PVS Emacs command {\tt M-x show-hidden} shows
the hidden formulas.  The command {\tt typepred} can be used to make
the subtype predicates on a given expression explicit as sequent
formulas.  The {\tt lift-if} command lifts {\tt IF}-branching to the
top-level of a sequent formula through $F${\tt (IF $A$ THEN $s$ ELSE
$t$ ENDIF)} being transformed to {\tt (IF $A$ THEN $F(s)$ ELSE $F(t)$
ENDIF)}.  The commands {\tt auto-rewrite} and {\tt
auto-rewrite-theory} are used to install rewrite rules to be used
automatically by the {\tt assert} command.


\subsubsection{Proof Checker Pragmatics}

The \pvs\ proofs in the tutorial examples reflect a very low level of
automation and should be viewed merely as pedagogical exercises.  The
proof checker actually provides several powerful commands for the
advanced user that make it possible to verify large classes of
theorems using only a small number of steps.  For example, the {\tt
grind} command is usually a good way to complete a proof that only
requires definition expansion, and arithmetic, equality, and
quantifier reasoning.  The decision procedure command {\tt assert} is
used very frequently since it does simplification, automatic
rewriting, and records the sequent formulas in the decision procedure
database.  The {\tt inst?}\ command is the most effective way to
automatically instantiate quantifiers of existential strength.  The
{\tt induct-and-simplify} command is a powerful way to construct
proofs by induction.  The commands {\tt induct-and-rewrite} and {\tt
induct-and-rewrite!}\ are variants of {\tt induct-and-simplify}\@.
These induction commands are able to automatically complete a fairly
large class of induction proofs.

It is not necessary to master all the proof commands in order to use
the \pvs\ proof checker effectively.  In general, it is advisable to
learn the most powerful commands first and only rely on the simpler
commands when the powerful ones fail.  For example, the initial step
in a proof is usually skolemization, and the preferred and most
powerful form here is {\tt skosimp*}. Similarly, {\tt
induct-and-simplify} or one of its variants should be used to initiate
induction proofs.

 Typically, the creative choices in a proof are:
 \begin{enumerate}
 \item The induction scheme: One of the above induction commands should be
 employed here.

 \item The case analysis: If the case analysis is not explicit in the
 propositional structure, then it might be implicit in an embedded {\tt
 IF-THEN-ELSE} or {\tt CASES} structure in which case the {\tt lift-if}
 command should be used to bring the case analyses to the surface of the
 sequent where they can be propositionally simplified.  Otherwise, the case
 analysis has to be supplied explicitly using the {\tt case} command.

\item The quantifier instantiations:  The instantiation of antecedent
universal and succedent existential quantifiers is done automatically by
the {\tt inst?}\ command.  When this fails, the more manual {\tt inst} and
{\tt inst-cp} commands should be used.
\end{enumerate}

The {\tt bddsimp} command is the most efficient way to do propositional
simplification, but {\tt prop} will do when efficiency is not important.
Propositional simplification has to be used with care because it can
generate many subgoals that share the same proof.  The {\tt flatten}
and {\tt split} commands are used to do the propositional
simplification more delicately.  

User-defined {\em proof strategies\/}, similar to the tactics and
tacticals of LCF, can be employed by more advanced \pvs\ users.  A
file containing definitions of basic strategies is distributed with
\pvs\ and provides a good introduction to this topic. The PVS Prover Checker
Reference Manual~\cite{PVS:prover} can be consulted for additional
information on user-defined proof strategies.

Finally, it is helpful to be familiar with the \pvs\ prelude theories,
which provide very useful background mathematics, as well as a rich
source of examples.









