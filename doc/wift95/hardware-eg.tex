% Master File: tutorial.tex
\sloppy

% full list of sections:
%\includeonly{title,intro,informal,pspace,undecide,conclu,ack,rules}

%\pagestyle{myheadings} % page number in upper right corner
%\markboth{Specification and Verification}{}
%\makeindex
%\newcommand{\allttinput}[1]{\hozline{\smaller\smaller\smaller\begin{alltt}\input{#1}\end{alltt}}\hozline}
%\newenvironment{pvsscript}{\hozline\smaller\smaller\smaller\begin{alltt}}{\end{alltt}\hozline}
%\newcommand{\ehdm}{{E{\smaller\smaller HDM}}}
%\newcommand{\Ehdm}{\ehdm}

%\topmargin -10pt
%\textheight 8.5in
%\textwidth 6.0in
%\headheight 15 pt
%\columnwidth \textwidth
%\oddsidemargin 0.5in
%\evensidemargin 0.5in   % fool system for page 0
%\setcounter{topnumber}{9}
%\renewcommand{\topfraction}{.99}
%\setcounter{bottomnumber}{9}
%\renewcommand{\bottomfraction}{.99}
%\setcounter{totalnumber}{10}
%\renewcommand{\textfraction}{.01}
%\renewcommand{\floatpagefraction}{.01}
%\newenvironment{smalltt}{\begin{alltt}\small}{\end{alltt}}
\raggedbottom

\font\largett=cmtt10 scaled\magstep2
\font\hugett=cmtt10 scaled\magstep4
\def\opt{{\smaller\sc {\smaller\smaller \&}optional}}
\def\rest{{\smaller\sc {\smaller\smaller \&}rest}}
\def\default#1{[\,{\tt #1}] }
\def\bkt#1{{$\langle$#1$\rangle$}}
\def\SetFigFont#1#2#3{\smaller\smaller\rm}

%\newenvironment{usage}[1]{\item[usage:\hspace*{-0.175in}]#1\begin{description}\setlength{\itemindent}{-0.2in}\setlength{\itemsep}{0.1in}}{\end{description}}
\renewcommand{\pvstheory}[3]
  {\begin{figure}[htb]\begin{boxedminipage}{\textwidth}%
   {\smaller\smaller\smaller\begin{alltt} \input{#1}\end{alltt}}\end{boxedminipage}%
   \caption{#2}\label{#3}\end{figure}}


\section{Two Hardware Examples}

In this final section, we develop two hardware examples that
illustrate a more sophisticated use of \pvs\ and suggest the
intellectual discipline involved in specifying and proving
industrial-strength applications.  The pipelined microprocessor and
n-bit ripple-carry adder examples provide an opportunity to touch on
modeling issues, specification styles, and hardware proof strategies,
as well as a chance to review many of the \pvs\ language and prover
features described in earlier sections of this tutorial.\footnote{One
point worth noting that may not be apparent in reading these examples
is that the process of specification and verification is an iterative one in
which proof is used not to certify a completed specification, but
as an aid to developing the specification.}



\subsection{A Pipelined Microprocessor}

We first develop a complete proof of a correctness property of the
controller logic of a simple pipelined processor design described at a
register-transfer level.  The design and the property verified are
both based on the processor example given in \cite{Clarke-etal90}.
The example has been used as a benchmark for evaluating how well
finite state-enumeration based tools, such as model checkers, can
handle datapath-oriented circuits with a large number of states by
varying the size of the datapath.  From the perspective of a theorem
prover, the size of the datapath is irrelevant because the
specification and proof are independent of the datapath size.  As a
theorem proving exercise, the challenge is to see if the proof can be
done as automatically as a model checker proof.

\begin{figure}[b]
\begin{center}
\setlength{\unitlength}{0.009375in}%
\begin{picture}(191,262)(112,399)
\thicklines
\put(190,567){\framebox(13,19){}}
\put(190,542){\framebox(13,19){}}
\put(190,514){\framebox(13,19){}}
\put(278,542){\framebox(12,19){}}
\put(278,514){\framebox(12,19){}}
\put(190,461){\framebox(13,38){}}
\put(190,411){\framebox(13,38){}}
\put(228,499){\line( 0,-1){ 38}}
\put(228,449){\line( 0,-1){ 38}}
\put(259,474){\line( 0,-1){ 38}}
\put(228,499){\line( 5,-4){ 31.098}}
\put(228,411){\line( 5, 4){ 31.098}}
\multiput(228,461)(0.66667,-0.33333){19}{\makebox(0.5926,0.8889){\SetFigFont{7}{8.4}{rm}.}}
\multiput(240,455)(-0.66667,-0.33333){19}{\makebox(0.5926,0.8889){\SetFigFont{7}{8.4}{rm}.}}
\put(278,436){\framebox(12,38){}}
\put(203,480){\vector( 1, 0){ 25}}
\put(203,427){\vector( 1, 0){ 25}}
\put(259,455){\vector( 1, 0){ 19}}
\put(265,455){\vector( 0,-1){ 56}}
\put(303,455){\line( 0,-1){ 56}}
\put(303,399){\line(-1, 0){107}}
\put(196,399){\vector( 0, 1){ 12}}
\put(203,524){\vector( 1, 0){ 75}}
\put(203,552){\vector( 1, 0){ 75}}
\put(178,661){\line( 0,-1){ 62}}
\put(165,630){\line( 1, 0){ 13}}
\put(165,599){\framebox(125,62){}}
\put(278,661){\line( 0,-1){ 62}}
\put(165,642){\line(-1, 0){ 25}}
\put(140,642){\line( 0,-1){212}}
\put(140,430){\vector( 1, 0){ 50}}
\put(128,580){\vector( 1, 0){ 62}}
\put(196,461){\line( 0,-1){ 12}}
\multiput(165,511)(8.73333,0.00000){16}{\makebox(0.5926,0.8889){\SetFigFont{10}{12}{rm}.}}
\multiput(165,589)(8.73333,0.00000){16}{\makebox(0.5926,0.8889){\SetFigFont{10}{12}{rm}.}}
\multiput(165,511)(0.00000,8.66667){10}{\makebox(0.5926,0.8889){\SetFigFont{10}{12}{rm}.}}
\multiput(296,511)(0.00000,8.66667){10}{\makebox(0.5926,0.8889){\SetFigFont{10}{12}{rm}.}}
\put(196,499){\vector( 0,-1){0}}
\multiput(196,511)(0.00000,-12.00000){1}{\makebox(0.5926,0.8889){\SetFigFont{10}{12}{rm}.}}
\put(243,486){\vector( 0,-1){0}}
\multiput(243,511)(0.00000,-8.33333){3}{\makebox(0.5926,0.8889){\SetFigFont{10}{12}{rm}.}}
\put(178,524){\vector( 1, 0){ 12}}
\put(128,552){\vector( 1, 0){ 62}}
\put(284,561){\vector( 0, 1){ 38}}
\put(265,455){\vector( 0, 1){ 56}}
\put(290,455){\line( 1, 0){ 13}}
\put(303,455){\line( 0, 1){178}}
\put(303,633){\vector(-1, 0){ 13}}
\put(146,655){\line( 0,-1){125}}
\put(128,614){\line( 1, 0){ 18}}
\put(146,655){\vector( 1, 0){ 19}}
\put(146,605){\vector( 1, 0){ 19}}
\put(146,530){\vector( 1, 0){ 19}}
\put(165,617){\line(-1, 0){ 12}}
\put(153,617){\line( 0,-1){140}}
\put(153,477){\vector( 1, 0){ 37}}
\put(168,518){\makebox(0,0)[lb]{\smash{\SetFigFont{5}{6.0}{bf}stall}}}
\put(209,627){\makebox(0,0)[lb]{\smash{\SetFigFont{5}{6.0}{bf}REGFILE}}}
\put(112,586){\makebox(0,0)[lb]{\smash{\SetFigFont{5}{6.0}{bf}opcode}}}
\put(243,443){\makebox(0,0)[lb]{\smash{\SetFigFont{5}{6.0}{bf}U}}}
\put(243,452){\makebox(0,0)[lb]{\smash{\SetFigFont{5}{6.0}{bf}L}}}
\put(243,461){\makebox(0,0)[lb]{\smash{\SetFigFont{5}{6.0}{bf}A}}}
\put(215,567){\makebox(0,0)[lb]{\smash{\SetFigFont{5}{6.0}{bf}CONTROL}}}
\put(240,542){\makebox(0,0)[lb]{\smash{\SetFigFont{5}{6.0}{bf}dsntdd}}}
\put(165,542){\makebox(0,0)[lb]{\smash{\SetFigFont{5}{6.0}{bf}dstnd}}}
\put(206,527){\makebox(0,0)[lb]{\smash{\SetFigFont{5}{6.0}{bf}stalld}}}
\put(243,514){\makebox(0,0)[lb]{\smash{\SetFigFont{5}{6.0}{bf}stalldd}}}
\put(271,427){\makebox(0,0)[lb]{\smash{\SetFigFont{5}{6.0}{bf}wbreg}}}
\put(153,418){\makebox(0,0)[lb]{\smash{\SetFigFont{5}{6.0}{bf}opreg2}}}
\put(156,461){\makebox(0,0)[lb]{\smash{\SetFigFont{5}{6.0}{bf}opreg1}}}
\put(115,542){\makebox(0,0)[lb]{\smash{\SetFigFont{5}{6.0}{bf}dstn}}}
\put(206,580){\makebox(0,0)[lb]{\smash{\SetFigFont{5}{6.0}{bf}opcoded}}}
\put(112,627){\makebox(0,0)[lb]{\smash{\SetFigFont{5}{6.0}{bf}src2}}}
\put(112,617){\makebox(0,0)[lb]{\smash{\SetFigFont{5}{6.0}{bf}src1}}}
\end{picture}

\end{center}
\caption{A Pipelined Microprocessor}
\label{clarkepipefig}
\end{figure}

\subsubsection{Informal Description}

Figure~\ref{clarkepipefig} shows a block diagram of the pipeline
design.  The processor executes instructions of
the form {\tt (opcode src1 src2 dstn)}, i.e., ``destination register
{\tt dstn} in the register file {\tt REGFILE} becomes some {\tt ALU}
function determined by {\tt opcode} of the contents of source registers
{\tt src1} and {\tt src2}.
Every instruction is executed in three stages (cycles) by the processor:
\begin{enumerate}
\item {\em Read}: Obtain the proper contents of the register file at {\tt src1}
and {\tt src2} and clock them into {\tt opreg1} and {\tt opreg2},
respectively.

\item {\em Compute}: Perform the ALU operation corresponding to the
opcode (remembered in {\tt opcoded}) of the instruction and clock the
result into {\tt wbreg}.

\item {\em Write}: Update the register file at the destination register
(remembered in {\tt dstndd}) of the instruction with the value in
{\tt wbreg}.
\end{enumerate}
The processor uses a three-stage pipeline to simultaneously execute
distinct stages of three successive instructions.  That is, 
the read stage of the current instruction is executed along with the
compute stage of the previous instruction and the write stage
of the previous-to-previous instruction.
Since the {\tt REGFILE} is not updated with the results of the previous and
previous-to-previous instructions while a read is being
performed for the current instruction, the controller
``bypasses'' {\tt REGFILE}, if necessary, to get the correct values for
the read.  The processor can abort, i.e., treat as {\tt NOP},
the instruction in the read stage by asserting the {\tt stall} signal true.
An instruction is aborted by inhibiting its write stage
by remembering the {\tt stall} signal until the write stage via
the registers {\tt stalld} and {\tt stalldd}.
We verify that an instruction entering the pipeline
at any time gets completed correctly, i.e., will write the correct result
into the register file, three cycles later, provided the instruction
is not aborted.

\subsubsection{Formal Specification}

\pvs\ specifications consist of a number of files, each of which
contains one or more theories.
%A theory is a collection of declarations:
%types, constants (including functions), axioms that express properties
%about the constants, and theorems and lemmas to be proved.
%Theories may import other theories.
%Every entity used in a theory must be either declared in an imported theory
%or be part of the prelude (the standard
%collection of theories built-in to \pvs\).
\pvstheory{pipeline-spec}{Microprocessor Specification}{pipeline-spec}
The microprocessor specification is organized into three theories, selected
parts of which are shown in Figures~\ref{pipeline-spec} and
\ref{signal-spec}.
(The complete specification can be found in~\cite{HW-Tutorial:Report}.)
The theory {\tt pipe} (Figure~\ref{pipeline-spec})
contains a specification
of the design and a statement of the correctness property to be
proved.
The theories {\tt signal} and {\tt time}, (Figure~\ref{signal-spec})
imported by {\tt pipe}, declares the types {\tt signal} and {\tt time}
used in {\tt pipe}.

\pvstheory{signal-spec}{Signal Specification}{signal-spec}
%\pvstheory{pipeline-spec}{Microprocessor Specification}{pipeline-spec-part2}

The theory {\tt pipe} is parameterized with respect to the types of the
register address, data, and the opcode field of the instructions.
A theory parameter in PVS can be either a type parameter or
a parameter belonging to a particular type, such as {\tt nat}.
Since {\tt pipe} does not impose any restriction on its parameters,
other than the requirement that they be nonempty, which is stated
in the {\tt ASSUMING} part of the theory,
one can instantiate them with any type.
Every entity declared in a parameterized theory is implicitly parameterized
with respect to the parameters of the theory.
For example, the type {\tt signal} declared in the parameterized
theory {\tt signal} is a parametric type denoting a function that
maps {\tt time} (a synonym for {\tt nat})
to the type parameter {\tt T}.  (The type {\tt signal} is used
to model the wires in our design.)
By importing the theory {\tt signal} uninstantiated in {\tt pipe},
we have the freedom to create any desired instances of the type
{\tt signal}.

In this example, we use a {\em functional} style of specification
to model register-transfer-level digital hardware in logic.
In this style, the inputs to the design and the outputs of every component
in the design are modeled as signals.
Every signal that is an output of a component is
specified as a function of the signals appearing at the inputs to
the component.

This style should be contrasted with
a {\em predicative} style, which is commonly used in most HOL applications.
In the predicative style every hardware component is specified as a
predicate relating the
input and output signals of the component and a design is
specified as a conjunction of the component predicates, with all
the internal signals used to connect the components
hidden by existential quantification.
A proof of correctness for a predicative style specification usually involves
executing a few additional steps at the start of the proof
to essentially transform
the predictative specification into an equivalent functional style.
After that, the proof proceeds similar to that of a proof in
a functional specification.
The additional proof steps required for a predicative specification
essentially unwind the component predicates using
their definitions and then appropriately
instantiate the existentially quantified variables.
An automatic way of performing this translation is discussed in
\cite{HW-Tutorial:Report}, which illustrates more examples
of hardware design verification using PVS.

Returning to our example, the microprocessor specification
in {\tt pipe} consists of two parts.
The first part declares all the signals
used in the design---the inputs
to the design and the internal wires that denote the outputs of components.
The composite state of {\tt REGFILE}, which is represented
as a function from {\tt addr} to {\tt data}, is modeled by the signal
{\tt regfile}.
The signals are declared as uninterpreted constants of appropriate types.
The second part consists of a set of AXIOMs that specify the
the values of the signals over time.
(To conserve space, we have only shown the specification of a subset
of the signals in the design.)
For example, the signal value at the output of the
register {\tt dstnd} at time {\tt t+1} is defined to be that of its
input a cycle earlier.
The output of the ALU, which is a combinational component, is defined
in terms of the inputs at the same time instant.

In PVS, we can use a descriptive style of definition, as illustrated
in this example, by selectively introducing properties of the
constants declared in a theory as AXIOMs.  Alternatively, we can use
the definitional forms provided by the language to define the
constants.  An advantage of using the definitions is that a
specification is guaranteed to be consistent. A disadvantage is that
the resulting specification may be overly specific (i.e.,
overspecified).  An advantage of the descriptive style is that it
gives better control over the degree to which an entity is defined For
example, we could have specified {\tt dstnd} prescriptively, using the
conventional function definition mechanism of PVS, which would have
forced us to specify the value of the signal at time {\tt t = 0} to
ensure that the function is total.  In the descriptive style used, we
have left the value of the signal at {\tt 0} unspecified.

In the present example, the specifications of the signals
{\tt opreg1} and {\tt opreg2} are the most interesting of all.
They have to check for any register collisions that might
exist between the instruction in the
read stage and the instructions in the later stages and bypass reading
from the register file in case of collisions.
The {\tt regfile} signal specification is recursive since the register
file state remains the same as its previous state except,
possibly, at a single register location.
The {\tt WITH} expression is an abbreviation for the result
of updating a function at a given point in the domain value with a new value.
Note that the function {\tt aluop} that denotes the operation ALU performs
for a given {\tt opcode} is left completely unspecified since it
is irrelevant to the controller logic.

The theorem ({\tt correctness}) to be proved states a correctness property
about the execution of the instruction that enters the pipeline at
{\tt t}, provided the instruction is not aborted, i.e., {\tt stall(t)} is
not true.
The equation in the conclusion of the implication compares the
actual value (left hand side) in the destination register three
cycles later, when the result of the instruction would be in place,
with the expected value.
The expected value is the result of applying the {\tt aluop} corresponding
to the opcode of the instruction to the values at the source field
registers in the register file at {\tt t+2}.
We use the state of the register file at {\tt t+2} rather than {\tt t}
to allow for the results of the two previous instructions in the pipeline
to be completed.

\subsubsection{Proof of Correctness}

Once the specification is complete, the next step is to typecheck the
file, which parses and checks for semantic errors, such as undeclared
names and ambiguous types.  As we have already seen, typechecking may
build new files or internal structures such as {\em type correctness
conditions} ({\em \tccs}) that represent {\em proof obligations\/}
that must be discharged before the {\tt pipe} theory can be considered
typechecked.  The typechecker does not generate any \tccs\ in the
present example.  If, for example, one of the assumptions, say for
{\tt addr}, in the {\tt ASSUMING} part of the theory was missing, the
typechecker would generate the following \tcc\ to show that the {\tt
addr} type is nonempty.  The declaration of the signal {\tt src1}
forces generation of this \tcc\ because a function is nonexistent if
its range is empty.

\mbox{}

\noindent
\begin{boxedminipage}{\textwidth}
\begin{alltt}
{\smaller\smaller
% Existence TCC generated (line 17) for src1: signal[addr]
% May need to add an assuming clause to prove this.
  % unproved
src1_TCC1: OBLIGATION (EXISTS (x1: signal[addr]): TRUE);
}
\end{alltt}
\end{boxedminipage}

\mbox{}

%The \pvs\ proof checker runs as a subprocess of Emacs.
%Once invoked on a theorem to be proved, it accepts commands
%directly from the user.

By way of review, the basic objective of developing a proof in \pvs\
as in other subgoal-directed proof checkers (e.g., HOL), is to
generate a {\em proof tree\/} in which all of the leaves are trivially
true.  The nodes of the proof tree are sequents, and in the
prover you are always looking at an unproved leaf of the tree.
The {\em current\/} branch of a proof is the branch leading back to
the root from the current sequent.  When a given branch is complete
(i.e., ends in a true leaf), the prover automatically moves on to the
next unproved branch, or, if there are no more unproven branches,
notifies you that the proof is complete.

%One reason why a proof in \pvs\ differs from a HOL proof is due to
The primitive inference steps in \pvs\ are a lot more
powerful than in HOL; it is not necessary to build complex tactics to
handle tedious lower level proofs in \pvs\@.  A knowledgeable \pvs\
user can typically get proofs to go through mostly automatically by
making a few critical decisions at the start of the proof.  However,
as noted previously, \pvs\ does provide the user with the equivalent
of HOL's tacticals, called {\em strategies}, and other features to
control the desired level of automation in a proof.

The proof of the microprocessor property shown below follows a certain
general pattern that works successfully for most hardware proofs.
This general proof pattern, variants of which have been used
in other verification
exercises \cite{mephisto,HOL:super}, consists of the following
sequence of general proof tasks.
\begin{description}
\item[Quantifier elimination:] Since the decision procedures work on
ground formulas, the user must eliminate the relevant universal
quantifiers by skolemization or selecting variables on which to induct
and existential quantifiers by suitable instantiation.

\item[Unfolding definitions:] The user may have to simplify selected
expressions
and defined function symbols in the goal by rewriting using definitions,
axioms or lemmas.
The user may also have to decide the level to which the function symbols have
to rewritten.

\item[Case analysis:] The user may have to split the proof based on
selected boolean expressions in the current goal and simplify the
resulting goals further.
\end{description}

Each of the above tasks can be accomplished automatically using a short
sequence of primitive PVS proof commands.
The complete proof of the theorem is shown below.
Selected parts of the proof session are reproduced below as we describe
the proof.

\mbox{}

\noindent
\begin{boxedminipage}{\textwidth}
\begin{alltt}
{\smaller\smaller
1: ({\em then*} (skosimp)
2:         (auto-rewrite-theory ``pipe'' :always? t)
3:         (repeat (do-rewrite))
4:         (apply (then* (repeat (lift-if))
5:                       (bddsimp)
6:                       (assert))))
}
\end{alltt}
\end{boxedminipage}

\mbox{}

In the proof, the names of strategies are shown in {\em italics} and
the primitive inference steps in {\tt type-writer font}.
(We have numbered the lines in the proof for reference.)
{\tt Then*} applies
the first command in the list that follows to the current goal;
the rest of the commands in the list are then applied
to each of the subgoals generated by the first command
application.
The {\tt apply} command used in line 5 makes the application of 
a compound proof step implemented by a strategy behave as
an atomic step.

The first goal in the proof session is shown below.
It consists of a
single formula (labeled {\tt \{1\}}) under a dashed line.  This is a
{\em sequent\/}; formulas above the dashed lines are called {\em
antecedents\/} and those below are called {\em succedents\/}.  The
interpretation of a sequent is that the conjunction of the antecedents
implies the disjunction of the succedents.
\comment{
Either or both of the
antecedents and succedents may be empty.\footnote{An empty antecedent
is equivalent to {\tt true}, and an empty succedent is equivalent to
{\tt false}, so if both are empty the sequent is unprovable.}}

\mbox{}

\noindent
\begin{boxedminipage}{\textwidth}
\begin{alltt}
{\smaller\smaller
correctness :   

  |-------
\{1\}   (FORALL t: NOT (stall(t))
                  IMPLIES regfile(t + 3)(dstn(t)) =
                     aluop(opcode(t), regfile(t + 2)(src1(t)),
                           regfile(t + 2)(src2(t))))
}
\end{alltt}
\end{boxedminipage}

\mbox{}

The quantifier elimination task of the proof is accomplished
by the command {\tt skosimp}, which skolemizes all the universally quantified
variables in a formula and flattens the sequent resulting in the following
goal.  Note that {\tt stall(t!1)} has been moved to the
succedent in the sequent because \pvs\ displays every atomic formula in
its positive form.

\mbox{}

\noindent
\begin{boxedminipage}{\textwidth}
\begin{alltt}
{\smaller\smaller
Rule? (skosimp)
Skolemizing and flattening, this simplifies to: 
correctness :   

  |-------
\{1\}   (stall(t!1))
\{2\}   regfile(t!1 + 3)(dstn(t!1))
        =
        aluop(opcode(t!1), regfile(t!1 + 2)(src1(t!1)),
              regfile(t!1 + 2)(src2(t!1)))
}
\end{alltt}
\end{boxedminipage}

\mbox{}

The next task---unfolding definitions---is performed by the commands
in lines 2 through 3.  \pvs\ provides a number of ways of unfolding
definitions ranging from unfolding one step at a time to automatic
rewriting that performs unfolding in a brute-force fashion.
Brute-force rewriting usually results in larger expressions than
controlled unfolding and, hence, potentially larger number of cases to
consider.  If a system provides automatic and efficient rewriting and
case analysis facilities, then the automatic approach is viable,
as illustrated here.  In \pvs\, automatic rewriting is performed
by first entering the definitions and AXIOMs to be used
for unfolding as rewrite rules.  Once entered, the commands that
perform rewriting as part of their repertoire, such as {\tt
do-rewrite} and {\tt assert}, repeatedly apply the rewrite rules until
none of the rules is applicable.  To control the size of the
expression resulting from rewriting and the potential for looping, the
rewriter uses the following restriction for stopping a rewrite: If the
right-hand-side of a rewrite is a conditional expression, then the
rule is applied only if the condition simplifies to true or false.
\comment{ Also, application of a recursive rewrite rule, such as {\tt
regfile\_ax} is inhibited on recursive instances of a function symbol
if the function is inside a conditional expression.}

Here our aim is to unfold every signal in
the sequent so that every signal expression contains only the
start time {\tt t!1}.
So, we make a rewrite rule out of every AXIOM in the theory {\tt pipe}
by means of the command {\tt auto-rewrite-theory} on line 2.
We also force an over-ride of the default restriction for stopping rewriting by
setting the tag\footnote{Tags are one of the ways in which \pvs\ permits the user to modify
the functionality of proof commands.} {\tt always?}\ to true in the {\tt auto-rewrite-theory}
command and embed {\tt do-rewrite} inside a {\tt repeat} loop to force
maximum rewriting.
In the present example, the rewriting
is guaranteed to terminate because every feedback loop is cut by a sequential
component.

At the end of automatic rewriting, the succedent we are trying to prove
is in the form of an equation on two deeply nested conditional expressions
as shown below in an abbreviated fashion.
The various cases in conditional expression shown above
arise as a result of
the different possible conflicts between instructions
in the pipeline.  The equation we are trying to prove contains
two distinct, but equivalent conditional expressions, as in
{\tt IF a THEN b ELSE c ENDIF = IF NOT a THEN c ELSE b ENDIF}, that
can only be proved equal by performing a case-split on one or more of the
conditions.  While {\tt assert} simplifies the leaves of a conditional
expression assuming every condition along the path to the leaves holds,
it does not split propositions.
One way to perform the case-splitting task automatically is
to ``lift'' all the {\tt IF-THEN-ELSE}s to the top so that the equation
is transformed into a propositional formula with unconditional equalities
as atomic predicates.
After performing such a lifting, we can try to reduce the resulting
proposition to true using the propositional simplification command
{\tt bddsimp}.  If {\tt bddsimp} does not simplify the proposition
to true, then it is most likely the case that equations at one or more
of the leaves of the proposition need to be further simplified, e.g., by
{\tt assert}, using the conditions along the path.
If the propositional formula does not reduce to true or false,
{\tt bddsimp} produces a set of subgoals to be proved.
In the present case, each of these goals can be discharged
by {\tt assert}.
The compound proof step appearing on lines 4 through 6 of the proof
accomplishes the case-splitting task.

\mbox{}

\noindent
\begin{boxedminipage}{\textwidth}
\begin{alltt}
{\smaller\smaller
correctness :   

  |-------
[1]   (stall(t!1))
\{2\}   aluop(opcode(t!1),
            IF src1(t!1) = dstnd(t!1) & NOT stalld(t!1)
              THEN aluop(opcoded(t!1), opreg1(t!1), opreg2(t!1))
            ELSIF src1(t!1) = dstndd(t!1) & NOT stalldd(t!1)
            THEN wbreg(t!1)
            ELSE regfile(t!1)(src1(t!1)) ENDIF,
            ....
            ENDIF)
        = aluop(opcode(t!1),
              IF stalld(t!1) THEN IF stalldd(t!1) THEN regfile(t!1)
                ELSE regfile(t!1) WITH [(dstndd(t!1)) := wbreg(t!1)]
                ENDIF
              ELSE ...
              ENDIF(src1(t!1)),
              IF stalld(t!1) THEN IF stalldd(t!1) THEN regfile(t!1)
                ELSE ... ENDIF
              ELSE ...
              ENDIF(src2(t!1)))
}
\end{alltt}
\end{boxedminipage}

\mbox{}

We have found that the sequence of steps shown above works successfully
for proving safety properties of finite state machines that relate
states of the machine that are finite distance apart.  If the strategy
does not succeed, the most likely cause is that either the
property is not true or that a certain property about some of the
functions in the specification unknown to the prover needs to be
proved as a lemma.  In either case, the unproven goals remaining at the
end of the proof provide information about the probable cause.


\subsection{An N-bit Ripple-Carry Adder}

The second example we consider is the verification of a parametrized
N-bit ripple-carry adder circuit.
The theory {\tt adder}, shown in Figure~\ref{adder-spec},
specifies a ripple-carry adder
circuit and a statement of correctness for the circuit.

\pvstheory{adder-spec}{Adder Specification}{adder-spec}

The theory is parameterized with respect to the length of the bit-vectors.
It imports the theories (not shown here)
{\tt full\_adder}, which contains a
specification of a full adder circuit ({\tt fa\_cout} and {\tt fa\_sum}),
and {\tt bv}, which specifies
the bit-vector type ({\tt bvec[N]}) and functions.
An N-bit bit-vector is represented as an array, i.e., a function, from
the the type {\tt below[N]}, a subtype of {\tt nat} ranging from
{\tt 0} through {\tt N-1}, to {\tt bool}; the index {\tt 0} denotes the least
significant bit.
Note that the parameter {\tt N} is constrained to be a {\tt posnat} since
we do not permit bit vectors of length {\tt 0}.
The {\tt adder} theory contains several declarations including a set
of initial variable declarations.
%\pvs\ allows logical variables to be declared globally within a theory
%so that the variables can be used later in function
%definitions and quantified formulas.

The carry bit that ripples through the full adder is specified recursively
by means of the function {\tt nth\_cin}.
%Associated with this definition is a
%{\em measure\/} function, following the {\tt MEASURE} keyword, which
%will be explained below.
The function {\tt bv\_cout} and {\tt bv\_sum} define the carry output
and the bit-vector sum of the adder, respectively.
The theorem {\tt adder\_correct} expresses the conventional correctness
statement of an adder circuit using {\tt bvec2nat}, which returns the
natural number equivalent of an N-bit bit-vector.
Note that variables that are left free in a formula are assumed to be
universally quantified.
We state and prove a more general lemma {\tt adder\_correct\_rec} of which
{\tt adder\_correct} is an instance.
For a given {\tt n < N},
{\tt bvec2nat\_rec} returns the natural number equivalent of
the least significant n-bits of a given bit-vector and {\tt bool2bit}
converts the boolean constants {\tt TRUE} and {\tt FALSE} into the natural
numbers {\tt 1} and {\tt 0}, respectively.

\subsubsection{Typechecking}
\index{typecheck|(}

The typechecker generates several \tccs\
(shown in Figure~\ref{adder-tccs} below) for {\tt adder}.
%These \tccs\ represent
%{\em proof obligations\/} that must be discharged before the {\tt adder}
%theory can be considered typechecked.  The proofs of the \tccs\
%may be postponed until it is convenient to prove them, though it is a good
%idea to view them to see if they are provable.

\pvstheory{adder-tccs}{\tccs\ for Theory {\tt adder}}{adder-tccs}

The first \tcc\ is due to the fact that the first argument to {\tt nth\_cin}
is of type {\tt below[N]}, but the type of the argument ({\tt n-1})
in the recursive
call to {\tt nth\_cin} is integer, since {\tt below[N]} is not closed
under subtraction.
Note that the \tcc\ includes the condition {\tt NOT n = 0}, which holds
in the branch of the {\tt IF-THEN-ELSE} in which the expression
{\tt n - 1} occurs.
A \tcc\ identical to this one is generated for each
of the two other occurrences of the expression {\tt n-1} because
{\tt bv1} and {\tt bv2} also expect arguments of type {\tt below[N]}.
These \tccs\ are not retained because they are subsumed by the first one.

The second \tcc\ is generated by the expression {\tt N-1} in the
definition of the theorem {\tt adder\_correct} because the first
argument to {\tt bv\_cout} is expected to be the subtype {\tt below[N]}.

There is yet another \tcc\ that is internally generated
by \pvs\ but is not even included in the \tccs\ file because
it can be discharged trivially by the typechecker, which calls
the prover to perform simple normalizations of expressions.
This \tcc\ is generated to ensure that the recursive definition
of {\tt nth\_cin} terminates.
\pvs\ does not directly support partial
functions, although its powerful subtyping mechanism allows \pvs\ to
express many operations that are traditionally regarded as partial.
As discussed earlier, the
measure function is used to show that recursive definitions are total by
requiring the measure to decrease with each recursive call.
For the definition of {\tt nth\_cin}, this entails showing
{\tt n-1 < n}, which the typechecker trivially deduces.

In the present case, all the remaining \tccs\ are simple, and in fact can be
discharged automatically
by using the {\tt typecheck-prove} command, which attempts
to prove all \tccs\ that have been generated using a predefined
proof strategy called {\tt tcc}.
\index{TCCs@\tccs|)}\index{typecheck|)}

\subsubsection{Proof of Adder\_correct\_n}
The proof of the lemma uses the same core strategy as in the
microprocessor proof except for the quantifier elimination step.
Since the specification is recursive in the length of the bit-vector,
we need to perform induction on the variable {\tt n}. As we've seen
in earlier proofs,
the user invokes an inductive proof in \pvs\ by means of the command
{\tt induct} with the variable to induct on ({\tt n}) and the induction
scheme to be used ({\tt below\_induction[N]}) as arguments.
The induction used in this case is defined in the \pvs\ prelude and
is parameterized, as is the type {\tt below[N]}, with respect to
the upper limit of the subrange.

This command  generates two subgoals:
the subgoal corresponding to the base case, which is the first goal presented
to prove, is shown in Figure~\ref{base-step}.

\pvstheory{base-step}{Base Step}{base-step}

The goal corresponding to the inductive case is shown below.

\pvstheory{siblings}{Inductive Step}{siblings}

The base and the inductive steps can be proved automatically
using essentially the same strategy used in the microprocessor proof.
A complete proof of {\tt adder\_correct\_n} is shown in~Figure~\ref{siblings}.

\begin{alltt}
{\smaller\smaller
 1: ({\em spread} (induct ``n'' 1 ``below_induction[N]'')
 2:   ( ({\em then*}  (skosimp*)
 3:              (auto-rewrite-defs :always? t)
 4:              (do-rewrite)
 5:              ({\em repeat} (lift-if))
 6:              ({\em apply} ({\em then*} (bddsimp)(assert))))
 7:     ({\em then*} (skosimp*)
 8:             (inst?)
 9:             (auto-rewrite-defs :always? t)
10:             (do-rewrite)
11:             ({\em repeat} (lift-if))
12:             ({\em apply} ({\em then*} (bddsimp)(assert))))))
}
\end{alltt}

The strategy {\em spread} used on line 1 applies the first proof step
({\tt induct})
and then applies the $i^{th}$ element of the list of commands that follow
to the $i^{th}$ subgoal resulting from the application of the first prof step.
Thus, the proof steps listed on lines 2 through 6 prove the base case
of induction, the steps on lines 7 through 12 prove the inductive case, and
the proof step on line 13 takes care of the third \tcc\ subgoal.

%\input{adder-auto-proof-script}

We consider the base case first.
The {\tt induct} command has already instantiated the variable {\tt n}
to {\tt 0}.
The remaining variables are skolemized away by {\tt skosimp*}.
To unfold the definitions in the resulting goal, we use
the command {\tt auto-rewrite-defs}, which makes rewrite rules out
of the definition of every function either directly or indirectly
used in the given formula.
The rest of the proof proceeds exactly as for the microprocessor.

The proof of the inductive step follows exactly the same pattern except
that we need to instantiate the induction hypothesis and use it in
the process of unfolding and case-analysis.
\pvs\ provides a command {\tt inst?}\ that tries to find instantiations
for existential-strength variables in a formula by searching for possible
matches between terms involving these variables with ground terms inside
formulas in the rest of the sequent.  This command finds the desired
instantiations in the present case.  The rest of the proof proceeds
as in the basis case.

Since the inductive proof pattern shown above is applicable to any
iteratively generated hardware designs, we have packaged it into a
general proof strategy called {\tt name-induct-and-bddrewrite}.  The
strategy is parameterized with respect to an induction scheme
and the set of rewrite rules to be used for unfolding.  We have used
the strategy to prove an N-bit ALU~\cite{cantu:alu} that executes 12
microoperations by cascading N 1-bit ALU slices.
