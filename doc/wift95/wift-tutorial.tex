% Master File: wift-tutintro.tex
% \documentstyle[11pt,epsf,part,notochead,nobibhead,nfltxsym,relative,cite,alltt,twoside,relative,fancyheadings,boxedminipage,url,/project/pvs/pvs]{article}
\documentclass[11pt,part]{article}
\usepackage{,notochead,nobibhead,relsize,cite,alltt,relsize,fancyheadings,boxedminipage,url}
\sloppy
\newcommand{\allttinput}[1]{\hozline{\smaller\smaller\smaller\begin{alltt}\input{#1}\end{alltt}}\hozline}
\newenvironment{pvsscript}{\hozline\smaller\smaller\smaller\begin{alltt}}{\end{alltt}\hozline}
\topmargin -10pt
\textheight 8.5in
\textwidth 6.0in
\headheight 15 pt
\columnwidth \textwidth
\oddsidemargin 0.5in
\evensidemargin 0.5in   % fool system for page 0
\setcounter{topnumber}{9}
\renewcommand{\topfraction}{.99}
\setcounter{bottomnumber}{9}
\renewcommand{\bottomfraction}{.99}
\setcounter{totalnumber}{10}
\renewcommand{\textfraction}{.01}
\renewcommand{\floatpagefraction}{.01}
\raggedbottom
\font\largett=cmtt10 scaled\magstep2
\font\hugett=cmtt10 scaled\magstep4
\def\opt{{\smaller\sc {\smaller\smaller \&}optional}}
\def\rest{{\smaller\sc {\smaller\smaller \&}rest}}
\def\default#1{[\,{\tt #1}] }
\def\bkt#1{{$\langle$#1$\rangle$}}
\newenvironment{usage}[1]{\item[usage:\hspace*{-0.175in}]#1\begin{description}\setlength{\itemindent}{-0.2in}\setlength{\itemsep}{0.1in}}{\end{description}}
%\renewcommand{\baselinestretch}{2}
\newenvironment{display}{\begin{alltt}\small\tt\vspace{0.3\baselineskip}}{\vspace{0.3\baselineskip}\end{alltt}}
\newcommand{\choice}{[\!]}
\newcommand{\normtt}[1]{{\obeyspaces {\tt #1 }}}
\newenvironment{pagegroup}{}{}
%\newenvironment{smalltt}{\begin{alltt}\small\tt}{\end{alltt}}
\newenvironment{tdisplay}{\begin{alltt}\footnotesize\tt\vspace{0.3\baselineskip}}{\vspace{0.3\baselineskip}\end{alltt}}
%% %
%
%    LaTeX  Macro File  /usr2/jcm/tex/macros.tex
%
%
\tracingonline=0 % shorter error messages (on screen)

%\long\def\comment#1{} % mulit-line comments
%\newcommand{\note}[1]{\fbox{#1}}

% font changes  (as function calls, scribe style)

\renewcommand{\i}[1]{{\it #1\/}}       % italics with space correction
%\newcommand{\emex}[1]{\/{\em #1}}   % emphasis in example, exercise, theorem, etc.
\renewcommand{\c}[1]{{\sc #1}}       % small caps (eliminates \c as cedilla)
%\newcommand{\r}[1]{{\rm #1}}            % for roman font in math mode
%\newcommand{\s}[1]{{\scr #1}}          % script (for use with tatex) 
\newcommand{\s}[1]{{\cal #1}}
\renewcommand{\b}[1]{{\bf #1}}           % bold face
\newcommand{\calg}[1]{{\cal #1}}          % caligraphic 

%\newcommand{\q}[1]{``#1''}    % matching quotes for in-line quotation

% numbered environments

%\newcounter{partcounter}
%\setcounter{partcounter}{0}
%\renewcommand{\part}[1]{\newpage \addtocounter{partcounter}{1}
%\noindent{\Large \bf Part \Roman{partcounter}. #1 } \\[1ex]}

\newtheorem{thm}{Theorem}[section]
\newtheorem{theorem}[thm]{Theorem}
\newtheorem{lemma}[thm]{Lemma}
\newtheorem{cor}[thm]{Corollary}
\newtheorem{corollary}[thm]{Corollary}
\newtheorem{claim}[thm]{Claim}
\newtheorem{prop}[thm]{Proposition}
\newtheorem{conj}[thm]{Conjecture}
\newtheorem{definition}[thm]{Definition}
\newtheorem{exercise}[thm]{Exercise}
\newtheorem{example}[thm]{Example}
\newtheorem{remark}[thm]{Remark}
\newtheorem{open}[thm]{Open Problem}

%\newcommand{\proof}{\\{\bf Proof.}\ }
\newenvironment{proof}{{\bf Proof. }}{\thmbox}

% axiom and inference rule (centered, in math mode,  with name at left)

\newcommand{\axiom}[2]
{\[ \hbox to \columnwidth
    { \rlap{$#2$} \hfil {$ #1 $} \hfil }
\]}

\newcommand{\infrule}[4]
{\[ \hbox to \columnwidth %\textwidth
    { \rlap{$#4$} \hfil $
      \frac {\strut\displaystyle #1 } {\strut\displaystyle #2 } \; \rlap{$#3$} \hfil $
      \hfil }
\]}
\newcommand{\infruletw}[4]
{\[ \hbox to \textwidth
    { \rlap{$#4$} \hfil $
      \frac {\strut\displaystyle #1 } {\strut\displaystyle #2 } \; \rlap{$#3$} \hfil $
      \hfil }
\]}

%\newcommand{\infrule}[4]
%{\[ \hbox to \columnwidth
%    { \rlap{$#4$} \hfil $
%      {{\displaystyle\strut #1}\over{\displaystyle\strut #2}}\quad\makebox[0pt][l]{\it #3} $
%      \hfil }
%\]}

% axiom and inference rule macros for use in tables, etc.
% presumed in math mode #1=axiom, #2=side condition
\newcommand{\Axiom}[2]{
{\displaystyle\strut #1}\qquad\makebox[0pt][l]{\it #2}
}
% presumed in math mode #1=top, #2=bottom, #3=side condition
\newcommand{\Infrule}[3]{
{{\displaystyle\strut #1}\over{\displaystyle\strut #2}}\;\mbox{\scriptsize$\bf #3$}
}

% sequence of #1's, numbered up to #2  (e.g., seq{x}{n} for x1, ..., xn   )

\newcommand{\seq}[2]{#1_{1} \ldots #1_{#2}}  
%\newcommand{\ie}{{\it i.e.}}
%\newcommand{\eg}{{\it e.g.}}
%\newcommand{\cf}{{\it c.f.\,}}

% common symbols

\newcommand{\union}{\cup}
\newcommand{\intersect}{\cap}
\newcommand{\subs}{\subseteq}
\newcommand{\el}{\in}
\newcommand{\nel}{\not\in}
\newcommand{\ns}{\emptyset}
\newcommand{\compose}{\circ}
\newcommand{\set}[2]{ \{\, #1 \,\mid\, #2 \,\}  } % set macro
\newcommand{\infinity}{\infty}
\newcommand{\pair}[1]{\langle #1 \rangle}
\newcommand{\tuple}[1]{\langle #1 \rangle}


\newcommand{\fa}{\forall}
\newcommand{\te}{\exists}
\newcommand{\imp}{\supset}
\renewcommand{\implies}{\supset}
\newcommand{\ts}{\vdash}
\newcommand{\dts}{\models}

\newcommand{\aro}{\mathord\rightarrow} % see pages 154-155 of TeX manual
\newcommand{\paro}{\rightharpoonup} 
\newcommand{\karo}{\mathop\Rightarrow} % see pages 154-155 of TeX manual
%\newcommand{\cross}{\times}
%\newcommand{\dlb}{\lbrack\!\lbrack}
%\newcommand{\drb}{\rbrack\!\rbrack}
\newcommand{\mean}[1]{\lbrack\!\lbrack #1 \rbrack\!\rbrack}
\newcommand{\lam}{\lambda}
\newcommand{\subst}[2]{{}[#1/#2]}
\renewcommand{\dot}{\mathrel{\bullet}}
\newcommand{\Dinf}{D_{\infty}}
\newcommand{\bottom}{\perp}

\mathcode`:="603A  % treat : as punctuation instead of relation in math mode
\mathchardef\colon="303A	% relation colon

\newcommand{\eqdef}{\mathrel{:=}}
\newcommand{\Dom}{\mathop{\rm dom}}
\newcommand{\Pow}{\mathop{\rm Pow}}

\newcommand{\aequiv}{\equiv_\alpha}
\newcommand{\baro}{\buildrel \beta \over \rightarrow}
\newcommand{\earo}{\buildrel \eta \over \rightarrow}
\newcommand{\red}{\rightarrow\!\!\!\!\rightarrow}
\newcommand{\backred}{\leftarrow\!\!\!\!\leftarrow}
\newcommand{\bred}{\buildrel \beta \over \red}
\newcommand{\ered}{\buildrel \eta \over \red}
\newcommand{\conv}{\leftrightarrow}
\newcommand{\beconv}{\buildrel {\beta, \eta} \over\leftrightarrow}

% lambda calculus abbreviations
\newcommand{\letdec}[3]{\b{let\ } #1 = #2 \b{\ in\ } #3}
\newcommand{\letrec}[3]{\b{letrec\ } #1 = #2 \b{\ in\ } #3}

\newcommand{\thmbox}
   {{\ \hfill\hbox{%
      \vrule width1.0ex height1.0ex
   }\parfillskip 0pt }}
\newcommand{\qed}{\thmbox}


% make single spacing

\newcommand{\singlespace}{\renewcommand{\baselinestretch}{1}\@normalsize}
\newcommand{\etal}{{\em et al.}}

% bycase command for definition by cases (pg 49 of LaTeX)

\newcommand{\bycase}[1]
	{\left\{ \begin{array}{ll}  #1  \end{array} \right. }


% make \cite put blanks after the comma (use in alpha style)

%\def\@citex[#1]#2{\if@filesw\immediate\write\@auxout{\string\citation{#2}}\fi
%  \def\@citea{}\@cite{\@for\@citeb:=#2\do
%    {\@citea\def\@citea{,\penalty100\hskip2.5pt plus1.5pt minus.8pt}%
%       \@ifundefined{b@\@citeb}{{\bf ?}\@warning
%       {Citation `\@citeb' on page \thepage \space undefined}}
%\hbox{\csname b@\@citeb\endcsname}}}{#1}}

%%  % zero-width inference rule markers
 % (if they have width, they throw off centering)
\newcommand{\zwb}[1]{\makebox[0cm][l]{{\scriptsize $#1$}}}

\newcommand{\mall}{{\sc mall}}
\newcommand{\ol}{\overline}
\newcommand{\ul}{\underline}
\newcommand{\lan}{\langle}
\newcommand{\ran}{\rangle}

 % Linear Arrow
\newcommand{\la}{\mbox{$-\!\circ$}} 

 % Backwards Linear Arrow
\newcommand{\bla}{\mbox{$\circ\!-$}} 

 % Par
%\newcommand{\Par}{\,{\parfont P}\,}
\newcommand{\Par}{\mathrel{\wp}}

 % With (Linear And) (&)
\newcommand{\with}{\,{\&}\,}

% Old par
% \newcommand{\Par}{\mbox{\raisebox{.4ex}{$\wp$}}} 

 % PDL's prettier bounded reuse operator
\newcommand{\bang}[1]{\stackrel{\scriptstyle \dagger}{\scriptstyle #1}}
 
 % Implies in Relevant Implication
\newcommand{\ra}{\mbox{$\rightarrow_{R_{\rightarrow}}$}}

 % Relevant Implication (Logic)
\newcommand{\rara}{\mbox{${R_{\rightarrow}}$}}

 % Multiplicative Linear Logic  
\newcommand{\lala}{\mbox{$LL_{\la, \otimes, !}$}}

 % (Propositional) Circular Logic
\newcommand{\cl}{\mbox{ CL }}

 % (Propositional) Multiplicative Circular Logic
\newcommand{\mcl}{\mbox{ MCL }}

 % Definition
\newcommand{\define}{\mbox{${\stackrel{\Delta}{=}}$}}

 % Derived From
\newcommand{\derive}{\mbox{${\stackrel{\vdots}{\vdash}}$}}

 % Semi-Thue Reduction
\newcommand{\stred}[1]{\mbox{$\Longrightarrow$}}

\newcommand{\stredstar}[1]{\mbox{$\Longrightarrow^*$}}

\newcommand{\deltared}[1]{\mbox{$\, \stackrel{\delta_{#1}}
                                             {\longrightarrow} \, $}}

\newcommand{\deltaredp}[1]{\mbox{$\, \stackrel{\delta'_{#1}}
                                             {\longrightarrow} \, $}}


%% \newenvironment{commentary}{\begin{quote}\small{\bf Commentary:}}{{\bf
End of Commentary}\normalsize\end{quote}} 
%\newenvironment{proof}{{\bf Proof:}}{$\Box$\\}
%\newtheorem{theorem}{Theorem}
%\newtheorem{lemma}{Lemma}
%\newtheorem{proposition}{Proposition}
%\newtheorem{definition}{Definition}
%\newtheorem{corollary}{Corollary}
%\newtheorem{example}{Example}
\newenvironment{mytheorem}{\begin{theorem}\rm}{\end{theorem}}
%\newcommand{\qed}[0] {\rule{1ex}{1ex} \vspace{2ex}}
\newcommand{\defdef}{\mbox{$\stackrel{\rm def}{=}$}}
\def\defn{\mathrel{\defdef}}
%\newcommand{\defn}{\mbox{$\stackrel{\rm def}{=}$}}
\newcommand{\intro}[1]{\begin{itemize}\item #1 \end{itemize}}
\newcommand{\simeqrel}[1]{\stackrel{#1}{\simeq}}
\newcommand{\eqrel}[1]{\stackrel{#1}{\approx}}
\newcommand{\simrel}[1]{\stackrel{#1}{\sim}}
\newcommand{\congrel}[1]{\stackrel{#1}{\cong}}
\def\maj{\mathop{\rm maj}}
\def\min{\mathop{\rm min}}
\def\rem{\mathop{\rm rem}}
\def\div{\mathop{\rm div}}
\newcommand{\rmand}{\mbox{\bf \ and }}
\newcommand{\rmif}{\mbox{\bf if\ }}
\newcommand{\rmiff}{\mbox{\bf \ iff \ }}
\newcommand{\rmthen}{\mbox{\bf \ then }}
\newcommand{\rmelse}{\mbox{\bf \ else }}
\newcommand{\rmend}{\mbox{\bf end}}
\newcommand{\rmendif}{\mbox{\bf \ endif}}
\newcommand{\rmotherwise}{\mbox{\bf otherwise}}
\newcommand{\rmwith}{\mbox{\bf \ with\ }}
\newcommand{\aless}{\mathrel{\mbox{\lower.9ex\hbox{$\stackrel{\textstyle <}{\sim}$}}}}
\newcommand{\amore}{\mathrel{\mbox{\lower.9ex\hbox{$\stackrel{\textstyle >}{\sim}$}}}}
\newcommand{\seqarrow}{\mathrel{\mbox{\boldmath $\rightarrow$}}}

%% %% Derived from John Rushby's prelude.tex, modified for NFSS2
%
% define variants of the \LaTeX macro that avoid using \sc
% for use in headings
%

% Define fonts that work in math or text mode
\def\dwimrm#1{\ifmmode\mathrm{#1}\else\textrm{#1}\fi}
\def\dwimsf#1{\ifmmode\mathsf{#1}\else\textsf{#1}\fi}
\def\dwimtt#1{\ifmmode\mathtt{#1}\else\texttt{#1}\fi}
\def\dwimbf#1{\ifmmode\mathbf{#1}\else\textbf{#1}\fi}
\def\dwimit#1{\ifmmode\mathit{#1}\else\textit{#1}\fi}
\def\dwimnormal#1{\ifmmode\mathnormal{#1}\else\textnormal{#1}\fi}

\def\BigLaTeX{{\rm L\kern-.36em\raise.3ex\hbox{\small\small A}\kern-.15em
    T\kern-.1667em\lower.7ex\hbox{E}\kern-.125emX}}
\def\BoldLaTeX{{\bf L\kern-.36em\raise.3ex\hbox{\small\small\bf A}\kern-.15em
    T\kern-.1667em\lower.7ex\hbox{E}\kern-.125emX}}
%\def\labelitemi{$\bullet$}
\def\labelitemii{$\circ$}
\def\labelitemiii{$\star$}
\def\labelitemiv{$\diamond$}
\newcommand{\tcc}{{\small\small TCC}}
\newcommand{\tccs}{\tcc s}
\newcommand{\emacs}{{Emacs}}
\newcommand{\Emacs}{\emacs}
\newcommand{\ehdm}{{E{\small\small HDM}}}
\newcommand{\Ehdm}{\ehdm}
\newcommand{\tm}{$^{\mbox{\tiny TM}}$}
\newcommand{\hozline}{{\noindent\rule{\textwidth}{0.4mm}}}

\newcommand{\allclear}%
  {\mbox{\boldmath$\stackrel{\raisebox{-.2ex}[0pt][0pt]%
              {$\textstyle\oslash$}}{\displaystyle\bot}$}}

\newenvironment{private}{}{}

\newenvironment{smalltt}{\begin{alltt}\small}{\end{alltt}}

\newlength{\hsbw}

\newenvironment{session}%
  {\begin{flushleft}
   \setlength{\hsbw}{\linewidth}
   \addtolength{\hsbw}{-\arrayrulewidth}
   \addtolength{\hsbw}{-\tabcolsep}
   \begin{tabular}{@{}|c@{}|@{}}\hline 
   \begin{minipage}[b]{\hsbw}
   \begingroup\small\mbox{ }\\[-1.8\baselineskip]\begin{alltt}}
  {\end{alltt}\endgroup\end{minipage}\\ \hline 
   \end{tabular}
   \end{flushleft}}

\newenvironment{smallsession}%
  {\begin{flushleft}
   \setlength{\hsbw}{\linewidth}
   \addtolength{\hsbw}{-\arrayrulewidth}
   \addtolength{\hsbw}{-\tabcolsep}
   \begin{tabular}{@{}|c@{}|@{}}\hline 
   \begin{minipage}[b]{\hsbw}
   \begingroup\footnotesize\mbox{ }\\[-1.8\baselineskip]\begin{alltt}}%
  {\end{alltt}\endgroup\end{minipage}\\ \hline 
   \end{tabular}
   \end{flushleft}}

\newenvironment{spec}%
  {\begin{flushleft}
   \setlength{\hsbw}{\textwidth}
   \addtolength{\hsbw}{-\arrayrulewidth}
   \addtolength{\hsbw}{-\tabcolsep}
   \begin{tabular}{@{}|c@{}|@{}}\hline 
   \begin{minipage}[b]{\hsbw}
   \begingroup\small\mbox{ }\\[-0.2\baselineskip]}%
  {\endgroup\end{minipage}\\ \hline 
   \end{tabular}
   \end{flushleft}}

\newcommand{\memo}[1]%
  {\mbox{}\par\vspace{0.25in}%
   \setlength{\hsbw}{\linewidth}\addtolength{\hsbw}{-1.5ex}%
   \noindent\fbox{\parbox{\hsbw}{{\bf Memo: }#1}}\vspace{0.25in}}

\newcommand{\nb}[1]%
  {\mbox{}\par\vspace{0.25in}%
   \setlength{\hsbw}{\linewidth}\addtolength{\hsbw}{-1.5ex}%
   \noindent\fbox{\parbox{\hsbw}{{\bf Note: }#1}}\vspace{0.25in}}

\newcommand{\comment}[1]{}
\newcommand{\exfootnote}[1]{}
%\newcommand{\ifelse}[2]{#1}
\sloppy
\clubpenalty=100000
\widowpenalty=100000
%\displaywidowpenalty=100000
\setcounter{secnumdepth}{3} 
\setcounter{tocdepth}{3}
\setcounter{topnumber}{9}
\setcounter{bottomnumber}{9}
\setcounter{totalnumber}{9}
\renewcommand{\topfraction}{.99}
\renewcommand{\bottomfraction}{.99}
\renewcommand{\floatpagefraction}{.01}
\renewcommand{\textfraction}{.2}
\font\largett=cmtt10 scaled\magstep1
\font\Largett=cmtt10 scaled\magstep2
\font\hugett=cmtt10 scaled\magstep3

\def\labelitemii{$\circ$}
\def\labelitemiii{$\star$}
\def\labelitemiv{$\diamond$}
\newcommand{\tcc}{{\small\small TCC}}
\newcommand{\tccs}{\tcc s}

%\renewcommand{\memo}[1]{\mbox{}\par\vspace{0.25in}\noindent\fbox{\parbox{.95\linewidth}{{\bf Memo: }#1}}\vspace{0.25in}}

\newcommand{\eg}{{\em e.g.\/},}
\newcommand{\ie}{{\em i.e.\/},}

\newcommand{\pvs}{PVS}

\newcommand{\ch}{\choice}
\newcommand{\rsv}[1]{{\rm\tt #1}}

\newcommand{\lpvstheory}[3]{\figurehead{\hozline\smaller\smaller\begin{alltt}}%
                           \figuretail{\end{alltt}\vspace{-0in}\hozline}%
                           \figurelabel{#3}\figurecap{#2}%
                           \begin{longfigure}\input{#1}\end{longfigure}}

\newcommand{\bpvstheory}[3]
{\begin{figure}[b]\begin{boxedminipage}{\textwidth}%
      {\smaller\smaller\begin{alltt} \input{#1}\end{alltt}}\end{boxedminipage}%
    \caption{#2}\label{#3}\end{figure}}

\newcommand{\spvstheory}[1]
{\vspace{0.1in}\par\noindent\begin{boxedminipage}{\textwidth}%
    {\smaller\smaller\begin{alltt} \input{#1}\end{alltt}}\end{boxedminipage}\vspace{0.1in}%
}

\CustomVerbatimEnvironment{pvsex}{Verbatim}{commandchars=\\\{\},frame=single,fontsize=\relsize{-1}}

\CustomVerbatimCommand{\pvsinput}{VerbatimInput}{commandchars=\\\{\},frame=single,fontsize=\relsize{-1}}

\newcommand{\pvstheory}[3]
  {\begin{figure}[htb]%
      \pvsinput{#1}%
      \caption{#2}\label{#3}%
    \end{figure}}

% \newenvironment{pvsex}%
%   {\setlength{\topsep}{0in}\smaller\begin{alltt}}%
%   {\end{alltt}}

\newcommand{\pvsbnf}[2]
  {\begin{figure}[htb]\begin{boxedminipage}{\textwidth}%
   \input{#1}\end{boxedminipage}\caption{#2}\label{#1}\end{figure}}

\newcommand{\spvsbnf}[1]
  {\begin{boxedminipage}{\textwidth}\input{#1}\end{boxedminipage}}

\newcommand{\pidx}[1]{{\rm #1}} % primary index entry
\newcommand{\sidx}[1]{{\rm #1}} % secondary index entry
\newcommand{\cmdindex}[1]{\index{#1@\cmd{#1}}}
\newcommand{\icmd}[1]{\cmd{#1}\cmdindex{#1}}
\newcommand{\iecmd}[1]{\ecmd{#1}\cmdindex{#1}}
\newcommand{\buf}[1]{\texttt{#1}}
\newcommand{\ibuf}[1]{\buf{#1}\index{#1 buffer@\buf{#1} buffer}\index{buffers!\buf{#1}}}

\newenvironment{pvscmds}%
  {\par\noindent\smaller%
   \begin{tabular*}{\textwidth}{|l@{\extracolsep{\fill}}l@{\extracolsep{\fill}}l|}\hline%
     {\it Command} & {\it Aliases} & {\it Function}\\ \hline}%
  {\hline\end{tabular*}\vspace{0.1in}}

\newenvironment{pvscmdsna}%
  {\par\noindent\smaller%
   \begin{tabular*}{\textwidth}{|l@{\extracolsep{\fill}}l|}\hline%
     {\it Command} & {\it \,\,Function}\\ \hline}%
  {\hline\end{tabular*}\vspace{0.1in}}

\newcommand{\cmd}[1]{\texttt{#1}}
\newcommand{\ecmd}[1]{{\tt M-x #1}}

\newcommand{\latex}{\LaTeX}                  %  LaTeX
\newcommand{\sun}{{S{\smaller\smaller UN}}}                 %  Sun
\newcommand{\sparc}{{S{\smaller\smaller PARC}}}             %  Sparc
\newcommand{\sunos}{{S{\smaller\smaller UN}OS}}             %  SunOS
\newcommand{\solaris}{{\em Solaris\/}}        %  Solaris
\newcommand{\sunview}{{S{\smaller\smaller UN}V{\smaller\smaller IEW}}} %SunView
\newcommand{\unix}{{U{\smaller\smaller NIX}}}               %  Unix
\newcommand{\lisp} {{\sc Lisp}}              %  Lisp
\newcommand{\gnu}{{Gnu Emacs}}           %  Gnu Emacs
\newcommand{\gnuemacs}{{Gnu Emacs}}      %  Gnu Emacs
\newcommand{\emacsl}{{Emacs-Lisp}}       %  Emacs Lisp
\newcommand{\shell}{{\sc Csh}}               %  C-shell

\newcommand{\update}[3]{#1\{#2\leftarrow #3\}}
\newcommand{\interp}[3]{\cal{M}\dlb {\tt #1 : #2 }\drb #3}
\newcommand{\myforall}[2]{(\forall{#1 .}\ #2)}
\newcommand{\myexists}[2]{(\exists{#1 .}\ #2)}
\newcommand{\mth}[1]{$ #1 $}
\newcommand{\labst}[2]{(\lambda{#1}.\ #2)}
\newcommand{\app}[2]{(#1\ #2)}
\newcommand{\problem}[1]{{\bf Exercise: } {\em #1}}
\newcommand{\rectype}[1]{[\# 1 \#]}
\newcommand{\recttype}[1]{{\tt [\# 1 \#]}}
\newcommand{\dlb}{\lbrack\!\lbrack}
\newcommand{\drb}{\rbrack\!\rbrack}
\newcommand{\cross}{\times}
\newcommand{\key}[1]{{\tt #1}}
\newcommand{\keyindex}[1]{\index{#1@\key{#1}}}
\newcommand{\ikey}[1]{\key{#1}\keyindex{#1}}
\newcommand{\keyword}[1]{{\smaller\texttt{#1}}}

\newenvironment{keybindings}%
  {\begin{center}\begin{tabular}{|l|l|}\hline Key & Function\\ \hline}%
  {\hline\end{tabular}\end{center}}
\def\rmif{\mbox{\bf if\ }}
\def\rmiff{\mbox{\bf \ iff \ }}
\def\rmthen{\mbox{\bf \ then }}
\def\rmelse{\mbox{\bf \ else }}
\def\rmend{\mbox{\bf end}}
\def\rmendif{\mbox{\bf \ endif}}
\def\rmotherwise{\mbox{\bf otherwise}}
\def\rmwith{\mbox{\bf \ with\ }}
\def\mapb{\char"7B\char"7B}
\def\mape{\char"7D\char"7D}
\def\setb{\char"7B}
\def\sete{\char"7D}

% ---------------------------------------------------------------------
% Macros for little PVS sessions displayed in boxes.
%
% Usage: (1) \setcounter{sessioncount}{1} resets the session counter
%
%        (2) \begin{session*}\label{thissession}
%             .
%              < lines from PVS session >
%             .
%            \end{session*}
%
%            typesets the session in a numbered box in ALLTT mode.
%
%  session instead of session* produces unnumbered boxes
%
%  Author: John Rushby
% ---------------------------------------------------------------------
\newlength{\hsbw}
\newenvironment{session}{\begin{flushleft}
 \setlength{\hsbw}{\linewidth}
 \addtolength{\hsbw}{-\arrayrulewidth}
 \addtolength{\hsbw}{-\tabcolsep}
 \begin{tabular}{@{}|c@{}|@{}}\hline 
 \begin{minipage}[b]{\hsbw}
% \begingroup\small\mbox{ }\\[-1.8\baselineskip]\begin{alltt}}{\end{alltt}\endgroup\end{minipage}\\ \hline
 \begingroup\sessionsize\vspace*{1.2ex}\begin{alltt}}{\end{alltt}\endgroup\end{minipage}\\ \hline
 \end{tabular}
 \end{flushleft}}
\newcounter{sessioncount}
\setcounter{sessioncount}{0}
\newenvironment{session*}{\begin{flushleft}
 \refstepcounter{sessioncount}
 \setlength{\hsbw}{\linewidth}
 \addtolength{\hsbw}{-\arrayrulewidth}
 \addtolength{\hsbw}{-\tabcolsep}
 \begin{tabular}{@{}|c@{}|@{}}\hline 
 \begin{minipage}[b]{\hsbw}
 \vspace*{-.5pt}
 \begin{flushright}
 \rule{0.01in}{.15in}\rule{0.3in}{0.01in}\hspace{-0.35in}
 \raisebox{0.04in}{\makebox[0.3in][c]{\footnotesize \thesessioncount}}
 \end{flushright}
 \vspace*{-.57in}
 \begingroup\small\vspace*{1.0ex}\begin{alltt}}{\end{alltt}\endgroup\end{minipage}\\ \hline 
 \end{tabular}
 \end{flushleft}}
\def\sessionsize{\footnotesize}
\def\smallsessionsize{\small}
\newenvironment{smallsession}{\begin{flushleft}
 \setlength{\hsbw}{\linewidth}
 \addtolength{\hsbw}{-\arrayrulewidth}
 \addtolength{\hsbw}{-\tabcolsep}
 \begin{tabular}{@{}|c@{}|@{}}\hline 
 \begin{minipage}[b]{\hsbw}
 \begingroup\smallsessionsize\mbox{ }\\[-1.8\baselineskip]\begin{alltt}}{\end{alltt}\endgroup\end{minipage}\\ \hline 
 \end{tabular}
 \end{flushleft}}
\newenvironment{spec}{\begin{flushleft}
 \setlength{\hsbw}{\textwidth}
 \addtolength{\hsbw}{-\arrayrulewidth}
 \addtolength{\hsbw}{-\tabcolsep}
 \begin{tabular}{@{}|c@{}|@{}}\hline 
 \begin{minipage}[b]{\hsbw}
 \begingroup\small\mbox{
}\\[-0.2\baselineskip]}{\endgroup\end{minipage}\\ \hline 
 \end{tabular}
 \end{flushleft}}
\newcommand{\memo}[1]{\mbox{}\par\vspace{0.25in}%
\setlength{\hsbw}{\linewidth}%
\addtolength{\hsbw}{-2\fboxsep}%
\addtolength{\hsbw}{-2\fboxrule}%
\noindent\fbox{\parbox{\hsbw}{{\bf Memo: }#1}}\vspace{0.25in}}
\newcommand{\nb}[1]{\mbox{}\par\vspace{0.25in}\setlength{\hsbw}{\linewidth}\addtolength{\hsbw}{-1.5ex}\noindent\fbox{\parbox{\hsbw}{{\bf Note: }#1}}\vspace{0.25in}}

%%% Local Variables: 
%%% mode: latex
%%% TeX-master: t
%%% End: 

%
%
%    LaTeX  Macro File  /usr2/jcm/tex/macros.tex
%
%
\tracingonline=0 % shorter error messages (on screen)

%\long\def\comment#1{} % mulit-line comments
%\newcommand{\note}[1]{\fbox{#1}}

% font changes  (as function calls, scribe style)

\renewcommand{\i}[1]{{\it #1\/}}       % italics with space correction
%\newcommand{\emex}[1]{\/{\em #1}}   % emphasis in example, exercise, theorem, etc.
\renewcommand{\c}[1]{{\sc #1}}       % small caps (eliminates \c as cedilla)
%\newcommand{\r}[1]{{\rm #1}}            % for roman font in math mode
%\newcommand{\s}[1]{{\scr #1}}          % script (for use with tatex) 
\newcommand{\s}[1]{{\cal #1}}
\renewcommand{\b}[1]{{\bf #1}}           % bold face
\newcommand{\calg}[1]{{\cal #1}}          % caligraphic 

%\newcommand{\q}[1]{``#1''}    % matching quotes for in-line quotation

% numbered environments

%\newcounter{partcounter}
%\setcounter{partcounter}{0}
%\renewcommand{\part}[1]{\newpage \addtocounter{partcounter}{1}
%\noindent{\Large \bf Part \Roman{partcounter}. #1 } \\[1ex]}

\newtheorem{thm}{Theorem}[section]
\newtheorem{theorem}[thm]{Theorem}
\newtheorem{lemma}[thm]{Lemma}
\newtheorem{cor}[thm]{Corollary}
\newtheorem{corollary}[thm]{Corollary}
\newtheorem{claim}[thm]{Claim}
\newtheorem{prop}[thm]{Proposition}
\newtheorem{conj}[thm]{Conjecture}
\newtheorem{definition}[thm]{Definition}
\newtheorem{exercise}[thm]{Exercise}
\newtheorem{example}[thm]{Example}
\newtheorem{remark}[thm]{Remark}
\newtheorem{open}[thm]{Open Problem}

%\newcommand{\proof}{\\{\bf Proof.}\ }
\newenvironment{proof}{{\bf Proof. }}{\thmbox}

% axiom and inference rule (centered, in math mode,  with name at left)

\newcommand{\axiom}[2]
{\[ \hbox to \columnwidth
    { \rlap{$#2$} \hfil {$ #1 $} \hfil }
\]}

\newcommand{\infrule}[4]
{\[ \hbox to \columnwidth %\textwidth
    { \rlap{$#4$} \hfil $
      \frac {\strut\displaystyle #1 } {\strut\displaystyle #2 } \; \rlap{$#3$} \hfil $
      \hfil }
\]}
\newcommand{\infruletw}[4]
{\[ \hbox to \textwidth
    { \rlap{$#4$} \hfil $
      \frac {\strut\displaystyle #1 } {\strut\displaystyle #2 } \; \rlap{$#3$} \hfil $
      \hfil }
\]}

%\newcommand{\infrule}[4]
%{\[ \hbox to \columnwidth
%    { \rlap{$#4$} \hfil $
%      {{\displaystyle\strut #1}\over{\displaystyle\strut #2}}\quad\makebox[0pt][l]{\it #3} $
%      \hfil }
%\]}

% axiom and inference rule macros for use in tables, etc.
% presumed in math mode #1=axiom, #2=side condition
\newcommand{\Axiom}[2]{
{\displaystyle\strut #1}\qquad\makebox[0pt][l]{\it #2}
}
% presumed in math mode #1=top, #2=bottom, #3=side condition
\newcommand{\Infrule}[3]{
{{\displaystyle\strut #1}\over{\displaystyle\strut #2}}\;\mbox{\scriptsize$\bf #3$}
}

% sequence of #1's, numbered up to #2  (e.g., seq{x}{n} for x1, ..., xn   )

\newcommand{\seq}[2]{#1_{1} \ldots #1_{#2}}  
%\newcommand{\ie}{{\it i.e.}}
%\newcommand{\eg}{{\it e.g.}}
%\newcommand{\cf}{{\it c.f.\,}}

% common symbols

\newcommand{\union}{\cup}
\newcommand{\intersect}{\cap}
\newcommand{\subs}{\subseteq}
\newcommand{\el}{\in}
\newcommand{\nel}{\not\in}
\newcommand{\ns}{\emptyset}
\newcommand{\compose}{\circ}
\newcommand{\set}[2]{ \{\, #1 \,\mid\, #2 \,\}  } % set macro
\newcommand{\infinity}{\infty}
\newcommand{\pair}[1]{\langle #1 \rangle}
\newcommand{\tuple}[1]{\langle #1 \rangle}


\newcommand{\fa}{\forall}
\newcommand{\te}{\exists}
\newcommand{\imp}{\supset}
\renewcommand{\implies}{\supset}
\newcommand{\ts}{\vdash}
\newcommand{\dts}{\models}

\newcommand{\aro}{\mathord\rightarrow} % see pages 154-155 of TeX manual
\newcommand{\paro}{\rightharpoonup} 
\newcommand{\karo}{\mathop\Rightarrow} % see pages 154-155 of TeX manual
%\newcommand{\cross}{\times}
%\newcommand{\dlb}{\lbrack\!\lbrack}
%\newcommand{\drb}{\rbrack\!\rbrack}
\newcommand{\mean}[1]{\lbrack\!\lbrack #1 \rbrack\!\rbrack}
\newcommand{\lam}{\lambda}
\newcommand{\subst}[2]{{}[#1/#2]}
\renewcommand{\dot}{\mathrel{\bullet}}
\newcommand{\Dinf}{D_{\infty}}
\newcommand{\bottom}{\perp}

\mathcode`:="603A  % treat : as punctuation instead of relation in math mode
\mathchardef\colon="303A	% relation colon

\newcommand{\eqdef}{\mathrel{:=}}
\newcommand{\Dom}{\mathop{\rm dom}}
\newcommand{\Pow}{\mathop{\rm Pow}}

\newcommand{\aequiv}{\equiv_\alpha}
\newcommand{\baro}{\buildrel \beta \over \rightarrow}
\newcommand{\earo}{\buildrel \eta \over \rightarrow}
\newcommand{\red}{\rightarrow\!\!\!\!\rightarrow}
\newcommand{\backred}{\leftarrow\!\!\!\!\leftarrow}
\newcommand{\bred}{\buildrel \beta \over \red}
\newcommand{\ered}{\buildrel \eta \over \red}
\newcommand{\conv}{\leftrightarrow}
\newcommand{\beconv}{\buildrel {\beta, \eta} \over\leftrightarrow}

% lambda calculus abbreviations
\newcommand{\letdec}[3]{\b{let\ } #1 = #2 \b{\ in\ } #3}
\newcommand{\letrec}[3]{\b{letrec\ } #1 = #2 \b{\ in\ } #3}

\newcommand{\thmbox}
   {{\ \hfill\hbox{%
      \vrule width1.0ex height1.0ex
   }\parfillskip 0pt }}
\newcommand{\qed}{\thmbox}


% make single spacing

\newcommand{\singlespace}{\renewcommand{\baselinestretch}{1}\@normalsize}
\newcommand{\etal}{{\em et al.}}

% bycase command for definition by cases (pg 49 of LaTeX)

\newcommand{\bycase}[1]
	{\left\{ \begin{array}{ll}  #1  \end{array} \right. }


% make \cite put blanks after the comma (use in alpha style)

%\def\@citex[#1]#2{\if@filesw\immediate\write\@auxout{\string\citation{#2}}\fi
%  \def\@citea{}\@cite{\@for\@citeb:=#2\do
%    {\@citea\def\@citea{,\penalty100\hskip2.5pt plus1.5pt minus.8pt}%
%       \@ifundefined{b@\@citeb}{{\bf ?}\@warning
%       {Citation `\@citeb' on page \thepage \space undefined}}
%\hbox{\csname b@\@citeb\endcsname}}}{#1}}

 % zero-width inference rule markers
 % (if they have width, they throw off centering)
\newcommand{\zwb}[1]{\makebox[0cm][l]{{\scriptsize $#1$}}}

\newcommand{\mall}{{\sc mall}}
\newcommand{\ol}{\overline}
\newcommand{\ul}{\underline}
\newcommand{\lan}{\langle}
\newcommand{\ran}{\rangle}

 % Linear Arrow
\newcommand{\la}{\mbox{$-\!\circ$}} 

 % Backwards Linear Arrow
\newcommand{\bla}{\mbox{$\circ\!-$}} 

 % Par
%\newcommand{\Par}{\,{\parfont P}\,}
\newcommand{\Par}{\mathrel{\wp}}

 % With (Linear And) (&)
\newcommand{\with}{\,{\&}\,}

% Old par
% \newcommand{\Par}{\mbox{\raisebox{.4ex}{$\wp$}}} 

 % PDL's prettier bounded reuse operator
\newcommand{\bang}[1]{\stackrel{\scriptstyle \dagger}{\scriptstyle #1}}
 
 % Implies in Relevant Implication
\newcommand{\ra}{\mbox{$\rightarrow_{R_{\rightarrow}}$}}

 % Relevant Implication (Logic)
\newcommand{\rara}{\mbox{${R_{\rightarrow}}$}}

 % Multiplicative Linear Logic  
\newcommand{\lala}{\mbox{$LL_{\la, \otimes, !}$}}

 % (Propositional) Circular Logic
\newcommand{\cl}{\mbox{ CL }}

 % (Propositional) Multiplicative Circular Logic
\newcommand{\mcl}{\mbox{ MCL }}

 % Definition
\newcommand{\define}{\mbox{${\stackrel{\Delta}{=}}$}}

 % Derived From
\newcommand{\derive}{\mbox{${\stackrel{\vdots}{\vdash}}$}}

 % Semi-Thue Reduction
\newcommand{\stred}[1]{\mbox{$\Longrightarrow$}}

\newcommand{\stredstar}[1]{\mbox{$\Longrightarrow^*$}}

\newcommand{\deltared}[1]{\mbox{$\, \stackrel{\delta_{#1}}
                                             {\longrightarrow} \, $}}

\newcommand{\deltaredp}[1]{\mbox{$\, \stackrel{\delta'_{#1}}
                                             {\longrightarrow} \, $}}


\newenvironment{commentary}{\begin{quote}\small{\bf Commentary:}}{{\bf
End of Commentary}\normalsize\end{quote}} 
%\newenvironment{proof}{{\bf Proof:}}{$\Box$\\}
%\newtheorem{theorem}{Theorem}
%\newtheorem{lemma}{Lemma}
%\newtheorem{proposition}{Proposition}
%\newtheorem{definition}{Definition}
%\newtheorem{corollary}{Corollary}
%\newtheorem{example}{Example}
\newenvironment{mytheorem}{\begin{theorem}\rm}{\end{theorem}}
%\newcommand{\qed}[0] {\rule{1ex}{1ex} \vspace{2ex}}
\newcommand{\defdef}{\mbox{$\stackrel{\rm def}{=}$}}
\def\defn{\mathrel{\defdef}}
%\newcommand{\defn}{\mbox{$\stackrel{\rm def}{=}$}}
\newcommand{\intro}[1]{\begin{itemize}\item #1 \end{itemize}}
\newcommand{\simeqrel}[1]{\stackrel{#1}{\simeq}}
\newcommand{\eqrel}[1]{\stackrel{#1}{\approx}}
\newcommand{\simrel}[1]{\stackrel{#1}{\sim}}
\newcommand{\congrel}[1]{\stackrel{#1}{\cong}}
\def\maj{\mathop{\rm maj}}
\def\min{\mathop{\rm min}}
\def\rem{\mathop{\rm rem}}
\def\div{\mathop{\rm div}}
\newcommand{\rmand}{\mbox{\bf \ and }}
\newcommand{\rmif}{\mbox{\bf if\ }}
\newcommand{\rmiff}{\mbox{\bf \ iff \ }}
\newcommand{\rmthen}{\mbox{\bf \ then }}
\newcommand{\rmelse}{\mbox{\bf \ else }}
\newcommand{\rmend}{\mbox{\bf end}}
\newcommand{\rmendif}{\mbox{\bf \ endif}}
\newcommand{\rmotherwise}{\mbox{\bf otherwise}}
\newcommand{\rmwith}{\mbox{\bf \ with\ }}
\newcommand{\aless}{\mathrel{\mbox{\lower.9ex\hbox{$\stackrel{\textstyle <}{\sim}$}}}}
\newcommand{\amore}{\mathrel{\mbox{\lower.9ex\hbox{$\stackrel{\textstyle >}{\sim}$}}}}
\newcommand{\seqarrow}{\mathrel{\mbox{\boldmath $\rightarrow$}}}

%% Derived from John Rushby's prelude.tex, modified for NFSS2
%
% define variants of the \LaTeX macro that avoid using \sc
% for use in headings
%

% Define fonts that work in math or text mode
\def\dwimrm#1{\ifmmode\mathrm{#1}\else\textrm{#1}\fi}
\def\dwimsf#1{\ifmmode\mathsf{#1}\else\textsf{#1}\fi}
\def\dwimtt#1{\ifmmode\mathtt{#1}\else\texttt{#1}\fi}
\def\dwimbf#1{\ifmmode\mathbf{#1}\else\textbf{#1}\fi}
\def\dwimit#1{\ifmmode\mathit{#1}\else\textit{#1}\fi}
\def\dwimnormal#1{\ifmmode\mathnormal{#1}\else\textnormal{#1}\fi}

\def\BigLaTeX{{\rm L\kern-.36em\raise.3ex\hbox{\small\small A}\kern-.15em
    T\kern-.1667em\lower.7ex\hbox{E}\kern-.125emX}}
\def\BoldLaTeX{{\bf L\kern-.36em\raise.3ex\hbox{\small\small\bf A}\kern-.15em
    T\kern-.1667em\lower.7ex\hbox{E}\kern-.125emX}}
%\def\labelitemi{$\bullet$}
\def\labelitemii{$\circ$}
\def\labelitemiii{$\star$}
\def\labelitemiv{$\diamond$}
\newcommand{\tcc}{{\small\small TCC}}
\newcommand{\tccs}{\tcc s}
\newcommand{\emacs}{{Emacs}}
\newcommand{\Emacs}{\emacs}
\newcommand{\ehdm}{{E{\small\small HDM}}}
\newcommand{\Ehdm}{\ehdm}
\newcommand{\tm}{$^{\mbox{\tiny TM}}$}
\newcommand{\hozline}{{\noindent\rule{\textwidth}{0.4mm}}}

\newcommand{\allclear}%
  {\mbox{\boldmath$\stackrel{\raisebox{-.2ex}[0pt][0pt]%
              {$\textstyle\oslash$}}{\displaystyle\bot}$}}

\newenvironment{private}{}{}

\newenvironment{smalltt}{\begin{alltt}\small}{\end{alltt}}

\newlength{\hsbw}

\newenvironment{session}%
  {\begin{flushleft}
   \setlength{\hsbw}{\linewidth}
   \addtolength{\hsbw}{-\arrayrulewidth}
   \addtolength{\hsbw}{-\tabcolsep}
   \begin{tabular}{@{}|c@{}|@{}}\hline 
   \begin{minipage}[b]{\hsbw}
   \begingroup\small\mbox{ }\\[-1.8\baselineskip]\begin{alltt}}
  {\end{alltt}\endgroup\end{minipage}\\ \hline 
   \end{tabular}
   \end{flushleft}}

\newenvironment{smallsession}%
  {\begin{flushleft}
   \setlength{\hsbw}{\linewidth}
   \addtolength{\hsbw}{-\arrayrulewidth}
   \addtolength{\hsbw}{-\tabcolsep}
   \begin{tabular}{@{}|c@{}|@{}}\hline 
   \begin{minipage}[b]{\hsbw}
   \begingroup\footnotesize\mbox{ }\\[-1.8\baselineskip]\begin{alltt}}%
  {\end{alltt}\endgroup\end{minipage}\\ \hline 
   \end{tabular}
   \end{flushleft}}

\newenvironment{spec}%
  {\begin{flushleft}
   \setlength{\hsbw}{\textwidth}
   \addtolength{\hsbw}{-\arrayrulewidth}
   \addtolength{\hsbw}{-\tabcolsep}
   \begin{tabular}{@{}|c@{}|@{}}\hline 
   \begin{minipage}[b]{\hsbw}
   \begingroup\small\mbox{ }\\[-0.2\baselineskip]}%
  {\endgroup\end{minipage}\\ \hline 
   \end{tabular}
   \end{flushleft}}

\newcommand{\memo}[1]%
  {\mbox{}\par\vspace{0.25in}%
   \setlength{\hsbw}{\linewidth}\addtolength{\hsbw}{-1.5ex}%
   \noindent\fbox{\parbox{\hsbw}{{\bf Memo: }#1}}\vspace{0.25in}}

\newcommand{\nb}[1]%
  {\mbox{}\par\vspace{0.25in}%
   \setlength{\hsbw}{\linewidth}\addtolength{\hsbw}{-1.5ex}%
   \noindent\fbox{\parbox{\hsbw}{{\bf Note: }#1}}\vspace{0.25in}}

\newcommand{\comment}[1]{}
\newcommand{\exfootnote}[1]{}
%\newcommand{\ifelse}[2]{#1}
\sloppy
\clubpenalty=100000
\widowpenalty=100000
%\displaywidowpenalty=100000
\setcounter{secnumdepth}{3} 
\setcounter{tocdepth}{3}
\setcounter{topnumber}{9}
\setcounter{bottomnumber}{9}
\setcounter{totalnumber}{9}
\renewcommand{\topfraction}{.99}
\renewcommand{\bottomfraction}{.99}
\renewcommand{\floatpagefraction}{.01}
\renewcommand{\textfraction}{.2}
\font\largett=cmtt10 scaled\magstep1
\font\Largett=cmtt10 scaled\magstep2
\font\hugett=cmtt10 scaled\magstep3

\def\labelitemii{$\circ$}
\def\labelitemiii{$\star$}
\def\labelitemiv{$\diamond$}
\newcommand{\tcc}{{\small\small TCC}}
\newcommand{\tccs}{\tcc s}

%\renewcommand{\memo}[1]{\mbox{}\par\vspace{0.25in}\noindent\fbox{\parbox{.95\linewidth}{{\bf Memo: }#1}}\vspace{0.25in}}

\newcommand{\eg}{{\em e.g.\/},}
\newcommand{\ie}{{\em i.e.\/},}

\newcommand{\pvs}{PVS}

\newcommand{\ch}{\choice}
\newcommand{\rsv}[1]{{\rm\tt #1}}

\newcommand{\lpvstheory}[3]{\figurehead{\hozline\smaller\smaller\begin{alltt}}%
                           \figuretail{\end{alltt}\vspace{-0in}\hozline}%
                           \figurelabel{#3}\figurecap{#2}%
                           \begin{longfigure}\input{#1}\end{longfigure}}

\newcommand{\bpvstheory}[3]
{\begin{figure}[b]\begin{boxedminipage}{\textwidth}%
      {\smaller\smaller\begin{alltt} \input{#1}\end{alltt}}\end{boxedminipage}%
    \caption{#2}\label{#3}\end{figure}}

\newcommand{\spvstheory}[1]
{\vspace{0.1in}\par\noindent\begin{boxedminipage}{\textwidth}%
    {\smaller\smaller\begin{alltt} \input{#1}\end{alltt}}\end{boxedminipage}\vspace{0.1in}%
}

\CustomVerbatimEnvironment{pvsex}{Verbatim}{commandchars=\\\{\},frame=single,fontsize=\relsize{-1}}

\CustomVerbatimCommand{\pvsinput}{VerbatimInput}{commandchars=\\\{\},frame=single,fontsize=\relsize{-1}}

\newcommand{\pvstheory}[3]
  {\begin{figure}[htb]%
      \pvsinput{#1}%
      \caption{#2}\label{#3}%
    \end{figure}}

% \newenvironment{pvsex}%
%   {\setlength{\topsep}{0in}\smaller\begin{alltt}}%
%   {\end{alltt}}

\newcommand{\pvsbnf}[2]
  {\begin{figure}[htb]\begin{boxedminipage}{\textwidth}%
   \input{#1}\end{boxedminipage}\caption{#2}\label{#1}\end{figure}}

\newcommand{\spvsbnf}[1]
  {\begin{boxedminipage}{\textwidth}\input{#1}\end{boxedminipage}}

\newcommand{\pidx}[1]{{\rm #1}} % primary index entry
\newcommand{\sidx}[1]{{\rm #1}} % secondary index entry
\newcommand{\cmdindex}[1]{\index{#1@\cmd{#1}}}
\newcommand{\icmd}[1]{\cmd{#1}\cmdindex{#1}}
\newcommand{\iecmd}[1]{\ecmd{#1}\cmdindex{#1}}
\newcommand{\buf}[1]{\texttt{#1}}
\newcommand{\ibuf}[1]{\buf{#1}\index{#1 buffer@\buf{#1} buffer}\index{buffers!\buf{#1}}}

\newenvironment{pvscmds}%
  {\par\noindent\smaller%
   \begin{tabular*}{\textwidth}{|l@{\extracolsep{\fill}}l@{\extracolsep{\fill}}l|}\hline%
     {\it Command} & {\it Aliases} & {\it Function}\\ \hline}%
  {\hline\end{tabular*}\vspace{0.1in}}

\newenvironment{pvscmdsna}%
  {\par\noindent\smaller%
   \begin{tabular*}{\textwidth}{|l@{\extracolsep{\fill}}l|}\hline%
     {\it Command} & {\it \,\,Function}\\ \hline}%
  {\hline\end{tabular*}\vspace{0.1in}}

\newcommand{\cmd}[1]{\texttt{#1}}
\newcommand{\ecmd}[1]{{\tt M-x #1}}

\newcommand{\latex}{\LaTeX}                  %  LaTeX
\newcommand{\sun}{{S{\smaller\smaller UN}}}                 %  Sun
\newcommand{\sparc}{{S{\smaller\smaller PARC}}}             %  Sparc
\newcommand{\sunos}{{S{\smaller\smaller UN}OS}}             %  SunOS
\newcommand{\solaris}{{\em Solaris\/}}        %  Solaris
\newcommand{\sunview}{{S{\smaller\smaller UN}V{\smaller\smaller IEW}}} %SunView
\newcommand{\unix}{{U{\smaller\smaller NIX}}}               %  Unix
\newcommand{\lisp} {{\sc Lisp}}              %  Lisp
\newcommand{\gnu}{{Gnu Emacs}}           %  Gnu Emacs
\newcommand{\gnuemacs}{{Gnu Emacs}}      %  Gnu Emacs
\newcommand{\emacsl}{{Emacs-Lisp}}       %  Emacs Lisp
\newcommand{\shell}{{\sc Csh}}               %  C-shell

\newcommand{\update}[3]{#1\{#2\leftarrow #3\}}
\newcommand{\interp}[3]{\cal{M}\dlb {\tt #1 : #2 }\drb #3}
\newcommand{\myforall}[2]{(\forall{#1 .}\ #2)}
\newcommand{\myexists}[2]{(\exists{#1 .}\ #2)}
\newcommand{\mth}[1]{$ #1 $}
\newcommand{\labst}[2]{(\lambda{#1}.\ #2)}
\newcommand{\app}[2]{(#1\ #2)}
\newcommand{\problem}[1]{{\bf Exercise: } {\em #1}}
\newcommand{\rectype}[1]{[\# 1 \#]}
\newcommand{\recttype}[1]{{\tt [\# 1 \#]}}
\newcommand{\dlb}{\lbrack\!\lbrack}
\newcommand{\drb}{\rbrack\!\rbrack}
\newcommand{\cross}{\times}
\newcommand{\key}[1]{{\tt #1}}
\newcommand{\keyindex}[1]{\index{#1@\key{#1}}}
\newcommand{\ikey}[1]{\key{#1}\keyindex{#1}}
\newcommand{\keyword}[1]{{\smaller\texttt{#1}}}

\newenvironment{keybindings}%
  {\begin{center}\begin{tabular}{|l|l|}\hline Key & Function\\ \hline}%
  {\hline\end{tabular}\end{center}}
\def\rmif{\mbox{\bf if\ }}
\def\rmiff{\mbox{\bf \ iff \ }}
\def\rmthen{\mbox{\bf \ then }}
\def\rmelse{\mbox{\bf \ else }}
\def\rmend{\mbox{\bf end}}
\def\rmendif{\mbox{\bf \ endif}}
\def\rmotherwise{\mbox{\bf otherwise}}
\def\rmwith{\mbox{\bf \ with\ }}
\def\mapb{\char"7B\char"7B}
\def\mape{\char"7D\char"7D}
\def\setb{\char"7B}
\def\sete{\char"7D}

% ---------------------------------------------------------------------
% Macros for little PVS sessions displayed in boxes.
%
% Usage: (1) \setcounter{sessioncount}{1} resets the session counter
%
%        (2) \begin{session*}\label{thissession}
%             .
%              < lines from PVS session >
%             .
%            \end{session*}
%
%            typesets the session in a numbered box in ALLTT mode.
%
%  session instead of session* produces unnumbered boxes
%
%  Author: John Rushby
% ---------------------------------------------------------------------
\newlength{\hsbw}
\newenvironment{session}{\begin{flushleft}
 \setlength{\hsbw}{\linewidth}
 \addtolength{\hsbw}{-\arrayrulewidth}
 \addtolength{\hsbw}{-\tabcolsep}
 \begin{tabular}{@{}|c@{}|@{}}\hline 
 \begin{minipage}[b]{\hsbw}
% \begingroup\small\mbox{ }\\[-1.8\baselineskip]\begin{alltt}}{\end{alltt}\endgroup\end{minipage}\\ \hline
 \begingroup\sessionsize\vspace*{1.2ex}\begin{alltt}}{\end{alltt}\endgroup\end{minipage}\\ \hline
 \end{tabular}
 \end{flushleft}}
\newcounter{sessioncount}
\setcounter{sessioncount}{0}
\newenvironment{session*}{\begin{flushleft}
 \refstepcounter{sessioncount}
 \setlength{\hsbw}{\linewidth}
 \addtolength{\hsbw}{-\arrayrulewidth}
 \addtolength{\hsbw}{-\tabcolsep}
 \begin{tabular}{@{}|c@{}|@{}}\hline 
 \begin{minipage}[b]{\hsbw}
 \vspace*{-.5pt}
 \begin{flushright}
 \rule{0.01in}{.15in}\rule{0.3in}{0.01in}\hspace{-0.35in}
 \raisebox{0.04in}{\makebox[0.3in][c]{\footnotesize \thesessioncount}}
 \end{flushright}
 \vspace*{-.57in}
 \begingroup\small\vspace*{1.0ex}\begin{alltt}}{\end{alltt}\endgroup\end{minipage}\\ \hline 
 \end{tabular}
 \end{flushleft}}
\def\sessionsize{\footnotesize}
\def\smallsessionsize{\small}
\newenvironment{smallsession}{\begin{flushleft}
 \setlength{\hsbw}{\linewidth}
 \addtolength{\hsbw}{-\arrayrulewidth}
 \addtolength{\hsbw}{-\tabcolsep}
 \begin{tabular}{@{}|c@{}|@{}}\hline 
 \begin{minipage}[b]{\hsbw}
 \begingroup\smallsessionsize\mbox{ }\\[-1.8\baselineskip]\begin{alltt}}{\end{alltt}\endgroup\end{minipage}\\ \hline 
 \end{tabular}
 \end{flushleft}}
\newenvironment{spec}{\begin{flushleft}
 \setlength{\hsbw}{\textwidth}
 \addtolength{\hsbw}{-\arrayrulewidth}
 \addtolength{\hsbw}{-\tabcolsep}
 \begin{tabular}{@{}|c@{}|@{}}\hline 
 \begin{minipage}[b]{\hsbw}
 \begingroup\small\mbox{
}\\[-0.2\baselineskip]}{\endgroup\end{minipage}\\ \hline 
 \end{tabular}
 \end{flushleft}}
\newcommand{\memo}[1]{\mbox{}\par\vspace{0.25in}%
\setlength{\hsbw}{\linewidth}%
\addtolength{\hsbw}{-2\fboxsep}%
\addtolength{\hsbw}{-2\fboxrule}%
\noindent\fbox{\parbox{\hsbw}{{\bf Memo: }#1}}\vspace{0.25in}}
\newcommand{\nb}[1]{\mbox{}\par\vspace{0.25in}\setlength{\hsbw}{\linewidth}\addtolength{\hsbw}{-1.5ex}\noindent\fbox{\parbox{\hsbw}{{\bf Note: }#1}}\vspace{0.25in}}

%%% Local Variables: 
%%% mode: latex
%%% TeX-master: t
%%% End: 

%\newlength{\hsbw}
\newenvironment{jmrsession}{\begin{flushleft}
 \setlength{\hsbw}{\linewidth}
 \addtolength{\hsbw}{-\arrayrulewidth}
 \addtolength{\hsbw}{-\tabcolsep}
 \begin{tabular}{@{}|c@{}|@{}}\hline 
 \begin{minipage}[b]{\hsbw}
 \begingroup\small\mbox{ }\\[-1.8\baselineskip]\begin{alltt}}{\end{alltt}\endgroup\end{minipage}\\ \hline 
 \end{tabular}
 \end{flushleft}}
\def\id#1{\hbox{\tt #1}} %changing ids from roman to tt.
\pagestyle{fancy}
\renewcommand{\sectionmark}[1]{\markright{{\em #1}}}
\renewcommand{\subsectionmark}[1]{}
%\lhead[\thepage]{\rightmark}
%\cfoot{\protect\small\bf \fbox{Beta Release}}
%\rhead[\leftmark]{\thepage}
\setcounter{secnumdepth}{2} 
\setcounter{tocdepth}{3}
\setcounter{page}{0} 
\title{{\Huge\bf A Tutorial Introduction to PVS}\\
\smaller  Presented at WIFT '95: Workshop on Industrial-Strength Formal
Specification Techniques, Boca Raton, Florida, April 1995}
\author{Judy Crow, Sam Owre, John Rushby, Natarajan Shankar, Mandayam
Srivas\thanks{Dave Stringer-Calvert provided valuable comments on
earlier versions of this tutorial, and also checked the specifications
and proofs appearing here.   Preparation of this tutorial was
partially funded by NASA Langley Research Center under Contract
NAS1-18969, and by the Advanced Research Projects Agency through NASA
Ames Research Center NASA-NAG-2-891 (Arpa order A721) to
Stanford Unversity.}\\
Computer Science Laboratory\\
SRI International\\
Menlo Park CA 94025 USA
\\ \mbox{ }\\
{\sc www}: {\tt http://www.csl.sri.com/sri-csl-fm.html}
%{\tt Rushby@csl.sri.com}
%\\Phone: +1 (415) 859-5456\ \ Fax: +1 (415) 859-2844
}
\date{Updated June 1995}
\begin{document} 
\maketitle
\begin{abstract}
This document provides an introductory example, a tutorial, and a
compact reference to the PVS verification system.  It is intended to
provide enough information to get you started using PVS, and to help
you appreciate the capabilities of the system and the purposes for
which it is suitable.

% Full documentation for PVS is available in three volumes (language,
% system, and prover), and a number of more advanced tutorial and other
% examples are also available.

\end{abstract}
\thispagestyle{empty}
\newpage
\evensidemargin 0.0in
\mbox{}
\thispagestyle{empty}
\newpage

\pagenumbering{roman}
\setcounter{page}{1} 

%\section*{Table of Contents}
\markboth{}{Contents}

\tableofcontents
\cleardoublepage

\setcounter{page}{0} 
\pagenumbering{arabic}

\section*{Overview}
\addcontentsline{toc}{section}{Overview}
\markboth{}{Overview}

PVS is a {\em verification system\/}:  an interactive environment for
writing formal specifications and checking formal proofs.  It builds on
nearly 20 years experience at SRI in building verification systems, and on
substantial experience with other systems.  The distinguishing feature of
PVS is its synergistic integration of an expressive specification
language and powerful theorem-proving capabilities.  PVS has been
applied successfully to large and difficult applications in both
academic and industrial settings.

PVS provides an expressive specification language that augments
classical higher-order logic with a sophisticated type system
containing predicate subtypes and dependent types, and with
parameterized theories and a mechanism for defining abstract datatypes
such as lists and trees.  The standard PVS types include numbers
(reals, rationals, integers, naturals, and the ordinals to
$\epsilon_{0}$), records, tuples, arrays, functions, sets, sequences,
lists, and trees, etc.  The combination of features in the PVS
type-system is very convenient for specification, but it makes
typechecking undecidable.  The PVS typechecker copes with this
undecidability by generating proof obligations for the PVS theorem
prover.  Most such proof obligations can be discharged automatically.
This liberation from purely algorithmic typechecking allows PVS to
provide relatively simple solutions to issues that are considered
difficult in some other systems (for example, accommodating partial
functions such as division within a logic of total functions), and it
allows PVS to enforce very strong checks on consistency and other
properties (such as preservation of invariants) in an entirely uniform
manner.

PVS has a powerful interactive theorem prover/proof checker.  The
basic deductive steps in PVS are large compared with many other
systems: there are atomic commands for induction, quantifier
reasoning, automatic conditional rewriting, simplification using
arithmetic and equality decision procedures and type information, and
propositional simplification using binary decision diagrams.  The PVS
proof checker manages the proof construction process by prompting the
user for a suitable command for a given subgoal.  The execution of the
given command can either generate further subgoals or complete a
subgoal and move the control over to the next subgoal in a proof.
User-defined proof strategies can be used to enhance the automation in
the proof checker.  Model-checking capabilities used for automatically
verifying temporal properties of finite-state systems have recently
been integrated into PVS\@.  PVS's automation suffices to prove many
straightforward results automatically; for hard proofs, the automation
takes care of the details and frees the user to concentrate on
directing the key steps.

PVS is implemented in Common Lisp---with ancillary functions provided
in C, Tcl/TK, and \LaTeX---and uses GNU Emacs for its interface.
Configured for Sun {\sc Sparc} Workstations running under SunOS
4.1.3,  the system is freely available under license from SRI.

PVS has been used at SRI to undertake proofs of difficult
fault-tolerant
algorithms~\cite{Lincoln&Rushby93:CAV,Lincoln&Rushby93:FTCS,Lincoln&Rushby94:FTP},
to verify the microcode for selected instructions of a complex,
pipelined, commercial microprocessor having 500,000 transistors where
seeded and unseeded errors were found~\cite{Miller&Srivas95}, to
provide an embedding for the Duration Calculus (an interval temporal
logic~\cite{Skakkebaek&Shankar94}), and for several other
applications.  PVS is installed at many sites worldwide, and is in
serious use at about a dozen of them.  There is a growing list of
significant applications undertaken using PVS by people outside SRI.
Many of these can be examined at the WWW site {\tt
http://www.csl.sri.com/sri-csl-fm.html}.

This tutorial is intended to give you an idea of the flavor of PVS, of
the opportunities created by effective mechanization of formal
methods, and an introduction to the use of the system itself.  PVS is
a big and complex system, so we can really only scratch the surface
here.  To make advanced use of the system, you should study the
manuals (there are three volumes: language~\cite{PVS:language},
prover~\cite{PVS:prover}, and system~\cite{PVS:userguide}), and some
of the more substantial applications.

There are three parts to this tutorial.
\begin{itemize}

\item {\em An Introduction to the Mechanized Analysis of Requirements
Specifications Using PVS.} This tutorial introduction shows how PVS
can be used to actively explore and analyze a simple requirements
specification.  It is intended to demonstrate the utility of
mechanized support for formal methods, and the opportunities for
validation and exploration that are created by effective
mechanization.

% \item {\em Formal Verification for Fault-Tolerant Architectures:
% Prolegomena to the Design of PVS}.  This paper from the February 1995
% issue of the IEEE Transactions on Software Engineering describes some
% of the verifications undertaken using PVS and motivates and describes
% some of the design decisions underlying PVS.  It also illustrates
% some of the more advanced language capabilities.

\item {\em Tutorial on Using PVS}.   This introduces many of
the capabilities of PVS by means of simple examples and takes you
through the process of using the system.  While it can be read as an
overview, it is best to have PVS available and to actively follow along.

\item {\em PVS Reference}.    This presents all PVS
system and prover commands, and illustrates the language constructs
in a very compact form.

\end{itemize}

A useful supplement to the material presented here
is~\cite{Owre95:prolegomena}, which describes some of the larger
verifications undertaken using PVS and also motivates and describes
some of the design decisions underlying PVS.

\part[Introduction to Mechanized Analysis of
Specifications Using PVS]{An Introduction to the Mechanized Analysis of
Requirements Specifications Using PVS}
\cleardoublepage
\markboth{Analyzing Specifications Using PVS}{}

\section{Introduction}

Simply using a formal notation does not ensure that specifications
will be correct: writing a correct formal specification is no easier
than writing a correct program or a correct description in English.
Specifications---especially {\em requirements} specifications, where
there is no higher-level specification against which they can be
verified---need to be {\em validated\/} against informal
expectations.   This is generally done by human review and inspection
(which can be very formalized processes), but with formal
specifications it is possible to do more.

The distinctive feature of {\em formal\/} specifications is that they
support formal deduction: it is possible to reduce certain questions
about a formal specification to a process that resembles calculation
and that can be checked by others or by machine.
Thus, reviews and inspections can be supplemented by {\em analyses\/}
of formal specifications, and those analyses can be mechanically checked.

In order to conduct mechanized analysis, it is necessary to support a
specification language with powerful tools including, primarily, a
theorem prover.  The needs of efficient theorem proving drive
specification language design in slightly different directions than
for unmechanized notations such as Z, but the presence of
mechanization also creates new linguistic opportunities---such as allowing
typechecking to use theorem proving---that can enhance the clarity and
precision of specifications.

PVS is a {\em verification system\/}: a specification language
tightly integrated with a powerful theorem prover and other tools.
This document is intended to serve as a first introduction to PVS: it
is not intended to teach the details of the PVS language and theorem
prover, but rather to give an appreciation of the opportunities
created by mechanized analysis in general, and of some of the
capabilities of PVS in particular.

\section{An Electronic Phone Book: Simple Version}
\markright{Phone Book: Simple Version}

Suppose we are to formally specify the requirements for an electronic phone
book, given the following informal description.\footnote{This example
is based on one by Ricky Butler and Sally Johnson of NASA
Langley~\cite{Butler&Johnson93}.}
\begin{itemize}
\item A phone book shall store the phone numbers of a city
\item It shall be possible to retrieve a phone number given a name
\item It shall be possible to add and delete entries from a phone book
\end{itemize}

Examining this description, we see that there are three types of
entities mentioned: {\em phone books}, {\em phone numbers} and {\em
names\/}; a phone book provides an association between names and
phone numbers.  We need three operations, which we can call {\em
FindPhone\/}, {\em AddPhone}, and {\em DelPhone\/}.  {\em FindPhone\/}
should take a phone book and a name and return the phone number
associated with that name.  The exact functionality of the other two
operations is less clear, so we have to make some design decisions.
We decide that {\em AddPhone} should take a phone book, a name, and a
phone number and should add the association between the name and
number to the phone book; and that {\em DelPhone\/} should take a
phone book and a name and delete the phone number associated with that
name (if any).

The next step is decide how to represent these entities and operations
in PVS\@.  If we were programming, we would have to choose some specific
representations for phone numbers and names---e.g., ascii strings, or
more structured representations such as records containing the
area-code and number---and would have to make several design decisions
at this point.  But for requirements specification, all we require is
that phone numbers and names are distinguishable {\em types\/} of
entities.  In PVS, we can specify this as follows (a \% sign
introduces a comment that extends to the end of the line).
\begin{jmrsession}
  N: TYPE              \% names
  P: TYPE              \% phone numbers
\end{jmrsession}
These types are {\em uninterpreted\/}, meaning that we know nothing
about their members---not even whether they are zero, many, or
infinite in number---except that elements of type {\tt N} are
distinguishable from those of type {\tt P}, and that there is an
equality predicate on each type (i.e., given two {\tt P}s, it is
possible to tell whether they are the same or not).

Next, we need to describe how phone books---associations between names
and numbers---are to be represented.  There are several possibilities:
one is to record each association as a {\tt (name, phone number)}
pair, so that a phone book is a set of such pairs; another is as a
function from names to phone numbers (you can think of a function as
an array if that notion is more familiar to you).  PVS is able to
reason very effectively with functions, so there is some advantage to
the latter representation.  We can specify this as follows.
\begin{jmrsession}
  B: TYPE = [N -> P]   \% phone books
\end{jmrsession}
This says that phone books have the {\em type\/} {\tt
B}, and are functions from names to phone numbers.

We must recognize that not all names will be in every phone book---a
phone book only records those names that have a phone number---so we
need some way to distinguish those names that have a phone number from
those that do not.  In the specification language Z, for example, this
would be accomplished by specifying that phone books are {\em
partial\/} functions.  Efficient theorem proving, however, strongly
encourages use of {\em total\/} functions, so PVS is a logic of total
functions.\footnote{PVS can represent partial functions very nicely
using {\em dependent\/} types, but that is an advanced topic.}  One
way to indicate that a name has no phone number is to identify some
particular phone number, represented by {\tt n0} say, to indicate this
fact.  Of course we need to mentally make note that this number must
be different from any ``real'' phone number (we will see later how we
can enforce this requirement, and later still we will see a better way
to deal this whole issue of names that have no phone number).  Give
this decision, we can next
specify the empty phone book as the (unique) phone book that maps all
names to {\tt n0}.  I will specify this axiomatically, later we will
see how to do it definitionally.
\begin{jmrsession}
  n0: P
  emptybook: B
  emptyax: AXIOM   FORALL (nm: N): emptybook(nm) = n0
\end{jmrsession}
If we were programming an implementation, a literal translation of
this representation would be grossly inefficient: it requires
``space'' for every possible name and it explicitly records for every
name that there is no number associated with the name.  When
programming, we would seek more compact representations that traded
off space for efficient access---perhaps a hash table or balanced
binary tree.   In requirements specification, however, the idea is
simply to record the functionality required, and it is not our concern
to suggest an efficient implementation.

We can specify the {\tt FindPhone} operation as a function that takes a
phone book and a name and returns the phone number associated with
that name.
\begin{jmrsession}
  FindPhone: [B, N -> P]
  Findax: AXIOM   FORALL (bk: B), (nm: N):  FindPhone(bk, nm) = bk(nm)
\end{jmrsession}
Notice that this is a {\em functional\/} specification style: the
``state'' of the system we are interested in (i.e., the phone book) is
passed to the {\tt FindPhone} function as an argument; this is in contrast
to a more procedural style of specification (as in Z, for example),
where there is a built-in notion of state.  Functional specifications
use conventional logic and can be mechanized straightforwardly,
whereas procedural specifications involve some kind of Hoare
logic---for which it is rather more difficult to provide mechanized
deduction.

The distinction between functional and procedural kinds of
specification is revealed more clearly in the case of our next
operation, {\tt AddPhone.}  In a procedural specification, this operation
would update the state of the phone book ``in place.''  In the
functional style used here, we model the operation by a function that
takes a phone book, a name, and a number, and gives us back a ``new''
phone book in which the association between the name and number has
been added.   
\begin{jmrsession}
  AddPhone: [B, N, P -> B]
  Addax: AXIOM   FORALL (bk: B), (nm: N), (pn: P): 
     AddPhone(bk, nm, pn) = bk WITH [(nm) := pn]
\end{jmrsession}
The {\tt WITH} construct is similar to function overriding in Z.

Now that we have specified two operations, perhaps we should check our
understanding of them.  If we were programming, we might run a couple
of test cases.  Some people advocate something similar (often called
``animation'') for specifications.  This is generally feasible only
with specifications that have a constructive character (i.e., that are
essentially very high-level programs).  Not all specifications are
best presented in this way, however, so the desire to make
specifications executable can distort their other characteristics.
Another way to probe a specification is by means of ``formal
challenges.''  These are putative theorems: general statements that we
think should to be true if our specification says what it ought to.
This can yield more information than an individual test case (it is
generally equivalent to running a whole class of test cases), and uses
theorem proving (i.e., search), rather than direct execution, so it is
possible even when the specification is not constructive.  (If the
specification is constructive---as in this example---then theorem
proving generally comes down to symbolic execution and is
very efficient.)  A suitable challenge for the specification
we have so far is: ``if I add a name {\tt nm} with phone number {\tt
pn} to a phone book and look up the name {\tt nm}, I should get back
the phone number {\tt pn}.''  We can write this as follows.
\begin{jmrsession}
  FindAdd: CONJECTURE  FORALL (bk: B), (nm: N), (pn: P):
    FindPhone(AddPhone(bk, nm, pn), nm) = pn
\end{jmrsession}

In order to test this conjecture, we have to extend the specification into a
complete PVS ``theory'' (as modules are called in PVS).  This is shown
in Figure~\ref{fig1}.  Then we load the specification into PVS, parse
and typecheck it, and start the prover.  The mechanics of doing this
are described in other PVS tutorial documents.  Briefly, PVS uses an
extended GNU Emacs as its interface, and PVS system functions are
invoked by Emacs keystrokes.  To invoke the prover, for example, place
the cursor on the {\tt CONJECTURE} and type {\tt M-x prove} (this will
automatically parse and typecheck if necessary).

\begin{figure}[htbp]
\begin{jmrsession}
phone_1: THEORY

BEGIN

  N: TYPE              \% names
  P: TYPE              \% phone numbers
  B: TYPE = [N -> P]   \% phone books

  n0: P
  emptybook: B
  emptyax: AXIOM   FORALL (nm: N): emptybook(nm) = n0

  FindPhone: [B, N -> P]
  Findax: AXIOM   FORALL (bk: B), (nm: N):  FindPhone(bk, nm) = bk(nm)

  AddPhone: [B, N, P -> B]
  Addax: AXIOM   FORALL (bk: B), (nm: N), (pn: P): 
     AddPhone(bk, nm, pn) = bk WITH [(nm) := pn]

  FindAdd: CONJECTURE  FORALL (bk: B), (nm: N), (pn: P):
    FindPhone(AddPhone(bk, nm, pn), nm) = pn

END phone_1
\end{jmrsession}
\caption{\label{fig1}Specification Ready for Checking the First Challenge}
\end{figure}

Starting the prover on the {\tt FindAdd} conjecture produces the following
display.
\begin{jmrsession}
FindAdd :  

  |-------
\{1\}   FORALL (bk: B), (nm: N), (pn: P): FindPhone(AddPhone(bk, nm, pn), nm) = pn

Rule? 
\end{jmrsession}
This is a {\em sequent\/}: in general there will be several numbered
formulas above the turnstile symbol {\tt |-------}, and several below.
The idea is that we have to establish that the conjunction (and) of the
formulas above the turnstile implies the disjunction (or) of the
formulas below the line.  The {\tt Rule?} prompt indicates that PVS is
waiting for us to type a prover command.  These use lisp syntax, with
pieces of PVS syntax embedded in quotes: for example:
\mbox{\tt (grind :theories ("phone\_1"))}.

This introduction is intended to describe the purpose and value of
mechanized theorem proving in analysis of requirements specification;
it is not intended as a tutorial on the PVS prover, so I will not
explain all the various choices and considerations at each step.  The
prover provides a number (about 20) basic commands, and a
similar-sized collection of higher-level commands called
``strategies'' that are programmed using the basic commands.  You type
a command at the {\tt Ready?} prompt, and the prover applies the
command and presents you with the transformed sequent and another
prompt.  When the prover recognizes that a sequent is trivially true,
it terminates that branch of the proof.  Some commands may split the
proof into branches, in which case you will be presented with one of
the branches, and the others will be remembered and popped up when the
current branch terminates.  When all branches are terminated the
theorem is proved.

On straightforward theorems (and the straightforward parts of
difficult theorems), it is generally best to use the highest-level,
most automated strategies, and only to resort to the basic commands
for crucial steps.  The highest-level strategy is called {\tt grind}.
It does skolemization, heuristic instantiation, propositional
simplification (using BDDs), if-lifting, rewriting, and applies
decision procedures for linear arithmetic and equality.  It takes
several optional arguments which mostly supply the names of the
formulas that can be used for automatic rewriting (i.e., replacing of
an instance of the left hand side of an equation by the corresponding
instance of the right hand side).  In this case, we need to tell it
that all the definitions and axioms in the theory {\tt phone\_1} may
be used as rewrites.  The command above does this, and is sufficient
to prove the challenge.
\begin{jmrsession}
Rule? (grind :theories ("phone_1"))
Addax rewrites AddPhone(bk, nm, pn)
  to bk WITH [(nm) := pn]
Findax rewrites FindPhone(bk WITH [(nm) := pn], nm)
  to pn
Trying repeated skolemization, instantiation, and if-lifting,
Q.E.D.
\end{jmrsession}

Encouraged by this small confirmation that we are on the right track
we can return to specifying the {\tt DelPhone} operation.  This is
specified in a similar way to {\tt AddPhone}.
\begin{jmrsession}
  DelPhone: [B, N -> B]
  Delax: AXIOM   FORALL (bk: B), (nm: N): DelPhone(bk, nm) = bk WITH [(nm) := n0]
\end{jmrsession}
We can similarly test our understanding of this specification by
checking the intuition that adding a name and phone number to a book
and then deleting them leaves the book unchanged.
\begin{jmrsession}
  DelAdd: CONJECTURE  FORALL (bk: B), (nm: N), (pn: P):
    DelPhone(AddPhone(bk, nm, pn), nm) = bk
\end{jmrsession}
The same proof strategy as before fails to prove the conjecture and
produces the following result.
\begin{jmrsession}
DelAdd :  

  |-------
\{1\}   FORALL (bk: B), (nm: N), (pn: P): DelPhone(AddPhone(bk, nm, pn), nm) = bk

Rule? (grind :theories ("phone_1"))
Addax rewrites AddPhone(bk, nm, pn)
  to bk WITH [(nm) := pn]
Delax rewrites DelPhone(bk WITH [(nm) := pn], nm)
  to bk WITH [(nm) := pn] WITH [(nm) := n0]
Trying repeated skolemization, instantiation, and if-lifting, this simplifies to: 
DelAdd :  

  |-------
\{1\}   bk!1 WITH [(nm!1) := pn!1] WITH [(nm!1) := n0] = bk!1

Rule?
\end{jmrsession}
The identifiers with {\tt !} in them are Skolem constants---arbitrary
representatives for quantified variables.  This sequent is requiring
us to prove that two functions (i.e., phone books) are the same: one
that has been modified by adding a name and then removing it, another
that is unchanged.  To prove that two functions are the same, we
appeal to the principle of {\em extensionality\/}, which says that
this is so if the values of the two functions are identical for every
point in their domains.
\begin{jmrsession}
Rule? (apply-extensionality)
  Applying extensionality, this simplifies to: 
DelAdd :  

  |-------
\{1\}   bk!1 WITH [(nm!1) := pn!1] WITH [(nm!1) := n0](x!1) = bk!1(x!1)
[2]   bk!1 WITH [(nm!1) := pn!1] WITH [(nm!1) := n0] = bk!1

Rule? (delete 2)
  Deleting some formulas, this simplifies to: 
DelAdd :  

  |-------
[1]   bk!1 WITH [(nm!1) := pn!1] WITH [(nm!1) := n0](x!1) = bk!1(x!1)

Rule? 
\end{jmrsession}
It is always possible to delete formulas from a sequent; here I have
deleted the original formula to reduce clutter, since it is the
extensional form that is interesting.  This sequent is asking us to
show that the phone number associated with an arbitrary name {\tt x!1}
is the same both before and after the phone book has been updated for
name {\tt nm!1}.  A case-analysis is appropriate here, according to
whether or not {\tt x!1 = nm!1}.  This can be accomplished by
the {\tt (lift-if)} command, which converts {\tt WITH} expressions to
their corresponding {\tt IF-THEN-ELSE} form.  The {\tt (ground)}
command (a slightly less muscular command than {\tt (grind)}) then
takes care of the various cases, except for one.
\begin{jmrsession}
DelAdd :  

  |-------
[1]   bk!1 WITH [(nm!1) := pn!1] WITH [(nm!1) := n0](x!1) = bk!1(x!1)

Rule? (lift-if)
Lifting IF-conditions to the top level,
this simplifies to: 
DelAdd :  

  |-------
\{1\}   IF nm!1 = x!1 THEN n0 = bk!1(x!1)
      ELSE IF nm!1 = x!1 THEN n0 = bk!1(x!1)
        ELSE bk!1(x!1) = bk!1(x!1)
        ENDIF
      ENDIF

Rule? (ground)
Applying propositional simplification and decision procedures,
this simplifies to: 
DelAdd :  

\{-1\}   nm!1 = x!1
  |-------
\{1\}   n0 = bk!1(x!1)

Rule?
\end{jmrsession}
(A {\tt (grind)} command would have performed both these steps.)
For this sequent to be true, we need to to demonstrate
that if {\tt x!1 = nm!1}, then the phone number originally associated
with {\tt x!1} is the special number {\tt n0}.  But, by virtue of the
equality, this is the same as asking us to prove that the phone number
originally associated with {\tt nm!1} is {\tt n0}---and there is no
reason why this should be true!  Suddenly, we understand the problem:
if the number associated with {\tt nm!1} beforehand was a real phone
number, {\tt nm!2}, say, then the {\tt AddPhone} operation {\em
changes\/} the association to the new number, and the {\tt DelPhone}
operation changes it again to {\tt n0}---which is not equal to {\tt
nm!2}.  Thus our conjecture is only true under the assumption that the
name we add to the phone book currently has no number associated with it.
We can test this by modifying the conjecture as follows.
\begin{jmrsession}
  DelAdd2: CONJECTURE  FORALL (bk: B), (nm: N), (pn: P):
    FindPhone(bk, nm) = n0 => DelPhone(AddPhone(bk, nm, pn), nm) = bk
\end{jmrsession}
And the {\tt (grind :theories ("phone\_1"))} strategy proves this.

Another conjecture is that the result of adding a name and then
deleting it is the same as just deleting it.
\begin{jmrsession}
  DelAdd3: CONJECTURE  FORALL (bk: B), (nm: N), (pn: P):
    DelPhone(AddPhone(bk, nm, pn), nm) = DelPhone(bk, nm)
\end{jmrsession}
The {\tt (grind :theories ("phone\_1"))} strategy proves this
conjecture also.

Notice how our inability to prove the original {\tt DelAdd} conjecture
exposed a deficiency in our specification and led us to discover the
source of the deficiency.  Individual test cases might have missed the
particular circumstance that exposes the problem, but the strict
requirements of mechanically checked proof systematically led us to
examine all the cases until we discovered the one that manifested the
problem.

Another conjecture we might try to prove is that after adding a name
and phone number to the phone book, the number stored for that name is
a ``real'' number (i.e., not {\tt n0}).
\begin{jmrsession}
  KnownAdd: CONJECTURE  FORALL (bk: B), (nm: N), (pn: P):
    FindPhone(AddPhone(bk, nm, pn), nm) /= n0
\end{jmrsession}
The same kind of exploration with the prover will rapidly show that
this is unprovable because there is nothing that requires the {\tt
pn} argument to {\tt AddPhone} to be a ``real'' phone number.

Our exploration of this specification has revealed a couple of
deficiencies.
\begin{enumerate}
\item {\tt AddPhone} has the side effect of changing the phone number
when applied to someone who already has a number.

\item Our specification does not rule out the possibility
      of giving someone {\tt n0} as a phone number
\end{enumerate}

We can deal with the second deficiency by introducing a type {\tt GP}
of ``good phone numbers'' as a {\em subtype\/} of P, with the
constraint that {\tt n0} is not a member of {\tt GP}.  In PVS, this is
done by means of a {\em predicate subtype}, which can be written as
follows.
\begin{jmrsession}
  GP: TYPE = \{ pn: P | pn /= n0 \}
\end{jmrsession}
We will see later that predicate subtypes are a very powerful element
of the PVS specification language.  Here we can make simple use of
them by changing the {\em signature\/} of the {\tt AddPhone}
function from {\tt [B, N, P -> B]} to {\tt [B, N, GP -> B]}, and this
will automatically prevent the addition of {\tt n0} to a phone book as
a real number.

We can deal with the first deficiency noted above by dividing the
functionality of {\tt AddPhone} in two: the revised {\tt AddPhone}
will make no change to the phone book if the name concerned already
has a phone number, and the new {\tt ChangePhone} operator will change
an existing number, but will not add a number to a name that
currently lacks one.

In order to specify these functions, it is convenient to add a
predicate {\tt Known?} that takes a phone book and a name and returns
{\tt true} if that name has a ``real'' phone number in the book
concerned.  (A {\em predicate\/} is just a function whose range type
is boolean.)  This can be specified as follows.
\begin{jmrsession}
  Known?: [B, N -> bool]
  Known_ax: AXIOM   FORALL (bk: B), (nm: N): Known?(bk, nm) = (bk(nm) /= n0)
\end{jmrsession}

This axiomatic style of specification has the disadvantage that axioms
can introduce inconsistencies.  An individual axiom is seldom
dangerous: rather, danger lies in the interactions among several
axioms.  For example, with the original signature and definition of
{\tt AddPhone}, adding the following axiom to that above yields an
inconsistent specification.
\begin{jmrsession}
  Whoops: AXIOM   FORALL (bk: B), (nm: N), (pn, P): Known?(AddPhone(bk, nm, pn), nm)
\end{jmrsession}

Inconsistent specifications are dangerous because they can be used to
prove anything at all,\footnote{For example, when used in conjunction
with the {\tt AXIOM}s {\tt emptyax}, {\tt Known\_ax}, and {\tt Addax},
{\tt Whoops} allows us to prove {\tt true = false}.}
and because they cannot be implemented.  It is
disturbingly easy to introduce inconsistent axioms, so it is generally
best to use them sparingly.  Axioms are really needed only when it is
necessary to constrain (rather than fully define) the values of a
function, or when it is necessary to constrain the interactions of
several functions.  When the intent is to fully define the values of
a function, it is generally better to state it as a {\em definition\/},
since PVS will then check that it is indeed a ``conservative
extension'' (and therefore does not introduce an inconsistency).

The predicate {\tt Known?} can be introduced by means of a definition
by replacing the two lines used earlier (the specification of its
signature and axiom) by the following single line.
\begin{jmrsession}
  Known?: [B, N -> bool] = LAMBDA (bk: B), (nm: N): bk(nm) /= n0
\end{jmrsession}
The use of {\tt LAMBDA} notation can be a little daunting, so PVS allows an
alternative, ``applicative,'' form of definition as follows.
\begin{jmrsession}
  Known?(bk: B, nm: N): bool = bk(nm) /= n0
\end{jmrsession}
The need to specify the types of the variables in this declaration
can be eliminated by declaring them separately.
\begin{jmrsession}
  bk: VAR B
  nm: VAR N
  Known?(bk, nm): bool = bk(nm) /= n0
\end{jmrsession}
In this way, the previous axiomatic specification for {\tt AddPhone} can be
changed to the following definition, which incorporates the refinement
that the function does not change the phone book if the name already
has a number known for it.
\begin{jmrsession}
  gp: VAR GP
  AddPhone(bk, nm, gp): B = 
    IF Known?(bk, nm) THEN bk ELSE bk WITH [(nm) := gp] ENDIF
\end{jmrsession}
We can check that these changes provide some of the properties we
expect by considering the following formal challenge.
\begin{jmrsession}
  KnownAdd: CONJECTURE  FORALL bk, nm, gp: Known?(AddPhone(bk, nm, gp), nm)
\end{jmrsession}
This says that a name is definitely known (i.e., has a ``real'' phone
number) after applying {\tt AddPhone} to it.   Notice that since the
variables {\tt bk}, {\tt nm}, and {\tt gp} have already been declared,
there is no need to specify their types in the {\tt FORALL} construction.
In fact, there is no need to provide the {\tt FORALL} construction at all:
the following specification is equivalent to the one above, since PVS
automatically interprets ``free'' variables as universally quantified
at the outermost level.
\begin{jmrsession}
  KnownAdd: CONJECTURE  Known?(AddPhone(bk, nm, gp), nm)
\end{jmrsession}
This conjecture is easily proved by the {\tt grind} strategy.

\begin{figure}[htbp]
\begin{jmrsession}
phone_2: THEORY

BEGIN

  N: TYPE              \% names
  P: TYPE              \% phone numbers
  B: TYPE = [N -> P]   \% phone books

  n0: P

  GP: TYPE = \{pn: P | pn /= n0\}

  nm: VAR N
  pn: VAR P
  bk: VAR B
  gp, gp1, gp2: VAR GP

  emptybook(nm): P = n0

  FindPhone(bk, nm): P = bk(nm)

  Known?(bk, nm): bool = bk(nm) /= n0

  AddPhone(bk, nm, gp): B = 
    IF Known?(bk, nm) THEN bk ELSE bk WITH [(nm) := gp] ENDIF

  ChangePhone(bk, nm, gp): B = 
    IF Known?(bk, nm) THEN bk WITH [(nm) := gp] ELSE bk ENDIF

  DelPhone(bk, nm): B = bk WITH [(nm) := n0]

  FindAdd: CONJECTURE
    NOT Known?(bk, nm) => FindPhone(AddPhone(bk, nm, gp), nm) = gp

  FindChange: CONJECTURE
    Known?(bk, nm) => FindPhone(ChangePhone(bk, nm, gp), nm) = gp

  DelAdd: CONJECTURE
    DelPhone(AddPhone(bk, nm, gp), nm) = DelPhone (bk, nm)

  KnownAdd: CONJECTURE Known?(AddPhone(bk, nm, gp), nm)

  AddChange: CONJECTURE
    ChangePhone(AddPhone(bk, nm, gp1), nm, gp2) =
      AddPhone(ChangePhone(bk, nm, gp2), nm, gp2)

END phone_2
\end{jmrsession}
\caption{\label{fig2}Revised Specification}
\end{figure}
Proceeding in this way, we can construct the theory {\tt phone\_2}
shown in Figure~\ref{fig2}.  All the conjectures in that theory
are proved by the simple command {\tt (grind)}.   There is no need to
specify auto-rewriting of the {\tt phone\_2} theory, since definitions
are automatically available for rewriting (another advantage that they
have over axioms).

If we try to add the dangerous {\tt AXIOM} {\tt Whoops} to this new
specification, PVS will note that the third argument supplied to {\tt
Addphone} {\tt (pn}) is a {\tt P}, whereas the signature of {\tt
AddPhone} says it requires a {\tt GP} in this position.  PVS allows a
value of a supertype to be used where one of a subtype is required,
provided the value can be proven, in its context, to satisfy the
predicate of the subtype concerned.  The corresponding proof
obligation is generated automatically by PVS as a Type-Correctness
Condition (TCC).  PVS does not consider a specification fully
typechecked until all its TCCs have been proved (though you can
postpone doing the proof until convenient).  TCCs are displayed by the
command {\tt M-x show-tccs}; in the present case, the TCC generated by
{\rm Whoops} is the following.
\begin{jmrsession}
% Subtype TCC generated (line 37) for pn
  % untried
whoops_TCC1: OBLIGATION (FORALL (pn: P): pn /= n0);
\end{jmrsession}
This is obviously unproveable (and untrue!), and the folly of adding
the axiom {\tt Whoops} is thereby brought to our attention.

Notice that if the {\tt pn} in {\tt Whoops} is changed to {\tt gp},
then the formula not only becomes harmless (and no TCC is generated),
but a proveable consequence of the definitions.

\newpage
\markright{Phone Book: Better Version}
\section{A Better Version of the Specification Using Sets}
\markright{Phone Book: Better Version}

The realization that {\tt AddPhone} had the effect of changing the
phone number associated with a name if that name already had a phone
number led us to revise the specification so that {\tt AddPhone} has
no effect when the name already has a phone number.  This treatment
assumes that names can have at most one phone number associated with
them.  On reflection, or after consultation with the customer, we may
decide that it is better to allow names to
have multiple numbers associated with them.  We can accommodate this
by changing the range of the phone book function from a single phone
number to a {\em set\/} of phone numbers as follows.
\begin{jmrsession}
  B: TYPE = [N -> setof[P]]   \% phone books
\end{jmrsession}
This approach has the benefit that we now have a ``natural''
representation for names that do not have phone numbers: they can be
associated with the emptyset of phone numbers.

A specification based on this approach is shown in Figure~\ref{fig3}.
The set-constructing functions such as {\tt add}, {\tt remove}, {\tt
emptyset}, etc., and the predicates on sets such as {\tt disjoint?}
are defined in a PVS {\em prelude\/} (i.e., built-in)
theory called {\tt set}.  You can inspect this theory with the command
{\tt M-x view-prelude-theory}.
\begin{figure}[htbp]
\begin{jmrsession}
phone_3 : THEORY

  BEGIN

  N: TYPE                     % names
  P: TYPE                     % phone numbers
  B: TYPE = [N -> setof[P]]   % phone books
  nm, x: VAR N
  pn: VAR P
  bk: VAR B
  
  emptybook(nm): setof[P] = emptyset[P]

  FindPhone(bk, nm): setof[P] = bk(nm)

  AddPhone(bk, nm, pn): B = bk WITH [(nm) := add(pn, bk(nm))]

  DelPhone(bk,nm): B = bk WITH [(nm) := emptyset[P]]

  DelPhoneNum(bk,nm,pn): B = bk WITH [(nm) := remove(pn, bk(nm))]

  FindAdd: CONJECTURE member(pn, FindPhone(AddPhone(bk, nm, pn), nm))

  DelAdd: CONJECTURE DelPhoneNum(AddPhone(bk, nm, pn), nm, pn) =
                DelPhoneNum(bk, nm, pn)

  END phone_3
\end{jmrsession}
\caption{\label{fig3}Specification Using Set Constructions}
\end{figure}
A rather more attractive rendition of this specification is shown in
Figure~\ref{fig3-latex}; this is produced by the command {\tt M-x
latex-theory}, which typesets the specification using \LaTeX.

\begin{figure}
% The following substitutions are from the file:
%   /tmp_mnt/homes/csla/rushby/tadpole/pvs-tex.sub
\def\removetwofn#1#2{{#2 \setminus \{#1\}}}% How to print function remove with arity (2)
\def\addtwofn#1#2{{\{#1\} \cup #2}}% How to print function add with arity (2)
\def\emptysetoneid#1{{\emptyset_#1}}% How to print name emptyset with (1) actuals
\def\singletononefn#1{{\{ #1 \}}}% How to print function singleton with arity (1)
\def\uniontwofn#1#2{{#1 \cup #2}}% How to print function union with arity (2)
\def\differencetwofn#1#2{{#1 \setminus #2}}% How to print function difference with arity (2)
\setlength{\fboxsep}{1pt} 
% The following substitutions are from the file:
%   /homes/EHDM/pvs/pvs-tex.sub
\def\membertwofn#1#2{{#1 \in #2}}% How to print function member with arity (2)
\setlength{\fboxsep}{1pt} 
 \begin{program} 
 \pvsid{phone\_3} :\mbox{ } \pvskey{THEORY} \pvsnewline{}
\zi\zi \pvskey{BEGIN} \pvsnewline{}
 \pvsnewline{}
 N :\mbox{ } \pvskey{TYPE} \pvsnewline{}
 \pvsnewline{}
 P :\mbox{ } \pvskey{TYPE} \pvsnewline{}
 \pvsnewline{}
 B :\mbox{ } \pvskey{TYPE\mbox{ }} \mbox{ }=\mbox{ }[\ii N \rightarrow\mbox{ } \pvsid{setof}[\ii P]] \pvsnewline{}
\\[-\baselineskip]\oo\oo\zi\zi\zi\zo\zo\zo \pvsnewline{}
 \pvsid{nm} , x :\mbox{ } \pvskey{VAR\mbox{ }} N \pvsnewline{}
 \pvsnewline{}
 \pvsid{pn} :\mbox{ } \pvskey{VAR\mbox{ }} P \pvsnewline{}
 \pvsnewline{}
 \pvsid{bk} :\mbox{ } \pvskey{VAR\mbox{ }} B \pvsnewline{}
 \pvsnewline{}
 \pvsid{emptybook}(\ii \pvsid{nm}) :\mbox{ } \pvsid{setof}[\ii P] \mbox{ }=\mbox{ } \emptysetoneid { P } \pvsnewline{}
\\[-\baselineskip]\oo\oo\zi\zo \pvsnewline{}
 \pvsid{FindPhone}(\ii \pvsid{bk} , \pvsid{ nm}) :\mbox{ } \pvsid{setof}[\ii P] \mbox{ }=\mbox{ } \pvsid{bk}(\ii \pvsid{nm}) \pvsnewline{}
\\[-\baselineskip]\oo\oo\oo\zi\zi\zo\zo \pvsnewline{}
 \pvsid{AddPhone}(\ii \pvsid{bk} , \pvsid{ nm} , \pvsid{ pn}) :\mbox{ } B \mbox{ }=\mbox{ } \pvsid{bk } \pvskey{WITH\mbox{ }}[\ii(\ii \pvsid{nm}) \mbox{ }:=\mbox{ }\ii \addtwofn { \pvsid{pn}}{ \pvsid{bk}( \pvsid{nm}) }] \pvsnewline{}
\\[-\baselineskip]\oo\oo\oo\oo\zi\zi\zi\zo\zo\zo \pvsnewline{}
 \pvsid{DelPhone}(\ii \pvsid{bk} , \pvsid{ nm}) :\mbox{ } B \mbox{ }=\mbox{ } \pvsid{bk } \pvskey{WITH\mbox{ }}[\ii(\ii \pvsid{nm}) \mbox{ }:=\mbox{ }\ii \emptysetoneid { P }] \pvsnewline{}
\\[-\baselineskip]\oo\oo\oo\oo\zi\zi\zi\zo\zo\zo \pvsnewline{}
 \pvsid{DelPhoneNum}(\ii \pvsid{bk} , \pvsid{ nm} , \pvsid{ pn}) :\mbox{ } B \mbox{ }=\mbox{ } \pvsid{bk } \pvskey{WITH\mbox{ }}[\ii(\ii \pvsid{nm}) \mbox{ }:=\mbox{ }\ii \removetwofn { \pvsid{pn}}{ \pvsid{bk}( \pvsid{nm}) }] \pvsnewline{}
\\[-\baselineskip]\oo\oo\oo\oo\zi\zi\zi\zo\zo\zo \pvsnewline{}
 \pvsid{FindAdd} :\mbox{ } \pvskey{CONJECTURE\mbox{ }} \membertwofn { \pvsid{pn}}{ \pvsid{FindPhone}( \pvsid{AddPhone}( \pvsid{bk} , \pvsid{ nm} , \pvsid{ pn}) , \pvsid{ nm}) } \pvsnewline{}
 \pvsnewline{}
 \pvsid{DelAdd} :\mbox{ } \pvskey{CONJECTURE} \pvsnewline{}
\zi\zi \pvsid{DelPhoneNum}(\ii \pvsid{AddPhone}(\ii \pvsid{bk} , \pvsid{ nm} , \pvsid{ pn}) , \pvsid{ nm} , \pvsid{ pn}) \mbox{ }=\mbox{ }\ii \pvsid{DelPhoneNum}(\ii \pvsid{bk} , \pvsid{ nm} , \pvsid{ pn}) \pvsnewline{}
\\[-\baselineskip]\oo\oo\oo\oo\zi\zi\zo\zi\zo\zo\zo\zo \pvsnewline{}
 \pvskey{END\mbox{ }} \pvsid{phone\_3} \pvsnewline{}
\zo \end{program}
\caption{\label{fig3-latex}\LaTeX-Printed Version of the Specification
in Figure~\protect\ref{fig3}}
\end{figure}
The first conjecture in this specification is easily proved using {\tt
(grind)}.   The second one requires the more complex proof shown
below.
\begin{jmrsession}
("" (GRIND)
    (APPLY-EXTENSIONALITY)
    (DELETE 2)
    (LIFT-IF)
    (GROUND)
    (APPLY-EXTENSIONALITY)
    (DELETE 2)
    (GRIND))
\end{jmrsession}
This is the form in which PVS proofs are stored for later replay.

We have specified single additions to the phone book, but it seems
likely that bulk additions will also be necessary.  This will give us
an opportunity to explore some more advanced features of the PVS
language and prover.  We would like to specify a function {\tt
AddList}, say, that takes a phone book and some collection of names
and phone numbers and adds all of those names and phone numbers to the
phone book.  Each name-and-number is a pair, which can be represented
in PVS by the tuple-type {\tt [N, P]}.  We could represent a
collection of such pairs by either a sequence, or a list---a list is
most convenient here, and is represented in PVS by the type {\tt
list[[N, P]]}.  In this expression, the outermost brackets enclose the
type parameter (here {\tt [N, P]}) to the generic {\tt list} theory
(e.g., a list of phone numbers would be {\tt list[P]}).  In order to
process such a list, we specify {\tt AddList} as a recursive function
that returns the phone book it is given if the list is empty, and
otherwise recurses by applying the tail of the list to the phone book
that results from applying {\tt AddPhone} to the first name and number
pair in the list.

\begin{jmrsession}
  updates: VAR list[[N, P]]

  AddList(bk, updates): RECURSIVE B = 
    CASES updates OF
      null: bk,
      cons(upd, rest): AddList(AddPhone(bk, proj_1(upd), proj_2(upd)), rest)
    ENDCASES
    MEASURE length(updates)
\end{jmrsession}

In this specification, the {\tt CASES} expression introduces a
pattern-matching enumeration over the constructors of an abstract data
type (here, {\tt list}), and the {\tt proj\_i} functions project out
the {\tt i}'th member of a tuple.  The {\tt MEASURE} clause indicates
the argument that decreases across recursive calls (more generally, it
specifies a function of the arguments, and an ordering relation
according to which it decreases).  PVS uses the {\tt MEASURE} to
generate a TCC to ensure that the function is total (i.e., that the
recursion always ``terminates'').  In this case, the TCC is
\begin{jmrsession}
% Termination TCC generated (line 48) for AddList

  AddList_TCC1: OBLIGATION
      (FORALL (rest: list[[N, P]], upd: [N, P], updates: list[[N, P]]):
         updates = cons[[N, P]](upd, rest)
           IMPLIES length[[N, P]](rest) < length[[N, P]](updates))
\end{jmrsession}
and it is proved automatically by PVS's standard strategy for proving
TCCs (this strategy, called {\tt (tcc)}, is a variety of {\tt (grind)}).

The {\tt list} datatype is specified in the PVS prelude using the
datatype construction (similar to a ``free type'' in Z).
\begin{jmrsession}
list [T: TYPE]: DATATYPE 
 BEGIN
  null: null?
  cons (car: T, cdr:list):cons?
 END list
\end{jmrsession}
This specifies that {\tt list} is a datatype that takes a single type
parameter and has constructors {\tt null} and {\tt cons}, with
corresponding recognizers and predicate subtypes {\tt null?} and {\tt
cons?}, and accessors {\tt car} and {\tt cdr}.  This specification
expands internally into many axioms and definitions that are
guaranteed to be conservative (i.e., not to introduce
inconsistencies), and that are used very efficiently by the prover.

To validate our understanding of this function, we can try a couple of
challenges.  A reasonable expectation is that if a number is a member
of the set of phone numbers for a given name, then it is still a member
of that set after an arbitrary list of names and phone numbers have
been added to the phone book.
\begin{jmrsession}
  AddList_member: CONJECTURE
    member(pn, FindPhone(bk, nm)) =>
       member(pn, FindPhone(AddList(bk, updates), nm))
\end{jmrsession}
Like most conjectures involving recursively-defined functions, this
one requires a proof by induction.  PVS provides some powerful
strategies for inductive proofs.  Here, the single strategy
{\tt (induct-and-simplify "updates" :defs t)}
is sufficient to prove the challenge.  The argument {\tt "updates"} is
the name of the variable on which to induct, and {\tt :defs t}
instructs PVS that it may treat all definitions as rewrites.  PVS
automatically selects the correct induction rule (here, induction on
lists), based on the type of the induction variable.  The induction
rule itself is defined automatically as part of the expansion of the
{\tt list} datatype definition.

A rather more complicated conjecture is that the set of phone numbers
associated with a given name is unchanged when a list of names and
phone numbers are added to the phone book if the given name is not
mentioned in the list.  This can be specified as follows.
\begin{jmrsession}
  FindList: CONJECTURE
    (every! (upd:[N, P]): proj_1(upd)/=nm) (updates) =>
      FindPhone(AddList2(bk, updates), nm) = FindPhone(bk, nm)
\end{jmrsession}
In this specification, {\tt every!} introduces the body of a predicate
that is true of all members of the list supplied as its argument
(here, {\tt updates}).  It is another of the constructions defined
automatically as a result of expanding the {\tt list} datatype
definition.  As with the previous example, the {\tt
induct-and-simplify} strategy is able to prove this conjecture
automatically.

%\newpage

\markright{Phone Book: Version with Invariant}
\section{Version of the Specification That Maintains An Invariant}
\markright{Phone Book: Version with Invariant}


A reasonable expectation is that the same phone number is never
assigned simultaneously to two different names.  We can extend the
specification to ensure this by adding a predicate {\tt
UnusedPhoneNum} that returns {\tt true} if a given number is not
assigned to any name in a given phone book, and  then modifying
{\tt AddPhone} to check the  number being added is indeed unused.
\begin{jmrsession}
  UnusedPhoneNum(bk, pn): bool =
     (FORALL nm: NOT member(pn,FindPhone(bk, nm)))

  AddPhone(bk, nm, pn): B = 
    IF UnusedPhoneNum(bk, pn) THEN bk WITH [(nm) := add(pn, bk(nm))]
      ELSE bk
    ENDIF
\end{jmrsession}

If we've got this right, then it ought to be the case that the sets of
phone numbers assigned to different names are always disjoint.  We
could generate a few challenges to check this, but we really want to
be sure that the disjointness condition is an {\em invariant\/} of the
specification.  Recognizing this, we could try to generate the proof
obligations that ensure this property.   It is tedious and
error-prone to generate proof obligations of this kind by hand, so
some systems have special provision for generating the proof
obligations necessary to guarantee invariants.  PVS, however, can
generate the necessary proof obligations as part of a much more
general mechanism.

We have already seen that PVS allows predicate subtypes.  The first
step is to define those phone books that are ``valid'' as the subtype {\tt VB}
of phone books in which the sets of numbers associated with different
names are disjoint.
\begin{jmrsession}
  VB: TYPE = \{ b:B | (FORALL (x,y: N): x /= y => disjoint?(b(x), b(y))) \}
\end{jmrsession}
Then we change the specification of {\tt FindPhone} to specify that it takes
a {\tt VB} and returns a {\tt VB}:
\begin{jmrsession}
  bk: VAR VB

  AddPhone(bk, nm, pn): VB = 
    IF UnusedPhoneNum(bk, pn) THEN bk WITH [(nm) := add(pn, bk(nm))]
      ELSE bk
    ENDIF
\end{jmrsession}
Now the expression {\tt bk WITH [(nm) := add(pn, bk(nm))]} appearing
here is a {\tt B}, but not necessarily a {\tt VB}.  But in order to
satisfy the return type specified for {\tt AddPhone}, this expression
must be a {\tt VB}\@.  As already explained, PVS allows a value of a
supertype to be used where one of a subtype is required, provided the
value can be proven, in its context, to satisfy the predicate of the
subtype concerned.  The context here is {\tt UnusedPhoneNum(bk, pn)},
so the proof obligation that needs to be discharged in order to ensure
this specification is well-typed is the following.
\begin{jmrsession}
% Subtype TCC generated (line 37) for bk WITH [(nm) := add(pn, bk(nm))]

AddPhone_TCC1: OBLIGATION
      (FORALL (bk: VB, nm: N, pn: P):
         UnusedPhoneNum(bk, pn) IMPLIES
           (FORALL (x, y: N): 
              x /= y =>
                disjoint?[P](bk WITH [(nm) := add[P](pn, bk(nm))](x),
                             bk WITH [(nm) := add[P](pn, bk(nm))](y))));
\end{jmrsession}
This proof obligation is called a Type-Correctness Condition (TCC) and
it is generated automatically by PVS\@.  
Proving it requires the following steps.
\begin{jmrsession}
("" (GRIND :IF-MATCH NIL)
    (("1" (GRIND)) ("2" (INST -1 "x!1" "y!1")
                        (GRIND))
                   ("3" (GRIND))
                   ("4" (INST -1 "x!1" "y!1")
                        (GRIND))
                   ("5" (INST -1 "x!1" "y!1")
                        (GRIND))))
\end{jmrsession}
Notice that the proof splits into several branches after the first
step.  PVS can generate a graphical display of the proof tree---which
can then be saved as a postscript file---using the command {\tt M-x
x-show-proof}.  The output for this proof is shown in
Figure~\ref{proof-pic},

\begin{figure}[htb]
\begin{center}
\leavevmode\epsfxsize=1.0\hsize\epsfbox{phone_4_AddPhone_TCC1.ps}\mbox{}
\end{center}
\caption{\label{proof-pic}Graphical Display of the Proof Tree for
TCC {\tt AddPhone\_TCC1}}
\end{figure}

The important point to note, however, is that the close integration
between language and prover in PVS allows the mechanization of very
strong checks on specifications.

The full version of the specification of the previous section,
adjusted to ensure that only valid phone books are generated is shown
in figure~\ref{fig4} and the TCCs generated are shown in
Figure~\ref{fig4-tccs}.

\begin{figure}
% The following substitutions are from the file:
%   /tmp_mnt/homes/csla/rushby/tadpole/pvs-tex.sub
\def\removetwofn#1#2{{#2 \setminus \{#1\}}}% How to print function remove with arity (2)
\def\addtwofn#1#2{{\{#1\} \cup #2}}% How to print function add with arity (2)
\def\emptysetoneid#1{{\emptyset_#1}}% How to print name emptyset with (1) actuals
\def\singletononefn#1{{\{ #1 \}}}% How to print function singleton with arity (1)
\def\uniontwofn#1#2{{#1 \cup #2}}% How to print function union with arity (2)
\def\differencetwofn#1#2{{#1 \setminus #2}}% How to print function difference with arity (2)
\setlength{\fboxsep}{1pt} 
% The following substitutions are from the file:
%   /homes/EHDM/pvs/pvs-tex.sub
\def\membertwofn#1#2{{#1 \in #2}}% How to print function member with arity (2)
\setlength{\fboxsep}{1pt} 
 \begin{program} 
 \pvsid{phone\_4} :\mbox{ } \pvskey{THEORY} \pvsnewline{}
\zi\zi \pvskey{BEGIN} \pvsnewline{}
 \pvsnewline{}
 N :\mbox{ } \pvskey{TYPE} \pvsnewline{}
 \pvsnewline{}
 P :\mbox{ } \pvskey{TYPE} \pvsnewline{}
 \pvsnewline{}
 B :\mbox{ } \pvskey{TYPE\mbox{ }} \mbox{ }=\mbox{ }[\ii N \rightarrow\mbox{ } \pvsid{setof}[\ii P]] \pvsnewline{}
\\[-\baselineskip]\oo\oo\zi\zi\zi\zo\zo\zo \pvsnewline{}
 \pvsid{VB} :\mbox{ } \pvskey{TYPE\mbox{ }} \mbox{ }=\mbox{ }\ii \{ b :\mbox{ } B \mbox{ }|\mbox{ }\mbox{ }(\ii \forall\mbox{ }(\ii x , y :\mbox{ } N) :\mbox{ } x \mbox{ }\neq\mbox{ }\ii y \mbox{ }\Rightarrow\mbox{ }\ii \pvsid{disjoint?}(\ii b(\ii x) , b(\ii y))) \} \pvsnewline{}
\\[-\baselineskip]\oo\oo\oo\oo\oo\oo\oo\oo\zi\zi\zi\zi\zi\zi\zi\zo\zo\zo\zo\zo\zo\zo \pvsnewline{}
 \pvsid{nm} , x :\mbox{ } \pvskey{VAR\mbox{ }} N \pvsnewline{}
 \pvsnewline{}
 \pvsid{pn} :\mbox{ } \pvskey{VAR\mbox{ }} P \pvsnewline{}
 \pvsnewline{}
 \pvsid{bk} :\mbox{ } \pvskey{VAR\mbox{ }} \pvsid{VB} \pvsnewline{}
 \pvsnewline{}
% \pvskey{IMPORTING\mbox{ }}\ii \pvsid{withexpr}[\ii N , \pvsid{ setof}[\ii P]] \pvsnewline{}
%\\[-\baselineskip]\oo\oo\oo  \pvsnewline{}
 \pvsid{emptybook} :\mbox{ } \pvsid{VB } \mbox{ }=\mbox{ }(\ii \lambda\mbox{ }(\ii x :\mbox{ } N) :\mbox{ } \emptysetoneid { P }) \pvsnewline{}
\\[-\baselineskip]\oo\oo\zi\zo \pvsnewline{}
 \pvsid{FindPhone}(\ii \pvsid{bk} , \pvsid{ nm}) :\mbox{ } \pvsid{setof}[\ii P] \mbox{ }=\mbox{ } \pvsid{bk}(\ii \pvsid{nm}) \pvsnewline{}
\\[-\baselineskip]\oo\oo\oo\zi\zi\zo\zo \pvsnewline{}
 \pvsid{UnusedPhoneNum}(\ii \pvsid{bk} , \pvsid{ pn}) :\mbox{ } \pvsid{bool } \mbox{ }=\mbox{ }(\ii \forall\mbox{ } \pvsid{nm} :\mbox{ } \neg\mbox{ } \membertwofn { \pvsid{pn}}{ \pvsid{FindPhone}( \pvsid{bk} , \pvsid{ nm}) }) \pvsnewline{}
\\[-\baselineskip]\oo\oo\zi\zo \pvsnewline{}
 \pvsid{AddPhone}(\ii \pvsid{bk} , \pvsid{ nm} , \pvsid{ pn}) :\mbox{ } \pvsid{VB } \mbox{ }= \pvsnewline{}
\\[-\baselineskip]\oo\zi \pvskey{IF\mbox{ }} \pvsid{UnusedPhoneNum}(\ii \pvsid{bk} , \pvsid{ pn}) \pvskey{\mbox{ }THEN\mbox{ }} \pvsid{bk } \pvskey{WITH\mbox{ }}[\ii(\ii \pvsid{nm}) \mbox{ }:=\mbox{ }\ii \addtwofn { \pvsid{pn}}{ \pvsid{bk}( \pvsid{nm}) }] \pvsnewline{}
\\[-\baselineskip]\oo\oo\oo\oo\zi\zi\zi\zo\zi\zo\zo\zo \pvskey{\mbox{ }ELSE\mbox{ }} \pvsid{bk} \pvsnewline{}
 \pvskey{\mbox{ }ENDIF} \pvsnewline{}
\zo \pvsnewline{}
 \pvsid{DelPhone}(\ii \pvsid{bk} , \pvsid{ nm}) :\mbox{ } \pvsid{VB } \mbox{ }=\mbox{ } \pvsid{bk } \pvskey{WITH\mbox{ }}[\ii(\ii \pvsid{nm}) \mbox{ }:=\mbox{ }\ii \emptysetoneid { P }] \pvsnewline{}
\\[-\baselineskip]\oo\oo\oo\oo\zi\zi\zi\zo\zo\zo \pvsnewline{}
 \pvsid{DelPhoneNum}(\ii \pvsid{bk} , \pvsid{ nm} , \pvsid{ pn}) :\mbox{ } \pvsid{VB } \mbox{ }=\mbox{ } \pvsid{bk } \pvskey{WITH\mbox{ }}[\ii(\ii \pvsid{nm}) \mbox{ }:=\mbox{ }\ii \removetwofn { \pvsid{pn}}{ \pvsid{bk}( \pvsid{nm}) }] \pvsnewline{}
\\[-\baselineskip]\oo\oo\oo\oo\zi\zi\zi\zo\zo\zo \pvsnewline{}
 \pvsid{FindAdd} :\mbox{ } \pvskey{CONJECTURE} \pvsnewline{}
\zi\zi \pvsid{UnusedPhoneNum}(\ii \pvsid{bk} , \pvsid{ pn}) \pvsnewline{}
\\[-\baselineskip]\oo\zi \mbox{ }\supset\mbox{ }\ii \membertwofn { \pvsid{pn}}{ \pvsid{FindPhone}( \pvsid{AddPhone}( \pvsid{bk} , \pvsid{ nm} , \pvsid{ pn}) , \pvsid{ nm}) } \pvsnewline{}
\\[-\baselineskip]\oo\zo\zo\zo \pvsnewline{}
 \pvsid{DelAdd} :\mbox{ } \pvskey{CONJECTURE} \pvsnewline{}
\zi\zi \pvsid{DelPhoneNum}(\ii \pvsid{AddPhone}(\ii \pvsid{bk} , \pvsid{ nm} , \pvsid{ pn}) , \pvsid{ nm} , \pvsid{ pn}) \mbox{ }=\mbox{ }\ii \pvsid{DelPhoneNum}(\ii \pvsid{bk} , \pvsid{ nm} , \pvsid{ pn}) \pvsnewline{}
\\[-\baselineskip]\oo\oo\oo\oo\zi\zi\zo\zi\zo\zo\zo\zo \pvsnewline{}
 \pvskey{END\mbox{ }} \pvsid{phone\_4} \pvsnewline{}
\zo \end{program}
\caption{\label{fig4}Specification Enforcing the Invariant that
Different Names Have Disjoint Sets of Phone Numbers}
\end{figure}

\begin{figure}
\begin{jmrsession}
% Subtype TCC generated (line 15) for (LAMBDA (x: N): emptyset[P])

emptybook_TCC1: OBLIGATION
      (FORALL (x, y: N): x /= y => disjoint?[P](emptyset[P], emptyset[P]));

% Subtype TCC generated (line 23) for bk WITH [(nm) := add(pn, bk(nm))]

AddPhone_TCC1: OBLIGATION
      (FORALL (bk: VB, nm: N, pn: P):
         UnusedPhoneNum(bk, pn) IMPLIES
           (FORALL (x, y: N): 
              x /= y =>
                disjoint?[P](bk WITH [(nm) := add[P](pn, bk(nm))](x),
                             bk WITH [(nm) := add[P](pn, bk(nm))](y))));

% Subtype TCC generated (line 28) for bk WITH [(nm) := emptyset[P]]

DelPhone_TCC1: OBLIGATION
      (FORALL (bk: VB), (nm: N), (x, y: N):
         x /= y =>
           disjoint?[P](bk WITH [(nm) := emptyset[P]](x),
                        bk WITH [(nm) := emptyset[P]](y)));

% Subtype TCC generated (line 30) for bk WITH [(nm) := remove(pn, bk(nm))]

DelPhoneNum_TCC1: OBLIGATION
      (FORALL (bk: VB), (nm: N), (pn: P), (x, y: N):
         x /= y =>
           disjoint?[P](bk WITH [(nm) := remove[P](pn, bk(nm))](x),
                        bk WITH [(nm) := remove[P](pn, bk(nm))](y)));

\end{jmrsession}
\caption{\label{fig4-tccs}TCCs for the Specification of Figure \protect\ref{fig4}}  
\end{figure}

\newpage

\section{Summary}

It is no easier to write correct specifications than to write correct
programs; just like programs, specifications need to be validated
against their informal requirements and expectations.  The
mechanization provided by PVS allows the human inspections and
reviews that are an essential element of validation to be
supplemented by mechanically checked analyses.

I hope the example considered here has conveyed some appreciation for
the opportunities created by mechanically supported formal
specification.  Other tutorials describe more of the mechanics of
using PVS, and give examples of its use to verify algorithm
correctness and to prove difficult theorems.



\newpage
\thispagestyle{plain}

%\part{Prolegomena}

\setcounter{section}{0}
\part{Tutorial on Using PVS}
\markboth{Using PVS}{}
\cleardoublepage

% Master File: wift-tutintro.tex

\section{Introducing PVS}

PVS stands for ``Prototype Verification System.''\footnote{A number of
people have contributed significantly to the design and implementation
of PVS.  They include David Cyrluk, Friedrich von~Henke, Pat Lincoln,
Steven Phillips, Sreeranga Rajan, Jens Skakkeb\ae{}k, Mandayam Srivas,
and Carl Witty.  We also thank Mark Moriconi, Director of the SRI
Computer Science Laboratory, for his support and encouragement.} It
consists of a specification language integrated with support tools and
a theorem prover.  PVS tries to provide the mechanization needed to
apply formal methods both rigorously and productively.

%We have tried to make it a very productive system to employ for
%most purposes that require mechanized support for formal methods.

% We think you will find that it is the most productive verification
% system available for many purposes.

The specification language of PVS is a higher-order logic with a rich
type-system, and is quite expressive; we have found that most of the
mathematical and computational concepts we wish to describe can be
formulated very directly and naturally in PVS\@.  Its theorem prover, or
proof checker (we use either term, though the latter is more correct),
is both interactive and highly mechanized: the user chooses each step
that is to be applied and PVS performs it, displays the result, and then
waits for the next command.  PVS differs from most other interactive
theorem provers in the power of its basic steps: these can invoke
decision procedures for arithmetic, automatic rewriting, induction, and
other relatively large units of deduction; it differs from other highly
automated theorem provers in being directly controlled by the user.  We
have been able to perform some significant new verifications quite
economically using PVS; we have also repeated some verifications first
undertaken in other systems and have usually been able to complete them
in a fraction of the original time (of course, these are previously
solved problems, which makes them much easier for us than for the
original developers).

PVS is the most recent in a line of specification languages, theorem
provers, and verification systems developed at SRI, dating back over
20 years.  That line includes the Jovial Verification
System~\cite{jovial:pv}, the Hierarchical Development Methodology
(HDM)~\cite{Robinson&Levitt,HDM:Handbook}, STP~\cite{STP}, and
\ehdm~\cite{Melliar-Smith&Rushby,EHDM:tutorial}.  We call PVS a
``Prototype Verification System,'' because it was built partly as a
lightweight prototype to explore ``next generation'' technology for
\ehdm, our main, heavyweight, verification system.  Another goal for
PVS was that it should be freely available, require no costly
licenses, and be relatively easy to install, maintain, and use.
Development of PVS was funded entirely by SRI International

%\memo{and
%by our unpaid work at nights and weekends.} and it is made available
%free of charge.

%SRI's investment is not entirely altruistic: we
%expect to significantly increase the size of the market for formal
%methods by making a productive verification system readily available.

In the rest of this introduction, we briefly sketch the purposes for
which PVS is intended and the rationale behind its design, mention
some of the uses that we and others are making of it, and explain how
to get a copy of the system.  In Section~2, we use a simple example to
briefly introduce the major functions of PVS; Sections~3 and 4 then
give more detail on the PVS language and theorem prover, respectively,
also using examples.  More realistic examples are provided in
Section~5.  The PVS language, system, and theorem prover each have
their own reference
manuals~\cite{PVS:language,PVS:prover,PVS:userguide}, which you will
need to study in order to make productive use of the system.  A pocket
reference card, summarizing all the features of the PVS language,
system, and prover is also available.

The purpose of this tutorial is not to introduce the general ideas of
formal methods, nor to explain how formal specification and
verification can best be applied to various problem domains; rather,
its purpose is to introduce some of the more unusual and powerful
capabilities that are provided by PVS.  Consequently, this
document, and the examples we use, are somewhat technical and are most
suitable for those who already have some experience with formal
methods and wish to understand how PVS provides mechanized support for
some of the more challenging aspects of formal methods.

\subsection{Design Goals for PVS}

PVS provides mechanized support for Formal Methods in Computer
Science.  ``Formal Methods'' refers to the use of concepts and
techniques from logic and discrete mathematics in the development of
computer systems, and we assume that you already have some
familiarity with this topic.


Formal methods can be undertaken for many different purposes, in many
different ways and styles, and with varying degrees of rigor.  The
earliest formal methods were concerned with proving programs
``correct'': a detailed specification was assumed to be available and
assumed to be correct, and the concern was to show that a program in
some concrete programming language satisfied the specification.  If
this kind of program verification is your interest, then PVS is not
for you.  You will probably be better served by a verification system
built around a programming language, such as Penelope~\cite{Prasad92} (for
Ada), or by some member of the Larch family~\cite{Larch85}.
Similarly, if your interests are gate-level hardware designs, you
will probably do best to consider model-checking and automatic
procedures based on BDDs~\cite{Clarke-etal90}.

\comment{
We have worked chiefly on problems in ``critical systems'' of various
kinds (e.g., digital flight control) and the design of PVS reflects
this background.  These critical systems are developed according to
very exacting procedures, involving extensive testing and review
(see, e.g., the standards for airborne systems for civil
aircraft~\cite{DO178B}).  The evidence is that few undetected faults are
introduced during the later stages of the lifecycle (i.e., detailed
design and coding) of systems constructed according to these
procedures.  Instead, the overwhelming evidence is that the serious
and undetected faults are introduced during the early stages of the
lifecycle---in requirements formulation, interface specification, and
mistaken or inconsistent assumptions about the behavior of the larger
system with which the computer system must interact.  For example, in
203 formal inspections of six projects at the Jet Propulsion
Laboratory, it was found that requirements documents averaged one
major defect every three pages, compared with one every 20 pages for
code.  Two-thirds of the defects in requirements were
omissions~\cite{Kelly-etal92}.  In other JPL data, 197 faults
detected during integration and system testing of the Voyager and
Galileo spacecraft were characterized as having potentially
significant or catastrophic effects (with respect to the spacecraft's
missions)~\cite{Lutz92:rqts}.  Of these 197 faults, 3 were coding
errors.  The remaining 194 faults were divided approximately 3:1
overall between ``function faults'' (faults within a single software
module) and interface faults (interactions with other modules or
system components).  Two thirds of function faults were attributed to
flawed requirements (of which omissions were the most common flaw);
the remaining one third were due to incorrect implementation of
requirements (i.e., faulty design or algorithms), and tended to
involve inherent technical complexity, rather than a failure to
follow the letter of the requirements.  Turning to hardware,
Keutzer~\cite{Keutzer:hol91} reports that over half of all VLSI chips contain
faults that are only detected after first fabrication.  None of these
faults come from the later stages of the development lifecycle
(probably because of the extensive simulation and analysis
performed); {\em all\/} are due to flawed requirements or to errors
introduced in the earliest stages of design.

Thus, in the fields from which our applications come, the later
stages of the lifecycle are considered under control (maybe
expensively and clumsily, but under control nonetheless).  All the
concern is with the early lifecycle: with requirements, overall
architectural design, interfaces, and critical algorithms.  The
latter arise, for instance, in fault-tolerant systems, where
malfunctioning channels and timing anomalies can lead to unexpected
behavior.  For example, in flight testing of the AFTI-F16 (which had
an early digital flight control system), ill-understood interactions
in the redundancy management and fault-tolerance mechanisms became
the {\em primary\/} source of system failure~\cite{Mackall:TR}.  
}

The design of PVS was shaped by our experience in doing or
contemplating early-lifecycle applications of formal methods.  Many of
the larger examples we have done concern algorithms and architectures
for fault-tolerance (see~\cite{Owre95:prolegomena} for an overview).  We
found that many of the published proofs that we attempted to check
were in fact, incorrect, as was one of the important algorithms.  We
have also found that many of our own specifications are subtly flawed
when first written.  For these reasons, PVS is designed to help in the
detection of errors as well as in the confirmation of ``correctness.''
One way it supports early error detection is by having a very rich
type-system and correspondingly rigorous typechecking.  A great deal
of specification can be embedded in PVS types (for example, the
invariant to be maintained by a state-machine can be expressed as a
type constraint), and typechecking can generate proof obligations that
amount to a very strong consistency check on some aspects of the
specification.\footnote{As a way to further strengthen error checking,
we are thinking of adding dimensions and dimensional analysis to the
PVS type system and typechecker.}

Another way PVS helps eliminate certain kinds of errors is by
providing very rich mechanisms for conservative extension---that is,
definitional forms that are guaranteed to preserve consistency.
Axiomatic specifications can very effective for certain kinds of
problem (e.g., for stating assumptions about the environment), but
axioms can also introduce inconsistencies---and our experience has
been that this does happen rather more often than one would wish.
Definitional constructs avoid this problem, but a limited repertoire
of such constructs (e.g., requiring everything to be specified as a
recursive function) can lead to excessively constructive
specifications: specifications that say ``how'' rather than ``what.''
PVS provides both the freedom of axiomatic specifications, and the
safety of a generous collection of definitional and constructive
forms, so that users may choose the style of specification most
appropriate to their problems.\footnote{Unlike \ehdm, PVS does not
provide special facilities for demonstrating the consistency of
axiomatic specifications.  We do expect to provide these in a later
release, but using a different approach than \ehdm.}

The third way that PVS supports error detection is by providing an
effective theorem prover.  Our experience has been that the act of
trying to prove properties about specifications is the most effective
way to truly understand their content and to identify errors.  This
can come about incidentally, while attempting to prove a ``real''
theorem, such as that an algorithm achieves its purpose, or it can be
done deliberately through the process of ``challenging''
specifications as part of a validation process.  A challenge has the
form ``if this specification is right, then the following ought to
follow''---it is a test case posed as a putative theorem; we
``execute'' the specification by proving theorems about
it.\footnote{Directly executable specification languages
(e.g.,~\cite{me-too,Hekmatpour&Ince:prototyping}) support validation
of specifications by running conventional test cases.  We think there
can be merit in this approach, but that it should not compromise the
effectiveness of the specification language as a tool for deductive
analysis; we are considering supporting an executable subset within
PVS.}

\comment{
Some challenges can be constructed systematically (e.g., it is
always useful to demonstrate that a predicate---a boolean valued
function---is capable of yielding both true and false results), but
most require imagination and insight.  Of course, challenges can be
undertaken without mechanized proof checking, but our experience is
that informal proofs are too often influenced by what the author
thinks the specification says (or ought to say); mechanized proof
checking helps the user understand what the specification really does
say.\footnote{For experienced users, the discovery of overlooked
cases is the most common benefit of performing challenges---which is
consistent with the observation that omissions are the most common
flaw in informal requirements specifications.  Inexperienced users
usually find that their specifications are essentially meaningless.}
Our own experience, and that of users of \ehdm\ and PVS, has been
such that we attach very little credibility to formal specifications
that have not been subjected to some degree of mechanized proof
checking.

A theorem prover that is to help in challenging specifications
obviously needs to be effective---so that the user can concentrate on
the substance of the problem, and not on incidental difficulties of
mechanization---but the notion of ``effective'' must comprehend more
than just ``efficiency'' or ``power.''  It will be obvious from what
we have said that many of the ``theorems'' on which the prover is
invoked will not, in fact, be theorems at all.  Most theorem provers
are set up to prove true theorems, and may waste a lot of time
exploring fruitless paths when confronted by a nontheorem.  Even when
they return with ``unproved,'' it may not be clear whether the
theorem is false, or inadequate heuristics were employed.  The PVS
theorem prover, on the other hand, is designed so that the user
understands and shapes the overall argument.  This not only helps
discover errors fairly effectively (typically, the user discovers an
``obviously false'' case due to some missing constraint or other
oversight), but it maximizes the information content and
understanding derived from successful proofs; our goal in PVS is to
shift the focus from theorems (``can you get it proved?'') to proofs
(``what did you learn by proving it?'').  As a side effect, the
approach to theorem proving used in PVS generally seems to allow
proofs of true theorems to be constructed far more quickly than with
any other theorem prover we know.
}

%\memo{Main use of FM should be to discover assumptions, test
%requirements---to do validation as much as verification.}
%
\comment{
%\section{Design Choices in PVS}

The first choice that must be made in design of a specification
language is selection of its logical foundation.  There are three
main traditions in mathematics and logic to draw on: the {\em
logicist\/} approach of Frege and Russell, which uses higher-order
logic, with the constraints of type theory to keep it consistent; the
{\em formalist\/} approach of Hilbert, Zermelo, and Fraenkel, which
uses first-order logic plus axiomatic set theory; and the {\em
intuitionistic\/} approach of Brouwer, which in computer science is
mostly shaped by Martin-L\"{o}f's notions of propositions-as-types.

Our decisions for PVS were largely pragmatic rather than
philosophical.  The intuitionistic approach is a fruitful area of
research in computer science, and may lead to new understandings of
computational processes, but it does not seem likely to lead, in the
short term at least, to the major gains in productivity of formal
methods that are our aim.

Set theory is the foundation of choice for most mathematicians, but
it is important to understand their motivations, and how they differ
from those who undertake formal methods in computer science.
Set theory was designed as a minimalist foundation within which
``all of mathematics'' could {\em in principle\/} be formalized.
The ``in principle'' is important: mathematicians seldom
actually formalize anything.  Formal methods in computer science, on
the other hand, is concerned with formalizing requirements, designs,
algorithms and programs, and with developing formal proofs {\em in
practice\/}.  Set theory requires mathematical knowledge to be built
from the bottom up by encoding concepts in terms of more primitive
concepts.   Proofs in computing are, however, best carried out with some
degree of abstraction that is unencumbered by details of particular
encodings of concepts.  

%Mathematical logic was developed by mathematicians to address matters
%of concern to them.  Initially, those concerns were to provide a
%minimal and self-evident foundation for mathematics; later, technical
%questions about logic itself became important.  For these reasons,
%much of mathematical logic is set up for {\em meta\/}mathematical
%purposes: to support (relatively) simple proofs of properties such as
%soundness and completeness, and to show that certain elementary
%concepts allow some parts of mathematics to be formalized in
%principle.  

The problems with set theory as a foundation for formal methods are
that many of its constructions are metamathematical (i.e., they are
constructions about set theory, not {\em in\/} set theory---although,
of course, they could performed within set theory {\em in
principle\/}) and use inperspicuous encodings (e.g., Kuratowski and
Wiener's ``programming trick'' for representing the ordered pair $(a,
b)$ as the set $\{a, \{a, b\}\}$).  Of course, these difficulties can
be mitigated: for example, we can provide parameterized modules or
``schemas'' in order to admit some metamathematical constructions,
and we can use a set theory with some of the basic data types built
in to avoid the more grotesque constructions.  Despite these
mitigations, however, set theory still has two problems that we
consider fatal.  First, it is an essentially untyped system:
everything is a set and can be freely combined with other sets.
Thus, a function is a set of pairs, and we can take its union with
some other set.  Our experience has been that strong typing is
essential for efficient and early detection of errors in
specifications, and is a useful discipline in its own right (e.g.,
simply writing down the types of the inputs to a function, or of the
components of a state, can be a valuable first step in a
specification).  Strong typing sits uncomfortably on set theory and
is essentially as kludge (since it must be escaped to perform certain
constructions).  Second, functions in set theory are sets of pairs,
and are inherently partial; total functions are a special case.
Efficient theorem proving usually requires that functions be total,
and can therefore be difficult to achieve in classical set theory.
But there is another difficulty to partial functions apart from this:
it is tricky to ascribe a semantics to expressions involving
functions that might be applied outside their domain.  Mathematicians
avoid this problem, by informally ensuring that their function
applications are always well-defined, but in formal methods we have
to find some way to enforce this mechanically.  The best choice is
probably a logic of partial terms~\cite{Beeson86}, in which terms can
be undefined, but expressions retain the familiar two truth values of
classical logic.  A less satisfactory choice is a logic of partial
functions in which ``undefined'' becomes a truth value and the whole
structure of the logic is modified to accommodate
it~\cite{Cheng&Jones90}.  In either case, the appealing simplicity of
set theory is compromised.  As with the other difficulties described
earlier, it is possible to mitigate these ones, but the mitigations
tend to move in the direction of type theory (also known as
higher-order logic), and it might seem best to simply adopt that
approach in the first place.

Higher-order logic\footnote{``Higher-order'' means that functions can
take functions as arguments and return functions as values, and
allows quantification to range over functions.} is the principal
alternative to axiomatic set theory as a classical (i.e.,
nonintuitionistic) foundation for mathematics.  Higher-order logic
needs the discipline of strong typing to keep it
consistent,\footnote{Russell's paradox shows that unrestricted
higher-order logic is inconsistent (as is unrestricted set theory).
Russell developed his theory of types to restore consistency to
higher-order logic.  His original theory had a notion of ``order'' as
well of type and is known as the ``Ramified Theory of Types.''
Ramsey observed that orders were not needed to exclude the
``logical'' paradoxes, and the theory with types but not orders is
called the ``Simple Theory of Types.''  Its modern formulation is due
to Church.  In current terminology ``Simple Type Theory'' is
equivalent to ``Higher-Order Logic.''} but this aligns with our
wishes anyway.  Many mathematicians find this discipline and some of
its associated distinctions irksome.  For example, the empty set is
not a single notion in type theory: there is a different empty set
for each type of elements.  Mathematicians call this ``reduplication
of notions'' ``repugnant''~\cite{Fraenkel-etal84}, but it is
perfectly defensible on linguistic grounds (e.g., is having no money
the same as having no worries?), and no trouble for formal methods in
practice.\footnote{Mathematicians also saddled themselves with opaque
notation for type theory: they reverse the order of the type symbols,
``curry'' all functions, and write applications without parentheses.
Computer Scientists are usually happy to ``declare'' the types of
variables and functions before use, and can take advantage of
computerized analysis that supports notational conveniences such as
name overloading and type-inference.}

One attraction of higher-order logic as a foundation for mechanized
formal methods is that it is very expressive: it is possible to say
a great deal in higher-order logic without metalogical assistance.
A second attraction is that it is inherently a typed system, and a
third is that functions are total, so that the lower levels of
theorem proving can be made relatively efficient.

For the reasons described, we have chosen higher-order logic as the
foundation for the PVS specification language.  Essentially similar
logics also provide the foundations for \ehdm, and for
HOL~\cite{Gordon:HOL88}, and the wide range of examples successfully
undertaken in those systems attest to the utility of higher-order
logic as a foundation for formal methods.  PVS differs from HOL in
supplying built-in interpretations for the {\tt integer} and {\tt
rational} numeric types, and in providing records, enumerations, and
certain tree-like data structures through built-in type constructors
(in addition to functions and tuples).  In addition PVS allows {\em
predicate subtypes\/} and {\em dependent types\/}.  These greatly
increase the ``precision'' with which terms may be typed---to such an
extent that typechecking is no longer a deterministic operation but
can require the assistance of the theorem prover.

As their name suggests, predicate subtypes use a predicate to induce
a subtype on some parent type.   For example, the natural numbers are
specified in PVS as:
\[ {\em nat}: {\bf type} = \{ n: {\em int} | n >= 0 \}. \]
More interestingly, the signature for the division operation (on the
rationals) is specified by
\[ / : [rat, nonzero\_rat -> rat] \]
where
\[{\em nonzero\_rat}: {\bf type} = \{ x : {\em rat} | x /= 0 \}\]
specifies the nonzero rational numbers.   
This constrains division to nonzero divisors, so that a formula
such as
\[ x /= y => (y-x)/(x-y) <0 \]
requires the typechecker to discharge the proof obligation
\[ x /= y => (x-y) /= 0 \]
in order to ensure that the occurrence of division is well-typed.
Proof obligations such as this are called Typecheck Correctness
Conditions, or TCCs; they are sufficient (though not always
necessary) conditions which ensure that the values of logical
expressions do not depend on functions applied outside their domains.
The decision procedures of the PVS theorem prover  can
instantly dispose of simple TCCs similar to this example.   More
complex TCCs are carried along as proof obligations that must
discharged (under control of the user) before analysis of the
specification is considered complete.

As the example of division illustrates, predicate subtypes allow
certain functions that are partial in some other treatments to remain
total (thereby avoiding the need for logics of partial terms or
three-valued logics).  Related constructions allow nice treatments of
errors, such as {\em pop\/}({\em empty\/}) in the theory of stacks.
Here we can type the stack operations as follows:
\begin{alltt}\rm
  {\em stack\/}: {\bf type}
  {\em empty\/}: {\em stack\/}
  {\em nonempty\_stack\/}: {\bf type} = \{({\em s: stack\/}) \(|\) {\em s\/} \(\neq\) {\em empty\/}\}

  {\em push\/}: [{\em elem\/}, {\em stack\/} \(\rightarrow\) {\em nonempty\_stack\/}]
  {\em pop\/}: [{\em nonempty\_stack} \(\rightarrow\) {\em stack\/}]
  {\em top\/}: [{\em nonempty\_stack} \(\rightarrow\) {\em elem\/}]
\end{alltt}
With these signatures, the expression ${\em pop\/}({\em empty\/})$ is
rejected during typechecking (because {\em pop\/} requires a {\em
nonempty\_stack\/} as its argument), and the theorem
\[ {\em push\/}(e, s) \neq {\em empty\/}\]
is an immediate consequence of the type definitions.   More
interestingly, the formula
\[{\em pop\/}({\em pop\/}({\em push\/}(x, {\em push\/} (y, s)))) = s\]
is shown to be well-typed by proving the TCC
\[{\em pop\/}({\em push\/}(x, {\em push\/}(y, s))) /= {\em empty\/},\]
which follows from the usual stack axioms.  

Untrue proof obligations indicate a type-error
in the specification, and have proved a potent method for the early
discovery of specification errors.
For example, the injections are specified as that subtype of the
 functions associated with the one-to-one property:
\[{\em injection}: {\bf type} = \{f: [t_{1} -> t_{2}] \;|
	 \;\forall (i, j: t_{1}): f(i) = f(j) => i = j\}\]
(here $t_{1}$ and $t_{2}$ are uninterpreted types introduced
in the module parameter list).
If we were later to specify the function {\em square\/} as an
injection on the integers by the declaration
\[ {\em square}: {\em injection} = 
	lambda (x: {\em int}): x \times x\]
then the PVS typechecker would require us to show that the body of
{\em square\/} satisfies the {\em injection\/} subtype predicate.
That is, it requires the proof obligation $i^{2} = j^{2} => i = j$ to be
proved in order to establish that the {\em square\/} function is
well-typed.  Since this theorem is untrue (e.g., $2^{2} = (-2)^{2}$
but $2 \neq -2$), we are led to discover a fault in this
specification.

Notice how use of predicate subtypes here has automatically led to
the generation of proof obligations that might require
special-purpose tools in other systems.  Yet another example of the
utility of predicate subtypes arises when modeling a system by means
of a state machine.  In this style of specification, we first
identify the components of the system state; an invariant then
specifies how the components of the system state are related, and
operations are required to preserve this relation.  With predicate
subtypes available, we can use the invariant to induce a subtype on
the state type, and can specify that each operation returns a value
of that subtype.  Typechecking the specification will then
automatically generate the proof obligations necessary to ensure that
the operations preserve the invariant.

Dependent types increase the expressive convenience of the language
still further.   We find them particularly convenient for dealing
with functions that would be partial in simpler type systems.
The standard ``challenge'' for treatments of partial
functions~\cite{Cheng&Jones90} is the function {\em subp\/} on the
integers defined by
\[{\em subp\/}(i, j) = \rmif i = j \rmthen 0 
\rmelse {\em subp\/}(i,j+1)+1 \rmendif.\]
This function is undefined if $i<j$ (when $i \geq j, {\em
subp\/}(i,j)=i-j$) and it is often argued that if a specification
language is to admit this kind of definition, then it must provide a
treatment for partial functions.  Fortunately, examples such as these
do {\em not\/} require partial functions: they can be admitted as
total functions on a very precisely specified domain.  {\em Dependent
types\/}, in which the {\em type\/} of one component of a structure
depends on the {\em value\/} of another, are the key to this.  For
example, in the language of PVS, {\em subp\/} can be specified as
follows.
\begin{alltt}\rm
  {\em subp}({\em i:int}, ({\em j:int} \(|\) \(i\geq j\))): {\bf recursive} {\em int\/} =
       (\rmif \(i=j\) \rmthen 0 \rmelse \({\em subp\/}(i, j+1)+1\) \rmendif)
   {\bf measure} \((\lambda (i:{\em int\/}), (j:{\em int\/} | i\geq j): i-j)\)\footnotemark
\end{alltt}
\footnotetext{The {\bf measure} clause specifies the function to
be used in the termination proof.}
Here, the domain of {\em subp\/} is the dependent tuple-type
\[ [i:{\em int\/}, \{j:{\em int\/} | i >= j\}]\]
(i.e., the pairs of integers in which the first component is greater
than or equal to the second) and the function is total on this domain.

PVS is not unique in providing predicate and dependent types;
Nuprl~\cite{Nuprl-book} and Veritas~\cite{Hanna89:Veritas}, for
example, also support these constructions.  PVS differs from others
in that we support a rich type-system within an entirely classical
framework (Nuprl mechanizes a constructive type theory, Veritas
provides the unusual combination of a Martin-L\"{o}f type system and
a classical higher-order logic).

A unique feature of PVS is the tightness of the integration between
the specification language and its typechecker, and the theorem
prover.  We have just seen examples how willingness to use theorem
proving in typechecking provides simple and sound solutions to
problems that can otherwise require very complex treatments.
Conversely, the PVS theorem prover can exploit information from the
typechecker to guide its search and to decide certain properties.
Furthermore, the prover uses the PVS language-processing tools, such
as the parser, typechecker and prettyprinters, so that its dialog
with the user is conducted entirely in terms of the PVS specification
language (even though much more austere forms are used internally),
and so that the user can modify or add the statement of a lemma or
definition during an ongoing proof.  Much of what happens during a
proof attempt is the discovery of inadequacies, oversights, and
faults in the specification that is intended to support the theorem.
Having to abandon the current proof attempt, correct the problem, and
then get back to the previous position in the proof, can be very time
consuming.  Allowing the underlying specification to be extended and
modified during a proof confers enormous gains in productivity.

The design of the PVS theorem prover was guided by our conviction
that proofs are at least as important as theorems: it is usually not
enough to know that a theorem is true, we need to understand {\em
why\/} it is true---because this understanding will be needed if (or,
more likely, when) we need to modify the specification to accommodate
changed assumptions, requirements, or designs.  The PVS prover is
therefore designed so that the main steps of the proof are given by
the user; the theorem prover automates the bookkeeping and the routine
steps and provides the interactive environment that allows the user
to explore and develop the main argument.  The ideal to which we
aspire is that developing a proof with PVS should be akin to
developing one with a knowledgeable but skeptical human colleague.
This means that the basic inferences that PVS performs automatically
must be rather powerful, so that the ``dialog'' between user and
machine is not interrupted by tedious subcases to establish trivial
facts of arithmetic or to expand and simplify definitions.

For the reasons just explained, the basic inference steps in PVS were
chosen to be powerful in comparison with the simple rules given in
textbook introductions to logic.  Each inference step is flexible, so
it can be used in a variety of related ways, and takes optional
parameters that adjust its behavior.  For example, the beta reduction
rule eliminates all redexes (and for flexibility many things are
regarded as redexes) from a set of formulas specified by a parameter
(the default is all formulas).  PVS also provides a mechanism for
composing basic inference steps into proof ``strategies'' (rather
like subroutines in a programming language, or the tacticals of
LCF-style provers), a facility for rerunning proofs, and another to
check that all secondary proof obligations (such as Type Correctness
Conditions) have been discharged.  Powerful primitive inferences make
the composed inference steps correspondingly more powerful, and allow
the proof to be represented in a manner that can be rerun efficiently
and that is robust in the face of small changes to the specification
or theorem.  A small and carefully chosen set of primitive inferences
also makes the system easier to learn and use.

Interaction between PVS and the user is organized in the manner of
Gentzen's Sequent Calculus.  This is explained in the documents
describing the theorem prover, but the basic idea is that a proof is
developed as a tree of {\em sequents\/}; each sequent can be
considered as a disjunction of ``antecedent'' formulas that implies a
conjunction of ``consequent'' formulas; at any instant, the focus of
attention is one of the leaf sequents in the proof tree; a proof step
(i.e., a primitive inference) either recognizes the current focus
sequent as true and shifts attention to some other leaf sequent in
the proof tree, or else it adds one or more children to the current
focus sequent and shifts attention to one of those children; the goal
is to develop a proof tree whose leaves are all recognized as true.
This is a ``backwards'' approach to proof, in that we start from the
conclusion to be proved and progressively apply inference steps to
generate subgoals until the subgoals are trivially provable.  The
attraction of the sequent calculus is that a sequent is a
very compact and clear way to represent all the information relevant
to the current step of the proof, and the basic steps are very
regular and intuitive.

The various tools and functions of PVS use the GNU Emacs editor
running under Unix as their interface: the theorem prover runs in its
own buffer and its commands are typed directly into that buffer; the
other tools and functions are invoked through extended Emacs
commands.  This means that you do need to learn Emacs in order to use
PVS effectively.  Apart from the extra functionality of the PVS
commands, the PVS Emacs is a perfectly standard Gnu Emacs and that
can be used for editing non-PVS files, reading mail and news, and so
on.  We chose Emacs as the interface for PVS partly for its
portability and economy---the rich functionality of Emacs and of
extensions such as ILISP allowed us to construct many attractive
capabilities rather easily and inexpensively---and partly from
personal preference: Emacs is our interface of choice for everything
else we use our computers for.

PVS specifications are composed of units called {\em theories\/},
which are stored in standard ascii text files with extension {\tt
pvs}.  Each PVS file contains one or more theories, and a collection
of such files stored in one directory make up a specification.  Any
proofs developed for the theories in a given PVS file are saved in a
file with the same name, but extension {\tt prf}.  The {\tt .pvs} and
{\tt .prf} files in a single directory constitute a PVS {\em
context\/} whose state of development is automatically saved and
restored from one PVS session to another.  A typical PVS session
begins by loading and modifying some existing PVS files, or creating
new some new ones, using standard Emacs editing capabilities.
Usually, the commands to parse and typecheck a PVS file will be given
next.  Either or both of these operations may detect errors in the
PVS specification whose correction may require some iteration of
these steps.  Next, a proof may be attempted; this is started by
moving the cursor to the formula to be proved and giving the
appropriate command.  If a proof has already been saved for the
formula concerned, the user is given the option of rerunning it or
developing a new proof from scratch.  If the former option is chosen,
the saved proof may succeed, or it may not (e.g., because the
specification has changed).  In the latter case, control is returned
to the user in the same state as if the saved proof had just been
developed interactively; the user can now choose to undo some proof
steps on certain branches of the proof tree in order to develop a
modified proof suitable to the changed specification.  An alternative
approach, which is especially suitable when a specification changes
in a very regular way (e.g., a function is renamed), is to use the PVS
facilities for editing saved proof scripts---this approach is most
suitable for more advanced users.  An edited proof can be attached to
a different (or additional) formula than the one it came from
originally.  This can be very convenient if several theorems have a
very similar form and should yield to similar proofs.  Another
approach in these cases is to develop a {\em strategy\/}, that is
series of PVS proof steps combined within a control mechanism that
can be used rather like a ``proof subroutine.''

When a specification is modified, all saved proofs associated with
the changed PVS files become ``suspect'': the system will once again
consider their corresponding theorems ``proved'' only when the PVS
prover has successfully rerun them.  PVS provides commands for
rerunning such proofs as a batch, and for discovering the current
status of the theories, files, and proofs that constitute a
specification.

Specifications and proofs generally need to be studied by others than
their original developers.  PVS provides a prettyprinter for
reformatting specifications in a very regular manner, and a rather
versatile \LaTeX-printer that can be used to typeset PVS
specifications.  The \LaTeX-printer can be customized by simple
user-supplied tables in ways that allow it to reproduce standard
mathematical notation.  The same capability can also be used to
typeset a proof transcript.  PVS is also able to generate a
cross-reference to the declarations of identifiers, and has functions
that allow the declaration or uses of an identifier under the cursor
to be viewed or visited.

%\section{What's in PVS?}

In this section we briefly list the capabilities and functions of the
PVS language, prover, and system.  

TBD.

%\section{Comparing PVS to other Verification Systems}

In this section we briefly compare the facilities, capabilities, and
design choices employed in PVS with those of a number of other
systems that you might be familiar with.  The purpose of this section
is not to suggest that PVS is better than these other very fine
systems, but to give you an idea how it differs from them, and
thereby to help you decide whether PVS is likely to provide the
services you need.

% three traditions.

% VDM, Z, and mural etc,
% BM, Otter, Paulson's system
% HOL, Eves, Imps

% tight leash or drag towards conclusion
}

\subsection{Uses of PVS}

PVS has so far been applied to several small demonstration examples,
and a growing number of significant verifications.  The smaller
examples include the specification and verification of ordered binary
tree insertion~\cite{Shankar:ADT}, a compiler for simple arithmetic
expressions~\cite{Rushby95:Movie}, and several small hardware examples
including pipeline and microcode correctness~\cite{Cyrluk94:TPCD}.
Examples of this scale can typically be completed within a day.  More
substantial examples include the correctness of a real-time railroad
crossing controller~\cite{Shankar93:CAV}, an embedding of the Duration
Calculus~\cite{Skakkebaek&Shankar94}, the correctness of some
transformations used in digital syntheses~\cite{Sree94:TR}, and the
correctness of distributed agreement protocols for a hybrid fault
model consisting of Byzantine, symmetric, and crash
faults~\cite{Lincoln&Rushby93:CAV,Lincoln&Rushby93:FTCS,Lincoln&Rushby94:FTP}.
These harder examples can take from several days to several weeks.
Industrial applications of PVS include verification of selected
elements of a commercial avionics microprocessor whose implementation
has 500,000 transistors~\cite{Miller&Srivas95}.
Some of these applications of PVS are summarized
in~\cite{Owre95:prolegomena}, which also
motivates and describes
some of the design decisions underlying PVS\@.
Applications of PVS undertaken independently of SRI
include~\cite{Hooman94,Butler:PVS-tut,Johnson94:TPCD,Miner94:circuit}.

\subsection{Getting and Using PVS}

At the moment, PVS is readily available only for Sun SPARC
workstations running SunOS 4.1.3, although versions of the system have
been run on IBM Risc 6000 (under AIX) and DECSystem 5000 (under
Ultrix).  PVS is implemented in Common Lisp (with CLOS), and has been
ported to Lucid, Allegro, AKCL, CMULISP, and Harlequin Lisps.
Only the Lucid and Allegro versions deliver acceptable performance.
All versions of PVS require \gnuemacs, which must be obtained
separately.  It is not particular about the window system, as long as
it supports \gnuemacs, although some facilities for presenting
graphical representaitons of theory dependencies and proof trees
(implemented in Tcl/TK) do require X-Windows.  In addition, \LaTeX\
and an appropriate viewer are needed to support certain optional
features of \pvs.

PVS is quite large, requiring about 50 megabytes of disk space.  In
addition, any system on which it is to be run should have a minimum of
100 megabytes of swap space and 48 megabytes of real memory (more is
better).  To obtain the \pvs\ system, send a request to {\tt
pvs-request@csl.sri.com}, and we will provide further instructions for
obtaining a tape or for getting the system by FTP\@.  Alternatively,
you may inspect the installation instructions over WWW at URL {\tt
http://www.csl.sri.com/pvs.html}.  All installations of PVS must be
licensed by SRI\@.  The Lucid Lisp version requires that you have a
runtime license for Lucid Lisp.  A nominal distribution fee is charged
for tapes; there is no charge for obtaining PVS by FTP.

% Document Type: LaTeX
% Master File: tutorial.tex
\section{A Brief Tour of \pvs}
\label{system-tutorial}

In this section we introduce the system by developing a theory and
doing a simple proof.  This will introduce the most useful commands
and provide a glimpse into the philosophy behind \pvs.  You will get
the most out of this section if you are sitting in front of a
workstation (or terminal) with \pvs\ installed.  In the following we
assume familiarity with Sun Unix and \gnu.

Start by going to a \unix\ shell window and creating a working
directory (using {\tt mkdir}). Next, connect ({\tt cd}) to that
working directory and start up \pvs\ by typing {\tt pvs}.\footnote{You
may need to include a pathname, depending on where and how \pvs\ is
installed.} This command executes a shell script which runs \gnu,
loads the necessary \pvs\ \emacs\ extensions, and starts the \pvs\
lisp image as a subprocess.\footnote{All the \gnu\ (and X-Windows or
Emacstool) command line flags can be added to the {\tt pvs} command
and passed through as appropriate; the {\tt -q} flag inhibits loading
of the user's {\tt .emacs} initialization file, and should be used if
difficulties are encountered starting \pvs\ or if there appear to be
conflicts in keybindings.  Do {\em not\/} report errors to us unless
they can be reproduced when the {\tt -q} flag is used.} 
After a few moments, you should see the
welcome screen indicating the version of \pvs\ being run, the current
directory, and instructions for getting help.  You may be asked
whether you want to create a new context in the directory; answer {\tt
yes} unless it is the wrong directory or you don't have write
permission there, in which case you should answer {\tt no} and provide
an alternative directory when prompted.

\pvs\ uses \emacs\ as its interface by extending \emacs\ with \pvs\
functions, but all the underlying capabilities of \emacs\ are available.
Thus the user can read mail and news, edit non\pvs\ files, or execute
commands in a shell buffer in the usual way.

In the following, \pvs\ \emacs\ commands are given first in their long
form, followed by an alternative abbreviation and/or key binding in
parentheses.  For example, the command for proving in \pvs\ is given
as \ecmd{prove} (\ecmd{pr}, \key{C-c p}).  This command can be entered
by typing the {\tt Escape} key, then an {\tt x}\footnote{Many
keyboards provide a {\tt Meta} key (hence the {\tt M-} prefix), and
this may be used instead.  On the \sun 3, the {\tt Meta} key is
normally labeled {\tt Left} and on the \sun 4 ({\sc sparc}), it is
labeled $\Diamond$.  The {\tt Meta} key is like the shift key; to use
it simply hold the {\tt Meta} key down while typing another key.}
followed by {\tt prove} (or {\tt pr}) and the {\tt Return} key.
Alternatively, hold the {\tt Control} key down while typing a {\tt c},
then let go and type a {\tt p}.  The {\tt Return} key does not need to
be pressed when giving the key binding form.  In \pvs\ all commands
and abbreviations are preceded by a {\tt M-x}; everything else is a
key-binding.  In later sections we will refer to commands by their
long form name, without the {\tt M-x} prefix.  Some of the commands
prompt for a theory or \pvs\ file name and specify a default; if the
default is the desired theory or file, you can simply type the {\tt
Return} key.  Although the basic keyword commands described here are
preferred by most serious users, \pvs\ commands are also available as
menu selections if you are running under \emacs\ 19.

To begin, type \iecmd{pvs-help} ({\tt C-h p}) for an overview of the
commands available in \pvs\ (type {\tt q} to exit the help buffer).
To exit \pvs, use \iecmd{exit-pvs} (\key{C-x C-c}).

\pvs\ specifications consist of a number of files, each of which
contains one or more theories.  Theories may import other theories;
imported theories must either be part of the prelude (the standard
collection of theories built-in to PVS), or the files containing them
must be in the same directory.\footnote{\pvs\ does support soft links,
thus supporting a limited capability for reusing theories.}
Specification files in \pvs\ all have a {\tt .pvs} extension.  As
specifications are developed, their proofs are kept in files of the
same name with {\tt .prf} extensions.  The specification and proof
files in a given directory constitute a PVS {\em context\/}; \pvs\
maintains the state of a specification between sessions by means of
the {\tt .pvscontext} file.  The {\tt .pvscontext} and {\tt .prf}
files are not meant to be modified by the user.  Other files used or
created by the system will be described as needed.  You may move to a
different context (\ie\ directory) using the \ecmd{change-context}
command, which is analogous to the \unix\ {\tt cd} command.

Now let's develop a small specification:
\begin{pvsex}
sum: THEORY
 BEGIN
  n: VAR nat
  sum(n): RECURSIVE nat =
   (IF n = 0 THEN 0 ELSE n + sum(n - 1) ENDIF)
   MEASURE (LAMBDA n: n)
  closed_form: THEOREM sum(n) = (n * (n + 1))/2
 END sum
\end{pvsex}
%
This is a specification for summation of the first $n$ natural numbers

This simple theory has no parameters and contains three declarations.
The first declares {\tt n} to be a variable of type {\tt nat}, the
built-in type of natural numbers.  The next declaration is a recursive
definition of the function {\tt sum(n)}, whose value is the sum of the
first {\tt n} natural numbers.  Associated with this definition is a
{\em measure\/} function, following the {\tt MEASURE} keyword, which
will be explained below.\footnote{In this case, the measure is the
identity function, which could have been written simply as {\tt MEASURE
n}.} The final declaration is a formula which gives the closed form of
the sum.

\subsection{Creating the Specification}

The {\tt sum} theory may be introduced to the system in a number of
ways, all of which create a file with a {\tt .pvs}
extension,\footnote{The file does not have to be named {\tt sum.pvs}, it
simply needs the {\tt .pvs} extension.} which can be done by
\begin{enumerate}

\item using the {\tt M-x new-pvs-file} command (\ecmd{nf}) to create a new
\pvs\ file, and typing {\tt sum} when prompted.  Then type in the {\tt
sum} specification.

\item Since the file is included on the distribution tape in the {\tt
Examples/tutorial} subdirectory of the main \pvs\ directory, it can be
imported with the {\tt M-x import-pvs-file} command (\ecmd{imf}).  Use
the \ecmd{whereis-pvs} command to find the path of the main \pvs\
directory.

\item Finally, any external means of introducing a file with extension
{\tt .pvs} into the current directory will make it available to the
system; for example, using {\tt vi} to type it in, or {\tt cp} to copy
it from the {\tt Examples/tutorial} subdirectory.

\end{enumerate}
The first two alternatives display the specification in a buffer.
The third option requires an explicit request such as a built-in \gnu\
file command (like {\tt M-x find-file}, {\tt C-x C-f}), or the {\tt M-x
find-pvs-file} ({\tt M-x ff} or {\tt C-c C-f}) command.  The latter is
more useful when there are multiple specification files, as it supports
completion on just the specification files, ignoring other files that
you or the system have created in the directory.

\subsection{Parsing}

Once the {\tt sum} specification is displayed, it can be parsed with the
{\tt M-x parse} ({\tt M-x pa}) command, which creates the internal
abstract representation for the theory described by the specification.
If the system finds an error during parsing, an error window will pop up
with an error message, and the cursor will be placed in the vicinity of
the error.  If you didn't get an error, introduce one (say by
misspelling the {\tt VAR} keyword), then move the cursor somewhere else and
parse the file again (note that the buffer is automatically saved).  Fix
the error and parse once more.  In practice, the parse command is rarely
used, as the system automatically parses the specification when it needs
to.

\subsection{Typechecking}
\index{typecheck|(}

The next step is to typecheck the file by typing \ecmd{typecheck}
(\ecmd{tc}, \key{C-c t}), which checks for semantic errors, such as
undeclared names and ambiguous types.  Typechecking may build new files
or internal structures such as \tccs.  When {\tt sum} has been
typechecked, a message is displayed in the minibuffer indicating that
two \tccs\index{TCCs@\tccs|(} were generated.  These \tccs\ represent
{\em proof obligations\/} that must be discharged before the {\tt sum}
theory can be considered typechecked.  The proofs of the \tccs\
may be postponed indefinitely, though it is a good idea to view them to
see if they are provable.  \tccs\ can be viewed using the \ecmd{show-tccs}
command, the results of which are shown in Figure~\ref{sum-tccs} below.

\pvstheory{sum-tccs}{\tccs\ for Theory {\tt sum}}{sum-tccs}

The first \tcc\ is due to the fact that {\tt sum} takes an argument of
type {\tt nat}, but the type of the argument in the recursive call to
{\tt sum} is integer, since {\tt nat} is not closed under subtraction.
Note that the \tcc\ includes the condition {\tt NOT n = 0}, which holds
in the branch of the {\tt IF-THEN-ELSE} in which the expression
{\tt n - 1} occurs.

The second \tcc\ is needed to ensure that the function {\tt sum} is
total, \ie\ terminates.  \pvs\ does not directly support partial
functions, although its powerful subtyping mechanism allows \pvs\ to
express many operations that are traditionally regarded as partial.  The
measure function is used to show that recursive definitions are total by
requiring the measure to decrease with each recursive call.

These \tccs\ are trivial, and in fact can be discharged automatically
by using the \ecmd{typecheck-prove} (\ecmd{tcp}) command, which attempts
to prove all \tccs\ that have been generated.  (Try it).
\index{TCCs@\tccs|)}\index{typecheck|)}

\subsection{Proving}

We are now ready to try to prove the main theorem.  Place the cursor on
the line containing the {\tt closed\_form} theorem, and type
\ecmd{prove} (\ecmd{pr} or \key{C-c p}).  A new buffer will pop up, the
formula will be displayed, and the cursor will appear at the {\tt Rule?}
prompt, indicating that the user can interact with the prover.  The
commands needed to prove this theorem constitute only a very small
subset of the commands available to the prover; more details can be
found in the prover guide~\cite{PVS:prover}.

First, notice the display (reproduced below), which consists of a
single formula (labeled {\tt \{1\}}) under a dashed line.  This is a
{\em sequent\/}; formulas above the dashed lines are called {\em
antecedents\/} and those below are called {\em succedents\/}.  The
interpretation of a sequent is that the conjunction of the antecedents
implies the disjunction of the succedents.  Either or both of the
antecedents and succedents may be empty.\footnote{An empty antecedent
is equivalent to {\tt true}, and an empty succedent is equivalent to
{\tt false}, so if both are empty the sequent is unprovable.} In our
case, we are trying to prove a single succedent.

The basic objective of the proof is to generate a {\em proof tree\/} in
which all of the leaves are trivially true.  The nodes of the proof tree
are sequents, and while in the prover you will always be looking at an
unproved leaf of the tree.  The {\em current\/} branch of a proof is the
branch leading back to the root from the current sequent.  When a given
branch is complete (\ie\ ends in a true leaf), the prover automatically
moves on to the next unproved branch, or, if there are no more unproven
branches, notifies you that the proof is complete.

Now back to the proof.  We will prove this formula by induction on {\tt
n}.  To do this, type {\tt (induct "n")}.\footnote{\pvs\ expressions are
case-sensitive, and must be put in double quotes when they appear as
arguments in prover commands.} This is not an \emacs\ command, rather it
is typed directly at the prompt, including the parentheses.  This
generates two subgoals; the one displayed is the base case, where {\tt
n} is {\tt 0}.  To see the inductive step, type {\tt (postpone)}, which
postpones the current subgoal and moves on to the next unproved one.
Type {\tt (postpone)} a second time to cycle back to the original
subgoal (labeled {\tt closed\_form.1}).\footnote{Three extremely useful
\emacs\ key sequences to know here are \key{M-p}, \key{M-n}, and
\key{M-s}.  \key{M-p} gets the last input typed to the prover; further
uses of \key{M-p} cycle back in the input history.  \key{M-n} works in
the opposite direction.  To use \key{M-s}, type the beginning of a
command that was previously input, and type \key{M-s}.  This will get
the previous input that matches the partial input; further uses of
\key{M-s} will find earlier matches.  Try these key sequences out; they
are easier to use than to explain.}

To prove the base case, we need to expand the definition of {\tt sum},
which is done by typing {\tt (expand "sum")}.  After expanding the
definition of {\tt sum}, we send the proof to the \pvs\ decision
procedures, which automatically decide certain fragments of
arithmetic, by typing {\tt (assert)}.\footnote{The {\tt
assert} command actually does a lot more than decide arithmetical
formulas, performing three basic tasks:
\begin{itemize}\def\itemsep{0in}
\item it tries to prove the subgoal using the decision procedures.

\item it stores the subgoal information in an underlying database,
allowing automatic use to be made of it later.

\item it simplifies the subgoal, again utilizing the underlying decision
procedures.
\end{itemize}
These arithmetic and equality procedures are the main workhorses to
most \pvs\ proofs.  You should learn to use them effectively in a
proof.} 
This completes the proof of this subgoal, and the system moves on to
the next subgoal, which is the inductive step.

The first thing to do here is to eliminate the {\tt FORALL} quantifier.
This can most easily be done with the {\tt skolem!}\
command\footnote{The exclamation point differentiates this command from
the {\tt skolem} command, where the new constants have to be provided by
the user.}, which provides new constants for the bound variables.  To
invoke this command type {\tt (skolem!)} at the prompt.  The resulting
formula may be simplified by typing {\tt (flatten)}, which will break up
the succedent into a new antecedent and succedent.  The obvious thing to
do now is to expand the definition of {\tt sum} in the succedent.  This
again is done with the {\tt expand} command, but this time we want to
control where it is expanded, as expanding it in the antecedent will not
help.  So we type {\tt (expand "sum" +)}, indicating that we want to
expand {\tt sum} in the succedent.\footnote{We could also have specified
the exact formula number (here {\tt 1}), but including formula numbers
in a proof tends to make it less robust in the face of changes.  There
is more discussion of this in the prover guide~\cite{PVS:prover}.}

The final step is to send the proof to the \pvs\ decision procedures
by typing {\tt (assert)}.  The proof is now complete, the system may
ask whether to save the new proof, and whether to display a brief
printout of the proof.  You should answer {\tt yes} to these questions
just to see how they work.  After responding to these questions, the
buffer from which the \cmd{prove} command was issued is redisplayed if
necessary, and the cursor is placed on the formula that was just
proved.  The entire proof transcript is shown below.  Yours may be
different, depending on your window size and the timings involved.

{\smaller\smaller\smaller\begin{alltt}
     closed_form :  
     
       |-------
     \{1\}   (FORALL (n: nat): sum(n) = (n * (n + 1)) / 2)
     
     Rule? {\bf (induct "n")}
     Inducting on n,
     this yields  2 subgoals: 
     closed_form.1 :  
     
       |-------
     \{1\}   sum(0) = (0 * (0 + 1)) / 2
     
     Rule? {\bf (postpone)}
     Postponing closed_form.1.
     
     closed_form.2 :  
     
       |-------
     \{1\}   (FORALL (j: nat):
              sum(j) = (j * (j + 1)) / 2
                IMPLIES sum(j + 1) = ((j + 1) * (j + 1 + 1)) / 2)
     
     Rule? {\bf (postpone)}
     Postponing closed_form.2.
     
     closed_form.1 :  
     
       |-------
     \{1\}   sum(0) = (0 * (0 + 1)) / 2
     
     Rule? {\bf (expand "sum")}
     (IF 0 = 0 THEN 0 ELSE 0 + sum(0 - 1) ENDIF) 
     simplifies to 0
     Expanding the definition of sum,
     this simplifies to: 
     closed_form.1 :  
     
       |-------
     \{1\}   0 = 0 / 2
     
     Rule? {\bf (assert)}
     Simplifying, rewriting, and recording with decision procedures,
     
     This completes the proof of closed_form.1.
     
     closed_form.2 :  
     
       |-------
     \{1\}   (FORALL (j: nat):
              sum(j) = (j * (j + 1)) / 2
                IMPLIES sum(j + 1) = ((j + 1) * (j + 1 + 1)) / 2)
     
     Rule? {\bf (skolem!)}
     Skolemizing,
     this simplifies to: 
     closed_form.2  |-------
     \{1\}   sum(j!1) = (j!1 * (j!1 + 1)) / 2
             IMPLIES sum(j!1 + 1) = ((j!1 + 1) * (j!1 + 1 + 1)) / 2
     
     Rule? {\bf (flatten)}
     Applying disjunctive simplification to flatten sequent,
     this simplifies to: 
     closed_form.2 :  
     
     \{-1\}   sum(j!1) = (j!1 * (j!1 + 1)) / 2
       |-------
     \{1\}   sum(j!1 + 1) = ((j!1 + 1) * (j!1 + 1 + 1)) / 2
     
     Rule? {\bf (expand "sum" +)}
     (IF j!1 + 1 = 0 THEN 0 ELSE j!1 + 1 + sum(j!1 + 1 - 1) ENDIF) 
     simplifies to 1 + sum(j!1) + j!1
     Expanding the definition of sum,
     this simplifies to: 
     closed_form.2 :  
     
     [-1]   sum(j!1) = (j!1 * (j!1 + 1)) / 2
       |-------
     \{1\}   1 + sum(j!1) + j!1 = (2 + j!1 + (j!1 * j!1 + 2 * j!1)) / 2
     
     Rule? {\bf (assert)}
     Simplifying, rewriting, and recording with decision procedures,
     
     This completes the proof of closed_form.2.
     
     Q.E.D.
     
     
     Run time  = 5.62 secs.
     Real time = 58.95 secs.

\end{alltt}}

Note: The proof presented here is a low-level interactive one chosen
for illustrative purposes.  In practice, trivial theorems such as this
are handled automatically by the higher-level strategies of PVS.  This
particular theorem, for example, is proved automatically by the single
command {\tt (induct-and-simplify "n" :defs T)}.


\subsection{Status}

Now type \iecmd{status-proof-theory} (\ecmd{spt}) and you will see a
buffer which displays the formulas in {\tt sum} (including the \tccs),
along with an indication of their proof status.  This command is
useful to see which formulas and \tccs\ still require proofs.  Another
useful command is \iecmd{status-proofchain} (\ecmd{spc}), which
analyzes a given proof to determine its dependencies.  To use this,
go to the {\tt sum.pvs} buffer, place the cursor on the {\tt
closed\_form} theorem, and enter the command.  A buffer will pop up
indicating whether the proof is complete, and that it depends on the
\tccs\ and the {\tt nat\_induction} axiom.

\subsection[Generating \LaTeX]{Generating \BoldLaTeX}

In order to try out this section, you must have access to \LaTeX\ and a
\TeX\ previewer, such as {\tt vitex} or {\tt dvitool} (for \sunview), or
{\tt xdvi} (for X-windows).  Otherwise this section may be skipped.

Type \iecmd{latex-theory-view} (\ecmd{ltv}).  You will be prompted for
the theory name---type {\tt sum}, or just {\tt Return} if {\tt sum} is
the default.  You will then be prompted for the \TeX\ previewer name.
Either the previewer must be in your path, or the entire pathname must
be given.  This information will only be prompted for once per session,
after that \pvs\ assumes that you want to use the same previewer.

\begin{figure}[ht]
\begin{center}
\begin{boxedminipage}{\textwidth}
{\smaller\smaller% The following substitutions are from the file:
%   /home/owre/pvs.git/pvs-tex.sub
\def\munderscoretimestwofn#1#2{{#1 \times #2}}% How to print function m_times with arity (2)
\def\fmunderscoretimestwofn#1#2{{#1 \times #2}}% How to print function fm_times with arity (2)
\def\sigmaunderscoretimestwofn#1#2{{#1 \times #2}}% How to print function sigma_times with arity (2)
\def\generatedunderscoresubsetunderscorealgebraonefn#1{{{\cal A}(#1)}}% How to print function generated_subset_algebra with arity (1)
\def\generatedunderscoresigmaunderscorealgebraonefn#1{{{\cal S}(#1)}}% How to print function generated_sigma_algebra with arity (1)
\def\aeunderscoredecreasingotheronefn#1{{\pvsid{decreasing?}(#1)~\mbox{\it a.e.}}}% How to print function ae_decreasing? with arity (1)
\def\aeunderscoreincreasingotheronefn#1{{\pvsid{increasing?}(#1)~\mbox{\it a.e.}}}% How to print function ae_increasing? with arity (1)
\def\aeunderscoreconvergenceothertwofn#1#2{{#1 \longrightarrow #2~\mbox{\it a.e.}}}% How to print function ae_convergence? with arity (2)
\def\aeunderscoreeqothertwofn#1#2{{#1 = #2~\mbox{\it a.e.}}}% How to print function ae_eq? with arity (2)
\def\aeunderscoreleothertwofn#1#2{{#1 \leq #2~\mbox{\it a.e.}}}% How to print function ae_le? with arity (2)
\def\aeunderscoreposotheronefn#1{{#1> 0~\mbox{\it a.e.}}}% How to print function ae_pos? with arity (1)
\def\aeunderscorenonnegotheronefn#1{{#1 \geq 0~\mbox{\it a.e.}}}% How to print function ae_nonneg? with arity (1)
\def\aeunderscorezerootheronefn#1{{#1 = 0~\mbox{\it a.e.}}}% How to print function ae_0? with arity (1)
\def\xunderscorelttwofn#1#2{{#1 < #2}}% How to print function x_lt with arity (2)
\def\xunderscoreletwofn#1#2{{#1 \leq #2}}% How to print function x_le with arity (2)
\def\xunderscoreeqtwofn#1#2{{#1 = #2}}% How to print function x_eq with arity (2)
\def\xunderscoretimestwofn#1#2{{#1 \times #2}}% How to print function x_times with arity (2)
\def\xunderscoreaddtwofn#1#2{{#1 + #2}}% How to print function x_add with arity (2)
\def\xunderscorelimitonefn#1{{\pvsid{limit}(#1)}}% How to print function x_limit with arity (1)
\def\xunderscoresumonefn#1{{\sum #1}}% How to print function x_sum with arity (1)
\def\xunderscoresigmathreefn#1#2#3{{\sum_{#1}^{#2} #3}}% How to print function x_sigma with arity (3)
\def\xunderscoresuponefn#1{{\pvsid{sup}(#1)}}% How to print function x_sup with arity (1)
\def\xunderscoreinfonefn#1{{\pvsid{inf}(#1)}}% How to print function x_inf with arity (1)
\def\pointwiseunderscoreconvergesunderscoredowntoothertwofn#1#2{{#1 \searrow #2}}% How to print function pointwise_converges_downto? with arity (2)
\def\pointwiseunderscoreconvergesunderscoreuptoothertwofn#1#2{{#1 \nearrow #2}}% How to print function pointwise_converges_upto? with arity (2)
\def\pointwiseunderscoreconvergenceothertwofn#1#2{{#1 \longrightarrow #2}}% How to print function pointwise_convergence? with arity (2)
\def\convergesunderscoredowntoothertwofn#1#2{{#1 \searrow #2}}% How to print function converges_downto? with arity (2)
\def\convergesunderscoreuptoothertwofn#1#2{{#1 \nearrow #2}}% How to print function converges_upto? with arity (2)
\def\convergenceothertwofn#1#2{{#1 \longrightarrow #2}}% How to print function convergence? with arity (2)
\def\convergencetwofn#1#2{{#1 \longrightarrow #2}}% How to print function convergence with arity (2)
\def\crossunderscoreproducttwofn#1#2{{#1 \times #2}}% How to print function cross_product with arity (2)
\def\conjugateonefn#1{{\overline{#1}}}% How to print function conjugate with arity (1)
\def\cunderscoredivtwofn#1#2{{#1/#2}}% How to print function c_div with arity (2)
\def\cunderscoremultwofn#1#2{{#1\times#2}}% How to print function c_mul with arity (2)
\def\cunderscoresubtwofn#1#2{{#1-#2}}% How to print function c_sub with arity (2)
\def\cunderscorenegonefn#1{{-#1}}% How to print function c_neg with arity (1)
\def\cunderscoreaddtwofn#1#2{{#1+#2}}% How to print function c_add with arity (2)
\def\Imonefn#1{{\Im(#1)}}% How to print function Im with arity (1)
\def\Reonefn#1{{\Re(#1)}}% How to print function Re with arity (1)
\def\Etwofn#1#2{{\mathbb{E}(#1~|~#2)}}% How to print function E with arity (2)
\def\Eonefn#1{{\mathbb{E}(#1)}}% How to print function E with arity (1)
\def\Ptwofn#1#2{{\mathbb{P}(#1~|~#2)}}% How to print function P with arity (2)
\def\Ponefn#1{{\mathbb{P}(#1)}}% How to print function P with arity (1)
\def\xtwofn#1#2{{#1\times#2}}% How to print function x with arity (2)
\def\asttwofn#1#2{{#1\ast#2}}% How to print function ast with arity (2)
\def\minusonefn#1{{{#1}^{-}}}% How to print function minus with arity (1)
\def\plusonefn#1{{{#1}^{+}}}% How to print function plus with arity (1)
\def\astonefn#1{{{#1}^{\ast}}}% How to print function ast with arity (1)
\def\dottwofn#1#2{{#1\bullet#2}}% How to print function dot with arity (2)
\def\integralthreefn#1#2#3{{\int_{#1}^{#2} #3}}% How to print function integral with arity (3)
\def\integraltwofn#1#2{{\int_{#1} #2}}% How to print function integral with arity (2)
\def\integralonefn#1{{\int#1}}% How to print function integral with arity (1)
\def\normonefn#1{{\left||{#1}\right||}}% How to print function norm with arity (1)
\def\phionefn#1{{\pvssubscript{\phi}{#1}}}% How to print function phi with arity (1)
\def\infunderscoreclosedonefn#1{{\left(-\infty,~#1\right]}}% How to print function inf_closed with arity (1)
\def\closedunderscoreinfonefn#1{{\left[#1,~\infty\right)}}% How to print function closed_inf with arity (1)
\def\infunderscoreopenonefn#1{(-\infty,~#1)}% How to print function inf_open with arity (1)
\def\openunderscoreinfonefn#1{(#1,~\infty)}% How to print function open_inf with arity (1)
\def\closedtwofn#1#2{{\left[#1,~#2\right]}}% How to print function closed with arity (2)
\def\opentwofn#1#2{(#1,~#2)}% How to print function open with arity (2)
\def\sigmathreefn#1#2#3{{\sum_{#1}^{#2} #3}}% How to print function sigma with arity (3)
\def\sigmatwofn#1#2{{\sum_{#1} {#2}}}% How to print function sigma with arity (2)
\def\ceilingonefn#1{{\lceil{#1}\rceil}}% How to print function ceiling with arity (1)
\def\flooronefn#1{{\lfloor{#1}\rfloor}}% How to print function floor with arity (1)
\def\absonefn#1{{\left|{#1}\right|}}% How to print function abs with arity (1)
\def\roottwofn#1#2{{\sqrt[#2]{#1}}}% How to print function root with arity (2)
\def\sqrtonefn#1{{\sqrt{#1}}}% How to print function sqrt with arity (1)
\def\sqonefn#1{{\pvssuperscript{#1}{2}}}% How to print function sq with arity (1)
\def\expttwofn#1#2{{\pvssuperscript{#1}{#2}}}% How to print function expt with arity (2)
\def\opcarettwofn#1#2{{\pvssuperscript{#1}{#2}}}% How to print function ^ with arity (2)
\def\indexedunderscoresetsotherIIntersectiononefn#1{{\bigcap #1}}% How to print function indexed_sets.IIntersection with arity (1)
\def\indexedunderscoresetsotherIUniononefn#1{{\bigcup #1}}% How to print function indexed_sets.IUnion with arity (1)
\def\setsotherIntersectiononefn#1{{\bigcap #1}}% How to print function sets.Intersection with arity (1)
\def\setsotherUniononefn#1{{\bigcup #1}}% How to print function sets.Union with arity (1)
\def\setsotherremovetwofn#1#2{{(#2 \setminus \{#1\})}}% How to print function sets.remove with arity (2)
\def\setsotheraddtwofn#1#2{{(#2 \cup \{#1\})}}% How to print function sets.add with arity (2)
\def\setsotherdifferencetwofn#1#2{{(#1 \setminus #2)}}% How to print function sets.difference with arity (2)
\def\setsothercomplementonefn#1{{\overline{#1}}}% How to print function sets.complement with arity (1)
\def\setsotherintersectiontwofn#1#2{{(#1 \cap #2)}}% How to print function sets.intersection with arity (2)
\def\setsotheruniontwofn#1#2{{(#1 \cup #2)}}% How to print function sets.union with arity (2)
\def\setsotherstrictunderscoresubsetothertwofn#1#2{{(#1 \subset #2)}}% How to print function sets.strict_subset? with arity (2)
\def\setsothersubsetothertwofn#1#2{{(#1 \subseteq #2)}}% How to print function sets.subset? with arity (2)
\def\setsothermembertwofn#1#2{{(#1 \in #2)}}% How to print function sets.member with arity (2)
\def\opohtwofn#1#2{{#1\circ#2}}% How to print function O with arity (2)
\def\opdividetwofn#1#2{{\frac{#1}{#2}}}% How to print function / with arity (2)
\def\optimestwofn#1#2{{#1\times#2}}% How to print function * with arity (2)
\def\opdifferenceonefn#1{{-#1}}% How to print function - with arity (1)
\def\opdifferencetwofn#1#2{{#1-#2}}% How to print function - with arity (2)
\def\opplustwofn#1#2{{#1+#2}}% How to print function + with arity (2)
\begin{alltt}
\pvsid{sum}: \pvskey{THEORY}
 \pvskey{BEGIN}

  \(n\): \pvskey{VAR} \(\mathbb{N}\)\vspace*{\pvsdeclspacing}

  \pvsid{sum}\pvsid{(}\(n\)\pvsid{)}: \pvskey{RECURSIVE} \(\mathbb{N}\) \pvskey{=} \pvsid{(}\pvskey{IF} \(n\) \(=\) \(0\) \pvskey{THEN} \(0\) \pvskey{ELSE} \(\opplustwofn{n}{\pvsid{sum}\pvsid{(}\opdifferencetwofn{n}{1}\pvsid{)}}\) \pvskey{ENDIF}\pvsid{)}
     \pvskey{MEASURE} \pvsid{(}\(\lambda\) \(n\): \(n\)\pvsid{)}\vspace*{\pvsdeclspacing}

  \pvsid{closed\_form}: \pvskey{THEOREM} \pvsid{sum}\pvsid{(}\(n\)\pvsid{)} \(=\) \(\opdividetwofn{\pvsid{(}\optimestwofn{n}{\pvsid{(}\opplustwofn{n}{1}\pvsid{)}}\pvsid{)}}{2}\)\vspace*{\pvsdeclspacing}

 \pvskey{END} \pvsid{sum}\end{alltt}
}
\end{boxedminipage}
\end{center}
\caption{Theory {\tt sum}}\label{sum-plain}
\end{figure}

After a few moments the previewer will pop up displaying the {\tt sum}
theory, as shown in Figure~\ref{sum-plain}.  Note that {\tt LAMBDA}
has been translated as $\lambda$.  This and other translations are
built into \pvs; the user may also specify translations for keywords
and identifiers (and override those built-in) by providing a
substitution file, {\tt pvs-tex.sub}, which contains commands to
customize the \LaTeX\ output.  For example, if the substitution file
contains the three lines

{\smaller\smaller\begin{alltt}
    THEORY key 7 \verb|{\large\bf Theory}|
    sum    1   2 \verb|{\sum_{i = 0}^{#1} i}|
\end{alltt}}
the output will look like Figure~\ref{sum-sub}.

\begin{figure}[ht]
\begin{center}
\begin{boxedminipage}{\textwidth}
{\smaller\smaller% The following substitutions are from the file:
%   /home/owre/pvs.git/pvs-tex.sub
\def\munderscoretimestwofn#1#2{{#1 \times #2}}% How to print function m_times with arity (2)
\def\fmunderscoretimestwofn#1#2{{#1 \times #2}}% How to print function fm_times with arity (2)
\def\sigmaunderscoretimestwofn#1#2{{#1 \times #2}}% How to print function sigma_times with arity (2)
\def\generatedunderscoresubsetunderscorealgebraonefn#1{{{\cal A}(#1)}}% How to print function generated_subset_algebra with arity (1)
\def\generatedunderscoresigmaunderscorealgebraonefn#1{{{\cal S}(#1)}}% How to print function generated_sigma_algebra with arity (1)
\def\aeunderscoredecreasingotheronefn#1{{\pvsid{decreasing?}(#1)~\mbox{\it a.e.}}}% How to print function ae_decreasing? with arity (1)
\def\aeunderscoreincreasingotheronefn#1{{\pvsid{increasing?}(#1)~\mbox{\it a.e.}}}% How to print function ae_increasing? with arity (1)
\def\aeunderscoreconvergenceothertwofn#1#2{{#1 \longrightarrow #2~\mbox{\it a.e.}}}% How to print function ae_convergence? with arity (2)
\def\aeunderscoreeqothertwofn#1#2{{#1 = #2~\mbox{\it a.e.}}}% How to print function ae_eq? with arity (2)
\def\aeunderscoreleothertwofn#1#2{{#1 \leq #2~\mbox{\it a.e.}}}% How to print function ae_le? with arity (2)
\def\aeunderscoreposotheronefn#1{{#1> 0~\mbox{\it a.e.}}}% How to print function ae_pos? with arity (1)
\def\aeunderscorenonnegotheronefn#1{{#1 \geq 0~\mbox{\it a.e.}}}% How to print function ae_nonneg? with arity (1)
\def\aeunderscorezerootheronefn#1{{#1 = 0~\mbox{\it a.e.}}}% How to print function ae_0? with arity (1)
\def\xunderscorelttwofn#1#2{{#1 < #2}}% How to print function x_lt with arity (2)
\def\xunderscoreletwofn#1#2{{#1 \leq #2}}% How to print function x_le with arity (2)
\def\xunderscoreeqtwofn#1#2{{#1 = #2}}% How to print function x_eq with arity (2)
\def\xunderscoretimestwofn#1#2{{#1 \times #2}}% How to print function x_times with arity (2)
\def\xunderscoreaddtwofn#1#2{{#1 + #2}}% How to print function x_add with arity (2)
\def\xunderscorelimitonefn#1{{\pvsid{limit}(#1)}}% How to print function x_limit with arity (1)
\def\xunderscoresumonefn#1{{\sum #1}}% How to print function x_sum with arity (1)
\def\xunderscoresigmathreefn#1#2#3{{\sum_{#1}^{#2} #3}}% How to print function x_sigma with arity (3)
\def\xunderscoresuponefn#1{{\pvsid{sup}(#1)}}% How to print function x_sup with arity (1)
\def\xunderscoreinfonefn#1{{\pvsid{inf}(#1)}}% How to print function x_inf with arity (1)
\def\pointwiseunderscoreconvergesunderscoredowntoothertwofn#1#2{{#1 \searrow #2}}% How to print function pointwise_converges_downto? with arity (2)
\def\pointwiseunderscoreconvergesunderscoreuptoothertwofn#1#2{{#1 \nearrow #2}}% How to print function pointwise_converges_upto? with arity (2)
\def\pointwiseunderscoreconvergenceothertwofn#1#2{{#1 \longrightarrow #2}}% How to print function pointwise_convergence? with arity (2)
\def\convergesunderscoredowntoothertwofn#1#2{{#1 \searrow #2}}% How to print function converges_downto? with arity (2)
\def\convergesunderscoreuptoothertwofn#1#2{{#1 \nearrow #2}}% How to print function converges_upto? with arity (2)
\def\convergenceothertwofn#1#2{{#1 \longrightarrow #2}}% How to print function convergence? with arity (2)
\def\convergencetwofn#1#2{{#1 \longrightarrow #2}}% How to print function convergence with arity (2)
\def\crossunderscoreproducttwofn#1#2{{#1 \times #2}}% How to print function cross_product with arity (2)
\def\conjugateonefn#1{{\overline{#1}}}% How to print function conjugate with arity (1)
\def\cunderscoredivtwofn#1#2{{#1/#2}}% How to print function c_div with arity (2)
\def\cunderscoremultwofn#1#2{{#1\times#2}}% How to print function c_mul with arity (2)
\def\cunderscoresubtwofn#1#2{{#1-#2}}% How to print function c_sub with arity (2)
\def\cunderscorenegonefn#1{{-#1}}% How to print function c_neg with arity (1)
\def\cunderscoreaddtwofn#1#2{{#1+#2}}% How to print function c_add with arity (2)
\def\Imonefn#1{{\Im(#1)}}% How to print function Im with arity (1)
\def\Reonefn#1{{\Re(#1)}}% How to print function Re with arity (1)
\def\Etwofn#1#2{{\mathbb{E}(#1~|~#2)}}% How to print function E with arity (2)
\def\Eonefn#1{{\mathbb{E}(#1)}}% How to print function E with arity (1)
\def\Ptwofn#1#2{{\mathbb{P}(#1~|~#2)}}% How to print function P with arity (2)
\def\Ponefn#1{{\mathbb{P}(#1)}}% How to print function P with arity (1)
\def\xtwofn#1#2{{#1\times#2}}% How to print function x with arity (2)
\def\asttwofn#1#2{{#1\ast#2}}% How to print function ast with arity (2)
\def\minusonefn#1{{{#1}^{-}}}% How to print function minus with arity (1)
\def\plusonefn#1{{{#1}^{+}}}% How to print function plus with arity (1)
\def\astonefn#1{{{#1}^{\ast}}}% How to print function ast with arity (1)
\def\dottwofn#1#2{{#1\bullet#2}}% How to print function dot with arity (2)
\def\integralthreefn#1#2#3{{\int_{#1}^{#2} #3}}% How to print function integral with arity (3)
\def\integraltwofn#1#2{{\int_{#1} #2}}% How to print function integral with arity (2)
\def\integralonefn#1{{\int#1}}% How to print function integral with arity (1)
\def\normonefn#1{{\left||{#1}\right||}}% How to print function norm with arity (1)
\def\phionefn#1{{\pvssubscript{\phi}{#1}}}% How to print function phi with arity (1)
\def\infunderscoreclosedonefn#1{{\left(-\infty,~#1\right]}}% How to print function inf_closed with arity (1)
\def\closedunderscoreinfonefn#1{{\left[#1,~\infty\right)}}% How to print function closed_inf with arity (1)
\def\infunderscoreopenonefn#1{(-\infty,~#1)}% How to print function inf_open with arity (1)
\def\openunderscoreinfonefn#1{(#1,~\infty)}% How to print function open_inf with arity (1)
\def\closedtwofn#1#2{{\left[#1,~#2\right]}}% How to print function closed with arity (2)
\def\opentwofn#1#2{(#1,~#2)}% How to print function open with arity (2)
\def\sigmathreefn#1#2#3{{\sum_{#1}^{#2} #3}}% How to print function sigma with arity (3)
\def\sigmatwofn#1#2{{\sum_{#1} {#2}}}% How to print function sigma with arity (2)
\def\ceilingonefn#1{{\lceil{#1}\rceil}}% How to print function ceiling with arity (1)
\def\flooronefn#1{{\lfloor{#1}\rfloor}}% How to print function floor with arity (1)
\def\absonefn#1{{\left|{#1}\right|}}% How to print function abs with arity (1)
\def\roottwofn#1#2{{\sqrt[#2]{#1}}}% How to print function root with arity (2)
\def\sqrtonefn#1{{\sqrt{#1}}}% How to print function sqrt with arity (1)
\def\sqonefn#1{{\pvssuperscript{#1}{2}}}% How to print function sq with arity (1)
\def\expttwofn#1#2{{\pvssuperscript{#1}{#2}}}% How to print function expt with arity (2)
\def\opcarettwofn#1#2{{\pvssuperscript{#1}{#2}}}% How to print function ^ with arity (2)
\def\indexedunderscoresetsotherIIntersectiononefn#1{{\bigcap #1}}% How to print function indexed_sets.IIntersection with arity (1)
\def\indexedunderscoresetsotherIUniononefn#1{{\bigcup #1}}% How to print function indexed_sets.IUnion with arity (1)
\def\setsotherIntersectiononefn#1{{\bigcap #1}}% How to print function sets.Intersection with arity (1)
\def\setsotherUniononefn#1{{\bigcup #1}}% How to print function sets.Union with arity (1)
\def\setsotherremovetwofn#1#2{{(#2 \setminus \{#1\})}}% How to print function sets.remove with arity (2)
\def\setsotheraddtwofn#1#2{{(#2 \cup \{#1\})}}% How to print function sets.add with arity (2)
\def\setsotherdifferencetwofn#1#2{{(#1 \setminus #2)}}% How to print function sets.difference with arity (2)
\def\setsothercomplementonefn#1{{\overline{#1}}}% How to print function sets.complement with arity (1)
\def\setsotherintersectiontwofn#1#2{{(#1 \cap #2)}}% How to print function sets.intersection with arity (2)
\def\setsotheruniontwofn#1#2{{(#1 \cup #2)}}% How to print function sets.union with arity (2)
\def\setsotherstrictunderscoresubsetothertwofn#1#2{{(#1 \subset #2)}}% How to print function sets.strict_subset? with arity (2)
\def\setsothersubsetothertwofn#1#2{{(#1 \subseteq #2)}}% How to print function sets.subset? with arity (2)
\def\setsothermembertwofn#1#2{{(#1 \in #2)}}% How to print function sets.member with arity (2)
\def\opohtwofn#1#2{{#1\circ#2}}% How to print function O with arity (2)
\def\opdividetwofn#1#2{{\frac{#1}{#2}}}% How to print function / with arity (2)
\def\optimestwofn#1#2{{#1\times#2}}% How to print function * with arity (2)
\def\opdifferenceonefn#1{{-#1}}% How to print function - with arity (1)
\def\opdifferencetwofn#1#2{{#1-#2}}% How to print function - with arity (2)
\def\opplustwofn#1#2{{#1+#2}}% How to print function + with arity (2)
% The following substitutions are from the file:
%   /home/owre/pvs.git/doc/wift95/pvs-tex.sub
\def\sumonefn#1{{\sum_{i = 0}^{#1} i}}% How to print function sum with arity (1)
\begin{alltt}
\pvsid{sum}: \({\large\bf Theory}\)
 \pvskey{BEGIN}

  \(n\): \pvskey{VAR} \(\mathbb{N}\)\vspace*{\pvsdeclspacing}

  \(\sumonefn{n}\): \pvskey{RECURSIVE} \(\mathbb{N}\) \pvskey{=} \pvsid{(}\pvskey{IF} \(n\) \(=\) \(0\) \pvskey{THEN} \(0\) \pvskey{ELSE} \(\opplustwofn{n}{\sumonefn{\opdifferencetwofn{n}{1}}}\) \pvskey{ENDIF}\pvsid{)}
     \pvskey{MEASURE} \pvsid{(}\(\lambda\) \(n\): \(n\)\pvsid{)}\vspace*{\pvsdeclspacing}

  \pvsid{closed\_form}: \pvskey{THEOREM} \(\sumonefn{n}\) \(=\) \(\opdividetwofn{\pvsid{(}\optimestwofn{n}{\pvsid{(}\opplustwofn{n}{1}\pvsid{)}}\pvsid{)}}{2}\)\vspace*{\pvsdeclspacing}

 \pvskey{END} \pvsid{sum}\end{alltt}
}
\end{boxedminipage}
\end{center}
\caption{Theory {\tt sum}}\label{sum-sub}
\end{figure}

Finally, using the \iecmd{latex-proof} command, it is possible to
generate a \LaTeX\ file from a proof.  A part of an example is shown
below; details are in the PVS system manual.

\noindent
\begin{boxedminipage}{\linewidth}
\def\sumonefn#1{{\sum_{i = 0}^{#1} i}}

Expanding the definition of sum

{\tt closed\_form.2:}

\vspace*{0.2in}\hspace*{0.2in}
\begin{tabular}{ll}
$\{-1\}$ &\begin{minipage}[t]{6in}{ \begin{program} 
 \sumonefn { j^{\prime} } \mbox{ }=\mbox{ }(\ii j^{\prime} \mbox{ }\times\mbox{ }(\ii j^{\prime} \mbox{ }+\mbox{ } 1)) \mbox{ }/\mbox{ } 2 \\ 
\oo\oo\zi\zi\zi\zi\zo\zo \end{program}}\end{minipage}\\\hline
$\{1\}$ &\begin{minipage}[t]{6in}{ \begin{program} 
(\ii \pvskey{IF\mbox{ }} j^{\prime} \mbox{ }+\mbox{ } 1 \mbox{ }=\mbox{ } 0 \pvskey{\mbox{ }THEN\mbox{ }} 0 \pvskey{\mbox{ }ELSE\mbox{ }} j^{\prime} \mbox{ }+\mbox{ } 1 \mbox{ }+\mbox{ } \sumonefn { j^{\prime} \mbox{ }+\mbox{ } 1 \mbox{ }-\mbox{ } 1 } \pvskey{\mbox{ }ENDIF}) \\
\oo\zi \mbox{ }=\mbox{ }(\ii(\ii j^{\prime} \mbox{ }+\mbox{ } 1) \mbox{ }\times\mbox{ }(\ii j^{\prime} \mbox{ }+\mbox{ } 1 \mbox{ }+\mbox{ } 1)) \mbox{ }/\mbox{ } 2 \\ 
\oo\oo\oo\zi\zi\zi\zo\zo \end{program}}\end{minipage}\\
\end{tabular}
\end{boxedminipage}

% Document Type: LaTeX
% Master File: language.tex
\documentclass[12pt]{book}
\usepackage{alltt}
\usepackage{makeidx}
\usepackage{relsize}
\usepackage{boxedminipage}
\usepackage{url}
\usepackage{../../pvs}
\usepackage{../makebnf}
\usepackage[chapter]{tocbibind}
\usepackage{fancyvrb}
\usepackage[dvipsnames,usenames]{color}

\usepackage{amssymb}
\usepackage{mathpazo}
\usepackage{fontspec}
\setmainfont[Ligatures=TeX]{XITS}
\setmonofont{DejaVu Sans Mono}[Scale=MatchLowercase]
%\setmonofont{Free Mono}[Scale=0.8]
\usepackage[math-style=ISO]{unicode-math}
\renewcommand{\leadsto}{\rightsquigarrow}
%\setmathfont{XITS Math}

\topmargin -10pt
\textheight 8.5in
\textwidth 6.0in
\headheight 15 pt
\columnwidth \textwidth
\oddsidemargin 0.5in
\evensidemargin 0.5in   % fool system for page 0
\setcounter{topnumber}{9}
\renewcommand{\topfraction}{.99}
\setcounter{bottomnumber}{9}
\renewcommand{\bottomfraction}{.99}
\setcounter{totalnumber}{10}
\renewcommand{\textfraction}{.5}
\renewcommand{\floatpagefraction}{.1}
\usepackage{fancyhdr}
\pagestyle{fancy}
\raggedbottom

%\setcounter{secnumdepth}{1}

\index{type correctness condition|see{TCC}}
\makeindex

\usepackage[bookmarks=true,hyperindex=true,colorlinks=true,linkcolor=Brown,citecolor=blue,backref=page,pagebackref=true,plainpages=false,pdfpagelabels]{hyperref}

%% Derived from John Rushby's prelude.tex, modified for NFSS2
%
% define variants of the \LaTeX macro that avoid using \sc
% for use in headings
%

% Define fonts that work in math or text mode
\def\dwimrm#1{\ifmmode\mathrm{#1}\else\textrm{#1}\fi}
\def\dwimsf#1{\ifmmode\mathsf{#1}\else\textsf{#1}\fi}
\def\dwimtt#1{\ifmmode\mathtt{#1}\else\texttt{#1}\fi}
\def\dwimbf#1{\ifmmode\mathbf{#1}\else\textbf{#1}\fi}
\def\dwimit#1{\ifmmode\mathit{#1}\else\textit{#1}\fi}
\def\dwimnormal#1{\ifmmode\mathnormal{#1}\else\textnormal{#1}\fi}

\def\BigLaTeX{{\rm L\kern-.36em\raise.3ex\hbox{\small\small A}\kern-.15em
    T\kern-.1667em\lower.7ex\hbox{E}\kern-.125emX}}
\def\BoldLaTeX{{\bf L\kern-.36em\raise.3ex\hbox{\small\small\bf A}\kern-.15em
    T\kern-.1667em\lower.7ex\hbox{E}\kern-.125emX}}
%\def\labelitemi{$\bullet$}
\def\labelitemii{$\circ$}
\def\labelitemiii{$\star$}
\def\labelitemiv{$\diamond$}
\newcommand{\tcc}{{\small\small TCC}}
\newcommand{\tccs}{\tcc s}
\newcommand{\emacs}{{Emacs}}
\newcommand{\Emacs}{\emacs}
\newcommand{\ehdm}{{E{\small\small HDM}}}
\newcommand{\Ehdm}{\ehdm}
\newcommand{\tm}{$^{\mbox{\tiny TM}}$}
\newcommand{\hozline}{{\noindent\rule{\textwidth}{0.4mm}}}

\newcommand{\allclear}%
  {\mbox{\boldmath$\stackrel{\raisebox{-.2ex}[0pt][0pt]%
              {$\textstyle\oslash$}}{\displaystyle\bot}$}}

\newenvironment{private}{}{}

\newenvironment{smalltt}{\begin{alltt}\small}{\end{alltt}}

\newlength{\hsbw}

\newenvironment{session}%
  {\begin{flushleft}
   \setlength{\hsbw}{\linewidth}
   \addtolength{\hsbw}{-\arrayrulewidth}
   \addtolength{\hsbw}{-\tabcolsep}
   \begin{tabular}{@{}|c@{}|@{}}\hline 
   \begin{minipage}[b]{\hsbw}
   \begingroup\small\mbox{ }\\[-1.8\baselineskip]\begin{alltt}}
  {\end{alltt}\endgroup\end{minipage}\\ \hline 
   \end{tabular}
   \end{flushleft}}

\newenvironment{smallsession}%
  {\begin{flushleft}
   \setlength{\hsbw}{\linewidth}
   \addtolength{\hsbw}{-\arrayrulewidth}
   \addtolength{\hsbw}{-\tabcolsep}
   \begin{tabular}{@{}|c@{}|@{}}\hline 
   \begin{minipage}[b]{\hsbw}
   \begingroup\footnotesize\mbox{ }\\[-1.8\baselineskip]\begin{alltt}}%
  {\end{alltt}\endgroup\end{minipage}\\ \hline 
   \end{tabular}
   \end{flushleft}}

\newenvironment{spec}%
  {\begin{flushleft}
   \setlength{\hsbw}{\textwidth}
   \addtolength{\hsbw}{-\arrayrulewidth}
   \addtolength{\hsbw}{-\tabcolsep}
   \begin{tabular}{@{}|c@{}|@{}}\hline 
   \begin{minipage}[b]{\hsbw}
   \begingroup\small\mbox{ }\\[-0.2\baselineskip]}%
  {\endgroup\end{minipage}\\ \hline 
   \end{tabular}
   \end{flushleft}}

\newcommand{\memo}[1]%
  {\mbox{}\par\vspace{0.25in}%
   \setlength{\hsbw}{\linewidth}\addtolength{\hsbw}{-1.5ex}%
   \noindent\fbox{\parbox{\hsbw}{{\bf Memo: }#1}}\vspace{0.25in}}

\newcommand{\nb}[1]%
  {\mbox{}\par\vspace{0.25in}%
   \setlength{\hsbw}{\linewidth}\addtolength{\hsbw}{-1.5ex}%
   \noindent\fbox{\parbox{\hsbw}{{\bf Note: }#1}}\vspace{0.25in}}

\newcommand{\comment}[1]{}
\newcommand{\exfootnote}[1]{}
%\newcommand{\ifelse}[2]{#1}
\sloppy
\clubpenalty=100000
\widowpenalty=100000
%\displaywidowpenalty=100000
\setcounter{secnumdepth}{3} 
\setcounter{tocdepth}{3}
\setcounter{topnumber}{9}
\setcounter{bottomnumber}{9}
\setcounter{totalnumber}{9}
\renewcommand{\topfraction}{.99}
\renewcommand{\bottomfraction}{.99}
\renewcommand{\floatpagefraction}{.01}
\renewcommand{\textfraction}{.2}
\font\largett=cmtt10 scaled\magstep1
\font\Largett=cmtt10 scaled\magstep2
\font\hugett=cmtt10 scaled\magstep3

\def\labelitemii{$\circ$}
\def\labelitemiii{$\star$}
\def\labelitemiv{$\diamond$}
\newcommand{\tcc}{{\small\small TCC}}
\newcommand{\tccs}{\tcc s}

%\renewcommand{\memo}[1]{\mbox{}\par\vspace{0.25in}\noindent\fbox{\parbox{.95\linewidth}{{\bf Memo: }#1}}\vspace{0.25in}}

\newcommand{\eg}{{\em e.g.\/},}
\newcommand{\ie}{{\em i.e.\/},}

\newcommand{\pvs}{PVS}

\newcommand{\ch}{\choice}
\newcommand{\rsv}[1]{{\rm\tt #1}}

\newcommand{\lpvstheory}[3]{\figurehead{\hozline\smaller\smaller\begin{alltt}}%
                           \figuretail{\end{alltt}\vspace{-0in}\hozline}%
                           \figurelabel{#3}\figurecap{#2}%
                           \begin{longfigure}\input{#1}\end{longfigure}}

\newcommand{\bpvstheory}[3]
{\begin{figure}[b]\begin{boxedminipage}{\textwidth}%
      {\smaller\smaller\begin{alltt} \input{#1}\end{alltt}}\end{boxedminipage}%
    \caption{#2}\label{#3}\end{figure}}

\newcommand{\spvstheory}[1]
{\vspace{0.1in}\par\noindent\begin{boxedminipage}{\textwidth}%
    {\smaller\smaller\begin{alltt} \input{#1}\end{alltt}}\end{boxedminipage}\vspace{0.1in}%
}

\CustomVerbatimEnvironment{pvsex}{Verbatim}{commandchars=\\\{\},frame=single,fontsize=\relsize{-1}}

\CustomVerbatimCommand{\pvsinput}{VerbatimInput}{commandchars=\\\{\},frame=single,fontsize=\relsize{-1}}

\newcommand{\pvstheory}[3]
  {\begin{figure}[htb]%
      \pvsinput{#1}%
      \caption{#2}\label{#3}%
    \end{figure}}

% \newenvironment{pvsex}%
%   {\setlength{\topsep}{0in}\smaller\begin{alltt}}%
%   {\end{alltt}}

\newcommand{\pvsbnf}[2]
  {\begin{figure}[htb]\begin{boxedminipage}{\textwidth}%
   \input{#1}\end{boxedminipage}\caption{#2}\label{#1}\end{figure}}

\newcommand{\spvsbnf}[1]
  {\begin{boxedminipage}{\textwidth}\input{#1}\end{boxedminipage}}

\newcommand{\pidx}[1]{{\rm #1}} % primary index entry
\newcommand{\sidx}[1]{{\rm #1}} % secondary index entry
\newcommand{\cmdindex}[1]{\index{#1@\cmd{#1}}}
\newcommand{\icmd}[1]{\cmd{#1}\cmdindex{#1}}
\newcommand{\iecmd}[1]{\ecmd{#1}\cmdindex{#1}}
\newcommand{\buf}[1]{\texttt{#1}}
\newcommand{\ibuf}[1]{\buf{#1}\index{#1 buffer@\buf{#1} buffer}\index{buffers!\buf{#1}}}

\newenvironment{pvscmds}%
  {\par\noindent\smaller%
   \begin{tabular*}{\textwidth}{|l@{\extracolsep{\fill}}l@{\extracolsep{\fill}}l|}\hline%
     {\it Command} & {\it Aliases} & {\it Function}\\ \hline}%
  {\hline\end{tabular*}\vspace{0.1in}}

\newenvironment{pvscmdsna}%
  {\par\noindent\smaller%
   \begin{tabular*}{\textwidth}{|l@{\extracolsep{\fill}}l|}\hline%
     {\it Command} & {\it \,\,Function}\\ \hline}%
  {\hline\end{tabular*}\vspace{0.1in}}

\newcommand{\cmd}[1]{\texttt{#1}}
\newcommand{\ecmd}[1]{{\tt M-x #1}}

\newcommand{\latex}{\LaTeX}                  %  LaTeX
\newcommand{\sun}{{S{\smaller\smaller UN}}}                 %  Sun
\newcommand{\sparc}{{S{\smaller\smaller PARC}}}             %  Sparc
\newcommand{\sunos}{{S{\smaller\smaller UN}OS}}             %  SunOS
\newcommand{\solaris}{{\em Solaris\/}}        %  Solaris
\newcommand{\sunview}{{S{\smaller\smaller UN}V{\smaller\smaller IEW}}} %SunView
\newcommand{\unix}{{U{\smaller\smaller NIX}}}               %  Unix
\newcommand{\lisp} {{\sc Lisp}}              %  Lisp
\newcommand{\gnu}{{Gnu Emacs}}           %  Gnu Emacs
\newcommand{\gnuemacs}{{Gnu Emacs}}      %  Gnu Emacs
\newcommand{\emacsl}{{Emacs-Lisp}}       %  Emacs Lisp
\newcommand{\shell}{{\sc Csh}}               %  C-shell

\newcommand{\update}[3]{#1\{#2\leftarrow #3\}}
\newcommand{\interp}[3]{\cal{M}\dlb {\tt #1 : #2 }\drb #3}
\newcommand{\myforall}[2]{(\forall{#1 .}\ #2)}
\newcommand{\myexists}[2]{(\exists{#1 .}\ #2)}
\newcommand{\mth}[1]{$ #1 $}
\newcommand{\labst}[2]{(\lambda{#1}.\ #2)}
\newcommand{\app}[2]{(#1\ #2)}
\newcommand{\problem}[1]{{\bf Exercise: } {\em #1}}
\newcommand{\rectype}[1]{[\# 1 \#]}
\newcommand{\recttype}[1]{{\tt [\# 1 \#]}}
\newcommand{\dlb}{\lbrack\!\lbrack}
\newcommand{\drb}{\rbrack\!\rbrack}
\newcommand{\cross}{\times}
\newcommand{\key}[1]{{\tt #1}}
\newcommand{\keyindex}[1]{\index{#1@\key{#1}}}
\newcommand{\ikey}[1]{\key{#1}\keyindex{#1}}
\newcommand{\keyword}[1]{{\smaller\texttt{#1}}}

\newenvironment{keybindings}%
  {\begin{center}\begin{tabular}{|l|l|}\hline Key & Function\\ \hline}%
  {\hline\end{tabular}\end{center}}
\def\rmif{\mbox{\bf if\ }}
\def\rmiff{\mbox{\bf \ iff \ }}
\def\rmthen{\mbox{\bf \ then }}
\def\rmelse{\mbox{\bf \ else }}
\def\rmend{\mbox{\bf end}}
\def\rmendif{\mbox{\bf \ endif}}
\def\rmotherwise{\mbox{\bf otherwise}}
\def\rmwith{\mbox{\bf \ with\ }}
\def\mapb{\char"7B\char"7B}
\def\mape{\char"7D\char"7D}
\def\setb{\char"7B}
\def\sete{\char"7D}

% ---------------------------------------------------------------------
% Macros for little PVS sessions displayed in boxes.
%
% Usage: (1) \setcounter{sessioncount}{1} resets the session counter
%
%        (2) \begin{session*}\label{thissession}
%             .
%              < lines from PVS session >
%             .
%            \end{session*}
%
%            typesets the session in a numbered box in ALLTT mode.
%
%  session instead of session* produces unnumbered boxes
%
%  Author: John Rushby
% ---------------------------------------------------------------------
\newlength{\hsbw}
\newenvironment{session}{\begin{flushleft}
 \setlength{\hsbw}{\linewidth}
 \addtolength{\hsbw}{-\arrayrulewidth}
 \addtolength{\hsbw}{-\tabcolsep}
 \begin{tabular}{@{}|c@{}|@{}}\hline 
 \begin{minipage}[b]{\hsbw}
% \begingroup\small\mbox{ }\\[-1.8\baselineskip]\begin{alltt}}{\end{alltt}\endgroup\end{minipage}\\ \hline
 \begingroup\sessionsize\vspace*{1.2ex}\begin{alltt}}{\end{alltt}\endgroup\end{minipage}\\ \hline
 \end{tabular}
 \end{flushleft}}
\newcounter{sessioncount}
\setcounter{sessioncount}{0}
\newenvironment{session*}{\begin{flushleft}
 \refstepcounter{sessioncount}
 \setlength{\hsbw}{\linewidth}
 \addtolength{\hsbw}{-\arrayrulewidth}
 \addtolength{\hsbw}{-\tabcolsep}
 \begin{tabular}{@{}|c@{}|@{}}\hline 
 \begin{minipage}[b]{\hsbw}
 \vspace*{-.5pt}
 \begin{flushright}
 \rule{0.01in}{.15in}\rule{0.3in}{0.01in}\hspace{-0.35in}
 \raisebox{0.04in}{\makebox[0.3in][c]{\footnotesize \thesessioncount}}
 \end{flushright}
 \vspace*{-.57in}
 \begingroup\small\vspace*{1.0ex}\begin{alltt}}{\end{alltt}\endgroup\end{minipage}\\ \hline 
 \end{tabular}
 \end{flushleft}}
\def\sessionsize{\footnotesize}
\def\smallsessionsize{\small}
\newenvironment{smallsession}{\begin{flushleft}
 \setlength{\hsbw}{\linewidth}
 \addtolength{\hsbw}{-\arrayrulewidth}
 \addtolength{\hsbw}{-\tabcolsep}
 \begin{tabular}{@{}|c@{}|@{}}\hline 
 \begin{minipage}[b]{\hsbw}
 \begingroup\smallsessionsize\mbox{ }\\[-1.8\baselineskip]\begin{alltt}}{\end{alltt}\endgroup\end{minipage}\\ \hline 
 \end{tabular}
 \end{flushleft}}
\newenvironment{spec}{\begin{flushleft}
 \setlength{\hsbw}{\textwidth}
 \addtolength{\hsbw}{-\arrayrulewidth}
 \addtolength{\hsbw}{-\tabcolsep}
 \begin{tabular}{@{}|c@{}|@{}}\hline 
 \begin{minipage}[b]{\hsbw}
 \begingroup\small\mbox{
}\\[-0.2\baselineskip]}{\endgroup\end{minipage}\\ \hline 
 \end{tabular}
 \end{flushleft}}
\newcommand{\memo}[1]{\mbox{}\par\vspace{0.25in}%
\setlength{\hsbw}{\linewidth}%
\addtolength{\hsbw}{-2\fboxsep}%
\addtolength{\hsbw}{-2\fboxrule}%
\noindent\fbox{\parbox{\hsbw}{{\bf Memo: }#1}}\vspace{0.25in}}
\newcommand{\nb}[1]{\mbox{}\par\vspace{0.25in}\setlength{\hsbw}{\linewidth}\addtolength{\hsbw}{-1.5ex}\noindent\fbox{\parbox{\hsbw}{{\bf Note: }#1}}\vspace{0.25in}}

%%% Local Variables: 
%%% mode: latex
%%% TeX-master: t
%%% End: 


\begin{document}

\begin{titlepage}
\renewcommand{\thepage}{title}
\vspace*{1in}
\noindent
\rule[1pt]{\textwidth}{2pt}
\begin{center}
\newfont{\pvstitle}{cmss17 scaled \magstep4}
\textbf{\pvstitle PVS Language Reference}
\end{center}
\begin{flushright}
{\Large Version 7.1 {\smaller$\bullet$} August 2020}
\end{flushright}
\rule[1in]{\textwidth}{2pt}
\vspace*{2in}
\begin{flushleft}
S.~Owre\\
N.~Shankar\\
J.~M.~Rushby\\
D.~W.~J.~Stringer-Calvert\\
{\smaller\url{{Owre,Shankar,Rushby,Dave_SC}@csl.sri.com}}\\
{\smaller\url{http://pvs.csl.sri.com/}}
\end{flushleft}
\vspace*{1in}
\vbox{\hbox to \textwidth{{\Large SRI International\hfill}}%
\hbox to \textwidth{{\small\sf%
Computer Science Laboratory $\bullet$ 333 Ravenswood Avenue $\bullet$ Menlo Park CA 94025\hfil}}}
\end{titlepage}

\renewcommand{\chaptermark}[1]{\markboth{{\em #1}}{}\markright{{\em #1}}}
\renewcommand{\sectionmark}[1]{\markright{\thesection \em \ #1}}
%\lhead[\thepage]{\rightmark}
%\cfoot{\protect\small\bf \fbox{PVS 2.3 DRAFT}}
%\cfoot{}
%\rhead[\leftmark]{\thepage}
\thispagestyle{empty}

\newpage
\renewcommand{\thepage}{ack}

\noindent\textbf{NOTE:} This manual is in the process of being updated.
Almost everything stated here is still correct, but incomplete due to the
many new features that have been introduced into PVS over the years.  The
release notes should be consulted for the most current information.

\vspace*{6in}\noindent
The initial development of PVS was funded by SRI International.
Subsequent enhancements were partially funded by SRI and by NASA
Contracts NAS1-18969 and NAS1-20334, NRL Contract N00014-96-C-2106,
NSF Grants CCR-9300044, CCR-9509931, and CCR-9712383, AFOSR contract
F49620-95-C0044, and DARPA Orders E276, A721, D431, D855, and E301.
\newpage
\pagenumbering{roman}
\setcounter{page}{1}

\tableofcontents
%\listoffigures

%\chapter{The PVS Specification Language}

%% Master File: language.tex
\addcontentsline{toc}{chapter}{\protect\numberline{}Preface}
\vspace{4in}
{\Huge\bf Preface}\linebreak
\vspace{.75in}

%\chapter{Preface}

This report presents a description of the \pvs\ specification language,
as implemented in Version 1.0 beta of the \pvs\ specification and
verification environment.  It is intended to provide a reference of all
of the features of the language, including the complete grammar, some
examples, and an informal semantics. This report is one of several
needed to effectively use \pvs.  Companion documents are devoted to the
use of the system~\cite{PVS:userguide}, the user of the
prover~\cite{PVS:prover}, a tutorial introduction~\cite{PVS:tutorial},
and a semantics~\cite{PVS:semantics}.

\memo{Give prerequisites to using \pvs.}

The \pvs\ system is the culmination of the effort of a large number of
people over many years, drawing heavily from the research and experience
gained from E{\sc
hdm}~\cite{EHDM:Userguide,EHDM:Language,EHDM:semantics,EHDM:supplement,EHDM:tutorial}.
The primary contributers to E{\sc hdm} in rough chronological order
were Michael Melliar-Smith, Richard Schwartz, Rob Shostak, Judith Crow,
Friedrich von Henke, Stan Jefferson, Rosanna Lee, John Rushby, Mark
Stickel, Natarajan Shankar, Sam Owre, David Cyrluk, Steven Phillips,
and Carl Witty.
%In addition to those named above, valuable contributions were made by
%Dorothy Denning, Brian Fromme, Allen van Gelder, Dwight Hare, Peter
%Ladkin, Sheralyn Listgarten, Jeff Miner, Paul Oppenheimer, Jeff
%Reninger, and Lorna Shinkle.
\pvs\ is primarily the work of John Rushby, Natarajan Shankar, Sam Owre,
Friedrich von Henke, David Cyrluk, and Carl Witty.

The present version of the \pvs\ Language Description was assembled by
Sam Owre, Natarajan Shankar, and John Rushby.



\cleardoublepage
\pagenumbering{arabic}
\setcounter{page}{1}

\setcounter{topnumber}{9}
\renewcommand{\topfraction}{.99}
\setcounter{bottomnumber}{9}
\renewcommand{\bottomfraction}{.99}
\setcounter{totalnumber}{10}
\renewcommand{\textfraction}{.01}
\renewcommand{\floatpagefraction}{.01}

% Document Type: LaTeX
% Master File: intro.tex

\chapter{Introduction}

PVS is a \emph{P}rototype \emph{V}erification \emph{S}ystem for the
development and analysis of formal specifications.  The PVS system
consists of a specification language, a parser, a typechecker, a prover,
specification libraries, and various browsing tools.  This document
primarily describes the specification language and is meant to be used as
a reference manual.  The \emph{PVS System Guide}~\cite{PVS:userguide} is to
be consulted for information on how to use the system to develop
specifications and proofs.  The \emph{PVS Prover Guide}~\cite{PVS:prover}
is a reference manual for the commands used to construct proofs.

In this section, we provide a brief summary of the PVS specification
language, enumerate the key design principles behind the language, and
provide a brief example.

The following sections provide more details on the various features of
the language.  The lexical aspects of the language are detailed in
Section~\ref{lexical}.  Section~\ref{declarations} describes
declarations, Section~\ref{types} describes type expressions,
Section~\ref{expressions} describes expressions, and
Section~\ref{theories} describes theories, theory parameters, and
imports and exports of names.  Section~\ref{names} describes names and
name resolution, and Section~\ref{adts} describes the datatype
facility of PVS.  Finally, 
Appendix~\ref{grammar} provides the grammar of the language.

\section{Summary of the PVS Language}

A PVS specification consists of a collection of \emph{theories}.
Each theory consists of a \emph{signature} for the type names and
constants introduced in the theory, and the axioms, definitions, and
theorems associated with the signature.  For example, a typical
specification for a queue would introduce the \texttt{queue} type and the
operations of \texttt{enq}, \texttt{deq}, and \texttt{front} with their
associated types.  In such a theory, one can either define a
representation for the \texttt{queue} type and its associated operations in
terms of some more primitive types and operations, or merely axiomatize
their properties.  A theory can build on other theories: for example, a
theory for ordered binary trees could build on the theory for
binary trees.  A theory can be \emph{parametric} in certain specified
types and values: as examples, a theory of queues can be parametric in
the maximum queue length, and a theory of ordered binary trees can be
parametric in the element type as well as the ordering relation.  It is
possible to place constraints, called \emph{assumptions}, on the
parameters of a theory so that, for instance, the ordering relation
parameter of an ordered binary tree can be constrained to be a total
ordering.

The PVS specification language is based on simply typed higher-order
logic.  Within a theory, \emph{types} can be defined starting from
\emph{base} types (Booleans, numbers, etc.) using the function, record,
and tuple type constructions.  The \emph{terms} of the language can be
constructed using function application, lambda abstraction, and record and
tuple construction.

There are a few significant enhancements to the simply typed language
above that lend considerable power and sophistication to PVS.  New
uninterpreted base types may be introduced.  One can define a
\emph{predicate subtype} of a given type as the subset of individuals in a
type satisfying a given predicate: the subtype of nonzero reals is
written as \texttt{\{x:real | x /= 0\}}.  One benefit of such
subtyping is that when an operation is not defined on all the elements of
a type, the signature can directly reflect this.  For example, the
division operation on reals is given a type where the denominator is
constrained to be nonzero.  Typechecking then ensures that
division is never applied to a zero denominator.  Since the predicate used
in defining a predicate subtype is arbitrary, typechecking is undecidable
and may lead to proof obligations called \emph{type correctness
conditions} (TCCs).  The user is expected to discharge these proof
obligations with the assistance of the PVS prover.  The PVS type system
also features dependent function, record, and tuple type constructions.
There is also a facility for defining a certain class of abstract datatype
(namely well-founded trees) theories automatically.

\section{PVS Language Design Principles}

There are several basic principles that have motivated the design of
PVS which are explicated in this section.

\paragraph{Specification vs. Programming Languages.}
A specification represents requirements or a design whereas a program
text represents an implementation of a design.  A program can be seen as
a specification, but a specification need not be a program.  Typically,
a specification expresses \emph{what} is being
computed whereas a program expresses \emph{how} it is computed.  A
specification can be incomplete and still be meaningful whereas an
incomplete program will typically not be executable.  A specification
need not be executable; it may use high-level constructs, quantifiers
and the like, that need have no computational meaning.  However, there
are a number of aspects of programming languages that a specification
language should include, such as:
\begin{itemize}

\item the usual basic types: booleans, integers, and rational numbers

\item the familiar datatypes of programming languages such as arrays,
records, lists, sequences, and abstract datatypes

\item the higher-order capabilities provided by modern functional
programming languages so that extremely general-purpose operations can
be defined

\item definition by recursion

\item support for dividing large specifications into parameterized
modules

\end{itemize}

It is clearly not enough to say that a specification language shares some
important features of a programming language but need not be executable.
Any useful formal language must have a clearly defined
semantics\footnote{The PVS semantics are presented in a technical
report~\cite{PVS:semantics}.} and must be capable of being manipulated in
ways that are meaningful relative to the semantics.  A programming
language for example can be given a denotational semantics so that the
execution of the program respects its denotational meaning.  The reason
one writes a specification in a formal language is typically to ensure
that it is sensible, to derive some useful consequences from it, and to
demonstrate that one specification implements another.  All of these
activities require the notion of a justification or a proof based on the
specification, a notion that can only be captured meaningfully within the
framework of logic.

\paragraph{Untyped set theory versus higher-order logic}
\index{set theory}\index{higher-order logic}
Which logic should be chosen?  There is a wide variety of choices:
simple propositional logics, which can be classical or intuitionistic,
equational logics, quantificational logics, modal and temporal logics,
set theory, higher-order logic, etc.  Some propositional and modal
logics are appropriate for dealing with finite state machines where one
is primarily interested in efficiently deciding certain finite state
machine properties.  For a general purpose specification language,
however, only a set theory or a higher-order logic would provide the
needed expressiveness.  Higher-order logic requires strict typing to
avoid inconsistencies whereas set theory restricts the rules for forming
sets.  Set theory is inherently untyped, and grafting a typechecker onto
a language based on set theory is likely to be too strict and arbitrary.
Typechecking, however, is an extremely important and easy way of
checking whether a specification makes semantic sense (although 
for an opposing view, the reader is referred to a report by Lamport
and Paulson~\cite{Lamport&Paulson97}).  Higher-order
logic does admit effective typechecking but at the expense of an
inflexible type system.  Recent advances in type theory have made it
possible to design more flexible type systems for higher-order logic
without losing the benefits of typechecking.  We have therefore chosen
to base PVS on higher-order logic.

\paragraph{Total versus partial functions}
\index{function!total}\index{function!partial} In the PVS higher-order
logic, an individual is either a function, a tuple, a record, or the
member of a base type.  Functions are extremely important in higher-order
logic.  They are \emph{first-class} individuals, i.e., variables can range
over functions.  In general, functions can represent either \emph{total}
or \emph{partial} maps.  A total map from domain $A$ to range $B$ maps
each element of $A$ to some element of $B$, whereas a partial map only
maps some of the elements of $A$ to elements of $B$.  Most traditional
logics build in the assumption that functions represent total maps.
Partial functions arise quite naturally in specifications.  For example,
the division operation is undefined on a zero denominator and the
operation of popping a stack is undefined on an empty stack.

Some recent logics, notably those of VDM~\cite{Jones:VDM},
LUTINS~\cite{Farmer:functions}, RAISE~\cite{RAISE-tutorial},
Beeson~\cite{Beeson:book} and Scott~\cite{Scott79}, admit partial
functions.  In these logics, some terms may be \emph{undefined} by not
denoting any individuals.  Some of these logics have mechanisms for
distinguishing defined and undefined terms, while others allow
``undefined'' to propagate from terms to expressions and therefore must
employ multiple truth values.  In all these cases, the ability to
formalize partially defined functions comes at the cost of complicating
the deductive apparatus, even when the specification does not involve any
partial functions.  Though logics that allow partial functions are
extremely interesting, we have chosen to avoid partial functions in PVS.
We have instead employed the notion of a \emph{predicate
subtype}\index{predicate subtype}, a type that consists of those elements
of a given type satisfying a given predicate.  Using predicate subtypes,
the type of the division operator, for example, can be constrained to
admit only nonzero denominators.  Division then becomes a total operation
on the domain consisting of arbitrary numerators and nonzero denominators.
The domain of a \emph{pop} operation on stacks can be similarly restricted
to nonempty stacks.  PVS thus admits partial functions within the
framework of a logic of total functions by enriching the type system to
include predicate subtypes.  We find this use of predicate subtypes to be
significantly in tune with conventional mathematical practice of being
explicit about the domain over which a function is defined.

\section{An Example: \texttt{stacks}}\label{stacks-example}
\index{stacks example@\texttt{stacks} example}

In this section we discuss a specific example, the theory of
\texttt{stacks}, in order to give a feel for the various aspects of the
PVS language before going into detail.  Apart from the basic notation for
defining a theory, this example illustrates the use of type parameters at
the theory level, the general format of declarations, the use of predicate
subtyping to define the type of nonempty stacks, and the generation of
typechecking obligations.

\pvstheory{stacks-alltt}{{Theory \texttt{stacks}}}{stacks-alltt}

Figure~\ref{stacks-alltt} illustrates a theory for stacks of an arbitrary
type with corresponding stack operations.  Note that this is not the
recommended approach to specifying stacks; a more convenient and complete
specification is provided in Section~\ref{stacks-adt},
page~\pageref{stacks-adt}.

The first line introduces a theory named \texttt{stacks} that is
parameterized by a type \texttt{t} (the \emph{formal parameter} of
\texttt{stacks}).  The keyword \texttt{TYPE+} indicates that \texttt{t} is
a \emph{non-empty} type.  The uninterpreted (nonempty) type \texttt{stack}
is declared, and the constant \texttt{empty} and variable \texttt{s} are
declared to be of type \texttt{stack}.  The defined predicate
\texttt{nonemptystack?}~is then declared on elements of type
\texttt{stack}; it is \texttt{true} for a given \texttt{stack} element
iff\footnote{Iff is short for ``if and only if''.} that element is not
equal to \texttt{empty}.  The functions \texttt{push}, \texttt{pop}, and
\texttt{top} are then declared.  Note that the predicate
\texttt{nonemptystack?}~is being used as a type in specifying the
signatures of these functions; any predicate may be used as a type simply
by putting parentheses around it.

The variables \texttt{x} and \texttt{y} are then declared, followed by the
usual axioms for \texttt{push}, \texttt{pop}, and \texttt{top}, which make
\texttt{push} a stack constructor and \texttt{pop} and \texttt{top} stack
accessors.  Finally, there is the theorem \texttt{pop2push2}, that can
easily be proved by two applications of the \texttt{pop\_push} axiom.

This simple theorem has an additional facet that shows up during
typechecking.  Note that \texttt{pop} expects an element of type
\texttt{(nonemptystack?)} and returns a value of type \texttt{stack}.
This works fine for the inner \texttt{pop} because it is applied to
\texttt{push}, which returns an element of type \texttt{(nonemptystack?)};
but the outer occurrence of \texttt{pop} cannot be seen to be type correct
by such syntactic means.  In cases like these, where a subtype is expected
but not directly provided, the system generates a \emph{type-correctness
condition} (TCC).  In this case, the TCC is
\begin{pvsex}
  pop2push2_TCC1: OBLIGATION
    (FORALL (s: stack, y: t, x: t):
      nonemptystack?(pop(push(x, push(y, s)))))
\end{pvsex}
and is easily proved using the \texttt{pop\_push} axiom.  The system keeps
track of all such obligations and will flag the unproved ones during proof
chain analysis.

Parameterized theories such as \texttt{stacks} introduce theory schemas,
where the type \texttt{t} may be instantiated with any other nonempty
type.  To use the types, constants, and formulas of the \texttt{stacks}
theory from another theory, the \texttt{stacks} theory must be imported,
with \emph{actual parameters} provided for the corresponding theory
parameters.  This is done by means of an \texttt{IMPORTING} clause. For
example, the theory
\begin{pvsex}
  ustacks : THEORY
    BEGIN
    IMPORTING stacks[int], stacks[stack[int]]

    si : stack[int]
    sos : stack[stack[int]] = push(si, empty)
    END ustacks
\end{pvsex}
imports stacks of integers and stacks of stacks of integers.  The constant
\texttt{si} is then declared to be a stack of integers, and the constant
\texttt{sos} is a stack of stacks of integers whose top element is
\texttt{si}.  Note that the system is able to determine which instance of
\texttt{push} and \texttt{empty} is meant from the type of the first
argument.  In general, the typechecker infers the type of an expression
from its context.  


% Document Type: LaTeX
% Master File: language.tex
\chapter{The Lexical Structure}\label{lexical}

PVS specifications are text files, each composed of a sequence of lexical
elements which in turn are made up of characters.  The lexical elements of
PVS are the identifiers, reserved words, special symbols, numbers,
whitespace characters, and comments.

Identifiers\index{identifiers} are composed of letters, digits, and the
characters \texttt{\_} or \texttt{?}; they must begin with a letter, which
are the usual ASCII letters, or any Unicode character that is not one of
the ASCII non-letter characters.  Note that keywords from
Figure~\ref{reserved-words} and operators from Figure~\ref{special-symbols} may
be embedded in identifiers, but may not be identifiers themselves.  Thus
\texttt{candy} is an identifier, though it contains the keyword
\texttt{and}, and ∧∧ is an identifier, though it contains the keyword ∧.  

They may be arbitrarily long.  Identifiers are case-sensitive;
\texttt{FOO}, \texttt{Foo}, and \texttt{foo} are different identifiers.
PVS strings contain any Unicode character: to include a \texttt{"} in the
string, use \texttt{\char'134 "} and to include a \texttt{\char'134} use
\texttt{\char'134\char'134}.  For more on Unicode, see the PVS User Guide.

\pvsbnf{bnf-lexical}{Lexical Syntax}

The reserved words\index{reserved words} are shown in
Figure~\ref{reserved-words}.  Unlike identifiers, they are not
case-sensitive.  In this document, reserved words are always displayed in
upper case.  Note that identifiers may have reserved words embedded in
them, thus \texttt{ARRAYALL} is a valid identifier and will not be
confused with the two embedded reserved words.  The meaning of the
reserved words are given in the appropriate sections; they are collected
here for reference.

\begin{figure}[tb]
{\smaller\tt
\begin{tabular}{|*{5}{p{1.03in}}|}\hline
\input{keywords}
\hline
\end{tabular}}
\caption{\pvs\ Reserved Words}\label{reserved-words}
\end{figure}

The special symbols\index{special symbols} are listed in
Figure~\ref{special-symbols}.  All of these symbols are separators; they
separate identifiers, numbers, and reserved words.

\begin{figure}[tb]
\begin{center}
  {\small\tt
    % \fontspec{Cambria Math}
    %\fontspec{TeX Gyre Pagella Math}
    % \fontspec{Latin Modern Math}
\begin{tabular}{|*{6}{@{\hspace*{.2in}}c@{\extracolsep{.5in}}}@{\hspace*{.25in}}|}\hline
\input{operator-table}
\hline
\end{tabular}}
\end{center}
\caption{\pvs\ Special Symbols}\label{special-symbols}
\end{figure}

The whitespace characters are space, tab, newline, return, and newpage;
they are used to separate other lexical elements.  At least one whitespace
character must separate adjacent identifiers, numbers, and reserved words.

Comments\index{comments} may appear anywhere that a whitespace character
is allowed.  They consist of the `\texttt{\%}'\index{\%@\texttt{\%}} character
followed by any sequence of characters and terminated by a newline.

The \emph{definable} symbols are shown in table~\ref{definable-symbols}.
These keywords and symbols may be given declarations.  Some of them have
declarations given in the prelude.\footnote{In particular,
\texttt{\char38}, \texttt{*}, \texttt{+}, \texttt{-}, \texttt{/},
\texttt{/=}, \texttt{<}, \texttt{<<}, \texttt{<=}, \texttt{<=>},
\texttt{=}, \texttt{=>}, \texttt{>}, \texttt{>=}, \texttt{AND},
\texttt{IFF}, \texttt{IMPLIES}, \texttt{NOT}, \texttt{O}, \texttt{OR},
\texttt{WHEN}, \texttt{XOR}, \texttt{\char94}, and \texttt{\char126} are
declared there.  Note that many of these are overloaded, for example,
\texttt{\char94} has three different definitions.}  Any of these may be
(re)declared any number of times, though this may lead to ambiguities.
Such ambiguities may be resolved by including the theory name, actual
parameters,  and possibly the type as a coercion.

Symbols that are binary infix (\hyperlink{Binop}{\emph{Binop}}), for
example \texttt{AND} and \texttt{+}, may be declared with any number of
arguments.  If they are declared with two arguments then they may
subsequently be used in prefix or infix form.  Otherwise they may only be
used in prefix form.  Similarly for unary operators, and the \texttt{IF}
operator, which may be used in \texttt{IF-THEN-ELSE-ENDIF} form if
declared with three arguments.

Note that when typing the operators \texttt{/\\} or \texttt{\\/} outside
of a specification, the backslash may need to be doubled (or in rare
cases, quadrupled).  This is because it is commonly used as an ``escape''
character, and the character following may be interpreted specially.

The symbol pairs \lit{[|} \lit{|]}, \lit{(|} \lit{|)}, 
\lit{$\{$|} \lit{|$\}$}, {\lit{$〈$} \lit{$〉$}, \lit{$⟦$} \lit{$⟧$},
  \lit{$«$} \lit{$»$},
  {⟪⟫},
  {⌈⌉},
  {⌊⌋},
  {⌜⌝}, and
  {⌞⌟} are
available as outfix operators.  They are
declared and may be used by concatenation,
for example, with the declaration \texttt{[||]:\ [bool, int -> int]} the
outfix term \texttt{[| TRUE, 0 |]} is equivalent to the prefix form
\texttt{[||](TRUE, 0)}.

\begin{figure}[tb]
\begin{center}
\renewcommand{\arraystretch}{1.2}
{\small\tt
    \fontspec{Latin Modern Math}
\begin{tabular}{|*{6}{@{\hspace*{.2in}}c@{\extracolsep{.4in minus .4in}}}@{\hspace*{.2in}}|}\hline
\char35\char35 &  \char47\char47 &  \char60\char124 &  AND &  ORELSE &  \char123\char124\char124\char125\\
\char38 &  \char47\char61 &  \char61 &  ANDTHEN &  TRUE &  \char124\char45\\
\char40\char124\char124\char41 &  \char47\char92\char92 &  \char61\char61 &  FALSE &  WHEN &  \char124\char61\\
\char42 &  \char60 &  \char61\char62 &  IF &  XOR &  \char124\char62\\
\char42\char42 &  \char60\char60 &  \char62 &  IFF &  \char91\char93 &  \char126\\
\char43 &  \char60\char60\char61 &  \char62\char61 &  IMPLIES &  \char91\char124\char124\char93 &  \\
\char43\char43 &  \char60\char61 &  \char62\char62 &  NOT &  \char92\char92\char47 &  \\
\char45 &  \char60\char61\char62 &  \char62\char62\char61 &  O &  \char94 &  \\
\char47 &  \char60\char62 &  \char64\char64 &  OR &  \char94\char94 &  \\


\hline
\end{tabular}}
\end{center}
\caption{\pvs\ Definable Symbols}\label{definable-symbols}
\end{figure}

%%% Local Variables: 
%%% mode: latex
%%% TeX-master: "language"
%%% End: 

% Document Type: LaTeX
% Master File: language.tex

\chapter{Declarations}\label{declarations}
\index{declaration|(pidx}

Entities of PVS are introduced by means of \emph{declarations}, which are
the main constituents of PVS specifications.  Declarations are used to
introduce types, variables, constants, formulas, judgements, conversions,
and other entities.  Most declarations have an \emph{identifier} and
belong to a unique theory.  Declarations also have a body which indicates
the \emph{kind} of the declaration and may provide a signature or
definition for the entity.  \emph{Top-level}
declarations\index{declaration!top-level} occur in the formal parameters,
the assertion section and the body of a theory.  \emph{Local}
declarations\index{declaration!local} for variables may be given, in
association with constant and recursive declarations and \emph{binding
expressions} (\eg\ involving \texttt{FORALL} or \texttt{LAMBDA}).
Declarations are ordered within a theory; earlier declarations may not
reference later ones.\footnote{Thus mutual recursion is not directly
supported.  The effect can be achieved with a single recursive function
that has an argument that serves as a switch for selecting between two or
more subexpressions.}

\index{exporting|(}\index{importing|(}
Declarations introduced in one theory may be referenced in another by
means of the \texttt{IMPORTING} and \texttt{EXPORTING} clauses.  The
\texttt{EXPORTING} clause of a theory indicates those entities that may be
referenced from outside the theory.  There is only one such clause for a
given theory.  The \texttt{IMPORTING} clauses provide access to the
entities exported by another theory.  There can be many \texttt{IMPORTING}
clauses in a theory; in general they may appear anywhere a top-level
declaration is allowed.  See Section~\ref{importings} for more details.
\index{importing|)}\index{exporting|)}

PVS allows the overloading\index{overloading} of declaration identifiers.
Thus a theory named \texttt{foo} may declare a constant \texttt{foo} and a
formula \texttt{foo}.  To support this \emph{ad hoc} overloading,
declarations are classified according to kind\index{declaration!kind}; in
PVS the primary kinds are \emph{type}\index{declaration!kind!type},
\emph{prop}\index{declaration!kind!prop},
\emph{expr}\index{declaration!kind!expr}, and
\emph{theory}\index{declaration!kind!theory}.  Type declarations are of
kind \emph{type}, and may be referenced in type declarations, actual
parameters, signatures, and expressions.  Formula declarations are of kind
\emph{prop}, and may be referenced in auto-rewrite declarations
(Section~\ref{auto-rewrite-decls}) or proofs (see the PVS Prover
Guide~\cite{PVS:prover}).  Variable, constant, and recursive definition
declarations are of kind \emph{expr}; these may be referenced in
expressions and actual parameters.  Newly introduced names need only be
unique within a kind, as there is no way, for example, to use an
expression where a type is expected.\footnote{There are a few exceptions,
for example the actual parameters of theories, since theories may be
instantiated with types or expressions.}

\pvsbnf{bnf-decls}{Declarations Syntax}
\index{syntax!declarations}
\pvsbnf{bnf-decls-aux}{Declarations Syntax (cont.)}
\index{syntax!declarations}

Declarations generally consist of an
\emph{identifier}\index{declaration!identifier}, an optional list of
\emph{bindings}\index{declaration!binding}, and a
\emph{body}\index{declaration!body}.  The body determines the kind of the
declaration, and the bindings and the body together determine the
signature and definition of the declared entity.  Multiple
declarations\index{declaration!multiple} may be given in compressed form
in which a common body is specified for multiple identifiers; for example
%
\begin{pvsex}
  x, y, z: VAR int
\end{pvsex}
In every case this is treated the same as the expanded form, thus the
above is equivalent to:
\begin{pvsex}
  x: VAR int
  y: VAR int
  z: VAR int
\end{pvsex}

In the rest of this chapter we describe declarations for types, variables,
constants, recursive definitions, macros, inductive and coinductive
definitions, formulas, judgements, conversions, libraries, and
auto-rewrites.  The declarations for theory parameters, importings,
exportings, and theory abbreviations are given in Chapter~\ref{theories}.
Figure~\ref{bnf-decls} gives the syntax for declarations.

\section{Type Declarations}\label{type-declarations}
\index{type declarations|(}

Type declarations are used to introduce new type names to the context.
There are four kinds of type declaration:

\begin{itemize}

\item \emph{uninterpreted type declaration}: \texttt{T:\ TYPE}
\index{uninterpreted type}\index{type!uninterpreted}

\item \emph{uninterpreted subtype declaration}: \texttt{S:\ TYPE FROM T}
\index{uninterpreted subtype}\index{type!uninterpreted subtype}

\item \emph{interpreted type declaration}: \texttt{T:\ TYPE =
int}\index{interpreted type}\index{type!interpreted}

\item \emph{enumeration type declarations}: \texttt{T:\ TYPE = \setb r,
g, b\sete} \index{enumeration types}\index{type!enumeration}

\end{itemize}

These type declarations introduce \emph{type names}\index{type!name}
that may be referenced in type expressions (see Section~\ref{types}).
They are introduced using one of the keywords
\keyword{TYPE}\index{type@\texttt{TYPE}},
\keyword{NONEMPTY\_TYPE}\index{type@\texttt{NONEMPTY\_TYPE}}, or
\keyword{TYPE+}\index{type+@\texttt{TYPE+}}.

\subsection{Uninterpreted Type Declarations}
\index{type!uninterpreted|(}

Uninterpreted types support abstraction by providing a means of
introducing a type with a minimum of assumptions on the type.  An
uninterpreted type imposes almost no constraints on an implementation of
the specification.  The only assumption made on an uninterpreted type
\texttt{T} is that it is disjoint from all other types, except for
subtypes of \texttt{T}.  For example,
\begin{pvsex}
  T1, T2, T3: TYPE
\end{pvsex}
%
introduces three new pairwise disjoint types.  If desired, further
constraints may be put on these types by means of axioms or assumptions
(see Section~\ref{formula-declarations} on
page~\pageref{formula-declarations}).

It should be emphasized that uninterpreted types are important in
providing the right level of abstraction in a specification.  Specifying
the type body may have the undesired effect of restricting the possible
implementations, and cluttering the specification with needless detail.

\index{type!uninterpreted|)}\index{uninterpreted type|)}


\subsection{Uninterpreted Subtype Declarations}
\index{uninterpreted subtype|(}

Uninterpreted subtype declarations are of the form
\begin{pvsex}
  s: TYPE FROM t
\end{pvsex}
\index{FROM@\texttt{FROM}}
This introduces an uninterpreted
\emph{subtype}\index{subtypes}\index{type!subtype} \texttt{s} of
the \emph{supertype}\index{supertype}\index{type!supertype}
\texttt{t}.  This has the same meaning as
\begin{pvsex}
  s_pred: [t -> bool]
  s: TYPE = (s_pred)
\end{pvsex}
%
in which a new predicate is introduced in the first line and the type
\texttt{s} is declared as a \emph{predicate} subtype in the second
line\footnote{This is described in Section~\ref{subtypes}
(page~\pageref{subtypes}).}.  No assumptions are made about uninterpreted
subtypes; in particular, they may or may not be empty, and two different
uninterpreted subtypes of the same supertype may or may not be disjoint.
Of course, if the supertypes themselves are disjoint, then the
uninterpreted subtypes are as well.

\index{uninterpreted subtype|)}

\subsection{Structural Subtypes}

PVS has support for structural subtyping for record and tuple types.  A
record type \texttt{S} is a structural subtype of record type \texttt{R}
if every field of \texttt{R} occurs in \texttt{S}, and similarly, a tuple
type \texttt{T} is a structural subtype of a tuple type forming a prefix
of \texttt{T}.  Section \ref{type-extensions} gives
examples, as \texttt{colored\_point} is a structural subtype of
\texttt{point}, and \texttt{R5} is a structural subtype of \texttt{R3}.
Structural subtypes are akin to the class hierarchy of object-oriented
systems, where the fields of a record can be viewed as the slots of a
class instance.  The PVS equivalent of setting a slot value is the
override expression (sometimes called update), and this works with
structural subtypes, allowing the equivalent of generic methods to be
defined.  Here is an example:
\begin{pvsex}
points: THEORY
BEGIN
 point: TYPE+ = [# x, y: real #]
END points

genpoints[(IMPORTING points) gpoint: TYPE <: point]: THEORY
BEGIN
 move(p: gpoint)(dx, dy: real): gpoint =
  p WITH [`x := p`x + dx, `y := p`y + dy]
END genpoints

colored_points: THEORY
BEGIN
 IMPORTING points
 Color: TYPE = {red, green, blue}
 colored_point: TYPE = point WITH [# color: Color #]
 IMPORTING genpoints[colored_point]
 p: colored_point
 move0: LEMMA move(p)(0, 0) = p
END colored_points
\end{pvsex}

The declaration for \texttt{gpoint} uses the structural subtype operator
\texttt{<:}.  This is analogous to the \texttt{FROM} keyword, which
introduces a (predicate) subtype.  This example also serves to explain
why we chose to separate structural and predicate subtyping.  If they
were treated uniformly, then \texttt{gpoint} could be instantiated with
the unit disk; but in that case the \texttt{move} operator would not
necessarily return a \texttt{gpoint}.  The TCC could not be generated
for the \texttt{move} declaration, but would have to be generated when
the \texttt{move} was referenced.  This both complicates typechecking,
and makes TCCs and error messages more inscrutable.  If both are
desired, simply include a structural subtype followed by a predicate
subtype, for example:
\begin{pvsex}
genpoints[(IMPORTING points) gpoint: TYPE <: point,
          spoint: TYPE FROM gpoint]: THEORY
\end{pvsex}
Now \texttt{move} may be applied to \texttt{gpoint}s, but if applied to a
\texttt{spoint} an unprovable TCC will result.

Structural subtypes are a work in progress.  In particular, structural
subtyping could be extended to function and datatypes.  And to have
real object-oriented PVS, we must be able to support a form of method
invocation.


\subsection{Empty and Singleton Record and Tuple Types}

Empty and singleton record and tuple types are now allowed in PVS.
Thus the following are valid declarations:
\begin{pvsex}
Tup0: TYPE = [ ]
Tup1: TYPE = [int]
Rec0: TYPE = [# #]
\end{pvsex}
Note that the space is important in the empty tuple type, as otherwise
it is taken to be an operator (the box operator).


\subsection{Interpreted Type Declarations}
\index{interpreted type declarations|(}\index{type!interpreted|(}

Interpreted type declarations are primarily a means for providing names
for type expressions.  For example,
\begin{pvsex}
  intfun: TYPE = [int -> int]
\end{pvsex}
%
introduces the type name \texttt{intfun} as an abbreviation for the type
of functions with integer domain and range.  Because PVS uses
\emph{structural equivalence}\index{structural equivalence} instead of
\emph{name equivalence}\index{name equivalence}, any type expression
\texttt{T} involving \texttt{intfun} is equivalent to the type expression
obtained by substituting \texttt{[int -> int]} for \texttt{intfun} in
\texttt{T}.  The available type expressions are described in
Chapter~\ref{types} on page~\pageref{types}.

Interpreted type declarations may be given
parameters.\index{parameterized type names} For example, the type of
integer subranges may be given as
\begin{pvsex}
  subrange(m, n: int): TYPE = \setb{}i:int | m <= i AND i <= n\sete
\end{pvsex}
and \texttt{subrange} with two integer parameters may subsequently be used
wherever a type is expected.  Any use of a parameterized type must include
all of the parameters, so currying of the parameters is not allowed.  Note
that \texttt{subrange} may be overloaded to declare a different type in
the same theory without any ambiguity, as long as the number or type of
parameters is different.

\index{type!interpreted|)}\index{interpreted type declarations|)}


\subsection{Enumeration Type Declarations}\label{enum-types}
\index{enumeration types|(}\index{type!enumeration|(}

Enumeration type declarations are of the form
\begin{pvsex}
  enum: TYPE = \setb{}e_1,\ldots, e_n\sete
\end{pvsex}
%
where the \texttt{e\_i} are distinct identifiers which are taken to
completely enumerate the type.  This is actually a shorthand for the
datatype specification
\begin{pvsex}
  enum: DATATYPE
    e_1: e_1?
         \(\vdots\)
    e_n: e_n?
  END enum
\end{pvsex}
%
explained in Chapter~\ref{adts}.  Because of this, enumeration types may
only be given as top-level declarations, and are \emph{not} type
expressions.  The advantage of treating them as datatypes is that the
necessary axioms are automatically generated, and the prover has built-in
facilities for handling datatypes.

\index{type!enumeration|)}\index{enumeration types|)}

\index{type declarations|)}


\subsection{Empty versus Nonempty Types}
\label{emptytypes}
\index{nonempty type}
\index{empty type}
\index{type!nonempty|(}\index{type!empty|(}

As noted before, PVS allows empty types, and the term \emph{type} refers
to either empty or nonempty types.  Constants declared to be of a given
type provide elements of the type, so the type must be nonempty or there
is an inconsistency.  Thus whenever a constant is declared, the system
checks whether the type is nonempty, and if it cannot decide that it is
nonempty it generates an \emph{existence TCC}.\index{existence
TCC}\index{TCC!existence} This is the simple explanation, but it is made
somewhat complicated by the considerations of formal parameters,
uninterpreted versus interpreted type declarations, explicit declarations
of nonemptiness, and
\keyword{CONTAINING}\index{CONTAINING@\texttt{CONTAINING}} clauses on type
declarationss, as well as a desire to keep the number of TCCs generated to
a minimum, while guaranteeing soundness.  The details are provided below.

First note that having variables range over an empty type causes no
difficulties,\footnote{If the type \texttt{T} is empty, then the following
two equivalences hold:
\begin{alltt}
  (FORALL (x: T): p(x)) IFF TRUE \quad \mbox{\textrm{and}} \quad (EXISTS (x: T): p(x)) IFF FALSE
\end{alltt}
}
so variable declarations and variable bindings never trigger the
nonemptiness check.

During typechecking, type declarations may indicate that the type is
nonempty, and constant declarations of a given type require that the type
be nonempty.  When a type is determined to be nonempty, it is marked as
such so that future checks of constants do not trigger more TCCs.  Below
we describe how type declarations are handled first for declarations in the
body of a theory, and then for type declarations that appear in the formal
parameters, as they require special handling.

\paragraph{Theory Body Type Declarations}

\begin{itemize}

\item Uninterpreted type or subtype declarations introduced with the
keyword \keyword{TYPE} may be empty.  Declaring a constant of that type
will lead to a TCC that is unprovable without further axioms.

\item Uninterpreted type declarations introduced with the keyword
\keyword{NONEMPTY\_TYPE}\index{nonempty_type@\keyword{NONEMPTY\_TYPE}}
or \keyword{TYPE+}\index{type+@\texttt{TYPE+}} are assumed to be nonempty.
Thus the type is marked nonempty.

\item Uninterpreted subtype declarations introduced with the keyword
\keyword{NONEMPTY\_TYPE} or \keyword{TYPE+} are assumed to be nonempty, as long as the
supertype is nonempty.  Thus the supertype is checked, and an existence
TCC is generated if the supertype is not known to be nonempty.  Then the
subtype is marked nonempty.

\item The type of an interpreted constant is nonempty, as the definition
provides a witness.

\item Interpreted type declarations introduced with the keyword
\keyword{TYPE} may be non\-emp\-ty, depending on the type definition.

\item Any interpreted type declaration with a \keyword{CONTAINING} clause
is marked nonempty, and the \keyword{CONTAINING} expression is typechecked
against the specified type.  In this case no existence TCC is generated,
since the \keyword{CONTAINING} expression is a witness to the type.  Of
course, other TCCs may be generated as a result of typechecking the
\keyword{CONTAINING} expression.

\end{itemize}

\paragraph{Formal Type Declarations}

Only uninterpreted (sub)type declarations may appear in the formal
parameters list.

\begin{itemize}

\item Formal type declarations introduced with the \texttt{TYPE} keyword may
be empty.  This is handled according to the occurrences of constant
declarations involving the type.

\item If there is a constant declaration of that type in the formal
parameter list, then no TCCs are generated, since
any instance of the theory will need to provide both the type and a
witness.  The type is marked nonempty in this case.

\item If the type declaration is a formal parameter and a constant is
declared whose type involves the type, but is not the type itself (for
example, if the formal theory parameters are \texttt{[t:\ TYPE, f:\ [t ->
t]]}), then a TCC may be generated, and a comment is added to the TCC
indicating that an assuming clause may be needed in order to discharge the
TCC.  This TCC will be generated only if an earlier constant declaration
hasn't already forced the type to be marked nonempty.  Note that there are
circumstances in which the formal type may be empty but the type
expression involving that type is nonempty.  This is discussed further
below.

\end{itemize}

\subsection{Checking Nonemptiness}\label{nonemptiness-check}
\index{type!nonempty}
The typechecker knows a type to be nonempty under the
following circumstances:
\begin{itemize}

\item The type was declared to be nonempty, using either the
\keyword{NONEMPTY\_TYPE}\index{nonempty_type@\keyword{NONEMPTY\_TYPE}} or
the synonymous \keyword{TYPE+}\index{type+@\texttt{TYPE+}} keyword.  If the
type is uninterpreted, this amounts to an assumption that the type is
nonempty.  If the type has a definition, then an existence TCC is
generated unless the defining type expression is known to be nonempty.

\item The type was declared to have an element using a
\keyword{CONTAINING}\index{CONTAINING@\texttt{CONTAINING}} expression.

\item A constant was declared for the type.  In this case an existence TCC
is generated for the first such constant, after which the type is marked
as nonempty.

\item It was marked as nonempty from an earlier check.

\end{itemize}

Once an unmarked type is determined to be nonempty, it is marked by the
typechecker so that later checks will not generate existence TCCs.  In
addition, the type components are marked as nonempty.  Thus the types that
make up a tuple type, the field types of a record type, and the supertype
of a subtype are all marked.

It is possible for two equivalent types to be marked differently, for
example:
\begin{pvsex}
  t1: TYPE = \setb{}x: int | x > 2\sete
  t2: TYPE = \setb{}x: int | x > 2\sete
  c1: t1
\end{pvsex}
only marks the first type (\texttt{t1}).  Hence, it is best to name your types and
to use those names uniformly.

\index{type!empty|)}
\index{type!nonempty|)}

\section{Variable Declarations}
\index{variables|(}\index{declaration!variables|(}

Variable declarations introduce new variables and associate a type with
them.  These are \emph{logical} variables, not program variables; they
have nothing to do with state---they simply provide a name and associated
type so that binding expressions and formulas can be succinct.
Variables may not be exported.  Variable
declarations also appear in binding expressions such as \texttt{FORALL} and
\texttt{LAMBDA}.  Such local declarations ``shadow'' any earlier
declarations.  For example, in
\begin{pvsex}
  x: VAR bool
  f: FORMULA (FORALL (x: int): (EXISTS (x: nat): p(x)) AND q(x))
\end{pvsex}
%
The occurrence of \texttt{x} as an argument to \texttt{p} is of type
\texttt{nat}, shadowing the one of type \texttt{int}.  Similarly, the
occurrence of \texttt{x} as an argument to \texttt{q} is of type
\texttt{int}, shadowing the one of type \texttt{bool}.

\index{variables|)}\index{declaration!variables|)}

\section{Constant Declarations}\label{constants}
\index{constants|(}\index{declaration!constants|(}

Constant declarations introduce new constants, specifying their type and
optionally providing a value.  Since PVS is a higher order logic, the term
\emph{constant} refers to functions and relations, as well as the usual
(0-ary) constants.  As with types, there are both \emph{uninterpreted} and
\emph{interpreted} \index{constants!interpreted}%
\index{constants!uninterpreted} constants.  Uninterpreted constants make
no assumptions, although they require that the type be nonempty (see
Section~\ref{nonemptiness-check}, page~\pageref{nonemptiness-check}).
Here are some examples of constant declarations:
\begin{pvsex}
  n: int
  c: int = 3
  f: [int -> int] = (lambda (x: int): x + 1)
  g(x: int): int = x + 1
\end{pvsex}
%
The declaration for \texttt{n} simply introduces a new integer constant.
Nothing is known about this constant other than its type, unless further
properties are provided by \texttt{AXIOM}s.  The other three constants are
interpreted.  Each is equivalent to specifying two declarations: \eg\
the third line is equivalent to
\begin{pvsex}
  f: [int -> int]
  f: AXIOM  f = (LAMBDA (x: int): x + 1)
\end{pvsex}
%
except that the definition is guaranteed to form a \emph{conservative
extension}\index{conservative extension} of the theory.  Thus the
theory remains consistent after the declaration is given if it was
consistent before.

The declarations for \texttt{f} and \texttt{g} above are two different ways to
declare the same function.  This extends to more complex arguments, for
example
\begin{pvsex}
  f: [int -> [int, nat -> [int -> int]]] =
     (LAMBDA (x: int): (LAMBDA (y: int), (z: nat): (LAMBDA (w: int):
       x * (y + w) - z)))
\end{pvsex}
%
is equivalent to
\begin{pvsex}
  f(x: int)(y: int, z: nat)(w: int): int = x * (y + w) - z
\end{pvsex}
%
This can be shortened even further if the variables are declared first:
\begin{pvsex}
  x, y, w: VAR int
  z: VAR nat
  f(x)(y,z)(w): int = x * (y + w) - z
\end{pvsex}
%
Finally, a mix of predeclared and locally declared variables is possible:
\begin{pvsex}
  x, y: VAR int
  f(x)(y,(z: nat))(w: int): int = x * (y + w) - z
\end{pvsex}
%
Note the parentheses around \texttt{z:\ nat}; without these, \texttt{y} would
also be treated as if it were declared to be of type \texttt{nat}.

A construct that is frequently encountered when subtypes are involved is
shown by this example
\begin{pvsex}
  f(x: \setb{}x: int | p(x)\sete): int = x + 1
\end{pvsex}
%
There are two useful abbreviations for this expression.  In the first, we
use the fact that the type \texttt{\setb{}x:\ int | p(x)\sete} is equivalent to
the type expression \texttt{(p)} when \texttt{p} has type \texttt{[int ->
bool]}, and we can write
\begin{pvsex}
  f(x: (p)): int = x + 1
\end{pvsex}
%
The second form of abbreviation basically removes the set braces and the
redundant references to the variable, though extra parentheses are
required:
\begin{pvsex}
  f((x: int | p(x))): int = x + 1
\end{pvsex}
%
Which of these forms to use is mostly a matter of taste; in general,
choose the form that is clearest to read for a given declaration.

Note that functions with range type \texttt{bool} are generally referred
to as \emph{predicates}, and can also be regarded as relations or sets.
For example, the set of positive odd numbers can be characterized by a
predicate as follows:
\begin{pvsex}
  odd: [nat -> bool] = (LAMBDA (n: nat): EXISTS (m: nat): n = 2 * m + 1)
\end{pvsex}
%
PVS allows an alternate syntax for predicates that encourages a
set-theoretic interpretation:
\begin{pvsex}
  odd: [nat -> bool] = \setb{}n: nat | EXISTS (m: nat): n = 2 * m + 1\sete
\end{pvsex}

\index{constants|)}

\section{Recursive Definitions}\label{recursive-definitions}
\index{recursive definitions|(}

Recursive definitions are treated as constant declarations, except that
the defining expression is required, and a \emph{measure}\index{measure
function} must be provided, along with an optional well-founded order
relation.\index{well-founded order releation} The same syntax for
arguments is available as for constant declarations; see the preceding
section.

PVS allows a restricted form of recursive definition; mutual
recursion\index{recursion!mutual}\index{mutual recursion} is not allowed,
and the function must be \emph{total},\index{total function} so that the
function is defined for every value of its domain.  In order to ensure
this, recursive functions must be specified with a
\emph{measure}\index{measure}, which is a function whose signature matches
that of the recursive function, but with range type the domain of the
order relation, which defaults to \texttt{<} on \texttt{nat} or
\texttt{ordinal}\index{ordinal}\index{type!ordinal}.  If the order
relation is provided, then it must be a binary relation on the range type
of the measure, and it must be well-founded; a \emph{well-founded} \tcc\
\index{well-founded TCC}\index{TCC!well-founded} is generated if the order
is not declared to be well-founded.

Here is the classic example of the
\texttt{factorial}\index{factorial@\texttt{factorial}} function:
%
\begin{pvsex}
  factorial(x: nat): RECURSIVE nat =
    IF x = 0 THEN 1 ELSE x * factorial(x - 1) ENDIF
    MEASURE (LAMBDA (x: nat): x)
\end{pvsex}
%
The measure is the expression following the \texttt{MEASURE} keyword (the
optional order relation follows a \texttt{BY} keyword after the
measure).  This definition generates a \emph{termination
TCC};\index{TCC!termination}\index{termination TCC} a proof obligation
which must be discharged in order that the function be well-defined.  In
this case the obligation is
%
\begin{pvsex}
  factorial_TCC2: OBLIGATION
    FORALL (x: nat): NOT x = 0 IMPLIES x - 1 < x
\end{pvsex}

It is possible to abbreviate the given \texttt{MEASURE} function by
leaving out the \texttt{LAMBDA} binding.  For example, the measure
function of the factorial definition may be abbreviated to:
\begin{pvsex}
  MEASURE x
\end{pvsex}
The typechecker will automatically insert a lambda binding corresponding
to the arguments to the recursive function if the measure is not already
of the correct type, and will generate a typecheck error if this process
cannot determine an appropriate function from what has been specified.

A termination \tcc\ is generated for each recursive occurrence of the
defined entity within the body of the definition.\footnote{Some of these
may be subsumed by earlier TCCs, and hence will not be displayed with the
\texttt{M-x show-tccs} command.}  It is obtained in one of two ways.  If a
given recursive reference has at least as many arguments provided as
needed by the measure, then the \tcc\ is generated by applying the measure
to the arguments of the recursive call and comparing that to the measure
applied to the original arguments using the order relation.  The
\texttt{factorial} \tcc\ is of this form.  The context of the occurrence
is included in the \tcc; in this case the occurrence is within the
\texttt{ELSE} part of an \texttt{IF-THEN-ELSE} so the negated condition is
an antecedent to the proof obligation.

If the reference does not have enough arguments available, then the
reference is actually given a \emph{recursive signature}\index{recursive
signature} derived from the recursive function as described below.  This
type constrains the domain to satisfy the measure, and the termination
\tcc\ is generated as a \emph{termination-subtype}
\tcc.\index{termination-subtype TCC}\index{TCC!termination-subtype}
Termination-subtype \tccs\ are recognized as such by the occurrence of the
order in the goal of the \tcc.  For example, we could define a
substitution function for terms as follows.
\begin{session}
  term: DATATYPE
  BEGIN
   mk_var(index: nat): var?
   mk_const(index: nat): const?
   mk_apply(fun: term, args: list[term]): apply?
  END term

  subst(x: (var?), y: term)(s: term): RECURSIVE term =
    (CASES s OF
      mk_var(i): (IF index(x) = i THEN y ELSE s ENDIF),
      mk_const(i): s,
      mk_apply(t, ss): mk_apply(subst(x, y)(t), map(subst(x, y))(ss))
     ENDCASES)
  MEASURE s BY <<
\end{session}
Now the first recursive occurrence of \texttt{subst} has all arguments
provided, so the termination TCC is as expected.  The second occurrence
does not have enough arguments.  The recursive signature of that
occurrence is
\begin{pvsex}
  [[(var?), term] -> [\setb{}z1: term | z1 << s\sete -> term]]
\end{pvsex}
Hence the signature of \texttt{subst(x, y)} is \texttt{[\setb{}z1:\ term | z1 <<
s\sete -> term]}, and map is instantiated to \texttt{map[\setb{}z1:\ term | z1 <<
s\sete, term]}, which leads to the TCC
\begin{pvsex}
 subst_TCC2: OBLIGATION
   FORALL (ss: list[term], t: term, s: term, x: (var?)):
     s = mk_apply(t, ss) IMPLIES every[term](LAMBDA (z: term): z << s)(ss);
\end{pvsex}
Note that this \texttt{map} instance could be given directly, just don't
make the mistake of providing \texttt{map[term, term]}, as this leads to a
TCC that says every \texttt{term} is \texttt{<<} \texttt{s}.
For the same reason, if the uncurried form of this definition is given,
then a lambda expression will have to be provided and the type will have
to include the measure, for example,
\begin{session}
   subst(x: (var?), y, s: term): RECURSIVE term =
     (CASES s OF
       mk_var(i): (IF index(x) = i THEN y ELSE s ENDIF),
       mk_const(i): s,
       mk_apply(t, ss): mk_apply(subst(x, y, t),
                                 map(LAMBDA (s1: \setb{}z: term|z<<s\sete):
                                       subst(x, y, s1))(ss))
      ENDCASES)
   MEASURE s BY <<
\end{session}
\renewcommand{\textfraction}{.1}

The recursive signature is generated based on the type of the recursive
function and the measure.  For curried functions, it may be that the
measure does not have the entire domain of the recursive function, but
only the first few.  For example, consider the measure for the function
\texttt{f}.
\begin{pvsex}
  f(r: real)(x, y: nat)(b: boolean): RECURSIVE boolean
    = ...
   MEASURE LAMBDA (r: real): LAMBDA (x, y: nat): x
\end{pvsex}
The type of the measure function is \texttt{[real -> [nat, nat -> nat]]},
which is a prefix of the function type.  In deriving the recursive
signature, the last domain type of the measure is constrained (using a
subtype) in the corresponding position of the recursive function type.  In
this case the recursive signature is
\begin{pvsex}
  [real -> [\setb{}z: [nat, nat] | z`1 < x\sete -> [boolean -> boolean]]]
\end{pvsex}
Note that the recursive signature is a dependent type that depends on the
arguments of the recursive function (\texttt{x} in this case), and hence
only applies within the body of the recursive definition.

The formal argument that typechecking the body of a recursive function
using the recursive signature is sound will appear in a future version of
the semantics manual, for now note that simple attempts to subvert this
mechanism do not work, as the following example illustrates.
\begin{pvsex}
  fbad: RECURSIVE [nat -> nat] = fbad
   MEASURE lambda (n: nat): n
\end{pvsex}
This leads an unprovable TCC.
\begin{pvsex}
  fbad_TCC1: OBLIGATION FORALL (x1: nat, x: nat): x < x1;
\end{pvsex}
The TCC results from the comaprison of the expected type \texttt{[nat ->
nat]} to the derived type \texttt{[\setb{}z:\ nat | z < x1\sete -> nat]}.  Remember
that in PVS domains of function types must be equal in order for the
function types to satisfy the subtype relation, and this is exactly what
the TCC states.

\pvstheory{f91-alltt}{Theory \texttt{f91}}{f91-alltt}
\index{f91@{\texttt{f91}}}

When a doubly recursive call is found, the inner recursive calls are
replaced by variables in the termination \tccs\ generated for the outer
calls.  For example, the theory of Figure~\ref{f91-alltt} generates the
termination TCC of Figure~\ref{f91-tcc}

\begin{figure}[ht]
\begin{session}
f91_TCC5: OBLIGATION
  FORALL (i: nat,
          v: [i1:
               \setb{}z: nat |
                        (IF z > 101 THEN 0 ELSE 101 - z ENDIF) <
                         (IF i > 101 THEN 0 ELSE 101 - i ENDIF)\sete ->
               \setb{}j: nat | IF i1 > 100 THEN j = i1 - 10 ELSE j = 91 ENDIF\sete]):
    NOT i > 100 IMPLIES
     IF i > 100 THEN v(v(i + 11)) = i - 10 ELSE v(v(i + 11)) = 91 ENDIF;
\end{session}
\caption{Termination TCC for \texttt{f91}}\label{f91-tcc}
\end{figure}
where the inner calls to \texttt{f91} have been replaced by the
higher-order variable \texttt{v}, with the recursive signature as shown.
Since the obligation forces us to prove the termination condition for all
functions whose type is that of \texttt{f91}, it will also hold for
\texttt{f91}.  This example also illustrates the use of dependent types,
discussed in Section~\ref{dependent-types}.

\pvstheory{ackerman-alltt}{Theory \texttt{ackerman}}{ackerman-alltt}
\index{ackerman@{\texttt{ackerman}}}

\renewcommand{\textfraction}{.01}

In some cases the natural numbers are not a convenient measure; PVS
also provides the \texttt{ordinal}s, which allow recursion with measures up
to $\varepsilon_0$.  This is primarily useful in handling
lexicographical orderings.  For example, in the definition of the
Ackerman function in Figure~\ref{ackerman-alltt},\footnote{There are
ways of specifying \texttt{ackerman} using higher-order functionals, in
which case the measure is again on the natural numbers.} there are two
termination \tccs\ generated (along with a number of subtype \tccs).
The first termination \tcc\ is
\begin{pvsex}
  ack_TCC2:
    OBLIGATION
      (FORALL m, n:
        NOT m = 0 AND n = 0 IMPLIES ackmeas(m - 1, 1) < ackmeas(m, n))
\end{pvsex}
%
and corresponds to the first recursive call of \texttt{ack} in the body of
\texttt{ack}.  In this occurrence, it is known that \texttt{m $\neq$ 0}
and \texttt{n = 0}.  The remaining expression says that the measure
applied to the arguments of the recursive call to \texttt{ack} is less
than the measure applied to the initial arguments of \texttt{ack}.  Note
that the \texttt{<} in this expression is over the \texttt{ordinal}s, not
the \texttt{real}s.

\index{recursive definitions|)}


\section{Macros}\label{macro-declarations}
\index{macros|(}

There are some definitions that are convenient to use, but it's preferable
to have them expanded whenever they are referenced.  To some extent this
can be accomplished using auto-rewrites in the prover, but rewriting is
restricted.  In particular terms in types or actual parameters are not
rewritten; \texttt{typepred} and \texttt{same-name} must be used.  These
both require the terms to be given as arguments, making it difficult to
automate proofs.

The \texttt{MACRO} declaration is used to indicate definitions that are
expanded at typecheck time.  Macro declarations are normal constant
declarations, with the \texttt{MACRO} keyword preceding the
type.\footnote{This is similar to the \texttt{==} form of E\textsc{hdm}.}
For example, after the declaration
\begin{pvsex}
  N: MACRO nat = 100
\end{pvsex}
any reference to \texttt{N} is now automatically replaced by \texttt{100},
including such forms as \texttt{below[N]}.

Macros are not expanded until they have been typechecked.  This is because
the name overloading allowed by PVS precludes expanding during parsing.
TCCs are generated before the definition is expanded.
\index{macros|)}

% Master File: language.tex
\section{Inductive and Coinductive Definitions}
\label{inductive-definitions}
\index{inductive definition|(}

\emph{Inductive} definitions~\cite{Aczel:Handbook} are used frequently in
mathematics.  In general, some rules are given that generate elements of a
set, and the inductively defined set is the smallest set that contains
those elements.  The obvious example of an inductive definition is the
natural numbers, where the rules are given by Peano's axioms, with the
induction scheme ensuring that the natural numbers are the smallest set
containing $1$ and the successor of any natural number.  Language
definitions are another example.  Most logics have a notion of
\emph{formulas}, and these are usually defined inductively.

Paulson~\cite{paulson-fixedpoint} notes that this is simply a \emph{least
fixedpoint} with respect to a given domain of elements and a set of rules,
which is well-defined if the rules are \emph{monotonic}, by the
well known Knaster-Tarski theorem.  From this perspective, the greatest
fixedpoint also exists and corresponds to \emph{coinductive} definitions.
Inductive and coinductive definitions are similar to recursive
definitions, in that they have induction principles, and both must satisfy
additional constraints to guarantee that they are well defined.

We will describe inductive definitions first, as they are more familiar.
The even integers provide a simple example of an inductive
definition:\footnote{This is an alternative to the more traditional
definition of \texttt{even?} in the prelude.}
\begin{pvsex}
  even(n: int): INDUCTIVE bool = n = 0 OR even(n - 2) OR even(n + 2)
\end{pvsex}
With this definition, it is easy to prove, for example, that \texttt{0} or
\texttt{1000} are even, simply by expanding the definition enough
times.\footnote{In the latter case, \texttt{(apply (repeat (then (expand
"even") (flatten) (assert))))} is a good strategy to use, though it should
be used with care since it does not terminate on \texttt{even} applied to
anything other than an even numeral.}  More is needed, however, in proving
general facts, such as if $n$ is even, then $n+1$ is not even.  To deal
with these, we need a means of stating that an integer is even iff it is
so as a result of this definition.  In PVS, this is accomplished by the
automatic creation of two induction schemas, that may be viewed using the
\texttt{M-x~prettyprint-expanded} command:
\begin{session}
  even_weak_induction: AXIOM
    FORALL (P: [int -> boolean]):
      (FORALL (n: int): n = 0 OR P(n - 2) OR P(n + 2) IMPLIES P(n)) IMPLIES
       (FORALL (n: int): even(n) IMPLIES P(n));

  even_induction: AXIOM
    FORALL (P: [int -> boolean]):
      (FORALL (n: int):
         n = 0 OR even(n - 2) AND P(n - 2) OR even(n + 2) AND P(n + 2)
          IMPLIES P(n))
       IMPLIES (FORALL (n: int): even(n) IMPLIES P(n));
\end{session}
The weak induction axiom states that if \texttt{P} is another predicate
that satisfies the \texttt{even} form, then any \texttt{even} number
satisfies \texttt{P}.  Thus \texttt{even} is the smallest such \texttt{P}.
The second (strong) axiom allows the \texttt{even} predicate to be carried
along, which can make proofs easier.  These axioms are used by the
\texttt{rule-induct} strategy described in the Prover
Guide~\cite{PVS:prover}.

Inductive definitions are predicates, hence must be functions with
eventual range type \texttt{boolean}.  For example, in
\begin{session}
  f1(n,m:int) INDUCTIVE int = n
  f2(n,m:int)(x,y:int)(z:int): INDUCTIVE [int,int,int -> bool] =
      LAMBDA (a,b,c:int): n = m IMPLIES f2(n,m)(x,y)(z)(a,b,c)
\end{session}
\texttt{f1} is illegal, while \texttt{f2} returns a boolean value if
applied to enough arguments, hence is valid.

To be monotonic, every occurrence of the definition within the defining
body must be \emph{positive}.\index{positive occurrence} For this we need
to define the parity of an occurrence of a term in an expression $A$: If a
term occurs in $A$ with a given parity, then the occurrence retains its
parity in \texttt{$A$ AND $B$}, \texttt{$A$ OR $B$}, \texttt{$B$ IMPLIES
$A$}, \texttt{FORALL y:$A$}, \texttt{EXISTS y:$A$}, and reverses it in
\texttt{$A$ IMPLIES $B$} and \texttt{NOT $A$}.  Any other occurrence is of
unknown parity.

The parity of the inductive definition in the definition body is checked,
and if some occurrence of the definition is negative, a type error is
generated.  If some occurrence is of unknown parity, then a
\emph{monotonicity TCC}\index{TCC!monotonicity}\index{monotonicity TCC} is
generated.  For example, given the declarations
\begin{session}
  f: [nat, bool -> bool]
  G(n:nat): INDUCTIVE bool =
    n = 0 OR f(n, G(n-1))
\end{session}
the monotonicity TCC has the form
\begin{session}
  (FORALL (P1: [nat -> boolean], P2: [nat -> boolean]):
     (FORALL (x: nat): P1(x) IMPLIES P2(x))
         IMPLIES
       (FORALL (x: nat):
          x = 0 OR f(x, P1(x - 1)) IMPLIES x = 0 OR f(x, P2(x - 1))));
\end{session}

Inductive definitions act as constants for the most part, so they may be
expanded or used as rewrite rules in proofs.  However, they are not usable
as auto-rewrite rules, as there is no easy way to determine when to stop
rewriting.

To provide induction schemes in the most usable form, they are generated
as follows.  First, the variables in the definition are partitioned into
fixed\index{fixed inductive variable} and non-fixed variables.  For
example, in the transitive-reflexive closure
\begin{pvsex}
  TC(R)(x, y) : INDUCTIVE bool =
     R(x, y) OR (EXISTS z: TC(R)(x, z) AND TC(R)(z, y))
\end{pvsex}
\texttt{R} is fixed since every occurrence of \texttt{TC} has \texttt{R}
as an argument in exactly the same position, whereas \texttt{x} and
\texttt{y} are not fixed.  The induction is then over predicates $P$ that
take the non-fixed variables as arguments.  If the inductive definition is
defined for variable $V$ partitioned into fixed variables $F$, and
non-fixed variables $N$, the general form of the (weak) induction scheme
is
\begin{session}
  FORALL (\(F\), \(P\)):
   (FORALL (\(N\)):
     \emph{inductive_body}(\(N\))\([P/\emph{def}]\) IMPLIES \(P\)(\(N\)))
      IMPLIES
     (FORALL (\(N\)): \emph{def}(\(V\)) IMPLIES \(P\)(\(N\)))
\end{session}
In the case of \texttt{TC}, this becomes
\begin{session}
  TC_weak_induction: AXIOM
        (FORALL (R: relation, P: [[T, T] -> boolean]):
           (FORALL (x: T, y: T):
              R(x, y) OR (EXISTS z: (P(x, z) AND P(z, y))) IMPLIES P(x, y))
               IMPLIES (FORALL (x: T, y: T): TC(R)(x, y) IMPLIES P(x, y)));
\end{session}

\index{coinductive definitions(|}
Coinductive definitions have the same form as inductive definitions, but
are introduced with the keyword \texttt{COINDUCTIVE}, and generate the
greatest fix point, rather than the least fix point.  The monotonicity
conditions are the same, but the coinduction axioms reverse some of the
implications.  Thus the general form of the (weak) coinduction scheme is
\begin{session}
  FORALL (\(F\), \(P\)):
   (FORALL (\(N\)):
     \(P\)(\(N\)) IMPLIES \emph{coinductive_body}(\(N\))\([P/\emph{def}]\))
      IMPLIES
     (FORALL (\(N\)): \(P\)(\(N\)) IMPLIES \emph{def}(\(V\)))
\end{session}

As noted earlier, inductive and coinductive definitions are really
fixedpoint definitions.  For example, the theory in
Figure~\ref{inductive-fixpoints} shows that an
inductive definition is a least fixedpoint, a coinductive definition is a
greatest fixpoint, an inductively defined set is a subset of a
coindutively defined set, and, if the universe contains a non-wellfounded
element, then the coinductively defined set is strictly larger.  These
results all build on the definitions in  the \texttt{mucalculus} theory of
the prelude.

{\begin{figure}[htb]\begin{boxedminipage}{\textwidth}%
{\smaller\smaller\begin{alltt}
inductive_fixpoint: THEORY
 BEGIN
  N: TYPE+
  n, m: VAR N
  0: N
  S: [N -> N]
  Sax1: AXIOM 0 /= S(n)
  Sax2: AXIOM S(m) = S(n) => m = n
  % Assume a non-wellfounded element
  nwf_exists: AXIOM EXISTS n: n = S(n)

  Nind(n):     INDUCTIVE bool = n = 0 OR Nind(S(n))
  Ncoind(n): COINDUCTIVE bool = n = 0 OR Ncoind(S(n))

  % NN is the predicate transformer corresponding to the (co)inductive defs
  NN(p: pred[N])(n): bool = n = 0 OR p(S(n))

  % These use the lfp and gfp defs from the prelude mucalculus theory
  ind_lfp: FORMULA Nind = lfp(NN)
  coind_gfp: FORMULA Ncoind = gfp(NN)

  % Repeat Nind_weak_induction, which is proved from lfp_induction
  Nind_weak_induction_repeated: FORMULA 
    FORALL (P: [N -> boolean]):
      (FORALL n: n = 0 OR P(S(n)) IMPLIES P(n)) IMPLIES
       (FORALL n: Nind(n) IMPLIES P(n));

  % Inductive definitions are a subset of coinductive
  ind_sub_co: FORMULA Nind(n) => Ncoind(n)

  % Because there is a non-wellfounded element, we can show that
  % the coinductive set is larger.
  co_has_more: FORMULA EXISTS n: Ncoind(n) & NOT Nind(n)
 END inductive_fixpoint
\end{alltt}}\end{boxedminipage}%
\caption{Inductive definitions and fixpoints}\label{inductive-fixpoints}\end{figure}}

\index{coinductive definition|)}
\index{inductive definition|)}


\section{Formula Declarations}\label{formula-declarations}
\index{formula declarations|(}\index{declaration!formulas|(}

Formula declarations introduce \emph{axioms}\index{axioms},
\emph{assumptions}\index{assumptions}, \emph{theorems}\index{theorems},
and \emph{obligations}\index{obligations}.  The identifier associated with
the declaration may be referenced in auto-rewrite declarations (see
Section~\ref{auto-rewrite-decls} and in proofs (see the \texttt{lemma} command
in the PVS Prover Guide~\cite{PVS:prover}).  The expression that makes up
the body of the formula is a boolean expression.  Axioms, assumptions, and
obligations are introduced with the keywords \texttt{AXIOM},
\texttt{ASSUMPTION}, and \texttt{OBLIGA\-TION}, respectively.  Axioms may
also be introduced using the keyword \texttt{POSTULATE}\index{postulate}.
In the prelude postulates are used to indicate axioms that are provable by
the decision procedures, but not from other axioms.  Theorems may be
introduced with any of the keywords
\texttt{CHALLENGE}\index{claim@{\texttt{CHALLENGE}}},
\texttt{CLAIM}\index{claim@{\texttt{CLAIM}}},
\texttt{CONJECTURE}\index{conjecture@{\texttt{CONJECTURE}}},
\texttt{COROLLARY}\index{corollary@{\texttt{COROLLARY}}},
\texttt{FACT}\index{fact@{\texttt{FACT}}},
\texttt{FORMULA}\index{formula@{\texttt{FORMULA}}},
\texttt{LAW}\index{law@{\texttt{LAW}}},
\texttt{LEMMA}\index{lemma@{\texttt{LEMMA}}},
\texttt{PROPOSITION}\index{proposition@{\texttt{PROPOSITION}}},
\texttt{SUBLEMMA}\index{sublemma@{\texttt{SUBLEMMA}}}, or
\texttt{THEOREM}\index{theorem@{\texttt{THEOREM}}}.

Assumptions are only allowed in assuming clauses (see
Section~\ref{assuming}).  Obligations are generated by the system for
\tccs, and cannot be specified by the user.  Axioms are treated
specially when a proof is analyzed, in that they are not expected to
have an associated proof.  Otherwise they are treated exactly like
theorems.  All the keywords associated with theorems have the same
semantics, they are there simply to allow for greater diversity in
classifying formulas.

Formula declarations may contain free variables\index{free variables}, in
which case they are equivalent to the universal closure\index{universal
closure} of the formula.\footnote{The universal closure of a formula is
obtained by surrounding the formula with a \texttt{FORALL} binding
operator whose bindings are the free variables of the formula.  For
example, the universal closure of \texttt{p(x,y) => q(z)} is
\texttt{(FORALL x,y,z:\ p(x,y) => q(z))} (assuming \texttt{x}, \texttt{y}
and \texttt{z} resolve to variables).} In fact, the prover actually uses
the universal closure when it introduces a formula to a proof.  Formula
declarations are the only declarations in which free variables are
allowed.

\index{declaration!formulas|)}\index{formula declarations|)}

% Document Type: LaTeX
% Master File: language.tex
\section{Judgements}
\label{judgements}\index{judgements|(}

The facility for defining predicate subtypes is one of the most useful
features provided by PVS, but it can lead to a lot of redundant TCCs.
\emph{Judgements}\footnote{We prefer this spelling, though many spell
checkers do not.} provide a means for controlling this by allowing
properties of operators on subtypes to be made available to the
typechecker.  There are several kinds of judgements available in PVS\@. 
Most of them indicate that an expression belongs to a given type, but the
\emph{subtype judgement} indicates that two types are in the subtype
relation.
\begin{description}
\item[Number judgement] - 
\item[Name judgement]
\item[Application judgement]
\item[Recursive judgement]
\item[Expression judgement]
\item[Subtype judgement]
\end{description}


The \emph{constant judgement}\index{constant judgement} states that a
particular constant (or number) has a type more specific than its declared
type.  The \emph{subtype judgement}\index{subtype judgement} states that
one type is a subtype of another.

\subsection{Number and Name Judgements}
\index{number-name-judgement}

Number and name judgements 

There are two kinds of constant judgements.  The simpler kind 
states that a constant or number belongs to a type different than its
declared type.\footnote{Remember that all numbers are implicitly declared to
be of type \texttt{real}.}  For example, the constant judgement
declaration
\begin{pvsex}
  JUDGEMENT c, 17 HAS_TYPE (prime?)
\end{pvsex}
simply states that the constant \texttt{c} and the number \texttt{17} are
both prime numbers.  This declaration leads to the TCC formulas
\texttt{prime?(c)} and \texttt{prime?(17)}, but in any context in which
this declaration is visible, the use of \texttt{c} or \texttt{17} where a
prime is expected will not generate TCCs.  Thus no TCCs are generated for
the formula \texttt{F} in
\begin{pvsex}
  RP: [(prime?), (prime?) -> bool]
  F: FORMULA RP(c, 17) IMPLIES RP(17, c)
\end{pvsex}

The second kind of constant judgement is for functions; argument types are
provided and the judgement states that when the function is applied to
arguments of the given types, then the result has the type following the
\texttt{HAS\_TYPE} keyword.  Here is an example that illustrates the need
for this kind of judgement:
\begin{pvsex}
  x, y: VAR real
  f(x,y): real = x*x - y*y
  n: int = IF f(1,2) > 0 THEN f(4,3) ELSE f(3,2) ENDIF
\end{pvsex}
This leads to two TCCs:
\begin{pvsex}
  n_TCC1: OBLIGATION
    f(1, 2) > 0 IMPLIES
      rational_pred(f(4, 3)) AND integer_pred(f(4, 3))
  n_TCC2: OBLIGATION
    NOT f(1, 2) > 0 IMPLIES
      rational_pred(f(3, 2)) AND integer_pred(f(3, 2))
\end{pvsex}
The problem here is that although we know that \texttt{f} is closed under
the integers, the typechecker does not.  If \texttt{f} is heavily used,
dealing with these TCCs becomes cumbersome.  We can try the \emph{ad hoc}
solution of adding new overloaded declarations for \texttt{f}:
\begin{pvsex}
  i, j: VAR nat
  f(i, j): int = f(i, j)
\end{pvsex}
But now proofs require an extra definition expansion, and such overloading
leads to confusion.\footnote{This is one of the motivations for providing
the \texttt{M-x~show-expanded-sequent} command.}  A more elegant solution
is to use a judgement declaration:
\begin{pvsex}
  f_int_is_int: JUDGEMENT f(i, j: int) HAS\_TYPE int
\end{pvsex}
This generates the TCC
\begin{pvsex}
  f_int_is_int: FORALL (x:int, y:int):
                  rational_pred(f(x, y)) AND integer_pred(f(x, y))
\end{pvsex}
But now the declaration of \texttt{n} given above generates \emph{no}
TCCs, as the typechecker ``knows'' that \texttt{f} is closed on the
integers.  Note that this is different than the simple judgement
\begin{pvsex}
  f_int: JUDGEMENT f HAS\_TYPE [int, int -> int]
\end{pvsex}
In this case, the TCC generated is unprovable:
\begin{pvsex}
  f_int: OBLIGATION
    ((FORALL (x: real): rational_pred(x) AND integer_pred(x)) AND
      (FORALL (x: real): rational_pred(x) AND integer_pred(x)))
     AND
     (FORALL (x1: [real, real]):
        rational_pred(f(x1)) AND integer_pred(f(x1)));
\end{pvsex}
A warning is generated when simple constant judgements are declared to be
of a function type.\footnote{Earlier versions of PVS simply interpreted
this form as a closure condition, but this is less flexible.}  In
addition, this judgement will not help with the declaration \texttt{n}
above; it can only be used in higher-order functions, for example:
\begin{pvsex}
  F: [[int, int -> int] -> bool]
  FF: FORMULA F(f)
\end{pvsex}

The arguments for a function judgement follow the syntax for function
declarations; so a curried function may be given multiple judgements:
\begin{pvsex}
  f(x, y: real)(z: real): real
  f_closed: JUDGEMENT f(x, y: nat)(z: int) HAS\_TYPE int
  f2_closed: JUDGEMENT f(x, y: int) HAS\_TYPE [real -> int]
\end{pvsex}

If a constant judgement declaration specifies a name, it must refer to a
unique constant and its type must be compatible with the type expression
following the \texttt{HAS\_TYPE} keyword.  If it is a number, then its
type must be compatible with the \texttt{number} type.

Constant judgements generally lead to TCCs.  If no TCC is generated, then
the judgement is not actually needed, and a warning to this effect is
produced.  Simple (non-functional) constant judgements generate TCCs
indicating that the constant belongs to the specified type.  Constant
function judgements generate TCCs that reflect closure conditions.

The judgement facility cannot be used to remove all redundant TCCs; the
variables used for arguments must be unique, and full expressions may not
be included.  Hence the following are not legal:
\begin{pvsex}
  x: VAR real
  x_times_x_is_nonneg: JUDGEMENT *(x, x) HAS\_TYPE nonneg_real
  c: real
  x_times_c_is_even: JUDGEMENT *(x, c) HAS\_TYPE (even?)
\end{pvsex}


\subsection{Subtype Judgements}
\index{subtype judgement}

The subtype judgement is used to fill in edges of the subtype graph that
otherwise are unknown to the typechecker.  For example, consider the
following declarations:
\begin{pvsex}
  nonzero_real: NONEMPTY_TYPE = \setb{}r: real | r /= 0\sete CONTAINING 1
  rational: NONEMPTY_TYPE FROM real
  nonneg_rat: NONEMPTY_TYPE = \setb{}r: rational | r >= 0\sete CONTAINING 0
  posrat: NONEMPTY_TYPE = \setb{}r: nonneg_rat | r > 0\sete  CONTAINING 1
  /: [real, nonzero_real -> real]
\end{pvsex}
For \texttt{r} of type \texttt{real} and \texttt{q} of type
\texttt{posrat}, the expression \texttt{r/q} leads to the TCC \texttt{q
/= 0}.  One solution, if \texttt{q} is a constant, is to use a constant
judgement as described above.  But if there are many constants involving
the type \texttt{posrat}, this requires a lot of judgement declarations,
and does not help at all for variables or compound expressions.  The
subtype judgement solves this by stating that \texttt{posrat} is a subtype
of \texttt{nzrat}.  Another subtype judgement states that \texttt{nzrat}
is a subtype of \texttt{nzreal}:
\begin{pvsex}
  JUDGEMENT posrat SUBTYPE_OF nzrat
  JUDGEMENT nzrat SUBTYPE_OF nzreal
\end{pvsex}
With these judgements, TCCs will not be generated for any denominator that
is of type \texttt{posrat}.  With the (prelude) judgement declarations
\begin{pvsex}
  nnrat_plus_posrat_is_posrat:   JUDGEMENT +(nnx, py) HAS_TYPE posrat
  posrat_times_posrat_is_posrat: JUDGEMENT *(px, py)  HAS_TYPE posrat
\end{pvsex}
not only are there no TCCs generated for \texttt{r/q}, but none are
generated for \texttt{r/(q + 2)}, \texttt{r/((q + 2) * q)}, etc.

Given a subtype judgement declaration of the form
\begin{pvsex}
  JUDGEMENT S SUBTYPE_OF T
\end{pvsex}
it is an error if \texttt{S} is already known to be a subtype of
\texttt{T}, or if they are not compatible.  Otherwise, \texttt{T} must be
of the form \texttt{\setb{}x:\ ST | p(x)\sete}, where \texttt{ST} is the least
compatible type of \texttt{S} and \texttt{T}, and a TCC will be generated
of the form \texttt{FORALL (x:S): p(x)}.  Remember that subtyping on
functions only works on range types, so the subtype judgement
\begin{pvsex}
  JUDGEMENT [nat -> nat] SUBTYPE_OF [int -> int]
\end{pvsex}
leads to the unprovable TCC
\begin{pvsex}
FORALL (x1:nat, y1:int): y1 >= 0 AND TRUE
\end{pvsex}

\subsection{Judgement Processing}

When a judgement declaration is typechecked, TCCs are generated as
explained above and the judgement is added to the current context for use
in typechecking expressions.  The typechecker typechecks expressions in
two passes; in the first pass it simply collects possible types for
subexpressions, and in the second pass it recursively tries to determine a
unique type based on the expected type, and generates TCCs accordingly;
this is where judgements are used.  If the expression is a constant (name
or number), then all non-functional judgements are collected for that
constant and used to generate a minimal TCC relative to the expected type.

If it is an application whose operator is a name, then functional
judgements of the corresponding arity are collected for the operator, and
those judgements for which the application arguments are all known to be
of the corresponding judgement argument types are extracted, and a minimal
TCC is generated from these.

In addition to inhibiting the generation of TCCs during typechecking,
judgements are also important to the prover; they are used explicitly in
the \texttt{typepred} command, and implicitly in \texttt{assert}, where
the judgement type information is provided to the ground decision
procedures.

Subtype judgements are used in determining when one type is a subtype of
another, which is tested frequently during typechecking and proving,
including in the test on argument types described above.

\index{judgements|)}


% Document Type: LaTeX
% Master File: language.tex
\section{Conversions}
\label{coercion-decls}\index{conversions|(}

Conversions are functions that the typechecker can insert automatically
whenever there is a type mismatch.  They are similar to the implicit
coercions for converting integers to floating point used in many
programming languages.  PVS provides some builtin conversions in the
prelude, but conversions may also be provided by the user using
\emph{conversion declarations}.  A conversion declaration consists of the
keyword \texttt{CONVERSION}, optionally followed by `\texttt{+}' or
`\texttt{-}' and an expression.  \texttt{CONVERSION+} is equivalent to
\texttt{CONVERSION}.  The expression must be of type a (subtype of) a
function type, where the domain and range are not compatible.  This is
because conversions are only triggered when there would otherwise be a
type error, and compatible types may lead to unproveable TCCs, but not to
type errors.  Judgements are the proper way to control the generation of
TCCs, see Section~\ref{judgements} for details.

\subsection{Conversion Examples}
\label{conversion-examples}

Here is a simple example.
\begin{pvsex}
  c: [int -> bool]
  CONVERSION c
  two: FORMULA 2
\end{pvsex}
Here, since formulas must be of type boolean, the typechecker
automatically invokes the conversion and changes the formula to
\texttt{c(2)}.  This is done internally, and is only visible to the user
on explicit command\footnote{The \texttt{M-x~prettyprint-expanded} command.}
and in the proof checker.

A more complex conversion is illustrated in the following example.
\begin{pvsex}
  g: [int -> int]
  F: [[nat -> int] -> bool]
  F_app: FORMULA F(g)
\end{pvsex}
As this stands, \texttt{F\_app} is not type-correct, because a function of
type \texttt{[int -> int]} is supplied where one of type \texttt{[nat ->
int]} is required, and PVS requires equality on domain types for function
types to be compatible.  However it is clear that \texttt{g} naturally
induces a function from \texttt{nat} to \texttt{int} by simply restricting
its domain.  Such a domain restriction is achieved by the
\texttt{restrict} conversion\index{restrict
conversion}\index{conversion!restrict} that is defined in the PVS prelude
as follows:
\begin{session}
  restrict [T: TYPE, S: TYPE FROM T, R: TYPE]: THEORY
   BEGIN
    f: VAR [T -> R]
    s: VAR S
    restrict(f)(s): R = f(s)
    CONVERSION restrict
   END restrict
\end{session}
The construction \texttt{S: TYPE FROM T} specifies that the actual
parameter supplied for \texttt{S} must be a subtype of the one supplied
for \texttt{T}.  The specification states that \texttt{restrict(f)} is a
function from \texttt{S} to \texttt{R} whose values agree with \texttt{f}
(which is defined on the larger domain \texttt{T}).  Using this approach,
a type correct version of \texttt{F\_app} can be written as
\texttt{F(restrict[int,nat,int](g))}.  This provides the convenience of
contravariant subtyping, but without the inherent complexity (in
particular, with contravariant subtyping the type of equality must be
correct in substituting equals for equals, making proofs less
perspicuous).

It is not so obvious how to expand the domain of a function in the general
case, so this approach does not work automatically in the other direction.
It does, however, work well for the important special case of sets (or,
equivalently, predicates): a set on some type \texttt{S} can be extended
naturally to one on a supertype \texttt{T} by assuming that the members of
the type-extended set are just those of the original set.  Thus, if
\texttt{extend(s)} is the type-extended version of the original set
\texttt{s}, we have \texttt{extend(s)(x) = s(x)} if \texttt{x} is in the
subtype \texttt{S}, and \texttt{extend(s)(x) = false} otherwise.  We can
say that \texttt{false} is the ``default'' value for the type-extended
function.  Building on this idea, we arrive at the following specification
for a general type-extension function.
\begin{session}
  extend [T: TYPE, S: TYPE FROM T, R: TYPE, d: R]: THEORY
   BEGIN
    f: VAR [S -> R]
    t: VAR T
    extend(f)(t): R = IF S_pred(t) THEN f(t) ELSE d ENDIF
   END extend
\end{session}
The function \texttt{extend(f)} has type \texttt{[T -> R]} and is
constructed from the function \texttt{f} of type \texttt{ [S -> R]} (where
\texttt{S} is a subtype of \texttt{T}) by supplying the default value
\texttt{d} whenever its argument is not in \texttt{S} (\texttt{S\_pred} is
the {\em recognizer\/} predicate for \texttt{ S}).  Because of the need to
supply the default \texttt{d}, this construction cannot be applied
automatically as a conversion.  However, as noted above, \texttt{false} is
a natural default for functions with range type \texttt{bool} (i.e., sets
and predicates), and the following theory establishes the corresponding
conversion.\index{extend\_bool conversion}\index{conversion!extend\_bool}
\begin{session}
extend_bool [T: TYPE, S: TYPE FROM T]: THEORY
 BEGIN
  CONVERSION extend[T, S, bool, false]
 END extend_bool
\end{session}
In the presence of this conversion, the type-incorrect formula
\texttt{B\_app} in the following specification
\begin{pvsex}
  b: [nat -> bool]
  B: [[int -> bool] -> bool]
  B_app: FORMULA B(b)
\end{pvsex}
is automatically transformed to \texttt{B(extend[int,nat,bool,false](b))}.

\subsection{Lambda conversions}\label{lambda-conversion}
\index{lambda conversion}\index{conversion!lambda}

Conversions are also useful (for example, in semantic encodings of dynamic
or temporal logics) in ``lifting'' operations to apply pointwise to
sequences over their argument types.  Here is an example, where
\texttt{state} is an uninterpreted (nonempty) type, and a state variable
\texttt{v} of type real is represented as a constant of type
\texttt{[state -> real]}.
\begin{session}
  th: THEORY
   BEGIN
    CONVERSION+ K_conversion
    state: TYPE+
    l: [state -> list[int]]
    x: [state -> real]
    b: [state -> bool]
    bv: VAR [state -> bool]
    s: VAR state
    box(bv): bool = FORALL s: bv(s)
    F1: FORMULA box(x > 1)
    F2: FORMULA box(b IMPLIES length(l) + 3 > x)
   END th
\end{session}
In this example, the formulas \texttt{F1} and \texttt{F2} are not type
correct as they stand, but with a \emph{lambda conversion},\index{lambda
conversion}\index{conversion!lambda} triggered by the
\texttt{K\_conversion} in the PVS prelude, these formulas are converted to
the forms
\begin{session}
  F1: FORMULA box(LAMBDA (x1: state): x(x1) > 1)
  
  F2: FORMULA
    box(LAMBDA (x3: state):
          b(x3) IMPLIES
           (LAMBDA (x2: state):
              (LAMBDA (x1: state):
                 (LAMBDA (x: state): length(l(x)))(x1) + 3)
                (x2)
             > x(x2))(x3))
\end{session}

\subsection{Conversions on Type Constructors}\label{type-conversions}
\index{conversions!type constructor}\index{type constructor conversiona}
\index{componentwise conversions}

Conversions for record, tuple, and function types may be found
componentwise, without having to create the corresponding conversion
declaration.  Here is an example.
\begin{session}
  bi: [bool -> int]
  ib: [int -> bool]
  CONVERSION+ bi, ib
  t: [int, int, int] = (true, false, 3)
  r: [# a, b: int #] = (# a := true, b := false #)
  f: [int, int -> int] = AND
\end{session}
With conversions displayed, this becomes the following.
\begin{session}
  t: [int, int, int] = (b2n(TRUE), b2n(FALSE), 3)

  r: [# a: int, b: int #] =
      (LAMBDA (x: [# a: bool, b: bool #]): (# a := bi(x`a), b := bi(x`b) #))
          ((# a := TRUE, b := FALSE #))

  f: [int, int -> int] =
      (LAMBDA (f: [[bool, bool] -> bool]):
         LAMBDA (x: [int, int]): bi(f(ib(x`1), ib(x`2))))
          (AND)
\end{session}
Note that for \texttt{f}, both a tuple conversion and a function
conversion are used.


\subsection{Conversion Processing}

In general, conversions are applied by the typechecker whenever it would
otherwise emit a type error.  In the simplest case, if an expression
\texttt{e} of type \texttt{T$_1$} occurs where an incompatible type
\texttt{T$_2$} is expected, the most recent compatible conversion
\texttt{C} is found in the context and the occurrence of \texttt{e} is
replaced by \texttt{C(e)}.  \texttt{C} is compatible if its type is
\texttt{[D -> R]}, where \texttt{D} is compatible with \texttt{T$_1$} and
\texttt{R} is compatible with \texttt{T$_2$}.

Conversions are ordered in the context; if multiple compatible conversions
are available,  the most recently declared conversion is used.  Hence, in

\begin{pvsex}
  CONVERSION c1
  \(\cdots\)
  IMPORTING th1, th2
  \(\cdots\)
  CONVERSION c2
  \(\cdots\)
  F: FORMULA 2
\end{pvsex}

For formula \texttt{F}, \texttt{c2} is the most recent conversion,
followed by the conversions in theory \texttt{th2}, those in \texttt{th1},
and finally \texttt{c1}.  Note that the relative order of the constant
declarations (e.g., \texttt{c1} and \texttt{c2} above) doesn't matter,
only the \texttt{CONVERSION} declarations.

When conversions are available on either the argument(s) or the operator
of an application, the arguments get precedence.

For an application \texttt{e(x$_1$, \ldots, x$_n$)} the possible types of
the operator \texttt{e}, and the arguments \texttt{x$_i$} are determined,
and for each operator type \texttt{[D$_1$, \ldots, D$_n$ -> R]} and
argument type \texttt{T$_i$}, if \texttt{D$_i$} is not compatible with
\texttt{T$_i$}, conversions of type \texttt{[T$_i$ -> D$_i$]} are
collected.  If such conversions are found for every argument that doesn't
have a compatible type, then those conversions are applied.  Otherwise an
operator conversion is looked for.

Note that compositions of conversion are never searched for, as this would
slow down processing too much.  If you want to use a composition, include
a conversion declaration for it.  Here is an example:
\begin{session}
  T1, T2, T3: TYPE+
  f1: [T1 -> T2]
  f2: [T2 -> T3]
  x: T1
  g: [T3 -> bool]
  CONVERSION f1, f2
  F1: FORMULA g(x)
  CONVERSION f2 o f1
  F2: FORMULA g(x)
\end{session}
In this example, \texttt{F1} leads to a type error, but when we make the
composition a conversion, the same expression in \texttt{F2} applies the
conversion rather than give a type error.

\subsection{Conversion Control}

As stated above, conversions are only applied when typechecking otherwise
fails.  In some cases, a conversion can allow a specification to
typecheck, but the meaning is different than what was intended.  This is
most likely for the \texttt{K\_conversion}, which was introduced when the
\texttt{mucalculus} theory was added to the prelude in support of the
model checker.  When a conversion is applied that fact is noted as a
message, and may be viewed using the \texttt{show-theory-messages}
command.  However, these messages are easily overlooked, so instead PVS
allows finer control over conversions.

Thus in addition to the \texttt{CONVERSION} form, the \texttt{CONVERSION-}
form is available allowing conversions to be turned off.  For uniformity,
the \texttt{CONVERSION+} form is also available as an alias for
\texttt{CONVERSION}.  \texttt{CONVERSION-} disables conversions.

The following theory illustrates the idea:
\begin{session}
  t1: THEORY
  BEGIN
   c: [int -> bool]
   CONVERSION+ c
   f1: FORMULA 3
   CONVERSION- c
   f2: FORMULA 3
  END t1
\end{session}
Here \texttt{f2} leads to a type error.

Another example is provided by the definition of the CTL temporal
operators in the prelude theory \texttt{ctlops}, which are surrounded by
\texttt{CONVERSION+} and \texttt{CONVERSION-} declarations that first
enable the \texttt{K\_conversion} then disable it at the end of the
theory.  All other conversions declared in the prelude remain enabled.
They may be disabled within any theory by using the \texttt{CONVERSION-}
form.

When theories containing conversion declarations are imported, the
conversions are imported as well.  Thus if \texttt{t2} enables the
\texttt{c} declaration without subsequently disabling it, then
\texttt{IMPORTING t1, t2} would enable the conversion, but
\texttt{IMPORTING t2, t1} would leave it disabled.

Conversion declarations may be generic or instantiated.  This
allows, for example, enabling the generic form of a conversion while
disabling particular instances.

\index{conversions|)}


\section{Library Declarations}
\label{library-decls}
\index{library declaration|(}
\index{declaration!library|(}

Library declarations are used to introduce a new PVS context\index{PVS Context} into a
specification.  Thus a specification may be developed in one context, and
used in many other contexts.  This provides more flexibility, at the cost
of less portability.  Any PVS context other than the current one may be
considered a library.  An example of a library declaration
is\index{LIBRARY@{\tt LIBRARY}}
\begin{pvsex}
  lib: LIBRARY = "~/pvs/protocols"
\end{pvsex}
When encountered, the system verifies that the directory specified within
the quotation marks exists, and that it has a PVS context file
\index{.pvscontext@{\tt .pvscontext}}%
(\texttt{.pvscontext}).  The library declaration is made use of by
including the library id in an importing name:\index{IMPORTING@{\tt IMPORTING}}
\begin{pvsex}
  IMPORTING lib@sliding_window[n]
\end{pvsex}
This has the effect of bringing in the \texttt{sliding\_window} theory,
exactly as if the theory belonged to the current context.

There are several libraries distributed with PVS, in the directory {\tt lib}.
It is not necessary to give a library declaration for libraries in this
directory, as it will be automatically searched for library importings.
For example, to import the finite sets library over the natural numbers:
\begin{pvsex}
  IMPORTING finite\_sets@finite\_sets[nat]
\end{pvsex}
An alternative approach (described in the \emph{PVS User
Guide}\cite{PVS:userguide}) is to use the {\tt M-x load-prelude-library},
which augments the PVS prelude with the the theories from a given context.

\index{declaration!library|)}
\index{library declaration|)}



\section{Auto-rewrite Declarations}
\label{auto-rewrite-decls}\index{auto-rewrites|(}

One of the problems with writing useful theories or libraries is that
there is no easy way to convey how the theory is to be used, other than in
comments or documentation.  In particular, the specifier of a theory
usually knows which lemmas should always be used as rewrites, and which
should never appear as rewrites.  Auto-rewrite declarations allow for both
forms of control.  Those that should always be used as rewrites are
declared with the \texttt{AUTO\_REWRITE+} or \texttt{AUTO\_REWRITE}
keyword, and those that should not are declared with
\texttt{AUTO\_REWRITE-}.  These will be referred to as
\emph{auto-rewrites} and \emph{stop-auto-rewrites} below.

When a proof is initiated for a given formula, all of the auto-rewrite
names in the current context that haven't subsequently been removed by
stop-auto-rewrite declarations are collected and added to the initial proof
state.  The stop-auto-rewrite declaration, in addition to removing
auto-rewrite names, also affects the following commands described in the
Prover manual.
\begin{itemize}
\setlength{\itemsep}{-5pt}
\item \texttt{auto-rewrite-theory},
\item \texttt{auto-rewrite-theories},
\item \texttt{auto-rewrite-theory-with-importings},
\item \texttt{simplify-with-rewrites},
\item \texttt{auto\-rewrite-defs},
\item \texttt{install-rewrites},
\item \texttt{auto-rewrite-explicit},
\item \texttt{grind},
\item \texttt{induct\-and-simplify},
\item \texttt{measure-induct-and-simplify}, and
\item \texttt{model-check}
\end{itemize}
These commands collect all definitions and formulas except those that
appear in \texttt{AUTO\_\-REWRITE-} declarations.  Thus suppose a theory
\texttt{T} contains the lemmas \texttt{lem1}, \texttt{lem2}, and
\texttt{lem3} and the declarations
\begin{alltt}
  AUTO_REWRITE+ lem1
  AUTO_REWRITE- lem3
\end{alltt}
Then in proving a formula of a theory that imports \texttt{T},
\texttt{lem1} is initially an auto-rewrite, and the command
\texttt{(auto-rewrite-theory "T")} will additionally install
\texttt{lem2}.  To auto-rewrite with \texttt{lem3}, simply use
\texttt{(auto-rewrite "lem3")}.  To exclude \texttt{lem1}, use
\texttt{(stop-auto-rewrite "lem1")} or \texttt{(auto-rewrite-theory "T"
:exclude "lem1")}.

The \texttt{autorewrites} theory shows a simple example.
\begin{session}
autorewrites: THEORY
BEGIN
 AUTO_REWRITE+ zero_times3
 a, b: real
 f1: FORMULA a * b = 0 AND a /= 0 IMPLIES b = 0
 AUTO_REWRITE- zero_times3
 f2: FORMULA a * b = 0 AND a /= 0 IMPLIES b = 0
END autorewrites
\end{session}
Here \texttt{f1} may be proved using only \texttt{assert}, but \texttt{f2}
requires more.

Rewrite names may have suffixes, for example, \texttt{foo!} or
\texttt{foo!!}.  Without the suffix, the rewrite is \emph{lazy}, meaning
that the rewrite will only take place if conditions and TCCs simplify to
true.  A condition in this case is a top-level \texttt{IF} or
\texttt{CASES} expression.  With a single exclamation point the
auto-rewrite is \emph{eager}, in which case the conditions are irrelevant,
though if it is a function definition it must have all arguments supplied.
With two exclamation points it is a \emph{macro} rewrite, and terms will
be rewritten even if not all arguments are provided.  See the prover guide
for more details; the notation is derived from the prover commands
\texttt{auto-rewrite}, \texttt{auto-rewrite!}, and
\texttt{auto-rewrite!!}.

In addition, a rewrite name may be disambiguated by stating that it is a
formula, or giving its type if it is a constant.  Without this any
definition or lemma in the context with the same name will be installed as
an auto-rewrite.

In order to be more uniform, these new forms of name are also available
for the \texttt{auto-rewrite} prover commands.  Thus the command
\begin{alltt}
  (auto-rewrite "A" ("B" "-2") "C" (("1" "D")))
\end{alltt}
may now be given instead as
\begin{alltt}
  (auto-rewrite "A" "B!" "-2!" "C" "1!!" "D!!")
\end{alltt}
The older form is still allowed, but is deprecated, and may not be mixed
with the new form.  Notice that in the auto-rewrite commands formula
numbers may also be used, and these may be followed by exclamation points,
but not by a formula keyword or type.

\index{auto-rewrites|)}


\index{declaration|)}

%%% Local Variables: 
%%% mode: latex
%%% TeX-master: "language"
%%% End: 

% Document Type: LaTeX
% Master File: language.tex

\chapter{Types}\label{types}
\index{type|(pidx}

PVS specifications are strongly typed, meaning that every expression has
an associated type (although it need not be unique, more on this later).
The PVS type system is based on \emph{structural
equivalence}\index{structural equivalence} instead of \emph{name
equivalence}\index{name equivalence}, so types are closely related to
sets, in that two types are equal iff they have the same elements.
Section~\ref{type-declarations} describes the introduction of type names,
which are the simplest type expressions.  More complex type
expressions\index{type expressions} are built from these using \emph{type
constructors}\index{type constructors}.  There are type constructors for
\emph{subtypes}\index{subtypes}\index{type!subtype}, \emph{function
types}\index{function types}\index{type!function}, \emph{tuple
types}\index{tuple types}\index{type!tuple}, \emph{cotuple
types}\index{cotuple types}\index{type!cotuple}, and \emph{record
types}\index{record types}\index{type!record}.  Function, record, and
tuple types may also be \emph{dependent}\index{dependent
types}\index{type!dependent}.  A form of \emph{type
application}\index{type application}\index{type!application} is provided
that makes it more convenient to specify parameterized subtypes.  There
are also provisions for creating \emph{abstract datatypes}, described in
Chapter~\ref{datatypes}.

Type expressions occur throughout a specification; in particular, they may
appear in theory parameters, type declarations, variable declarations,
constant declarations, recursive and inductive definitions, conversions,
and judgements.  In addition, they may appear in certain expressions
(coercions and local bindings, see pages~\pageref{coercions}
and~\pageref{binding-expressions}, respectively), and as actual parameters
in names (page~\pageref{names}).  In the many examples which follow, type
expressions will be presented in the context of type declarations; but it
must be remembered that they can appear in any of the above places.

\pvsbnf{bnf-type-expr}{Type Expression Syntax}

\section{Subtypes}\label{subtypes}
\index{subtypes|(}\index{type!subtype|(}

Any collection of elements of a given type itself forms a type, called a
\emph{subtype}.  The type from which the elements are taken is called the
\emph{supertype}\index{supertype}.  The elements which form the subtype
are determined by a \emph{subtype predicate}\index{subtype predicate} on
the supertype.

Subtypes in PVS provide much of the expressive power of the language,
at the cost of making typechecking undecidable.  There are two forms of
subtypes.  The first is similar to the notation used to define a set:
\begin{pvsex}
  t: TYPE = \setb{}x: s | p(x)\sete
\end{pvsex}
%
where \texttt{p} is a predicate on the type \texttt{s}.\footnote{If \texttt{x}
has been previously declared as a variable of type \texttt{s}, then the
``\texttt{:~s}'' may be omitted.} This has the usual set-theoretical
meaning, since types in PVS are modeled as sets.  Subtypes may also
be presented in an abbreviated form, by giving a predicate surrounded by
parentheses:
\begin{pvsex}
  t: TYPE = (p)
\end{pvsex}
%
This is equivalent to the form above.

Note that if the predicate \texttt{p} is everywhere false, then the type
is empty.  PVS supports empty types\index{empty type}\index{type!empty},
and the term \emph{type} is used to refer to any type, including the empty
type.  This is discussed in Section~\ref{type-declarations} (page~\pageref{type-declarations}).

Subtypes tend to make specifications more succinct and easier to read.
For example, in a specification such as
\begin{pvsex}
  FORALL (i:int):
    (i >= 0 IMPLIES (EXISTS (j:int): j >= 0 AND j > i))
\end{pvsex}
it is much more difficult to see what is being stated than in the
equivalent
\begin{pvsex}
  FORALL (i:nat): (EXISTS (j:nat): j > i))
\end{pvsex}
%
where \texttt{nat} is defined in the prelude as
\begin{pvsex}
  naturalnumber: NONEMPTY\_TYPE = \setb{}i:integer | i >= 0\sete CONTAINING 0
  nat: NONEMPTY\_TYPE = naturalnumber
\end{pvsex}

Subtype constructors consist of a \emph{supertype}\index{supertype} and a
\emph{subtype predicate}\index{subtype predicate} on the supertype.  The
primary property of a subtype is that any element which belongs to the
subtype automatically belongs to the supertype.  In addition, functions
defined on a type automatically apply to the subtype.

\index{TCC|(}

There are two \emph{type-correctness conditions} (\tccs) associated with
subtypes.  The first concerns \emph{empty types}\index{empty
type}\index{type!empty} as described in section~\ref{emptytypes}.  The
second \tcc\ associated with subtypes is the \emph{subtype}
\tcc,\index{subtype TCC}\index{TCC!subtype}, which comes about from the
use of operations defined on subtypes that are applied to elements of the
supertype.  By this means partial functions may be handled directly,
without recourse to a partial term logic or some form of multi-valued
logic.  For instance, division in PVS is a total function, with signature
\texttt{[real, nonzero\_real -> real]}.  So given the formula
\begin{pvsex}
  div_form: FORMULA (FORALL (x, y: int):
                      x /= y IMPLIES (x - y)/(y - x) = -1)
\end{pvsex}
%
the denominator is of type integer, but the signature for \texttt{/}
demands a \texttt{nonzero\_real}.  The typechecker thus generates a
\emph{subtype} \tcc\ whose conclusion is \texttt{(y - x) /= 0}.  The
premises of the \tcc\ are obtained from the expressions
\emph{context}\index{context}---the conditions which lead to the
\texttt{/} operator---in this case \texttt{x /= y}.\footnote{As described
in the Formal Semantics~\cite{PVS:semantics}, the context containing
declarations is extended to allow boolean expressions.}  The \tcc\ is then
\begin{pvsex}
  div_form_TCC1: OBLIGATION
    (FORALL (x,y: int): x /= y IMPLIES (y - x) /= 0)
\end{pvsex}
which is easily discharged by the prover.  In general, the context of an
expression is obtained from expressions involving \texttt{IF-THEN-ELSE},
\texttt{AND}, \texttt{OR}, and \texttt{IMPLIES} by translating to the \texttt{IF-THEN-ELSE} form.  Specifically,
\begin{center}
\begin{tabular}{|lc|} \hline
Expression & Context for $e$ \\ \hline
\texttt{IF $a$ THEN $e$ ELSE $c$ ENDIF} & $a$ \\
\texttt{IF $a$ THEN $b$ ELSE $e$ ENDIF} & \texttt{NOT $a$} \\
\texttt{$a$ AND $e$} & $a$ \\
\texttt{$a$ OR $e$} & \texttt{NOT $a$} \\
\texttt{$a$ IMPLIES $e$} & $a$ \\ \hline
\end{tabular}
\end{center}
Note that only these operators are treated this way; if, for example,
\texttt{IMPLIES} is overloaded it will not include the left-hand side in
the context for typechecking the right-hand side.  The \tccs\ generated
from the context of expression involving a subtype are sufficient, but not
necessary conditions that ensure that the value of the expression does
not depend on the value of functions applied outside their domain.

Subtype \tccs\ may occur anywhere there is a mismatch between the type of
a term and the use of it, not just in function applications.  For example,
the following use of record types leads to an unprovable subtype \tcc.
\begin{pvsex}
  r: [# a, b: nzint #] = (# a := 0, b := 0 #)
\end{pvsex}

\index{TCC|)}

\index{type!subtype|)}\index{subtypes|)}

\section{Function Types}\label{function-types}
\index{function types|(}\index{type!function|(}

Function types have three equivalent forms:
\begin{itemize}
\item \texttt{[t\(_1\), \ldots, t\(_n\) -> t]}

\item \texttt{FUNCTION[t\(_1\), \ldots, t\(_n\) -> t]}

\item \texttt{ARRAY[t\(_1\), \ldots, t\(_n\) -> t]}
\end{itemize}
%
where each \texttt{t$_i$} is a type expression.  An element of this type is
simply a function whose domain is the sequence of types \texttt{t$_1$},
\ldots, \texttt{t$_n$}, and whose range is \texttt{t}.  A function type is empty
if the range is empty and the domain is not.  There is no difference in
meaning between these three forms; they are provided to support different
intensional uses of the type, and may suggest how to handle the given type
when an implementation is created for the specification.

The two forms \texttt{pred[t]}\index{pred@{\texttt{pred}}} and \texttt{setof[t]}\index{setof@{\texttt{setof}}} are both provided in the
prelude as shorthand for \texttt{[t ->
bool]}.  There is no difference in semantics, as sets in
PVS are represented as predicates.  The different keywords are
provided to support different intentions; \texttt{pred} focuses on
properties while \texttt{setof} tends to emphasize elements.

A function type \texttt{[t$_1$,\ldots,t$_n$ -> t]} is a subtype of
\texttt{[s$_1$,\ldots,s$_m$ -> s]} iff \texttt{s} is a subtype of
\texttt{t}, $n = m$, and \texttt{s$_i$} = \texttt{t$_i$} for $1 \leq i \leq n$.
This leads to subtype \tccs\ (called \emph{domain mismatch
\tccs})\index{domain mismatch TCC}\index{TCC!domain mismatch} that state
the equivalence of the domain types.  For example, given
\begin{pvsex}
  p, q: pred[int]
  f: [\setb{}x: int | p(x)\sete -> int]
  g: [\setb{}x: int | q(x)\sete -> int]
  h: [int -> int]
  eq1: FORMULA f = g
  eq2: FORMULA f = h
\end{pvsex}
%
The following \tccs\ are generated:
\begin{pvsex}
eq1_TCC1: OBLIGATION
  (FORALL (x1: \setb{}x : int | q(x)\sete, y1 : \setb{}x : int | p(x)\sete) :
     q(y1) AND p(x1))

eq2_TCC1: OBLIGATION
  (FORALL (x1: int, y1 : \setb{}x : int | p(x)\sete) :
     TRUE AND p(x1))
\end{pvsex}

Section~\ref{conversion-examples} on page~\pageref{conversion-examples}
explains how the \texttt{restrict} conversion may be automatically applied
in some cases to eliminate the production of these \tccs.

\index{type!function|)}\index{function types|)}


\section{Tuple Types}\label{tuple-types}
\index{tuple types|(}\index{type!tuple|(}

Tuple types (also called product types) have the form \texttt{[t$_1$,
\ldots, t$_n$]}, where the \texttt{t$_i$} are type expressions.  Note that
the 0-ary tuple type is not allowed.  Elements of this type are tuples
whose components are elements of the corresponding type.  For example,
\texttt{(1, TRUE, (LAMBDA (x:int):\ x + 1))} is an expression of type
\texttt{[int, bool, [int -> int]]}.  Order is important.  Associated with
every $n$-tuple type is a set of projection functions: \texttt{`1},
\texttt{`2}, \ldots, (or \texttt{proj\_1}, \texttt{proj\_2}, \ldots) where
the $i$th projection is of type \texttt{[[t$_1$, \ldots, t$_n$] ->
t$_i$]}.  A tuple type is empty if any of its component types is empty.
Function type domains and tuple types are closely related, as the types
\texttt{[t$_1$,\ldots, t$_n$ -> t]} and \texttt{[[t$_1$,\ldots, t$_n$] ->
t]} are equivalent; see Section~\ref{tuple-exprs} for more details.

\index{type!tuple|)}\index{tuple types|)}

\section{Record Types}\label{record-types}
\index{record types|(}\index{type!record|(}

Record types are of the form \texttt{[\# a$_1$:t$_1$, \ldots, a$_n$:t$_n$
\#]}.  The \texttt{a$_i$} are called \emph{record accessors}\index{record
accessors} or fields and the \texttt{t$_i$} are types.  Record types are
similar to tuple types, except that the order is unimportant and accessors
are used instead of projections.  Record types are empty if any of the
component types is empty.

Note that the fields of a record type must be applied, they are not
understood as functions.  See Section~\ref{record-expressions}.

\index{type!record|)}\index{record types|)}

\section{Dependent types}\label{dependent-types}
\index{dependent types|(}\index{type!dependent|(}

Function, tuple, and record types may be dependent; in other words, some
of the type components may depend on earlier components.  Here are some
examples:
\begin{pvsex}
  rem: [nat, d: \setb{}n: nat | n /= 0\sete -> \setb{}r: nat | r < d\sete]
  pfn: [d:pred[dom], [(d) -> ran]]
  stack: [\# size: nat, elements: [\setb{}n:nat | n < size\sete -> t] \#]
\end{pvsex}
The declaration for \texttt{rem} indicates explicitly the range of the
remainder function, which depends on the second argument.  Function types
may also have dependencies within the domain types; \eg\ the second domain
type may depend on the first.  Note that for function and tuple dependent
types, local identifiers need to be given only for those types on which
later types depend.

The tuple type \texttt{pfn} encodes partial functions as pairs consisting
of a predicate on the domain type and a function from the subtype
defined by that predicate to the range \texttt{ran}.  If the second
component were given instead as a function of type \texttt{[dom -> ran]},
then equality no longer works as intended.  For example, the absolute
value function \texttt{abs} and the identity function \texttt{id} are the same
on the domain \texttt{nat}, so we would like to have
\begin{pvsex}
  ((LAMBDA (x:int):x >= 0),abs) = ((LAMBDA (x:int):x >= 0),id)
\end{pvsex}
%
but without the dependency this would be equivalent to \texttt{abs = id}.

\texttt{stack} encodes a stack as a pair consisting of a size and an array
mapping initial segments of the natural numbers to \texttt{t}.  This is
similar to the \texttt{pfn} example---in fact, if we were willing to use a
tuple instead of a record encoding, \texttt{stack} could be declared as an
instance of the type of \texttt{pfn}.

Another example, presented in~\cite{Cheng&Jones90} as a ``challenge'' to
specification languages without partial functions, is easily handled
with dependent types as shown below.
\begin{pvsex}
  subp(i:int,(j:int | i >= j)): RECURSIVE int =
       (IF (i=j) THEN 0 ELSE (subp(i, j+1)+1) ENDIF)
    MEASURE i - j
\end{pvsex}
However, some formulas that are valid with partial functions are not even
well-formed in PVS:
\begin{pvsex}
  subp_lemma: LEMMA subp(i, 0) = i OR subp(0, i) = i
\end{pvsex}
This generates unprovable \tccs.  In practice this is rarely a problem.

\index{type!dependent|)}\index{dependent types|)}
\section{Cotuple Types}\label{cotuple-types}
\index{cotuple types|(}\index{type!cotuple|(}
\index{coproduct types|see{cotuple type}}
\index{sum types|see{cotuple type}}

\emph{Cotuple types} (also called \emph{coproduct} or \emph{sum} types)
provide a way to form the disjoint union of types.  The syntax is similar
to that for tuple types, but with `\texttt{+}' in place of `\texttt{,}',
so have the form \texttt{[t$_1$ + \ldots + t$_n$]}.  Elements of this type
are essentially pairs consisting of an index and a value for the type
corresponding to the index.  In PVS the syntax for this is
\texttt{IN\_$i(e)$}, where $e$ is an expression of type \texttt{t$_i$}.
For example, \texttt{IN\_2(3)} is an expression of type \texttt{[bool + int
+ [int -> int]]}, or any other cotuple type whose second component type
contains \texttt{3}.  A cotuple type is empty iff all its component types
are empty.

\index{type!cotuple|)}\index{cotuple types|)}

\index{type|)}

% Document Type: LaTeX
% Master File: language.tex

\chapter{Expressions}\label{expressions}
\index{expressions|(}

The PVS language offers the usual panoply of expression constructs,
including logical and arithmetic operators, quantifiers, lambda
abstractions, function application, tuples, a polymorphic
\texttt{IF-THEN-ELSE}, and function and record overrides.  Expressions may
appear in the body of a formula or constant declaration, as the predicate
of a subtype, or as an actual parameter of a theory instance.  The syntax
for PVS expressions is shown in Figures~\ref{bnf-expr} and~\ref{bnf-expr-aux}.

\pvsbnf{bnf-expr}{Expression syntax}

\pvsbnf{bnf-expr-aux}{Expression syntax (continued)}

\index{precedence|(} The language has a number of predefined operators
(although not all of these have a predefined meaning).  These are given in
Figure~\ref{precedenceops} below, along with their relative precedence
from lowest to highest.  Most of these operators are described in the
following sections.  \texttt{IN} is a part of \texttt{ LET} expressions,
\texttt{WITH} goes with override expressions, and the double colon
(\texttt{::}) is for coercion expressions.  The \texttt{o} operator is
defined in the prelude as the function composition operator.  Note that
most operators may be overloaded, see Chapter~\ref{lexical}
(page~\pageref{lexical}) for details.

\begin{figure}[htb]
\begin{center}{\small\tt
\begin{tabular}{|l|l|} \hline
{\rm Operators} & {\rm Associativity} \\ \hline
FORALL, EXISTS, LAMBDA, IN & None \\
\verb/|/ & Left \\
\verb/|-/, \verb/|=/ & Right \\
IFF, <=> & Right \\
IMPLIES, =>, WHEN & Right \\
OR, \verb|\/|, XOR, ORELSE & Right \\
AND, \&, \&\&, \verb|/\|, ANDTHEN & Right \\
NOT, \verb|~| & None \\
=, /=, ==, <, <=, >, >=, <<, >>, <<=, >>=, <|, |> & Left \\
WITH & Left \\
WHERE & Left \\
@, \# & Left \\
@@, \#\#, || & Left \\
+, -, ++, ~ & Left \\
*, /, **, // & Left \\
- & None \\
o & Left \\
:, ::, HAS\_TYPE & Left \\
\verb|[]|, <> & None \\
\verb|^|, \verb|^^| & Left \\
` & Left \\ \hline
\end{tabular}}
\end{center}\caption{Precedence Table}\label{precedenceops}
\end{figure}
\index{precedence|)}

\index{operator symbols|(}

Many of the operators may be overloaded by the user and retain their
precedence and form (\eg\ infix).  All of the infix operators may also be
given in prefix form; \texttt{x + 1} and \texttt{+(x,1)} are semantically equivalent.  Care must be taken in redefining these operators---if the
preceding declaration ends in an expression there could be an ambiguity.
To handle this situation the language allows declarations to be terminated
with a '\texttt{;}'.  For example,
\begin{pvsex}
  AND: [state, state -> state] = (LAMBDA a,b: (LAMBDA t: a(t) AND b(t)));
  OR: [state, state -> state] = (LAMBDA a,b: (LAMBDA t: a(t) OR b(t)));
\end{pvsex}
%
without the semicolon the second declaration would be seen as an infix
\texttt{OR} and the result would be a parse error.

Another common mistake when overloading operators with predefined meanings
is the assumption that overloading, for example, {\tt IMPLIES} automatically
provides an overloading for {\tt =>}.  This is not the case---they are distinct
operators (which happen to have the same meaning by default) and not syntactic
sugar.

\index{operator symbols|)}

\section{Boolean Expressions}\label{bool-exprs}
\index{boolean expressions}

The Boolean expressions include the constants \texttt{TRUE}\index{true@{\texttt{TRUE}}} and
\texttt{FALSE}\index{false@{\texttt{FALSE}}},
the unary operator \texttt{NOT}\index{not@{\texttt{NOT}}}, and
the binary operators \texttt{AND}\index{and@{\texttt{AND}}} (also written
\texttt{ \&}\index{\&}), \texttt{OR}\index{or@{\texttt{OR}}}, \texttt{
IMPLIES}\index{implies@{\texttt{IMPLIES}}}
(\texttt{=>}\index{=>@\texttt{=>}}),
\texttt{WHEN}\index{when@{\texttt{WHEN}}}, and
\texttt{IFF}\index{iff@{\texttt{IFF}}}
(\texttt{<=>}\index{<=>@\texttt{<=>}}).  The declarations for these are in
the \texttt{booleans} prelude theory.  All of these have their standard
meaning, except for \texttt{WHEN}, which is the converse of
\texttt{IMPLIES} (\ie\ $A$ \texttt{WHEN} $B$ $\equiv$ $B$ \texttt{IMPLIES}
$A$).

Equality\index{equality} (\texttt{=}\index{=}) and
disequality\index{disequality} (\texttt{/=}\index{/=}) are declared in the
prelude theories \texttt{equalities} and \texttt{notequal}.  They are both
polymorphic, the type depending on the types of the left- and right-hand
sides.  If the types are compatible, meaning that there is a common
supertype, then the (dis)equality is of the greatest common supertype.  Otherwise it is a type
error.  For example,
\begin{pvsex}
  S,T: TYPE
  s: VAR S
  t: VAR T
  eq1: FORMULA s = t
  i: VAR \setb{}x: int | x < 10\sete
  j: VAR \setb{}x: int | x > 100\sete
  eq2: FORMULA i = j
\end{pvsex}
%
\texttt{eq1} will cause a type error---remember that \texttt{S} and \texttt{T}
are assumed to be disjoint.  \texttt{eq2} is perfectly typesafe because
they have a common supertype \texttt{int} even though the subtypes have no
elements in common; the equality simply has the value \texttt{FALSE}.

When the equality is between terms of type \texttt{bool}, the semantics
are the same as for \texttt{IFF}.  There is a pragmatic difference in the
way the PVS prover processes these operators.  Equalities may be
used for rewriting, which makes for efficient proofs but is incomplete,
\ie\ the prover may fail to find the proof of a true formula.  On the other
hand the \texttt{IFF} form is complete, but may lead to a large number of
cases.  When in doubt, use equality as the prover provides commands
that turn an equality into an \texttt{IFF}.

%The decision to disallow \texttt{eq1} is a pragmatic one; the
%utility of such a declaration is questionable, and most likely the user
%has made an error in the specification.


\section{\texttt{IF-THEN-ELSE} Expressions}
\index{if-then-else@{\texttt{IF-THEN-ELSE}}}

The \texttt{IF-THEN-ELSE} expression \texttt{IF} {\em cond\/} \texttt{THEN} {\em
expr1\/} \texttt{ELSE} {\em expr2\/} \texttt{ENDIF} is polymorphic; its type is the
common type of {\em expr1\/} and {\em expr2\/}.  The {\em cond\/} must
be of type \texttt{boolean}.  Note that the \texttt{ELSE} part is not
optional as this is an expression, not an operational statement.  The
declaration for \texttt{IF} is in the \texttt{if\_def} prelude theory.  \texttt{
IF-THEN-ELSE} may be redeclared by the user in the same way as \texttt{
AND}, \texttt{OR}, etc.  Note that only \texttt{IF} is explicitly redeclared,
the \texttt{THEN} and \texttt{ELSE} are implicit.

Any number of \texttt{ELSIF} clauses may be present; they are translated into nested
\texttt{IF-THEN-ELSE} expressions.  Thus the expression
\begin{pvsex}
  IF A THEN B
  ELSIF C THEN D
  ELSE E
  ENDIF
\end{pvsex}
%
translates to
\begin{pvsex}
  IF A THEN B
  ELSE (IF C THEN D
        ELSE E
        ENDIF)
  ENDIF
\end{pvsex}

\section{Numeric Expressions}
\index{numeric expressions}

The numeric expressions include the \emph{numerals}\index{numerals} (0, 1,
2, \ldots), the unary operator \texttt{-}\index{-}, and the binary infix
operators \texttt{\char94}\index{\^}, \texttt{+}\index{+},
\texttt{-}\index{-}, \texttt{*}\index{*}, and \texttt{/}\index{/}.  The
numerals are all of type \texttt{real}\index{real@\texttt{real}}.
The typechecker has implicit judgements on numbers; \texttt{0} is known to
be \texttt{real}, \texttt{rat}, \texttt{int} and \texttt{nat}; all others
are known to be non zero and greater than zero.  The relational operators
on numeric types are \texttt{<}\index{<@\texttt{<}}, \texttt{
<=}\index{<=@\texttt{<=}}, \texttt{>}\index{>@\texttt{>}}, and
\texttt{>=}\index{>=@\texttt{>=}}.  The numeric operators and axioms are
all defined in the prelude.  As with the boolean operators, all of these
operators may be defined on new types and retain their original
precedences.

The numerals may also be treated as names, and
overloaded.\index{overloading numberals}\index{numerals!overloading} This
is particularly useful for defining algebraic structures such as groups
and rings, where it is natural to overload `\texttt{0}' and `\texttt{1}'.
Note that such use may include actual parameters, just as for names.  Thus
\texttt{groups[int].0} or \texttt{0[int]} might refer to the group zero
instantiated with the integer carrier set.

\section{Applications}
\index{application expressions}

Function application is specified as in ordinary mathematics; thus the
application of function \texttt{f} to expression \texttt{x} is denoted \texttt{
f(x)}.  Those operator symbols that are binary functions, and their
applications, may be written in prefix or the usual infix notation.  For
example, \texttt{(3 + 5) = (2 * 4)} may be written as \texttt{=(+(3,5),
*(2,4))}.

PVS supports higher-order types, so that functions may yield functions
as values or be curried\index{curried applications}.  For example, given
\texttt{f} of type \texttt{[int -> [int, int -> int]]}, \texttt{f(0)(2,3)}
yields an \texttt{int}.

If the application involves a dependent function type then the result
type of the application is substituted for accordingly.  For example,
\begin{pvsex}
  f: [a:int, b:\setb{}x:int | a < x\sete -> \setb{}y:int | a < y & y <= b\sete]
\end{pvsex}
the application \texttt{f(2,3)} is of type \texttt{\setb{}y:int | 2 < y \& y <=
3\sete}.  This application will also lead to the subtype \tcc\ \texttt{2 < 3}.

Application and tuple expressions have a special relation, due to the
type equivalence of \texttt{[t$_1$,\ldots,t$_n$ -> t]} and \texttt{
[[t$_1$,\ldots,t$_n$] -> t]}, see Section~\ref{tuple-exprs} for details.

\section{Binding Expressions}\label{binding-expressions}
\index{binding expressions}

The binding expressions are those which create a local scope for
variables, including the quantified expressions and
$\lambda$-expressions.  Binding expressions consist of an operator, a
list of bindings, and an expression.  The operator is one of the
keywords \texttt{FORALL}\index{forall@\texttt{FORALL}}, \texttt{
EXISTS}\index{exists@\texttt{EXISTS}}, or \texttt{LAMBDA}\index{lambda@{\texttt{LAMBDA}}}.\footnote{Set
expressions are also binding expressions; see Section~\ref{set-exprs} (page~\pageref{set-exprs}).}
The bindings specify the variables bound by the operator; each variable
has an id and may also include a type or a constraint.  Here is a
contrived example:
\begin{pvsex}
  x,y,z,d,e: VAR real
  ex1: AXIOM FORALL x,y,z: (x + y) + z = x + (y + z)
  ex2: AXIOM FORALL (x,y,z: nat): x * (y + z) = (x * y) + (x * z)
  ex3: AXIOM FORALL (n: num | n /= 0): EXISTS (x | x /= 0): x = 1/n
\end{pvsex}
%
In \texttt{ex1}, variables \texttt{x}, \texttt{y}, and \texttt{z} are all of type
\texttt{real}.  In \texttt{ex2} these same variables are of type \texttt{nat},
shadowing the global declarations.  \texttt{ex3} illustrates
the use of constraints; this is equivalent to the declaration
\begin{pvsex}
  ex3: AXIOM FORALL (n: \setb{}n: num | n /= 0\sete):
               EXISTS (x: \setb{}x | x /= 0\sete): x = 1/n
\end{pvsex}

Quantified expressions\index{quantified expressions} are introduced with
the keywords \texttt{FORALL} and \texttt{EXISTS}.  These expressions are
of type \texttt{boolean}.

Lambda expressions\index{lambda expressions} denote unnamed functions.
For example, the function which adds \texttt{3} to an integer may be
written
\begin{pvsex}
  (LAMBDA (x: int): x + 3)
\end{pvsex}
%
The type of this expression is the function type \texttt{[int ->
numfield]}.\footnote{\texttt{numfield} sits between \texttt{number} and
\texttt{real}, and is where the field operators are introduced.  See
Section~{prelude-numbers}.}  In addition, when the range is \texttt{bool},
a lambda expression may be represented as a set expression; see
Section~\ref{set-exprs}.

All of the binding expressions may involve dependent
types\index{dependent types} in the bindings, \eg
\begin{pvsex}
  FORALL (x: int), (y: \setb{}z: int | x < z\sete): p(x,y)
\end{pvsex}
%
Note that in the instantiation of such an expression during a proof will
generally lead to a subtype \tcc.  For example, substituting \texttt{e$_1$} for
\texttt{x} and \texttt{e$_2$} for \texttt{y} will lead to the \tcc\ \texttt{e$_1$ <
e$_2$}.\footnote{Such \tccs\ may never be seen, as they tend to be
proved automatically during a proof; more complicated examples may be
given, for which the prover would need help from the user.  In addition,
a false \tcc\ can show up, \eg\ substituting \texttt{2} for \texttt{x} and
\texttt{1} for \texttt{y}.  This means that the corresponding expression is
not type correct.}

Constant names may be treated as binding expressions by using a
\texttt{!}  suffix.  For example,
\begin{pvsex}
foo! (x : int) : e
\end{pvsex}
is equivalent to
\begin{pvsex}
foo( LAMBDA (x : int) : e)
\end{pvsex}

\section{\texttt{LET} and \texttt{WHERE} Expressions}
\index{let expressions@{\texttt{LET} expressions}}
\index{where expressions@{\texttt{WHERE} expressions}}

\texttt{LET} and \texttt{WHERE} expressions are provided for convenience,
making some forms easier to read.  Both of these forms provide local
bindings for variables that may then be referenced in the body of the
expression, thus reducing redundancy and allowing names to be provided for common subterms.
Here are two examples:
\begin{pvsex}
  LET x:int = 2, y:int = x * x IN x + y
  x + y WHERE x:int = 2, y:int = x * x
\end{pvsex}
%
The value of each of these expressions is 6.

\texttt{LET} and \texttt{WHERE} expressions are internally translated to
applications of lambda expressions; in this case both expressions
translate to
\begin{pvsex}
  (LAMBDA (x:int) : (LAMBDA (y:int) : x + y)(x * x))(2)
\end{pvsex}
%
These translations should be kept in mind when the semantics of these
expressions is in question.

The type declaration is optional, so the above could be written as
\begin{pvsex}
  LET x = 2, y = x * x IN x + y
  x + y WHERE x = 2, y = x * x
\end{pvsex}
In this case the typechecking of these expressions depends on whether
\texttt{x} and/or \texttt{y} have been previously declared as variables.
If they have, then those delarations are used to determine the type.
Otherwise, the right-hand side of the \texttt{=} is typechecked, and if it
is unambiguous is used to determine the type of the variable.  This is 
one way in which these expressions differ from their translation.
It is usually better to either reference a variable or give the type, as
the typechecker uses the ``natural'' type of the expression as the type of
the variable, which can lead to extra \tccs.

The \texttt{LET} expression has a limited form of pattern matching over
tuples.  An example is
\begin{pvsex}
  p: VAR [int, int]
  +(p): int = LET (m, n) = p IN m + n
\end{pvsex}
which is shorter than the equivalent
\begin{pvsex}
  p: VAR [int, int]
  +(p): int = LET m = p`1, n = p`2 IN m + n
\end{pvsex}


\section{Set Expressions}\label{set-exprs}

In PVS, sets of elements of a type \texttt{t} are represented as
predicates, \ie\ functions from \texttt{t} to \texttt{bool}.  The type of a
set may be given as \texttt{[t -> bool]}, \texttt{pred[t]}, or \texttt{
setof[t]}, which are all type equivalent.\footnote{The prelude theory
\texttt{defined\_types} also defines \texttt{PRED}, \texttt{predicate}, \texttt{
PREDICATE}, and \texttt{SETOF} as alternate equivalents.}
The choice depends wholly on the intended use of the type.
Similarly, a set may be given in the form \texttt{(LAMBDA (x:\ t):\
p(x))} or \texttt{\setb{}x:\ t | p(x)\sete}; these are equivalent
expressions.\footnote{In fact, internally they are represented by the
same abstract syntax, they simply print differently.} Note that the
latter form may also represent a type---this usually causes no
confusion as the context generally makes it clear which is expected.
The usual functions and properties of sets are provided in the prelude
theory \texttt{sets}.


\section{Tuple Expressions}\label{tuple-exprs}
\index{tuple expressions}

A tuple expression of the type \texttt{[t$_1$,\ldots,t$_n$]} has the form
\texttt{(e$_1$,\ldots,e$_n$)}.  For example, \texttt{(2, TRUE, (LAMBDA x:\ x +
1))} is of type \texttt{[nat, bool, [nat -> nat]]}.  0-tuples are not
allowed, and 1-tuples are treated simply as parenthesized expressions.
The following relation holds between function types and tuple types:
\begin{pvsex}
  [[t\(\sb{1}\),\ldots,t\(\sb{n}\)] -> t] \(\equiv\) [t\(\sb{1}\),\ldots,t\(\sb{n}\) -> t]
\end{pvsex}
%
This equivalence is most important in theory parameters; it allows one
theory to take the place of many.  For example the \texttt{functions}
theory from the prelude may be instantiated by the reference
\texttt{injective?[[int,int,int],int]}.  Applications of an element \texttt{f} of
this type include \texttt{f(1,2,3)}, \texttt{f((1,2,3))}, and \texttt{f(e)},
where \texttt{e} is of type \texttt{[int,int,int]}.

\section{Projection Expressions}\label{projection-exprs}
\index{projection expressions}

The components of an expression whose type is a tuple can be accessed
using the projection operators \texttt{`1}, \texttt{`2}, \ldots or
\texttt{PROJ\_1}, \texttt{PROJ\_2}, \ldots.  The former are preferred.
Like reserved words, projection expressions are case insensitive and may
not be redeclared.  For the most part, projection expressions are
analogous to field accessors for record types.  For example,
\begin{pvsex}
  t: [int, bool, [int -> int]]
  ft: FORMULA t`2 AND t`1 > t`3(0)
  ft_deprecated: FORMULA PROJ_2(t) AND PROJ_1(t) > (PROJ_3(t))(0)
\end{pvsex}

Projection expressions may be used without an argument as long as the
context determines the tuple type involved.  For example, in the following
it is obvious what tuple type is involved.
\begin{pvsex}
  F: [[[int, bool, [int -> int]] -> bool] -> bool]
  FP: FORMULA F(PROJ_2)
\end{pvsex}
Note that the \texttt{PROJ} keyword must be used in such cases, as, e.g.,
\texttt{`2} is not an expression.  In the following example we see that
the context does not provide enough information.
\begin{pvsex}
  PP: FORMULA PROJ_2 = PROJ_2
\end{pvsex}
To deal with such situations, the syntax for projections has been extended
to allow the tuple type to be provided.
\begin{pvsex}
  PP: FORMULA PROJ_2[[int, bool, [int -> int]]] = PROJ_2
\end{pvsex}
In this case only one of the operators needs to be annotated.  This looks
like a use of actual parameters, but it is not, as the \texttt{PROJ} is
not a name, and does not belong to a theory.


\section{Record Expressions}\label{record-expressions}
\index{record expressions}

Record expressions are of the form \texttt{(\# a$_1$ := e$_1$, \ldots,
a$_n$ := e$_n$ \#)}, which has type \texttt{[\# a$_1$:\ t$_1$, \ldots,
a$_n$:\ t$_n$ \#]}, where each \texttt{e$_i$} is of type \texttt{t$_i$}.
Partial record expressions are not allowed; all fields must be given.  If
it is desired to give a partial record, declare an uninterpreted constant
or variable of the record type, and use override expressions to specify
the given record at the fields of interest.  For example,
\begin{pvsex}
  rc: [# a, b : int #]
  re: [# a, b : int #] = rc WITH [`a := 0]
\end{pvsex}

The type of a record expression is determined by the type of its
components.  Thus \texttt{(\# a := 3, b := 2 \#)} is of type \texttt{[\# a,
b: real \#]}.  This means that a record expression is never of a dependent
record type directly, though it may be used where a dependent record is
expected, and \tccs\ may be generated as a result.  For example,
\begin{pvsex}
  R: TYPE = [# a: int, b: \setb{}x: int | x < a\sete #]
  r: R = (# a := 3, b := 4 #)
\end{pvsex}
%
leads to the (unprovable) \tcc\ \texttt{4 < 3}.

Record expressions may be introduced without introducing the record type
first, and the type of a record expression is determined by its
components, independently of any previously declared record type.  For
this reason record types do not automatically generate associated accessor
functions.

\section{Record Accessors}

The components of an expression of a record type are accessed using the
corresponding field name.  There are two forms of access.  For example if
\texttt{r} is of type \texttt{[\# x, y: real \#]}, the x-component may be
accessed using either \texttt{r`x} or \texttt{x(r)}.  The first form is
preferred as there is less chance for ambiguity.

As noted above, accessors are not stand-alone functions.  However, you can
define your own functions to provide this capability, and even use the
same name.  For example:
\begin{pvsex}
  point: TYPE = [# x, y: real #]
  x(p:point): real = p`x
  y(p:point): real = p`y
\end{pvsex}
Now \texttt{x} and \texttt{y} may be provided wherever a function is
expected.  Note that this means that a subsequent expression of the form
\texttt{x(p)} could be ambiguous, but the record field accessor is always
preferred, so in practice such ambiguities don't arise.

\section{Cotuple Expressions}\label{cotuple-expressions}
\index{cotuple expression}

Elements of cotuple types \texttt{[t$_1$ + \ldots + t$_n$]} are constructed
with the \emph{injection} operators \texttt{IN\_$i$} of type
\texttt{[t$_i$ -> [t$_1$ + \ldots + t$_n$]]}.  Thus if $e$ is of type
\texttt{t$_i$}, \texttt{IN\_$i$($e$)} is of the cotuple type.  If $x$ is
an element of a cotuple type, \texttt{IN?\_$i$($x$)} is a boolean that
tests if $x$ belongs to the $i^{th}$ component, and if it does,
\texttt{OUT\_$i$($x$)} returns the associated value of type
\texttt{t$_i$}.  Note that this is similar to a datatype of the form
\begin{pvsex}
  cotup: DATATYPE
   BEGIN
    IN_1(OUT_1: t\(\sb{1}\)): IN?_1
    \(\cdots\)
    IN_\(n\)(OUT_\(n\): t\(\sb{n}\)): IN?_\(n\)
   END cotup
\end{pvsex}
The differences are that cotuples are not recursive, do not generate all
the functions and axioms associated with datatypes, and allow for any
number of component types---using datatypes a new one would have to be
given for each arity.

The analogy works also for the \texttt{CASES} expression described in
Section~\ref{cases-expressions}.  This allows access to the values of a
cotuple element.  It has the form
\begin{pvsex}
  CASES \(e\) OF
    IN_1(x1): f\(\sb{1}\)(x1),
    \vdots
    IN_\(n\)(x\(n\)): f\(\sb{n}\)(x\(n\))
  ENDCASES
\end{pvsex}
where each \texttt{f$_i$} is an expression of type \texttt{[t$_i$ ->
$T$]}, and the common return type $T$ is the type of the \texttt{CASES}
expression.  For example, if \texttt{x} is of type \texttt{[int + bool +
[int -> int]}, the following expression will return a boolean value.
\begin{pvsex}
  CASES x OF
    IN_1(i): i > 0,
    IN_2(b): NOT b,
    IN_3(f): FORALL (n: int): f(f(n)) = f(n)
  ENDCASES
\end{pvsex}
If there are any missing components in the \texttt{CASES} expression, a
\emph{cases \tcc}\index{cases TCC}\index{TCC!cases} will be generated
stating that the cotuple expression must be one of the given selections,
unless there is an \texttt{ELSE} selection.

Like the projection operators \texttt{PROJ\_$i$}, the \texttt{IN\_$i$},
\texttt{OUT\_$i$} and \texttt{IN?\_$i$} operators make be disambiguated by
adding the cotuple type reference to the operator, for example,
\texttt{IN\_2[int + int](3)} or \texttt{IN?\_1[coT]}.  Note that although
they have the form of actual parameters, they are not, as these operators
are built in and not associated with any theory.  Also, for brevity, only
the cotuple type is given, not the full type of the operator.  There are a
number of axioms associated with cotuples that are built in to the PVS
typechecker and prover.


\section{Override Expressions}
\index{override expression}
\index{update expression}
\index{with expression}

Functions, tuples, records, and datatype elements may be ``modified'' by
means of the override expression.  The result of an override expression is
a function, tuple, record, or datatype element that is exactly the same as
the original, except that at the specified arguments it takes the new
values.  For example,
\begin{pvsex}
  identity WITH [(0) := 1, (1) := 2]
\end{pvsex}
%
is the same function as the \texttt{identity} function (defined in the
prelude) except at argument values \texttt{0} and \texttt{1}.  This is exactly
the same expression as either of
\begin{pvsex}
  (identity WITH [(0) := 1]) WITH [(1) := 2] {\rm or}
  (LAMBDA x: IF x = 1 THEN 2 ELSIF x = 0 THEN 1 ELSE identity(x))
\end{pvsex}

This order of evaluation ensures that functions remain total, and allows
for the possibility of expressions such as
\begin{pvsex}
  identity WITH [(c) := 1, (d) := 2]
\end{pvsex}
where \texttt{c} and \texttt{d} may or may not be equal.  If they are
equal, then the value of the override expression at the common argument is
\texttt{2}.

More complex overrides can be made using nested arguments; for example,
\begin{pvsex}
  R: TYPE = [# a: int, b: [int -> [int, int]] #]
  r1: R
  r2: R = r1 WITH [`a := 0, `b(1)`2 := 4]
\end{pvsex}
{\tt r2} is equivalent to
\begin{pvsex}
  (# a := 0,
     b := LAMBDA (x: int):
           IF x = 1
           THEN (r1`b(x)`1, 4)
           ELSE r2`b(x)
           ENDIF #)
\end{pvsex}

Updating a datatype element amounts to updating the accessor(s) associated
with a constructor.  For example, if \texttt{lst} is of type
\texttt{(cons?[nat])}, then \texttt{lst WITH [`car := 3]} returns a list
that is the same as \texttt{lst}, but whose first element is \texttt{3}.
If \texttt{lst} is given type \texttt{list[nat]}, then the same override
expression generates a \tcc\ obligation to prove that \texttt{lst} is a
\texttt{cons?}.  Because accessors may be both dependent and overloaded,
\tccs\ may get complicated.  For example,
\begin{pvsex}
  dt: DATATYPE
  BEGIN
   c0: c0?
   c1(x: int, a: \{z: (even?) | z > x\}, b: int): c1?
   c2(x: int, a: \{n: nat | n > x\}, c: int): c2?
  END dt
\end{pvsex}
If \texttt{d} is of type \texttt{dt}, the update expression \texttt{d WITH
[a := y]} leads to the \tcc
\begin{pvsex}
  f1_TCC1: OBLIGATION
    (c1?(d) AND even?(y) AND y > x(d)) OR
     (c2?(d) AND y >= 0 AND y > x(d));
\end{pvsex}

Another form of override expression is the maplet, indicated using
\texttt{|->} in place of \texttt{:=}.  This is used to extend the domain
of the corresponding element; for example, if \texttt{f:[nat -> int]} is
given, then \texttt{f WITH [(-1) |-> 0]} is a function of type
\texttt{[\setb{}i:int | i >= 0 OR i = -1\sete -> int]}.  This is especially useful
with dependent types, see Section~\ref{dependent-types}.  Domain extension
is also possible for record and tuple types; for example, \texttt{r1 WITH
[`c |-> 3]} is of type \texttt{[\# a:\ int, b:\ [int -> [int,int]], c:\ int
\#]}, and if \texttt{t1} is of type \texttt{[int, bool]}, then \texttt{t1
WITH [`3 |-> 1]} is of type \texttt{[int, bool, int]}.  It is an error to
extend a tuple type such that gaps are left, so \texttt{t1 WITH [`4 |->
1]} is illegal, though \texttt{t1 WITH [`3 |-> 1, `4 |-> 1]} is allowed.
Gaps would also be left if nested arguments were given, so \texttt{r1 WITH
[`c(0) |-> 0]} is also illegal.  It would have to be given as \texttt{r1
WITH [`c := LAMBDA (x:\ int):\ IF x = 0 THEN 0 ELSE $\cdots$ ENDIF]}, where
the gap $\cdots$ now has to be filled in.  Domain extension is not
possible for datatype elements, as a new datatype theory would need to be
generated for each such extension.

In the past, the two forms of assignment (using \texttt{:=} and
\texttt{|->}) were merely alternative notation, and domains would be
extended automatically whenever the typechecker could not determine that
the argument belonged to the domain.  In most cases, extending the domain
unnecessarily is harmless.  However, when terms get large, the types can
get cumbersome, slowing down the system dramatically.  Even worse, when
domains are extended and matched against a rewrite rule with the original
type, the match can fail, and the automatic rewrite will not be triggered.
For this reason, it is always best to use the maplet on function types
only when actually extending the domain.

\section{Coercion Expressions}\label{coercions}

Coercion expressions are of the form \texttt{expr ::\ type-expr}, indicating
that the expression \texttt{expr} is expected to be of type \texttt{
type-expr}.  This serves two purposes.  First, although PVS allows a
liberal amount of overloading, it cannot always disambiguate things for
itself, and coercion may be needed.  For example, in
\begin{pvsex}
  foo: int
  foo: [int -> int]
  foo: LEMMA foo = foo::int
\end{pvsex}
%
the coercion of \texttt{foo} to \texttt{int} is needed, because otherwise the
typechecker cannot determine the type.  Note that only one of the sides
of the equation needs to be disambiguated.

The second purpose of coercion is as an aid to typechecking; by
providing the expected type in key places within complex expressions,
the resulting \tccs\ may be considerably simplified.

% Master File: language.tex

\section{Tables}
\index{tables|(}

Many expressions are easier to express and to read when presented in
tabular form, as described in~\cite{Heitmeyer94:SCR-checks,Parnas92}.
There are many types of tables, ten different interpretations are
described in~\cite{Parnas92} alone.  Rather than provide support for all
these tables, we chose to support a simple form of table initially,
providing extensions in later versions of PVS as the need arises.

PVS provides a form of table expressions that allows simple
tables\footnote{In Parnas' terms~\cite{Parnas92}, these tables are
\emph{normal function tables} of one or two dimensions.} to be presented,
and supports \emph{table consistency conditions}\index{table consistency}.
One of the consistency conditions (the \emph{Mutual Exclusion
Property}\index{mutual exclusion property} or
\emph{disjointness}\index{disjointness property}) requires the pairwise
conjunction of a set of formulas to be false; another (the \emph{Coverage
Property}\index{coverage property}) requires the disjunction of a set of
formulas to be true.

Tables are supported by means of the more generic \texttt{COND}
expression, which provides the semantic foundation.  In the following
sections, we first describe the \texttt{COND} expression, and then
\texttt{TABLE} expressions.

\subsection{\texttt{COND} Expressions}
\index{cond expressions@\texttt{COND} expressions|(}

The \texttt{COND} construct is a multi-way extension to the polymorphic
\texttt{IF-THEN-ELSE} construct of PVS\@.  Its form is
\begin{pvsex}
  COND
      be\_1 -> e\_1,
      be\_2 -> e\_2,
        \ldots
      be\_n -> e\_n
  ENDCOND
\end{pvsex}
where the \texttt{be\_}i's are boolean expressions, and the \texttt{e\_}i's are
expressions of some common supertype.  It is required that the \texttt{be\_}i's are pairwise disjoint and that their disjunction is a tautology:
these constraints are generated as \emph{disjointness}\index{disjointness
TCC}\index{TCC!disjointness} and \emph{coverage}\index{coverage
TCC}\index{TCC!coverage} TCCs that must be discharged before PVS will
consider a \texttt{COND} expression fully type-correct.
\begin{pvsex}
  foo_TCC1: OBLIGATION NOT (be\_1 AND be\_2) AND\ldots{}AND NOT (be\_n-1 AND be\_n)
  foo_TCC2: OBLIGATION be\_1 OR be\_2 OR\ldots{}OR be\_n
\end{pvsex}

Notice that a \texttt{COND} expression with $n$ clauses generates $O(n^2)$
clauses in its disjointness TCC\@.  

Assuming its associated TCCs are discharged, the schematic \texttt{COND}
shown above is equivalent to the following \texttt{IF-THEN-ELSE} form,
which is its semantic definition.

\begin{pvsex}
  IF be\_1 THEN e\_1
  ELSIF be\_2 THEN e\_2
          \ldots
  ELSIF be\_n-1 THEN e\_n-1
  ELSE e\_n
\end{pvsex}

The \texttt{COND} may include an \texttt{ELSE} clause:
\begin{pvsex}
  COND
      be\_1 -> e\_1,
      be\_2 -> e\_2,
        \ldots
      ELSE -> e\_n
  ENDCOND
\end{pvsex}
This form does not require the coverage TCC and is equivalent to the
\texttt{IF-THEN-ELSE} form shown above.

Using \texttt{COND}, we can translate the following tabular
specification of the \emph{sign} function
\[
\begin{array}{|c||c|c|c|}
\hline
 & x<0 & x=0 & x>0\\
\hline
\emph{sign}(x) & -1 & 0 & 1\\
\hline
\end{array}
\]
into
\begin{pvsex}
  sign(x): int = COND
                    x<0 -> -1, 
                    x=0 -> 0,
                    x>0 -> 1
                 ENDCOND
\end{pvsex}

Two dimensional tables can be generated by nested \texttt{COND}s.  For
example, the following table defining the value for \texttt{safety\_injection}
\begin{center}
\begin{tabular}{|c||c|c|}
\hline
modes & \multicolumn{2}{c|}{conditions} \\
\hline
\hline
normal & false & true\\
\hline
low & not overridden & overridden \\
\hline
voter\_failure & true & false\\
\hline
\hline
safety\_injection & on & off\\
\hline
\end{tabular}
\end{center}
can be represented as
\begin{pvsex}
  safety\_injection(mode, overridden): on\_off = 
    COND
      mode=normal -> off,
      mode=low -> (COND NOT overridden -> on, overridden -> off ENDCOND),
      mode=voter\_failure -> on
    ENDCOND
\end{pvsex}

Notice that \texttt{mode=low} provides the ``left context'' used in
generating the TCCs for the nested \texttt{COND}.  This causes some
redundancy in highly structured two dimensional tables as the
following example shows.
\begin{center}
\begin{tabular}{|c||c|c|}
\hline
 & \multicolumn{2}{c|}{input}\\
\cline{2-3}
state & x & y\\
\hline
\hline
a & a & b\\
\hline
b & b & b\\
\hline
\end{tabular}
\end{center}
This translates to
\begin{pvsex}
  COND
    state=a -> COND input=x -> a,input=y -> b ENDCOND,
    state=b -> COND input=x -> b,input=y -> b ENDCOND
  ENDCOND
\end{pvsex}
The coverage TCCs generated for the two inner \texttt{COND}s will have the form
\begin{pvsex}
   foo\_TCC2 : OBLIGATION state=a IMPLIES input=x OR input=y
   foo\_TCC3 : OBLIGATION state=b IMPLIES input=x OR input=y
\end{pvsex}
whereas, because of the disjointness and coverage of $\{$\texttt{a}, \texttt{b}$\}$, the correct TCC is the simpler form
\begin{pvsex}
   foo\_TCC: OBLIGATION input=x OR input=y
\end{pvsex}
The source of the error here is that our translation of the original
table is too simple-minded.  A better translation is the following.
\begin{pvsex}
  LET
    x1 = COND input=x -> a, input=y -> b ENDCOND,
    x2 = COND input=x -> b, input=y -> b ENDCOND
  IN
    COND state=a -> x1, state=b -> x2 ENDCOND
\end{pvsex}
And this generates the correct TCCs.

Note that if the \texttt{be\_i}'s are members of an enumerated type, then
the standard PVS \texttt{CASES} construct should be used instead of \texttt{COND}, since there is no need to generate TCCs in these cases.  For
example, if in the previous example $\{$ \texttt{a}, \texttt{b} $\}$ and
$\{$ \texttt{x}, \texttt{y} $\}$ had been enumerated types, then the table
could have been expressed as
\begin{pvsex}
  CASES state OF
    a: CASES input OF x: a, y: b ENDCASES,
    b: CASES input OF x: b, y: b ENDCASES
  ENDCASES
\end{pvsex}
and no TCCs would be generated.

If the \texttt{be\_i}'s are all equalities with the same left hand side,
whose right hand sides are ground arithmetic terms (involving only
numbers, \texttt{+}, \texttt{-}, \texttt{*}, \texttt{/}) then the typechecker
directly checks for coverage and disjointness so no \tccs\ are generated
in this case.

\index{cond expressions@\texttt{COND} expressions|)}


\subsection{Table Expressions}

The \texttt{COND} and \texttt{CASES} constructs (see datatypes on page~\pageref{datatypes}) provide the semantic
foundation for our treatment of tables in PVS; for convenience, we
also provide a \texttt{TABLE} construct that provides more attractive
syntax for the important special cases of regular one and
two-dimensional tables.  The example above can be written in the
alternative form.
\begin{pvsex}
   TABLE
%               ---------------------
               |[ input=x | input=y ]|
%     -------------------------------
      | state=a |    a    |    b    ||
%     -------------------------------
      | state=b |    b    |    b    ||
%     -------------------------------
   ENDTABLE
\end{pvsex}
This will translate internally into the \texttt{LET} and \texttt{COND} form
shown earlier.  Note that the horizontal lines are simply PVS
comments.\footnote{The \LaTeX\ generation translates these constructs into attractively
typeset tables.  See the PVS System Guide~\cite{PVS:userguide} for details.}

The row and column headers to a \texttt{TABLE} construct are arbitrary
boolean expressions.   In cases where the expressions are all of the
form \texttt{id=x}, the \texttt{id} can be factored out to produce simpler
tables of the following form.
\begin{pvsex}
   TABLE state,    input
%               ----------
               |[ x | y ]|
%        -----------------
         |  a   | a | b ||
%        -----------------
         |  b   | b | b ||
%        -----------------
   ENDTABLE
\end{pvsex}
In this form, as the headings are enumeration constructs this is
internally represented as a \texttt{CASES} construct, and so generates
no \tccs\ (the previous version generates 5 \tccs).

One-dimensional tables can be presented in both ``horizontal'' and
``vertical'' forms.  The \emph{sign} function example can be
presented as a ``vertical'' table as follows.
\begin{pvsex}
  sign(x): int = TABLE 
%                ------------
                |[ x<0 | -1 ]|
%                ------------
                 | x=0 |  0 ||
%                ------------
                 | x>0 |  1 ||
%                ------------
   ENDTABLE
\end{pvsex}

And as a horizontal one as follows.
\begin{pvsex}
  sign(x): int = TABLE 
%                --------------------
                 |[ x<0 | x=0 | x>0 ]|
%                -------------------
                 |   -1 |  0  |  1  ||
%                --------------------
   ENDTABLE
\end{pvsex}


A more complex two-dimensional example is provided by the mode
transition tables used in SCR\@.  These have the following form.
\[
\begin{array}{|c|c|c|}
\hline
\mbox{current mode} & \mbox{Event} & \mbox{New Mode} \\
\hline
m_{1} & e_{1,1} & m_{1,1} \\
    & e_{1,2} & m_{1,2} \\
    & \ldots & \ldots\\
    & e_{1,k_1} & m_{1,k_1} \\
\hline
m_{2} & e_{2,1} & m_{2,1} \\
    & e_{2,2} & m_{2,2} \\
    & \ldots & \ldots\\
    & e_{2,k_2} & m_{2,k_2} \\
\hline
\ldots  & \ldots & \ldots\\  
\hline
m_{p} & e_{p,1} & m_{p,1} \\
    & e_{p,2} & m_{p,2} \\
    & \ldots & \ldots\\
    & e_{p,k_p} & m_{p,k_p} \\
\hline
\end{array}
\]
And translate to the following form.
\smaller
\begin{center}
\begin{pvsex}
TABLE mode
%--------------------------------
     |  m\_1 | TABLE event
                  | e\_1,1 | m\_1,1 ||
                  | e\_1,2 | m\_1,2 ||
           \ldots
                  | e\_1,k1| m\_1,k1||
             ENDTABLE ||
%--------------------------------
     |  m\_2 | TABLE event
                  | e\_2,1 | m\_2,1 ||
                  | e\_2,2 | m\_2,2 ||
           \ldots
                  | e\_2,k2| m\_2,k2||
             ENDTABLE ||
%--------------------------------
        \ldots
%--------------------------------
     |  m\_p | TABLE event
                  | e\_p,1 | m\_p,1 ||
                  | e\_p,2 | m\_p,2 ||
           \ldots
                  | e\_p,kp| m\_p,kp||
             ENDTABLE ||
%--------------------------------
ENDTABLE
\end{pvsex}
\end{center}

The last row or column heading in a table may contain the \texttt{ELSE}
keyword, which has the same meaning as for the corresponding \texttt{COND}
or \texttt{CASES} expression.

The table may also have blank entries (except in the headings).  These
represent illegal values; in other words the entry may never be reached.
This is represented by generation of a TCC indicating that the
formulas corresponding to the row and column headings for that entry
cannot both be true.

Note that this is different than having ``don't care'' values.  If you
want to add don't care entries, make sure that you use an array; the table
\begin{pvsex}
DC: int
TABLE
         |[ x < 0 | x = 0 | x > 0 ]|
  | y < 0 |   1   |   0   |   DC  ||
  | y = 0 |  DC   |   2   |    3  ||
  | y > 0 |   -2  |  DC   |    0  ||
  ENDTABLE
\end{pvsex}
may seem like any integer may appear in place of \texttt{DC}, but it must
always be the same integer, which is probably not intended.  The right way
to do this is
\begin{pvsex}
DC(n:nat): int
TABLE
         |[ x < 0 | x = 0 | x > 0 ]|
  | y < 0 |   1   |   0   | DC(2) ||
  | y = 0 | DC(0) |   2   |    3  ||
  | y > 0 |   -2  | DC(1) |    0  ||
  ENDTABLE
\end{pvsex}

\index{tables|)}


\index{expression|)}

% Document Type: LaTeX
% Master File: language.tex

\chapter{Theories}\label{theories}
\index{theories}

Specifications in \pvs\ are built from \emph{theories}, which provide
genericity, reusability, and structuring.  \pvs\ theories may be
parameterized.  A theory consists of a \emph{theory
identifier}, a list of formal \emph{parameters}, an \texttt{EXPORTING}
clause, an \emph{assuming part}, a \emph{theory body}, and an ending
id.  The syntax for theories is shown in Figure~\ref{bnf-theory}.

\pvsbnf{bnf-theory}{Theory Syntax}

\pvsbnf{bnf-assuming}{Assuming Syntax}

\pvsbnf{bnf-theory-part}{Theory Part Syntax}

Everything is optional except the identifiers and the keywords.  Thus
the simplest theory has the form
\begin{pvsex}
  triv : THEORY
    BEGIN
    END triv
\end{pvsex}

The formal parameters, assuming, and theory body consist of declarations
and \emph{importings}.  The various declarations are described in
Section~\ref{declarations}.  In this section we discuss the restrictions
on the allowable declarations within each section, the formal parameters,
the assuming part, and the exportings and importings.

The \texttt{groups} theory below illustrates these concepts.  It views a
group as a 4-tuple consisting of a type \texttt{G}, an identity element
\texttt{e} of \texttt{G}, and operations \texttt{o}\footnote{Recall that
\texttt{o} is an infix operator.} and \texttt{inv}.  Note the use of the
type parameter \texttt{G} in the rest of the formal parameter list.  The
assuming part provides the group axioms.  Any use of the \texttt{groups}
theory incurs the obligation to prove all of the \texttt{ASSUMPTION}s.
The body of the \texttt{groups} theory consists of two theorems, which can
be proved from the assumptions.

\pvstheory{groups-alltt}{Theory \texttt{groups}}{groups-alltt}

\section{Theory Identifiers}

The theory identifier introduces a name for a theory; as described in
Section~\ref{names}, this identifier can be used to help disambiguate
references to declarations of the theory.

In the \pvs\ system, the set of theories currently available to the
session form a \emph{context}.  Within the context theory names must be
unique.  There is an initial context available, called the prelude
% (described in Appendix~\ref{prelude}),
that provides, among other things,
the Boolean operators, equality, and the \texttt{real}, \texttt{rational},
\texttt{integer}, and \texttt{naturalnumber} types and their associated
properties.  The only difference between the prelude and user-defined
theories is that the prelude is automatically imported in every theory,
without requiring an explicit \rsv{IMPORTING} clause.

The end identifier must match the theory identifier, or an error is
signaled.


\section{Theory Parameters}\label{parameters}
\index{theory parameters|(}
\index{formal parameters|see{theory parameters}}

The theory parameters allow theory schemas to be specified.  This
provides support for \emph{universal polymorphism}\index{polymorphism}

Theory parameters may be types, subtypes, constants, or
theories,\footnote{This is discussed in Chapter~\ref{interpretations}.}
interspersed with importings.  Theory parameters must have unique
identifiers.  The parameters are ordered, allowing later parameters to
refer to earlier parameters or imported entities.  This is another form of
dependency, akin to dependent types (see Section~\ref{dependent-types}).
A theory is \emph{ instantiated} from within another theory by providing
\emph{actual parameters}\index{actual parameters} to substitute for the
formals.  Actual parameters may occur in importings, exportings, theory
declarations, and names.  In each case they are enclosed in braces
(\texttt{[} and \texttt{]}) and separated with commas.

The actuals must match the formals in number, kind, and (where
applicable) type.  In this matching process the importings, which
must be enclosed in parentheses, are ignored.  For example, given the
theory declaration

\begin{pvsex}
  T [t: TYPE,
     subt: TYPE FROM t
     (IMPORTING orders[subt]) <=: (partial_order?),
     c: subt,
     d: \setb{}x:subt | c <= x\sete]
\end{pvsex}
a valid instance has five actual parameters; an example is
\begin{pvsex}
  T[int, \setb{}x:nat | x < 10\sete, <=, 5, 6]
\end{pvsex}
%
Note that the matching process may lead to the generation of \emph{actual}
\tccs.\index{actual TCC}\index{TCC!actual}

\index{theory parameters|)}


\section{Importings and Exportings}\label{importings}

The importing and exporting clauses form a hierarchy, much like the
subroutine hierarchy of a programming language.

Names declared in a theory may be made available to other theories in the
same context by means of the \texttt{EXPORTING} clause.  Names exported by
a given theory may be imported into a second theory by means of the
\texttt{IMPORTING} clause.  Names that are exported from one theory are
said to be \emph{visible} to any theory which uses the given theory.  In
this section we describe the syntax of the \texttt{EXPORTING} and
\texttt{IMPORTING} clauses and give some simple examples.

\pvsbnf{bnf-exporting}{Importing and Exporting Syntax}


\subsection{The \texttt{EXPORTING} Clause}
\index{exporting@\texttt{EXPORTING}|(}

The \texttt{EXPORTING} clause specifies the names declared in the theory
which are to be made available to any theory \texttt{IMPORTING} it.  It may
also specify instances of the theories which it imported to be exported.
The syntax of the \texttt{EXPORTING} clause is given in
Figure~\ref{bnf-exporting}.

\noindent
The \texttt{EXPORTING} clause is optional; if omitted, it defaults to
\begin{alltt}
  EXPORTING ALL WITH ALL
\end{alltt}

Any declared name may be exported except for variable declarations and formal parameters.
When \texttt{ALL} is specified for the \emph{ExportingNames}, all
entities declared in the theory aside from the variables are exported.
If a list of names is specified, then these are exported.  Finally, when
a list of names follows \texttt{ALL BUT}, all names aside from these are exported.

Since PVS supports overloading, it is possible that the exported name will
be ambiguous.  Such names may be disambiguated by including the type, if
it is a constant, or by including one of the keywords \texttt{TYPE} or
\texttt{FORMULA}.  The keyword \texttt{TYPE} is used for any type
declaration, and \texttt{FORMULA} is used for any formula declaration
(including \texttt{AXIOM}s, \texttt{LEMMA}s, etc.)  If not disambiguated,
all declarations (except variables and formals) with the specified id will
be exported.

When names are specified they are checked for \emph{completeness}.
This means that when a name is exported all of the names on which the
corresponding declaration(s) depend must also be exported.  Thus, for
example, given the following declarations
\begin{alltt}
  sometype: TYPE
  someconst: sometype
\end{alltt}
it would be illegal to export \texttt{someconst} without also exporting
\texttt{sometype}.  Note that this check is unnecessary if exporting
\texttt{ALL} without the \texttt{BUT} keyword.

In some cases it is desirable (or necessary for completeness) to
export some of the instances of the theories which are used by the
given theory.  This is done by specifying a \texttt{WITH} subclause as a
part of the \texttt{EXPORTING} clause.  The \texttt{WITH} subclause may be
\texttt{ALL}, indicating that all instances of theories used by the given
theory are exported.  If \texttt{CLOSURE} is specified, then the
typechecker determines the instances to be exported by a \emph{
completion analysis}\index{completion analysis} on the exported
names.  Completion analysis determines those entities that are
directly or indirectly referenced by one of the exported
names.\footnote{Proofs are not used in completion analysis.} Finally,
a list of theory names may be given; in this case the theory names
must be complete in the sense that if an exported name refers to an
entity in another theory instance, then that theory instance must be
exported also.  Other theory instances may also be exported even if
not actually needed for completeness in this sense.  The \texttt{WITH}
subclause may only reference theory instances, \ie\ theory names with
actuals provided for all of the corresponding formal parameters.

As a practical matter, it is probably best not to include an
\texttt{EXPORTING} clause unless there is a good reason.  That way
everything that is declared will be visible at higher levels of the
\texttt{IMPORTING} chain.

\index{exporting@\texttt{EXPORTING}|)}

\subsection{\texttt{IMPORTING} Clauses}
\index{importings|(}

\texttt{IMPORTING} clauses import the visible names of another theory.
\texttt{IMPORTING} clauses may appear in the formal parameters list, the
assuming part, or the theory part of a theory.  In addition, theory
abbreviations implicitly import the theory name that they abbreviate (see
Section~\ref{theory-abbreviations}).

The names appearing in an \texttt{IMPORTING} or theory abbreviation
specifies a theory and optionally gives an instance of that theory, by
providing actual parameters corresponding to the formal parameters of the
theory used or mappings for the uninterpreted types and constants (see
Chapter~\ref{interpretations}).  \texttt{IMPORTING}s are cumulative;
entities made visible at some point in a theory are visible to every
declaration following.

An \texttt{IMPORTING} with actual parameters provided is said to be a \emph{
theory instance}.\index{theory instance} We use the same terminology for
an \texttt{IMPORTING} that refers to theory that has no formal parameters.
Otherwise it is referred to as a \emph{generic}\index{generic reference}
reference.

A single theory may appear in any number of \texttt{IMPORTING}s of another
theory, both instantiated and generic.  Obviously, any time there is
more than one \texttt{IMPORTING} of a given theory there is a chance for
ambiguity.  Section~\ref{names} discusses such ambiguities, explaining
how the system attempts to resolve them and how the user can
disambiguate in situations where the system cannot.

An \texttt{IMPORTING} forms a relation between the theory containing the
\texttt{IMPORTING} and the theory referenced.  The transitive closure of
the \texttt{IMPORTING} relation is called the \emph{importing chain} of a
theory.  The importing chain must form a directed acyclic graph; hence a
theory may not end up importing itself, directly or indirectly.
\index{importings|)}


\subsubsection{Theory Abbreviations}\label{theory-abbreviations}
\index{theory abbreviations}

A theory abbreviation is a form of importing that introduces a new name
for a theory instance, providing an alternate means for referring to the
instance.  For example, given the importing\footnote{Prior to the
introduction of theory interpretations, this was written as
\texttt{fsets:\ THEORY = sets[[integer -> integer]].}}
\begin{pvsex}
  IMPORTING sets[[integer -> integer]] AS fsets
\end{pvsex}
where \texttt{sets} is a theory in which the function \texttt{member} is
declared, the name \texttt{fsets.member} is equivalent to
\texttt{sets[[integer -> integer]].member}.


\section{Assuming Part}\label{assuming}

The assuming part consists of top-level declarations and
\texttt{IMPORTING}s.  The assuming part precedes the theory part, so the
theory part may refer to entities declared in the assuming part.  The
grammar for the assuming part is given in Figure~\ref{bnf-assuming}.

The primary purpose of the assuming part is to provide constraints on the
use of the theory, by means of \texttt{ASSUMPTION}s.  These are formulas
expressing properties that are expected to hold of any instance of the
theory.  They are generally stated in terms of the formal parameters, and
when instantiated they become \emph{assuming} \tccs.\index{assuming
TCC}\index{TCC!assuming} For example, given the theory \texttt{groups}
above, the importing
\begin{pvsex}
  IMPORTING groups[int, 0, +, -]
\end{pvsex}
generates the following obligations
\begin{pvsex}
  IMP_groups_TCC1: OBLIGATION FORALL (a, b, c: int): a + (b + c) = (a + b) + c;

  IMP_groups_TCC2: OBLIGATION FORALL (a: int): 0 + a = a AND a + 0 = a;

  IMP_groups_TCC3: OBLIGATION FORALL (a: int): (-)(a) + a = 0 AND a + (-)(a) = 0;
\end{pvsex}

Except for the variable declarations, the declarations of the assumings
are all externally visible.  
  
The dynamic semantics of an \emph{assuming} part of a theory is as
follows.  Internal to the theory, assumptions are used exactly as axioms
would be used.  Externally, for each import of a theory, the assumptions
have to be discharged (i.e., proved) with the actual parameters replacing
the formal parameters.  Note that in terms of the proof chain, every proof
in a theory depends on the proofs of the assumptions.

Assuming \tccs\ are generated when a theory is instantiated, which may or
may not occur when it is imported.  Thus if a theory with assumptions is
imported generically, the assuming \tccs\ are not generated until some
reference is instantiated.  If a theory instance is imported, then the
assuming \tccs\ precede the importing in the dynamic semantics.  Note that
this may not make sense, as the assumings may refer to entities that are
not visible until after the theory is imported.  Thus the following is
illegal.
\begin{session}
  assuming_test[n: nat, m: {x:int | x < n}]: THEORY
  BEGIN
   ASSUMING
    rel_prime?(x, y: int): bool = EXISTS (a, b: int): x*a + y*b = 1
    rel_prime: ASSUMPTION rel_prime?(n,m)
   ENDASSUMING
  END assuming_test

  assimp: THEORY
  BEGIN
   IMPORTING assuming_test[4, 2]
  END assimp
\end{session}
And leads to the following error message.
\begin{pvsex}
  Error: assumption refers to 
    assuming_test[4, 2].rel_prime?,
  which is not visible in the current theory
\end{pvsex}
There are a number of ways to solve this problem.  Perhaps the simplest is
to first import the theory generically, then import the instance.
\begin{pvsex}
   IMPORTING assuming_test
   IMPORTING assuming_test[4, 2]
\end{pvsex}
Now the reference to \texttt{rel\_prime?} makes sense in the assuming
\tcc\ generated for the second importing.

In this case, another solution is to simply define \texttt{rel\_prime?} as
a \emph{macro} (see Section~\ref{macro-declarations}).
\begin{pvsex}
  rel_prime?(x, y: int): MACRO bool = EXISTS (a, b: int): x*a + y*b = 1
\end{pvsex}
Of course, this will not work if the declaration in question is a
recursive or inductive definition.

Another solution is to provide the declaration in a theory that is
imported in both the theory with the assuming and the theory importing
that theory.
\begin{session}
  rel_prime[y:int]: THEORY
  BEGIN
   rel_prime?(x: int): bool = EXISTS (a, b: int): x*a + y*b = 1
  END assth2

  assuming_test[n: nat, m: {x:int | x < n}]: THEORY
  BEGIN
   ASSUMING
    IMPORTING rel_prime[m]
    rel_prime: ASSUMPTION rel_prime?(n)
   ENDASSUMING
  END assuming_test2

  assuming_imp: THEORY
  BEGIN
   IMPORTING rel_prime[2], assuming_test[4, 2]
  END assuming_imp
\end{session}
Now the reference to \texttt{rel\_prime?} in the assuming \tcc\ associated
with \texttt{assuming\_test[4, 2]} is the same as the previously imported
instance, so there is no problem.  In the theory \texttt{assuming\_imp},
\texttt{rel\_prime} may also be imported generically.  However, if
\texttt{rel\_prime} is not imported, or is imported with a different
parameter (e.g., \texttt{rel\_prime[3]}) then the above error is produced.


\section{Theory Part}

The theory part consists of top-level declarations and \texttt{IMPORTING}s.
Declarations are ordered; references may not be made to declarations
which occur later in the theory.  The theory part usually contains the
main body of the theory.  Assuming declarations are not allowed in the
theory part.  The grammar for the theory part is given in
Figure~\ref{bnf-theory-part}.

%%% Local Variables: 
%%% TeX-command-default: "Make"
%%% mode: latex
%%% TeX-master: "language"
%%% End: 

% Document Type: LaTeX
% Master File: interpretations-final.tex
\documentclass[11pt,twoside,openright,titlepage]{cslreport}
\usepackage{relsize}
\usepackage{makebnf}
\usepackage{alltt}
%\usepackage{doublespace}

\usepackage{cite}
\usepackage{/homes/rushby/tex/oz}
%\usepackage{/homes/rushby/tex/cslrep}
\usepackage{url}
\usepackage{psfig}
\usepackage{times}
\usepackage{/homes/owre/tex/session}
\usepackage{boxedminipage}
\def\mapb{\char"7B\char"7B}
\def\mape{\char"7D\char"7D}
\def\setb{\char"7B}
\def\sete{\char"7D}
\newcommand{\specware}{{\sc Specware}}
\textwidth 5.5in
\oddsidemargin .65in
\evensidemargin .41in
\raggedbottom
\sloppy


\begin{document}
\begin{titlepage}
\title{\textbf{\larger Theory Interpretations in PVS}}
\author{Sam Owre \and N. Shankar
\date{April 2001}
\cslreportnumber{SRI-CSL-01-01}
\maketitle
\noindent
%\hspace*{-1in}
\raisebox{-0.8cm}[1cm][1cm]{\srilogo}
\acknowledge{Funded by NASA Langley Research Center contract numbers
NAS1-20334 and NAS1-0079 and DARPA/AFRL contract number F33615-00-C-3043.}
\end{titlepage}

\cleardoublepageblank
\pagenumbering{roman}

\begin{abstract}
\thispagestyle{plain}

We describe a mechanism for theory interpretations in PVS.  The
mechanization makes it possible to show that one collection of theories is
correctly interpreted by another collection of theories under a
user-specified interpretation for the uninterpreted types and constants.
A theory instance is generated and imported, while the axiom instances are
generated as proof obligations to ensure that the interpretation is valid.
Interpretations can be used to show that an implementation is a correct
refinement of a specification, that an axiomatically defined specification
is consistent, or that a axiomatically defined specification captures its
intended models.

In addition, the theory parameter mechanism has been extended with a
notion of \emph{theory as parameter} so that a theory instance can be
given as an actual parameter to an imported theory.  Theory
interpretations can thus be used to refine an abstract specification or to
demonstrate the consistency of an axiomatic theory.  In this report we
describe the mechanism in detail.  This extension is a part of PVS version
3.0, which will be publicly released in mid-2001.

\end{abstract}

\tableofcontents
\cleardoublepage
\setcounter{page}{0} 
\pagenumbering{arabic}

\chapter{Introduction}

Theory interpretations have a long history in first-order
logic~\cite{Shoenfield,Enderton,Monk76}\@.  They are used to show that the
language of a given source theory $S$ can be interpreted within a target
theory $T$ such that the corresponding interpretation of axioms of $S$
become theorems of $T$\@.  This demonstrates the consistency of $S$
relative to $T$, and also the decidability of $S$ modulo the decidability
of $T$\@.  Theories and theory interpretations have also become important
in higher-order logic and type theory with languages such as {\sc
Ehdm}~\cite{EHDM:manuals}, IMPS~\cite{Farmer:interpretations},
HOL~\cite{Windley92}, Maude~\cite{Maude}, Extended
ML~\cite{SannellaDT:essential-concepts97}, and
\specware~\cite{SrinivasJullig95}\@.  In these languages, theories are
used as structuring mechanisms for large specifications so that abstract
theories can be refined into more concrete ones through interpretation.
In this report, we describe a theory interpretation mechanism for the PVS
specification language.

Specification languages and programming languages usually have some
mechanism for packaging groups of definitions into modules.  Lisp and Ada
have \emph{packages}\@.  Standard ML has a module system consisting of
signatures, structures corresponding to a signature, and functors that map
between structures.  Packages can be made generic by allowing certain
declarations to serve as parameters that can be instantiated when the
package is imported.  Ada has \emph{generic} packages that allow
parameters.  SML \emph{functors} can be used to construct parametric
modules.  C++ allows \emph{templates}.

In specification languages, a \emph{theory} groups together related
declarations of constants, types, axioms, definitions, and theorems.  One
way of demonstrating the consistency of such a theory is by providing an
interpretation for the uninterpreted types and constants under which the
axioms are valid.  The definitions and theorems corresponding to a valid
interpretation can then be taken as valid without further proof as long as
they have been verified in the source theory.  The technique of
interpreting one axiomatic theory in another has many uses.  It can be
used to demonstrate the consistency or decidability of the former theory
with respect to the latter theory.  It can also be used to refine an
abstract theory down to an executable implementation.

Interpretations are also useful in showing that the axioms capture the
intended models.  For example, a clock synchronization algorithm was
developed in \textsc{Ehdm} and was later shown to be consistent using the
mappings, but it turned out that in one place $<$ was used instead of
$\leq$, and because of this a set of perfectly synchronized clocks was
actually disallowed by the model.  Using interpretations in this way is
similar to testing in allowing for the exploration of the space of models
for the theory.

Parametric theories in PVS share some of the features of theory
interpretations.  Such theories can be defined with formal parameters
ranging over types and individuals, for example,\footnote{This exploits a
new feature of PVS version 3.0, in which numbers may be overloaded as
names.}
{\smaller\begin{alltt}
group[G: TYPE, + : [G, G -> G], 0: G, -: [G -> G]]: THEORY
  BEGIN
    \vdots
  END group
\end{alltt}}
An instance of the theory \texttt{group} can be imported by supplying
actual parameters, the type \texttt{int} of integers, integer addition
{\tt +}, zero \texttt{0}, and integer negation {\tt -}, corresponding to
the formal parameters, as in {\tt group[int, +, 0, -]}\@.  A theory can
include assumptions about the parameters that have to be discharged when
the actual parameters are supplied.  For example, the group axioms can
be given as assumptions in the \texttt{group} theory above.  However,
there are some crucial differences between parametric theories and theory
interpretations.  In particular, if axioms are always specified as
assumptions, then the theory can be imported only by discharging these
assumptions.  It is necessary to have separate mechanisms for importing a
theory with the axioms, and for interpreting a theory by supplying a valid
interpretation, that is, one that satisfies its axioms.

The PVS theory interpretation mechanism is quite similar to that for
theory parameterization.  The axiomatic specification of groups could
alternately be given in a theory
{\smaller\begin{alltt}
group: THEORY
 BEGIN
  G: TYPE+
  +: [G, G -> G]
  0: G
  -: [G -> G]
   \vdots
 END group
\end{alltt}}
The group axioms are declared in the body of the theory.  Such a theory
can be interpreted by writing \texttt{\smaller group\mapb{}G := int, + :=
+, 0 := 0, - := -\mape{}}\@.  Here the left-hand sides refer to the
uninterpreted types and constants of theory \texttt{group}, and the
right-hand sides are the interpretations.  This notation resembles that of
theory parameterization and is used in contexts where a theory is
imported.  The corresponding instances of the group axioms are generated
as proof obligations at the point where the theory is imported.  The
result is a theory that consists of the corresponding mapping of the
remaining declarations in the theory \texttt{group}\@.  This allows the
theory \texttt{group} to be used in other theories, such as rings and
fields, and also allows the theory \texttt{group} to be suitably
instantiated by group structures.

Theory interpretations largely subsume parametric theories in the sense
that the theory parameters and the corresponding assumings can instead be
presented as uninterpreted types and constants and axioms so that the
actual parameters are given by means of an interpretation.  However, a
parametric theory with both assumings and axioms involving the parameters
is not equivalent to any interpreted theory, as the parameters may be
instantiated without the need to prove the axioms.  It is also useful to
have parametric theories as a convenient way of grouping together all the
parameters that must be provided whenever the theory is used.  For
example, typical theory parameters such as the size of an array, or the
element type of an aggregate datatype such as an array, list, or tree, are
required as inputs whenever the corresponding theories are used.  While
this kind of parameterization can be captured by theory interpretations,
it would not capture the intent that these parameters are \emph{required}
inputs wherever the theory is used.  Furthermore, when an operation from a
parametric theory is used, PVS attempts to figure out the actual
parameters based on the context of its use.  It can do this because the
formal parameters are precisely delimited.  The corresponding inference is
harder for theory interpretations since there might be many possible
interpretations that are compatible with the context of the operations
use.

In addition to the uninterpreted types and constants in a source theory
$S$, the PVS theory interpretation mechanism can also be used to interpret
any theories that are imported into $S$ by means of the \texttt{THEORY}
declaration.  The interpretation of a theory declaration for $S'$ imported
within $S$ must itself be a theory interpretation of $S'$\@.  Two distinct
importations of a theory $S'$ within $S$ can be given distinct
interpretations.  A typical situation is when two theories $R_1$ and $R_2$
both import a theory $S$ as $S_1$ and $S_2$, respectively.  A theory $T$
importing both $R_1$ and $R_2$ might wish to identify $S_1$ and $S_2$
since, otherwise, these would be regarded as distinct within $T$\@.  This
can be done by importing an instance $S'$ of $S$ into $T$ and importing
$R_1$ with $S_1$ interpreted by $S'$ and $R_2$ with $S_2$ interpreted as
$S'$\@.  With theory interpretations, we have also extended parametric
theories in PVS to take theories as parameters.  For example, we might
have a theory \texttt{group\_homomorphism} of group homomorphisms that
takes two groups \texttt{G1} and \texttt{G2} as parameters as in the
declaration
\begin{alltt}
 group_homomorphism[G1, G2: THEORY group]: THEORY \ldots
\end{alltt}
The actual parameters for these theory formals must be
interpretations \texttt{G1'} and \texttt{G2'}
of the theory \texttt{group}\@.

Another typical requirement in a theory interpretation mechanism is
the ability to map a source type to some quotient with respect to
an equivalence relation over a target type.
For example, rational numbers can be interpreted by means of
a pair of integers corresponding to the numerator and denominator,
but the same rational number can have multiple such representations.
We show how it is possible to define quotient types in PVS and use
these types to capture interpretations where the equality over a
source type is mapped to an equivalence relation over a target type. 

The implementation of theory interpretation in PVS is described in the
following chapters.  This report assumes the reader is already familiar
with the PVS language; for details see the PVS Language
Manual~\cite{PVS:language}.  Chapter 2 deals with mappings, explaining the
basic concepts and introduces the grammar.  Chapter 3 introduces theory
declarations and theories as parameters which allow any valid
interpretation of the formal parameter theory as an actual parameter.
Chapter 4 describes a new command for viewing theory instances.  Chapter 5
compares PVS interpretations with other systems, Chapter 6 describes
future work, and we conclude with Chapter 7.


\chapter{Mappings}\label{mappings}

Theory interpretations in PVS provide mappings for uninterpreted types and
constants of the \emph{source} theory into the current
(\emph{interpreting}) theory.  Applying a mapping to a source theory
yields an \emph{interpretation} (or \emph{target}) theory.  A mapping is
specified by means of the \emph{mapping} construct, which associates
uninterpreted entities of the source theory with expressions of the target
theory.  The mapping construct is an extension to the PVS notion of
``name''.  The changes to the existing grammar are given in
Figure~\ref{mapping-bnf}.

\begin{figure}
\setlength{\sessionboxwidth}{\linewidth}
\addtolength{\sessionboxwidth}{-\arrayrulewidth}
\addtolength{\sessionboxwidth}{-\tabcolsep}
\begin{boxedminipage}[b]{\sessionboxwidth}
\begin{bnf}

\production{TheoryName}
{\opt{Id \lit{@}} Id \opt{Actuals} \opt{Mappings}}

\production{Name}
{\opt{Id \lit{@}} IdOp \opt{Actuals} \opt{Mappings} \opt{\lit{.} IdOp}}

\production{Mappings}
{\lit{\mapb{}} \ites{Mapping}{,} \lit{\mape{}}}

\production{Mapping}
{MappingLhs MappingRhs}

\production{MappingLhs}
{IdOp \rep{Bindings} \opt{\lit{:} \brc{\lit{TYPE} \choice \lit{THEORY}
\choice TypeExpr}}}

\production{MappingRhs}
{\lit{:=} \brc{Expr \choice TypeExpr}}

\end{bnf}
\end{boxedminipage}
\caption{Grammar for Names with Mappings}\label{mapping-bnf}
\end{figure}

The mapping construct defines the basic translation, but to be a theory
interpretation the mapping must be consistent: if type \texttt{T} is
mapped to the type expression \emph{E}, then a constant \texttt{t} of type
\texttt{T} must be mapped to an expression \emph{e} of type \emph{E}.  In
addition, all axioms and theorems of the source theory must be shown to
hold in the target theory under the mapping.  Since the theorems are
provable from the axioms, it is enough to show that the translation of the
axioms hold.  Axioms whose translations do not involve any
uninterpreted types or constants of the source theory are converted to
proof obligations.  Otherwise they remain axioms.

Theory interpretation may be viewed as an extension of theory
parameterization.  Given a theory named \texttt{T}, the instance
\texttt{T[a$_1$,\ldots,a$_n$]\mapb{}c$_1$:= e$_1$,\ldots,c$_m$:=
e$_m$\mape{}} is the same as the original theory, with the \emph{actuals}
\texttt{a$_i$} substituted for the corresponding formal parameters, and
e$_i$ substituted for \texttt{c$_i$}, which must be an uninterpreted type
or constant declaration.  Declarations that appear in the target of a
substitution in the mapping are not visible in the importing theory.  Some
axioms are translated to proof obligations.  The substituted forms of any
remaining axioms, definitions, and lemmas are available for use, and are
considered proved if they are proved in the uninterpreted theory.

The following simple example illustrates the
basic concepts.
\begin{session}
th1[T: TYPE, e: T]: THEORY
 BEGIN
  t: TYPE+
  c: t
  f: [t -> T]
  ax: AXIOM EXISTS (x, y: t): f(x) /= f(y)
  lem1: LEMMA EXISTS (x:T): x /= e
 END th1
\end{session}
\begin{session}
th2: THEORY
 BEGIN
  IMPORTING th1[int, 0]
               \mapb{} t := bool,
                  c := true,
                  f(x: bool) := IF x THEN 1 ELSE 0 ENDIF \mape{}
  lem2: LEMMA EXISTS (x:int): x /= 0
 END th2
\end{session}
\noindent Here theory \texttt{th1} has both actual parameters and
uninterpreted types and constants, as well as an axiom and
a lemma.  Theory \texttt{th2} imports \texttt{th1}, making the
following substitutions:
\setlength{\jot}{-2pt}
\setlength{\abovedisplayskip}{0pt}
\setlength{\belowdisplayskip}{0pt}
{\smaller\begin{eqnarray*}
\texttt{T} & \leftarrow & \texttt{int} \\
\texttt{e} & \leftarrow & \texttt{0} \\
\texttt{t} & \leftarrow & \texttt{bool} \\
\texttt{c} & \leftarrow & \texttt{true} \\
\texttt{f} & \leftarrow & \texttt{LAMBDA (x:\ bool):\ IF x THEN 1 ELSE 0 ENDIF} \\
\end{eqnarray*}}
Note that the mapping for \texttt{f} uses an abbreviated form of
substitution.  Typechecking this leads to the following proof obligation.
\begin{session}
IMP_th1_TCC1: OBLIGATION
  EXISTS (x, y: bool):
    IF x THEN 1 ELSE 0 ENDIF /= IF y THEN 1 ELSE 0 ENDIF;
\end{session}
This is simply the interpretation of the \texttt{ax} axiom and is easily
proved.  The lemma \texttt{lem1} can be proved from the axiom, and may
be used directly in proving \texttt{lem2} using the proof command
\texttt{(LEMMA "lem1")}.

Note that once the TCC has been proved, we know that \texttt{th1} is
consistent.  If we had left out the mapping for \texttt{f}, then the TCC
would not be generated, and the translation of theory \texttt{th1} would
still contain an axiom and not necessarily be consistent.

% Note that we used a lambda form in the axiom,
% rather than \texttt{f}.  This is because logically the generated proof
% obligation precedes the importing, which is only meaningful if the
% obligation is provable.  Hence \texttt{f} is not visible in the proof
% obligation and should not appear in any axiom of the theory.\footnote{We
% may allow this in future versions of PVS by automatically expanding
% non-recursive definitions as a part of substitution, treating them as a
% kind of macro.}  After the importing, of course, \texttt{f} is visible as
% seen in \texttt{lem2}.

% Note that mappings make theory parameters optional---they may be
% eliminated by moving the formal parameters to the theory body and turning
% assumptions into axioms.  The theory could then be instantiated using
% mappings instead of actual parameters.  Theory parameters are still
% useful, however.  First, they may be used to distinguish between the
% parameters to the system being specified and the entities defined by the
% system.  For example, in describing a protocol that works for any number
% of processes, it is more natural to make the number of processes a formal
% parameter rather than an uninterpreted constant.  Second, the
% typechecker can frequently infer the values of the actual parameters when
% a theory is imported generically, but mappings must be explicitly given.
% Although in principle the typechecker might be extended to infer mappings,
% it is hard to see how to do this efficiently.

One advantage to using mappings instead of parameters is that not all
uninterpreted entities need be mapped, whereas for parameters either all
or none must be given.  For example, consider the following theory.
\begin{session}
example1[T: TYPE, c: T]: THEORY
 BEGIN
  f(x: T): int = IF x = c THEN 0 ELSE 1 ENDIF
 END example1
\end{session}
\noindent It may be desirable to import this where \texttt{T} is always
\texttt{real}, and \texttt{c} is left as a parameter, but there is
currently no mechanism for this.  One could envision partial importings
such as \texttt{IMPORTING example1[real, \_]}, but it is not clear that
this is actually practical---in particular, the syntax for providing the
missing parameters is not obvious.  With mappings, on the other hand, we
can define \texttt{example1} as follows.
\begin{session}
example1: THEORY
 BEGIN
  T: TYPE
  c: T
  f(x: T): int = IF x = c THEN 0 ELSE 1 ENDIF
 END example1
\end{session}
\noindent Then we can refer to this theory from another theory as in the
following.
\begin{session}
example2: THEORY
 BEGIN
  th: THEORY = example1\mapb{}T := real\mape{}
  frm: FORMULA f\mapb{}c := 3\mape{} = f
 END example2
\end{session}
\noindent The \texttt{th} theory declaration just instantiates \texttt{T},
leaving \texttt{c} uninterpreted.  The first reference to \texttt{f} maps
\texttt{c} to \texttt{3}, whereas the second reference leaves it
uninterpreted though it is still a \texttt{real}.  Note that formula
\texttt{frm} is unprovable, since the uninterpreted \texttt{c} from the
second reference may or may not be equal to \texttt{3}.

As described in the introduction, an important aspect of mappings is the
support for quotient types.  In \textsc{Ehdm} this was done by
interpreting equality, but in PVS we instead define a theory of
equivalence classes, and allow the user to map constants to equivalence
classes under congruences.  For example, the \texttt{stacks} datatype
might be implemented using an array as follows.
\begin{session}
stack[t:TYPE]: DATATYPE
 BEGIN
  empty: empty?
  push(top:t, pop: stack): nonempty?
 END stack
\end{session}
\begin{session}\label{cstack}
cstack[t: TYPE+]: THEORY
 BEGIN
  cstack: TYPE = [# size: nat, elems: [nat -> t] #]
  cempty?(s: cstack): bool = (s`size = 0)
  some_t: t = epsilon(LAMBDA (x:t): true)
  cempty: (cempty?) =
    (# size := 0,
       elems := LAMBDA (n: nat): some_t #)
  cnonempty?(s: cstack): bool = (s`size /= 0)
  ctop(s: (cnonempty?)): t = s`elems(s`size - 1)
  cpop(s: (cnonempty?)): cstack = s WITH [`size := s`size - 1]
  cpush(x: t)(s: cstack): (cnonempty?) =
    (# size := s`size + 1,
       elems := s`elems WITH [(s`size) := x] #)
  ce: equivalence[cstack] =
    LAMBDA (s1, s2: cstack):
     s1`size = s2`size AND
     FORALL (n: below(s1`size)): s1`elems(n) = s2`elems(n)

  estack: TYPE = Quotient(ce)
  CONVERSION+ EquivClass(ce), rep(ce), lift(ce)
  \ldots
\end{session}
\texttt{Quotient}, \texttt{EquivClass}, and \texttt{rep} are defined in
the prelude theory \texttt{QuotientDefinition}, shown here in part.
\begin{session}
QuotientDefinition[T : TYPE] : THEORY
BEGIN
  R : VAR set[[T, T]]
  S : VAR equivalence[T]
  x, y, z : VAR T

  EquivClass(R)(x) : set[T] = { z | R(x, z) }
  \ldots
  Quotient(S) : TYPE =
    { P : set[T] | EXISTS x : P = EquivClass(S)(x) }
  \ldots
  rep(S)(P: Quotient(S)): T = choose(P)
  \ldots
END QuotientDefinition
\end{session}
The \texttt{lift} function is defined in the prelude theory
\texttt{QuotientExtensionProperties} as follows.
\begin{session}
QuotientExtensionProperties[X, Y : TYPE] : THEORY
BEGIN
  S : VAR equivalence[X]

  lift(S)(g : (PreservesEq[X, Y](S)))(P : Quotient(S)) : Y
    = g(rep(S)(P))
  \ldots
END QuotientExtensionProperties
\end{session}
This allows functions on concrete stacks to be lifted to functions on
equivalence classes, so long they satisfy the \texttt{PreservesEq}
relation, i.e., they produce the same values on \texttt{S}-equivalent
elements.

With these conversions in place, we can finish the specification of
\texttt{cstack} as follows.
\begin{session}
  \ldots
  IMPORTING stack[t]\mapb{} stack := estack,
                       empty? := cempty?,
                       nonempty? := cnonempty?,
                       empty := cempty,
                       top(s: (cnonempty?)) := ctop(s),
                       pop(s: (cnonempty?)) := cpop(s),
                       push(x: t, s: cstack) := cpush(x)(s) \mape{}
 END cstack
\end{session}
\noindent Here the source type \texttt{stack} is mapped to the quotient
type \texttt{estack} defined by the concrete equality \texttt{ce}.  The
\texttt{empty?} and \texttt{nonempty?} predicates are mapped to predicates
on \texttt{estack}s, using the \texttt{rep(ce)} conversion.  The
\texttt{empty} constructor is then mapped to its equivalence class.

\texttt{top}, \texttt{pop},

The mapping for \texttt{push} is more involved; \texttt{cpush} must first
be lifted in order to apply it to the abstract stack \texttt{s}.  This is
applied automatically by the conversion mechanism of PVS.  The application
of \texttt{lift} generates the proof obligation that \texttt{cpush}
preserves the equivalences, that is, it is a congruence.  This mapping
generates a large number of proof obligations, because the \texttt{stack}
datatype generates a \texttt{stacks\_adt} theory with a large number of
axioms, for example, extensionality, well-foundedness, and induction.

The PVS interpretations mechanism is much simpler to implement than the
one in \textsc{Ehdm}---equality is not a special case, but simply an
aspect of mapping a type to an equivalence class.  The technique of
mapping types to equivalence classes is quite useful, and captures the
notion of behavioral equivalence outlined
in~\cite{SannellaDT:essential-concepts97}.  In fact it is more general, in
that it works for any equivalence relation, not just those based on
observable sorts.


\chapter{Theory Declarations}

With the mapping mechanism, it is easy to specify a general theory and
have it stand for any number of instances.  For example, groups, rings,
and fields are all structures that can be given axiomatically in terms of
uninterpreted types and constants.  This works well when considering one
such structure at a time, but it is difficult to specify theories that
involve more than one structure, for example, group homomorphisms.
Importing the original theory twice is the same as importing it once, and
an attempted definition of a homomorphism would turn into an automorphism.
In this case what is needed is a way to specify multiple different
``copies'' of the original theory.  This is accomplished with \emph{theory
declarations}, which may appear in either the theory parameters or the
body of a theory.  A theory declaration in the formal parameters is
referred to as a \emph{theory as parameter}.\footnote{The term
\emph{theory parameter} refers to a parameter of a theory, so we use the
term \emph{theory as parameter} instead.}  Theory declarations allow
theories to be encapsulated, and instantiated copies of the implicitly
imported theory are generated.
\begin{figure}[!b]
\setlength{\sessionboxwidth}{\linewidth}
\addtolength{\sessionboxwidth}{-\arrayrulewidth}
\addtolength{\sessionboxwidth}{-\tabcolsep}
\begin{boxedminipage}[b]{\sessionboxwidth}
\begin{bnf}\smaller

\production{TheoryFormalDecl}
{TheoryFormalType \choice TheoryFormalConst \choice TheoryDecl}

\production{TheoryDecl}
{Id \lit{:} \lit{THEORY} TheoryDeclName}

\production{TheoryDeclName}
{\opt{Id \lit{@}} Id \opt{Actuals} \opt{TheoryDeclMappings}}

\production{TheoryDeclMappings}
{\lit{\mapb{}} \ites{TheoryDeclMapping}{,} \lit{\mape{}}}

\production{TheoryDeclMapping}
{MappingLhs TheoryDeclMappingRhs}

\production{TheoryDeclMappingRhs}
{MappingSubst \choice MappingDef \choice MappingRename}

\production{MappingSubst}
{\lit{:=} \brc{Expr \choice TypeExpr}}

\production{MappingDef}
{\lit{=} \brc{Expr \choice TypeExpr}}

\production{MappingRename}
{\lit{::=} \brc{IdOp \choice Number}}

\end{bnf}
\end{boxedminipage}
\caption{Grammar for Theory Declarations}\label{theory-parameter-bnf}
\end{figure}

For example, an (additive) group is normally thought of as a 4-tuple
consisting of a set $G$, a binary operator $+$, an identity element $0$,
and an inverse operator $-$ that satisfies the usual group axioms.  Using
theory interpretations, we simply define this as follows:
\begin{session}
group: THEORY
 BEGIN
  G: TYPE+
  +: [G, G -> G]
  0: G
  -: [G -> G]
  x, y, z: VAR G
  associative_ax: AXIOM FORALL x, y, z: x + (y + z) = (x + y) + z
  identity_ax: AXIOM FORALL x: x + 0 = x
  inverse_ax: AXIOM FORALL x: x + -x = 0 AND -x + x = 0
  idempotent_is_identity: LEMMA x + x = x => x = 0
 END group
\end{session}

As described in Chapter~\ref{mappings}, we can use mappings to create
specific instances of groups.  For example, {\smaller\begin{alltt}
group\mapb{}G := int, + := +, 0 := 0, - := -\mape{}
\end{alltt}}
\noindent is the additive group of integers, whereas
{\smaller\begin{alltt}
group\mapb{}G := nzreal, + := *, 0 := 1, - := LAMBDA (r:nzreal):\ 1/r\mape{}
\end{alltt}}
\noindent is the multiplicative group of nonzero reals.

This works nicely, until we try to define the notion of a group
homomorphism.  At this point we need two groups, both individually
instantiable.  We could simply duplicate the group specification, but
this is obviously inelegant and error prone.  Using theories as
parameters, we may define group homomorphisms as follows.
\begin{session}
group_homomorphism[G1, G2: THEORY group]: THEORY
 BEGIN
  x, y: VAR G1.G
  f: VAR [G1.G -> G2.G]
  homomorphism?(f): bool = FORALL x, y: f(x + y) = f(x) + f(y)
  hom_exists: LEMMA EXISTS f: homomorphism?(f)
 END group_homomorphism
\end{session}
\noindent Here \texttt{G1} and \texttt{G2} are theories as parameters to a
generic homomorphism theory that may be instantiated with two different
groups.  Hence we may import \texttt{group\_homomorphism}, for example, as
\begin{session}
IMPORTING group_homomorphism[group\mapb{}G := int, + := +, 0 := 0, - := -\mape{}
                             group\mapb{}G := nzreal, + := *, 0 := 1,
                                 - := LAMBDA (x: nzreal): 1/x\mape{}]
\end{session}

There is a subtlety here that needs emphasizing; \texttt{G1} and
\texttt{G2} are two \emph{distinct} versions of theory \texttt{group}.
For example, consider the addition of the following lemma to
\texttt{group\_homomorphism}.
\begin{session}
oops: LEMMA G1.0 = G2.0
\end{session}
\noindent If \texttt{G1} and \texttt{G2} are treated as the same
\texttt{group} theory, this is a provable lemma.  But then after the
importing given above we would be able to show that \texttt{0 = 1}.  Even
worse, the two different instances of groups may not even be type
compatible, so the \texttt{oops} lemma should not even typecheck.

We have solved this in PVS by making new theories \texttt{G1} and
\texttt{G2} that are copies of the original \texttt{group} theory.
Declarations within these copies are distinct from each other and from the
original.  Thus the \texttt{oops} lemma generates a type error, as
\texttt{G1.G} and \texttt{G2.G} are incompatible types.

This introduces new possibilities.  When creating copies of a theory the
mappings are substituted and the original declarations disappear.
However, it may be preferable to create definitions rather than
substitutions.  In addition, it is sometimes useful to simply rename the
types or constants of a theory.  For example, consider the following group
instance
\begin{session}
G1: THEORY = group\mapb{}G := int, + := +, 0 := 0, - := -\mape{}
\end{session}
\noindent which generates the following theory.
\label{group-instances-start}
\begin{session}
G1: THEORY
 BEGIN
  x, y, z: VAR int
  idempotent_is_identity: LEMMA x + x = x => x = 0
 END G1
\end{session}
To create definitions, use \texttt{=} instead of \texttt{:=}, as
in the following.
\begin{session}
G2: THEORY = group\mapb{}G = int, + = +, 0 = 0, - = -\mape{}
\end{session}
\noindent Now we get the following theory.
\begin{session}
G2: THEORY
 BEGIN
  G: TYPE+ = int
  +: [G, G -> G] = +
  0: G = 0
  -: [G -> G] = -
  x, y, z: VAR G
  idempotent_is_identity: LEMMA x + x = x => x = 0
 END G2
\end{session}
Finally, to simply rename the uninterpreted types and constants, use
\texttt{::=} as in the following.
\begin{session}
G3: THEORY = group\mapb{}G ::= MG, + ::= *, 0 ::= 1, - ::= inv\mape{}
\end{session}
\noindent The generated theory instance specifies multiplicative groups as
follows.
\begin{session}
G3: THEORY
 BEGIN
  MG: TYPE+
  *: [MG, MG -> MG]
  1: MG
  inv: [MG -> MG]
  x, y, z: VAR MG
  associative_ax: AXIOM FORALL x, y, z: x * (y * z) = (x * y) * z
  identity_ax: AXIOM FORALL x: x * 1 = x
  inverse_ax: AXIOM FORALL x: x * inv(x) = 1 AND inv(x) * x = 1
  idempotent_is_identity: LEMMA x * x = x => x = 1
 END G3
\end{session}
The right-hand side of a renaming mapping must be an identifier, operator,
or number, and must not create ambiguities within the generated theory.
Note that renamed declarations are still uninterpreted, and may themselves
be given interpretations, as in
\begin{session}
G3i: THEORY = G3\mapb{}MG := nzreal, * := *, 1 := 1,
                  inv := LAMBDA (r: nzreal): 1/r\mape{}
\end{session}

Finally, we can mix the different forms of mapping, to give a partial
mapping.
\begin{session}
G4: THEORY = group\mapb{}G = nzreal, + := *, 0 ::= one\mape{}
\end{session}
\noindent This generates the following theory instance.
\begin{session}
G4: THEORY
 BEGIN
  G: TYPE+ = nzreal;
  one: nzreal;
  -: [nzreal -> nzreal]
  x, y, z: VAR nzreal
  identity_ax: AXIOM FORALL (x: nzreal): x * one = x
  inverse_ax: AXIOM FORALL (x: nzreal):
                      x * -x = one AND -x * x = one
  idempotent_is_identity: LEMMA x * x = x => x = one
 END G4
\end{session}\label{group-instances-end}
Note that \texttt{associative\_ax} has disappeared---it has become a TCC
of the importing theory---whereas the other axioms are not so transformed
because they still reference uninterpreted types or constants.

With theories as parameters we have another situation in which mappings
are more convenient than theory parameters.  Many times the same set of
parameters is passed through an entire theory hierarchy.  If there are
assumings, then these must be copied.  For example, consider the
following theory.
\begin{session}
th[T: TYPE, a, b: T]: THEORY
 BEGIN
  ASSUMING
   A: ASSUMPTION a /= b
  ENDASSUMING
  ...
 END th
\end{session}
\noindent To import this theory, you simply provide a type and two
different elements of that type.  But suppose you wish to import this
theory from a theory that has the same parameters.  In this case the
assumption must also be copied, as there is otherwise no way to prove the
resulting obligation.  This can (and frequently does) lead to a tower of
theories, all with the same parameters and copies of the same assumptions,
as well as proofs of the same obligations.

There are ways around this, of course.  Most assumptions may be turned
into type constraints, as in the following.
\begin{session}
th[T: TYPE, a: T, b: \setb{}x: T | a /= x\sete{}]: THEORY
 ...
\end{session}
\noindent But this introduces an asymmetry in that \texttt{a} and
\texttt{b} now belong to different types, and the type predicate still
must be provided up the entire hierarchy.

Using a theory as a parameter, we may instead define \texttt{th} as
follows.
\begin{session}
th: THEORY
 BEGIN
  T: TYPE,
  a, b: T
  A: AXIOM a /= b
  ...
 END th
\end{session}
\noindent We then parameterize using this theory (which is implicitly
imported):
\begin{session}
th_1[t: THEORY th]: THEORY ...
\end{session}
\noindent We have encapsulated the uninterpreted types and constants into
a theory, and this is now represented as a single parameter.  Axiom
\texttt{A} is visible within theory \texttt{th\_1}, and no proof
obligations are generated since no mapping was given for \texttt{th}.  Now
we can continue defining new theories as follows.
\begin{session}
th_2[t: THEORY th]: THEORY IMPORTING th_1[t] ...
th_3[t: THEORY th]: THEORY IMPORTING th_2[t] ...
  \vdots
\end{session}
\noindent None of these generate proof obligations, as no mappings are
provided.

We may now instantiate \texttt{th\_n}, for example, with the following.
\begin{session}
IMPORTING th_n[th\mapb{}T := int, a := 0, b := 1\mape{}]
\end{session}
\noindent Now the substituted form of the axiom becomes a proof obligation
which, when proved, provides evidence that the theory \texttt{th} is
consistent.

% \chapter{Theory Declarations and Theory Abbreviations}

With the introduction of theories as parameters, it is natural to allow
theory declarations that may be mapped, in the same way that instances may
be provided for theories as parameters.  Thus the
\texttt{group\_homomorphism} may be rewritten as follows:
\begin{session}
group_homomorphism: THEORY
 BEGIN
  G1, G2: THEORY group
  x, y: VAR G1.G
  f: VAR [G1.G -> G2.G]
  homomorphism?(f): bool = FORALL x, y: f(x + y) = f(x) + f(y)
  hom_exists: LEMMA EXISTS f: homomorphism?(f)
 END group_homomorphism
\end{session}
\noindent Again, the choice between using theories as parameters or theory
declarations is really a question of taste, as they are largely
interchangeable.

As with theories as parameters, copies must be made for \texttt{G1} and
\texttt{G2}.  Note that this means that there is a difference between
theory abbreviations and theory declarations, as the former do not involve
any copying.  We decided to use the old form of theory abbreviation to
define theory declarations, and to extend the \texttt{IMPORTING} expressions to
allow abbreviations, as shown in Figure~\ref{importing-bnf}.  Thus instead of
\begin{session}
funset: THEORY = sets[[int -> int]]
\end{session}
\noindent which creates a copy of sets, use
\begin{session}
IMPORTING sets[[int -> int]] AS funset
\end{session}
\noindent which imports \texttt{sets[[int -> int]]} and abbreviates it as
\texttt{funset}.

\begin{figure}
\setlength{\sessionboxwidth}{\linewidth}
\addtolength{\sessionboxwidth}{-\arrayrulewidth}
\addtolength{\sessionboxwidth}{-\tabcolsep}
\begin{boxedminipage}[b]{\sessionboxwidth}
\begin{bnf}

\production{Importing}
{\lit{IMPORTING} \ites{ImportingItem}{,}}

\production{ImportingItem}
{TheoryName \opt{\lit{AS} Id}}

\end{bnf}
\end{boxedminipage}
\caption{Grammar for Importings}\label{importing-bnf}
\end{figure}

\chapter{Prettyprinting Theory Instances}

Mappings can get fairly complex, especially if actual parameters are
involved, and it may be desirable to see the specified theory instance
displayed with all the substitutions performed.  To support this, we have
provided a new PVS command: \texttt{prettyprint-theory-instance}
(\texttt{M-x ppti}).  This takes two arguments: a theory instance, which
in general is a theory name with actual parameters and/or mappings, and a
context theory, in which the theory instance may be typechecked.  The
simplest way to use this command is to put the cursor on the theory name
as it appears in a theory as parameter, theory declaration, or
importing---when the command is issued it then defaults to the theory
instance under the cursor and the current theory is the default
context theory.  For example, putting the cursor on
\texttt{group\_homomorphism} in the following and typing \texttt{M-x ppti}
followed by two carriage returns\footnote{The first uses the theory name
instance at the cursor, and the second uses the current theory as the
context.} generates a buffer named \texttt{group\_homomorphism.ppi}.
All instances of a given theory generate the same buffer name.
\begin{session}
IMPORTING group_homomorphism[\mapb{}G := int, + := +, 0 := 0, - := -\mape{}
                             \mapb{}G := nzreal, + := *, 0 := 1,
                               - := LAMBDA (x: nzreal): 1/x\mape{}]
\end{session}
\noindent This buffer has the following contents.
\begin{session}
% Theory instance for
  % group_homomorphism[groups\mapb{} G := int, + := +,
  %                             - := -, 0 := 0 \mape{},
  %                    groups\mapb{} G := nzreal, + := *,
  %                             - := (LAMBDA (x: nzreal): 1 / x),
  %                             0 := 1 \mape{}]
group_homomorphism_instance: THEORY
 BEGIN

  IMPORTING groups\mapb{} G := int, + := +, - := -, 0 := 0 \mape{}

  IMPORTING groups\mapb{} G := nzreal, + := *,
                     - := (LAMBDA (x: nzreal): 1 / x), 0 := 1 \mape{}

  x, y: VAR int

  f: VAR [int -> nzreal]

  homomorphism?(f): bool =
    FORALL (x: int), (y: int): f(x + y) = f(x) * f(y)

  hom_exists: LEMMA EXISTS (f: [int -> nzreal]): homomorphism?(f)
 END group_homomorphism_instance
\end{session}
The group instances shown on
pages~\pageref{group-instances-start}--\pageref{group-instances-end}
provide more examples of the output produced by
\texttt{prettyprint-theory-instance}.

\chapter{Comparison with Other Systems}

In this chapter we compare PVS theory interpretations to existing
programming and specification mechanisms of other systems.
The \textsc{Ehdm} system~\cite{EHDM:Language} has a notion of a mapping
module that maps a source module to a target module.  When a mapping
module is typechecked, a new module is automatically created that
represents the substitution of the interpretations for the body of the
source theory.  Equality is allowed to be mapped in \textsc{Ehdm}, in
which case it must be mapped to an equivalence relation.  In PVS, mappings
are provided as a syntactic component of names, and are essentially an
extension of theory parameters.  Equality is not treated specially, but is
handled by mapping a given type to a quotient type.

IMPS~\cite{Farmer:imps-cade,Farmer94} also supports theory
interpretations.  It is similar to \textsc{Ehdm} in that it has a special
\texttt{def-translation} form that takes a source theory, target
theory, sort association list, and constant association list, and generates a
theory translation.  Obligations may be generated that ensure that every
axiom of the source theory is a theorem of the target theory.  If these
are proved the translation is treated as an interpretation.  There is no
mechanism for mapping equality.  As with both PVS and \textsc{Ehdm},
defined sorts and constants of the source theory are automatically
translated.  A more detailed comparison between IMPS and an earlier
version of PVS appears in an unpublished report by
Kamm\"{u}ller~\cite{Kammuller:comparison}.

In Maude~\cite{Maude} and its precursor OBJ~\cite{OBJ:intro} it is
possible to
define \texttt{modules} that represent transition systems of a rewrite
theory whose states are equivalence classes of ground terms and whose
transitions are inference rules in \emph{rewriting logic}.  A given module
may import another module, either \texttt{protecting} it, which means that
the importing module adds no \emph{junk} or \emph{confusion}, or
\texttt{including} it, which imposes no such restrictions.  In addition to
modules, Maude has \emph{theories}, which are used to declare module
interfaces.  These may appear as module parameters, as in
$M[X_{1}::T_{1},\ldots,X_{n}::T_{n}]$, where the $X_{i}$ are \emph{labels}
and the $T_{i}$ are names of theories.  These theory parameters (source
theories) may be instantiated by target theories or modules using
\emph{views}, which indicate how each sort, function, class, and message
of the source theory is mapped to the target theory.  However, Maude
currently does not support the generation of proof obligations from source
theory axioms, so views are simply theory translations, not
interpretations.

The programming language Standard ML~\cite{ML-report} has a module
system where modules are given by \emph{structures} with a given
\emph{signature}, and parametric modules are \emph{functors} mapping
structures of a given signature to structures.  The PVS mechanism
of using theories as parameters resembles SML functors but for a
specification language rather than a programming language. 
Sannella and Tarlecki~\cite{SannellaDT:essential-concepts97} describe a
version of the ML module system in which there are \emph{specifications}
containing \emph{sorts}, \emph{operations}, and \emph{axioms}.  For
example, the signature of stacks is the following.
\begin{eqnarray*}
\emph{STACK} = & \textbf{sorts} & \emph{stack} \\
               & \textbf{opns} & \emph{empty} : \emph{stack} \\
               &               & \emph{push} : \texttt{int} \times \emph{stack} \rightarrow \emph{stack} \\
               &               & \emph{pop} :
                                 \emph{stack} \rightarrow \emph{stack} \\
               &               & \emph{top} :
                                 \emph{stack} \rightarrow \texttt{int} \\
               &               & \emph{is\_empty} :
                                 \emph{stack} \rightarrow \texttt{bool} \\
               & \textbf{axioms} & \emph{is\_empty}(\emph{empty}) =
                                   \texttt{true} \\
               &               &
               \forall\emph{s}:\emph{stack}.\forall\emph{n}:\texttt{int}.
                 \emph{is\_empty}(\emph{push}(\emph{n},\emph{s}))
                     = \texttt{false} \\
               &               &
               \forall\emph{s}:\emph{stack}.\forall\emph{n}:\texttt{int}.
                 \emph{top}(\emph{push}(\emph{n},\emph{s})) = \emph{n} \\
               &               &
               \forall\emph{s}:\emph{stack}.\forall\emph{n}:\texttt{int}.
                 \emph{pop}(\emph{push}(\emph{n},\emph{s})) = \emph{s} \\
\end{eqnarray*}
The following algebra is a \emph{realization} of the above specification
that corresponds to that of \texttt{cstack} on page~\pageref{cstack}.
{\smaller\begin{alltt}
  structure S2 : STACK =
      struct
          type stack = (int -> int) * int
          val empty = ((fn k => 0), 0)
          fun push (n, (f, i))
                = ((fn k => if k = i then n else f k), i+1)
          fun pop (f, i) = if i = 0 then (f, 0) else (f, i-1)
          fun top (f, i) = if i = 0 then 0 else f(i-1)
          fun is_empty (f, i) = (i=0)
\end{alltt}}
Note however, that the stacks \emph{empty} and
\emph{pop}(\emph{push}(\texttt{6},\emph{empty})) are not equal.  Thus they
distinguish the \emph{observable} sorts, in this case \texttt{int} and
\texttt{bool}, which are the only data directly visible to the user.  The
above two terms are not \emph{observable computations}, so it does not
matter that they are different.  In general, two different algebras are
\emph{behaviorally equivalent} if all observable computations yield the
same results. Note that choosing observable values based on sorts is a bit
coarse: for example, there may be two \texttt{int}-valued variables, one of
which is observable and one that represents an internal pointer.  Mapping
to equivalence classes is more general, as it is easy to capture
behavioral equivalence.

The induction theorem prover Nqthm~\cite{boyer-moore88,BoyerGoldschlag91}
has a feature called \texttt{FUNCTIONALLY-INSTANTIATE} that can be used to
derive an instance of a theorem by supplying an interpretation for some of
the function symbols used in defining the theorem.  The corresponding
instances of any axioms concerning these function symbols must be
discharged.  Such axioms can be introduced as conservative extensions as
definitions with the \texttt{DEFUN} declaration or through witnessed
constraints using the \texttt{CONSTRAIN} declaration, or they can be
introduced nonconservatively through an \texttt{ADD-AXIOM}
declaration.  While the functional instantiation mechanism is similar in
flavor to PVS theory interpretations, the underlying logic of Nqthm is a
fragment of first-order logic whose expressive power is more limited
than the higher-order logic of PVS.  In addition, Nqthm lacks types and
structuring mechanisms such as parametric theories.

The \specware{} language~\cite{SrinivasJullig95} employs theory
interpretations as a mechanism for the stepwise refinement of
specifications into executable code.  \specware{} has constructs for
composing specifications while identifying the common components, and for
compositionally refining specifications so that the refinement of a
specification can be composed from the refinement of its components.
Unlike PVS, \specware{} has the ability to incorporate multiple logics
and translate specifications between these logics.  A theory is an
independent unit of specification in PVS and hence there is no support for
composing theories from other theories.  However, the operations in
\specware{} can largely be simulated by means of theories and theory
interpretations in PVS.

In summary, theory interpretation has been a standard tool in
specification languages since the early work on HDM~\cite{HDM:Handbook}
and Clear~\cite{BURSTALL&GOGUEN}.  PVS implements theory interpretations
as a simple extension of the mechanism for importing parametric theories.
PVS theory interpretations subsume the corresponding capabilities
available in other specification frameworks.


\chapter{Future Work}

A number of interesting extensions may be contemplated for
the future.

\paragraph{Mapping of interpreted types and constants---}

There are two aspects: one is simply a convenience where, for
example, we might have a tuple type declaration \texttt{T: TYPE = [T1, T2,
T3]} and want to map it to \texttt{position: TYPE = [real, real, real]} by
simply giving the map \texttt{\mapb{}T := position\mape{}}.

The second aspect is where the mapping is between two different kinds, for
example mapping a record type to a function type.  This requires
determining the corresponding components as well as making explicit the
underlying axioms.  For example, record types satisfy extensionality, and
if they are mapped to a different type the implicit extensionality axiom
must be translated to a proof obligation.

\paragraph{Rewriting with congruences---}

In theory substitution, if a type is mapped to a quotient type then
equality over this type is mapped to equality over the quotient type.
If $T$ is an uninterpreted type, $\equiv$ an equivalence relation over
$T'$, and $T'/\equiv$ the quotient type, then \texttt{=[$T$]} is mapped to
\texttt{=[$T'/\equiv$]}, which is equivalent to $\equiv$.  An equational
formula thus still has the form of a rewrite.  However, to apply such a
rewrite one generally needs to do some lifting.  The following is a simple
example.
\begin{session}
th: THEORY
 BEGIN
  T: TYPE
  a, b: T
  f, g: [T -> T]
  \ldots \emph{Some axioms involving f, g, a, and b}
  lem: LEMMA f(a) = g(b)
 END th
th2: THEORY
 BEGIN
  ==(x, y: int): bool = divides(3, x - y)
  IMPORTING th\mapb{}T := E(==),
                a := equiv_class(==)(2),
                b := equiv_class(==)(1),
                f := LAMBDA (x: E(==)): equiv_class(rep(x) - 1),
                g := LAMBDA (x: E(==)): equiv_class(rep(x) - 2)\mape{}
  \ldots
 END th2
\end{session}
\noindent To rewrite with \texttt{lem}, \texttt{a} must first be lifted to
its equivalence class, then the rewrite is applied and the result is then
projected back using \texttt{rep}.  To do this requires some modification
to the rewriting mechanism of the prover.

\paragraph{Consistency Analysis---}

With a single independent theory such as groups, it is easy to generate a
mapping in which all axioms become proof obligations, and see directly
that the theory is consistent.  On the other hand, if many theories are
involved in which compositions of mappings are involved, this may become
quite difficult.  What is needed is a tool that analyzes a mapped theory
to see if it is consistent, and reports on any remaining axioms and
uninterpreted declarations.  This is similar in spirit to proof chain
analysis, but works at the theory level rather than for individual
formulas.

\paragraph{Semantics of Mappings---}

The semantics of theory interpretations needs to be formalized and added
to the PVS semantics report~\cite{PVS-Semantics:TR}.

\chapter{Conclusion}

Theory interpretations are used to embed an interpretation of an abstract
theory in a more concrete one.  In this way, they allow an abstract
development to be reused at the more concrete level.  Theory
interpretations can be used to refine a specification down to code.
Theory interpretations can also be used to demonstrate the consistency of
an axiomatic theory relative to another theory.

Parametric theories in PVS provide some but not all of the functionality
of theory interpretations.  In particular, they do not allow an abstract
theory to be imported with only a partial parameterization.  Theory
interpretations have been implemented in PVS version 3.0, which will be
released in mid-2001.  The current implementation allows the
interpretation of uninterpreted types and constants in a theory, as well
as theory declarations.  PVS has also been extended so that a theory may
appear as a formal parameter of another theory.  This allows related sets
of parameters to be packaged as a theory.  Quotient types have been
defined within PVS and used to admit interpretations of types where the
equality on a source type is treated as an equivalence relation on a
target type.

Theory interpretations have been implemented in PVS as an extension of the
theory parameter mechanism.  This way, theory interpretations are
an extension of an already familiar concept in PVS and can be used in
place of theory parameters where there is a need for greater
flexibility in the instantiation.  The proof obligations generated by
theory interpretations are similar to those for parametric theories
with assumptions.  

A number of extensions related to theory interpretations remain to be
implemented.  First, we plan to extend theory interpretations to the case
of interpreted types and constants.  This poses some challenges since
there are implicit operations and axioms associated with certain type
constructors.  Second, the rewriting mechanisms of the PVS prover need to
be extended to rewrite relative to a congruence.  This means that if we
are only interested in $f(a)$ up to some equivalence that is preserved by
$f$, then we could rewrite $a$ up to equivalence rather than equality.
Third, the PVS semantics have to be extended to incorporate theory
interpretations.  Finally, the PVS ground evaluator has to be extended to
handle theory interpretations.  Currently, the ground evaluator generates
code corresponding to a parametric theory and this code is reused with the
actual parameters used as arguments to the operations.  Theory
interpretations cannot be treated as arguments in this manner since there
is no fixed set of parameters; parameters can vary according to the
interpretation.  Also, non-executable operations can become executable as
a result of the interpretation.

In summary, we believe that theory interpretations are a significant
extension to the PVS specification language.  Our implementation of this
in PVS3.0 is simple yet powerful.  We expect theory interpretations to be
a widely used feature of PVS.

\newpage
\bibliographystyle{alpha}
\addcontentsline{toc}{chapter}{Bibliography}
\bibliography{/homes/rushby/jmr,/homes/owre/tex/sam,/homes/shankar/tex}
\end{document}

% Master File: language.tex
% Document Type: LaTeX

\chapter{Name Resolution}\label{names}\label{resolution}

Names in \pvs\ are used to denote theories, variables, constants, and
formulas.  New names are introduced by declarations.  The syntax of names
is given in Figure~\ref{bnf-names}.

\pvsbnf{bnf-names}{Name Syntax}

The simplest form of a name is an \emph{idop}, \ie\ an identifier or
operator symbol.  This is generally all that is needed, unless names are
overloaded.

The overloading of names, both from different theories and within a single
theory, is allowed as long as there is some way for the system to
distinguish references to them.  Names from different theories may be
distinguished by prefixing them with the theory name.  Within a theory,
all names of the same kind must be unique, except for expression kinds;
which need only be unique up to the signature.  This is because the
signature is enough to distinguish these declarations.  For example, if
{\tt <} is declared to have signature {\tt [bool,int -> bool]}, the system
will recognize from the context that {\tt TRUE < 3} contains a reference
to this declaration, whereas {\tt 2 < 3} does not.\footnote{Of course,
this assumes that \texttt{TRUE} has not itself been overloaded.}  If the
use of the name is not enough to distinguish, coercion may be used to
specify the signature directly (see page~\pageref{coercions}).  Theory
parameters must be unique across all kinds.

There are three possible forms for names (two for theory names, which
appear in {\tt IMPORTING}s, {\tt EXPORTING} {\tt WITH}s, and theory
declarations).  Given a theory named {\em theoryid\/}, with formal
parameters $f_1,\ldots,f_n$, that contains a declaration named {\em
id\/}, the following three forms may be used to reference the
declaration in a theory that imports {\em theoryid\/}:
\begin{itemize}
\item {\em theoryid\/}{\tt [$a_1,\ldots,a_n$]}.{\em id\/}

\item {\em id\/}{\tt [$a_1,\ldots,a_n$]}

\item {\em id}

\end{itemize}
where the $a_i$ are expressions or type expressions that are compatible
with the formal parameters as described in Section~\ref{parameters}.
These forms are listed in order of increasing ambiguity---that is, names
that are given with just an id are far more likely to produce an ambiguity
than those further up.  Note that even the top form may be
ambiguous, as {\em id\/} may be declared more than once in {\em theoryid\/}.
If this is the case, then either the context will disambiguate the name or
a type will have to be supplied in the form of a coercion expression, \eg\
{\tt {\em id\/}:~nat}.  This kind of ambiguity is allowed only for
constants (including functions and recursive functions) and variables.

Names are resolved based on the expected type and the number and types of
arguments to which the name is applied.  The expected type is generally
determined from the context of the name, for example in
\begin{pvsex}
  c1: int = c2
\end{pvsex}
\texttt{c2} has expected type \texttt{int}.  For most expressions, this is
straight-forward, but applications create special problems.  For example,
in
\begin{pvsex}
  f: FORMULA c1 = c2
\end{pvsex}
we know that the equality (which \emph{is} an application) has range type
\texttt{boolean} since it is a formula, but this gives no information
about the types of the arguments.  We will first describe the simpler
situation, and then explain how names used as operators of an application
are resolved.

In general, the typechecker works by first collecting possible types for
the expressions, and then chooses from among the possible types using the
expected type, which is determined from the context of the expression.
The expected type is used to resolve ambiguities, but otherwise does not
contribute to the type of an expression.  Thus if \texttt{2 + 3}
typechecks, and \texttt{+} has not been redeclared, then it has type
\texttt{real} regardless of its context.  However, for the purpose of
checking for TCCs, it may be treated as having a different type depending
on the expected type and the available judgements.


% Document Type: LaTeX
% Master File: language.tex

\chapter{Abstract Datatypes}\label{datatypes}\label{adts}

PVS provides a powerful mechanism for defining abstract datatypes.  This
mechanism is akin to, but more sophisticated than, the \emph{shell}
principle of the Boyer-Moore prover~\cite{Boyer-Moore79}).  A PVS datatype
is specified by providing a set of \emph{constructors} along with
associated \emph{accessors} and \emph{recognizers}.  When a datatype is
typechecked, a new theory is created that provides the axioms and
induction principles needed to ensure that the datatype is the initial
algebra defined by the constructors.

\pvsbnf{bnf-adts}{Datatype Syntax}

The syntax for PVS datatypes is given in Figure~\ref{bnf-adts}.  Datatypes
may appear at the \emph{top-level} as with theory declarations, or
\emph{in-line} as a declaration within a theory.\footnote{Enumeration
types are actually in-line datatypes---see Section~\ref{enum-types}.}
Typechecking a top-level datatype named \texttt{foo} causes the generation
of a new PVS file named \texttt{foo\_adt.pvs} containing up to three
theories as described below.  Typechecking an in-line datatype has the
effect of adding new declarations to the current theory, effectively
replacing the in-line datatype.  In-line datatypes are more restricted:
they may not have formal parameters or assuming parts, and they will not
generate the recursive combinators described below.  The declarations
generated for an in-line datatype may be viewed using the
\texttt{M-x~prettyprint-expanded} command (see the \emph{PVS System
Guide}~\cite{PVS:userguide}).

\section{A Datatype Example: \texttt{stack}}\label{stacks-adt}
An example of a datatype is \texttt{stack}:
\begin{session}
  stack[T: TYPE]: DATATYPE
   BEGIN
    empty: empty?
    push(top:T, pop:stack): nonempty?
   END stack
\end{session}
The \texttt{stack} datatype has two \emph{constructors}, \texttt{empty} and
\texttt{push}, that allow stack elements to be constructed.  For example,
the term \texttt{push(1, empty)} is an element of type \texttt{stack[int]}.
The \emph{recognizers} \texttt{empty?}\ and \texttt{nonempty?}\ are predicates
over the \texttt{stack} datatype that are true when their argument is
constructed using the corresponding constructor.  Given a \texttt{stack}
element that is known to be \texttt{nonempty?}, the \emph{accessors}
\texttt{top} and \texttt{pop} may be used to extract the first and second
arguments.

Typechecking the \texttt{stack} specification automatically creates a new
file \texttt{stack\_adt.pvs}, that contains the material found in
the next five figures.  This new file contains three theories:
\texttt{stack\_adt}, \texttt{stack\_adt\_map}, and
\texttt{stack\_adt\_reduce}.

\pvstheory{stack_adtA-alltt}{Theory \texttt{stack\_adt} (continues)}{stack_adtA-alltt}
\pvstheory{stack_adtB-alltt}{Theory \texttt{stack\_adt} (continues)}{stack_adtB-alltt}
\pvstheory{stack_adtC-alltt}{Theory \texttt{stack\_adt} (continues)}{stack_adtC-alltt}
\pvstheory{stack_adtD-alltt}{Theory \texttt{stack\_adt\_map}}{stack_adtD-alltt}
\pvstheory{stack_adtE-alltt}{Theory \texttt{stack\_adt\_reduce}}{stack_adtE-alltt}

The first theory \texttt{stack\_adt} is parametric in type \texttt{T}.
This is a specification of ``stacks of \texttt{T}'', where \texttt{T} may
be instantiated by any defined type when the stacks datatype is imported.
Thus ``stacks of integers'' as well as ``stacks of stacks of integers''
may be defined using this theory.  The first few lines of the theory
define the main type of stacks \texttt{stack}, the recognizers
\texttt{emptystack?} and \texttt{nonemptystack?}, the constructors
\texttt{empty} and \texttt{push}, and the accessors \texttt{top} and
\texttt{pop} are declared.

The \texttt{stack\_ord} function is defined, and an axiom provided for
it's definition.  This is provided instead of a disjointness axiom,
because the disjointness axiom becomes difficult to generate and use if
the number of constructors is large.  The disjointness comes from the fact
that the natural numbers are distinct.  The \texttt{ord} function is then
defined to return \texttt{0} on an empty stack and \texttt{1} on a
nonempty stack.  This is the same function as \texttt{stack\_ord}, but is
easier to use.

Then a series of axioms are given.  The
\texttt{stack\_empty\_extensionality} axiom states that there is only one
bottom element of the datatype: \texttt{empty}.
\texttt{stack\_push\_extensionality} states that any two stacks that have
the same \texttt{top} and \texttt{pop} (have the same components) are the
same.  The \texttt{stack\_push\_eta} axiom states that \texttt{pop}ping
and \texttt{push}ing the same element off and onto a stack results in a
stack identical to the original.  \texttt{stack\_top\_push} says that if
you \texttt{push} and element on a stack, you get that same element when
you \texttt{pop} it back off.  \texttt{stack\_pop\_push} says that pushing
something on a stack and then popping it back off results in the original
stack.

The \texttt{stack\_inclusive} axiom states that all stacks are either
\texttt{empty?} or \texttt{nonempty?}.  The PVS prover builds this axiom
in, so that it rarely needs be cited by a user.

\newpage
The next axiom, \texttt{stack\_induction}, introduces an induction formula
for stacks stating that any predicate $p$ of stacks that
\begin{enumerate}
\item holds for the empty stack (the base case), and
\item if $p$ holds for some stack then $p$ holds for the result of
\texttt{push}ing anything of the right type onto that stack (the induction
step),
\end{enumerate}
then $p$ holds for all stacks.

Then some useful functions are defined over stacks.  The stack predicate
\texttt{every} takes as arguments a predicate over \texttt{T} and a stack
and returns \texttt{TRUE} iff all elements on the stack satisfy the given
predicate.  \texttt{every} is introduced in both curried and uncurried
forms.  The stack predicate \texttt{some} is dual to \texttt{every},
returning \texttt{TRUE} iff there is some element on the stack that
satisfies the predicate.  The \texttt{subterm} predicate takes two stacks
and returns \texttt{TRUE} if and only if the first argument stack is a
subterm of the second.  That is, if the second stack consists of the first
stack with some (perhaps zero) elements pushed onto it.  The \texttt{<<}
predicate is the strict (irreflexive) \texttt{subterm} predicate.  Thus
for all stacks $s$, \texttt{subterm}$(s,s)$ holds, but for no stack $s$
does \texttt{<<}$(s,s)$ hold.  An alternative equivalent definition of
\texttt{<<} is as follows:
\begin{pvsex}
  <<(x: stack, y: stack): boolean = subterm(x,y) AND NOT x = y
\end{pvsex}
However, this definition is more awkward to use in a proof, as the
recursion is hidden in the definition of \texttt{subterm}.  For this
reason the definitions for \texttt{every}, \texttt{some},
\texttt{subterm}, and \texttt{<<}, are each defined as standalone
functions, though some of them could be defined in terms of the others.

The last four declarations of the theory \texttt{stack\_adt} are functions
which reduce a stack to a natural number or to an ordinal.  These
functions are useful for simplifying the proof of termination of
user-defined functions over stacks.  Recall that PVS requires recursive
functions to include a \emph{measure}, which is used to generate
termination conditions.  The primary use of the recursive combinator is to
allow measure functions to be specified.  The function
\texttt{reduce\_nat} takes a natural number and a function.  The natural
number is used for the empty stack, and then for each element on the
stack, the input function is applied to the element from the stack and the
current reduced natural number, returning a natural number.  The function
\texttt{reduce\_nat} returns the final natural number.  The function
\texttt{REDUCE\_nat} is analogous to \texttt{reduce\_nat}, except that the
reducing function is also given the entire contents of the stack.  This
version of reduction can be useful for complicated measures that involve,
for example, the number of repeated elements appearing on the stack.  The
simpler form of reduce is difficult to apply to such situations.  The
functions \texttt{reduce\_ordinal} and \texttt{REDUCE\_ordinal} are
analogous to \texttt{reduce\_nat} and \texttt{REDUCE\_nat} except that
they return ordinal numbers instead of natural numbers.  It is rare that a
termination argument requires the use of ordinals, so the simpler
\texttt{reduce\_nat} form is more often used.  This completes the
description of the \texttt{stack\_adt} theory.

The second theory in the file \texttt{stack\_adt.pvs} is
\texttt{stack\_adt\_map}.  This theory takes two types \texttt{T} and
\texttt{T1} as parameters, imports the \texttt{stack\_adt} theory, and
defines a mapping from \texttt{stacks[T]} to \texttt{stacks[T1]}.  The
higher-order \texttt{map} function takes a function \texttt{f} of type
\texttt{[T -> T1]}, and a stack of \texttt{T}, and returns a stack of
\texttt{T1} obtained by applying \texttt{f} to each element on the input
stack.  \texttt{map} is defined in both curried and uncurried forms.
\texttt{map} couldn't reside in the \texttt{stack\_adt} theory because
that theory has only one type parameter, while the \texttt{map} functions
require two: In order to construct and access stacks in two theories,
\texttt{map} must be parameterized in the two types.

Also in the \texttt{stack\_adt\_map} is a relational \texttt{every}
function.  It lifts a relation \texttt{R} between \texttt{T} and \texttt{T1},
to stacks of \texttt{T} and \texttt{T1}.  It is true if the stacks are the
same size, and corresponding elements satisfy \texttt{R}.

The third and final theory generated from \texttt{stack\_pvs} is
\texttt{stack\_adt\_reduce}.  This theory provides a generalized version
of \texttt{reduce\_nat} and \texttt{REDUCE\_nat}.  It takes as parameters
a type \texttt{T} and a range type \texttt{range}.  It defines a
generalized \texttt{reduce} which reduces stacks of \texttt{T} to elements
of \texttt{range}.  The functions \texttt{reduce\_nat},
\texttt{REDUCE\_nat}, \texttt{reduce\_ordinal}, and
\texttt{REDUCE\_ordinal} could have been defined using
\texttt{stack\_adt\_reduce}, but the direct definitions are provided for
additional user convenience.  The generalized \texttt{reduce} can be used
to provide evidence of termination of user-defined functions, but the
predefined versions such as \texttt{reduce\_nat} are easier to use in most
cases.

\section{Datatype Details}

In general, a datatype declaration has the form
\begin{pvsex}
  adt: DATATYPE WITH SUBTYPES S\(\sb{1}\), \ldots, S\(\sb{n}\)
    BEGIN
     cons\(\sb1\)(acc\(\sb{11}\): T\(\sb{11}\), \ldots, acc\(\sb{1{n\sb1}}\): T\(\sb{1{n\sb1}}\)): rec\(\sb1\) : S\(\sb{i\sb{1}}\)
     \vdots
     cons\(\sb{m}\)(acc\(\sb{m1}\): T\(\sb{m1}\), \ldots, acc\(\sb{1n\sb{m}}\): T\(\sb{1n\sb{m}}\)): rec\(\sb{m}\) : S\(\sb{i\sb{m}}\)
    END adt
\end{pvsex}
%
where the \texttt{cons$_i$} are the
\emph{constructors}\index{constructor}\index{datatype!constructor}, the
\texttt{acc$_{ij}$} are the
\emph{accessors}\index{accessor}\index{datatype!accessor}, the
\texttt{T$_{ij}$} are type expressions, and the \texttt{rec$_i$} are
\emph{recognizers}\index{recognizer}\index{datatype!recognizer}.  Each
line is referred to as a \emph{constructor
specification}\index{constructor specification}\index{datatype!constructor
specification}.  There are a number of restrictions enforced on
constructor specifications:
\begin{itemize}

\item The datatype identifier may not be used for a recognizer,
accessor, or subtype:\newline
($\texttt{adt} \not\equiv \texttt{rec}_i$ for all $i$, $\texttt{adt}
\not\equiv \texttt{acc}_{ij}$ for all $i$ and $j$, and $\texttt{adt}
\not\equiv \texttt{S}_i$ for all $i$).

\item The subtype names must be unique:
($i \neq j \Rightarrow \texttt{S}_i \not\equiv \texttt{S}_j$)

\item Each subtype name must be used at least once.

\item The constructor names must be unique:
($i \neq j \Rightarrow \texttt{cons}_i \not\equiv \texttt{cons}_j$).

\item The recognizer names must be unique:
($i \neq j \Rightarrow \texttt{rec}_i \not\equiv \texttt{rec}_j$).

\item No identifier may be used as both a constructor and a recognizer:\newline
($\texttt{cons}_i \not\equiv \texttt{rec}_j$ forall $i$ and $j$).

\item Duplicate accessor identifiers are not allowed within a single
constructor specification:
($j \neq k \Rightarrow \texttt{acc}_{ij} \not\equiv \texttt{acc}_{ik}$).

\end{itemize}

As seen in the \texttt{stack} example, datatypes may be recursive; this is
the case when the type of one or more of the accessors reference the
datatype.  In PVS, all such occurrences must be positive, where a type
occurrence \texttt{T} is positive in a type expression $\tau$ iff either
\begin{itemize}
\item $\tau\equiv \texttt{T}$.

\item $\tau\equiv \{x:\tau'|p(x)\}$ and the occurrence \texttt{T} is
positive in $\tau'$.

\item $\tau\equiv [{\tau_1} \rightarrow {\tau_2}]$ and the occurrence
\texttt{T} is positive in $\tau_2$\@.  For example, \texttt{T} occurs
positively in \texttt{sequence[T]} where \texttt{sequence[T]} is defined
in the PVS prelude as the function type \texttt{[nat -> T]}\@.

\item $\tau \equiv [\tau_1,\ldots, \tau_n]$ and the occurrence \texttt{T}
is positive in some $\tau_i$.

\item $\tau\equiv [\#\ l_1 : \tau_1, \ldots, l_n : \tau_n\ \#]$ and the occurrence \texttt{T} is positive in some $\tau_i$\@. 

\item $\tau\equiv \mbox{\emph{datatype}}[\tau_1,\ldots, \tau_n]$, where
\emph{datatype} is a previously defined datatype and the occurrence
\texttt{T} is positive in $\tau_i$, where $\tau_i$ is a \emph{positive
parameter} of \emph{datatype}\@.
\end{itemize}

When a top-level datatype is given with formal type parameters, they are
checked for whether their occurrences are all positive; this is used as
described above for any datatype that imports this one, as well as
determining some of the declarations described below.

When a datatype is typechecked, a number of new declarations are
generated:
\begin{itemize}

\item The datatype identifier is used to create an uninterpreted type
declaration.  In general, the term \emph{datatype} refers to this type.

\item Each recognizer is used to declare an uninterpreted subtype of the
datatype.

\item Each subtype identifier is used to declare an interpreted type that
is the disjunction of the types given by the recognizers that reference
the subtype identifier in the constructor specification.

\item Each constructor and accessor is used to generate a constant
declaration.

\item An \texttt{\emph{id}\_ord} uninterpreted function is created, and an
axiom \texttt{\emph{id}\_ord\_defaxiom} defines its values.  This is
provided instead of a disjointness axiom, because the disjointness axiom
becomes difficult to generate and use when the number of constructors is
large.

\item An \texttt{ord} function is generated that gives a zero-based number
to each constructor (e.g., \texttt{ord(null) = 0} and \texttt{cons(1,null)
= 1}).  This is mostly useful for enumeration types.

\item An extensionality axiom is generated for each constructor
specification.

\item An eta axiom is generated for each constructor specification
that has accessors.

\item For each accessor an axiom is created that says that the accessor
composed with the corresponding constructor returns the correct value; \eg\
\begin{pvsex}
  acc\(\sb{ij}\)(cons\(\sb{i}\)(e\(\sb{i1}\),\ldots, e\(\sb{i{m\sb{i}}})\) = e\(\sb{ij}\)
\end{pvsex}

\item An inclusive axiom is generated that says that every element of
the datatype belongs to at least one recognizer subtype.  This axiom is
not actually needed in practice as the prover checks for this directly.

\item Two induction schemes are provided for proving properties of the
datatype.

\item If there is at least one constructor with accessors,\footnote{Note
that enumeration types have no accessors.}  and there are positive type
parameters to the datatype, then \texttt{every} and \texttt{some}
functions are defined that provide a predicate on the datatype in terms of
the positive types.

\item The \texttt{subterm} and \texttt{<<} (irreflexive subterm) functions
are defined, and an axiom is generated that states that \texttt{<<} is
well-founded.  This allows it to be used as an ordering relation in
recursive function definitions.

\item If there is at least one constructor with
accessors,\addtocounter{footnote}{-1}\footnotemark{} the
\texttt{reduce\_nat}, \texttt{REDUCE\_nat}, \texttt{reduce\_ordinal}, and
\texttt{REDUCE\_ordinal} recursion combinators are defined.  These provide
a means for defining notions like the size or depth of a datatype term.

Note that accessor subtypes involving the datatype are
``lifted''.  The following example shows why.
\begin{pvsex}
  dt: DATATYPE
   BEGIN
    c0: c0?
    c1(a1: \setb{}x: list[dt] | length(x) > 0\sete): c1?
    c2(a2: \setb{}x: list[dt] | every(c0?)(x)\sete): c2?
   END dt
\end{pvsex}
Consider the \texttt{reduc\_nat} function.  The signature for the lifted
mapping function for \texttt{c1} and \texttt{c2} are the same:
\texttt{[list[nat] -> nat]}.  It's obvious the mapping function for
\texttt{c2} function could have the signature \texttt{[\setb{}x: list[nat]
| length(x) > 0\sete{} -> nat]}, but there is no obvious way to map
\texttt{c2} without lifting it.  Since it is not trivial to determine
which predicates map nicely, we lift them all.  In the future we may
provide heuristics that refine this.

\item If some type parameter is positive a \texttt{map} function is
generated in a separate theory.  Every positive type parameter in the
datatype is associated with a pair of \texttt{map} parameters, which form
the domain and range of a corresponding function argument.  Given a set of
such functions and a term of the datatype, \texttt{map} returns a term
that has the same structure, but with the ``leaf'' elements replaced by
the function values.

\item A separate theory is generated for the \texttt{reduce} and
\texttt{REDUCE} functions.  These generalize the \texttt{reduce} functions
above to an arbitrary range type.

\end{itemize}

Note that in the stack example, the \texttt{stack} type is nonempty, since
\texttt{empty} is an element of \texttt{stack} even if the parameter type
\texttt{T} is instantiated with an empty type.  However, there is no
requirement that a datatype be nonempty, though if it is imported and a
constant is declared to be of that type, a TCC will be generated as
described on page~\pageref{emptytypes} in section~\ref{emptytypes}.

The \texttt{stack\_adt} theory is parameterized in the type \texttt{T},
and introduces the uninterpreted type \texttt{stack}.  Under normal
circumstances, this would imply no relation between, for example,
\texttt{stack[nat]} and \texttt{stack[int]}.  However, since every
occurrence of \texttt{T} in the accessor types is positive, we can infer
that \texttt{stack[nat]} is a subtype of \texttt{stack[int]}.  In general,
given a type $T$ and a subtype $S \equiv \setb{}x:T | p(x)\sete$, then
\texttt{stack[$S$]} is treated the same as $\setb{}s:
\texttt{stack[}T\texttt{]} | \texttt{every}(p)(s)\sete$.  When a datatype
has a mix of positive and nonpositive type parameters, the subtype
relation only holds for the positive ones.  For example, in the datatype
\begin{session}
  dt[T1, T2: TYPE, c: T1]: DATATYPE
   BEGIN
    c(a1: T1, a2: [T2 -> T1]): c?
   END dt
\end{session}
\texttt{T1} is positive and \texttt{T2} is not, so \texttt{dt[nat, nat,
0]} is a subtype of \texttt{dt[int, nat, 0]}, but is not a subtype of
\texttt{dt[nat, int, 0]}, nor is it a subtype of \texttt{dt[nat, nat, 1]}.

More complex datatypes lead to correspondingly more complex declarations;
for example, in the following contrived datatype
\begin{session}
  adt1[t1,t2: TYPE, c:t1]: DATATYPE
   BEGIN
    bottom: bottom?
    c1(a11:t1, a12: [t2 -> int]): c1?
    c2(a21:adt1, a22:[nat -> adt1], a23: list[adt1]): c2?
    c3(a31:[list[int] -> adt1],
       a32:[# a: adt1, b: [int -> adt1] #],
       a33:[adt1, [set[int] -> adt1]]) : c3?
   END adt1
\end{session}
the curried \texttt{every} is generated as follows:
\begin{session}
  every(p: PRED[t1])(a1: adt1):  boolean =
      CASES a1
        OF bottom: TRUE,
           c1(c11_var, c12_var): p(c11_var),
           c2(c21_var, c22_var, c23_var):
             every(p)(c21_var) AND
              every(every(p))(c22_var) AND every[adt1](every(p))(c23_var),
           c3(c31_var, c32_var, c33_var):
                  (FORALL (x1: list[int]): every(p)(c31_var(x1)))
              AND every(p)(a(c32_var))
              AND FORALL (x: int): every(p)(b(c32_var)(x))
              AND every(p)(c33_var`1)
              AND FORALL (x: set[int]): every(p)(c33_var`2(x))
        ENDCASES;
\end{session}
Note that this is only defined for predicates over \texttt{t1}, since
the occurrence of \texttt{t2} in the constructor specification for
\texttt{c2} is not positive.

As with record types, constructor selectors may be dependent.  Here is a
simple example.
\begin{session}
  depdt: DATATYPE
   BEGIN
    b: b?
    c(x: int, y: \setb{}z: int | z < x\sete): c?
   END depdt
\end{session}

\section{Datatype Subtypes}

The \texttt{WITH SUBTYPES} keyword introduces a set of subtype names.
These are useful, for example, in defining the nonterminals of a language.
For example, we might try to describe a simple typed lambda calculus:
\begin{eqnarray*}
T & ::= & B \;|\; T \rightarrow T \\
E & ::= & x \;|\; \lambda x:T.E \;|\; E(E)
\end{eqnarray*}
This is difficult to express using datatypes without subtypes, but is
reasonably straightforward with them:\footnote{\texttt{TYPE},
\texttt{LAMBDA}, and \texttt{VAR} are PVS keywords, so variants are used
here.}
\begin{session}
tlc: DATATYPE WITH SUBTYPES typ, expr
 BEGIN
 base_type(n:nat): base_type? : typ
 fun_type(dom, ran: typ): fun_type? : typ
 expr_var(n:nat): expr_var? : expr
 lambda_expr(lvar:(expr_var?), ltype: typ, lexpr: expr)
                            : lambda_expr? : expr
 application(fun, arg: expr): application? : expr
 END tlc
\end{session}
In addition to the usual generated declarations, this generates
\begin{session}
  typ((x: tlc)): boolean = base_type?(x) OR fun_type?(x);
  typ: TYPE = \setb{}x: tlc | base_type?(x) OR fun_type?(x)\sete
  expr((x: tlc)): boolean =
     expr_var?(x) OR lambda_expr?(x) OR application?(x);
  expr: TYPE =
     \setb{}x: tlc | expr_var?(x) OR lambda_expr?(x) OR application?(x)\sete
\end{session}
immediately after the declarations generated for the recognizers, so they
may be referenced in the accessor types.  Note that only a single
induction scheme is generated.  To induct over a particular subtype,
extend the property of interest to the entire datatype so that it returns
true for everything else.


\section{\texttt{CASES} Expressions}\label{cases-expressions}
\index{cases expressions}

The \texttt{CASES} expression uses a simple form of pattern-matching on
abstract datatypes.  Patterns are of the form $c(x_1,\ldots, x_n)$ where
$c$ is an $n$-ary constructor and $x_1,\ldots, x_n$ is a list of distinct
variables.  Patterns here are simple so that certain logical properties of
the expression are easy to check.  Patterns are not defined in the grammar
but in the type rules, since the notion of a variable or a constructor is
only defined in the type rules.

For example, if \texttt{x} is of type \texttt{stack}, the cases expression
\begin{pvsex}
  CASES x OF
    empty : FALSE,
    push(y, z) : even?(y) AND empty?(z)
  ENDCASES
\end{pvsex}
is \texttt{TRUE} if \texttt{x} is a singleton even integer, and otherwise is
false.  This expression can be translated into
\begin{pvsex}
  IF empty?(x)
     THEN FALSE
     ELSE LET (y, z) = (car(x), cdr(x))
           IN even?(y) AND empty?(z)
  ENDIF
\end{pvsex}

The \texttt{CASES} expression also allows an \texttt{ELSE} clause, which
comes last and covers all constructors not previously mentioned in a
pattern.  If the \texttt{ELSE} clause is missing, and not all constructors
have been mentioned, then a \emph{cases TCC}\index{cases
TCC}\index{TCC!cases} is generated which states that the expression is not
any of the missing elements.  For example, if the \texttt{x} above is
declared to be a subtype of \texttt{stack} in which \texttt{empty} is
excluded, then the \texttt{empty} case can safely be left out, and a \tcc\
will be generated that obligates the user to prove that \texttt{x} is not
\texttt{empty}.  There is a trade-off here between simpler specifications
and simpler verifications; if the \texttt{empty} case is left in, then
there is no obligation to prove, but the extra case clutters up the
specification, and can mislead the reader into thinking that the
\texttt{empty} case is possible.  In general, we feel that the
specification should be as perspicuous as possible, even if it means a
little more work behind the scenes.

%%% Local Variables: 
%%% mode: latex
%%% TeX-master: "language"
%%% End: 


\appendix
\chapter{The Grammar}\label{grammar}

The complete \pvs\ grammar is presented in this Appendix, along with a
discussion of the notation used in presenting the grammar.

The conventions used in the presentation of the syntax are as follows.
\index{syntax!conventions}

\begin{itemize}

\item Names in {\it italics\/} indicate syntactic classes and
metavariables ranging over syntactic classes.

\item The reserved words of the language are
      printed in \lit{tt font, UPPERCASE}.

\item An optional part {\it A\/} of a clause is enclosed in square brackets:
\opt{{\it A\/}}.

\item Alternatives in a syntax production are separated by a bar
(``\choice''); a list of alternatives that is embedded in the right-hand
side of a syntax production is enclosed in brackets, as in

\begin{bnf}
\production{ExportingName}
{IdOp \opt{\lit{:} \brc{TypeExpr \choice \lit{TYPE} \choice \lit{FORMULA}}}}
\end{bnf}


\item Iteration of a clause {\it B\/} one or more times is indicated by
enclosing it in brackets followed by a plus sign: \ite{{\it B\/}};
repetition zero or more times is indicated by an asterisk instead of the
plus sign: \rep{{\it B\/}}.

\item A double plus or double asterisk indicates a clause separator; for
example, \reps{{\it B\/}}{,} indicates zero or more repetitions of the
clause {\it B} separated by commas.

\item Other items printed in tt font on the right hand side of
      productions are literals.  Be careful to distinguish where BNF
symbols occur as literals, \eg\ the BNF brackets \brc{} versus the
literal brackets \lit{\{\}}.

\end{itemize}

\subsubsection*{Specification}
\par\noindent
\spvsbnf{bnf-theory}

\subsubsection*{Importings and Exportings}
\par\noindent
\spvsbnf{bnf-exporting}

\subsubsection*{Assumings}
\par\noindent
\spvsbnf{bnf-assuming}

\subsubsection*{Theory Part}
\par\noindent
\spvsbnf{bnf-theory-part}

\subsubsection*{Declarations}
\par\noindent
\spvsbnf{bnf-decls}

\subsubsection*{Type Expressions}
\par\noindent
\spvsbnf{bnf-type-expr}

\subsubsection*{Expressions}
\par\noindent
\spvsbnf{bnf-expr}

\subsubsection*{Expressions (continued)}
\par\noindent
\spvsbnf{bnf-expr-aux}

\subsubsection*{Names}
\par\noindent
\spvsbnf{bnf-names}

\subsubsection*{Identifiers}
\par\noindent
\spvsbnf{bnf-lexical}

\subsubsection*{Datatypes}
\par\noindent
\spvsbnf{bnf-adts}

%% Derived from John Rushby's prelude.tex, modified for NFSS2
%
% define variants of the \LaTeX macro that avoid using \sc
% for use in headings
%

% Define fonts that work in math or text mode
\def\dwimrm#1{\ifmmode\mathrm{#1}\else\textrm{#1}\fi}
\def\dwimsf#1{\ifmmode\mathsf{#1}\else\textsf{#1}\fi}
\def\dwimtt#1{\ifmmode\mathtt{#1}\else\texttt{#1}\fi}
\def\dwimbf#1{\ifmmode\mathbf{#1}\else\textbf{#1}\fi}
\def\dwimit#1{\ifmmode\mathit{#1}\else\textit{#1}\fi}
\def\dwimnormal#1{\ifmmode\mathnormal{#1}\else\textnormal{#1}\fi}

\def\BigLaTeX{{\rm L\kern-.36em\raise.3ex\hbox{\small\small A}\kern-.15em
    T\kern-.1667em\lower.7ex\hbox{E}\kern-.125emX}}
\def\BoldLaTeX{{\bf L\kern-.36em\raise.3ex\hbox{\small\small\bf A}\kern-.15em
    T\kern-.1667em\lower.7ex\hbox{E}\kern-.125emX}}
%\def\labelitemi{$\bullet$}
\def\labelitemii{$\circ$}
\def\labelitemiii{$\star$}
\def\labelitemiv{$\diamond$}
\newcommand{\tcc}{{\small\small TCC}}
\newcommand{\tccs}{\tcc s}
\newcommand{\emacs}{{Emacs}}
\newcommand{\Emacs}{\emacs}
\newcommand{\ehdm}{{E{\small\small HDM}}}
\newcommand{\Ehdm}{\ehdm}
\newcommand{\tm}{$^{\mbox{\tiny TM}}$}
\newcommand{\hozline}{{\noindent\rule{\textwidth}{0.4mm}}}

\newcommand{\allclear}%
  {\mbox{\boldmath$\stackrel{\raisebox{-.2ex}[0pt][0pt]%
              {$\textstyle\oslash$}}{\displaystyle\bot}$}}

\newenvironment{private}{}{}

\newenvironment{smalltt}{\begin{alltt}\small}{\end{alltt}}

\newlength{\hsbw}

\newenvironment{session}%
  {\begin{flushleft}
   \setlength{\hsbw}{\linewidth}
   \addtolength{\hsbw}{-\arrayrulewidth}
   \addtolength{\hsbw}{-\tabcolsep}
   \begin{tabular}{@{}|c@{}|@{}}\hline 
   \begin{minipage}[b]{\hsbw}
   \begingroup\small\mbox{ }\\[-1.8\baselineskip]\begin{alltt}}
  {\end{alltt}\endgroup\end{minipage}\\ \hline 
   \end{tabular}
   \end{flushleft}}

\newenvironment{smallsession}%
  {\begin{flushleft}
   \setlength{\hsbw}{\linewidth}
   \addtolength{\hsbw}{-\arrayrulewidth}
   \addtolength{\hsbw}{-\tabcolsep}
   \begin{tabular}{@{}|c@{}|@{}}\hline 
   \begin{minipage}[b]{\hsbw}
   \begingroup\footnotesize\mbox{ }\\[-1.8\baselineskip]\begin{alltt}}%
  {\end{alltt}\endgroup\end{minipage}\\ \hline 
   \end{tabular}
   \end{flushleft}}

\newenvironment{spec}%
  {\begin{flushleft}
   \setlength{\hsbw}{\textwidth}
   \addtolength{\hsbw}{-\arrayrulewidth}
   \addtolength{\hsbw}{-\tabcolsep}
   \begin{tabular}{@{}|c@{}|@{}}\hline 
   \begin{minipage}[b]{\hsbw}
   \begingroup\small\mbox{ }\\[-0.2\baselineskip]}%
  {\endgroup\end{minipage}\\ \hline 
   \end{tabular}
   \end{flushleft}}

\newcommand{\memo}[1]%
  {\mbox{}\par\vspace{0.25in}%
   \setlength{\hsbw}{\linewidth}\addtolength{\hsbw}{-1.5ex}%
   \noindent\fbox{\parbox{\hsbw}{{\bf Memo: }#1}}\vspace{0.25in}}

\newcommand{\nb}[1]%
  {\mbox{}\par\vspace{0.25in}%
   \setlength{\hsbw}{\linewidth}\addtolength{\hsbw}{-1.5ex}%
   \noindent\fbox{\parbox{\hsbw}{{\bf Note: }#1}}\vspace{0.25in}}

\newcommand{\comment}[1]{}
\newcommand{\exfootnote}[1]{}
%\newcommand{\ifelse}[2]{#1}
\sloppy
\clubpenalty=100000
\widowpenalty=100000
%\displaywidowpenalty=100000
\setcounter{secnumdepth}{3} 
\setcounter{tocdepth}{3}
\setcounter{topnumber}{9}
\setcounter{bottomnumber}{9}
\setcounter{totalnumber}{9}
\renewcommand{\topfraction}{.99}
\renewcommand{\bottomfraction}{.99}
\renewcommand{\floatpagefraction}{.01}
\renewcommand{\textfraction}{.2}
\font\largett=cmtt10 scaled\magstep1
\font\Largett=cmtt10 scaled\magstep2
\font\hugett=cmtt10 scaled\magstep3


%\addcontentsline{toc}{chapter}{Bibliography}
\bibliographystyle{plain}
\bibliography{../pvs}

%\addcontentsline{toc}{chapter}{Index}   %% Put entry in T-O-C
%%\printindex  %% printindex makes extra call to "theindex"
{\smaller
\printindex
%% Document Type: LaTeX
% Master File: language.tex
\documentclass[12pt]{book}
\usepackage{alltt}
\usepackage{makeidx}
\usepackage{relsize}
\usepackage{boxedminipage}
\usepackage{url}
\usepackage{../../pvs}
\usepackage{../makebnf}
\usepackage[chapter]{tocbibind}
\usepackage{fancyvrb}
\usepackage[dvipsnames,usenames]{color}

\usepackage{amssymb}
\usepackage{mathpazo}
\usepackage{fontspec}
\setmainfont[Ligatures=TeX]{XITS}
\setmonofont{DejaVu Sans Mono}[Scale=MatchLowercase]
%\setmonofont{Free Mono}[Scale=0.8]
\usepackage[math-style=ISO]{unicode-math}
\renewcommand{\leadsto}{\rightsquigarrow}
%\setmathfont{XITS Math}

\topmargin -10pt
\textheight 8.5in
\textwidth 6.0in
\headheight 15 pt
\columnwidth \textwidth
\oddsidemargin 0.5in
\evensidemargin 0.5in   % fool system for page 0
\setcounter{topnumber}{9}
\renewcommand{\topfraction}{.99}
\setcounter{bottomnumber}{9}
\renewcommand{\bottomfraction}{.99}
\setcounter{totalnumber}{10}
\renewcommand{\textfraction}{.5}
\renewcommand{\floatpagefraction}{.1}
\usepackage{fancyhdr}
\pagestyle{fancy}
\raggedbottom

%\setcounter{secnumdepth}{1}

\index{type correctness condition|see{TCC}}
\makeindex

\usepackage[bookmarks=true,hyperindex=true,colorlinks=true,linkcolor=Brown,citecolor=blue,backref=page,pagebackref=true,plainpages=false,pdfpagelabels]{hyperref}

%\input{/project/pvs/doc/prelude}
\def\labelitemii{$\circ$}
\def\labelitemiii{$\star$}
\def\labelitemiv{$\diamond$}
\newcommand{\tcc}{{\small\small TCC}}
\newcommand{\tccs}{\tcc s}

%\renewcommand{\memo}[1]{\mbox{}\par\vspace{0.25in}\noindent\fbox{\parbox{.95\linewidth}{{\bf Memo: }#1}}\vspace{0.25in}}

\newcommand{\eg}{{\em e.g.\/},}
\newcommand{\ie}{{\em i.e.\/},}

\newcommand{\pvs}{PVS}

\newcommand{\ch}{\choice}
\newcommand{\rsv}[1]{{\rm\tt #1}}

\newcommand{\lpvstheory}[3]{\figurehead{\hozline\smaller\smaller\begin{alltt}}%
                           \figuretail{\end{alltt}\vspace{-0in}\hozline}%
                           \figurelabel{#3}\figurecap{#2}%
                           \begin{longfigure}\input{#1}\end{longfigure}}

\newcommand{\bpvstheory}[3]
{\begin{figure}[b]\begin{boxedminipage}{\textwidth}%
      {\smaller\smaller\begin{alltt} \input{#1}\end{alltt}}\end{boxedminipage}%
    \caption{#2}\label{#3}\end{figure}}

\newcommand{\spvstheory}[1]
{\vspace{0.1in}\par\noindent\begin{boxedminipage}{\textwidth}%
    {\smaller\smaller\begin{alltt} \input{#1}\end{alltt}}\end{boxedminipage}\vspace{0.1in}%
}

\CustomVerbatimEnvironment{pvsex}{Verbatim}{commandchars=\\\{\},frame=single,fontsize=\relsize{-1}}

\CustomVerbatimCommand{\pvsinput}{VerbatimInput}{commandchars=\\\{\},frame=single,fontsize=\relsize{-1}}

\newcommand{\pvstheory}[3]
  {\begin{figure}[htb]%
      \pvsinput{#1}%
      \caption{#2}\label{#3}%
    \end{figure}}

% \newenvironment{pvsex}%
%   {\setlength{\topsep}{0in}\smaller\begin{alltt}}%
%   {\end{alltt}}

\newcommand{\pvsbnf}[2]
  {\begin{figure}[htb]\begin{boxedminipage}{\textwidth}%
   \input{#1}\end{boxedminipage}\caption{#2}\label{#1}\end{figure}}

\newcommand{\spvsbnf}[1]
  {\begin{boxedminipage}{\textwidth}\input{#1}\end{boxedminipage}}

\newcommand{\pidx}[1]{{\rm #1}} % primary index entry
\newcommand{\sidx}[1]{{\rm #1}} % secondary index entry
\newcommand{\cmdindex}[1]{\index{#1@\cmd{#1}}}
\newcommand{\icmd}[1]{\cmd{#1}\cmdindex{#1}}
\newcommand{\iecmd}[1]{\ecmd{#1}\cmdindex{#1}}
\newcommand{\buf}[1]{\texttt{#1}}
\newcommand{\ibuf}[1]{\buf{#1}\index{#1 buffer@\buf{#1} buffer}\index{buffers!\buf{#1}}}

\newenvironment{pvscmds}%
  {\par\noindent\smaller%
   \begin{tabular*}{\textwidth}{|l@{\extracolsep{\fill}}l@{\extracolsep{\fill}}l|}\hline%
     {\it Command} & {\it Aliases} & {\it Function}\\ \hline}%
  {\hline\end{tabular*}\vspace{0.1in}}

\newenvironment{pvscmdsna}%
  {\par\noindent\smaller%
   \begin{tabular*}{\textwidth}{|l@{\extracolsep{\fill}}l|}\hline%
     {\it Command} & {\it \,\,Function}\\ \hline}%
  {\hline\end{tabular*}\vspace{0.1in}}

\newcommand{\cmd}[1]{\texttt{#1}}
\newcommand{\ecmd}[1]{{\tt M-x #1}}

\newcommand{\latex}{\LaTeX}                  %  LaTeX
\newcommand{\sun}{{S{\smaller\smaller UN}}}                 %  Sun
\newcommand{\sparc}{{S{\smaller\smaller PARC}}}             %  Sparc
\newcommand{\sunos}{{S{\smaller\smaller UN}OS}}             %  SunOS
\newcommand{\solaris}{{\em Solaris\/}}        %  Solaris
\newcommand{\sunview}{{S{\smaller\smaller UN}V{\smaller\smaller IEW}}} %SunView
\newcommand{\unix}{{U{\smaller\smaller NIX}}}               %  Unix
\newcommand{\lisp} {{\sc Lisp}}              %  Lisp
\newcommand{\gnu}{{Gnu Emacs}}           %  Gnu Emacs
\newcommand{\gnuemacs}{{Gnu Emacs}}      %  Gnu Emacs
\newcommand{\emacsl}{{Emacs-Lisp}}       %  Emacs Lisp
\newcommand{\shell}{{\sc Csh}}               %  C-shell

\newcommand{\update}[3]{#1\{#2\leftarrow #3\}}
\newcommand{\interp}[3]{\cal{M}\dlb {\tt #1 : #2 }\drb #3}
\newcommand{\myforall}[2]{(\forall{#1 .}\ #2)}
\newcommand{\myexists}[2]{(\exists{#1 .}\ #2)}
\newcommand{\mth}[1]{$ #1 $}
\newcommand{\labst}[2]{(\lambda{#1}.\ #2)}
\newcommand{\app}[2]{(#1\ #2)}
\newcommand{\problem}[1]{{\bf Exercise: } {\em #1}}
\newcommand{\rectype}[1]{[\# 1 \#]}
\newcommand{\recttype}[1]{{\tt [\# 1 \#]}}
\newcommand{\dlb}{\lbrack\!\lbrack}
\newcommand{\drb}{\rbrack\!\rbrack}
\newcommand{\cross}{\times}
\newcommand{\key}[1]{{\tt #1}}
\newcommand{\keyindex}[1]{\index{#1@\key{#1}}}
\newcommand{\ikey}[1]{\key{#1}\keyindex{#1}}
\newcommand{\keyword}[1]{{\smaller\texttt{#1}}}

\newenvironment{keybindings}%
  {\begin{center}\begin{tabular}{|l|l|}\hline Key & Function\\ \hline}%
  {\hline\end{tabular}\end{center}}
\def\rmif{\mbox{\bf if\ }}
\def\rmiff{\mbox{\bf \ iff \ }}
\def\rmthen{\mbox{\bf \ then }}
\def\rmelse{\mbox{\bf \ else }}
\def\rmend{\mbox{\bf end}}
\def\rmendif{\mbox{\bf \ endif}}
\def\rmotherwise{\mbox{\bf otherwise}}
\def\rmwith{\mbox{\bf \ with\ }}
\def\mapb{\char"7B\char"7B}
\def\mape{\char"7D\char"7D}
\def\setb{\char"7B}
\def\sete{\char"7D}

% ---------------------------------------------------------------------
% Macros for little PVS sessions displayed in boxes.
%
% Usage: (1) \setcounter{sessioncount}{1} resets the session counter
%
%        (2) \begin{session*}\label{thissession}
%             .
%              < lines from PVS session >
%             .
%            \end{session*}
%
%            typesets the session in a numbered box in ALLTT mode.
%
%  session instead of session* produces unnumbered boxes
%
%  Author: John Rushby
% ---------------------------------------------------------------------
\newlength{\hsbw}
\newenvironment{session}{\begin{flushleft}
 \setlength{\hsbw}{\linewidth}
 \addtolength{\hsbw}{-\arrayrulewidth}
 \addtolength{\hsbw}{-\tabcolsep}
 \begin{tabular}{@{}|c@{}|@{}}\hline 
 \begin{minipage}[b]{\hsbw}
% \begingroup\small\mbox{ }\\[-1.8\baselineskip]\begin{alltt}}{\end{alltt}\endgroup\end{minipage}\\ \hline
 \begingroup\sessionsize\vspace*{1.2ex}\begin{alltt}}{\end{alltt}\endgroup\end{minipage}\\ \hline
 \end{tabular}
 \end{flushleft}}
\newcounter{sessioncount}
\setcounter{sessioncount}{0}
\newenvironment{session*}{\begin{flushleft}
 \refstepcounter{sessioncount}
 \setlength{\hsbw}{\linewidth}
 \addtolength{\hsbw}{-\arrayrulewidth}
 \addtolength{\hsbw}{-\tabcolsep}
 \begin{tabular}{@{}|c@{}|@{}}\hline 
 \begin{minipage}[b]{\hsbw}
 \vspace*{-.5pt}
 \begin{flushright}
 \rule{0.01in}{.15in}\rule{0.3in}{0.01in}\hspace{-0.35in}
 \raisebox{0.04in}{\makebox[0.3in][c]{\footnotesize \thesessioncount}}
 \end{flushright}
 \vspace*{-.57in}
 \begingroup\small\vspace*{1.0ex}\begin{alltt}}{\end{alltt}\endgroup\end{minipage}\\ \hline 
 \end{tabular}
 \end{flushleft}}
\def\sessionsize{\footnotesize}
\def\smallsessionsize{\small}
\newenvironment{smallsession}{\begin{flushleft}
 \setlength{\hsbw}{\linewidth}
 \addtolength{\hsbw}{-\arrayrulewidth}
 \addtolength{\hsbw}{-\tabcolsep}
 \begin{tabular}{@{}|c@{}|@{}}\hline 
 \begin{minipage}[b]{\hsbw}
 \begingroup\smallsessionsize\mbox{ }\\[-1.8\baselineskip]\begin{alltt}}{\end{alltt}\endgroup\end{minipage}\\ \hline 
 \end{tabular}
 \end{flushleft}}
\newenvironment{spec}{\begin{flushleft}
 \setlength{\hsbw}{\textwidth}
 \addtolength{\hsbw}{-\arrayrulewidth}
 \addtolength{\hsbw}{-\tabcolsep}
 \begin{tabular}{@{}|c@{}|@{}}\hline 
 \begin{minipage}[b]{\hsbw}
 \begingroup\small\mbox{
}\\[-0.2\baselineskip]}{\endgroup\end{minipage}\\ \hline 
 \end{tabular}
 \end{flushleft}}
\newcommand{\memo}[1]{\mbox{}\par\vspace{0.25in}%
\setlength{\hsbw}{\linewidth}%
\addtolength{\hsbw}{-2\fboxsep}%
\addtolength{\hsbw}{-2\fboxrule}%
\noindent\fbox{\parbox{\hsbw}{{\bf Memo: }#1}}\vspace{0.25in}}
\newcommand{\nb}[1]{\mbox{}\par\vspace{0.25in}\setlength{\hsbw}{\linewidth}\addtolength{\hsbw}{-1.5ex}\noindent\fbox{\parbox{\hsbw}{{\bf Note: }#1}}\vspace{0.25in}}

%%% Local Variables: 
%%% mode: latex
%%% TeX-master: t
%%% End: 


\begin{document}

\begin{titlepage}
\renewcommand{\thepage}{title}
\vspace*{1in}
\noindent
\rule[1pt]{\textwidth}{2pt}
\begin{center}
\newfont{\pvstitle}{cmss17 scaled \magstep4}
\textbf{\pvstitle PVS Language Reference}
\end{center}
\begin{flushright}
{\Large Version 7.1 {\smaller$\bullet$} August 2020}
\end{flushright}
\rule[1in]{\textwidth}{2pt}
\vspace*{2in}
\begin{flushleft}
S.~Owre\\
N.~Shankar\\
J.~M.~Rushby\\
D.~W.~J.~Stringer-Calvert\\
{\smaller\url{{Owre,Shankar,Rushby,Dave_SC}@csl.sri.com}}\\
{\smaller\url{http://pvs.csl.sri.com/}}
\end{flushleft}
\vspace*{1in}
\vbox{\hbox to \textwidth{{\Large SRI International\hfill}}%
\hbox to \textwidth{{\small\sf%
Computer Science Laboratory $\bullet$ 333 Ravenswood Avenue $\bullet$ Menlo Park CA 94025\hfil}}}
\end{titlepage}

\renewcommand{\chaptermark}[1]{\markboth{{\em #1}}{}\markright{{\em #1}}}
\renewcommand{\sectionmark}[1]{\markright{\thesection \em \ #1}}
%\lhead[\thepage]{\rightmark}
%\cfoot{\protect\small\bf \fbox{PVS 2.3 DRAFT}}
%\cfoot{}
%\rhead[\leftmark]{\thepage}
\thispagestyle{empty}

\newpage
\renewcommand{\thepage}{ack}

\noindent\textbf{NOTE:} This manual is in the process of being updated.
Almost everything stated here is still correct, but incomplete due to the
many new features that have been introduced into PVS over the years.  The
release notes should be consulted for the most current information.

\vspace*{6in}\noindent
The initial development of PVS was funded by SRI International.
Subsequent enhancements were partially funded by SRI and by NASA
Contracts NAS1-18969 and NAS1-20334, NRL Contract N00014-96-C-2106,
NSF Grants CCR-9300044, CCR-9509931, and CCR-9712383, AFOSR contract
F49620-95-C0044, and DARPA Orders E276, A721, D431, D855, and E301.
\newpage
\pagenumbering{roman}
\setcounter{page}{1}

\tableofcontents
%\listoffigures

%\chapter{The PVS Specification Language}

%% Master File: language.tex
\addcontentsline{toc}{chapter}{\protect\numberline{}Preface}
\vspace{4in}
{\Huge\bf Preface}\linebreak
\vspace{.75in}

%\chapter{Preface}

This report presents a description of the \pvs\ specification language,
as implemented in Version 1.0 beta of the \pvs\ specification and
verification environment.  It is intended to provide a reference of all
of the features of the language, including the complete grammar, some
examples, and an informal semantics. This report is one of several
needed to effectively use \pvs.  Companion documents are devoted to the
use of the system~\cite{PVS:userguide}, the user of the
prover~\cite{PVS:prover}, a tutorial introduction~\cite{PVS:tutorial},
and a semantics~\cite{PVS:semantics}.

\memo{Give prerequisites to using \pvs.}

The \pvs\ system is the culmination of the effort of a large number of
people over many years, drawing heavily from the research and experience
gained from E{\sc
hdm}~\cite{EHDM:Userguide,EHDM:Language,EHDM:semantics,EHDM:supplement,EHDM:tutorial}.
The primary contributers to E{\sc hdm} in rough chronological order
were Michael Melliar-Smith, Richard Schwartz, Rob Shostak, Judith Crow,
Friedrich von Henke, Stan Jefferson, Rosanna Lee, John Rushby, Mark
Stickel, Natarajan Shankar, Sam Owre, David Cyrluk, Steven Phillips,
and Carl Witty.
%In addition to those named above, valuable contributions were made by
%Dorothy Denning, Brian Fromme, Allen van Gelder, Dwight Hare, Peter
%Ladkin, Sheralyn Listgarten, Jeff Miner, Paul Oppenheimer, Jeff
%Reninger, and Lorna Shinkle.
\pvs\ is primarily the work of John Rushby, Natarajan Shankar, Sam Owre,
Friedrich von Henke, David Cyrluk, and Carl Witty.

The present version of the \pvs\ Language Description was assembled by
Sam Owre, Natarajan Shankar, and John Rushby.



\cleardoublepage
\pagenumbering{arabic}
\setcounter{page}{1}

\setcounter{topnumber}{9}
\renewcommand{\topfraction}{.99}
\setcounter{bottomnumber}{9}
\renewcommand{\bottomfraction}{.99}
\setcounter{totalnumber}{10}
\renewcommand{\textfraction}{.01}
\renewcommand{\floatpagefraction}{.01}

% Document Type: LaTeX
% Master File: intro.tex

\chapter{Introduction}

PVS is a \emph{P}rototype \emph{V}erification \emph{S}ystem for the
development and analysis of formal specifications.  The PVS system
consists of a specification language, a parser, a typechecker, a prover,
specification libraries, and various browsing tools.  This document
primarily describes the specification language and is meant to be used as
a reference manual.  The \emph{PVS System Guide}~\cite{PVS:userguide} is to
be consulted for information on how to use the system to develop
specifications and proofs.  The \emph{PVS Prover Guide}~\cite{PVS:prover}
is a reference manual for the commands used to construct proofs.

In this section, we provide a brief summary of the PVS specification
language, enumerate the key design principles behind the language, and
provide a brief example.

The following sections provide more details on the various features of
the language.  The lexical aspects of the language are detailed in
Section~\ref{lexical}.  Section~\ref{declarations} describes
declarations, Section~\ref{types} describes type expressions,
Section~\ref{expressions} describes expressions, and
Section~\ref{theories} describes theories, theory parameters, and
imports and exports of names.  Section~\ref{names} describes names and
name resolution, and Section~\ref{adts} describes the datatype
facility of PVS.  Finally, 
Appendix~\ref{grammar} provides the grammar of the language.

\section{Summary of the PVS Language}

A PVS specification consists of a collection of \emph{theories}.
Each theory consists of a \emph{signature} for the type names and
constants introduced in the theory, and the axioms, definitions, and
theorems associated with the signature.  For example, a typical
specification for a queue would introduce the \texttt{queue} type and the
operations of \texttt{enq}, \texttt{deq}, and \texttt{front} with their
associated types.  In such a theory, one can either define a
representation for the \texttt{queue} type and its associated operations in
terms of some more primitive types and operations, or merely axiomatize
their properties.  A theory can build on other theories: for example, a
theory for ordered binary trees could build on the theory for
binary trees.  A theory can be \emph{parametric} in certain specified
types and values: as examples, a theory of queues can be parametric in
the maximum queue length, and a theory of ordered binary trees can be
parametric in the element type as well as the ordering relation.  It is
possible to place constraints, called \emph{assumptions}, on the
parameters of a theory so that, for instance, the ordering relation
parameter of an ordered binary tree can be constrained to be a total
ordering.

The PVS specification language is based on simply typed higher-order
logic.  Within a theory, \emph{types} can be defined starting from
\emph{base} types (Booleans, numbers, etc.) using the function, record,
and tuple type constructions.  The \emph{terms} of the language can be
constructed using function application, lambda abstraction, and record and
tuple construction.

There are a few significant enhancements to the simply typed language
above that lend considerable power and sophistication to PVS.  New
uninterpreted base types may be introduced.  One can define a
\emph{predicate subtype} of a given type as the subset of individuals in a
type satisfying a given predicate: the subtype of nonzero reals is
written as \texttt{\{x:real | x /= 0\}}.  One benefit of such
subtyping is that when an operation is not defined on all the elements of
a type, the signature can directly reflect this.  For example, the
division operation on reals is given a type where the denominator is
constrained to be nonzero.  Typechecking then ensures that
division is never applied to a zero denominator.  Since the predicate used
in defining a predicate subtype is arbitrary, typechecking is undecidable
and may lead to proof obligations called \emph{type correctness
conditions} (TCCs).  The user is expected to discharge these proof
obligations with the assistance of the PVS prover.  The PVS type system
also features dependent function, record, and tuple type constructions.
There is also a facility for defining a certain class of abstract datatype
(namely well-founded trees) theories automatically.

\section{PVS Language Design Principles}

There are several basic principles that have motivated the design of
PVS which are explicated in this section.

\paragraph{Specification vs. Programming Languages.}
A specification represents requirements or a design whereas a program
text represents an implementation of a design.  A program can be seen as
a specification, but a specification need not be a program.  Typically,
a specification expresses \emph{what} is being
computed whereas a program expresses \emph{how} it is computed.  A
specification can be incomplete and still be meaningful whereas an
incomplete program will typically not be executable.  A specification
need not be executable; it may use high-level constructs, quantifiers
and the like, that need have no computational meaning.  However, there
are a number of aspects of programming languages that a specification
language should include, such as:
\begin{itemize}

\item the usual basic types: booleans, integers, and rational numbers

\item the familiar datatypes of programming languages such as arrays,
records, lists, sequences, and abstract datatypes

\item the higher-order capabilities provided by modern functional
programming languages so that extremely general-purpose operations can
be defined

\item definition by recursion

\item support for dividing large specifications into parameterized
modules

\end{itemize}

It is clearly not enough to say that a specification language shares some
important features of a programming language but need not be executable.
Any useful formal language must have a clearly defined
semantics\footnote{The PVS semantics are presented in a technical
report~\cite{PVS:semantics}.} and must be capable of being manipulated in
ways that are meaningful relative to the semantics.  A programming
language for example can be given a denotational semantics so that the
execution of the program respects its denotational meaning.  The reason
one writes a specification in a formal language is typically to ensure
that it is sensible, to derive some useful consequences from it, and to
demonstrate that one specification implements another.  All of these
activities require the notion of a justification or a proof based on the
specification, a notion that can only be captured meaningfully within the
framework of logic.

\paragraph{Untyped set theory versus higher-order logic}
\index{set theory}\index{higher-order logic}
Which logic should be chosen?  There is a wide variety of choices:
simple propositional logics, which can be classical or intuitionistic,
equational logics, quantificational logics, modal and temporal logics,
set theory, higher-order logic, etc.  Some propositional and modal
logics are appropriate for dealing with finite state machines where one
is primarily interested in efficiently deciding certain finite state
machine properties.  For a general purpose specification language,
however, only a set theory or a higher-order logic would provide the
needed expressiveness.  Higher-order logic requires strict typing to
avoid inconsistencies whereas set theory restricts the rules for forming
sets.  Set theory is inherently untyped, and grafting a typechecker onto
a language based on set theory is likely to be too strict and arbitrary.
Typechecking, however, is an extremely important and easy way of
checking whether a specification makes semantic sense (although 
for an opposing view, the reader is referred to a report by Lamport
and Paulson~\cite{Lamport&Paulson97}).  Higher-order
logic does admit effective typechecking but at the expense of an
inflexible type system.  Recent advances in type theory have made it
possible to design more flexible type systems for higher-order logic
without losing the benefits of typechecking.  We have therefore chosen
to base PVS on higher-order logic.

\paragraph{Total versus partial functions}
\index{function!total}\index{function!partial} In the PVS higher-order
logic, an individual is either a function, a tuple, a record, or the
member of a base type.  Functions are extremely important in higher-order
logic.  They are \emph{first-class} individuals, i.e., variables can range
over functions.  In general, functions can represent either \emph{total}
or \emph{partial} maps.  A total map from domain $A$ to range $B$ maps
each element of $A$ to some element of $B$, whereas a partial map only
maps some of the elements of $A$ to elements of $B$.  Most traditional
logics build in the assumption that functions represent total maps.
Partial functions arise quite naturally in specifications.  For example,
the division operation is undefined on a zero denominator and the
operation of popping a stack is undefined on an empty stack.

Some recent logics, notably those of VDM~\cite{Jones:VDM},
LUTINS~\cite{Farmer:functions}, RAISE~\cite{RAISE-tutorial},
Beeson~\cite{Beeson:book} and Scott~\cite{Scott79}, admit partial
functions.  In these logics, some terms may be \emph{undefined} by not
denoting any individuals.  Some of these logics have mechanisms for
distinguishing defined and undefined terms, while others allow
``undefined'' to propagate from terms to expressions and therefore must
employ multiple truth values.  In all these cases, the ability to
formalize partially defined functions comes at the cost of complicating
the deductive apparatus, even when the specification does not involve any
partial functions.  Though logics that allow partial functions are
extremely interesting, we have chosen to avoid partial functions in PVS.
We have instead employed the notion of a \emph{predicate
subtype}\index{predicate subtype}, a type that consists of those elements
of a given type satisfying a given predicate.  Using predicate subtypes,
the type of the division operator, for example, can be constrained to
admit only nonzero denominators.  Division then becomes a total operation
on the domain consisting of arbitrary numerators and nonzero denominators.
The domain of a \emph{pop} operation on stacks can be similarly restricted
to nonempty stacks.  PVS thus admits partial functions within the
framework of a logic of total functions by enriching the type system to
include predicate subtypes.  We find this use of predicate subtypes to be
significantly in tune with conventional mathematical practice of being
explicit about the domain over which a function is defined.

\section{An Example: \texttt{stacks}}\label{stacks-example}
\index{stacks example@\texttt{stacks} example}

In this section we discuss a specific example, the theory of
\texttt{stacks}, in order to give a feel for the various aspects of the
PVS language before going into detail.  Apart from the basic notation for
defining a theory, this example illustrates the use of type parameters at
the theory level, the general format of declarations, the use of predicate
subtyping to define the type of nonempty stacks, and the generation of
typechecking obligations.

\pvstheory{stacks-alltt}{{Theory \texttt{stacks}}}{stacks-alltt}

Figure~\ref{stacks-alltt} illustrates a theory for stacks of an arbitrary
type with corresponding stack operations.  Note that this is not the
recommended approach to specifying stacks; a more convenient and complete
specification is provided in Section~\ref{stacks-adt},
page~\pageref{stacks-adt}.

The first line introduces a theory named \texttt{stacks} that is
parameterized by a type \texttt{t} (the \emph{formal parameter} of
\texttt{stacks}).  The keyword \texttt{TYPE+} indicates that \texttt{t} is
a \emph{non-empty} type.  The uninterpreted (nonempty) type \texttt{stack}
is declared, and the constant \texttt{empty} and variable \texttt{s} are
declared to be of type \texttt{stack}.  The defined predicate
\texttt{nonemptystack?}~is then declared on elements of type
\texttt{stack}; it is \texttt{true} for a given \texttt{stack} element
iff\footnote{Iff is short for ``if and only if''.} that element is not
equal to \texttt{empty}.  The functions \texttt{push}, \texttt{pop}, and
\texttt{top} are then declared.  Note that the predicate
\texttt{nonemptystack?}~is being used as a type in specifying the
signatures of these functions; any predicate may be used as a type simply
by putting parentheses around it.

The variables \texttt{x} and \texttt{y} are then declared, followed by the
usual axioms for \texttt{push}, \texttt{pop}, and \texttt{top}, which make
\texttt{push} a stack constructor and \texttt{pop} and \texttt{top} stack
accessors.  Finally, there is the theorem \texttt{pop2push2}, that can
easily be proved by two applications of the \texttt{pop\_push} axiom.

This simple theorem has an additional facet that shows up during
typechecking.  Note that \texttt{pop} expects an element of type
\texttt{(nonemptystack?)} and returns a value of type \texttt{stack}.
This works fine for the inner \texttt{pop} because it is applied to
\texttt{push}, which returns an element of type \texttt{(nonemptystack?)};
but the outer occurrence of \texttt{pop} cannot be seen to be type correct
by such syntactic means.  In cases like these, where a subtype is expected
but not directly provided, the system generates a \emph{type-correctness
condition} (TCC).  In this case, the TCC is
\begin{pvsex}
  pop2push2_TCC1: OBLIGATION
    (FORALL (s: stack, y: t, x: t):
      nonemptystack?(pop(push(x, push(y, s)))))
\end{pvsex}
and is easily proved using the \texttt{pop\_push} axiom.  The system keeps
track of all such obligations and will flag the unproved ones during proof
chain analysis.

Parameterized theories such as \texttt{stacks} introduce theory schemas,
where the type \texttt{t} may be instantiated with any other nonempty
type.  To use the types, constants, and formulas of the \texttt{stacks}
theory from another theory, the \texttt{stacks} theory must be imported,
with \emph{actual parameters} provided for the corresponding theory
parameters.  This is done by means of an \texttt{IMPORTING} clause. For
example, the theory
\begin{pvsex}
  ustacks : THEORY
    BEGIN
    IMPORTING stacks[int], stacks[stack[int]]

    si : stack[int]
    sos : stack[stack[int]] = push(si, empty)
    END ustacks
\end{pvsex}
imports stacks of integers and stacks of stacks of integers.  The constant
\texttt{si} is then declared to be a stack of integers, and the constant
\texttt{sos} is a stack of stacks of integers whose top element is
\texttt{si}.  Note that the system is able to determine which instance of
\texttt{push} and \texttt{empty} is meant from the type of the first
argument.  In general, the typechecker infers the type of an expression
from its context.  


% Document Type: LaTeX
% Master File: language.tex
\chapter{The Lexical Structure}\label{lexical}

PVS specifications are text files, each composed of a sequence of lexical
elements which in turn are made up of characters.  The lexical elements of
PVS are the identifiers, reserved words, special symbols, numbers,
whitespace characters, and comments.

Identifiers\index{identifiers} are composed of letters, digits, and the
characters \texttt{\_} or \texttt{?}; they must begin with a letter, which
are the usual ASCII letters, or any Unicode character that is not one of
the ASCII non-letter characters.  Note that keywords from
Figure~\ref{reserved-words} and operators from Figure~\ref{special-symbols} may
be embedded in identifiers, but may not be identifiers themselves.  Thus
\texttt{candy} is an identifier, though it contains the keyword
\texttt{and}, and ∧∧ is an identifier, though it contains the keyword ∧.  

They may be arbitrarily long.  Identifiers are case-sensitive;
\texttt{FOO}, \texttt{Foo}, and \texttt{foo} are different identifiers.
PVS strings contain any Unicode character: to include a \texttt{"} in the
string, use \texttt{\char'134 "} and to include a \texttt{\char'134} use
\texttt{\char'134\char'134}.  For more on Unicode, see the PVS User Guide.

\pvsbnf{bnf-lexical}{Lexical Syntax}

The reserved words\index{reserved words} are shown in
Figure~\ref{reserved-words}.  Unlike identifiers, they are not
case-sensitive.  In this document, reserved words are always displayed in
upper case.  Note that identifiers may have reserved words embedded in
them, thus \texttt{ARRAYALL} is a valid identifier and will not be
confused with the two embedded reserved words.  The meaning of the
reserved words are given in the appropriate sections; they are collected
here for reference.

\begin{figure}[tb]
{\smaller\tt
\begin{tabular}{|*{5}{p{1.03in}}|}\hline
\input{keywords}
\hline
\end{tabular}}
\caption{\pvs\ Reserved Words}\label{reserved-words}
\end{figure}

The special symbols\index{special symbols} are listed in
Figure~\ref{special-symbols}.  All of these symbols are separators; they
separate identifiers, numbers, and reserved words.

\begin{figure}[tb]
\begin{center}
  {\small\tt
    % \fontspec{Cambria Math}
    %\fontspec{TeX Gyre Pagella Math}
    % \fontspec{Latin Modern Math}
\begin{tabular}{|*{6}{@{\hspace*{.2in}}c@{\extracolsep{.5in}}}@{\hspace*{.25in}}|}\hline
\input{operator-table}
\hline
\end{tabular}}
\end{center}
\caption{\pvs\ Special Symbols}\label{special-symbols}
\end{figure}

The whitespace characters are space, tab, newline, return, and newpage;
they are used to separate other lexical elements.  At least one whitespace
character must separate adjacent identifiers, numbers, and reserved words.

Comments\index{comments} may appear anywhere that a whitespace character
is allowed.  They consist of the `\texttt{\%}'\index{\%@\texttt{\%}} character
followed by any sequence of characters and terminated by a newline.

The \emph{definable} symbols are shown in table~\ref{definable-symbols}.
These keywords and symbols may be given declarations.  Some of them have
declarations given in the prelude.\footnote{In particular,
\texttt{\char38}, \texttt{*}, \texttt{+}, \texttt{-}, \texttt{/},
\texttt{/=}, \texttt{<}, \texttt{<<}, \texttt{<=}, \texttt{<=>},
\texttt{=}, \texttt{=>}, \texttt{>}, \texttt{>=}, \texttt{AND},
\texttt{IFF}, \texttt{IMPLIES}, \texttt{NOT}, \texttt{O}, \texttt{OR},
\texttt{WHEN}, \texttt{XOR}, \texttt{\char94}, and \texttt{\char126} are
declared there.  Note that many of these are overloaded, for example,
\texttt{\char94} has three different definitions.}  Any of these may be
(re)declared any number of times, though this may lead to ambiguities.
Such ambiguities may be resolved by including the theory name, actual
parameters,  and possibly the type as a coercion.

Symbols that are binary infix (\hyperlink{Binop}{\emph{Binop}}), for
example \texttt{AND} and \texttt{+}, may be declared with any number of
arguments.  If they are declared with two arguments then they may
subsequently be used in prefix or infix form.  Otherwise they may only be
used in prefix form.  Similarly for unary operators, and the \texttt{IF}
operator, which may be used in \texttt{IF-THEN-ELSE-ENDIF} form if
declared with three arguments.

Note that when typing the operators \texttt{/\\} or \texttt{\\/} outside
of a specification, the backslash may need to be doubled (or in rare
cases, quadrupled).  This is because it is commonly used as an ``escape''
character, and the character following may be interpreted specially.

The symbol pairs \lit{[|} \lit{|]}, \lit{(|} \lit{|)}, 
\lit{$\{$|} \lit{|$\}$}, {\lit{$〈$} \lit{$〉$}, \lit{$⟦$} \lit{$⟧$},
  \lit{$«$} \lit{$»$},
  {⟪⟫},
  {⌈⌉},
  {⌊⌋},
  {⌜⌝}, and
  {⌞⌟} are
available as outfix operators.  They are
declared and may be used by concatenation,
for example, with the declaration \texttt{[||]:\ [bool, int -> int]} the
outfix term \texttt{[| TRUE, 0 |]} is equivalent to the prefix form
\texttt{[||](TRUE, 0)}.

\begin{figure}[tb]
\begin{center}
\renewcommand{\arraystretch}{1.2}
{\small\tt
    \fontspec{Latin Modern Math}
\begin{tabular}{|*{6}{@{\hspace*{.2in}}c@{\extracolsep{.4in minus .4in}}}@{\hspace*{.2in}}|}\hline
\input{opsym-table}
\hline
\end{tabular}}
\end{center}
\caption{\pvs\ Definable Symbols}\label{definable-symbols}
\end{figure}

%%% Local Variables: 
%%% mode: latex
%%% TeX-master: "language"
%%% End: 

% Document Type: LaTeX
% Master File: language.tex

\chapter{Declarations}\label{declarations}
\index{declaration|(pidx}

Entities of PVS are introduced by means of \emph{declarations}, which are
the main constituents of PVS specifications.  Declarations are used to
introduce types, variables, constants, formulas, judgements, conversions,
and other entities.  Most declarations have an \emph{identifier} and
belong to a unique theory.  Declarations also have a body which indicates
the \emph{kind} of the declaration and may provide a signature or
definition for the entity.  \emph{Top-level}
declarations\index{declaration!top-level} occur in the formal parameters,
the assertion section and the body of a theory.  \emph{Local}
declarations\index{declaration!local} for variables may be given, in
association with constant and recursive declarations and \emph{binding
expressions} (\eg\ involving \texttt{FORALL} or \texttt{LAMBDA}).
Declarations are ordered within a theory; earlier declarations may not
reference later ones.\footnote{Thus mutual recursion is not directly
supported.  The effect can be achieved with a single recursive function
that has an argument that serves as a switch for selecting between two or
more subexpressions.}

\index{exporting|(}\index{importing|(}
Declarations introduced in one theory may be referenced in another by
means of the \texttt{IMPORTING} and \texttt{EXPORTING} clauses.  The
\texttt{EXPORTING} clause of a theory indicates those entities that may be
referenced from outside the theory.  There is only one such clause for a
given theory.  The \texttt{IMPORTING} clauses provide access to the
entities exported by another theory.  There can be many \texttt{IMPORTING}
clauses in a theory; in general they may appear anywhere a top-level
declaration is allowed.  See Section~\ref{importings} for more details.
\index{importing|)}\index{exporting|)}

PVS allows the overloading\index{overloading} of declaration identifiers.
Thus a theory named \texttt{foo} may declare a constant \texttt{foo} and a
formula \texttt{foo}.  To support this \emph{ad hoc} overloading,
declarations are classified according to kind\index{declaration!kind}; in
PVS the primary kinds are \emph{type}\index{declaration!kind!type},
\emph{prop}\index{declaration!kind!prop},
\emph{expr}\index{declaration!kind!expr}, and
\emph{theory}\index{declaration!kind!theory}.  Type declarations are of
kind \emph{type}, and may be referenced in type declarations, actual
parameters, signatures, and expressions.  Formula declarations are of kind
\emph{prop}, and may be referenced in auto-rewrite declarations
(Section~\ref{auto-rewrite-decls}) or proofs (see the PVS Prover
Guide~\cite{PVS:prover}).  Variable, constant, and recursive definition
declarations are of kind \emph{expr}; these may be referenced in
expressions and actual parameters.  Newly introduced names need only be
unique within a kind, as there is no way, for example, to use an
expression where a type is expected.\footnote{There are a few exceptions,
for example the actual parameters of theories, since theories may be
instantiated with types or expressions.}

\pvsbnf{bnf-decls}{Declarations Syntax}
\index{syntax!declarations}
\pvsbnf{bnf-decls-aux}{Declarations Syntax (cont.)}
\index{syntax!declarations}

Declarations generally consist of an
\emph{identifier}\index{declaration!identifier}, an optional list of
\emph{bindings}\index{declaration!binding}, and a
\emph{body}\index{declaration!body}.  The body determines the kind of the
declaration, and the bindings and the body together determine the
signature and definition of the declared entity.  Multiple
declarations\index{declaration!multiple} may be given in compressed form
in which a common body is specified for multiple identifiers; for example
%
\begin{pvsex}
  x, y, z: VAR int
\end{pvsex}
In every case this is treated the same as the expanded form, thus the
above is equivalent to:
\begin{pvsex}
  x: VAR int
  y: VAR int
  z: VAR int
\end{pvsex}

In the rest of this chapter we describe declarations for types, variables,
constants, recursive definitions, macros, inductive and coinductive
definitions, formulas, judgements, conversions, libraries, and
auto-rewrites.  The declarations for theory parameters, importings,
exportings, and theory abbreviations are given in Chapter~\ref{theories}.
Figure~\ref{bnf-decls} gives the syntax for declarations.

\section{Type Declarations}\label{type-declarations}
\index{type declarations|(}

Type declarations are used to introduce new type names to the context.
There are four kinds of type declaration:

\begin{itemize}

\item \emph{uninterpreted type declaration}: \texttt{T:\ TYPE}
\index{uninterpreted type}\index{type!uninterpreted}

\item \emph{uninterpreted subtype declaration}: \texttt{S:\ TYPE FROM T}
\index{uninterpreted subtype}\index{type!uninterpreted subtype}

\item \emph{interpreted type declaration}: \texttt{T:\ TYPE =
int}\index{interpreted type}\index{type!interpreted}

\item \emph{enumeration type declarations}: \texttt{T:\ TYPE = \setb r,
g, b\sete} \index{enumeration types}\index{type!enumeration}

\end{itemize}

These type declarations introduce \emph{type names}\index{type!name}
that may be referenced in type expressions (see Section~\ref{types}).
They are introduced using one of the keywords
\keyword{TYPE}\index{type@\texttt{TYPE}},
\keyword{NONEMPTY\_TYPE}\index{type@\texttt{NONEMPTY\_TYPE}}, or
\keyword{TYPE+}\index{type+@\texttt{TYPE+}}.

\subsection{Uninterpreted Type Declarations}
\index{type!uninterpreted|(}

Uninterpreted types support abstraction by providing a means of
introducing a type with a minimum of assumptions on the type.  An
uninterpreted type imposes almost no constraints on an implementation of
the specification.  The only assumption made on an uninterpreted type
\texttt{T} is that it is disjoint from all other types, except for
subtypes of \texttt{T}.  For example,
\begin{pvsex}
  T1, T2, T3: TYPE
\end{pvsex}
%
introduces three new pairwise disjoint types.  If desired, further
constraints may be put on these types by means of axioms or assumptions
(see Section~\ref{formula-declarations} on
page~\pageref{formula-declarations}).

It should be emphasized that uninterpreted types are important in
providing the right level of abstraction in a specification.  Specifying
the type body may have the undesired effect of restricting the possible
implementations, and cluttering the specification with needless detail.

\index{type!uninterpreted|)}\index{uninterpreted type|)}


\subsection{Uninterpreted Subtype Declarations}
\index{uninterpreted subtype|(}

Uninterpreted subtype declarations are of the form
\begin{pvsex}
  s: TYPE FROM t
\end{pvsex}
\index{FROM@\texttt{FROM}}
This introduces an uninterpreted
\emph{subtype}\index{subtypes}\index{type!subtype} \texttt{s} of
the \emph{supertype}\index{supertype}\index{type!supertype}
\texttt{t}.  This has the same meaning as
\begin{pvsex}
  s_pred: [t -> bool]
  s: TYPE = (s_pred)
\end{pvsex}
%
in which a new predicate is introduced in the first line and the type
\texttt{s} is declared as a \emph{predicate} subtype in the second
line\footnote{This is described in Section~\ref{subtypes}
(page~\pageref{subtypes}).}.  No assumptions are made about uninterpreted
subtypes; in particular, they may or may not be empty, and two different
uninterpreted subtypes of the same supertype may or may not be disjoint.
Of course, if the supertypes themselves are disjoint, then the
uninterpreted subtypes are as well.

\index{uninterpreted subtype|)}

\subsection{Structural Subtypes}

PVS has support for structural subtyping for record and tuple types.  A
record type \texttt{S} is a structural subtype of record type \texttt{R}
if every field of \texttt{R} occurs in \texttt{S}, and similarly, a tuple
type \texttt{T} is a structural subtype of a tuple type forming a prefix
of \texttt{T}.  Section \ref{type-extensions} gives
examples, as \texttt{colored\_point} is a structural subtype of
\texttt{point}, and \texttt{R5} is a structural subtype of \texttt{R3}.
Structural subtypes are akin to the class hierarchy of object-oriented
systems, where the fields of a record can be viewed as the slots of a
class instance.  The PVS equivalent of setting a slot value is the
override expression (sometimes called update), and this works with
structural subtypes, allowing the equivalent of generic methods to be
defined.  Here is an example:
\begin{pvsex}
points: THEORY
BEGIN
 point: TYPE+ = [# x, y: real #]
END points

genpoints[(IMPORTING points) gpoint: TYPE <: point]: THEORY
BEGIN
 move(p: gpoint)(dx, dy: real): gpoint =
  p WITH [`x := p`x + dx, `y := p`y + dy]
END genpoints

colored_points: THEORY
BEGIN
 IMPORTING points
 Color: TYPE = {red, green, blue}
 colored_point: TYPE = point WITH [# color: Color #]
 IMPORTING genpoints[colored_point]
 p: colored_point
 move0: LEMMA move(p)(0, 0) = p
END colored_points
\end{pvsex}

The declaration for \texttt{gpoint} uses the structural subtype operator
\texttt{<:}.  This is analogous to the \texttt{FROM} keyword, which
introduces a (predicate) subtype.  This example also serves to explain
why we chose to separate structural and predicate subtyping.  If they
were treated uniformly, then \texttt{gpoint} could be instantiated with
the unit disk; but in that case the \texttt{move} operator would not
necessarily return a \texttt{gpoint}.  The TCC could not be generated
for the \texttt{move} declaration, but would have to be generated when
the \texttt{move} was referenced.  This both complicates typechecking,
and makes TCCs and error messages more inscrutable.  If both are
desired, simply include a structural subtype followed by a predicate
subtype, for example:
\begin{pvsex}
genpoints[(IMPORTING points) gpoint: TYPE <: point,
          spoint: TYPE FROM gpoint]: THEORY
\end{pvsex}
Now \texttt{move} may be applied to \texttt{gpoint}s, but if applied to a
\texttt{spoint} an unprovable TCC will result.

Structural subtypes are a work in progress.  In particular, structural
subtyping could be extended to function and datatypes.  And to have
real object-oriented PVS, we must be able to support a form of method
invocation.


\subsection{Empty and Singleton Record and Tuple Types}

Empty and singleton record and tuple types are now allowed in PVS.
Thus the following are valid declarations:
\begin{pvsex}
Tup0: TYPE = [ ]
Tup1: TYPE = [int]
Rec0: TYPE = [# #]
\end{pvsex}
Note that the space is important in the empty tuple type, as otherwise
it is taken to be an operator (the box operator).


\subsection{Interpreted Type Declarations}
\index{interpreted type declarations|(}\index{type!interpreted|(}

Interpreted type declarations are primarily a means for providing names
for type expressions.  For example,
\begin{pvsex}
  intfun: TYPE = [int -> int]
\end{pvsex}
%
introduces the type name \texttt{intfun} as an abbreviation for the type
of functions with integer domain and range.  Because PVS uses
\emph{structural equivalence}\index{structural equivalence} instead of
\emph{name equivalence}\index{name equivalence}, any type expression
\texttt{T} involving \texttt{intfun} is equivalent to the type expression
obtained by substituting \texttt{[int -> int]} for \texttt{intfun} in
\texttt{T}.  The available type expressions are described in
Chapter~\ref{types} on page~\pageref{types}.

Interpreted type declarations may be given
parameters.\index{parameterized type names} For example, the type of
integer subranges may be given as
\begin{pvsex}
  subrange(m, n: int): TYPE = \setb{}i:int | m <= i AND i <= n\sete
\end{pvsex}
and \texttt{subrange} with two integer parameters may subsequently be used
wherever a type is expected.  Any use of a parameterized type must include
all of the parameters, so currying of the parameters is not allowed.  Note
that \texttt{subrange} may be overloaded to declare a different type in
the same theory without any ambiguity, as long as the number or type of
parameters is different.

\index{type!interpreted|)}\index{interpreted type declarations|)}


\subsection{Enumeration Type Declarations}\label{enum-types}
\index{enumeration types|(}\index{type!enumeration|(}

Enumeration type declarations are of the form
\begin{pvsex}
  enum: TYPE = \setb{}e_1,\ldots, e_n\sete
\end{pvsex}
%
where the \texttt{e\_i} are distinct identifiers which are taken to
completely enumerate the type.  This is actually a shorthand for the
datatype specification
\begin{pvsex}
  enum: DATATYPE
    e_1: e_1?
         \(\vdots\)
    e_n: e_n?
  END enum
\end{pvsex}
%
explained in Chapter~\ref{adts}.  Because of this, enumeration types may
only be given as top-level declarations, and are \emph{not} type
expressions.  The advantage of treating them as datatypes is that the
necessary axioms are automatically generated, and the prover has built-in
facilities for handling datatypes.

\index{type!enumeration|)}\index{enumeration types|)}

\index{type declarations|)}


\subsection{Empty versus Nonempty Types}
\label{emptytypes}
\index{nonempty type}
\index{empty type}
\index{type!nonempty|(}\index{type!empty|(}

As noted before, PVS allows empty types, and the term \emph{type} refers
to either empty or nonempty types.  Constants declared to be of a given
type provide elements of the type, so the type must be nonempty or there
is an inconsistency.  Thus whenever a constant is declared, the system
checks whether the type is nonempty, and if it cannot decide that it is
nonempty it generates an \emph{existence TCC}.\index{existence
TCC}\index{TCC!existence} This is the simple explanation, but it is made
somewhat complicated by the considerations of formal parameters,
uninterpreted versus interpreted type declarations, explicit declarations
of nonemptiness, and
\keyword{CONTAINING}\index{CONTAINING@\texttt{CONTAINING}} clauses on type
declarationss, as well as a desire to keep the number of TCCs generated to
a minimum, while guaranteeing soundness.  The details are provided below.

First note that having variables range over an empty type causes no
difficulties,\footnote{If the type \texttt{T} is empty, then the following
two equivalences hold:
\begin{alltt}
  (FORALL (x: T): p(x)) IFF TRUE \quad \mbox{\textrm{and}} \quad (EXISTS (x: T): p(x)) IFF FALSE
\end{alltt}
}
so variable declarations and variable bindings never trigger the
nonemptiness check.

During typechecking, type declarations may indicate that the type is
nonempty, and constant declarations of a given type require that the type
be nonempty.  When a type is determined to be nonempty, it is marked as
such so that future checks of constants do not trigger more TCCs.  Below
we describe how type declarations are handled first for declarations in the
body of a theory, and then for type declarations that appear in the formal
parameters, as they require special handling.

\paragraph{Theory Body Type Declarations}

\begin{itemize}

\item Uninterpreted type or subtype declarations introduced with the
keyword \keyword{TYPE} may be empty.  Declaring a constant of that type
will lead to a TCC that is unprovable without further axioms.

\item Uninterpreted type declarations introduced with the keyword
\keyword{NONEMPTY\_TYPE}\index{nonempty_type@\keyword{NONEMPTY\_TYPE}}
or \keyword{TYPE+}\index{type+@\texttt{TYPE+}} are assumed to be nonempty.
Thus the type is marked nonempty.

\item Uninterpreted subtype declarations introduced with the keyword
\keyword{NONEMPTY\_TYPE} or \keyword{TYPE+} are assumed to be nonempty, as long as the
supertype is nonempty.  Thus the supertype is checked, and an existence
TCC is generated if the supertype is not known to be nonempty.  Then the
subtype is marked nonempty.

\item The type of an interpreted constant is nonempty, as the definition
provides a witness.

\item Interpreted type declarations introduced with the keyword
\keyword{TYPE} may be non\-emp\-ty, depending on the type definition.

\item Any interpreted type declaration with a \keyword{CONTAINING} clause
is marked nonempty, and the \keyword{CONTAINING} expression is typechecked
against the specified type.  In this case no existence TCC is generated,
since the \keyword{CONTAINING} expression is a witness to the type.  Of
course, other TCCs may be generated as a result of typechecking the
\keyword{CONTAINING} expression.

\end{itemize}

\paragraph{Formal Type Declarations}

Only uninterpreted (sub)type declarations may appear in the formal
parameters list.

\begin{itemize}

\item Formal type declarations introduced with the \texttt{TYPE} keyword may
be empty.  This is handled according to the occurrences of constant
declarations involving the type.

\item If there is a constant declaration of that type in the formal
parameter list, then no TCCs are generated, since
any instance of the theory will need to provide both the type and a
witness.  The type is marked nonempty in this case.

\item If the type declaration is a formal parameter and a constant is
declared whose type involves the type, but is not the type itself (for
example, if the formal theory parameters are \texttt{[t:\ TYPE, f:\ [t ->
t]]}), then a TCC may be generated, and a comment is added to the TCC
indicating that an assuming clause may be needed in order to discharge the
TCC.  This TCC will be generated only if an earlier constant declaration
hasn't already forced the type to be marked nonempty.  Note that there are
circumstances in which the formal type may be empty but the type
expression involving that type is nonempty.  This is discussed further
below.

\end{itemize}

\subsection{Checking Nonemptiness}\label{nonemptiness-check}
\index{type!nonempty}
The typechecker knows a type to be nonempty under the
following circumstances:
\begin{itemize}

\item The type was declared to be nonempty, using either the
\keyword{NONEMPTY\_TYPE}\index{nonempty_type@\keyword{NONEMPTY\_TYPE}} or
the synonymous \keyword{TYPE+}\index{type+@\texttt{TYPE+}} keyword.  If the
type is uninterpreted, this amounts to an assumption that the type is
nonempty.  If the type has a definition, then an existence TCC is
generated unless the defining type expression is known to be nonempty.

\item The type was declared to have an element using a
\keyword{CONTAINING}\index{CONTAINING@\texttt{CONTAINING}} expression.

\item A constant was declared for the type.  In this case an existence TCC
is generated for the first such constant, after which the type is marked
as nonempty.

\item It was marked as nonempty from an earlier check.

\end{itemize}

Once an unmarked type is determined to be nonempty, it is marked by the
typechecker so that later checks will not generate existence TCCs.  In
addition, the type components are marked as nonempty.  Thus the types that
make up a tuple type, the field types of a record type, and the supertype
of a subtype are all marked.

It is possible for two equivalent types to be marked differently, for
example:
\begin{pvsex}
  t1: TYPE = \setb{}x: int | x > 2\sete
  t2: TYPE = \setb{}x: int | x > 2\sete
  c1: t1
\end{pvsex}
only marks the first type (\texttt{t1}).  Hence, it is best to name your types and
to use those names uniformly.

\index{type!empty|)}
\index{type!nonempty|)}

\section{Variable Declarations}
\index{variables|(}\index{declaration!variables|(}

Variable declarations introduce new variables and associate a type with
them.  These are \emph{logical} variables, not program variables; they
have nothing to do with state---they simply provide a name and associated
type so that binding expressions and formulas can be succinct.
Variables may not be exported.  Variable
declarations also appear in binding expressions such as \texttt{FORALL} and
\texttt{LAMBDA}.  Such local declarations ``shadow'' any earlier
declarations.  For example, in
\begin{pvsex}
  x: VAR bool
  f: FORMULA (FORALL (x: int): (EXISTS (x: nat): p(x)) AND q(x))
\end{pvsex}
%
The occurrence of \texttt{x} as an argument to \texttt{p} is of type
\texttt{nat}, shadowing the one of type \texttt{int}.  Similarly, the
occurrence of \texttt{x} as an argument to \texttt{q} is of type
\texttt{int}, shadowing the one of type \texttt{bool}.

\index{variables|)}\index{declaration!variables|)}

\section{Constant Declarations}\label{constants}
\index{constants|(}\index{declaration!constants|(}

Constant declarations introduce new constants, specifying their type and
optionally providing a value.  Since PVS is a higher order logic, the term
\emph{constant} refers to functions and relations, as well as the usual
(0-ary) constants.  As with types, there are both \emph{uninterpreted} and
\emph{interpreted} \index{constants!interpreted}%
\index{constants!uninterpreted} constants.  Uninterpreted constants make
no assumptions, although they require that the type be nonempty (see
Section~\ref{nonemptiness-check}, page~\pageref{nonemptiness-check}).
Here are some examples of constant declarations:
\begin{pvsex}
  n: int
  c: int = 3
  f: [int -> int] = (lambda (x: int): x + 1)
  g(x: int): int = x + 1
\end{pvsex}
%
The declaration for \texttt{n} simply introduces a new integer constant.
Nothing is known about this constant other than its type, unless further
properties are provided by \texttt{AXIOM}s.  The other three constants are
interpreted.  Each is equivalent to specifying two declarations: \eg\
the third line is equivalent to
\begin{pvsex}
  f: [int -> int]
  f: AXIOM  f = (LAMBDA (x: int): x + 1)
\end{pvsex}
%
except that the definition is guaranteed to form a \emph{conservative
extension}\index{conservative extension} of the theory.  Thus the
theory remains consistent after the declaration is given if it was
consistent before.

The declarations for \texttt{f} and \texttt{g} above are two different ways to
declare the same function.  This extends to more complex arguments, for
example
\begin{pvsex}
  f: [int -> [int, nat -> [int -> int]]] =
     (LAMBDA (x: int): (LAMBDA (y: int), (z: nat): (LAMBDA (w: int):
       x * (y + w) - z)))
\end{pvsex}
%
is equivalent to
\begin{pvsex}
  f(x: int)(y: int, z: nat)(w: int): int = x * (y + w) - z
\end{pvsex}
%
This can be shortened even further if the variables are declared first:
\begin{pvsex}
  x, y, w: VAR int
  z: VAR nat
  f(x)(y,z)(w): int = x * (y + w) - z
\end{pvsex}
%
Finally, a mix of predeclared and locally declared variables is possible:
\begin{pvsex}
  x, y: VAR int
  f(x)(y,(z: nat))(w: int): int = x * (y + w) - z
\end{pvsex}
%
Note the parentheses around \texttt{z:\ nat}; without these, \texttt{y} would
also be treated as if it were declared to be of type \texttt{nat}.

A construct that is frequently encountered when subtypes are involved is
shown by this example
\begin{pvsex}
  f(x: \setb{}x: int | p(x)\sete): int = x + 1
\end{pvsex}
%
There are two useful abbreviations for this expression.  In the first, we
use the fact that the type \texttt{\setb{}x:\ int | p(x)\sete} is equivalent to
the type expression \texttt{(p)} when \texttt{p} has type \texttt{[int ->
bool]}, and we can write
\begin{pvsex}
  f(x: (p)): int = x + 1
\end{pvsex}
%
The second form of abbreviation basically removes the set braces and the
redundant references to the variable, though extra parentheses are
required:
\begin{pvsex}
  f((x: int | p(x))): int = x + 1
\end{pvsex}
%
Which of these forms to use is mostly a matter of taste; in general,
choose the form that is clearest to read for a given declaration.

Note that functions with range type \texttt{bool} are generally referred
to as \emph{predicates}, and can also be regarded as relations or sets.
For example, the set of positive odd numbers can be characterized by a
predicate as follows:
\begin{pvsex}
  odd: [nat -> bool] = (LAMBDA (n: nat): EXISTS (m: nat): n = 2 * m + 1)
\end{pvsex}
%
PVS allows an alternate syntax for predicates that encourages a
set-theoretic interpretation:
\begin{pvsex}
  odd: [nat -> bool] = \setb{}n: nat | EXISTS (m: nat): n = 2 * m + 1\sete
\end{pvsex}

\index{constants|)}

\section{Recursive Definitions}\label{recursive-definitions}
\index{recursive definitions|(}

Recursive definitions are treated as constant declarations, except that
the defining expression is required, and a \emph{measure}\index{measure
function} must be provided, along with an optional well-founded order
relation.\index{well-founded order releation} The same syntax for
arguments is available as for constant declarations; see the preceding
section.

PVS allows a restricted form of recursive definition; mutual
recursion\index{recursion!mutual}\index{mutual recursion} is not allowed,
and the function must be \emph{total},\index{total function} so that the
function is defined for every value of its domain.  In order to ensure
this, recursive functions must be specified with a
\emph{measure}\index{measure}, which is a function whose signature matches
that of the recursive function, but with range type the domain of the
order relation, which defaults to \texttt{<} on \texttt{nat} or
\texttt{ordinal}\index{ordinal}\index{type!ordinal}.  If the order
relation is provided, then it must be a binary relation on the range type
of the measure, and it must be well-founded; a \emph{well-founded} \tcc\
\index{well-founded TCC}\index{TCC!well-founded} is generated if the order
is not declared to be well-founded.

Here is the classic example of the
\texttt{factorial}\index{factorial@\texttt{factorial}} function:
%
\begin{pvsex}
  factorial(x: nat): RECURSIVE nat =
    IF x = 0 THEN 1 ELSE x * factorial(x - 1) ENDIF
    MEASURE (LAMBDA (x: nat): x)
\end{pvsex}
%
The measure is the expression following the \texttt{MEASURE} keyword (the
optional order relation follows a \texttt{BY} keyword after the
measure).  This definition generates a \emph{termination
TCC};\index{TCC!termination}\index{termination TCC} a proof obligation
which must be discharged in order that the function be well-defined.  In
this case the obligation is
%
\begin{pvsex}
  factorial_TCC2: OBLIGATION
    FORALL (x: nat): NOT x = 0 IMPLIES x - 1 < x
\end{pvsex}

It is possible to abbreviate the given \texttt{MEASURE} function by
leaving out the \texttt{LAMBDA} binding.  For example, the measure
function of the factorial definition may be abbreviated to:
\begin{pvsex}
  MEASURE x
\end{pvsex}
The typechecker will automatically insert a lambda binding corresponding
to the arguments to the recursive function if the measure is not already
of the correct type, and will generate a typecheck error if this process
cannot determine an appropriate function from what has been specified.

A termination \tcc\ is generated for each recursive occurrence of the
defined entity within the body of the definition.\footnote{Some of these
may be subsumed by earlier TCCs, and hence will not be displayed with the
\texttt{M-x show-tccs} command.}  It is obtained in one of two ways.  If a
given recursive reference has at least as many arguments provided as
needed by the measure, then the \tcc\ is generated by applying the measure
to the arguments of the recursive call and comparing that to the measure
applied to the original arguments using the order relation.  The
\texttt{factorial} \tcc\ is of this form.  The context of the occurrence
is included in the \tcc; in this case the occurrence is within the
\texttt{ELSE} part of an \texttt{IF-THEN-ELSE} so the negated condition is
an antecedent to the proof obligation.

If the reference does not have enough arguments available, then the
reference is actually given a \emph{recursive signature}\index{recursive
signature} derived from the recursive function as described below.  This
type constrains the domain to satisfy the measure, and the termination
\tcc\ is generated as a \emph{termination-subtype}
\tcc.\index{termination-subtype TCC}\index{TCC!termination-subtype}
Termination-subtype \tccs\ are recognized as such by the occurrence of the
order in the goal of the \tcc.  For example, we could define a
substitution function for terms as follows.
\begin{session}
  term: DATATYPE
  BEGIN
   mk_var(index: nat): var?
   mk_const(index: nat): const?
   mk_apply(fun: term, args: list[term]): apply?
  END term

  subst(x: (var?), y: term)(s: term): RECURSIVE term =
    (CASES s OF
      mk_var(i): (IF index(x) = i THEN y ELSE s ENDIF),
      mk_const(i): s,
      mk_apply(t, ss): mk_apply(subst(x, y)(t), map(subst(x, y))(ss))
     ENDCASES)
  MEASURE s BY <<
\end{session}
Now the first recursive occurrence of \texttt{subst} has all arguments
provided, so the termination TCC is as expected.  The second occurrence
does not have enough arguments.  The recursive signature of that
occurrence is
\begin{pvsex}
  [[(var?), term] -> [\setb{}z1: term | z1 << s\sete -> term]]
\end{pvsex}
Hence the signature of \texttt{subst(x, y)} is \texttt{[\setb{}z1:\ term | z1 <<
s\sete -> term]}, and map is instantiated to \texttt{map[\setb{}z1:\ term | z1 <<
s\sete, term]}, which leads to the TCC
\begin{pvsex}
 subst_TCC2: OBLIGATION
   FORALL (ss: list[term], t: term, s: term, x: (var?)):
     s = mk_apply(t, ss) IMPLIES every[term](LAMBDA (z: term): z << s)(ss);
\end{pvsex}
Note that this \texttt{map} instance could be given directly, just don't
make the mistake of providing \texttt{map[term, term]}, as this leads to a
TCC that says every \texttt{term} is \texttt{<<} \texttt{s}.
For the same reason, if the uncurried form of this definition is given,
then a lambda expression will have to be provided and the type will have
to include the measure, for example,
\begin{session}
   subst(x: (var?), y, s: term): RECURSIVE term =
     (CASES s OF
       mk_var(i): (IF index(x) = i THEN y ELSE s ENDIF),
       mk_const(i): s,
       mk_apply(t, ss): mk_apply(subst(x, y, t),
                                 map(LAMBDA (s1: \setb{}z: term|z<<s\sete):
                                       subst(x, y, s1))(ss))
      ENDCASES)
   MEASURE s BY <<
\end{session}
\renewcommand{\textfraction}{.1}

The recursive signature is generated based on the type of the recursive
function and the measure.  For curried functions, it may be that the
measure does not have the entire domain of the recursive function, but
only the first few.  For example, consider the measure for the function
\texttt{f}.
\begin{pvsex}
  f(r: real)(x, y: nat)(b: boolean): RECURSIVE boolean
    = ...
   MEASURE LAMBDA (r: real): LAMBDA (x, y: nat): x
\end{pvsex}
The type of the measure function is \texttt{[real -> [nat, nat -> nat]]},
which is a prefix of the function type.  In deriving the recursive
signature, the last domain type of the measure is constrained (using a
subtype) in the corresponding position of the recursive function type.  In
this case the recursive signature is
\begin{pvsex}
  [real -> [\setb{}z: [nat, nat] | z`1 < x\sete -> [boolean -> boolean]]]
\end{pvsex}
Note that the recursive signature is a dependent type that depends on the
arguments of the recursive function (\texttt{x} in this case), and hence
only applies within the body of the recursive definition.

The formal argument that typechecking the body of a recursive function
using the recursive signature is sound will appear in a future version of
the semantics manual, for now note that simple attempts to subvert this
mechanism do not work, as the following example illustrates.
\begin{pvsex}
  fbad: RECURSIVE [nat -> nat] = fbad
   MEASURE lambda (n: nat): n
\end{pvsex}
This leads an unprovable TCC.
\begin{pvsex}
  fbad_TCC1: OBLIGATION FORALL (x1: nat, x: nat): x < x1;
\end{pvsex}
The TCC results from the comaprison of the expected type \texttt{[nat ->
nat]} to the derived type \texttt{[\setb{}z:\ nat | z < x1\sete -> nat]}.  Remember
that in PVS domains of function types must be equal in order for the
function types to satisfy the subtype relation, and this is exactly what
the TCC states.

\pvstheory{f91-alltt}{Theory \texttt{f91}}{f91-alltt}
\index{f91@{\texttt{f91}}}

When a doubly recursive call is found, the inner recursive calls are
replaced by variables in the termination \tccs\ generated for the outer
calls.  For example, the theory of Figure~\ref{f91-alltt} generates the
termination TCC of Figure~\ref{f91-tcc}

\begin{figure}[ht]
\begin{session}
f91_TCC5: OBLIGATION
  FORALL (i: nat,
          v: [i1:
               \setb{}z: nat |
                        (IF z > 101 THEN 0 ELSE 101 - z ENDIF) <
                         (IF i > 101 THEN 0 ELSE 101 - i ENDIF)\sete ->
               \setb{}j: nat | IF i1 > 100 THEN j = i1 - 10 ELSE j = 91 ENDIF\sete]):
    NOT i > 100 IMPLIES
     IF i > 100 THEN v(v(i + 11)) = i - 10 ELSE v(v(i + 11)) = 91 ENDIF;
\end{session}
\caption{Termination TCC for \texttt{f91}}\label{f91-tcc}
\end{figure}
where the inner calls to \texttt{f91} have been replaced by the
higher-order variable \texttt{v}, with the recursive signature as shown.
Since the obligation forces us to prove the termination condition for all
functions whose type is that of \texttt{f91}, it will also hold for
\texttt{f91}.  This example also illustrates the use of dependent types,
discussed in Section~\ref{dependent-types}.

\pvstheory{ackerman-alltt}{Theory \texttt{ackerman}}{ackerman-alltt}
\index{ackerman@{\texttt{ackerman}}}

\renewcommand{\textfraction}{.01}

In some cases the natural numbers are not a convenient measure; PVS
also provides the \texttt{ordinal}s, which allow recursion with measures up
to $\varepsilon_0$.  This is primarily useful in handling
lexicographical orderings.  For example, in the definition of the
Ackerman function in Figure~\ref{ackerman-alltt},\footnote{There are
ways of specifying \texttt{ackerman} using higher-order functionals, in
which case the measure is again on the natural numbers.} there are two
termination \tccs\ generated (along with a number of subtype \tccs).
The first termination \tcc\ is
\begin{pvsex}
  ack_TCC2:
    OBLIGATION
      (FORALL m, n:
        NOT m = 0 AND n = 0 IMPLIES ackmeas(m - 1, 1) < ackmeas(m, n))
\end{pvsex}
%
and corresponds to the first recursive call of \texttt{ack} in the body of
\texttt{ack}.  In this occurrence, it is known that \texttt{m $\neq$ 0}
and \texttt{n = 0}.  The remaining expression says that the measure
applied to the arguments of the recursive call to \texttt{ack} is less
than the measure applied to the initial arguments of \texttt{ack}.  Note
that the \texttt{<} in this expression is over the \texttt{ordinal}s, not
the \texttt{real}s.

\index{recursive definitions|)}


\section{Macros}\label{macro-declarations}
\index{macros|(}

There are some definitions that are convenient to use, but it's preferable
to have them expanded whenever they are referenced.  To some extent this
can be accomplished using auto-rewrites in the prover, but rewriting is
restricted.  In particular terms in types or actual parameters are not
rewritten; \texttt{typepred} and \texttt{same-name} must be used.  These
both require the terms to be given as arguments, making it difficult to
automate proofs.

The \texttt{MACRO} declaration is used to indicate definitions that are
expanded at typecheck time.  Macro declarations are normal constant
declarations, with the \texttt{MACRO} keyword preceding the
type.\footnote{This is similar to the \texttt{==} form of E\textsc{hdm}.}
For example, after the declaration
\begin{pvsex}
  N: MACRO nat = 100
\end{pvsex}
any reference to \texttt{N} is now automatically replaced by \texttt{100},
including such forms as \texttt{below[N]}.

Macros are not expanded until they have been typechecked.  This is because
the name overloading allowed by PVS precludes expanding during parsing.
TCCs are generated before the definition is expanded.
\index{macros|)}

\input{inductive_defs}

\section{Formula Declarations}\label{formula-declarations}
\index{formula declarations|(}\index{declaration!formulas|(}

Formula declarations introduce \emph{axioms}\index{axioms},
\emph{assumptions}\index{assumptions}, \emph{theorems}\index{theorems},
and \emph{obligations}\index{obligations}.  The identifier associated with
the declaration may be referenced in auto-rewrite declarations (see
Section~\ref{auto-rewrite-decls} and in proofs (see the \texttt{lemma} command
in the PVS Prover Guide~\cite{PVS:prover}).  The expression that makes up
the body of the formula is a boolean expression.  Axioms, assumptions, and
obligations are introduced with the keywords \texttt{AXIOM},
\texttt{ASSUMPTION}, and \texttt{OBLIGA\-TION}, respectively.  Axioms may
also be introduced using the keyword \texttt{POSTULATE}\index{postulate}.
In the prelude postulates are used to indicate axioms that are provable by
the decision procedures, but not from other axioms.  Theorems may be
introduced with any of the keywords
\texttt{CHALLENGE}\index{claim@{\texttt{CHALLENGE}}},
\texttt{CLAIM}\index{claim@{\texttt{CLAIM}}},
\texttt{CONJECTURE}\index{conjecture@{\texttt{CONJECTURE}}},
\texttt{COROLLARY}\index{corollary@{\texttt{COROLLARY}}},
\texttt{FACT}\index{fact@{\texttt{FACT}}},
\texttt{FORMULA}\index{formula@{\texttt{FORMULA}}},
\texttt{LAW}\index{law@{\texttt{LAW}}},
\texttt{LEMMA}\index{lemma@{\texttt{LEMMA}}},
\texttt{PROPOSITION}\index{proposition@{\texttt{PROPOSITION}}},
\texttt{SUBLEMMA}\index{sublemma@{\texttt{SUBLEMMA}}}, or
\texttt{THEOREM}\index{theorem@{\texttt{THEOREM}}}.

Assumptions are only allowed in assuming clauses (see
Section~\ref{assuming}).  Obligations are generated by the system for
\tccs, and cannot be specified by the user.  Axioms are treated
specially when a proof is analyzed, in that they are not expected to
have an associated proof.  Otherwise they are treated exactly like
theorems.  All the keywords associated with theorems have the same
semantics, they are there simply to allow for greater diversity in
classifying formulas.

Formula declarations may contain free variables\index{free variables}, in
which case they are equivalent to the universal closure\index{universal
closure} of the formula.\footnote{The universal closure of a formula is
obtained by surrounding the formula with a \texttt{FORALL} binding
operator whose bindings are the free variables of the formula.  For
example, the universal closure of \texttt{p(x,y) => q(z)} is
\texttt{(FORALL x,y,z:\ p(x,y) => q(z))} (assuming \texttt{x}, \texttt{y}
and \texttt{z} resolve to variables).} In fact, the prover actually uses
the universal closure when it introduces a formula to a proof.  Formula
declarations are the only declarations in which free variables are
allowed.

\index{declaration!formulas|)}\index{formula declarations|)}

\input{judgements}

\input{conversions}

\input{libraries}

\input{auto-rewrite}

\index{declaration|)}

%%% Local Variables: 
%%% mode: latex
%%% TeX-master: "language"
%%% End: 

% Document Type: LaTeX
% Master File: language.tex

\chapter{Types}\label{types}
\index{type|(pidx}

PVS specifications are strongly typed, meaning that every expression has
an associated type (although it need not be unique, more on this later).
The PVS type system is based on \emph{structural
equivalence}\index{structural equivalence} instead of \emph{name
equivalence}\index{name equivalence}, so types are closely related to
sets, in that two types are equal iff they have the same elements.
Section~\ref{type-declarations} describes the introduction of type names,
which are the simplest type expressions.  More complex type
expressions\index{type expressions} are built from these using \emph{type
constructors}\index{type constructors}.  There are type constructors for
\emph{subtypes}\index{subtypes}\index{type!subtype}, \emph{function
types}\index{function types}\index{type!function}, \emph{tuple
types}\index{tuple types}\index{type!tuple}, \emph{cotuple
types}\index{cotuple types}\index{type!cotuple}, and \emph{record
types}\index{record types}\index{type!record}.  Function, record, and
tuple types may also be \emph{dependent}\index{dependent
types}\index{type!dependent}.  A form of \emph{type
application}\index{type application}\index{type!application} is provided
that makes it more convenient to specify parameterized subtypes.  There
are also provisions for creating \emph{abstract datatypes}, described in
Chapter~\ref{datatypes}.

Type expressions occur throughout a specification; in particular, they may
appear in theory parameters, type declarations, variable declarations,
constant declarations, recursive and inductive definitions, conversions,
and judgements.  In addition, they may appear in certain expressions
(coercions and local bindings, see pages~\pageref{coercions}
and~\pageref{binding-expressions}, respectively), and as actual parameters
in names (page~\pageref{names}).  In the many examples which follow, type
expressions will be presented in the context of type declarations; but it
must be remembered that they can appear in any of the above places.

\pvsbnf{bnf-type-expr}{Type Expression Syntax}

\section{Subtypes}\label{subtypes}
\index{subtypes|(}\index{type!subtype|(}

Any collection of elements of a given type itself forms a type, called a
\emph{subtype}.  The type from which the elements are taken is called the
\emph{supertype}\index{supertype}.  The elements which form the subtype
are determined by a \emph{subtype predicate}\index{subtype predicate} on
the supertype.

Subtypes in PVS provide much of the expressive power of the language,
at the cost of making typechecking undecidable.  There are two forms of
subtypes.  The first is similar to the notation used to define a set:
\begin{pvsex}
  t: TYPE = \setb{}x: s | p(x)\sete
\end{pvsex}
%
where \texttt{p} is a predicate on the type \texttt{s}.\footnote{If \texttt{x}
has been previously declared as a variable of type \texttt{s}, then the
``\texttt{:~s}'' may be omitted.} This has the usual set-theoretical
meaning, since types in PVS are modeled as sets.  Subtypes may also
be presented in an abbreviated form, by giving a predicate surrounded by
parentheses:
\begin{pvsex}
  t: TYPE = (p)
\end{pvsex}
%
This is equivalent to the form above.

Note that if the predicate \texttt{p} is everywhere false, then the type
is empty.  PVS supports empty types\index{empty type}\index{type!empty},
and the term \emph{type} is used to refer to any type, including the empty
type.  This is discussed in Section~\ref{type-declarations} (page~\pageref{type-declarations}).

Subtypes tend to make specifications more succinct and easier to read.
For example, in a specification such as
\begin{pvsex}
  FORALL (i:int):
    (i >= 0 IMPLIES (EXISTS (j:int): j >= 0 AND j > i))
\end{pvsex}
it is much more difficult to see what is being stated than in the
equivalent
\begin{pvsex}
  FORALL (i:nat): (EXISTS (j:nat): j > i))
\end{pvsex}
%
where \texttt{nat} is defined in the prelude as
\begin{pvsex}
  naturalnumber: NONEMPTY\_TYPE = \setb{}i:integer | i >= 0\sete CONTAINING 0
  nat: NONEMPTY\_TYPE = naturalnumber
\end{pvsex}

Subtype constructors consist of a \emph{supertype}\index{supertype} and a
\emph{subtype predicate}\index{subtype predicate} on the supertype.  The
primary property of a subtype is that any element which belongs to the
subtype automatically belongs to the supertype.  In addition, functions
defined on a type automatically apply to the subtype.

\index{TCC|(}

There are two \emph{type-correctness conditions} (\tccs) associated with
subtypes.  The first concerns \emph{empty types}\index{empty
type}\index{type!empty} as described in section~\ref{emptytypes}.  The
second \tcc\ associated with subtypes is the \emph{subtype}
\tcc,\index{subtype TCC}\index{TCC!subtype}, which comes about from the
use of operations defined on subtypes that are applied to elements of the
supertype.  By this means partial functions may be handled directly,
without recourse to a partial term logic or some form of multi-valued
logic.  For instance, division in PVS is a total function, with signature
\texttt{[real, nonzero\_real -> real]}.  So given the formula
\begin{pvsex}
  div_form: FORMULA (FORALL (x, y: int):
                      x /= y IMPLIES (x - y)/(y - x) = -1)
\end{pvsex}
%
the denominator is of type integer, but the signature for \texttt{/}
demands a \texttt{nonzero\_real}.  The typechecker thus generates a
\emph{subtype} \tcc\ whose conclusion is \texttt{(y - x) /= 0}.  The
premises of the \tcc\ are obtained from the expressions
\emph{context}\index{context}---the conditions which lead to the
\texttt{/} operator---in this case \texttt{x /= y}.\footnote{As described
in the Formal Semantics~\cite{PVS:semantics}, the context containing
declarations is extended to allow boolean expressions.}  The \tcc\ is then
\begin{pvsex}
  div_form_TCC1: OBLIGATION
    (FORALL (x,y: int): x /= y IMPLIES (y - x) /= 0)
\end{pvsex}
which is easily discharged by the prover.  In general, the context of an
expression is obtained from expressions involving \texttt{IF-THEN-ELSE},
\texttt{AND}, \texttt{OR}, and \texttt{IMPLIES} by translating to the \texttt{IF-THEN-ELSE} form.  Specifically,
\begin{center}
\begin{tabular}{|lc|} \hline
Expression & Context for $e$ \\ \hline
\texttt{IF $a$ THEN $e$ ELSE $c$ ENDIF} & $a$ \\
\texttt{IF $a$ THEN $b$ ELSE $e$ ENDIF} & \texttt{NOT $a$} \\
\texttt{$a$ AND $e$} & $a$ \\
\texttt{$a$ OR $e$} & \texttt{NOT $a$} \\
\texttt{$a$ IMPLIES $e$} & $a$ \\ \hline
\end{tabular}
\end{center}
Note that only these operators are treated this way; if, for example,
\texttt{IMPLIES} is overloaded it will not include the left-hand side in
the context for typechecking the right-hand side.  The \tccs\ generated
from the context of expression involving a subtype are sufficient, but not
necessary conditions that ensure that the value of the expression does
not depend on the value of functions applied outside their domain.

Subtype \tccs\ may occur anywhere there is a mismatch between the type of
a term and the use of it, not just in function applications.  For example,
the following use of record types leads to an unprovable subtype \tcc.
\begin{pvsex}
  r: [# a, b: nzint #] = (# a := 0, b := 0 #)
\end{pvsex}

\index{TCC|)}

\index{type!subtype|)}\index{subtypes|)}

\section{Function Types}\label{function-types}
\index{function types|(}\index{type!function|(}

Function types have three equivalent forms:
\begin{itemize}
\item \texttt{[t\(_1\), \ldots, t\(_n\) -> t]}

\item \texttt{FUNCTION[t\(_1\), \ldots, t\(_n\) -> t]}

\item \texttt{ARRAY[t\(_1\), \ldots, t\(_n\) -> t]}
\end{itemize}
%
where each \texttt{t$_i$} is a type expression.  An element of this type is
simply a function whose domain is the sequence of types \texttt{t$_1$},
\ldots, \texttt{t$_n$}, and whose range is \texttt{t}.  A function type is empty
if the range is empty and the domain is not.  There is no difference in
meaning between these three forms; they are provided to support different
intensional uses of the type, and may suggest how to handle the given type
when an implementation is created for the specification.

The two forms \texttt{pred[t]}\index{pred@{\texttt{pred}}} and \texttt{setof[t]}\index{setof@{\texttt{setof}}} are both provided in the
prelude as shorthand for \texttt{[t ->
bool]}.  There is no difference in semantics, as sets in
PVS are represented as predicates.  The different keywords are
provided to support different intentions; \texttt{pred} focuses on
properties while \texttt{setof} tends to emphasize elements.

A function type \texttt{[t$_1$,\ldots,t$_n$ -> t]} is a subtype of
\texttt{[s$_1$,\ldots,s$_m$ -> s]} iff \texttt{s} is a subtype of
\texttt{t}, $n = m$, and \texttt{s$_i$} = \texttt{t$_i$} for $1 \leq i \leq n$.
This leads to subtype \tccs\ (called \emph{domain mismatch
\tccs})\index{domain mismatch TCC}\index{TCC!domain mismatch} that state
the equivalence of the domain types.  For example, given
\begin{pvsex}
  p, q: pred[int]
  f: [\setb{}x: int | p(x)\sete -> int]
  g: [\setb{}x: int | q(x)\sete -> int]
  h: [int -> int]
  eq1: FORMULA f = g
  eq2: FORMULA f = h
\end{pvsex}
%
The following \tccs\ are generated:
\begin{pvsex}
eq1_TCC1: OBLIGATION
  (FORALL (x1: \setb{}x : int | q(x)\sete, y1 : \setb{}x : int | p(x)\sete) :
     q(y1) AND p(x1))

eq2_TCC1: OBLIGATION
  (FORALL (x1: int, y1 : \setb{}x : int | p(x)\sete) :
     TRUE AND p(x1))
\end{pvsex}

Section~\ref{conversion-examples} on page~\pageref{conversion-examples}
explains how the \texttt{restrict} conversion may be automatically applied
in some cases to eliminate the production of these \tccs.

\index{type!function|)}\index{function types|)}


\section{Tuple Types}\label{tuple-types}
\index{tuple types|(}\index{type!tuple|(}

Tuple types (also called product types) have the form \texttt{[t$_1$,
\ldots, t$_n$]}, where the \texttt{t$_i$} are type expressions.  Note that
the 0-ary tuple type is not allowed.  Elements of this type are tuples
whose components are elements of the corresponding type.  For example,
\texttt{(1, TRUE, (LAMBDA (x:int):\ x + 1))} is an expression of type
\texttt{[int, bool, [int -> int]]}.  Order is important.  Associated with
every $n$-tuple type is a set of projection functions: \texttt{`1},
\texttt{`2}, \ldots, (or \texttt{proj\_1}, \texttt{proj\_2}, \ldots) where
the $i$th projection is of type \texttt{[[t$_1$, \ldots, t$_n$] ->
t$_i$]}.  A tuple type is empty if any of its component types is empty.
Function type domains and tuple types are closely related, as the types
\texttt{[t$_1$,\ldots, t$_n$ -> t]} and \texttt{[[t$_1$,\ldots, t$_n$] ->
t]} are equivalent; see Section~\ref{tuple-exprs} for more details.

\index{type!tuple|)}\index{tuple types|)}

\section{Record Types}\label{record-types}
\index{record types|(}\index{type!record|(}

Record types are of the form \texttt{[\# a$_1$:t$_1$, \ldots, a$_n$:t$_n$
\#]}.  The \texttt{a$_i$} are called \emph{record accessors}\index{record
accessors} or fields and the \texttt{t$_i$} are types.  Record types are
similar to tuple types, except that the order is unimportant and accessors
are used instead of projections.  Record types are empty if any of the
component types is empty.

Note that the fields of a record type must be applied, they are not
understood as functions.  See Section~\ref{record-expressions}.

\index{type!record|)}\index{record types|)}

\section{Dependent types}\label{dependent-types}
\index{dependent types|(}\index{type!dependent|(}

Function, tuple, and record types may be dependent; in other words, some
of the type components may depend on earlier components.  Here are some
examples:
\begin{pvsex}
  rem: [nat, d: \setb{}n: nat | n /= 0\sete -> \setb{}r: nat | r < d\sete]
  pfn: [d:pred[dom], [(d) -> ran]]
  stack: [\# size: nat, elements: [\setb{}n:nat | n < size\sete -> t] \#]
\end{pvsex}
The declaration for \texttt{rem} indicates explicitly the range of the
remainder function, which depends on the second argument.  Function types
may also have dependencies within the domain types; \eg\ the second domain
type may depend on the first.  Note that for function and tuple dependent
types, local identifiers need to be given only for those types on which
later types depend.

The tuple type \texttt{pfn} encodes partial functions as pairs consisting
of a predicate on the domain type and a function from the subtype
defined by that predicate to the range \texttt{ran}.  If the second
component were given instead as a function of type \texttt{[dom -> ran]},
then equality no longer works as intended.  For example, the absolute
value function \texttt{abs} and the identity function \texttt{id} are the same
on the domain \texttt{nat}, so we would like to have
\begin{pvsex}
  ((LAMBDA (x:int):x >= 0),abs) = ((LAMBDA (x:int):x >= 0),id)
\end{pvsex}
%
but without the dependency this would be equivalent to \texttt{abs = id}.

\texttt{stack} encodes a stack as a pair consisting of a size and an array
mapping initial segments of the natural numbers to \texttt{t}.  This is
similar to the \texttt{pfn} example---in fact, if we were willing to use a
tuple instead of a record encoding, \texttt{stack} could be declared as an
instance of the type of \texttt{pfn}.

Another example, presented in~\cite{Cheng&Jones90} as a ``challenge'' to
specification languages without partial functions, is easily handled
with dependent types as shown below.
\begin{pvsex}
  subp(i:int,(j:int | i >= j)): RECURSIVE int =
       (IF (i=j) THEN 0 ELSE (subp(i, j+1)+1) ENDIF)
    MEASURE i - j
\end{pvsex}
However, some formulas that are valid with partial functions are not even
well-formed in PVS:
\begin{pvsex}
  subp_lemma: LEMMA subp(i, 0) = i OR subp(0, i) = i
\end{pvsex}
This generates unprovable \tccs.  In practice this is rarely a problem.

\index{type!dependent|)}\index{dependent types|)}
\section{Cotuple Types}\label{cotuple-types}
\index{cotuple types|(}\index{type!cotuple|(}
\index{coproduct types|see{cotuple type}}
\index{sum types|see{cotuple type}}

\emph{Cotuple types} (also called \emph{coproduct} or \emph{sum} types)
provide a way to form the disjoint union of types.  The syntax is similar
to that for tuple types, but with `\texttt{+}' in place of `\texttt{,}',
so have the form \texttt{[t$_1$ + \ldots + t$_n$]}.  Elements of this type
are essentially pairs consisting of an index and a value for the type
corresponding to the index.  In PVS the syntax for this is
\texttt{IN\_$i(e)$}, where $e$ is an expression of type \texttt{t$_i$}.
For example, \texttt{IN\_2(3)} is an expression of type \texttt{[bool + int
+ [int -> int]]}, or any other cotuple type whose second component type
contains \texttt{3}.  A cotuple type is empty iff all its component types
are empty.

\index{type!cotuple|)}\index{cotuple types|)}

\index{type|)}

% Document Type: LaTeX
% Master File: language.tex

\chapter{Expressions}\label{expressions}
\index{expressions|(}

The PVS language offers the usual panoply of expression constructs,
including logical and arithmetic operators, quantifiers, lambda
abstractions, function application, tuples, a polymorphic
\texttt{IF-THEN-ELSE}, and function and record overrides.  Expressions may
appear in the body of a formula or constant declaration, as the predicate
of a subtype, or as an actual parameter of a theory instance.  The syntax
for PVS expressions is shown in Figures~\ref{bnf-expr} and~\ref{bnf-expr-aux}.

\pvsbnf{bnf-expr}{Expression syntax}

\pvsbnf{bnf-expr-aux}{Expression syntax (continued)}

\index{precedence|(} The language has a number of predefined operators
(although not all of these have a predefined meaning).  These are given in
Figure~\ref{precedenceops} below, along with their relative precedence
from lowest to highest.  Most of these operators are described in the
following sections.  \texttt{IN} is a part of \texttt{ LET} expressions,
\texttt{WITH} goes with override expressions, and the double colon
(\texttt{::}) is for coercion expressions.  The \texttt{o} operator is
defined in the prelude as the function composition operator.  Note that
most operators may be overloaded, see Chapter~\ref{lexical}
(page~\pageref{lexical}) for details.

\begin{figure}[htb]
\begin{center}{\small\tt
\begin{tabular}{|l|l|} \hline
{\rm Operators} & {\rm Associativity} \\ \hline
FORALL, EXISTS, LAMBDA, IN & None \\
\verb/|/ & Left \\
\verb/|-/, \verb/|=/ & Right \\
IFF, <=> & Right \\
IMPLIES, =>, WHEN & Right \\
OR, \verb|\/|, XOR, ORELSE & Right \\
AND, \&, \&\&, \verb|/\|, ANDTHEN & Right \\
NOT, \verb|~| & None \\
=, /=, ==, <, <=, >, >=, <<, >>, <<=, >>=, <|, |> & Left \\
WITH & Left \\
WHERE & Left \\
@, \# & Left \\
@@, \#\#, || & Left \\
+, -, ++, ~ & Left \\
*, /, **, // & Left \\
- & None \\
o & Left \\
:, ::, HAS\_TYPE & Left \\
\verb|[]|, <> & None \\
\verb|^|, \verb|^^| & Left \\
` & Left \\ \hline
\end{tabular}}
\end{center}\caption{Precedence Table}\label{precedenceops}
\end{figure}
\index{precedence|)}

\index{operator symbols|(}

Many of the operators may be overloaded by the user and retain their
precedence and form (\eg\ infix).  All of the infix operators may also be
given in prefix form; \texttt{x + 1} and \texttt{+(x,1)} are semantically equivalent.  Care must be taken in redefining these operators---if the
preceding declaration ends in an expression there could be an ambiguity.
To handle this situation the language allows declarations to be terminated
with a '\texttt{;}'.  For example,
\begin{pvsex}
  AND: [state, state -> state] = (LAMBDA a,b: (LAMBDA t: a(t) AND b(t)));
  OR: [state, state -> state] = (LAMBDA a,b: (LAMBDA t: a(t) OR b(t)));
\end{pvsex}
%
without the semicolon the second declaration would be seen as an infix
\texttt{OR} and the result would be a parse error.

Another common mistake when overloading operators with predefined meanings
is the assumption that overloading, for example, {\tt IMPLIES} automatically
provides an overloading for {\tt =>}.  This is not the case---they are distinct
operators (which happen to have the same meaning by default) and not syntactic
sugar.

\index{operator symbols|)}

\section{Boolean Expressions}\label{bool-exprs}
\index{boolean expressions}

The Boolean expressions include the constants \texttt{TRUE}\index{true@{\texttt{TRUE}}} and
\texttt{FALSE}\index{false@{\texttt{FALSE}}},
the unary operator \texttt{NOT}\index{not@{\texttt{NOT}}}, and
the binary operators \texttt{AND}\index{and@{\texttt{AND}}} (also written
\texttt{ \&}\index{\&}), \texttt{OR}\index{or@{\texttt{OR}}}, \texttt{
IMPLIES}\index{implies@{\texttt{IMPLIES}}}
(\texttt{=>}\index{=>@\texttt{=>}}),
\texttt{WHEN}\index{when@{\texttt{WHEN}}}, and
\texttt{IFF}\index{iff@{\texttt{IFF}}}
(\texttt{<=>}\index{<=>@\texttt{<=>}}).  The declarations for these are in
the \texttt{booleans} prelude theory.  All of these have their standard
meaning, except for \texttt{WHEN}, which is the converse of
\texttt{IMPLIES} (\ie\ $A$ \texttt{WHEN} $B$ $\equiv$ $B$ \texttt{IMPLIES}
$A$).

Equality\index{equality} (\texttt{=}\index{=}) and
disequality\index{disequality} (\texttt{/=}\index{/=}) are declared in the
prelude theories \texttt{equalities} and \texttt{notequal}.  They are both
polymorphic, the type depending on the types of the left- and right-hand
sides.  If the types are compatible, meaning that there is a common
supertype, then the (dis)equality is of the greatest common supertype.  Otherwise it is a type
error.  For example,
\begin{pvsex}
  S,T: TYPE
  s: VAR S
  t: VAR T
  eq1: FORMULA s = t
  i: VAR \setb{}x: int | x < 10\sete
  j: VAR \setb{}x: int | x > 100\sete
  eq2: FORMULA i = j
\end{pvsex}
%
\texttt{eq1} will cause a type error---remember that \texttt{S} and \texttt{T}
are assumed to be disjoint.  \texttt{eq2} is perfectly typesafe because
they have a common supertype \texttt{int} even though the subtypes have no
elements in common; the equality simply has the value \texttt{FALSE}.

When the equality is between terms of type \texttt{bool}, the semantics
are the same as for \texttt{IFF}.  There is a pragmatic difference in the
way the PVS prover processes these operators.  Equalities may be
used for rewriting, which makes for efficient proofs but is incomplete,
\ie\ the prover may fail to find the proof of a true formula.  On the other
hand the \texttt{IFF} form is complete, but may lead to a large number of
cases.  When in doubt, use equality as the prover provides commands
that turn an equality into an \texttt{IFF}.

%The decision to disallow \texttt{eq1} is a pragmatic one; the
%utility of such a declaration is questionable, and most likely the user
%has made an error in the specification.


\section{\texttt{IF-THEN-ELSE} Expressions}
\index{if-then-else@{\texttt{IF-THEN-ELSE}}}

The \texttt{IF-THEN-ELSE} expression \texttt{IF} {\em cond\/} \texttt{THEN} {\em
expr1\/} \texttt{ELSE} {\em expr2\/} \texttt{ENDIF} is polymorphic; its type is the
common type of {\em expr1\/} and {\em expr2\/}.  The {\em cond\/} must
be of type \texttt{boolean}.  Note that the \texttt{ELSE} part is not
optional as this is an expression, not an operational statement.  The
declaration for \texttt{IF} is in the \texttt{if\_def} prelude theory.  \texttt{
IF-THEN-ELSE} may be redeclared by the user in the same way as \texttt{
AND}, \texttt{OR}, etc.  Note that only \texttt{IF} is explicitly redeclared,
the \texttt{THEN} and \texttt{ELSE} are implicit.

Any number of \texttt{ELSIF} clauses may be present; they are translated into nested
\texttt{IF-THEN-ELSE} expressions.  Thus the expression
\begin{pvsex}
  IF A THEN B
  ELSIF C THEN D
  ELSE E
  ENDIF
\end{pvsex}
%
translates to
\begin{pvsex}
  IF A THEN B
  ELSE (IF C THEN D
        ELSE E
        ENDIF)
  ENDIF
\end{pvsex}

\section{Numeric Expressions}
\index{numeric expressions}

The numeric expressions include the \emph{numerals}\index{numerals} (0, 1,
2, \ldots), the unary operator \texttt{-}\index{-}, and the binary infix
operators \texttt{\char94}\index{\^}, \texttt{+}\index{+},
\texttt{-}\index{-}, \texttt{*}\index{*}, and \texttt{/}\index{/}.  The
numerals are all of type \texttt{real}\index{real@\texttt{real}}.
The typechecker has implicit judgements on numbers; \texttt{0} is known to
be \texttt{real}, \texttt{rat}, \texttt{int} and \texttt{nat}; all others
are known to be non zero and greater than zero.  The relational operators
on numeric types are \texttt{<}\index{<@\texttt{<}}, \texttt{
<=}\index{<=@\texttt{<=}}, \texttt{>}\index{>@\texttt{>}}, and
\texttt{>=}\index{>=@\texttt{>=}}.  The numeric operators and axioms are
all defined in the prelude.  As with the boolean operators, all of these
operators may be defined on new types and retain their original
precedences.

The numerals may also be treated as names, and
overloaded.\index{overloading numberals}\index{numerals!overloading} This
is particularly useful for defining algebraic structures such as groups
and rings, where it is natural to overload `\texttt{0}' and `\texttt{1}'.
Note that such use may include actual parameters, just as for names.  Thus
\texttt{groups[int].0} or \texttt{0[int]} might refer to the group zero
instantiated with the integer carrier set.

\section{Applications}
\index{application expressions}

Function application is specified as in ordinary mathematics; thus the
application of function \texttt{f} to expression \texttt{x} is denoted \texttt{
f(x)}.  Those operator symbols that are binary functions, and their
applications, may be written in prefix or the usual infix notation.  For
example, \texttt{(3 + 5) = (2 * 4)} may be written as \texttt{=(+(3,5),
*(2,4))}.

PVS supports higher-order types, so that functions may yield functions
as values or be curried\index{curried applications}.  For example, given
\texttt{f} of type \texttt{[int -> [int, int -> int]]}, \texttt{f(0)(2,3)}
yields an \texttt{int}.

If the application involves a dependent function type then the result
type of the application is substituted for accordingly.  For example,
\begin{pvsex}
  f: [a:int, b:\setb{}x:int | a < x\sete -> \setb{}y:int | a < y & y <= b\sete]
\end{pvsex}
the application \texttt{f(2,3)} is of type \texttt{\setb{}y:int | 2 < y \& y <=
3\sete}.  This application will also lead to the subtype \tcc\ \texttt{2 < 3}.

Application and tuple expressions have a special relation, due to the
type equivalence of \texttt{[t$_1$,\ldots,t$_n$ -> t]} and \texttt{
[[t$_1$,\ldots,t$_n$] -> t]}, see Section~\ref{tuple-exprs} for details.

\section{Binding Expressions}\label{binding-expressions}
\index{binding expressions}

The binding expressions are those which create a local scope for
variables, including the quantified expressions and
$\lambda$-expressions.  Binding expressions consist of an operator, a
list of bindings, and an expression.  The operator is one of the
keywords \texttt{FORALL}\index{forall@\texttt{FORALL}}, \texttt{
EXISTS}\index{exists@\texttt{EXISTS}}, or \texttt{LAMBDA}\index{lambda@{\texttt{LAMBDA}}}.\footnote{Set
expressions are also binding expressions; see Section~\ref{set-exprs} (page~\pageref{set-exprs}).}
The bindings specify the variables bound by the operator; each variable
has an id and may also include a type or a constraint.  Here is a
contrived example:
\begin{pvsex}
  x,y,z,d,e: VAR real
  ex1: AXIOM FORALL x,y,z: (x + y) + z = x + (y + z)
  ex2: AXIOM FORALL (x,y,z: nat): x * (y + z) = (x * y) + (x * z)
  ex3: AXIOM FORALL (n: num | n /= 0): EXISTS (x | x /= 0): x = 1/n
\end{pvsex}
%
In \texttt{ex1}, variables \texttt{x}, \texttt{y}, and \texttt{z} are all of type
\texttt{real}.  In \texttt{ex2} these same variables are of type \texttt{nat},
shadowing the global declarations.  \texttt{ex3} illustrates
the use of constraints; this is equivalent to the declaration
\begin{pvsex}
  ex3: AXIOM FORALL (n: \setb{}n: num | n /= 0\sete):
               EXISTS (x: \setb{}x | x /= 0\sete): x = 1/n
\end{pvsex}

Quantified expressions\index{quantified expressions} are introduced with
the keywords \texttt{FORALL} and \texttt{EXISTS}.  These expressions are
of type \texttt{boolean}.

Lambda expressions\index{lambda expressions} denote unnamed functions.
For example, the function which adds \texttt{3} to an integer may be
written
\begin{pvsex}
  (LAMBDA (x: int): x + 3)
\end{pvsex}
%
The type of this expression is the function type \texttt{[int ->
numfield]}.\footnote{\texttt{numfield} sits between \texttt{number} and
\texttt{real}, and is where the field operators are introduced.  See
Section~{prelude-numbers}.}  In addition, when the range is \texttt{bool},
a lambda expression may be represented as a set expression; see
Section~\ref{set-exprs}.

All of the binding expressions may involve dependent
types\index{dependent types} in the bindings, \eg
\begin{pvsex}
  FORALL (x: int), (y: \setb{}z: int | x < z\sete): p(x,y)
\end{pvsex}
%
Note that in the instantiation of such an expression during a proof will
generally lead to a subtype \tcc.  For example, substituting \texttt{e$_1$} for
\texttt{x} and \texttt{e$_2$} for \texttt{y} will lead to the \tcc\ \texttt{e$_1$ <
e$_2$}.\footnote{Such \tccs\ may never be seen, as they tend to be
proved automatically during a proof; more complicated examples may be
given, for which the prover would need help from the user.  In addition,
a false \tcc\ can show up, \eg\ substituting \texttt{2} for \texttt{x} and
\texttt{1} for \texttt{y}.  This means that the corresponding expression is
not type correct.}

Constant names may be treated as binding expressions by using a
\texttt{!}  suffix.  For example,
\begin{pvsex}
foo! (x : int) : e
\end{pvsex}
is equivalent to
\begin{pvsex}
foo( LAMBDA (x : int) : e)
\end{pvsex}

\section{\texttt{LET} and \texttt{WHERE} Expressions}
\index{let expressions@{\texttt{LET} expressions}}
\index{where expressions@{\texttt{WHERE} expressions}}

\texttt{LET} and \texttt{WHERE} expressions are provided for convenience,
making some forms easier to read.  Both of these forms provide local
bindings for variables that may then be referenced in the body of the
expression, thus reducing redundancy and allowing names to be provided for common subterms.
Here are two examples:
\begin{pvsex}
  LET x:int = 2, y:int = x * x IN x + y
  x + y WHERE x:int = 2, y:int = x * x
\end{pvsex}
%
The value of each of these expressions is 6.

\texttt{LET} and \texttt{WHERE} expressions are internally translated to
applications of lambda expressions; in this case both expressions
translate to
\begin{pvsex}
  (LAMBDA (x:int) : (LAMBDA (y:int) : x + y)(x * x))(2)
\end{pvsex}
%
These translations should be kept in mind when the semantics of these
expressions is in question.

The type declaration is optional, so the above could be written as
\begin{pvsex}
  LET x = 2, y = x * x IN x + y
  x + y WHERE x = 2, y = x * x
\end{pvsex}
In this case the typechecking of these expressions depends on whether
\texttt{x} and/or \texttt{y} have been previously declared as variables.
If they have, then those delarations are used to determine the type.
Otherwise, the right-hand side of the \texttt{=} is typechecked, and if it
is unambiguous is used to determine the type of the variable.  This is 
one way in which these expressions differ from their translation.
It is usually better to either reference a variable or give the type, as
the typechecker uses the ``natural'' type of the expression as the type of
the variable, which can lead to extra \tccs.

The \texttt{LET} expression has a limited form of pattern matching over
tuples.  An example is
\begin{pvsex}
  p: VAR [int, int]
  +(p): int = LET (m, n) = p IN m + n
\end{pvsex}
which is shorter than the equivalent
\begin{pvsex}
  p: VAR [int, int]
  +(p): int = LET m = p`1, n = p`2 IN m + n
\end{pvsex}


\section{Set Expressions}\label{set-exprs}

In PVS, sets of elements of a type \texttt{t} are represented as
predicates, \ie\ functions from \texttt{t} to \texttt{bool}.  The type of a
set may be given as \texttt{[t -> bool]}, \texttt{pred[t]}, or \texttt{
setof[t]}, which are all type equivalent.\footnote{The prelude theory
\texttt{defined\_types} also defines \texttt{PRED}, \texttt{predicate}, \texttt{
PREDICATE}, and \texttt{SETOF} as alternate equivalents.}
The choice depends wholly on the intended use of the type.
Similarly, a set may be given in the form \texttt{(LAMBDA (x:\ t):\
p(x))} or \texttt{\setb{}x:\ t | p(x)\sete}; these are equivalent
expressions.\footnote{In fact, internally they are represented by the
same abstract syntax, they simply print differently.} Note that the
latter form may also represent a type---this usually causes no
confusion as the context generally makes it clear which is expected.
The usual functions and properties of sets are provided in the prelude
theory \texttt{sets}.


\section{Tuple Expressions}\label{tuple-exprs}
\index{tuple expressions}

A tuple expression of the type \texttt{[t$_1$,\ldots,t$_n$]} has the form
\texttt{(e$_1$,\ldots,e$_n$)}.  For example, \texttt{(2, TRUE, (LAMBDA x:\ x +
1))} is of type \texttt{[nat, bool, [nat -> nat]]}.  0-tuples are not
allowed, and 1-tuples are treated simply as parenthesized expressions.
The following relation holds between function types and tuple types:
\begin{pvsex}
  [[t\(\sb{1}\),\ldots,t\(\sb{n}\)] -> t] \(\equiv\) [t\(\sb{1}\),\ldots,t\(\sb{n}\) -> t]
\end{pvsex}
%
This equivalence is most important in theory parameters; it allows one
theory to take the place of many.  For example the \texttt{functions}
theory from the prelude may be instantiated by the reference
\texttt{injective?[[int,int,int],int]}.  Applications of an element \texttt{f} of
this type include \texttt{f(1,2,3)}, \texttt{f((1,2,3))}, and \texttt{f(e)},
where \texttt{e} is of type \texttt{[int,int,int]}.

\section{Projection Expressions}\label{projection-exprs}
\index{projection expressions}

The components of an expression whose type is a tuple can be accessed
using the projection operators \texttt{`1}, \texttt{`2}, \ldots or
\texttt{PROJ\_1}, \texttt{PROJ\_2}, \ldots.  The former are preferred.
Like reserved words, projection expressions are case insensitive and may
not be redeclared.  For the most part, projection expressions are
analogous to field accessors for record types.  For example,
\begin{pvsex}
  t: [int, bool, [int -> int]]
  ft: FORMULA t`2 AND t`1 > t`3(0)
  ft_deprecated: FORMULA PROJ_2(t) AND PROJ_1(t) > (PROJ_3(t))(0)
\end{pvsex}

Projection expressions may be used without an argument as long as the
context determines the tuple type involved.  For example, in the following
it is obvious what tuple type is involved.
\begin{pvsex}
  F: [[[int, bool, [int -> int]] -> bool] -> bool]
  FP: FORMULA F(PROJ_2)
\end{pvsex}
Note that the \texttt{PROJ} keyword must be used in such cases, as, e.g.,
\texttt{`2} is not an expression.  In the following example we see that
the context does not provide enough information.
\begin{pvsex}
  PP: FORMULA PROJ_2 = PROJ_2
\end{pvsex}
To deal with such situations, the syntax for projections has been extended
to allow the tuple type to be provided.
\begin{pvsex}
  PP: FORMULA PROJ_2[[int, bool, [int -> int]]] = PROJ_2
\end{pvsex}
In this case only one of the operators needs to be annotated.  This looks
like a use of actual parameters, but it is not, as the \texttt{PROJ} is
not a name, and does not belong to a theory.


\section{Record Expressions}\label{record-expressions}
\index{record expressions}

Record expressions are of the form \texttt{(\# a$_1$ := e$_1$, \ldots,
a$_n$ := e$_n$ \#)}, which has type \texttt{[\# a$_1$:\ t$_1$, \ldots,
a$_n$:\ t$_n$ \#]}, where each \texttt{e$_i$} is of type \texttt{t$_i$}.
Partial record expressions are not allowed; all fields must be given.  If
it is desired to give a partial record, declare an uninterpreted constant
or variable of the record type, and use override expressions to specify
the given record at the fields of interest.  For example,
\begin{pvsex}
  rc: [# a, b : int #]
  re: [# a, b : int #] = rc WITH [`a := 0]
\end{pvsex}

The type of a record expression is determined by the type of its
components.  Thus \texttt{(\# a := 3, b := 2 \#)} is of type \texttt{[\# a,
b: real \#]}.  This means that a record expression is never of a dependent
record type directly, though it may be used where a dependent record is
expected, and \tccs\ may be generated as a result.  For example,
\begin{pvsex}
  R: TYPE = [# a: int, b: \setb{}x: int | x < a\sete #]
  r: R = (# a := 3, b := 4 #)
\end{pvsex}
%
leads to the (unprovable) \tcc\ \texttt{4 < 3}.

Record expressions may be introduced without introducing the record type
first, and the type of a record expression is determined by its
components, independently of any previously declared record type.  For
this reason record types do not automatically generate associated accessor
functions.

\section{Record Accessors}

The components of an expression of a record type are accessed using the
corresponding field name.  There are two forms of access.  For example if
\texttt{r} is of type \texttt{[\# x, y: real \#]}, the x-component may be
accessed using either \texttt{r`x} or \texttt{x(r)}.  The first form is
preferred as there is less chance for ambiguity.

As noted above, accessors are not stand-alone functions.  However, you can
define your own functions to provide this capability, and even use the
same name.  For example:
\begin{pvsex}
  point: TYPE = [# x, y: real #]
  x(p:point): real = p`x
  y(p:point): real = p`y
\end{pvsex}
Now \texttt{x} and \texttt{y} may be provided wherever a function is
expected.  Note that this means that a subsequent expression of the form
\texttt{x(p)} could be ambiguous, but the record field accessor is always
preferred, so in practice such ambiguities don't arise.

\section{Cotuple Expressions}\label{cotuple-expressions}
\index{cotuple expression}

Elements of cotuple types \texttt{[t$_1$ + \ldots + t$_n$]} are constructed
with the \emph{injection} operators \texttt{IN\_$i$} of type
\texttt{[t$_i$ -> [t$_1$ + \ldots + t$_n$]]}.  Thus if $e$ is of type
\texttt{t$_i$}, \texttt{IN\_$i$($e$)} is of the cotuple type.  If $x$ is
an element of a cotuple type, \texttt{IN?\_$i$($x$)} is a boolean that
tests if $x$ belongs to the $i^{th}$ component, and if it does,
\texttt{OUT\_$i$($x$)} returns the associated value of type
\texttt{t$_i$}.  Note that this is similar to a datatype of the form
\begin{pvsex}
  cotup: DATATYPE
   BEGIN
    IN_1(OUT_1: t\(\sb{1}\)): IN?_1
    \(\cdots\)
    IN_\(n\)(OUT_\(n\): t\(\sb{n}\)): IN?_\(n\)
   END cotup
\end{pvsex}
The differences are that cotuples are not recursive, do not generate all
the functions and axioms associated with datatypes, and allow for any
number of component types---using datatypes a new one would have to be
given for each arity.

The analogy works also for the \texttt{CASES} expression described in
Section~\ref{cases-expressions}.  This allows access to the values of a
cotuple element.  It has the form
\begin{pvsex}
  CASES \(e\) OF
    IN_1(x1): f\(\sb{1}\)(x1),
    \vdots
    IN_\(n\)(x\(n\)): f\(\sb{n}\)(x\(n\))
  ENDCASES
\end{pvsex}
where each \texttt{f$_i$} is an expression of type \texttt{[t$_i$ ->
$T$]}, and the common return type $T$ is the type of the \texttt{CASES}
expression.  For example, if \texttt{x} is of type \texttt{[int + bool +
[int -> int]}, the following expression will return a boolean value.
\begin{pvsex}
  CASES x OF
    IN_1(i): i > 0,
    IN_2(b): NOT b,
    IN_3(f): FORALL (n: int): f(f(n)) = f(n)
  ENDCASES
\end{pvsex}
If there are any missing components in the \texttt{CASES} expression, a
\emph{cases \tcc}\index{cases TCC}\index{TCC!cases} will be generated
stating that the cotuple expression must be one of the given selections,
unless there is an \texttt{ELSE} selection.

Like the projection operators \texttt{PROJ\_$i$}, the \texttt{IN\_$i$},
\texttt{OUT\_$i$} and \texttt{IN?\_$i$} operators make be disambiguated by
adding the cotuple type reference to the operator, for example,
\texttt{IN\_2[int + int](3)} or \texttt{IN?\_1[coT]}.  Note that although
they have the form of actual parameters, they are not, as these operators
are built in and not associated with any theory.  Also, for brevity, only
the cotuple type is given, not the full type of the operator.  There are a
number of axioms associated with cotuples that are built in to the PVS
typechecker and prover.


\section{Override Expressions}
\index{override expression}
\index{update expression}
\index{with expression}

Functions, tuples, records, and datatype elements may be ``modified'' by
means of the override expression.  The result of an override expression is
a function, tuple, record, or datatype element that is exactly the same as
the original, except that at the specified arguments it takes the new
values.  For example,
\begin{pvsex}
  identity WITH [(0) := 1, (1) := 2]
\end{pvsex}
%
is the same function as the \texttt{identity} function (defined in the
prelude) except at argument values \texttt{0} and \texttt{1}.  This is exactly
the same expression as either of
\begin{pvsex}
  (identity WITH [(0) := 1]) WITH [(1) := 2] {\rm or}
  (LAMBDA x: IF x = 1 THEN 2 ELSIF x = 0 THEN 1 ELSE identity(x))
\end{pvsex}

This order of evaluation ensures that functions remain total, and allows
for the possibility of expressions such as
\begin{pvsex}
  identity WITH [(c) := 1, (d) := 2]
\end{pvsex}
where \texttt{c} and \texttt{d} may or may not be equal.  If they are
equal, then the value of the override expression at the common argument is
\texttt{2}.

More complex overrides can be made using nested arguments; for example,
\begin{pvsex}
  R: TYPE = [# a: int, b: [int -> [int, int]] #]
  r1: R
  r2: R = r1 WITH [`a := 0, `b(1)`2 := 4]
\end{pvsex}
{\tt r2} is equivalent to
\begin{pvsex}
  (# a := 0,
     b := LAMBDA (x: int):
           IF x = 1
           THEN (r1`b(x)`1, 4)
           ELSE r2`b(x)
           ENDIF #)
\end{pvsex}

Updating a datatype element amounts to updating the accessor(s) associated
with a constructor.  For example, if \texttt{lst} is of type
\texttt{(cons?[nat])}, then \texttt{lst WITH [`car := 3]} returns a list
that is the same as \texttt{lst}, but whose first element is \texttt{3}.
If \texttt{lst} is given type \texttt{list[nat]}, then the same override
expression generates a \tcc\ obligation to prove that \texttt{lst} is a
\texttt{cons?}.  Because accessors may be both dependent and overloaded,
\tccs\ may get complicated.  For example,
\begin{pvsex}
  dt: DATATYPE
  BEGIN
   c0: c0?
   c1(x: int, a: \{z: (even?) | z > x\}, b: int): c1?
   c2(x: int, a: \{n: nat | n > x\}, c: int): c2?
  END dt
\end{pvsex}
If \texttt{d} is of type \texttt{dt}, the update expression \texttt{d WITH
[a := y]} leads to the \tcc
\begin{pvsex}
  f1_TCC1: OBLIGATION
    (c1?(d) AND even?(y) AND y > x(d)) OR
     (c2?(d) AND y >= 0 AND y > x(d));
\end{pvsex}

Another form of override expression is the maplet, indicated using
\texttt{|->} in place of \texttt{:=}.  This is used to extend the domain
of the corresponding element; for example, if \texttt{f:[nat -> int]} is
given, then \texttt{f WITH [(-1) |-> 0]} is a function of type
\texttt{[\setb{}i:int | i >= 0 OR i = -1\sete -> int]}.  This is especially useful
with dependent types, see Section~\ref{dependent-types}.  Domain extension
is also possible for record and tuple types; for example, \texttt{r1 WITH
[`c |-> 3]} is of type \texttt{[\# a:\ int, b:\ [int -> [int,int]], c:\ int
\#]}, and if \texttt{t1} is of type \texttt{[int, bool]}, then \texttt{t1
WITH [`3 |-> 1]} is of type \texttt{[int, bool, int]}.  It is an error to
extend a tuple type such that gaps are left, so \texttt{t1 WITH [`4 |->
1]} is illegal, though \texttt{t1 WITH [`3 |-> 1, `4 |-> 1]} is allowed.
Gaps would also be left if nested arguments were given, so \texttt{r1 WITH
[`c(0) |-> 0]} is also illegal.  It would have to be given as \texttt{r1
WITH [`c := LAMBDA (x:\ int):\ IF x = 0 THEN 0 ELSE $\cdots$ ENDIF]}, where
the gap $\cdots$ now has to be filled in.  Domain extension is not
possible for datatype elements, as a new datatype theory would need to be
generated for each such extension.

In the past, the two forms of assignment (using \texttt{:=} and
\texttt{|->}) were merely alternative notation, and domains would be
extended automatically whenever the typechecker could not determine that
the argument belonged to the domain.  In most cases, extending the domain
unnecessarily is harmless.  However, when terms get large, the types can
get cumbersome, slowing down the system dramatically.  Even worse, when
domains are extended and matched against a rewrite rule with the original
type, the match can fail, and the automatic rewrite will not be triggered.
For this reason, it is always best to use the maplet on function types
only when actually extending the domain.

\section{Coercion Expressions}\label{coercions}

Coercion expressions are of the form \texttt{expr ::\ type-expr}, indicating
that the expression \texttt{expr} is expected to be of type \texttt{
type-expr}.  This serves two purposes.  First, although PVS allows a
liberal amount of overloading, it cannot always disambiguate things for
itself, and coercion may be needed.  For example, in
\begin{pvsex}
  foo: int
  foo: [int -> int]
  foo: LEMMA foo = foo::int
\end{pvsex}
%
the coercion of \texttt{foo} to \texttt{int} is needed, because otherwise the
typechecker cannot determine the type.  Note that only one of the sides
of the equation needs to be disambiguated.

The second purpose of coercion is as an aid to typechecking; by
providing the expected type in key places within complex expressions,
the resulting \tccs\ may be considerably simplified.

\input{tables}

\index{expression|)}

% Document Type: LaTeX
% Master File: language.tex

\chapter{Theories}\label{theories}
\index{theories}

Specifications in \pvs\ are built from \emph{theories}, which provide
genericity, reusability, and structuring.  \pvs\ theories may be
parameterized.  A theory consists of a \emph{theory
identifier}, a list of formal \emph{parameters}, an \texttt{EXPORTING}
clause, an \emph{assuming part}, a \emph{theory body}, and an ending
id.  The syntax for theories is shown in Figure~\ref{bnf-theory}.

\pvsbnf{bnf-theory}{Theory Syntax}

\pvsbnf{bnf-assuming}{Assuming Syntax}

\pvsbnf{bnf-theory-part}{Theory Part Syntax}

Everything is optional except the identifiers and the keywords.  Thus
the simplest theory has the form
\begin{pvsex}
  triv : THEORY
    BEGIN
    END triv
\end{pvsex}

The formal parameters, assuming, and theory body consist of declarations
and \emph{importings}.  The various declarations are described in
Section~\ref{declarations}.  In this section we discuss the restrictions
on the allowable declarations within each section, the formal parameters,
the assuming part, and the exportings and importings.

The \texttt{groups} theory below illustrates these concepts.  It views a
group as a 4-tuple consisting of a type \texttt{G}, an identity element
\texttt{e} of \texttt{G}, and operations \texttt{o}\footnote{Recall that
\texttt{o} is an infix operator.} and \texttt{inv}.  Note the use of the
type parameter \texttt{G} in the rest of the formal parameter list.  The
assuming part provides the group axioms.  Any use of the \texttt{groups}
theory incurs the obligation to prove all of the \texttt{ASSUMPTION}s.
The body of the \texttt{groups} theory consists of two theorems, which can
be proved from the assumptions.

\pvstheory{groups-alltt}{Theory \texttt{groups}}{groups-alltt}

\section{Theory Identifiers}

The theory identifier introduces a name for a theory; as described in
Section~\ref{names}, this identifier can be used to help disambiguate
references to declarations of the theory.

In the \pvs\ system, the set of theories currently available to the
session form a \emph{context}.  Within the context theory names must be
unique.  There is an initial context available, called the prelude
% (described in Appendix~\ref{prelude}),
that provides, among other things,
the Boolean operators, equality, and the \texttt{real}, \texttt{rational},
\texttt{integer}, and \texttt{naturalnumber} types and their associated
properties.  The only difference between the prelude and user-defined
theories is that the prelude is automatically imported in every theory,
without requiring an explicit \rsv{IMPORTING} clause.

The end identifier must match the theory identifier, or an error is
signaled.


\section{Theory Parameters}\label{parameters}
\index{theory parameters|(}
\index{formal parameters|see{theory parameters}}

The theory parameters allow theory schemas to be specified.  This
provides support for \emph{universal polymorphism}\index{polymorphism}

Theory parameters may be types, subtypes, constants, or
theories,\footnote{This is discussed in Chapter~\ref{interpretations}.}
interspersed with importings.  Theory parameters must have unique
identifiers.  The parameters are ordered, allowing later parameters to
refer to earlier parameters or imported entities.  This is another form of
dependency, akin to dependent types (see Section~\ref{dependent-types}).
A theory is \emph{ instantiated} from within another theory by providing
\emph{actual parameters}\index{actual parameters} to substitute for the
formals.  Actual parameters may occur in importings, exportings, theory
declarations, and names.  In each case they are enclosed in braces
(\texttt{[} and \texttt{]}) and separated with commas.

The actuals must match the formals in number, kind, and (where
applicable) type.  In this matching process the importings, which
must be enclosed in parentheses, are ignored.  For example, given the
theory declaration

\begin{pvsex}
  T [t: TYPE,
     subt: TYPE FROM t
     (IMPORTING orders[subt]) <=: (partial_order?),
     c: subt,
     d: \setb{}x:subt | c <= x\sete]
\end{pvsex}
a valid instance has five actual parameters; an example is
\begin{pvsex}
  T[int, \setb{}x:nat | x < 10\sete, <=, 5, 6]
\end{pvsex}
%
Note that the matching process may lead to the generation of \emph{actual}
\tccs.\index{actual TCC}\index{TCC!actual}

\index{theory parameters|)}


\section{Importings and Exportings}\label{importings}

The importing and exporting clauses form a hierarchy, much like the
subroutine hierarchy of a programming language.

Names declared in a theory may be made available to other theories in the
same context by means of the \texttt{EXPORTING} clause.  Names exported by
a given theory may be imported into a second theory by means of the
\texttt{IMPORTING} clause.  Names that are exported from one theory are
said to be \emph{visible} to any theory which uses the given theory.  In
this section we describe the syntax of the \texttt{EXPORTING} and
\texttt{IMPORTING} clauses and give some simple examples.

\pvsbnf{bnf-exporting}{Importing and Exporting Syntax}


\subsection{The \texttt{EXPORTING} Clause}
\index{exporting@\texttt{EXPORTING}|(}

The \texttt{EXPORTING} clause specifies the names declared in the theory
which are to be made available to any theory \texttt{IMPORTING} it.  It may
also specify instances of the theories which it imported to be exported.
The syntax of the \texttt{EXPORTING} clause is given in
Figure~\ref{bnf-exporting}.

\noindent
The \texttt{EXPORTING} clause is optional; if omitted, it defaults to
\begin{alltt}
  EXPORTING ALL WITH ALL
\end{alltt}

Any declared name may be exported except for variable declarations and formal parameters.
When \texttt{ALL} is specified for the \emph{ExportingNames}, all
entities declared in the theory aside from the variables are exported.
If a list of names is specified, then these are exported.  Finally, when
a list of names follows \texttt{ALL BUT}, all names aside from these are exported.

Since PVS supports overloading, it is possible that the exported name will
be ambiguous.  Such names may be disambiguated by including the type, if
it is a constant, or by including one of the keywords \texttt{TYPE} or
\texttt{FORMULA}.  The keyword \texttt{TYPE} is used for any type
declaration, and \texttt{FORMULA} is used for any formula declaration
(including \texttt{AXIOM}s, \texttt{LEMMA}s, etc.)  If not disambiguated,
all declarations (except variables and formals) with the specified id will
be exported.

When names are specified they are checked for \emph{completeness}.
This means that when a name is exported all of the names on which the
corresponding declaration(s) depend must also be exported.  Thus, for
example, given the following declarations
\begin{alltt}
  sometype: TYPE
  someconst: sometype
\end{alltt}
it would be illegal to export \texttt{someconst} without also exporting
\texttt{sometype}.  Note that this check is unnecessary if exporting
\texttt{ALL} without the \texttt{BUT} keyword.

In some cases it is desirable (or necessary for completeness) to
export some of the instances of the theories which are used by the
given theory.  This is done by specifying a \texttt{WITH} subclause as a
part of the \texttt{EXPORTING} clause.  The \texttt{WITH} subclause may be
\texttt{ALL}, indicating that all instances of theories used by the given
theory are exported.  If \texttt{CLOSURE} is specified, then the
typechecker determines the instances to be exported by a \emph{
completion analysis}\index{completion analysis} on the exported
names.  Completion analysis determines those entities that are
directly or indirectly referenced by one of the exported
names.\footnote{Proofs are not used in completion analysis.} Finally,
a list of theory names may be given; in this case the theory names
must be complete in the sense that if an exported name refers to an
entity in another theory instance, then that theory instance must be
exported also.  Other theory instances may also be exported even if
not actually needed for completeness in this sense.  The \texttt{WITH}
subclause may only reference theory instances, \ie\ theory names with
actuals provided for all of the corresponding formal parameters.

As a practical matter, it is probably best not to include an
\texttt{EXPORTING} clause unless there is a good reason.  That way
everything that is declared will be visible at higher levels of the
\texttt{IMPORTING} chain.

\index{exporting@\texttt{EXPORTING}|)}

\subsection{\texttt{IMPORTING} Clauses}
\index{importings|(}

\texttt{IMPORTING} clauses import the visible names of another theory.
\texttt{IMPORTING} clauses may appear in the formal parameters list, the
assuming part, or the theory part of a theory.  In addition, theory
abbreviations implicitly import the theory name that they abbreviate (see
Section~\ref{theory-abbreviations}).

The names appearing in an \texttt{IMPORTING} or theory abbreviation
specifies a theory and optionally gives an instance of that theory, by
providing actual parameters corresponding to the formal parameters of the
theory used or mappings for the uninterpreted types and constants (see
Chapter~\ref{interpretations}).  \texttt{IMPORTING}s are cumulative;
entities made visible at some point in a theory are visible to every
declaration following.

An \texttt{IMPORTING} with actual parameters provided is said to be a \emph{
theory instance}.\index{theory instance} We use the same terminology for
an \texttt{IMPORTING} that refers to theory that has no formal parameters.
Otherwise it is referred to as a \emph{generic}\index{generic reference}
reference.

A single theory may appear in any number of \texttt{IMPORTING}s of another
theory, both instantiated and generic.  Obviously, any time there is
more than one \texttt{IMPORTING} of a given theory there is a chance for
ambiguity.  Section~\ref{names} discusses such ambiguities, explaining
how the system attempts to resolve them and how the user can
disambiguate in situations where the system cannot.

An \texttt{IMPORTING} forms a relation between the theory containing the
\texttt{IMPORTING} and the theory referenced.  The transitive closure of
the \texttt{IMPORTING} relation is called the \emph{importing chain} of a
theory.  The importing chain must form a directed acyclic graph; hence a
theory may not end up importing itself, directly or indirectly.
\index{importings|)}


\subsubsection{Theory Abbreviations}\label{theory-abbreviations}
\index{theory abbreviations}

A theory abbreviation is a form of importing that introduces a new name
for a theory instance, providing an alternate means for referring to the
instance.  For example, given the importing\footnote{Prior to the
introduction of theory interpretations, this was written as
\texttt{fsets:\ THEORY = sets[[integer -> integer]].}}
\begin{pvsex}
  IMPORTING sets[[integer -> integer]] AS fsets
\end{pvsex}
where \texttt{sets} is a theory in which the function \texttt{member} is
declared, the name \texttt{fsets.member} is equivalent to
\texttt{sets[[integer -> integer]].member}.


\section{Assuming Part}\label{assuming}

The assuming part consists of top-level declarations and
\texttt{IMPORTING}s.  The assuming part precedes the theory part, so the
theory part may refer to entities declared in the assuming part.  The
grammar for the assuming part is given in Figure~\ref{bnf-assuming}.

The primary purpose of the assuming part is to provide constraints on the
use of the theory, by means of \texttt{ASSUMPTION}s.  These are formulas
expressing properties that are expected to hold of any instance of the
theory.  They are generally stated in terms of the formal parameters, and
when instantiated they become \emph{assuming} \tccs.\index{assuming
TCC}\index{TCC!assuming} For example, given the theory \texttt{groups}
above, the importing
\begin{pvsex}
  IMPORTING groups[int, 0, +, -]
\end{pvsex}
generates the following obligations
\begin{pvsex}
  IMP_groups_TCC1: OBLIGATION FORALL (a, b, c: int): a + (b + c) = (a + b) + c;

  IMP_groups_TCC2: OBLIGATION FORALL (a: int): 0 + a = a AND a + 0 = a;

  IMP_groups_TCC3: OBLIGATION FORALL (a: int): (-)(a) + a = 0 AND a + (-)(a) = 0;
\end{pvsex}

Except for the variable declarations, the declarations of the assumings
are all externally visible.  
  
The dynamic semantics of an \emph{assuming} part of a theory is as
follows.  Internal to the theory, assumptions are used exactly as axioms
would be used.  Externally, for each import of a theory, the assumptions
have to be discharged (i.e., proved) with the actual parameters replacing
the formal parameters.  Note that in terms of the proof chain, every proof
in a theory depends on the proofs of the assumptions.

Assuming \tccs\ are generated when a theory is instantiated, which may or
may not occur when it is imported.  Thus if a theory with assumptions is
imported generically, the assuming \tccs\ are not generated until some
reference is instantiated.  If a theory instance is imported, then the
assuming \tccs\ precede the importing in the dynamic semantics.  Note that
this may not make sense, as the assumings may refer to entities that are
not visible until after the theory is imported.  Thus the following is
illegal.
\begin{session}
  assuming_test[n: nat, m: {x:int | x < n}]: THEORY
  BEGIN
   ASSUMING
    rel_prime?(x, y: int): bool = EXISTS (a, b: int): x*a + y*b = 1
    rel_prime: ASSUMPTION rel_prime?(n,m)
   ENDASSUMING
  END assuming_test

  assimp: THEORY
  BEGIN
   IMPORTING assuming_test[4, 2]
  END assimp
\end{session}
And leads to the following error message.
\begin{pvsex}
  Error: assumption refers to 
    assuming_test[4, 2].rel_prime?,
  which is not visible in the current theory
\end{pvsex}
There are a number of ways to solve this problem.  Perhaps the simplest is
to first import the theory generically, then import the instance.
\begin{pvsex}
   IMPORTING assuming_test
   IMPORTING assuming_test[4, 2]
\end{pvsex}
Now the reference to \texttt{rel\_prime?} makes sense in the assuming
\tcc\ generated for the second importing.

In this case, another solution is to simply define \texttt{rel\_prime?} as
a \emph{macro} (see Section~\ref{macro-declarations}).
\begin{pvsex}
  rel_prime?(x, y: int): MACRO bool = EXISTS (a, b: int): x*a + y*b = 1
\end{pvsex}
Of course, this will not work if the declaration in question is a
recursive or inductive definition.

Another solution is to provide the declaration in a theory that is
imported in both the theory with the assuming and the theory importing
that theory.
\begin{session}
  rel_prime[y:int]: THEORY
  BEGIN
   rel_prime?(x: int): bool = EXISTS (a, b: int): x*a + y*b = 1
  END assth2

  assuming_test[n: nat, m: {x:int | x < n}]: THEORY
  BEGIN
   ASSUMING
    IMPORTING rel_prime[m]
    rel_prime: ASSUMPTION rel_prime?(n)
   ENDASSUMING
  END assuming_test2

  assuming_imp: THEORY
  BEGIN
   IMPORTING rel_prime[2], assuming_test[4, 2]
  END assuming_imp
\end{session}
Now the reference to \texttt{rel\_prime?} in the assuming \tcc\ associated
with \texttt{assuming\_test[4, 2]} is the same as the previously imported
instance, so there is no problem.  In the theory \texttt{assuming\_imp},
\texttt{rel\_prime} may also be imported generically.  However, if
\texttt{rel\_prime} is not imported, or is imported with a different
parameter (e.g., \texttt{rel\_prime[3]}) then the above error is produced.


\section{Theory Part}

The theory part consists of top-level declarations and \texttt{IMPORTING}s.
Declarations are ordered; references may not be made to declarations
which occur later in the theory.  The theory part usually contains the
main body of the theory.  Assuming declarations are not allowed in the
theory part.  The grammar for the theory part is given in
Figure~\ref{bnf-theory-part}.

%%% Local Variables: 
%%% TeX-command-default: "Make"
%%% mode: latex
%%% TeX-master: "language"
%%% End: 

% Document Type: LaTeX
% Master File: interpretations-final.tex
\documentclass[11pt,twoside,openright,titlepage]{cslreport}
\usepackage{relsize}
\usepackage{makebnf}
\usepackage{alltt}
%\usepackage{doublespace}

\usepackage{cite}
\usepackage{/homes/rushby/tex/oz}
%\usepackage{/homes/rushby/tex/cslrep}
\usepackage{url}
\usepackage{psfig}
\usepackage{times}
\usepackage{/homes/owre/tex/session}
\usepackage{boxedminipage}
\def\mapb{\char"7B\char"7B}
\def\mape{\char"7D\char"7D}
\def\setb{\char"7B}
\def\sete{\char"7D}
\newcommand{\specware}{{\sc Specware}}
\textwidth 5.5in
\oddsidemargin .65in
\evensidemargin .41in
\raggedbottom
\sloppy


\begin{document}
\begin{titlepage}
\title{\textbf{\larger Theory Interpretations in PVS}}
\author{Sam Owre \and N. Shankar
\date{April 2001}
\cslreportnumber{SRI-CSL-01-01}
\maketitle
\noindent
%\hspace*{-1in}
\raisebox{-0.8cm}[1cm][1cm]{\srilogo}
\acknowledge{Funded by NASA Langley Research Center contract numbers
NAS1-20334 and NAS1-0079 and DARPA/AFRL contract number F33615-00-C-3043.}
\end{titlepage}

\cleardoublepageblank
\pagenumbering{roman}

\begin{abstract}
\thispagestyle{plain}

We describe a mechanism for theory interpretations in PVS.  The
mechanization makes it possible to show that one collection of theories is
correctly interpreted by another collection of theories under a
user-specified interpretation for the uninterpreted types and constants.
A theory instance is generated and imported, while the axiom instances are
generated as proof obligations to ensure that the interpretation is valid.
Interpretations can be used to show that an implementation is a correct
refinement of a specification, that an axiomatically defined specification
is consistent, or that a axiomatically defined specification captures its
intended models.

In addition, the theory parameter mechanism has been extended with a
notion of \emph{theory as parameter} so that a theory instance can be
given as an actual parameter to an imported theory.  Theory
interpretations can thus be used to refine an abstract specification or to
demonstrate the consistency of an axiomatic theory.  In this report we
describe the mechanism in detail.  This extension is a part of PVS version
3.0, which will be publicly released in mid-2001.

\end{abstract}

\tableofcontents
\cleardoublepage
\setcounter{page}{0} 
\pagenumbering{arabic}

\chapter{Introduction}

Theory interpretations have a long history in first-order
logic~\cite{Shoenfield,Enderton,Monk76}\@.  They are used to show that the
language of a given source theory $S$ can be interpreted within a target
theory $T$ such that the corresponding interpretation of axioms of $S$
become theorems of $T$\@.  This demonstrates the consistency of $S$
relative to $T$, and also the decidability of $S$ modulo the decidability
of $T$\@.  Theories and theory interpretations have also become important
in higher-order logic and type theory with languages such as {\sc
Ehdm}~\cite{EHDM:manuals}, IMPS~\cite{Farmer:interpretations},
HOL~\cite{Windley92}, Maude~\cite{Maude}, Extended
ML~\cite{SannellaDT:essential-concepts97}, and
\specware~\cite{SrinivasJullig95}\@.  In these languages, theories are
used as structuring mechanisms for large specifications so that abstract
theories can be refined into more concrete ones through interpretation.
In this report, we describe a theory interpretation mechanism for the PVS
specification language.

Specification languages and programming languages usually have some
mechanism for packaging groups of definitions into modules.  Lisp and Ada
have \emph{packages}\@.  Standard ML has a module system consisting of
signatures, structures corresponding to a signature, and functors that map
between structures.  Packages can be made generic by allowing certain
declarations to serve as parameters that can be instantiated when the
package is imported.  Ada has \emph{generic} packages that allow
parameters.  SML \emph{functors} can be used to construct parametric
modules.  C++ allows \emph{templates}.

In specification languages, a \emph{theory} groups together related
declarations of constants, types, axioms, definitions, and theorems.  One
way of demonstrating the consistency of such a theory is by providing an
interpretation for the uninterpreted types and constants under which the
axioms are valid.  The definitions and theorems corresponding to a valid
interpretation can then be taken as valid without further proof as long as
they have been verified in the source theory.  The technique of
interpreting one axiomatic theory in another has many uses.  It can be
used to demonstrate the consistency or decidability of the former theory
with respect to the latter theory.  It can also be used to refine an
abstract theory down to an executable implementation.

Interpretations are also useful in showing that the axioms capture the
intended models.  For example, a clock synchronization algorithm was
developed in \textsc{Ehdm} and was later shown to be consistent using the
mappings, but it turned out that in one place $<$ was used instead of
$\leq$, and because of this a set of perfectly synchronized clocks was
actually disallowed by the model.  Using interpretations in this way is
similar to testing in allowing for the exploration of the space of models
for the theory.

Parametric theories in PVS share some of the features of theory
interpretations.  Such theories can be defined with formal parameters
ranging over types and individuals, for example,\footnote{This exploits a
new feature of PVS version 3.0, in which numbers may be overloaded as
names.}
{\smaller\begin{alltt}
group[G: TYPE, + : [G, G -> G], 0: G, -: [G -> G]]: THEORY
  BEGIN
    \vdots
  END group
\end{alltt}}
An instance of the theory \texttt{group} can be imported by supplying
actual parameters, the type \texttt{int} of integers, integer addition
{\tt +}, zero \texttt{0}, and integer negation {\tt -}, corresponding to
the formal parameters, as in {\tt group[int, +, 0, -]}\@.  A theory can
include assumptions about the parameters that have to be discharged when
the actual parameters are supplied.  For example, the group axioms can
be given as assumptions in the \texttt{group} theory above.  However,
there are some crucial differences between parametric theories and theory
interpretations.  In particular, if axioms are always specified as
assumptions, then the theory can be imported only by discharging these
assumptions.  It is necessary to have separate mechanisms for importing a
theory with the axioms, and for interpreting a theory by supplying a valid
interpretation, that is, one that satisfies its axioms.

The PVS theory interpretation mechanism is quite similar to that for
theory parameterization.  The axiomatic specification of groups could
alternately be given in a theory
{\smaller\begin{alltt}
group: THEORY
 BEGIN
  G: TYPE+
  +: [G, G -> G]
  0: G
  -: [G -> G]
   \vdots
 END group
\end{alltt}}
The group axioms are declared in the body of the theory.  Such a theory
can be interpreted by writing \texttt{\smaller group\mapb{}G := int, + :=
+, 0 := 0, - := -\mape{}}\@.  Here the left-hand sides refer to the
uninterpreted types and constants of theory \texttt{group}, and the
right-hand sides are the interpretations.  This notation resembles that of
theory parameterization and is used in contexts where a theory is
imported.  The corresponding instances of the group axioms are generated
as proof obligations at the point where the theory is imported.  The
result is a theory that consists of the corresponding mapping of the
remaining declarations in the theory \texttt{group}\@.  This allows the
theory \texttt{group} to be used in other theories, such as rings and
fields, and also allows the theory \texttt{group} to be suitably
instantiated by group structures.

Theory interpretations largely subsume parametric theories in the sense
that the theory parameters and the corresponding assumings can instead be
presented as uninterpreted types and constants and axioms so that the
actual parameters are given by means of an interpretation.  However, a
parametric theory with both assumings and axioms involving the parameters
is not equivalent to any interpreted theory, as the parameters may be
instantiated without the need to prove the axioms.  It is also useful to
have parametric theories as a convenient way of grouping together all the
parameters that must be provided whenever the theory is used.  For
example, typical theory parameters such as the size of an array, or the
element type of an aggregate datatype such as an array, list, or tree, are
required as inputs whenever the corresponding theories are used.  While
this kind of parameterization can be captured by theory interpretations,
it would not capture the intent that these parameters are \emph{required}
inputs wherever the theory is used.  Furthermore, when an operation from a
parametric theory is used, PVS attempts to figure out the actual
parameters based on the context of its use.  It can do this because the
formal parameters are precisely delimited.  The corresponding inference is
harder for theory interpretations since there might be many possible
interpretations that are compatible with the context of the operations
use.

In addition to the uninterpreted types and constants in a source theory
$S$, the PVS theory interpretation mechanism can also be used to interpret
any theories that are imported into $S$ by means of the \texttt{THEORY}
declaration.  The interpretation of a theory declaration for $S'$ imported
within $S$ must itself be a theory interpretation of $S'$\@.  Two distinct
importations of a theory $S'$ within $S$ can be given distinct
interpretations.  A typical situation is when two theories $R_1$ and $R_2$
both import a theory $S$ as $S_1$ and $S_2$, respectively.  A theory $T$
importing both $R_1$ and $R_2$ might wish to identify $S_1$ and $S_2$
since, otherwise, these would be regarded as distinct within $T$\@.  This
can be done by importing an instance $S'$ of $S$ into $T$ and importing
$R_1$ with $S_1$ interpreted by $S'$ and $R_2$ with $S_2$ interpreted as
$S'$\@.  With theory interpretations, we have also extended parametric
theories in PVS to take theories as parameters.  For example, we might
have a theory \texttt{group\_homomorphism} of group homomorphisms that
takes two groups \texttt{G1} and \texttt{G2} as parameters as in the
declaration
\begin{alltt}
 group_homomorphism[G1, G2: THEORY group]: THEORY \ldots
\end{alltt}
The actual parameters for these theory formals must be
interpretations \texttt{G1'} and \texttt{G2'}
of the theory \texttt{group}\@.

Another typical requirement in a theory interpretation mechanism is
the ability to map a source type to some quotient with respect to
an equivalence relation over a target type.
For example, rational numbers can be interpreted by means of
a pair of integers corresponding to the numerator and denominator,
but the same rational number can have multiple such representations.
We show how it is possible to define quotient types in PVS and use
these types to capture interpretations where the equality over a
source type is mapped to an equivalence relation over a target type. 

The implementation of theory interpretation in PVS is described in the
following chapters.  This report assumes the reader is already familiar
with the PVS language; for details see the PVS Language
Manual~\cite{PVS:language}.  Chapter 2 deals with mappings, explaining the
basic concepts and introduces the grammar.  Chapter 3 introduces theory
declarations and theories as parameters which allow any valid
interpretation of the formal parameter theory as an actual parameter.
Chapter 4 describes a new command for viewing theory instances.  Chapter 5
compares PVS interpretations with other systems, Chapter 6 describes
future work, and we conclude with Chapter 7.


\chapter{Mappings}\label{mappings}

Theory interpretations in PVS provide mappings for uninterpreted types and
constants of the \emph{source} theory into the current
(\emph{interpreting}) theory.  Applying a mapping to a source theory
yields an \emph{interpretation} (or \emph{target}) theory.  A mapping is
specified by means of the \emph{mapping} construct, which associates
uninterpreted entities of the source theory with expressions of the target
theory.  The mapping construct is an extension to the PVS notion of
``name''.  The changes to the existing grammar are given in
Figure~\ref{mapping-bnf}.

\begin{figure}
\setlength{\sessionboxwidth}{\linewidth}
\addtolength{\sessionboxwidth}{-\arrayrulewidth}
\addtolength{\sessionboxwidth}{-\tabcolsep}
\begin{boxedminipage}[b]{\sessionboxwidth}
\begin{bnf}

\production{TheoryName}
{\opt{Id \lit{@}} Id \opt{Actuals} \opt{Mappings}}

\production{Name}
{\opt{Id \lit{@}} IdOp \opt{Actuals} \opt{Mappings} \opt{\lit{.} IdOp}}

\production{Mappings}
{\lit{\mapb{}} \ites{Mapping}{,} \lit{\mape{}}}

\production{Mapping}
{MappingLhs MappingRhs}

\production{MappingLhs}
{IdOp \rep{Bindings} \opt{\lit{:} \brc{\lit{TYPE} \choice \lit{THEORY}
\choice TypeExpr}}}

\production{MappingRhs}
{\lit{:=} \brc{Expr \choice TypeExpr}}

\end{bnf}
\end{boxedminipage}
\caption{Grammar for Names with Mappings}\label{mapping-bnf}
\end{figure}

The mapping construct defines the basic translation, but to be a theory
interpretation the mapping must be consistent: if type \texttt{T} is
mapped to the type expression \emph{E}, then a constant \texttt{t} of type
\texttt{T} must be mapped to an expression \emph{e} of type \emph{E}.  In
addition, all axioms and theorems of the source theory must be shown to
hold in the target theory under the mapping.  Since the theorems are
provable from the axioms, it is enough to show that the translation of the
axioms hold.  Axioms whose translations do not involve any
uninterpreted types or constants of the source theory are converted to
proof obligations.  Otherwise they remain axioms.

Theory interpretation may be viewed as an extension of theory
parameterization.  Given a theory named \texttt{T}, the instance
\texttt{T[a$_1$,\ldots,a$_n$]\mapb{}c$_1$:= e$_1$,\ldots,c$_m$:=
e$_m$\mape{}} is the same as the original theory, with the \emph{actuals}
\texttt{a$_i$} substituted for the corresponding formal parameters, and
e$_i$ substituted for \texttt{c$_i$}, which must be an uninterpreted type
or constant declaration.  Declarations that appear in the target of a
substitution in the mapping are not visible in the importing theory.  Some
axioms are translated to proof obligations.  The substituted forms of any
remaining axioms, definitions, and lemmas are available for use, and are
considered proved if they are proved in the uninterpreted theory.

The following simple example illustrates the
basic concepts.
\begin{session}
th1[T: TYPE, e: T]: THEORY
 BEGIN
  t: TYPE+
  c: t
  f: [t -> T]
  ax: AXIOM EXISTS (x, y: t): f(x) /= f(y)
  lem1: LEMMA EXISTS (x:T): x /= e
 END th1
\end{session}
\begin{session}
th2: THEORY
 BEGIN
  IMPORTING th1[int, 0]
               \mapb{} t := bool,
                  c := true,
                  f(x: bool) := IF x THEN 1 ELSE 0 ENDIF \mape{}
  lem2: LEMMA EXISTS (x:int): x /= 0
 END th2
\end{session}
\noindent Here theory \texttt{th1} has both actual parameters and
uninterpreted types and constants, as well as an axiom and
a lemma.  Theory \texttt{th2} imports \texttt{th1}, making the
following substitutions:
\setlength{\jot}{-2pt}
\setlength{\abovedisplayskip}{0pt}
\setlength{\belowdisplayskip}{0pt}
{\smaller\begin{eqnarray*}
\texttt{T} & \leftarrow & \texttt{int} \\
\texttt{e} & \leftarrow & \texttt{0} \\
\texttt{t} & \leftarrow & \texttt{bool} \\
\texttt{c} & \leftarrow & \texttt{true} \\
\texttt{f} & \leftarrow & \texttt{LAMBDA (x:\ bool):\ IF x THEN 1 ELSE 0 ENDIF} \\
\end{eqnarray*}}
Note that the mapping for \texttt{f} uses an abbreviated form of
substitution.  Typechecking this leads to the following proof obligation.
\begin{session}
IMP_th1_TCC1: OBLIGATION
  EXISTS (x, y: bool):
    IF x THEN 1 ELSE 0 ENDIF /= IF y THEN 1 ELSE 0 ENDIF;
\end{session}
This is simply the interpretation of the \texttt{ax} axiom and is easily
proved.  The lemma \texttt{lem1} can be proved from the axiom, and may
be used directly in proving \texttt{lem2} using the proof command
\texttt{(LEMMA "lem1")}.

Note that once the TCC has been proved, we know that \texttt{th1} is
consistent.  If we had left out the mapping for \texttt{f}, then the TCC
would not be generated, and the translation of theory \texttt{th1} would
still contain an axiom and not necessarily be consistent.

% Note that we used a lambda form in the axiom,
% rather than \texttt{f}.  This is because logically the generated proof
% obligation precedes the importing, which is only meaningful if the
% obligation is provable.  Hence \texttt{f} is not visible in the proof
% obligation and should not appear in any axiom of the theory.\footnote{We
% may allow this in future versions of PVS by automatically expanding
% non-recursive definitions as a part of substitution, treating them as a
% kind of macro.}  After the importing, of course, \texttt{f} is visible as
% seen in \texttt{lem2}.

% Note that mappings make theory parameters optional---they may be
% eliminated by moving the formal parameters to the theory body and turning
% assumptions into axioms.  The theory could then be instantiated using
% mappings instead of actual parameters.  Theory parameters are still
% useful, however.  First, they may be used to distinguish between the
% parameters to the system being specified and the entities defined by the
% system.  For example, in describing a protocol that works for any number
% of processes, it is more natural to make the number of processes a formal
% parameter rather than an uninterpreted constant.  Second, the
% typechecker can frequently infer the values of the actual parameters when
% a theory is imported generically, but mappings must be explicitly given.
% Although in principle the typechecker might be extended to infer mappings,
% it is hard to see how to do this efficiently.

One advantage to using mappings instead of parameters is that not all
uninterpreted entities need be mapped, whereas for parameters either all
or none must be given.  For example, consider the following theory.
\begin{session}
example1[T: TYPE, c: T]: THEORY
 BEGIN
  f(x: T): int = IF x = c THEN 0 ELSE 1 ENDIF
 END example1
\end{session}
\noindent It may be desirable to import this where \texttt{T} is always
\texttt{real}, and \texttt{c} is left as a parameter, but there is
currently no mechanism for this.  One could envision partial importings
such as \texttt{IMPORTING example1[real, \_]}, but it is not clear that
this is actually practical---in particular, the syntax for providing the
missing parameters is not obvious.  With mappings, on the other hand, we
can define \texttt{example1} as follows.
\begin{session}
example1: THEORY
 BEGIN
  T: TYPE
  c: T
  f(x: T): int = IF x = c THEN 0 ELSE 1 ENDIF
 END example1
\end{session}
\noindent Then we can refer to this theory from another theory as in the
following.
\begin{session}
example2: THEORY
 BEGIN
  th: THEORY = example1\mapb{}T := real\mape{}
  frm: FORMULA f\mapb{}c := 3\mape{} = f
 END example2
\end{session}
\noindent The \texttt{th} theory declaration just instantiates \texttt{T},
leaving \texttt{c} uninterpreted.  The first reference to \texttt{f} maps
\texttt{c} to \texttt{3}, whereas the second reference leaves it
uninterpreted though it is still a \texttt{real}.  Note that formula
\texttt{frm} is unprovable, since the uninterpreted \texttt{c} from the
second reference may or may not be equal to \texttt{3}.

As described in the introduction, an important aspect of mappings is the
support for quotient types.  In \textsc{Ehdm} this was done by
interpreting equality, but in PVS we instead define a theory of
equivalence classes, and allow the user to map constants to equivalence
classes under congruences.  For example, the \texttt{stacks} datatype
might be implemented using an array as follows.
\begin{session}
stack[t:TYPE]: DATATYPE
 BEGIN
  empty: empty?
  push(top:t, pop: stack): nonempty?
 END stack
\end{session}
\begin{session}\label{cstack}
cstack[t: TYPE+]: THEORY
 BEGIN
  cstack: TYPE = [# size: nat, elems: [nat -> t] #]
  cempty?(s: cstack): bool = (s`size = 0)
  some_t: t = epsilon(LAMBDA (x:t): true)
  cempty: (cempty?) =
    (# size := 0,
       elems := LAMBDA (n: nat): some_t #)
  cnonempty?(s: cstack): bool = (s`size /= 0)
  ctop(s: (cnonempty?)): t = s`elems(s`size - 1)
  cpop(s: (cnonempty?)): cstack = s WITH [`size := s`size - 1]
  cpush(x: t)(s: cstack): (cnonempty?) =
    (# size := s`size + 1,
       elems := s`elems WITH [(s`size) := x] #)
  ce: equivalence[cstack] =
    LAMBDA (s1, s2: cstack):
     s1`size = s2`size AND
     FORALL (n: below(s1`size)): s1`elems(n) = s2`elems(n)

  estack: TYPE = Quotient(ce)
  CONVERSION+ EquivClass(ce), rep(ce), lift(ce)
  \ldots
\end{session}
\texttt{Quotient}, \texttt{EquivClass}, and \texttt{rep} are defined in
the prelude theory \texttt{QuotientDefinition}, shown here in part.
\begin{session}
QuotientDefinition[T : TYPE] : THEORY
BEGIN
  R : VAR set[[T, T]]
  S : VAR equivalence[T]
  x, y, z : VAR T

  EquivClass(R)(x) : set[T] = { z | R(x, z) }
  \ldots
  Quotient(S) : TYPE =
    { P : set[T] | EXISTS x : P = EquivClass(S)(x) }
  \ldots
  rep(S)(P: Quotient(S)): T = choose(P)
  \ldots
END QuotientDefinition
\end{session}
The \texttt{lift} function is defined in the prelude theory
\texttt{QuotientExtensionProperties} as follows.
\begin{session}
QuotientExtensionProperties[X, Y : TYPE] : THEORY
BEGIN
  S : VAR equivalence[X]

  lift(S)(g : (PreservesEq[X, Y](S)))(P : Quotient(S)) : Y
    = g(rep(S)(P))
  \ldots
END QuotientExtensionProperties
\end{session}
This allows functions on concrete stacks to be lifted to functions on
equivalence classes, so long they satisfy the \texttt{PreservesEq}
relation, i.e., they produce the same values on \texttt{S}-equivalent
elements.

With these conversions in place, we can finish the specification of
\texttt{cstack} as follows.
\begin{session}
  \ldots
  IMPORTING stack[t]\mapb{} stack := estack,
                       empty? := cempty?,
                       nonempty? := cnonempty?,
                       empty := cempty,
                       top(s: (cnonempty?)) := ctop(s),
                       pop(s: (cnonempty?)) := cpop(s),
                       push(x: t, s: cstack) := cpush(x)(s) \mape{}
 END cstack
\end{session}
\noindent Here the source type \texttt{stack} is mapped to the quotient
type \texttt{estack} defined by the concrete equality \texttt{ce}.  The
\texttt{empty?} and \texttt{nonempty?} predicates are mapped to predicates
on \texttt{estack}s, using the \texttt{rep(ce)} conversion.  The
\texttt{empty} constructor is then mapped to its equivalence class.

\texttt{top}, \texttt{pop},

The mapping for \texttt{push} is more involved; \texttt{cpush} must first
be lifted in order to apply it to the abstract stack \texttt{s}.  This is
applied automatically by the conversion mechanism of PVS.  The application
of \texttt{lift} generates the proof obligation that \texttt{cpush}
preserves the equivalences, that is, it is a congruence.  This mapping
generates a large number of proof obligations, because the \texttt{stack}
datatype generates a \texttt{stacks\_adt} theory with a large number of
axioms, for example, extensionality, well-foundedness, and induction.

The PVS interpretations mechanism is much simpler to implement than the
one in \textsc{Ehdm}---equality is not a special case, but simply an
aspect of mapping a type to an equivalence class.  The technique of
mapping types to equivalence classes is quite useful, and captures the
notion of behavioral equivalence outlined
in~\cite{SannellaDT:essential-concepts97}.  In fact it is more general, in
that it works for any equivalence relation, not just those based on
observable sorts.


\chapter{Theory Declarations}

With the mapping mechanism, it is easy to specify a general theory and
have it stand for any number of instances.  For example, groups, rings,
and fields are all structures that can be given axiomatically in terms of
uninterpreted types and constants.  This works well when considering one
such structure at a time, but it is difficult to specify theories that
involve more than one structure, for example, group homomorphisms.
Importing the original theory twice is the same as importing it once, and
an attempted definition of a homomorphism would turn into an automorphism.
In this case what is needed is a way to specify multiple different
``copies'' of the original theory.  This is accomplished with \emph{theory
declarations}, which may appear in either the theory parameters or the
body of a theory.  A theory declaration in the formal parameters is
referred to as a \emph{theory as parameter}.\footnote{The term
\emph{theory parameter} refers to a parameter of a theory, so we use the
term \emph{theory as parameter} instead.}  Theory declarations allow
theories to be encapsulated, and instantiated copies of the implicitly
imported theory are generated.
\begin{figure}[!b]
\setlength{\sessionboxwidth}{\linewidth}
\addtolength{\sessionboxwidth}{-\arrayrulewidth}
\addtolength{\sessionboxwidth}{-\tabcolsep}
\begin{boxedminipage}[b]{\sessionboxwidth}
\begin{bnf}\smaller

\production{TheoryFormalDecl}
{TheoryFormalType \choice TheoryFormalConst \choice TheoryDecl}

\production{TheoryDecl}
{Id \lit{:} \lit{THEORY} TheoryDeclName}

\production{TheoryDeclName}
{\opt{Id \lit{@}} Id \opt{Actuals} \opt{TheoryDeclMappings}}

\production{TheoryDeclMappings}
{\lit{\mapb{}} \ites{TheoryDeclMapping}{,} \lit{\mape{}}}

\production{TheoryDeclMapping}
{MappingLhs TheoryDeclMappingRhs}

\production{TheoryDeclMappingRhs}
{MappingSubst \choice MappingDef \choice MappingRename}

\production{MappingSubst}
{\lit{:=} \brc{Expr \choice TypeExpr}}

\production{MappingDef}
{\lit{=} \brc{Expr \choice TypeExpr}}

\production{MappingRename}
{\lit{::=} \brc{IdOp \choice Number}}

\end{bnf}
\end{boxedminipage}
\caption{Grammar for Theory Declarations}\label{theory-parameter-bnf}
\end{figure}

For example, an (additive) group is normally thought of as a 4-tuple
consisting of a set $G$, a binary operator $+$, an identity element $0$,
and an inverse operator $-$ that satisfies the usual group axioms.  Using
theory interpretations, we simply define this as follows:
\begin{session}
group: THEORY
 BEGIN
  G: TYPE+
  +: [G, G -> G]
  0: G
  -: [G -> G]
  x, y, z: VAR G
  associative_ax: AXIOM FORALL x, y, z: x + (y + z) = (x + y) + z
  identity_ax: AXIOM FORALL x: x + 0 = x
  inverse_ax: AXIOM FORALL x: x + -x = 0 AND -x + x = 0
  idempotent_is_identity: LEMMA x + x = x => x = 0
 END group
\end{session}

As described in Chapter~\ref{mappings}, we can use mappings to create
specific instances of groups.  For example, {\smaller\begin{alltt}
group\mapb{}G := int, + := +, 0 := 0, - := -\mape{}
\end{alltt}}
\noindent is the additive group of integers, whereas
{\smaller\begin{alltt}
group\mapb{}G := nzreal, + := *, 0 := 1, - := LAMBDA (r:nzreal):\ 1/r\mape{}
\end{alltt}}
\noindent is the multiplicative group of nonzero reals.

This works nicely, until we try to define the notion of a group
homomorphism.  At this point we need two groups, both individually
instantiable.  We could simply duplicate the group specification, but
this is obviously inelegant and error prone.  Using theories as
parameters, we may define group homomorphisms as follows.
\begin{session}
group_homomorphism[G1, G2: THEORY group]: THEORY
 BEGIN
  x, y: VAR G1.G
  f: VAR [G1.G -> G2.G]
  homomorphism?(f): bool = FORALL x, y: f(x + y) = f(x) + f(y)
  hom_exists: LEMMA EXISTS f: homomorphism?(f)
 END group_homomorphism
\end{session}
\noindent Here \texttt{G1} and \texttt{G2} are theories as parameters to a
generic homomorphism theory that may be instantiated with two different
groups.  Hence we may import \texttt{group\_homomorphism}, for example, as
\begin{session}
IMPORTING group_homomorphism[group\mapb{}G := int, + := +, 0 := 0, - := -\mape{}
                             group\mapb{}G := nzreal, + := *, 0 := 1,
                                 - := LAMBDA (x: nzreal): 1/x\mape{}]
\end{session}

There is a subtlety here that needs emphasizing; \texttt{G1} and
\texttt{G2} are two \emph{distinct} versions of theory \texttt{group}.
For example, consider the addition of the following lemma to
\texttt{group\_homomorphism}.
\begin{session}
oops: LEMMA G1.0 = G2.0
\end{session}
\noindent If \texttt{G1} and \texttt{G2} are treated as the same
\texttt{group} theory, this is a provable lemma.  But then after the
importing given above we would be able to show that \texttt{0 = 1}.  Even
worse, the two different instances of groups may not even be type
compatible, so the \texttt{oops} lemma should not even typecheck.

We have solved this in PVS by making new theories \texttt{G1} and
\texttt{G2} that are copies of the original \texttt{group} theory.
Declarations within these copies are distinct from each other and from the
original.  Thus the \texttt{oops} lemma generates a type error, as
\texttt{G1.G} and \texttt{G2.G} are incompatible types.

This introduces new possibilities.  When creating copies of a theory the
mappings are substituted and the original declarations disappear.
However, it may be preferable to create definitions rather than
substitutions.  In addition, it is sometimes useful to simply rename the
types or constants of a theory.  For example, consider the following group
instance
\begin{session}
G1: THEORY = group\mapb{}G := int, + := +, 0 := 0, - := -\mape{}
\end{session}
\noindent which generates the following theory.
\label{group-instances-start}
\begin{session}
G1: THEORY
 BEGIN
  x, y, z: VAR int
  idempotent_is_identity: LEMMA x + x = x => x = 0
 END G1
\end{session}
To create definitions, use \texttt{=} instead of \texttt{:=}, as
in the following.
\begin{session}
G2: THEORY = group\mapb{}G = int, + = +, 0 = 0, - = -\mape{}
\end{session}
\noindent Now we get the following theory.
\begin{session}
G2: THEORY
 BEGIN
  G: TYPE+ = int
  +: [G, G -> G] = +
  0: G = 0
  -: [G -> G] = -
  x, y, z: VAR G
  idempotent_is_identity: LEMMA x + x = x => x = 0
 END G2
\end{session}
Finally, to simply rename the uninterpreted types and constants, use
\texttt{::=} as in the following.
\begin{session}
G3: THEORY = group\mapb{}G ::= MG, + ::= *, 0 ::= 1, - ::= inv\mape{}
\end{session}
\noindent The generated theory instance specifies multiplicative groups as
follows.
\begin{session}
G3: THEORY
 BEGIN
  MG: TYPE+
  *: [MG, MG -> MG]
  1: MG
  inv: [MG -> MG]
  x, y, z: VAR MG
  associative_ax: AXIOM FORALL x, y, z: x * (y * z) = (x * y) * z
  identity_ax: AXIOM FORALL x: x * 1 = x
  inverse_ax: AXIOM FORALL x: x * inv(x) = 1 AND inv(x) * x = 1
  idempotent_is_identity: LEMMA x * x = x => x = 1
 END G3
\end{session}
The right-hand side of a renaming mapping must be an identifier, operator,
or number, and must not create ambiguities within the generated theory.
Note that renamed declarations are still uninterpreted, and may themselves
be given interpretations, as in
\begin{session}
G3i: THEORY = G3\mapb{}MG := nzreal, * := *, 1 := 1,
                  inv := LAMBDA (r: nzreal): 1/r\mape{}
\end{session}

Finally, we can mix the different forms of mapping, to give a partial
mapping.
\begin{session}
G4: THEORY = group\mapb{}G = nzreal, + := *, 0 ::= one\mape{}
\end{session}
\noindent This generates the following theory instance.
\begin{session}
G4: THEORY
 BEGIN
  G: TYPE+ = nzreal;
  one: nzreal;
  -: [nzreal -> nzreal]
  x, y, z: VAR nzreal
  identity_ax: AXIOM FORALL (x: nzreal): x * one = x
  inverse_ax: AXIOM FORALL (x: nzreal):
                      x * -x = one AND -x * x = one
  idempotent_is_identity: LEMMA x * x = x => x = one
 END G4
\end{session}\label{group-instances-end}
Note that \texttt{associative\_ax} has disappeared---it has become a TCC
of the importing theory---whereas the other axioms are not so transformed
because they still reference uninterpreted types or constants.

With theories as parameters we have another situation in which mappings
are more convenient than theory parameters.  Many times the same set of
parameters is passed through an entire theory hierarchy.  If there are
assumings, then these must be copied.  For example, consider the
following theory.
\begin{session}
th[T: TYPE, a, b: T]: THEORY
 BEGIN
  ASSUMING
   A: ASSUMPTION a /= b
  ENDASSUMING
  ...
 END th
\end{session}
\noindent To import this theory, you simply provide a type and two
different elements of that type.  But suppose you wish to import this
theory from a theory that has the same parameters.  In this case the
assumption must also be copied, as there is otherwise no way to prove the
resulting obligation.  This can (and frequently does) lead to a tower of
theories, all with the same parameters and copies of the same assumptions,
as well as proofs of the same obligations.

There are ways around this, of course.  Most assumptions may be turned
into type constraints, as in the following.
\begin{session}
th[T: TYPE, a: T, b: \setb{}x: T | a /= x\sete{}]: THEORY
 ...
\end{session}
\noindent But this introduces an asymmetry in that \texttt{a} and
\texttt{b} now belong to different types, and the type predicate still
must be provided up the entire hierarchy.

Using a theory as a parameter, we may instead define \texttt{th} as
follows.
\begin{session}
th: THEORY
 BEGIN
  T: TYPE,
  a, b: T
  A: AXIOM a /= b
  ...
 END th
\end{session}
\noindent We then parameterize using this theory (which is implicitly
imported):
\begin{session}
th_1[t: THEORY th]: THEORY ...
\end{session}
\noindent We have encapsulated the uninterpreted types and constants into
a theory, and this is now represented as a single parameter.  Axiom
\texttt{A} is visible within theory \texttt{th\_1}, and no proof
obligations are generated since no mapping was given for \texttt{th}.  Now
we can continue defining new theories as follows.
\begin{session}
th_2[t: THEORY th]: THEORY IMPORTING th_1[t] ...
th_3[t: THEORY th]: THEORY IMPORTING th_2[t] ...
  \vdots
\end{session}
\noindent None of these generate proof obligations, as no mappings are
provided.

We may now instantiate \texttt{th\_n}, for example, with the following.
\begin{session}
IMPORTING th_n[th\mapb{}T := int, a := 0, b := 1\mape{}]
\end{session}
\noindent Now the substituted form of the axiom becomes a proof obligation
which, when proved, provides evidence that the theory \texttt{th} is
consistent.

% \chapter{Theory Declarations and Theory Abbreviations}

With the introduction of theories as parameters, it is natural to allow
theory declarations that may be mapped, in the same way that instances may
be provided for theories as parameters.  Thus the
\texttt{group\_homomorphism} may be rewritten as follows:
\begin{session}
group_homomorphism: THEORY
 BEGIN
  G1, G2: THEORY group
  x, y: VAR G1.G
  f: VAR [G1.G -> G2.G]
  homomorphism?(f): bool = FORALL x, y: f(x + y) = f(x) + f(y)
  hom_exists: LEMMA EXISTS f: homomorphism?(f)
 END group_homomorphism
\end{session}
\noindent Again, the choice between using theories as parameters or theory
declarations is really a question of taste, as they are largely
interchangeable.

As with theories as parameters, copies must be made for \texttt{G1} and
\texttt{G2}.  Note that this means that there is a difference between
theory abbreviations and theory declarations, as the former do not involve
any copying.  We decided to use the old form of theory abbreviation to
define theory declarations, and to extend the \texttt{IMPORTING} expressions to
allow abbreviations, as shown in Figure~\ref{importing-bnf}.  Thus instead of
\begin{session}
funset: THEORY = sets[[int -> int]]
\end{session}
\noindent which creates a copy of sets, use
\begin{session}
IMPORTING sets[[int -> int]] AS funset
\end{session}
\noindent which imports \texttt{sets[[int -> int]]} and abbreviates it as
\texttt{funset}.

\begin{figure}
\setlength{\sessionboxwidth}{\linewidth}
\addtolength{\sessionboxwidth}{-\arrayrulewidth}
\addtolength{\sessionboxwidth}{-\tabcolsep}
\begin{boxedminipage}[b]{\sessionboxwidth}
\begin{bnf}

\production{Importing}
{\lit{IMPORTING} \ites{ImportingItem}{,}}

\production{ImportingItem}
{TheoryName \opt{\lit{AS} Id}}

\end{bnf}
\end{boxedminipage}
\caption{Grammar for Importings}\label{importing-bnf}
\end{figure}

\chapter{Prettyprinting Theory Instances}

Mappings can get fairly complex, especially if actual parameters are
involved, and it may be desirable to see the specified theory instance
displayed with all the substitutions performed.  To support this, we have
provided a new PVS command: \texttt{prettyprint-theory-instance}
(\texttt{M-x ppti}).  This takes two arguments: a theory instance, which
in general is a theory name with actual parameters and/or mappings, and a
context theory, in which the theory instance may be typechecked.  The
simplest way to use this command is to put the cursor on the theory name
as it appears in a theory as parameter, theory declaration, or
importing---when the command is issued it then defaults to the theory
instance under the cursor and the current theory is the default
context theory.  For example, putting the cursor on
\texttt{group\_homomorphism} in the following and typing \texttt{M-x ppti}
followed by two carriage returns\footnote{The first uses the theory name
instance at the cursor, and the second uses the current theory as the
context.} generates a buffer named \texttt{group\_homomorphism.ppi}.
All instances of a given theory generate the same buffer name.
\begin{session}
IMPORTING group_homomorphism[\mapb{}G := int, + := +, 0 := 0, - := -\mape{}
                             \mapb{}G := nzreal, + := *, 0 := 1,
                               - := LAMBDA (x: nzreal): 1/x\mape{}]
\end{session}
\noindent This buffer has the following contents.
\begin{session}
% Theory instance for
  % group_homomorphism[groups\mapb{} G := int, + := +,
  %                             - := -, 0 := 0 \mape{},
  %                    groups\mapb{} G := nzreal, + := *,
  %                             - := (LAMBDA (x: nzreal): 1 / x),
  %                             0 := 1 \mape{}]
group_homomorphism_instance: THEORY
 BEGIN

  IMPORTING groups\mapb{} G := int, + := +, - := -, 0 := 0 \mape{}

  IMPORTING groups\mapb{} G := nzreal, + := *,
                     - := (LAMBDA (x: nzreal): 1 / x), 0 := 1 \mape{}

  x, y: VAR int

  f: VAR [int -> nzreal]

  homomorphism?(f): bool =
    FORALL (x: int), (y: int): f(x + y) = f(x) * f(y)

  hom_exists: LEMMA EXISTS (f: [int -> nzreal]): homomorphism?(f)
 END group_homomorphism_instance
\end{session}
The group instances shown on
pages~\pageref{group-instances-start}--\pageref{group-instances-end}
provide more examples of the output produced by
\texttt{prettyprint-theory-instance}.

\chapter{Comparison with Other Systems}

In this chapter we compare PVS theory interpretations to existing
programming and specification mechanisms of other systems.
The \textsc{Ehdm} system~\cite{EHDM:Language} has a notion of a mapping
module that maps a source module to a target module.  When a mapping
module is typechecked, a new module is automatically created that
represents the substitution of the interpretations for the body of the
source theory.  Equality is allowed to be mapped in \textsc{Ehdm}, in
which case it must be mapped to an equivalence relation.  In PVS, mappings
are provided as a syntactic component of names, and are essentially an
extension of theory parameters.  Equality is not treated specially, but is
handled by mapping a given type to a quotient type.

IMPS~\cite{Farmer:imps-cade,Farmer94} also supports theory
interpretations.  It is similar to \textsc{Ehdm} in that it has a special
\texttt{def-translation} form that takes a source theory, target
theory, sort association list, and constant association list, and generates a
theory translation.  Obligations may be generated that ensure that every
axiom of the source theory is a theorem of the target theory.  If these
are proved the translation is treated as an interpretation.  There is no
mechanism for mapping equality.  As with both PVS and \textsc{Ehdm},
defined sorts and constants of the source theory are automatically
translated.  A more detailed comparison between IMPS and an earlier
version of PVS appears in an unpublished report by
Kamm\"{u}ller~\cite{Kammuller:comparison}.

In Maude~\cite{Maude} and its precursor OBJ~\cite{OBJ:intro} it is
possible to
define \texttt{modules} that represent transition systems of a rewrite
theory whose states are equivalence classes of ground terms and whose
transitions are inference rules in \emph{rewriting logic}.  A given module
may import another module, either \texttt{protecting} it, which means that
the importing module adds no \emph{junk} or \emph{confusion}, or
\texttt{including} it, which imposes no such restrictions.  In addition to
modules, Maude has \emph{theories}, which are used to declare module
interfaces.  These may appear as module parameters, as in
$M[X_{1}::T_{1},\ldots,X_{n}::T_{n}]$, where the $X_{i}$ are \emph{labels}
and the $T_{i}$ are names of theories.  These theory parameters (source
theories) may be instantiated by target theories or modules using
\emph{views}, which indicate how each sort, function, class, and message
of the source theory is mapped to the target theory.  However, Maude
currently does not support the generation of proof obligations from source
theory axioms, so views are simply theory translations, not
interpretations.

The programming language Standard ML~\cite{ML-report} has a module
system where modules are given by \emph{structures} with a given
\emph{signature}, and parametric modules are \emph{functors} mapping
structures of a given signature to structures.  The PVS mechanism
of using theories as parameters resembles SML functors but for a
specification language rather than a programming language. 
Sannella and Tarlecki~\cite{SannellaDT:essential-concepts97} describe a
version of the ML module system in which there are \emph{specifications}
containing \emph{sorts}, \emph{operations}, and \emph{axioms}.  For
example, the signature of stacks is the following.
\begin{eqnarray*}
\emph{STACK} = & \textbf{sorts} & \emph{stack} \\
               & \textbf{opns} & \emph{empty} : \emph{stack} \\
               &               & \emph{push} : \texttt{int} \times \emph{stack} \rightarrow \emph{stack} \\
               &               & \emph{pop} :
                                 \emph{stack} \rightarrow \emph{stack} \\
               &               & \emph{top} :
                                 \emph{stack} \rightarrow \texttt{int} \\
               &               & \emph{is\_empty} :
                                 \emph{stack} \rightarrow \texttt{bool} \\
               & \textbf{axioms} & \emph{is\_empty}(\emph{empty}) =
                                   \texttt{true} \\
               &               &
               \forall\emph{s}:\emph{stack}.\forall\emph{n}:\texttt{int}.
                 \emph{is\_empty}(\emph{push}(\emph{n},\emph{s}))
                     = \texttt{false} \\
               &               &
               \forall\emph{s}:\emph{stack}.\forall\emph{n}:\texttt{int}.
                 \emph{top}(\emph{push}(\emph{n},\emph{s})) = \emph{n} \\
               &               &
               \forall\emph{s}:\emph{stack}.\forall\emph{n}:\texttt{int}.
                 \emph{pop}(\emph{push}(\emph{n},\emph{s})) = \emph{s} \\
\end{eqnarray*}
The following algebra is a \emph{realization} of the above specification
that corresponds to that of \texttt{cstack} on page~\pageref{cstack}.
{\smaller\begin{alltt}
  structure S2 : STACK =
      struct
          type stack = (int -> int) * int
          val empty = ((fn k => 0), 0)
          fun push (n, (f, i))
                = ((fn k => if k = i then n else f k), i+1)
          fun pop (f, i) = if i = 0 then (f, 0) else (f, i-1)
          fun top (f, i) = if i = 0 then 0 else f(i-1)
          fun is_empty (f, i) = (i=0)
\end{alltt}}
Note however, that the stacks \emph{empty} and
\emph{pop}(\emph{push}(\texttt{6},\emph{empty})) are not equal.  Thus they
distinguish the \emph{observable} sorts, in this case \texttt{int} and
\texttt{bool}, which are the only data directly visible to the user.  The
above two terms are not \emph{observable computations}, so it does not
matter that they are different.  In general, two different algebras are
\emph{behaviorally equivalent} if all observable computations yield the
same results. Note that choosing observable values based on sorts is a bit
coarse: for example, there may be two \texttt{int}-valued variables, one of
which is observable and one that represents an internal pointer.  Mapping
to equivalence classes is more general, as it is easy to capture
behavioral equivalence.

The induction theorem prover Nqthm~\cite{boyer-moore88,BoyerGoldschlag91}
has a feature called \texttt{FUNCTIONALLY-INSTANTIATE} that can be used to
derive an instance of a theorem by supplying an interpretation for some of
the function symbols used in defining the theorem.  The corresponding
instances of any axioms concerning these function symbols must be
discharged.  Such axioms can be introduced as conservative extensions as
definitions with the \texttt{DEFUN} declaration or through witnessed
constraints using the \texttt{CONSTRAIN} declaration, or they can be
introduced nonconservatively through an \texttt{ADD-AXIOM}
declaration.  While the functional instantiation mechanism is similar in
flavor to PVS theory interpretations, the underlying logic of Nqthm is a
fragment of first-order logic whose expressive power is more limited
than the higher-order logic of PVS.  In addition, Nqthm lacks types and
structuring mechanisms such as parametric theories.

The \specware{} language~\cite{SrinivasJullig95} employs theory
interpretations as a mechanism for the stepwise refinement of
specifications into executable code.  \specware{} has constructs for
composing specifications while identifying the common components, and for
compositionally refining specifications so that the refinement of a
specification can be composed from the refinement of its components.
Unlike PVS, \specware{} has the ability to incorporate multiple logics
and translate specifications between these logics.  A theory is an
independent unit of specification in PVS and hence there is no support for
composing theories from other theories.  However, the operations in
\specware{} can largely be simulated by means of theories and theory
interpretations in PVS.

In summary, theory interpretation has been a standard tool in
specification languages since the early work on HDM~\cite{HDM:Handbook}
and Clear~\cite{BURSTALL&GOGUEN}.  PVS implements theory interpretations
as a simple extension of the mechanism for importing parametric theories.
PVS theory interpretations subsume the corresponding capabilities
available in other specification frameworks.


\chapter{Future Work}

A number of interesting extensions may be contemplated for
the future.

\paragraph{Mapping of interpreted types and constants---}

There are two aspects: one is simply a convenience where, for
example, we might have a tuple type declaration \texttt{T: TYPE = [T1, T2,
T3]} and want to map it to \texttt{position: TYPE = [real, real, real]} by
simply giving the map \texttt{\mapb{}T := position\mape{}}.

The second aspect is where the mapping is between two different kinds, for
example mapping a record type to a function type.  This requires
determining the corresponding components as well as making explicit the
underlying axioms.  For example, record types satisfy extensionality, and
if they are mapped to a different type the implicit extensionality axiom
must be translated to a proof obligation.

\paragraph{Rewriting with congruences---}

In theory substitution, if a type is mapped to a quotient type then
equality over this type is mapped to equality over the quotient type.
If $T$ is an uninterpreted type, $\equiv$ an equivalence relation over
$T'$, and $T'/\equiv$ the quotient type, then \texttt{=[$T$]} is mapped to
\texttt{=[$T'/\equiv$]}, which is equivalent to $\equiv$.  An equational
formula thus still has the form of a rewrite.  However, to apply such a
rewrite one generally needs to do some lifting.  The following is a simple
example.
\begin{session}
th: THEORY
 BEGIN
  T: TYPE
  a, b: T
  f, g: [T -> T]
  \ldots \emph{Some axioms involving f, g, a, and b}
  lem: LEMMA f(a) = g(b)
 END th
th2: THEORY
 BEGIN
  ==(x, y: int): bool = divides(3, x - y)
  IMPORTING th\mapb{}T := E(==),
                a := equiv_class(==)(2),
                b := equiv_class(==)(1),
                f := LAMBDA (x: E(==)): equiv_class(rep(x) - 1),
                g := LAMBDA (x: E(==)): equiv_class(rep(x) - 2)\mape{}
  \ldots
 END th2
\end{session}
\noindent To rewrite with \texttt{lem}, \texttt{a} must first be lifted to
its equivalence class, then the rewrite is applied and the result is then
projected back using \texttt{rep}.  To do this requires some modification
to the rewriting mechanism of the prover.

\paragraph{Consistency Analysis---}

With a single independent theory such as groups, it is easy to generate a
mapping in which all axioms become proof obligations, and see directly
that the theory is consistent.  On the other hand, if many theories are
involved in which compositions of mappings are involved, this may become
quite difficult.  What is needed is a tool that analyzes a mapped theory
to see if it is consistent, and reports on any remaining axioms and
uninterpreted declarations.  This is similar in spirit to proof chain
analysis, but works at the theory level rather than for individual
formulas.

\paragraph{Semantics of Mappings---}

The semantics of theory interpretations needs to be formalized and added
to the PVS semantics report~\cite{PVS-Semantics:TR}.

\chapter{Conclusion}

Theory interpretations are used to embed an interpretation of an abstract
theory in a more concrete one.  In this way, they allow an abstract
development to be reused at the more concrete level.  Theory
interpretations can be used to refine a specification down to code.
Theory interpretations can also be used to demonstrate the consistency of
an axiomatic theory relative to another theory.

Parametric theories in PVS provide some but not all of the functionality
of theory interpretations.  In particular, they do not allow an abstract
theory to be imported with only a partial parameterization.  Theory
interpretations have been implemented in PVS version 3.0, which will be
released in mid-2001.  The current implementation allows the
interpretation of uninterpreted types and constants in a theory, as well
as theory declarations.  PVS has also been extended so that a theory may
appear as a formal parameter of another theory.  This allows related sets
of parameters to be packaged as a theory.  Quotient types have been
defined within PVS and used to admit interpretations of types where the
equality on a source type is treated as an equivalence relation on a
target type.

Theory interpretations have been implemented in PVS as an extension of the
theory parameter mechanism.  This way, theory interpretations are
an extension of an already familiar concept in PVS and can be used in
place of theory parameters where there is a need for greater
flexibility in the instantiation.  The proof obligations generated by
theory interpretations are similar to those for parametric theories
with assumptions.  

A number of extensions related to theory interpretations remain to be
implemented.  First, we plan to extend theory interpretations to the case
of interpreted types and constants.  This poses some challenges since
there are implicit operations and axioms associated with certain type
constructors.  Second, the rewriting mechanisms of the PVS prover need to
be extended to rewrite relative to a congruence.  This means that if we
are only interested in $f(a)$ up to some equivalence that is preserved by
$f$, then we could rewrite $a$ up to equivalence rather than equality.
Third, the PVS semantics have to be extended to incorporate theory
interpretations.  Finally, the PVS ground evaluator has to be extended to
handle theory interpretations.  Currently, the ground evaluator generates
code corresponding to a parametric theory and this code is reused with the
actual parameters used as arguments to the operations.  Theory
interpretations cannot be treated as arguments in this manner since there
is no fixed set of parameters; parameters can vary according to the
interpretation.  Also, non-executable operations can become executable as
a result of the interpretation.

In summary, we believe that theory interpretations are a significant
extension to the PVS specification language.  Our implementation of this
in PVS3.0 is simple yet powerful.  We expect theory interpretations to be
a widely used feature of PVS.

\newpage
\bibliographystyle{alpha}
\addcontentsline{toc}{chapter}{Bibliography}
\bibliography{/homes/rushby/jmr,/homes/owre/tex/sam,/homes/shankar/tex}
\end{document}

% Master File: language.tex
% Document Type: LaTeX

\chapter{Name Resolution}\label{names}\label{resolution}

Names in \pvs\ are used to denote theories, variables, constants, and
formulas.  New names are introduced by declarations.  The syntax of names
is given in Figure~\ref{bnf-names}.

\pvsbnf{bnf-names}{Name Syntax}

The simplest form of a name is an \emph{idop}, \ie\ an identifier or
operator symbol.  This is generally all that is needed, unless names are
overloaded.

The overloading of names, both from different theories and within a single
theory, is allowed as long as there is some way for the system to
distinguish references to them.  Names from different theories may be
distinguished by prefixing them with the theory name.  Within a theory,
all names of the same kind must be unique, except for expression kinds;
which need only be unique up to the signature.  This is because the
signature is enough to distinguish these declarations.  For example, if
{\tt <} is declared to have signature {\tt [bool,int -> bool]}, the system
will recognize from the context that {\tt TRUE < 3} contains a reference
to this declaration, whereas {\tt 2 < 3} does not.\footnote{Of course,
this assumes that \texttt{TRUE} has not itself been overloaded.}  If the
use of the name is not enough to distinguish, coercion may be used to
specify the signature directly (see page~\pageref{coercions}).  Theory
parameters must be unique across all kinds.

There are three possible forms for names (two for theory names, which
appear in {\tt IMPORTING}s, {\tt EXPORTING} {\tt WITH}s, and theory
declarations).  Given a theory named {\em theoryid\/}, with formal
parameters $f_1,\ldots,f_n$, that contains a declaration named {\em
id\/}, the following three forms may be used to reference the
declaration in a theory that imports {\em theoryid\/}:
\begin{itemize}
\item {\em theoryid\/}{\tt [$a_1,\ldots,a_n$]}.{\em id\/}

\item {\em id\/}{\tt [$a_1,\ldots,a_n$]}

\item {\em id}

\end{itemize}
where the $a_i$ are expressions or type expressions that are compatible
with the formal parameters as described in Section~\ref{parameters}.
These forms are listed in order of increasing ambiguity---that is, names
that are given with just an id are far more likely to produce an ambiguity
than those further up.  Note that even the top form may be
ambiguous, as {\em id\/} may be declared more than once in {\em theoryid\/}.
If this is the case, then either the context will disambiguate the name or
a type will have to be supplied in the form of a coercion expression, \eg\
{\tt {\em id\/}:~nat}.  This kind of ambiguity is allowed only for
constants (including functions and recursive functions) and variables.

Names are resolved based on the expected type and the number and types of
arguments to which the name is applied.  The expected type is generally
determined from the context of the name, for example in
\begin{pvsex}
  c1: int = c2
\end{pvsex}
\texttt{c2} has expected type \texttt{int}.  For most expressions, this is
straight-forward, but applications create special problems.  For example,
in
\begin{pvsex}
  f: FORMULA c1 = c2
\end{pvsex}
we know that the equality (which \emph{is} an application) has range type
\texttt{boolean} since it is a formula, but this gives no information
about the types of the arguments.  We will first describe the simpler
situation, and then explain how names used as operators of an application
are resolved.

In general, the typechecker works by first collecting possible types for
the expressions, and then chooses from among the possible types using the
expected type, which is determined from the context of the expression.
The expected type is used to resolve ambiguities, but otherwise does not
contribute to the type of an expression.  Thus if \texttt{2 + 3}
typechecks, and \texttt{+} has not been redeclared, then it has type
\texttt{real} regardless of its context.  However, for the purpose of
checking for TCCs, it may be treated as having a different type depending
on the expected type and the available judgements.


% Document Type: LaTeX
% Master File: language.tex

\chapter{Abstract Datatypes}\label{datatypes}\label{adts}

PVS provides a powerful mechanism for defining abstract datatypes.  This
mechanism is akin to, but more sophisticated than, the \emph{shell}
principle of the Boyer-Moore prover~\cite{Boyer-Moore79}).  A PVS datatype
is specified by providing a set of \emph{constructors} along with
associated \emph{accessors} and \emph{recognizers}.  When a datatype is
typechecked, a new theory is created that provides the axioms and
induction principles needed to ensure that the datatype is the initial
algebra defined by the constructors.

\pvsbnf{bnf-adts}{Datatype Syntax}

The syntax for PVS datatypes is given in Figure~\ref{bnf-adts}.  Datatypes
may appear at the \emph{top-level} as with theory declarations, or
\emph{in-line} as a declaration within a theory.\footnote{Enumeration
types are actually in-line datatypes---see Section~\ref{enum-types}.}
Typechecking a top-level datatype named \texttt{foo} causes the generation
of a new PVS file named \texttt{foo\_adt.pvs} containing up to three
theories as described below.  Typechecking an in-line datatype has the
effect of adding new declarations to the current theory, effectively
replacing the in-line datatype.  In-line datatypes are more restricted:
they may not have formal parameters or assuming parts, and they will not
generate the recursive combinators described below.  The declarations
generated for an in-line datatype may be viewed using the
\texttt{M-x~prettyprint-expanded} command (see the \emph{PVS System
Guide}~\cite{PVS:userguide}).

\section{A Datatype Example: \texttt{stack}}\label{stacks-adt}
An example of a datatype is \texttt{stack}:
\begin{session}
  stack[T: TYPE]: DATATYPE
   BEGIN
    empty: empty?
    push(top:T, pop:stack): nonempty?
   END stack
\end{session}
The \texttt{stack} datatype has two \emph{constructors}, \texttt{empty} and
\texttt{push}, that allow stack elements to be constructed.  For example,
the term \texttt{push(1, empty)} is an element of type \texttt{stack[int]}.
The \emph{recognizers} \texttt{empty?}\ and \texttt{nonempty?}\ are predicates
over the \texttt{stack} datatype that are true when their argument is
constructed using the corresponding constructor.  Given a \texttt{stack}
element that is known to be \texttt{nonempty?}, the \emph{accessors}
\texttt{top} and \texttt{pop} may be used to extract the first and second
arguments.

Typechecking the \texttt{stack} specification automatically creates a new
file \texttt{stack\_adt.pvs}, that contains the material found in
the next five figures.  This new file contains three theories:
\texttt{stack\_adt}, \texttt{stack\_adt\_map}, and
\texttt{stack\_adt\_reduce}.

\pvstheory{stack_adtA-alltt}{Theory \texttt{stack\_adt} (continues)}{stack_adtA-alltt}
\pvstheory{stack_adtB-alltt}{Theory \texttt{stack\_adt} (continues)}{stack_adtB-alltt}
\pvstheory{stack_adtC-alltt}{Theory \texttt{stack\_adt} (continues)}{stack_adtC-alltt}
\pvstheory{stack_adtD-alltt}{Theory \texttt{stack\_adt\_map}}{stack_adtD-alltt}
\pvstheory{stack_adtE-alltt}{Theory \texttt{stack\_adt\_reduce}}{stack_adtE-alltt}

The first theory \texttt{stack\_adt} is parametric in type \texttt{T}.
This is a specification of ``stacks of \texttt{T}'', where \texttt{T} may
be instantiated by any defined type when the stacks datatype is imported.
Thus ``stacks of integers'' as well as ``stacks of stacks of integers''
may be defined using this theory.  The first few lines of the theory
define the main type of stacks \texttt{stack}, the recognizers
\texttt{emptystack?} and \texttt{nonemptystack?}, the constructors
\texttt{empty} and \texttt{push}, and the accessors \texttt{top} and
\texttt{pop} are declared.

The \texttt{stack\_ord} function is defined, and an axiom provided for
it's definition.  This is provided instead of a disjointness axiom,
because the disjointness axiom becomes difficult to generate and use if
the number of constructors is large.  The disjointness comes from the fact
that the natural numbers are distinct.  The \texttt{ord} function is then
defined to return \texttt{0} on an empty stack and \texttt{1} on a
nonempty stack.  This is the same function as \texttt{stack\_ord}, but is
easier to use.

Then a series of axioms are given.  The
\texttt{stack\_empty\_extensionality} axiom states that there is only one
bottom element of the datatype: \texttt{empty}.
\texttt{stack\_push\_extensionality} states that any two stacks that have
the same \texttt{top} and \texttt{pop} (have the same components) are the
same.  The \texttt{stack\_push\_eta} axiom states that \texttt{pop}ping
and \texttt{push}ing the same element off and onto a stack results in a
stack identical to the original.  \texttt{stack\_top\_push} says that if
you \texttt{push} and element on a stack, you get that same element when
you \texttt{pop} it back off.  \texttt{stack\_pop\_push} says that pushing
something on a stack and then popping it back off results in the original
stack.

The \texttt{stack\_inclusive} axiom states that all stacks are either
\texttt{empty?} or \texttt{nonempty?}.  The PVS prover builds this axiom
in, so that it rarely needs be cited by a user.

\newpage
The next axiom, \texttt{stack\_induction}, introduces an induction formula
for stacks stating that any predicate $p$ of stacks that
\begin{enumerate}
\item holds for the empty stack (the base case), and
\item if $p$ holds for some stack then $p$ holds for the result of
\texttt{push}ing anything of the right type onto that stack (the induction
step),
\end{enumerate}
then $p$ holds for all stacks.

Then some useful functions are defined over stacks.  The stack predicate
\texttt{every} takes as arguments a predicate over \texttt{T} and a stack
and returns \texttt{TRUE} iff all elements on the stack satisfy the given
predicate.  \texttt{every} is introduced in both curried and uncurried
forms.  The stack predicate \texttt{some} is dual to \texttt{every},
returning \texttt{TRUE} iff there is some element on the stack that
satisfies the predicate.  The \texttt{subterm} predicate takes two stacks
and returns \texttt{TRUE} if and only if the first argument stack is a
subterm of the second.  That is, if the second stack consists of the first
stack with some (perhaps zero) elements pushed onto it.  The \texttt{<<}
predicate is the strict (irreflexive) \texttt{subterm} predicate.  Thus
for all stacks $s$, \texttt{subterm}$(s,s)$ holds, but for no stack $s$
does \texttt{<<}$(s,s)$ hold.  An alternative equivalent definition of
\texttt{<<} is as follows:
\begin{pvsex}
  <<(x: stack, y: stack): boolean = subterm(x,y) AND NOT x = y
\end{pvsex}
However, this definition is more awkward to use in a proof, as the
recursion is hidden in the definition of \texttt{subterm}.  For this
reason the definitions for \texttt{every}, \texttt{some},
\texttt{subterm}, and \texttt{<<}, are each defined as standalone
functions, though some of them could be defined in terms of the others.

The last four declarations of the theory \texttt{stack\_adt} are functions
which reduce a stack to a natural number or to an ordinal.  These
functions are useful for simplifying the proof of termination of
user-defined functions over stacks.  Recall that PVS requires recursive
functions to include a \emph{measure}, which is used to generate
termination conditions.  The primary use of the recursive combinator is to
allow measure functions to be specified.  The function
\texttt{reduce\_nat} takes a natural number and a function.  The natural
number is used for the empty stack, and then for each element on the
stack, the input function is applied to the element from the stack and the
current reduced natural number, returning a natural number.  The function
\texttt{reduce\_nat} returns the final natural number.  The function
\texttt{REDUCE\_nat} is analogous to \texttt{reduce\_nat}, except that the
reducing function is also given the entire contents of the stack.  This
version of reduction can be useful for complicated measures that involve,
for example, the number of repeated elements appearing on the stack.  The
simpler form of reduce is difficult to apply to such situations.  The
functions \texttt{reduce\_ordinal} and \texttt{REDUCE\_ordinal} are
analogous to \texttt{reduce\_nat} and \texttt{REDUCE\_nat} except that
they return ordinal numbers instead of natural numbers.  It is rare that a
termination argument requires the use of ordinals, so the simpler
\texttt{reduce\_nat} form is more often used.  This completes the
description of the \texttt{stack\_adt} theory.

The second theory in the file \texttt{stack\_adt.pvs} is
\texttt{stack\_adt\_map}.  This theory takes two types \texttt{T} and
\texttt{T1} as parameters, imports the \texttt{stack\_adt} theory, and
defines a mapping from \texttt{stacks[T]} to \texttt{stacks[T1]}.  The
higher-order \texttt{map} function takes a function \texttt{f} of type
\texttt{[T -> T1]}, and a stack of \texttt{T}, and returns a stack of
\texttt{T1} obtained by applying \texttt{f} to each element on the input
stack.  \texttt{map} is defined in both curried and uncurried forms.
\texttt{map} couldn't reside in the \texttt{stack\_adt} theory because
that theory has only one type parameter, while the \texttt{map} functions
require two: In order to construct and access stacks in two theories,
\texttt{map} must be parameterized in the two types.

Also in the \texttt{stack\_adt\_map} is a relational \texttt{every}
function.  It lifts a relation \texttt{R} between \texttt{T} and \texttt{T1},
to stacks of \texttt{T} and \texttt{T1}.  It is true if the stacks are the
same size, and corresponding elements satisfy \texttt{R}.

The third and final theory generated from \texttt{stack\_pvs} is
\texttt{stack\_adt\_reduce}.  This theory provides a generalized version
of \texttt{reduce\_nat} and \texttt{REDUCE\_nat}.  It takes as parameters
a type \texttt{T} and a range type \texttt{range}.  It defines a
generalized \texttt{reduce} which reduces stacks of \texttt{T} to elements
of \texttt{range}.  The functions \texttt{reduce\_nat},
\texttt{REDUCE\_nat}, \texttt{reduce\_ordinal}, and
\texttt{REDUCE\_ordinal} could have been defined using
\texttt{stack\_adt\_reduce}, but the direct definitions are provided for
additional user convenience.  The generalized \texttt{reduce} can be used
to provide evidence of termination of user-defined functions, but the
predefined versions such as \texttt{reduce\_nat} are easier to use in most
cases.

\section{Datatype Details}

In general, a datatype declaration has the form
\begin{pvsex}
  adt: DATATYPE WITH SUBTYPES S\(\sb{1}\), \ldots, S\(\sb{n}\)
    BEGIN
     cons\(\sb1\)(acc\(\sb{11}\): T\(\sb{11}\), \ldots, acc\(\sb{1{n\sb1}}\): T\(\sb{1{n\sb1}}\)): rec\(\sb1\) : S\(\sb{i\sb{1}}\)
     \vdots
     cons\(\sb{m}\)(acc\(\sb{m1}\): T\(\sb{m1}\), \ldots, acc\(\sb{1n\sb{m}}\): T\(\sb{1n\sb{m}}\)): rec\(\sb{m}\) : S\(\sb{i\sb{m}}\)
    END adt
\end{pvsex}
%
where the \texttt{cons$_i$} are the
\emph{constructors}\index{constructor}\index{datatype!constructor}, the
\texttt{acc$_{ij}$} are the
\emph{accessors}\index{accessor}\index{datatype!accessor}, the
\texttt{T$_{ij}$} are type expressions, and the \texttt{rec$_i$} are
\emph{recognizers}\index{recognizer}\index{datatype!recognizer}.  Each
line is referred to as a \emph{constructor
specification}\index{constructor specification}\index{datatype!constructor
specification}.  There are a number of restrictions enforced on
constructor specifications:
\begin{itemize}

\item The datatype identifier may not be used for a recognizer,
accessor, or subtype:\newline
($\texttt{adt} \not\equiv \texttt{rec}_i$ for all $i$, $\texttt{adt}
\not\equiv \texttt{acc}_{ij}$ for all $i$ and $j$, and $\texttt{adt}
\not\equiv \texttt{S}_i$ for all $i$).

\item The subtype names must be unique:
($i \neq j \Rightarrow \texttt{S}_i \not\equiv \texttt{S}_j$)

\item Each subtype name must be used at least once.

\item The constructor names must be unique:
($i \neq j \Rightarrow \texttt{cons}_i \not\equiv \texttt{cons}_j$).

\item The recognizer names must be unique:
($i \neq j \Rightarrow \texttt{rec}_i \not\equiv \texttt{rec}_j$).

\item No identifier may be used as both a constructor and a recognizer:\newline
($\texttt{cons}_i \not\equiv \texttt{rec}_j$ forall $i$ and $j$).

\item Duplicate accessor identifiers are not allowed within a single
constructor specification:
($j \neq k \Rightarrow \texttt{acc}_{ij} \not\equiv \texttt{acc}_{ik}$).

\end{itemize}

As seen in the \texttt{stack} example, datatypes may be recursive; this is
the case when the type of one or more of the accessors reference the
datatype.  In PVS, all such occurrences must be positive, where a type
occurrence \texttt{T} is positive in a type expression $\tau$ iff either
\begin{itemize}
\item $\tau\equiv \texttt{T}$.

\item $\tau\equiv \{x:\tau'|p(x)\}$ and the occurrence \texttt{T} is
positive in $\tau'$.

\item $\tau\equiv [{\tau_1} \rightarrow {\tau_2}]$ and the occurrence
\texttt{T} is positive in $\tau_2$\@.  For example, \texttt{T} occurs
positively in \texttt{sequence[T]} where \texttt{sequence[T]} is defined
in the PVS prelude as the function type \texttt{[nat -> T]}\@.

\item $\tau \equiv [\tau_1,\ldots, \tau_n]$ and the occurrence \texttt{T}
is positive in some $\tau_i$.

\item $\tau\equiv [\#\ l_1 : \tau_1, \ldots, l_n : \tau_n\ \#]$ and the occurrence \texttt{T} is positive in some $\tau_i$\@. 

\item $\tau\equiv \mbox{\emph{datatype}}[\tau_1,\ldots, \tau_n]$, where
\emph{datatype} is a previously defined datatype and the occurrence
\texttt{T} is positive in $\tau_i$, where $\tau_i$ is a \emph{positive
parameter} of \emph{datatype}\@.
\end{itemize}

When a top-level datatype is given with formal type parameters, they are
checked for whether their occurrences are all positive; this is used as
described above for any datatype that imports this one, as well as
determining some of the declarations described below.

When a datatype is typechecked, a number of new declarations are
generated:
\begin{itemize}

\item The datatype identifier is used to create an uninterpreted type
declaration.  In general, the term \emph{datatype} refers to this type.

\item Each recognizer is used to declare an uninterpreted subtype of the
datatype.

\item Each subtype identifier is used to declare an interpreted type that
is the disjunction of the types given by the recognizers that reference
the subtype identifier in the constructor specification.

\item Each constructor and accessor is used to generate a constant
declaration.

\item An \texttt{\emph{id}\_ord} uninterpreted function is created, and an
axiom \texttt{\emph{id}\_ord\_defaxiom} defines its values.  This is
provided instead of a disjointness axiom, because the disjointness axiom
becomes difficult to generate and use when the number of constructors is
large.

\item An \texttt{ord} function is generated that gives a zero-based number
to each constructor (e.g., \texttt{ord(null) = 0} and \texttt{cons(1,null)
= 1}).  This is mostly useful for enumeration types.

\item An extensionality axiom is generated for each constructor
specification.

\item An eta axiom is generated for each constructor specification
that has accessors.

\item For each accessor an axiom is created that says that the accessor
composed with the corresponding constructor returns the correct value; \eg\
\begin{pvsex}
  acc\(\sb{ij}\)(cons\(\sb{i}\)(e\(\sb{i1}\),\ldots, e\(\sb{i{m\sb{i}}})\) = e\(\sb{ij}\)
\end{pvsex}

\item An inclusive axiom is generated that says that every element of
the datatype belongs to at least one recognizer subtype.  This axiom is
not actually needed in practice as the prover checks for this directly.

\item Two induction schemes are provided for proving properties of the
datatype.

\item If there is at least one constructor with accessors,\footnote{Note
that enumeration types have no accessors.}  and there are positive type
parameters to the datatype, then \texttt{every} and \texttt{some}
functions are defined that provide a predicate on the datatype in terms of
the positive types.

\item The \texttt{subterm} and \texttt{<<} (irreflexive subterm) functions
are defined, and an axiom is generated that states that \texttt{<<} is
well-founded.  This allows it to be used as an ordering relation in
recursive function definitions.

\item If there is at least one constructor with
accessors,\addtocounter{footnote}{-1}\footnotemark{} the
\texttt{reduce\_nat}, \texttt{REDUCE\_nat}, \texttt{reduce\_ordinal}, and
\texttt{REDUCE\_ordinal} recursion combinators are defined.  These provide
a means for defining notions like the size or depth of a datatype term.

Note that accessor subtypes involving the datatype are
``lifted''.  The following example shows why.
\begin{pvsex}
  dt: DATATYPE
   BEGIN
    c0: c0?
    c1(a1: \setb{}x: list[dt] | length(x) > 0\sete): c1?
    c2(a2: \setb{}x: list[dt] | every(c0?)(x)\sete): c2?
   END dt
\end{pvsex}
Consider the \texttt{reduc\_nat} function.  The signature for the lifted
mapping function for \texttt{c1} and \texttt{c2} are the same:
\texttt{[list[nat] -> nat]}.  It's obvious the mapping function for
\texttt{c2} function could have the signature \texttt{[\setb{}x: list[nat]
| length(x) > 0\sete{} -> nat]}, but there is no obvious way to map
\texttt{c2} without lifting it.  Since it is not trivial to determine
which predicates map nicely, we lift them all.  In the future we may
provide heuristics that refine this.

\item If some type parameter is positive a \texttt{map} function is
generated in a separate theory.  Every positive type parameter in the
datatype is associated with a pair of \texttt{map} parameters, which form
the domain and range of a corresponding function argument.  Given a set of
such functions and a term of the datatype, \texttt{map} returns a term
that has the same structure, but with the ``leaf'' elements replaced by
the function values.

\item A separate theory is generated for the \texttt{reduce} and
\texttt{REDUCE} functions.  These generalize the \texttt{reduce} functions
above to an arbitrary range type.

\end{itemize}

Note that in the stack example, the \texttt{stack} type is nonempty, since
\texttt{empty} is an element of \texttt{stack} even if the parameter type
\texttt{T} is instantiated with an empty type.  However, there is no
requirement that a datatype be nonempty, though if it is imported and a
constant is declared to be of that type, a TCC will be generated as
described on page~\pageref{emptytypes} in section~\ref{emptytypes}.

The \texttt{stack\_adt} theory is parameterized in the type \texttt{T},
and introduces the uninterpreted type \texttt{stack}.  Under normal
circumstances, this would imply no relation between, for example,
\texttt{stack[nat]} and \texttt{stack[int]}.  However, since every
occurrence of \texttt{T} in the accessor types is positive, we can infer
that \texttt{stack[nat]} is a subtype of \texttt{stack[int]}.  In general,
given a type $T$ and a subtype $S \equiv \setb{}x:T | p(x)\sete$, then
\texttt{stack[$S$]} is treated the same as $\setb{}s:
\texttt{stack[}T\texttt{]} | \texttt{every}(p)(s)\sete$.  When a datatype
has a mix of positive and nonpositive type parameters, the subtype
relation only holds for the positive ones.  For example, in the datatype
\begin{session}
  dt[T1, T2: TYPE, c: T1]: DATATYPE
   BEGIN
    c(a1: T1, a2: [T2 -> T1]): c?
   END dt
\end{session}
\texttt{T1} is positive and \texttt{T2} is not, so \texttt{dt[nat, nat,
0]} is a subtype of \texttt{dt[int, nat, 0]}, but is not a subtype of
\texttt{dt[nat, int, 0]}, nor is it a subtype of \texttt{dt[nat, nat, 1]}.

More complex datatypes lead to correspondingly more complex declarations;
for example, in the following contrived datatype
\begin{session}
  adt1[t1,t2: TYPE, c:t1]: DATATYPE
   BEGIN
    bottom: bottom?
    c1(a11:t1, a12: [t2 -> int]): c1?
    c2(a21:adt1, a22:[nat -> adt1], a23: list[adt1]): c2?
    c3(a31:[list[int] -> adt1],
       a32:[# a: adt1, b: [int -> adt1] #],
       a33:[adt1, [set[int] -> adt1]]) : c3?
   END adt1
\end{session}
the curried \texttt{every} is generated as follows:
\begin{session}
  every(p: PRED[t1])(a1: adt1):  boolean =
      CASES a1
        OF bottom: TRUE,
           c1(c11_var, c12_var): p(c11_var),
           c2(c21_var, c22_var, c23_var):
             every(p)(c21_var) AND
              every(every(p))(c22_var) AND every[adt1](every(p))(c23_var),
           c3(c31_var, c32_var, c33_var):
                  (FORALL (x1: list[int]): every(p)(c31_var(x1)))
              AND every(p)(a(c32_var))
              AND FORALL (x: int): every(p)(b(c32_var)(x))
              AND every(p)(c33_var`1)
              AND FORALL (x: set[int]): every(p)(c33_var`2(x))
        ENDCASES;
\end{session}
Note that this is only defined for predicates over \texttt{t1}, since
the occurrence of \texttt{t2} in the constructor specification for
\texttt{c2} is not positive.

As with record types, constructor selectors may be dependent.  Here is a
simple example.
\begin{session}
  depdt: DATATYPE
   BEGIN
    b: b?
    c(x: int, y: \setb{}z: int | z < x\sete): c?
   END depdt
\end{session}

\section{Datatype Subtypes}

The \texttt{WITH SUBTYPES} keyword introduces a set of subtype names.
These are useful, for example, in defining the nonterminals of a language.
For example, we might try to describe a simple typed lambda calculus:
\begin{eqnarray*}
T & ::= & B \;|\; T \rightarrow T \\
E & ::= & x \;|\; \lambda x:T.E \;|\; E(E)
\end{eqnarray*}
This is difficult to express using datatypes without subtypes, but is
reasonably straightforward with them:\footnote{\texttt{TYPE},
\texttt{LAMBDA}, and \texttt{VAR} are PVS keywords, so variants are used
here.}
\begin{session}
tlc: DATATYPE WITH SUBTYPES typ, expr
 BEGIN
 base_type(n:nat): base_type? : typ
 fun_type(dom, ran: typ): fun_type? : typ
 expr_var(n:nat): expr_var? : expr
 lambda_expr(lvar:(expr_var?), ltype: typ, lexpr: expr)
                            : lambda_expr? : expr
 application(fun, arg: expr): application? : expr
 END tlc
\end{session}
In addition to the usual generated declarations, this generates
\begin{session}
  typ((x: tlc)): boolean = base_type?(x) OR fun_type?(x);
  typ: TYPE = \setb{}x: tlc | base_type?(x) OR fun_type?(x)\sete
  expr((x: tlc)): boolean =
     expr_var?(x) OR lambda_expr?(x) OR application?(x);
  expr: TYPE =
     \setb{}x: tlc | expr_var?(x) OR lambda_expr?(x) OR application?(x)\sete
\end{session}
immediately after the declarations generated for the recognizers, so they
may be referenced in the accessor types.  Note that only a single
induction scheme is generated.  To induct over a particular subtype,
extend the property of interest to the entire datatype so that it returns
true for everything else.


\section{\texttt{CASES} Expressions}\label{cases-expressions}
\index{cases expressions}

The \texttt{CASES} expression uses a simple form of pattern-matching on
abstract datatypes.  Patterns are of the form $c(x_1,\ldots, x_n)$ where
$c$ is an $n$-ary constructor and $x_1,\ldots, x_n$ is a list of distinct
variables.  Patterns here are simple so that certain logical properties of
the expression are easy to check.  Patterns are not defined in the grammar
but in the type rules, since the notion of a variable or a constructor is
only defined in the type rules.

For example, if \texttt{x} is of type \texttt{stack}, the cases expression
\begin{pvsex}
  CASES x OF
    empty : FALSE,
    push(y, z) : even?(y) AND empty?(z)
  ENDCASES
\end{pvsex}
is \texttt{TRUE} if \texttt{x} is a singleton even integer, and otherwise is
false.  This expression can be translated into
\begin{pvsex}
  IF empty?(x)
     THEN FALSE
     ELSE LET (y, z) = (car(x), cdr(x))
           IN even?(y) AND empty?(z)
  ENDIF
\end{pvsex}

The \texttt{CASES} expression also allows an \texttt{ELSE} clause, which
comes last and covers all constructors not previously mentioned in a
pattern.  If the \texttt{ELSE} clause is missing, and not all constructors
have been mentioned, then a \emph{cases TCC}\index{cases
TCC}\index{TCC!cases} is generated which states that the expression is not
any of the missing elements.  For example, if the \texttt{x} above is
declared to be a subtype of \texttt{stack} in which \texttt{empty} is
excluded, then the \texttt{empty} case can safely be left out, and a \tcc\
will be generated that obligates the user to prove that \texttt{x} is not
\texttt{empty}.  There is a trade-off here between simpler specifications
and simpler verifications; if the \texttt{empty} case is left in, then
there is no obligation to prove, but the extra case clutters up the
specification, and can mislead the reader into thinking that the
\texttt{empty} case is possible.  In general, we feel that the
specification should be as perspicuous as possible, even if it means a
little more work behind the scenes.

%%% Local Variables: 
%%% mode: latex
%%% TeX-master: "language"
%%% End: 


\appendix
\chapter{The Grammar}\label{grammar}

The complete \pvs\ grammar is presented in this Appendix, along with a
discussion of the notation used in presenting the grammar.

The conventions used in the presentation of the syntax are as follows.
\index{syntax!conventions}

\begin{itemize}

\item Names in {\it italics\/} indicate syntactic classes and
metavariables ranging over syntactic classes.

\item The reserved words of the language are
      printed in \lit{tt font, UPPERCASE}.

\item An optional part {\it A\/} of a clause is enclosed in square brackets:
\opt{{\it A\/}}.

\item Alternatives in a syntax production are separated by a bar
(``\choice''); a list of alternatives that is embedded in the right-hand
side of a syntax production is enclosed in brackets, as in

\begin{bnf}
\production{ExportingName}
{IdOp \opt{\lit{:} \brc{TypeExpr \choice \lit{TYPE} \choice \lit{FORMULA}}}}
\end{bnf}


\item Iteration of a clause {\it B\/} one or more times is indicated by
enclosing it in brackets followed by a plus sign: \ite{{\it B\/}};
repetition zero or more times is indicated by an asterisk instead of the
plus sign: \rep{{\it B\/}}.

\item A double plus or double asterisk indicates a clause separator; for
example, \reps{{\it B\/}}{,} indicates zero or more repetitions of the
clause {\it B} separated by commas.

\item Other items printed in tt font on the right hand side of
      productions are literals.  Be careful to distinguish where BNF
symbols occur as literals, \eg\ the BNF brackets \brc{} versus the
literal brackets \lit{\{\}}.

\end{itemize}

\subsubsection*{Specification}
\par\noindent
\spvsbnf{bnf-theory}

\subsubsection*{Importings and Exportings}
\par\noindent
\spvsbnf{bnf-exporting}

\subsubsection*{Assumings}
\par\noindent
\spvsbnf{bnf-assuming}

\subsubsection*{Theory Part}
\par\noindent
\spvsbnf{bnf-theory-part}

\subsubsection*{Declarations}
\par\noindent
\spvsbnf{bnf-decls}

\subsubsection*{Type Expressions}
\par\noindent
\spvsbnf{bnf-type-expr}

\subsubsection*{Expressions}
\par\noindent
\spvsbnf{bnf-expr}

\subsubsection*{Expressions (continued)}
\par\noindent
\spvsbnf{bnf-expr-aux}

\subsubsection*{Names}
\par\noindent
\spvsbnf{bnf-names}

\subsubsection*{Identifiers}
\par\noindent
\spvsbnf{bnf-lexical}

\subsubsection*{Datatypes}
\par\noindent
\spvsbnf{bnf-adts}

%% Derived from John Rushby's prelude.tex, modified for NFSS2
%
% define variants of the \LaTeX macro that avoid using \sc
% for use in headings
%

% Define fonts that work in math or text mode
\def\dwimrm#1{\ifmmode\mathrm{#1}\else\textrm{#1}\fi}
\def\dwimsf#1{\ifmmode\mathsf{#1}\else\textsf{#1}\fi}
\def\dwimtt#1{\ifmmode\mathtt{#1}\else\texttt{#1}\fi}
\def\dwimbf#1{\ifmmode\mathbf{#1}\else\textbf{#1}\fi}
\def\dwimit#1{\ifmmode\mathit{#1}\else\textit{#1}\fi}
\def\dwimnormal#1{\ifmmode\mathnormal{#1}\else\textnormal{#1}\fi}

\def\BigLaTeX{{\rm L\kern-.36em\raise.3ex\hbox{\small\small A}\kern-.15em
    T\kern-.1667em\lower.7ex\hbox{E}\kern-.125emX}}
\def\BoldLaTeX{{\bf L\kern-.36em\raise.3ex\hbox{\small\small\bf A}\kern-.15em
    T\kern-.1667em\lower.7ex\hbox{E}\kern-.125emX}}
%\def\labelitemi{$\bullet$}
\def\labelitemii{$\circ$}
\def\labelitemiii{$\star$}
\def\labelitemiv{$\diamond$}
\newcommand{\tcc}{{\small\small TCC}}
\newcommand{\tccs}{\tcc s}
\newcommand{\emacs}{{Emacs}}
\newcommand{\Emacs}{\emacs}
\newcommand{\ehdm}{{E{\small\small HDM}}}
\newcommand{\Ehdm}{\ehdm}
\newcommand{\tm}{$^{\mbox{\tiny TM}}$}
\newcommand{\hozline}{{\noindent\rule{\textwidth}{0.4mm}}}

\newcommand{\allclear}%
  {\mbox{\boldmath$\stackrel{\raisebox{-.2ex}[0pt][0pt]%
              {$\textstyle\oslash$}}{\displaystyle\bot}$}}

\newenvironment{private}{}{}

\newenvironment{smalltt}{\begin{alltt}\small}{\end{alltt}}

\newlength{\hsbw}

\newenvironment{session}%
  {\begin{flushleft}
   \setlength{\hsbw}{\linewidth}
   \addtolength{\hsbw}{-\arrayrulewidth}
   \addtolength{\hsbw}{-\tabcolsep}
   \begin{tabular}{@{}|c@{}|@{}}\hline 
   \begin{minipage}[b]{\hsbw}
   \begingroup\small\mbox{ }\\[-1.8\baselineskip]\begin{alltt}}
  {\end{alltt}\endgroup\end{minipage}\\ \hline 
   \end{tabular}
   \end{flushleft}}

\newenvironment{smallsession}%
  {\begin{flushleft}
   \setlength{\hsbw}{\linewidth}
   \addtolength{\hsbw}{-\arrayrulewidth}
   \addtolength{\hsbw}{-\tabcolsep}
   \begin{tabular}{@{}|c@{}|@{}}\hline 
   \begin{minipage}[b]{\hsbw}
   \begingroup\footnotesize\mbox{ }\\[-1.8\baselineskip]\begin{alltt}}%
  {\end{alltt}\endgroup\end{minipage}\\ \hline 
   \end{tabular}
   \end{flushleft}}

\newenvironment{spec}%
  {\begin{flushleft}
   \setlength{\hsbw}{\textwidth}
   \addtolength{\hsbw}{-\arrayrulewidth}
   \addtolength{\hsbw}{-\tabcolsep}
   \begin{tabular}{@{}|c@{}|@{}}\hline 
   \begin{minipage}[b]{\hsbw}
   \begingroup\small\mbox{ }\\[-0.2\baselineskip]}%
  {\endgroup\end{minipage}\\ \hline 
   \end{tabular}
   \end{flushleft}}

\newcommand{\memo}[1]%
  {\mbox{}\par\vspace{0.25in}%
   \setlength{\hsbw}{\linewidth}\addtolength{\hsbw}{-1.5ex}%
   \noindent\fbox{\parbox{\hsbw}{{\bf Memo: }#1}}\vspace{0.25in}}

\newcommand{\nb}[1]%
  {\mbox{}\par\vspace{0.25in}%
   \setlength{\hsbw}{\linewidth}\addtolength{\hsbw}{-1.5ex}%
   \noindent\fbox{\parbox{\hsbw}{{\bf Note: }#1}}\vspace{0.25in}}

\newcommand{\comment}[1]{}
\newcommand{\exfootnote}[1]{}
%\newcommand{\ifelse}[2]{#1}
\sloppy
\clubpenalty=100000
\widowpenalty=100000
%\displaywidowpenalty=100000
\setcounter{secnumdepth}{3} 
\setcounter{tocdepth}{3}
\setcounter{topnumber}{9}
\setcounter{bottomnumber}{9}
\setcounter{totalnumber}{9}
\renewcommand{\topfraction}{.99}
\renewcommand{\bottomfraction}{.99}
\renewcommand{\floatpagefraction}{.01}
\renewcommand{\textfraction}{.2}
\font\largett=cmtt10 scaled\magstep1
\font\Largett=cmtt10 scaled\magstep2
\font\hugett=cmtt10 scaled\magstep3


%\addcontentsline{toc}{chapter}{Bibliography}
\bibliographystyle{plain}
\bibliography{../pvs}

%\addcontentsline{toc}{chapter}{Index}   %% Put entry in T-O-C
%%\printindex  %% printindex makes extra call to "theindex"
{\smaller
\printindex
%% Document Type: LaTeX
% Master File: language.tex
\documentclass[12pt]{book}
\usepackage{alltt}
\usepackage{makeidx}
\usepackage{relsize}
\usepackage{boxedminipage}
\usepackage{url}
\usepackage{../../pvs}
\usepackage{../makebnf}
\usepackage[chapter]{tocbibind}
\usepackage{fancyvrb}
\usepackage[dvipsnames,usenames]{color}

\usepackage{amssymb}
\usepackage{mathpazo}
\usepackage{fontspec}
\setmainfont[Ligatures=TeX]{XITS}
\setmonofont{DejaVu Sans Mono}[Scale=MatchLowercase]
%\setmonofont{Free Mono}[Scale=0.8]
\usepackage[math-style=ISO]{unicode-math}
\renewcommand{\leadsto}{\rightsquigarrow}
%\setmathfont{XITS Math}

\topmargin -10pt
\textheight 8.5in
\textwidth 6.0in
\headheight 15 pt
\columnwidth \textwidth
\oddsidemargin 0.5in
\evensidemargin 0.5in   % fool system for page 0
\setcounter{topnumber}{9}
\renewcommand{\topfraction}{.99}
\setcounter{bottomnumber}{9}
\renewcommand{\bottomfraction}{.99}
\setcounter{totalnumber}{10}
\renewcommand{\textfraction}{.5}
\renewcommand{\floatpagefraction}{.1}
\usepackage{fancyhdr}
\pagestyle{fancy}
\raggedbottom

%\setcounter{secnumdepth}{1}

\index{type correctness condition|see{TCC}}
\makeindex

\usepackage[bookmarks=true,hyperindex=true,colorlinks=true,linkcolor=Brown,citecolor=blue,backref=page,pagebackref=true,plainpages=false,pdfpagelabels]{hyperref}

\input{../pvstex}

\begin{document}

\begin{titlepage}
\renewcommand{\thepage}{title}
\vspace*{1in}
\noindent
\rule[1pt]{\textwidth}{2pt}
\begin{center}
\newfont{\pvstitle}{cmss17 scaled \magstep4}
\textbf{\pvstitle PVS Language Reference}
\end{center}
\begin{flushright}
{\Large Version 7.1 {\smaller$\bullet$} August 2020}
\end{flushright}
\rule[1in]{\textwidth}{2pt}
\vspace*{2in}
\begin{flushleft}
S.~Owre\\
N.~Shankar\\
J.~M.~Rushby\\
D.~W.~J.~Stringer-Calvert\\
{\smaller\url{{Owre,Shankar,Rushby,Dave_SC}@csl.sri.com}}\\
{\smaller\url{http://pvs.csl.sri.com/}}
\end{flushleft}
\vspace*{1in}
\vbox{\hbox to \textwidth{{\Large SRI International\hfill}}%
\hbox to \textwidth{{\small\sf%
Computer Science Laboratory $\bullet$ 333 Ravenswood Avenue $\bullet$ Menlo Park CA 94025\hfil}}}
\end{titlepage}

\renewcommand{\chaptermark}[1]{\markboth{{\em #1}}{}\markright{{\em #1}}}
\renewcommand{\sectionmark}[1]{\markright{\thesection \em \ #1}}
%\lhead[\thepage]{\rightmark}
%\cfoot{\protect\small\bf \fbox{PVS 2.3 DRAFT}}
%\cfoot{}
%\rhead[\leftmark]{\thepage}
\thispagestyle{empty}

\newpage
\renewcommand{\thepage}{ack}

\noindent\textbf{NOTE:} This manual is in the process of being updated.
Almost everything stated here is still correct, but incomplete due to the
many new features that have been introduced into PVS over the years.  The
release notes should be consulted for the most current information.

\vspace*{6in}\noindent
The initial development of PVS was funded by SRI International.
Subsequent enhancements were partially funded by SRI and by NASA
Contracts NAS1-18969 and NAS1-20334, NRL Contract N00014-96-C-2106,
NSF Grants CCR-9300044, CCR-9509931, and CCR-9712383, AFOSR contract
F49620-95-C0044, and DARPA Orders E276, A721, D431, D855, and E301.
\newpage
\pagenumbering{roman}
\setcounter{page}{1}

\tableofcontents
%\listoffigures

%\chapter{The PVS Specification Language}

%\include{preface}
\cleardoublepage
\pagenumbering{arabic}
\setcounter{page}{1}

\setcounter{topnumber}{9}
\renewcommand{\topfraction}{.99}
\setcounter{bottomnumber}{9}
\renewcommand{\bottomfraction}{.99}
\setcounter{totalnumber}{10}
\renewcommand{\textfraction}{.01}
\renewcommand{\floatpagefraction}{.01}

\include{intro}
\include{lexical}
\include{declarations}
\include{types}
\include{expressions}
\include{theories}
\include{interpretations}
\include{names}
\include{adts}

\appendix
\include{grammar}
%\include{prelude}

%\addcontentsline{toc}{chapter}{Bibliography}
\bibliographystyle{plain}
\bibliography{../pvs}

%\addcontentsline{toc}{chapter}{Index}   %% Put entry in T-O-C
%%\printindex  %% printindex makes extra call to "theindex"
{\smaller
\printindex
%\input{language.ind}
}

\end{document}

%%% Local Variables: 
%%% TeX-command-default: "Make"
%%% mode: latex
%%% TeX-master: "language"
%%% End: 

}

\end{document}

%%% Local Variables: 
%%% TeX-command-default: "Make"
%%% mode: latex
%%% TeX-master: "language"
%%% End: 

}

\end{document}

%%% Local Variables: 
%%% TeX-command-default: "Make"
%%% mode: latex
%%% TeX-master: "language"
%%% End: 


\section{The PVS Proof Checker}

The PVS Proof Checker is also referred to as an interactive theorem
prover.  It is much more automated than the low-level ``proof
editors'' that support some specification notations, but operates
under the user's direction and is therefore more controllable than
purely automatic theorem provers.


\subsection{Introduction}

Just as we execute programs to check if they return the desired result,
we subject high-level functional descriptions of a system to challenges
by demanding proofs of desirable properties.  We call such challenges
{\em putative theorems}.  Here are some simple examples:
\begin{itemize}
  \item If a function that reverses a list has been
correctly specified, then we should be able to prove that we get the
original list by reversing a list twice.

  \item When a train is allowed into a railroad crossing the gates must
be down.  

  \item If the operational semantics is correct, then it should conform
to the denotational semantics.
\end{itemize}
%
The point of these challenges is that the process of proving putative
theorems quickly highlights the gaps, errors, and inadequacies in the
functional description.  In some cases, such a proof could also prove
that the system as described meets its complete specification, in other
words, that it is {\em correct\/}.  But it is the rare proof that
succeeds.  The typical proof attempt fails, but such failure usually
yields valuable insights that can be used to correct oversights in the
specification or formulation of the challenge.  A useful automated proof
assistant must therefore play the role of an intelligent but implacable
skeptic in rejecting any argument that is not entirely watertight.
Furthermore, in rejecting these arguments, such a skeptic must pinpoint
the source of the failure so that the argument can be corrected and the
dialogue resumed.  The PVS proof checker is intended to serve as the
skeptical party in such a dialogue.  The user supplies the steps in the
argument and PVS applies them to the goal of the proof progressively
breaking them into simpler subgoals or to obvious truths or falsehoods.
If all of the subgoals are reduced to obvious truths, the proof attempt
has succeeded.  Otherwise, the proof attempts fails either because the
argument or the conjecture is incorrect.

The central design assumptions in PVS are therefore that
\begin{itemize}
  \item The purpose of an automated proof checker is not merely to prove
theorems but also to provide useful feedback from failed and partial proofs
by serving as a rigorous skeptic.
  \item The most straightforward mechanizable criterion for a rigorous
argument is that of a formal proof.
  \item Automation serves to minimize the tedious aspects of
formal reasoning  while maintaining a high level of accuracy in the
book-keeping and formal manipulations.
  \item Automation should also be used to capture repetitive patterns
of argumentation.
  \item The end product of a proof attempt should be a proof that, with
only a small amount of work, can be made humanly readable so that it can
be subjected to the {\em social process\/} of mathematical scrutiny.
\end{itemize}
In following these design assumptions, the PVS proof checker 
is more automated than a low-level proof checker such as
AUTOMATH~\cite{deBruijn80}, LCF~\cite{LCF}, Nuprl~\cite{Constable86},
Coq~\cite{Coquand-Huet85}, and HOL~\cite{Gordon:HOL}, but provides more
user control over the structure of the proof than highly
automated systems such as Nqthm~\cite{Boyer-Moore79,boyer-moore88} and
Otter~\cite{Otter90}.  We feel that the low-level systems over-emphasize
the formal correctness of proofs at the expense of their cogency, and
the highly automated systems emphasize theorems at the expense of their
proofs.

What is unusual about PVS is the extent to which aspects of the
language, the typechecker, and proof checker are intertwined.  The
typechecker invokes the proof checker in order to discharge proof
obligations that arise from typechecking expressions involving predicate
subtypes or dependent types.  The proof checker also makes heavy use of
the typechecker to ensure that all expressions involved in a proof are
well-typed.  This use of the typechecker can also generate proof
obligations that are either discharged automatically or are presented as
additional subgoals.  Several aspects of the language, particularly the
type system, are built into the proof checker.  These include the
automatic use of type constraints by the decision procedures, the
simplifications given by the abstract datatype axioms, and forms of
beta-reduction and extensionality, Another less unusual aspect of PVS is
the extent to which decision procedures involving equalities and linear
arithmetic inequalities are employed.\footnote{The Ontic
system~\cite{mcallester89} is a proof checker where decision procedures
are ubiquitously used.  Nqthm~\cite{Boyer-Moore79,boyer-moore88},
Eves~\cite{Pase-Saaltink}, and IMPS~\cite{IMPS91} also rely heavily on
the use of decision procedures.} The most direct consequence of this is
that the trivial, obvious, or tedious parts of the proof are often
entirely hidden from the displayed proof so that the user can focus on
the intellectually demanding parts of the proof, and the resulting proof
is also easier to read.

As with much else in PVS, the implementation philosophy of the proof
checker has been guided by the 80-20 rule, namely that 80\% of the
functionality of a nearly perfect system can be built with 20\% of the
effort, and the remaining 20\% of the functionality can take up the
remaining 80\% of the effort.  PVS attempts to provide much of the 80\%
of the functionality that is easily implemented.  Each PVS proof
commands performs the function that, in our experience, is typically
required of it.  To some reasonable extent, the less typical
functionality can be obtained by providing optional arguments to these
proof commands.  In atypical instances, the burden of carrying out some
manipulation falls squarely on the user.  Even in these instances, it is
not too tedious to achieve one's ends with the existing proof commands
in some fairly simple ways.  The reader should let us know if any of our
design decisions are found to be ill-considered.

In order to learn how to use the PVS proof checker, one must first
understand the sequent representation used by PVS to represent proof
goals, the commands used to move around and undo parts of the proof
tree, and the commands used to get help.  One must then understand the
syntax and effects of proof commands used to build proofs.  Many of these
commands are extremely powerful even in their simplest usage.  Several
of these commands can be more carefully directed by supplying them with
one or more optional arguments.  The advanced user will also need to
understand how to define proof strategies that capture repetitive
patterns of proof commands, and commands used for displaying, editing,
and replaying proofs.

Section~\ref{prelims} provides the basic information needed to get
started with the PVS proof checker.  The remaining sections give a
collection of typical examples of how the proof checker is used.
The PVS Proof Checker Reference
Manual~\cite{PVS:prover} contains detailed descriptions of the PVS
proof commands.

\subsection{Preliminaries}\label{prelims}

\paragraph{Sequent Representation of Proof Goals.}
Each goal or subgoal in a PVS proof attempt is a sequent
of the form $\Gamma\vdash\Delta$, where $\Gamma$ is a sequence of
{\em antecedent\/} formulas and $\Delta$ is a 
sequence of {\em consequent\/} formulas.  The actual displayed form of a
PVS sequent is
\begin{center}
\begin{tabular}{ll}
  {\tt \{-1\}} & $A_1$\\
  {\tt \{-2\}} & $A_2$\\
  {\tt [-3]} & $A_3$\\
 & \vdots\\
 \multicolumn{2}{l}{\tt |-------}\\
  {\tt [1]} & $B_1$\\
  {\tt \{2\}} & $B_2$\\
  {\tt \{3\}} & $B_3$\\
 & \vdots
\end{tabular}
\end{center}
where each $A_i$ is an antecedent formula and each $B_i$ is a consequent
formula.  The intuitive reading of such a sequent is as the formula
$$(A_1\wedge A_2 \wedge A_3 \wedge \ldots ) \supset (B_1 \vee B_2 \vee
B_3 \vee \ldots).$$ Note that the antecedent formulas are numbered with
negative integers and the consequent formulas with positive integers.
These numberings are used in directing the PVS proof commands. If a
formula number $n$ appears as {\tt [$n$]} in the sequent, it is an
indication that the formula was unaffected by the proof step that
created the sequent.  It is a good heuristic is to examine the new
formulas (\ie\ the formulas whose number appears as {\tt \{$n$\}}) in
the sequent to formulate the next proof step.


\paragraph{Starting and Quitting Proofs. }

As indicated earlier, the PVS Emacs command {\tt M-x pr} initiates a
proof with the cursor on the formula to be proved.   This brings up the
{\tt *pvs*} buffer with the goal sequent and a {\tt Rule?}\ prompt.
Typing the PVS Emacs command {\tt M-x help-pvs-prover} brings up
help on the prover commands.
To quit out of an existing proof attempt, type {\tt q} or {\tt
quit} at the {\tt Rule?}\ prompt.   You will be asked whether you wish to
save the partial proof.  Remember that if you answer {\tt yes}, the old
proof will be overwritten, and if you answer {\tt no}, you will lose the
partial proof that you have developed up to this point.


\comment{The main steps in initiating a proof are to: 
\begin{enumerate}
  \item Set up a working directory for each specification.
  \item Copy any {\tt .pvs} files you need into this directory.
  \item Start \pvs\ by invoking ``{\tt pvs}'' (provided you have the
relevant pvs directory in your path.
  \item To exit \pvs, type {\tt C-x C-c} in Emacs.
  \item {\tt M-x cc} to change the context to your working directory.
  \item {\tt M-x nf} starts a new pvs file called {\tt \bkt{\em
name}.pvs}, where \bkt{\em name} is your response to the prompt.
  \item {\tt M-x ps} parses a buffer, {\tt M-x tc} typechecks a buffer,
and {\tt M-x tcp} typechecks a buffer with proof attempts on the \tccs.
Each PVS command does 
the parsing and typechecking of the buffer as needed.  {\tt M-x ppe} shows the
expanded and pretty printed version of the buffer with all the \tccs\ and
datatype axioms.
  \item {M-x pr} on a declaration starts a proof attempt on the declaration.
If the formula has been proved, you are asked whether you wish to
continue the proof attempt.  If there is an existing proof, you are
queried if you wish to rerun that proof.
  \item Proof commands are typed in response to the ``{\tt Rule?}'' prompt.
  \item Typing {\tt (help)} prints a brief line describing each of the
proof commands.  Typing {\tt (help {\em command-name})} prints the
information for a specific proof command.
\end{enumerate}
}


Since PVS proof construction is carried out in a Lisp buffer, there is a
small chance that you could find yourself at a Lisp breakpoint with a
`{\tt ->}' prompt.  Typing {\tt (restore)} at this point should almost
always  take you back to the nearest sensible proof goal and a  {\tt
Rule?}\ prompt.   


\paragraph{The Structure of PVS Proofs. }
In the course of a proof, PVS builds up a tree of sequents where
each sequent is a subgoal generated from its parent sequent by a PVS
proof command.  At any point in a proof attempt, the control is at a
leaf sequent of such a proof tree.  At this point a PVS proof command
can either 
\begin{itemize}
  \item  cause control to be transferred to next proof sequent in the
tree ({\tt postpone})
  \item undo a subtree by causing control to move up to some ancestor node
in the proof tree ({\tt undo})
  \item prove the {\em current sequent\/} causing control to move to the
next remaining leaf sequent in the tree
  \item generate subgoals so that control moves to the first of these
subgoals, or
  \item leave the proof tree unchanged while providing some useful status information.
\end{itemize}
A proof is completed when there are no remaining unproved leaf sequents
in the proof tree.  The resulting proof is saved and can be edited and
rerun on the same or a different conjecture.


\subsection{Using the Proof Checker}\label{using}

\subsubsection{Propositional Proof Commands}

Now that we have gotten past the preliminaries, we can look at
examples of some simple interactions with the PVS proof checker.
We start with the following PVS theory named {\tt propositions}
that declares three Boolean constants {\tt A}, {\tt B}, and {\tt C}, and
states a theorem named {\tt prop} asserting that the conjunction of
$( \mbox{\tt A} \supset ( \mbox{\tt B} \supset \mbox{\tt C}))$ and $(
\mbox{\tt A} \supset \mbox{\tt B})$ and $ \mbox{\tt A}$ implies $
\mbox{\tt C}$.

\begin{pvsscript}
propositions : THEORY
  BEGIN

  A, B, C: bool

  prop: THEOREM (A IMPLIES (B IMPLIES C)) AND (A IMPLIES B) AND A
               IMPLIES C

  END propositions
\end{pvsscript}

The proof script displayed below is the result of typing the PVS Emacs
command {\tt M-x pr} on the formula {\tt prop} and typing the inputs
(shown in bold-face) in response to the {\tt Rule?}\ prompt or to other
queries from PVS.  The {\tt (flatten)} command eliminates the
disjunctive connectives in the formula so as to flatten the formula out
into the sequent.  The next proof command {\tt (split)} picks the first
available conjunctive formula, in this case {\tt (A IMPLIES (B IMPLIES
C))}, and generates the three subgoals resulting from the conjunctive
splitting of this formula.  PVS then observes that the first of these
subgoals is trivially true since it has {\tt C} in both the antecedent
and consequent.  The {\tt (split)} command applied to the second subgoal
generates two further subgoals which are both recognized as being
trivially true, as is the remaining subgoal from the earlier {\tt
(split)} command.  The proof has now been successfully completed
generating the {\tt Q.E.D.} message, and  the new proof is automatically
saved.  The system inquires whether the user would like to see an
abbreviated version of the proof which is then printed out following the
{\tt yes} response.  For space reasons, we only display a few lines of
this printout in the script below.  The two timings printed out at the
end provide the machine time and the human time for the
proof attempt,  respectively.  The Emacs command {\tt M-x
show-last-proof} can be used to bring up an abbreviated version
of the most recently completed proof that can be used as a guide
in developing an informal presentation of the proof.  It displays
the sequents at the branch points in the proof and the commentary in
between.  

\begin{pvsscript}
     prop :   

       |-------
     \{1\}   (A IMPLIES (B IMPLIES C)) AND (A IMPLIES B) AND A IMPLIES C

     Rule? {\bf (flatten)}
     Applying disjunctive simplification to flatten sequent,
     this simplifies to: 
     prop :   

     \{-1\}   (A IMPLIES (B IMPLIES C))
     \{-2\}   (A IMPLIES B)
     \{-3\}   A
       |-------
     \{1\}   C

     Rule? {\bf (split)}
     Splitting conjunctions,
     this yields  3 subgoals: 
     prop.1 :   

     \{-1\}   C
     [-2]   (A IMPLIES B)
     [-3]   A
       |-------
     [1]   C

     which is trivially true.

     This completes the proof of prop.1.

     prop.2 :   

     [-1]   (A IMPLIES B)
     [-2]   A
       |-------
     \{1\}   B
     [2]   C

     Rule? {\bf (split)}
     Splitting conjunctions,
     this yields  2 subgoals: 
     prop.2.1 :   

     \{-1\}   B
     [-2]   A
       |-------
     [1]   B
     [2]   C

     which is trivially true.

     This completes the proof of prop.2.1.

     prop.2.2 :   

     [-1]   A
       |-------
     \{1\}   A
     [2]   B
     [3]   C

     which is trivially true.

     This completes the proof of prop.2.2.


     This completes the proof of prop.2.

     prop.3 :   

     [-1]   (A IMPLIES B)
     [-2]   A
       |-------
     \{1\}   A
     [2]   C

     which is trivially true.

     This completes the proof of prop.3.

     Q.E.D.


     Run time  = 0.52 secs.
     Real time = 14.32 secs.
\end{pvsscript}

\paragraph{Summary. }
The PVS Emacs command {\tt M-x pr} is used to invoke the PVS proof
checker.  Proof goals are represented as sequents with the formulas
numbered.  The command {\tt (flatten)} flattens the top-level
disjunctive structure of all of the sequent formulas so that there are
no disjunctive formulas in the resulting subgoal sequent.  (Variations:
{\tt (flatten *)} is the same as {\tt (flatten)}.  {\tt (flatten +)}
flattens only the consequent formulas, and {\tt (flatten -)} the
antecedent formulas.  {\tt (flatten -2 3 4)} flattens formulas numbered
{\tt -2}, {\tt 3}, and {\tt 4} in the goal sequent.)  The command {\tt
(split)} picks the first top-level conjunctive sequent formula and
generates the subgoals that result from splitting this conjunction.
As with {\tt flatten}, {\tt (split *)} is the same as {\tt (split)},
{\tt (split -)} splits the first antecedent conjunction, {\tt (split +)}
the first consequent conjunction, and {\tt (split -3)} splits the
formula numbered {\tt -3}.  

With the same example, we can now attempt to repeat the proof in order
to explore some other commands.  When we now type {\tt M-x pr} at the
formula {\tt prop} in the theory {\tt proposition}, PVS informs us that
the formula has already been proved and asks whether we wish to retry
proving the formula.  If we respond that we do, then PVS inquires
whether the existing proof should be rerun.  If we choose to rerun the
existing proof, the following script is automatically generated.

\begin{pvsscript}
    prop :   

      |-------
    \{1\}   (A IMPLIES (B IMPLIES C)) AND (A IMPLIES B) AND A IMPLIES C

    Rerunning step: (FLATTEN)
    Applying disjunctive simplification to flatten sequent,
    this simplifies to: 
    prop :   

    \{-1\}   (A IMPLIES (B IMPLIES C))
    \{-2\}   (A IMPLIES B)
    \{-3\}   A
      |-------
    \{1\}   C

    Rerunning step: (SPLIT)
    Splitting conjunctions,
    this yields  3 subgoals: 
    prop.1 :   

    \{-1\}   C
    [-2]   (A IMPLIES B)
    [-3]   A
      |-------
    [1]   C

    which is trivially true.

    This completes the proof of prop.1.

       \vdots
\end{pvsscript}

\paragraph{Summary. }  Proofs can be rerun by responding suitably to the
mini-buffer query when {\tt M-x pr} is invoked on a formula that has a
proof or a partial proof.  Another way to rerun the existing proof is to
type {\tt (rerun)} as the first step in a manual proof.

We can retry the same example to explore some further proof commands.
In this version, we choose not to rerun the existing proof.
Typing the inappropriate command {\tt (split)} results in
{\tt No change} to the proof state since there is no top level
conjunctive formula in the sequent.  We then type {\tt (flatten)} which
flattens the formula followed by {\tt (split)} which generates three
subgoals, the first of which is trivially true.  We then type {\tt
(postpone)} at the second subgoal.  This causes the control to shift to
the third subgoal which is also trivially true.  The control now returns
to the second subgoal.  A further {\tt (postpone)} brings us back to the
same subgoal since there are no other pending subgoals.  At this point,
we simply choose to quit the proof by typing {\tt q} at the {\tt Rule?}
prompt.  At the query, we choose to save the partial proof from the
current proof attempt.  

\begin{pvsscript}
     prop :   

       |-------
     \{1\}   (A IMPLIES (B IMPLIES C)) AND (A IMPLIES B) AND A IMPLIES C

     Rule? {\bf (split)}
     No change on: (SPLIT) 
     prop :   

       |-------
     \{1\}   (A IMPLIES (B IMPLIES C)) AND (A IMPLIES B) AND A IMPLIES C

     Rule? {\bf (flatten)}
     Applying disjunctive simplification to flatten sequent,
     this simplifies to: 
     prop :   

     \{-1\}   (A IMPLIES (B IMPLIES C))
     \{-2\}   (A IMPLIES B)
     \{-3\}   A
       |-------
     \{1\}   C

     Rule? {\bf (split)}
     Splitting conjunctions,
     this yields  3 subgoals: 
     prop.1 :   

     \{-1\}   C
     [-2]   (A IMPLIES B)
     [-3]   A
       |-------
     [1]   C

     which is trivially true.

     This completes the proof of prop.1.

     prop.2 :   

     [-1]   (A IMPLIES B)
     [-2]   A
       |-------
     \{1\}   B
     [2]   C

     Rule? {\bf (postpone)}
     Postponing prop.2.

     prop.3 :   

     [-1]   (A IMPLIES B)
     [-2]   A
       |-------
     \{1\}   A
     [2]   C

     which is trivially true.

     This completes the proof of prop.3.

     prop.2 :   

     [-1]   (A IMPLIES B)
     [-2]   A
       |-------
     \{1\}   B
     [2]   C

     Rule? {\bf (postpone)}
     Postponing prop.2.

     prop.2 :   

     [-1]   (A IMPLIES B)
     [-2]   A
       |-------
     \{1\}   B
     [2]   C

     Rule? {\bf q}
     Do you really want to quit?   (Y or N): {\bf y}
     Would you like the partial proof to be saved? 
     (***Old proof will be overwritten.***)
      (Yes or No) {\bf yes}
     Use M-x revert-proof to revert to previous proof.

     Run time  = 0.77 secs.
     Real time = 22.63 secs.
\end{pvsscript}

We can again type {\tt M-x pr} and this time we can rerun the partial
proof that we saved.  
Notice that we are back at the subgoal where we quit the proof
since this is the only unfinished subgoal in the proof.

\paragraph{Summary. }  The command {\tt (postpone)} is used to navigate
cyclically around the unproved subgoals in a proof.  The PVS Emacs
command {\tt M-x siblings} displays all those subgoals that share the
same parent goal as the current subgoal in the proof.  The PVS Emacs
command {\tt M-x ancestry} displays the chain of goals leading back from
the current goal back to the root node of the proof tree.  A {\tt q} or
{\tt quit} can be used to quit out of a proof-in-progress with the
option of saving the partial proof.  If a previous proof is overwritten
as a result, then the PVS Emacs command {\tt M-x revert-proof}
can be used to recover the earlier proof.  
The PVS Emacs command {\tt M-x show-proof} can be used to display a
proof in progress in such a way that parts of it can be edited and
used as input to the {\tt rerun} proof command.  The PVS Emacs command
{\tt M-x edit-proof} with the cursor positioned on a formula in a theory
brings up a buffer containing the proof of the formula displayed as a
tree of commands.  This displayed proof can also be edited and rerun.


\comment{
\begin{pvsscript}
> ;;;
prop :   

  |-------
\{1\}   (A IMPLIES (B IMPLIES C)) AND (A IMPLIES B) AND A IMPLIES C

Rerunning step: (FLATTEN)
Applying disjunctive simplification to flatten sequent,
this simplifies to: 
prop :   

\{-1\}   (A IMPLIES (B IMPLIES C))
\{-2\}   (A IMPLIES B)
\{-3\}   A
  |-------
\{1\}   C

Rerunning step: (SPLIT)
Splitting conjunctions,
this yields  3 subgoals: 
prop.1 :   

\{-1\}   C
[-2]   (A IMPLIES B)
[-3]   A
  |-------
[1]   C

which is trivially true.

This completes the proof of prop.1.

prop.2 :   

[-1]   (A IMPLIES B)
[-2]   A
  |-------
\{1\}   B
[2]   C

Rerunning step: (POSTPONE)
prop.3 :   

[-1]   (A IMPLIES B)
[-2]   A
  |-------
\{1\}   A
[2]   C

which is trivially true.

This completes the proof of prop.3.

prop.2 :   

[-1]   (A IMPLIES B)
[-2]   A
  |-------
\{1\}   B
[2]   C

Rule? {\bf q}
Do you really want to quit?   (Y or N): {\bf y}
Would you like the partial proof to be saved? 
(***Old proof will be overwritten.***)
 (Yes or No) {\bf no}

Run time  = 0.61 secs.
Real time = 137.33 secs.
\end{pvsscript}
}

\subsubsection{Quantifier Proof Commands}

We now consider a simple example involving quantifiers displayed in the
theory {\tt predicate} below.

\begin{pvsscript}
predicate: THEORY
  BEGIN
   T : TYPE
   x, y, z: VAR T
   P, Q : [T -> bool]

   pred_calc: THEOREM
      (FORALL x: P(x) AND Q(x))
      IMPLIES (FORALL x: P(x)) AND (FORALL x: Q(x))

  END predicate
\end{pvsscript}

The proof script for this example starts with the application of {\tt
(flatten)} to the given conjecture followed by the {\tt (split)} command
to break the consequent conjunction.  In the first branch of the proof,
we use the {\tt (skolem)} command to replace the universally quantified
variable {\tt x} in the consequent formula numbered {\tt 1} with the
(Skolem) constant {\tt X}, where {\tt X} is new (\ie\ undeclared) in the
present context.  The next step is to instantiate the universally
quantified variable {\tt x} in the antecedent formula numbered {\tt -1}
with the constant {\tt X} using the {\tt (inst)} command.  The first
branch of the proof is then easily completed by propositional reasoning.
Note that the two quantifier steps, {\tt skolem} and {\tt inst}, only
affect the outermost quantifier of a formula in the sequent.  Also,
universally quantified variables in consequent formulas are replaced by
new constants, whereas antecedent universally quantified variables are
instantiated with terms.  Existentially quantified variables behave
dually.  The second branch of the proof employs minor variants of the
{\tt skolem} and {\tt inst}.  Here the {\tt (skolem!)} command picks
the first ``skolemizable'' sequent formula and replaces the quantified
variables with internally generated constants (containing exclamations).
The {\tt (inst?)} command picks the first instantiable sequent formula
and tries to find an instantiation for the quantified variables by
matching against the rest of the sequent.

\begin{pvsscript}
    pred_calc :   

      |-------
    \{1\}   (FORALL x: P(x) AND Q(x)) IMPLIES (FORALL x: P(x)) AND (FORALL x: Q(x))

    Rule? {\bf (flatten)}
    Applying disjunctive simplification to flatten sequent,
    this simplifies to: 
    pred_calc :   

    \{-1\}   (FORALL x: P(x) AND Q(x))
      |-------
    \{1\}   (FORALL x: P(x)) AND (FORALL x: Q(x))

    Rule? {\bf (split)}
    Splitting conjunctions,
    this yields  2 subgoals: 
    pred_calc.1 :   

    [-1]   (FORALL x: P(x) AND Q(x))
      |-------
    \{1\}   (FORALL x: P(x))

    Rule? {\bf (skolem 1 {\tt "X"})}
    For the top quantifier in 1, we introduce Skolem constants: X
    this simplifies to: 
    pred_calc.1 :   

    [-1]   (FORALL x: P(x) AND Q(x))
      |-------
    \{1\}   P(X)

    Rule? {\bf (inst -1 {\tt "X"})}
    Instantiating the top quantifier in -1 with the terms: 
     X
    this simplifies to: 
    pred_calc.1 :   

    \{-1\}   P(X) AND Q(X)
      |-------
    [1]   P(X)

    Rule? {\bf (prop)}
    By propositional simplification,

    This completes the proof of pred_calc.1.

    pred_calc.2 :   

    [-1]   (FORALL x: P(x) AND Q(x))
      |-------
    \{1\}   (FORALL x: Q(x))

    Rule? {\bf (skolem!)}
    Skolemizing,
    this simplifies to: 
    pred_calc.2 :   

    [-1]   (FORALL x: P(x) AND Q(x))
      |-------
    \{1\}   Q(x!1)

    Rule? {\bf (inst?)}
    Found substitution:
    x gets x!1,
    Instantiating quantified variables,
    this simplifies to: 
    pred_calc.2 :   

    \{-1\}   P(x!1) AND Q(x!1)
      |-------
    [1]   Q(x!1)

    Rule? {\bf (prop)}
    By propositional simplification,

    This completes the proof of pred_calc.2.

    Q.E.D.
\end{pvsscript}

\comment{
Types in PVS are not necessarily nonempty so that one cannot prove
statements like {\tt (FORALL x:\ P(x)) IMPLIES (EXISTS x:\ P(x))} unless
the variable {\tt x} is known to range over a nonempty type.
}

\paragraph{Summary. }
The command {\tt (skolem 1 "X")} is used to introduce a new constant
{\tt X} in place of the universally quantified variable in the formula
numbered {\tt 1}.  {\tt (skolem 1 ("X" "\_" "Z"))} is to be used if
there are three variables bound by the universal quantifier and only the
first and third are to be replaced by constants.  {\tt (skolem + "X")}
carries out the skolemization step for the first consequent universally
quantified formula, and {\tt (skolem - "X")} for the first antecedent
existentially quantified formula.  The variations of the instantiation
command {\tt inst} are similar to those of {\tt skolem}.  The command
forms {\tt (skolem!)}, {\tt (skolem! 1)}, {\tt (skolem! -)}, etc., are
variants of {\tt skolem} where the new constant names are internally
generated.  The command {\tt (inst?)} is a version of {\tt inst} that
tries to find a matching substitution for a chosen quantified formula.
It can also be supplied a partial substitution to disambiguate the
matching process as in {\tt (inst? - :subst ("x" "X"))}.  Both {\tt
inst} and {\tt inst?}\ take an optional {\tt :copy?}\ argument that can
be given as {\tt T} in order to retain a copy of the original quantified
formula in the sequent in case further instances of the formula are
needed, as in {\tt (inst + ("x" "X") :copy? T)}.  The PVS rule
{\tt inst-cp} is a version of the {\tt inst} that automatically
copies the quantified formula, and {\tt inst} is the non-copying
variant.  
Note that optional
arguments to PVS proof commands can be given by order or by keyword.
To find out the order, the keywords, and defaults for each of the proof
commands, use {\tt M-x help-pvs-prover}.

\subsubsection{Decision Procedures}

The equality and linear inequality decision procedures are the
workhorses of almost any nontrivial PVS proof.  The theory {\tt
decisions} displayed below illustrates some of the power of these
decision procedures.  The formulas marked {\tt THEOREM} are those that
can be proved using the decision procedures, and the ones marked
{\tt CONJECTURE} are either true but cannot be proved solely by the
decision procedures (like {\tt badarith1}) or false (like {\tt badarith}
and {\tt badarith2}) and hence unprovable.  The reader should
invoke {\tt M-x pr} on each of the formulas in {\tt decisions} and
type either {\tt (then (skolem!)(ground))} or
{\tt (then* (skolem!)(flatten)(assert))} to the {\tt Rule?}\ prompt
to observe the effects of the decision procedures.  The command
{\tt assert} is used to either record equality or inequality information
into the data-structures used by the decision procedures, or to simplify
propositional or {\tt IF-THEN-ELSE} structures in a formula, or
carry out the automatic rewrites (to be described below).
The command {\tt (ground)} is a combination of {\tt (prop)} and {\tt (assert)}.

\begin{pvsscript}
decisions: THEORY
  BEGIN
   x,y,v: VAR number
   f: [number -> number]

   eq1: THEOREM x = f(x) IMPLIES f(f(f(x))) = x

   g : [number, number -> number]

   eq2: THEOREM x = f(y) IMPLIES g(f(y + 2 - 2), x + 2) = g(x, f(y) + 2)

   arith: THEOREM  %Proved by decision procedures
     x < 2*y AND y < 3*v IMPLIES 3*x < 18*v

   badarith: CONJECTURE %Not proved; statement is false.
     x < 2*y AND y < 3*v IMPLIES 3*x < 17*v

   badarith1: CONJECTURE %Not proved; statement true but non-linear
     x<0 AND y<0 IMPLIES x*y>0

   i, j, k: VAR int
 
   intarith: THEOREM %Proved by decision procedures  
     2*i < 5 AND i > 1 IMPLIES i = 2

   badarith2: CONJECTURE  %Not proved; stmt. true of integers but not reals.
     2*x < 5 AND x > 1 IMPLIES x = 2

   range : THEOREM  %Proved by decision procedures
     i > 0 AND i < 3 IMPLIES i = 1 OR i = 2

  END decisions
\end{pvsscript}

We now consider an example proof that further illustrates the use of
decision procedures.  The theory {\tt stamps} below contains the formula
asserting that any postage requirement of 8~cents or more can be met
solely with 3 and 5~cent stamps, \ie\ is the sum of some multiple of 3
and some multiple of 5.

\begin{pvsscript}
stamps: THEORY
  BEGIN
  
  i, three, five: VAR nat
  
  stamps: LEMMA (FORALL i: (EXISTS three, five: i + 8 = 3 * three + 5 * five))
  
  END stamps
\end{pvsscript}

In abstract terms, the proof proceeds by induction on {\tt i}.  In the
base case, when {\tt i} is {\tt 0}, the left-hand side is 8.  Letting
{\tt m} and {\tt n} both be {\tt 1} fulfills the equality.  In the
induction case, we know that {\tt that i + 8} can be expressed as {\tt
3*M + 5*N} for some {\tt M} and {\tt N} and we need to find {\tt m} and
{\tt n} such that {\tt i + 8 + 1} is {\tt 3*m + 5*n}.  If {\tt N = 0},
then {\tt M} is at least {\tt 3}.  We then let {\tt m} be {\tt  M - 3}
and {\tt n} be {\tt 2}, \ie\ we remove three 3~cent stamps and add two
5~cent stamps to get postage worth {\tt i + 8 + 1}.  If {\tt N > 0},
then we simply remove a 5~cent stamp and add two 3~cent stamps to prove
the induction conclusion.

In the proof script below, the first command {\tt (induct "i")} directs
PVS to use induction on {\tt i}.  PVS deduces from the type {\tt nat} of
{\tt i} that natural number induction is to be used and formulates an
induction predicate based on the formula number 1 in the sequent.  The
command {\tt induct}, like {\tt prop} and {\tt ground}, is a compound
step or a {\em proof strategy\/}.  Two subgoals are generated
corresponding to the base and induction cases.  In the base case, the
{\tt inst} command is used to instantiate {\tt three} with {\tt 1} and {\tt
five} with {\tt 1}.  The decision procedures are invoked to prove the
resulting trivial arithmetic equality.  In the induction case, the {\tt
skolem} command followed by {\tt flatten} results in a sequent
containing the induction hypothesis in its antecedent and the conclusion
in its consequent part.  The witnesses corresponding to the induction
hypothesis are produced by the {\tt skolem!}\ command.  The case-split
according to {\tt five!1 = 0} is created by the {\tt case} command.  In the
first {\tt five!1 = 0} case, we instantiate the existential quantifiers in
the conclusion as required by the abstract proof.  Since the bound
variable {\tt three} has type {\tt nat} (which is a subtype of the {\tt
integer} type consisting of the non-negative integers), the {\tt inst}
command generates a second (type correctness) subgoal demanding proof
that {\tt three!1 - 3} is at least {\tt 0}.  Both subgoals are discharged
through the use of {\tt assert}.  In the case when {\tt five!1 = 0} is
false, note that the assumption of falsity is indicated by the formula
{\tt five!1 = 0} appearing in the consequent part of the goal sequent.  We
now follow an approach that is slightly different from that of the
previous branch; we use {\tt assert} at this point.  This has no visible
effect on the sequent to be proved, but the falsity of {\tt five!1 = 0} is
noted by the decision procedures for use deeper in the proof.  Now note
that the {\tt inst} command instantiating {\tt five} with {\tt five!1 - 1}
does not generate the type correctness subgoal that was generated in the
previous branch since the decision procedures were able to automatically
demonstrate that {\tt five!1 - 1} was non-negative from the known
information. 

\begin{pvsscript}
     stamps :  

       |-------
     \{1\}   (FORALL i: (EXISTS three, five: i + 8 = 3 * three + 5 * five))
     
     Rule? {\bf (induct "i")}
     Inducting on i,
     this yields  2 subgoals: 
     stamps.1 :  
     
       |-------
     \{1\}   (EXISTS (three: nat), (five: nat): 0 + 8 = 3 * three + 5 * five)
     
     Rule? {\bf (inst 1 1 1)}
     Instantiating the top quantifier in 1 with the terms: 
      1, 1,
     this simplifies to: 
     stamps.1 :  
     
       |-------
     \{1\}   0 + 8 = 3 * 1 + 5 * 1
     
     Rule? {\bf (assert)}
     Simplifying, rewriting, and recording with decision procedures,
     
     This completes the proof of stamps.1.
     
     stamps.2 :  
     
       |-------
     \{1\}   (FORALL (j: nat):
              (EXISTS (three: nat), (five: nat): j + 8 = 3 * three + 5 * five)
                IMPLIES
                (EXISTS (three: nat), (five: nat):
                   j + 1 + 8 = 3 * three + 5 * five))
     
     Rule? {\bf (skolem + "JJ")}
     For the top quantifier in +, we introduce Skolem constants: JJ,
     this simplifies to: 
     stamps.2 :  
     
       |-------
     \{1\}   (EXISTS (three: nat), (five: nat): JJ + 8 = 3 * three + 5 * five)
             IMPLIES
             (EXISTS (three: nat), (five: nat):
                JJ + 1 + 8 = 3 * three + 5 * five)
     
     Rule? {\bf (flatten)}
     Applying disjunctive simplification to flatten sequent,
     this simplifies to: 
     stamps.2 :  
     
     \{-1\}   (EXISTS (three: nat), (five: nat): JJ + 8 = 3 * three + 5 * five)
       |-------
     \{1\}   (EXISTS (three: nat), (five: nat): JJ + 1 + 8 = 3 * three + 5 * five)
     
     Rule? {\bf (skolem!)}
     Skolemizing,
     this simplifies to: 
     stamps.2 :  
     
     \{-1\}   JJ + 8 = 3 * three!1 + 5 * five!1
       |-------
     [1]   (EXISTS (three: nat), (five: nat): JJ + 1 + 8 = 3 * three + 5 * five)
     
     Rule? {\bf (case "five!1 = 0")}
     Case splitting on 
        five!1 = 0, 
     this yields  2 subgoals: 
     stamps.2.1 :  
     
     \{-1\}   five!1 = 0
     [-2]   JJ + 8 = 3 * three!1 + 5 * five!1
       |-------
     [1]   (EXISTS (three: nat), (five: nat): JJ + 1 + 8 = 3 * three + 5 * five)
     
     Rule? {\bf (inst + "three!1 - 3" 2)}
     Instantiating the top quantifier in + with the terms: 
      three!1 - 3, 2,
     this yields  2 subgoals: 
     stamps.2.1.1 :  
     
     [-1]   five!1 = 0
     [-2]   JJ + 8 = 3 * three!1 + 5 * five!1
       |-------
     \{1\}   JJ + 1 + 8 = 3 * (three!1 - 3) + 5 * 2
     
     Rule? {\bf (assert)}
     Simplifying, rewriting, and recording with decision procedures,
     
     This completes the proof of stamps.2.1.1.
     
     stamps.2.1.2 (TCC):   
     
     [-1]   five!1 = 0
     [-2]   JJ + 8 = 3 * three!1 + 5 * five!1
       |-------
     \{1\}   three!1 - 3 >= 0
     
     Rule? {\bf (assert)}
     Simplifying, rewriting, and recording with decision procedures,
     
     This completes the proof of stamps.2.1.2.
     
     
     This completes the proof of stamps.2.1.
     
     stamps.2.2 :  
     
     [-1]   JJ + 8 = 3 * three!1 + 5 * five!1
       |-------
     \{1\}   five!1 = 0
     [2]   (EXISTS (three: nat), (five: nat): JJ + 1 + 8 = 3 * three + 5 * five)
     
     Rule? {\bf (assert)}
     Simplifying, rewriting, and recording with decision procedures,
     this simplifies to: 
     stamps.2.2 :  
     
     \{-1\}   8 + JJ = 5 * five!1 + 3 * three!1
       |-------
     [1]   five!1 = 0
     \{2\}   (EXISTS (three: nat), (five: nat): 9 + JJ = 5 * five + 3 * three)
     
     Rule? {\bf (inst + "three!1 + 2" "five!1 - 1")}
     Instantiating the top quantifier in + with the terms: 
      three!1 + 2, five!1 - 1,
     this simplifies to: 
     stamps.2.2 :  
     
     [-1]   8 + JJ = 5 * five!1 + 3 * three!1
       |-------
     [1]   five!1 = 0
     \{2\}   9 + JJ = 5 * (five!1 - 1) + 3 * (three!1 + 2)
     
     Rule? {\bf (assert)}
     Simplifying, rewriting, and recording with decision procedures,
     
     This completes the proof of stamps.2.2.
     
     
     This completes the proof of stamps.2.
     
     Q.E.D.


\end{pvsscript}

\paragraph{Summary. }
PVS proofs make heavy use of decision procedures to simplify tedious
equality and arithmetic reasoning so that the number
of trivial subgoals can be minimized and to keep the sequent formulas 
simple.  The equality decision procedure employs congruence closure
to propagate equality information along the term structure to quickly
decide whether a sequent containing equalities and other propositions is
true.  An antecedent formula $P$ that is not an equality can be treated
as $P \mbox{\tt = TRUE}$, and a consequent formula $P$ as the equality
$P \mbox{\tt = FALSE}$.  The {\tt assert} rule is the most powerful form
in which decision procedures are applied.  It is a combination of
the {\tt record} rule which records sequent formulas in the
data-structures used by the decision procedures, {\tt simplify} which
simplifies branching and propositional structure using the decision
procedures, {\tt beta} which beta-reduces record, tuple,
function-update, {\tt LAMBDA}, and abstract datatype redexes,
and {\tt do-rewrite} which applies the rewrites specified by
{\tt auto-rewrite} and {\tt auto-rewrite-theory}.  

The {\tt (case $\langle formula\rangle^*$)} command used in the above
proof is extremely useful for case-splitting on a formula.  For example,
if there is no straightforward way to simplify a formula $A$ to another
formula $A'$, then one can case-split on $A'$ so that we can use $A'$ on
one branch and prove it from $A$ on the other branch.  The {\tt case}
command can also be used to replace a term $s$ by $s'$ by case-splitting
on $s = s'$ and using the {\tt replace} proof command (which is not
explained here) to carry out the replacement.

\subsubsection{Using Definitions and Lemmas}

For the purpose of this discussion, we use the following very simple
example of a recursive function that halves a given natural number.

\begin{pvsscript}\label{half}
half: THEORY
  BEGIN
  
  i, j, k: VAR nat
  
  half(i): RECURSIVE nat =
      (IF i = 0 THEN 0 ELSIF i = 1 THEN 0 ELSE half(i - 2) + 1 ENDIF)
        MEASURE (LAMBDA i: i)
  
  half_halves: THEOREM half(2 * i) = i
  
  half_half: THEOREM half(2 * half(2 * i)) = i
  
  END half
\end{pvsscript}

We show a segment of the proof of {\tt half\_halves} where the
definition of {\tt half} is expanded.  Notice that the first use
of {\tt expand} brings in an unsimplified expansion of the definition of
{\tt half}.  When we {\tt undo} this proof step and retry the same
{\tt expand} step following an {\tt assert}, not only is the expansion
simplified, but the equality is itself reduced to {\tt TRUE}.  

\begin{pvsscript}
       \vdots
    half_halves.2 :   

    \{-1\}   half(2 * J) = J
      |-------
    \{1\}   half(2 * (J + 1)) = J + 1

    Rule? {\bf (expand {\tt "half"} +)}
    Expanding the definition of half
    this simplifies to: 
    half_halves.2 :   

    [-1]   half(2 * J) = J
      |-------
    \{1\}   (IF 2 * (J + 1) = 0 THEN 0 ELSE half(2 * (J + 1) - 2) + 1 ENDIF) = J + 1

    Rule? {\bf (undo)}
    This will undo the proof to: 
    half_halves.2 :   

    \{-1\}   half(2 * J) = J
      |-------
    \{1\}   half(2 * (J + 1)) = J + 1
    Sure? (Y or N): y
    half_halves.2 :   

    \{-1\}   half(2 * J) = J
      |-------
    \{1\}   half(2 * (J + 1)) = J + 1

    Rule? {\bf (assert)}
    Invoking decision procedures,
    this simplifies to: 
    half_halves.2 :   

    [-1]   half(2 * J) = J
      |-------
    [1]   half(2 * (J + 1)) = J + 1

    Rule? {\bf (expand {\tt "half"} +)}
    Expanding the definition of half
    this simplifies to: 
    half_halves.2 :   

    [-1]   half(2 * J) = J
      |-------
    \{1\}   TRUE

    which is trivially true.
       \vdots
\end{pvsscript}

The {\tt rewrite} command is an alternative to {\tt expand}, though {\tt
rewrite} can be used to rewrite with both formulas and definitions.  In
the script below, the {\tt rewrite} step replaces the second of the
above applications of {\tt expand}.  Notice that {\tt rewrite} behaves
slightly differently from {\tt expand}, but it too is sensitive to the
facts recorded by the decision procedures from a previous {\tt assert}.

\begin{pvsscript}
       \vdots
    half_halves.2 :   

    [-1]   half(2 * J) = J
      |-------
    [1]   half(2 * (J + 1)) = J + 1

    Rule? {\bf (rewrite {\tt "half"} +)}
    Rewriting using half,  
    this simplifies to: 
    half_halves.2 :   

    [-1]   half(2 * J) = J
      |-------
    \{1\}   half(2 * (J + 1) - 2) + 1 = J + 1

    Rule? {\bf (assert)}
    Invoking decision procedures,

    This completes the proof of half_halves.2.
       \vdots
\end{pvsscript}

In summary, {\tt expand} is used to expand definitions, and {\tt
rewrite} is used to rewrite using definitions and formulas.  Both employ
decision procedures for simplification during rewriting.  Decision
procedures are also used to discharge any conditions (arising from a
conditional rewrite rule) and the type-correctness conditions arising
from the lemma instantiation applied by {\tt rewrite}.  The {\tt expand}
step is the preferred way to expand definitions.
%Here is an example of
%the use of rewrite on a formula taken from the proof of {\tt half\_half}.
%
%\begin{pvsscript}
%       \vdots
%    half_half :   
%
%      |-------
%    \{1\}   half(2 * half(2 * i!1)) = i!1
%
%    Rule? {\bf (rewrite {\tt "half\_halves"})}
%    Rewriting using half_halves,  
%    Q.E.D.
%\end{pvsscript}

\paragraph{Other Commands.}
We have described some typical commands, but have not mentioned many
others.  A partial account of some of those we've omitted is given
below; a complete, annotated list of prover commands can be found
in The PVS Prover Checker Reference Manual~\cite{PVS:prover}.
The {\tt lemma} command is used to bring in an instance of a
lemma as an antecedent sequent formula.  The {\tt extensionality}
proof command is similarly used to bring in the extensionality scheme
given a suitable type expression, \ie\ a function, record, or tuple
type or an abstract datatype.  The {\tt beta} rule is used to carry
out beta-reduction of redexes including those involving {\tt
LAMBDA}-abstraction, record access, tuple access, function updates,
and datatype expressions.  The command {\tt delete} can be used to
drop irrelevant sequent formulas; {\tt hide} is a more conservative
form of {\tt delete} where the formula can be restored using the {\tt
reveal} command.  The PVS Emacs command {\tt M-x show-hidden} shows
the hidden formulas.  The command {\tt typepred} can be used to make
the subtype predicates on a given expression explicit as sequent
formulas.  The {\tt lift-if} command lifts {\tt IF}-branching to the
top-level of a sequent formula through $F${\tt (IF $A$ THEN $s$ ELSE
$t$ ENDIF)} being transformed to {\tt (IF $A$ THEN $F(s)$ ELSE $F(t)$
ENDIF)}.  The commands {\tt auto-rewrite} and {\tt
auto-rewrite-theory} are used to install rewrite rules to be used
automatically by the {\tt assert} command.


\subsubsection{Proof Checker Pragmatics}

The \pvs\ proofs in the tutorial examples reflect a very low level of
automation and should be viewed merely as pedagogical exercises.  The
proof checker actually provides several powerful commands for the
advanced user that make it possible to verify large classes of
theorems using only a small number of steps.  For example, the {\tt
grind} command is usually a good way to complete a proof that only
requires definition expansion, and arithmetic, equality, and
quantifier reasoning.  The decision procedure command {\tt assert} is
used very frequently since it does simplification, automatic
rewriting, and records the sequent formulas in the decision procedure
database.  The {\tt inst?}\ command is the most effective way to
automatically instantiate quantifiers of existential strength.  The
{\tt induct-and-simplify} command is a powerful way to construct
proofs by induction.  The commands {\tt induct-and-rewrite} and {\tt
induct-and-rewrite!}\ are variants of {\tt induct-and-simplify}\@.
These induction commands are able to automatically complete a fairly
large class of induction proofs.

It is not necessary to master all the proof commands in order to use
the \pvs\ proof checker effectively.  In general, it is advisable to
learn the most powerful commands first and only rely on the simpler
commands when the powerful ones fail.  For example, the initial step
in a proof is usually skolemization, and the preferred and most
powerful form here is {\tt skosimp*}. Similarly, {\tt
induct-and-simplify} or one of its variants should be used to initiate
induction proofs.

 Typically, the creative choices in a proof are:
 \begin{enumerate}
 \item The induction scheme: One of the above induction commands should be
 employed here.

 \item The case analysis: If the case analysis is not explicit in the
 propositional structure, then it might be implicit in an embedded {\tt
 IF-THEN-ELSE} or {\tt CASES} structure in which case the {\tt lift-if}
 command should be used to bring the case analyses to the surface of the
 sequent where they can be propositionally simplified.  Otherwise, the case
 analysis has to be supplied explicitly using the {\tt case} command.

\item The quantifier instantiations:  The instantiation of antecedent
universal and succedent existential quantifiers is done automatically by
the {\tt inst?}\ command.  When this fails, the more manual {\tt inst} and
{\tt inst-cp} commands should be used.
\end{enumerate}

The {\tt bddsimp} command is the most efficient way to do propositional
simplification, but {\tt prop} will do when efficiency is not important.
Propositional simplification has to be used with care because it can
generate many subgoals that share the same proof.  The {\tt flatten}
and {\tt split} commands are used to do the propositional
simplification more delicately.  

User-defined {\em proof strategies\/}, similar to the tactics and
tacticals of LCF, can be employed by more advanced \pvs\ users.  A
file containing definitions of basic strategies is distributed with
\pvs\ and provides a good introduction to this topic. The PVS Prover Checker
Reference Manual~\cite{PVS:prover} can be consulted for additional
information on user-defined proof strategies.

Finally, it is helpful to be familiar with the \pvs\ prelude theories,
which provide very useful background mathematics, as well as a rich
source of examples.










% Master File: tutorial.tex
\sloppy

% full list of sections:
%\includeonly{title,intro,informal,pspace,undecide,conclu,ack,rules}

%\pagestyle{myheadings} % page number in upper right corner
%\markboth{Specification and Verification}{}
%\makeindex
%\newcommand{\allttinput}[1]{\hozline{\smaller\smaller\smaller\begin{alltt}\input{#1}\end{alltt}}\hozline}
%\newenvironment{pvsscript}{\hozline\smaller\smaller\smaller\begin{alltt}}{\end{alltt}\hozline}
%\newcommand{\ehdm}{{E{\smaller\smaller HDM}}}
%\newcommand{\Ehdm}{\ehdm}

%\topmargin -10pt
%\textheight 8.5in
%\textwidth 6.0in
%\headheight 15 pt
%\columnwidth \textwidth
%\oddsidemargin 0.5in
%\evensidemargin 0.5in   % fool system for page 0
%\setcounter{topnumber}{9}
%\renewcommand{\topfraction}{.99}
%\setcounter{bottomnumber}{9}
%\renewcommand{\bottomfraction}{.99}
%\setcounter{totalnumber}{10}
%\renewcommand{\textfraction}{.01}
%\renewcommand{\floatpagefraction}{.01}
%\newenvironment{smalltt}{\begin{alltt}\small}{\end{alltt}}
\raggedbottom

\font\largett=cmtt10 scaled\magstep2
\font\hugett=cmtt10 scaled\magstep4
\def\opt{{\smaller\sc {\smaller\smaller \&}optional}}
\def\rest{{\smaller\sc {\smaller\smaller \&}rest}}
\def\default#1{[\,{\tt #1}] }
\def\bkt#1{{$\langle$#1$\rangle$}}
\def\SetFigFont#1#2#3{\smaller\smaller\rm}

%\newenvironment{usage}[1]{\item[usage:\hspace*{-0.175in}]#1\begin{description}\setlength{\itemindent}{-0.2in}\setlength{\itemsep}{0.1in}}{\end{description}}
\renewcommand{\pvstheory}[3]
  {\begin{figure}[htb]\begin{boxedminipage}{\textwidth}%
   {\smaller\smaller\smaller\begin{alltt} \input{#1}\end{alltt}}\end{boxedminipage}%
   \caption{#2}\label{#3}\end{figure}}


\section{Two Hardware Examples}

In this final section, we develop two hardware examples that
illustrate a more sophisticated use of \pvs\ and suggest the
intellectual discipline involved in specifying and proving
industrial-strength applications.  The pipelined microprocessor and
n-bit ripple-carry adder examples provide an opportunity to touch on
modeling issues, specification styles, and hardware proof strategies,
as well as a chance to review many of the \pvs\ language and prover
features described in earlier sections of this tutorial.\footnote{One
point worth noting that may not be apparent in reading these examples
is that the process of specification and verification is an iterative one in
which proof is used not to certify a completed specification, but
as an aid to developing the specification.}



\subsection{A Pipelined Microprocessor}

We first develop a complete proof of a correctness property of the
controller logic of a simple pipelined processor design described at a
register-transfer level.  The design and the property verified are
both based on the processor example given in \cite{Clarke-etal90}.
The example has been used as a benchmark for evaluating how well
finite state-enumeration based tools, such as model checkers, can
handle datapath-oriented circuits with a large number of states by
varying the size of the datapath.  From the perspective of a theorem
prover, the size of the datapath is irrelevant because the
specification and proof are independent of the datapath size.  As a
theorem proving exercise, the challenge is to see if the proof can be
done as automatically as a model checker proof.

\begin{figure}[b]
\begin{center}
\setlength{\unitlength}{0.009375in}%
\begin{picture}(191,262)(112,399)
\thicklines
\put(190,567){\framebox(13,19){}}
\put(190,542){\framebox(13,19){}}
\put(190,514){\framebox(13,19){}}
\put(278,542){\framebox(12,19){}}
\put(278,514){\framebox(12,19){}}
\put(190,461){\framebox(13,38){}}
\put(190,411){\framebox(13,38){}}
\put(228,499){\line( 0,-1){ 38}}
\put(228,449){\line( 0,-1){ 38}}
\put(259,474){\line( 0,-1){ 38}}
\put(228,499){\line( 5,-4){ 31.098}}
\put(228,411){\line( 5, 4){ 31.098}}
\multiput(228,461)(0.66667,-0.33333){19}{\makebox(0.5926,0.8889){\SetFigFont{7}{8.4}{rm}.}}
\multiput(240,455)(-0.66667,-0.33333){19}{\makebox(0.5926,0.8889){\SetFigFont{7}{8.4}{rm}.}}
\put(278,436){\framebox(12,38){}}
\put(203,480){\vector( 1, 0){ 25}}
\put(203,427){\vector( 1, 0){ 25}}
\put(259,455){\vector( 1, 0){ 19}}
\put(265,455){\vector( 0,-1){ 56}}
\put(303,455){\line( 0,-1){ 56}}
\put(303,399){\line(-1, 0){107}}
\put(196,399){\vector( 0, 1){ 12}}
\put(203,524){\vector( 1, 0){ 75}}
\put(203,552){\vector( 1, 0){ 75}}
\put(178,661){\line( 0,-1){ 62}}
\put(165,630){\line( 1, 0){ 13}}
\put(165,599){\framebox(125,62){}}
\put(278,661){\line( 0,-1){ 62}}
\put(165,642){\line(-1, 0){ 25}}
\put(140,642){\line( 0,-1){212}}
\put(140,430){\vector( 1, 0){ 50}}
\put(128,580){\vector( 1, 0){ 62}}
\put(196,461){\line( 0,-1){ 12}}
\multiput(165,511)(8.73333,0.00000){16}{\makebox(0.5926,0.8889){\SetFigFont{10}{12}{rm}.}}
\multiput(165,589)(8.73333,0.00000){16}{\makebox(0.5926,0.8889){\SetFigFont{10}{12}{rm}.}}
\multiput(165,511)(0.00000,8.66667){10}{\makebox(0.5926,0.8889){\SetFigFont{10}{12}{rm}.}}
\multiput(296,511)(0.00000,8.66667){10}{\makebox(0.5926,0.8889){\SetFigFont{10}{12}{rm}.}}
\put(196,499){\vector( 0,-1){0}}
\multiput(196,511)(0.00000,-12.00000){1}{\makebox(0.5926,0.8889){\SetFigFont{10}{12}{rm}.}}
\put(243,486){\vector( 0,-1){0}}
\multiput(243,511)(0.00000,-8.33333){3}{\makebox(0.5926,0.8889){\SetFigFont{10}{12}{rm}.}}
\put(178,524){\vector( 1, 0){ 12}}
\put(128,552){\vector( 1, 0){ 62}}
\put(284,561){\vector( 0, 1){ 38}}
\put(265,455){\vector( 0, 1){ 56}}
\put(290,455){\line( 1, 0){ 13}}
\put(303,455){\line( 0, 1){178}}
\put(303,633){\vector(-1, 0){ 13}}
\put(146,655){\line( 0,-1){125}}
\put(128,614){\line( 1, 0){ 18}}
\put(146,655){\vector( 1, 0){ 19}}
\put(146,605){\vector( 1, 0){ 19}}
\put(146,530){\vector( 1, 0){ 19}}
\put(165,617){\line(-1, 0){ 12}}
\put(153,617){\line( 0,-1){140}}
\put(153,477){\vector( 1, 0){ 37}}
\put(168,518){\makebox(0,0)[lb]{\smash{\SetFigFont{5}{6.0}{bf}stall}}}
\put(209,627){\makebox(0,0)[lb]{\smash{\SetFigFont{5}{6.0}{bf}REGFILE}}}
\put(112,586){\makebox(0,0)[lb]{\smash{\SetFigFont{5}{6.0}{bf}opcode}}}
\put(243,443){\makebox(0,0)[lb]{\smash{\SetFigFont{5}{6.0}{bf}U}}}
\put(243,452){\makebox(0,0)[lb]{\smash{\SetFigFont{5}{6.0}{bf}L}}}
\put(243,461){\makebox(0,0)[lb]{\smash{\SetFigFont{5}{6.0}{bf}A}}}
\put(215,567){\makebox(0,0)[lb]{\smash{\SetFigFont{5}{6.0}{bf}CONTROL}}}
\put(240,542){\makebox(0,0)[lb]{\smash{\SetFigFont{5}{6.0}{bf}dsntdd}}}
\put(165,542){\makebox(0,0)[lb]{\smash{\SetFigFont{5}{6.0}{bf}dstnd}}}
\put(206,527){\makebox(0,0)[lb]{\smash{\SetFigFont{5}{6.0}{bf}stalld}}}
\put(243,514){\makebox(0,0)[lb]{\smash{\SetFigFont{5}{6.0}{bf}stalldd}}}
\put(271,427){\makebox(0,0)[lb]{\smash{\SetFigFont{5}{6.0}{bf}wbreg}}}
\put(153,418){\makebox(0,0)[lb]{\smash{\SetFigFont{5}{6.0}{bf}opreg2}}}
\put(156,461){\makebox(0,0)[lb]{\smash{\SetFigFont{5}{6.0}{bf}opreg1}}}
\put(115,542){\makebox(0,0)[lb]{\smash{\SetFigFont{5}{6.0}{bf}dstn}}}
\put(206,580){\makebox(0,0)[lb]{\smash{\SetFigFont{5}{6.0}{bf}opcoded}}}
\put(112,627){\makebox(0,0)[lb]{\smash{\SetFigFont{5}{6.0}{bf}src2}}}
\put(112,617){\makebox(0,0)[lb]{\smash{\SetFigFont{5}{6.0}{bf}src1}}}
\end{picture}

\end{center}
\caption{A Pipelined Microprocessor}
\label{clarkepipefig}
\end{figure}

\subsubsection{Informal Description}

Figure~\ref{clarkepipefig} shows a block diagram of the pipeline
design.  The processor executes instructions of
the form {\tt (opcode src1 src2 dstn)}, i.e., ``destination register
{\tt dstn} in the register file {\tt REGFILE} becomes some {\tt ALU}
function determined by {\tt opcode} of the contents of source registers
{\tt src1} and {\tt src2}.
Every instruction is executed in three stages (cycles) by the processor:
\begin{enumerate}
\item {\em Read}: Obtain the proper contents of the register file at {\tt src1}
and {\tt src2} and clock them into {\tt opreg1} and {\tt opreg2},
respectively.

\item {\em Compute}: Perform the ALU operation corresponding to the
opcode (remembered in {\tt opcoded}) of the instruction and clock the
result into {\tt wbreg}.

\item {\em Write}: Update the register file at the destination register
(remembered in {\tt dstndd}) of the instruction with the value in
{\tt wbreg}.
\end{enumerate}
The processor uses a three-stage pipeline to simultaneously execute
distinct stages of three successive instructions.  That is, 
the read stage of the current instruction is executed along with the
compute stage of the previous instruction and the write stage
of the previous-to-previous instruction.
Since the {\tt REGFILE} is not updated with the results of the previous and
previous-to-previous instructions while a read is being
performed for the current instruction, the controller
``bypasses'' {\tt REGFILE}, if necessary, to get the correct values for
the read.  The processor can abort, i.e., treat as {\tt NOP},
the instruction in the read stage by asserting the {\tt stall} signal true.
An instruction is aborted by inhibiting its write stage
by remembering the {\tt stall} signal until the write stage via
the registers {\tt stalld} and {\tt stalldd}.
We verify that an instruction entering the pipeline
at any time gets completed correctly, i.e., will write the correct result
into the register file, three cycles later, provided the instruction
is not aborted.

\subsubsection{Formal Specification}

\pvs\ specifications consist of a number of files, each of which
contains one or more theories.
%A theory is a collection of declarations:
%types, constants (including functions), axioms that express properties
%about the constants, and theorems and lemmas to be proved.
%Theories may import other theories.
%Every entity used in a theory must be either declared in an imported theory
%or be part of the prelude (the standard
%collection of theories built-in to \pvs\).
\pvstheory{pipeline-spec}{Microprocessor Specification}{pipeline-spec}
The microprocessor specification is organized into three theories, selected
parts of which are shown in Figures~\ref{pipeline-spec} and
\ref{signal-spec}.
(The complete specification can be found in~\cite{HW-Tutorial:Report}.)
The theory {\tt pipe} (Figure~\ref{pipeline-spec})
contains a specification
of the design and a statement of the correctness property to be
proved.
The theories {\tt signal} and {\tt time}, (Figure~\ref{signal-spec})
imported by {\tt pipe}, declares the types {\tt signal} and {\tt time}
used in {\tt pipe}.

\pvstheory{signal-spec}{Signal Specification}{signal-spec}
%\pvstheory{pipeline-spec}{Microprocessor Specification}{pipeline-spec-part2}

The theory {\tt pipe} is parameterized with respect to the types of the
register address, data, and the opcode field of the instructions.
A theory parameter in PVS can be either a type parameter or
a parameter belonging to a particular type, such as {\tt nat}.
Since {\tt pipe} does not impose any restriction on its parameters,
other than the requirement that they be nonempty, which is stated
in the {\tt ASSUMING} part of the theory,
one can instantiate them with any type.
Every entity declared in a parameterized theory is implicitly parameterized
with respect to the parameters of the theory.
For example, the type {\tt signal} declared in the parameterized
theory {\tt signal} is a parametric type denoting a function that
maps {\tt time} (a synonym for {\tt nat})
to the type parameter {\tt T}.  (The type {\tt signal} is used
to model the wires in our design.)
By importing the theory {\tt signal} uninstantiated in {\tt pipe},
we have the freedom to create any desired instances of the type
{\tt signal}.

In this example, we use a {\em functional} style of specification
to model register-transfer-level digital hardware in logic.
In this style, the inputs to the design and the outputs of every component
in the design are modeled as signals.
Every signal that is an output of a component is
specified as a function of the signals appearing at the inputs to
the component.

This style should be contrasted with
a {\em predicative} style, which is commonly used in most HOL applications.
In the predicative style every hardware component is specified as a
predicate relating the
input and output signals of the component and a design is
specified as a conjunction of the component predicates, with all
the internal signals used to connect the components
hidden by existential quantification.
A proof of correctness for a predicative style specification usually involves
executing a few additional steps at the start of the proof
to essentially transform
the predictative specification into an equivalent functional style.
After that, the proof proceeds similar to that of a proof in
a functional specification.
The additional proof steps required for a predicative specification
essentially unwind the component predicates using
their definitions and then appropriately
instantiate the existentially quantified variables.
An automatic way of performing this translation is discussed in
\cite{HW-Tutorial:Report}, which illustrates more examples
of hardware design verification using PVS.

Returning to our example, the microprocessor specification
in {\tt pipe} consists of two parts.
The first part declares all the signals
used in the design---the inputs
to the design and the internal wires that denote the outputs of components.
The composite state of {\tt REGFILE}, which is represented
as a function from {\tt addr} to {\tt data}, is modeled by the signal
{\tt regfile}.
The signals are declared as uninterpreted constants of appropriate types.
The second part consists of a set of AXIOMs that specify the
the values of the signals over time.
(To conserve space, we have only shown the specification of a subset
of the signals in the design.)
For example, the signal value at the output of the
register {\tt dstnd} at time {\tt t+1} is defined to be that of its
input a cycle earlier.
The output of the ALU, which is a combinational component, is defined
in terms of the inputs at the same time instant.

In PVS, we can use a descriptive style of definition, as illustrated
in this example, by selectively introducing properties of the
constants declared in a theory as AXIOMs.  Alternatively, we can use
the definitional forms provided by the language to define the
constants.  An advantage of using the definitions is that a
specification is guaranteed to be consistent. A disadvantage is that
the resulting specification may be overly specific (i.e.,
overspecified).  An advantage of the descriptive style is that it
gives better control over the degree to which an entity is defined For
example, we could have specified {\tt dstnd} prescriptively, using the
conventional function definition mechanism of PVS, which would have
forced us to specify the value of the signal at time {\tt t = 0} to
ensure that the function is total.  In the descriptive style used, we
have left the value of the signal at {\tt 0} unspecified.

In the present example, the specifications of the signals
{\tt opreg1} and {\tt opreg2} are the most interesting of all.
They have to check for any register collisions that might
exist between the instruction in the
read stage and the instructions in the later stages and bypass reading
from the register file in case of collisions.
The {\tt regfile} signal specification is recursive since the register
file state remains the same as its previous state except,
possibly, at a single register location.
The {\tt WITH} expression is an abbreviation for the result
of updating a function at a given point in the domain value with a new value.
Note that the function {\tt aluop} that denotes the operation ALU performs
for a given {\tt opcode} is left completely unspecified since it
is irrelevant to the controller logic.

The theorem ({\tt correctness}) to be proved states a correctness property
about the execution of the instruction that enters the pipeline at
{\tt t}, provided the instruction is not aborted, i.e., {\tt stall(t)} is
not true.
The equation in the conclusion of the implication compares the
actual value (left hand side) in the destination register three
cycles later, when the result of the instruction would be in place,
with the expected value.
The expected value is the result of applying the {\tt aluop} corresponding
to the opcode of the instruction to the values at the source field
registers in the register file at {\tt t+2}.
We use the state of the register file at {\tt t+2} rather than {\tt t}
to allow for the results of the two previous instructions in the pipeline
to be completed.

\subsubsection{Proof of Correctness}

Once the specification is complete, the next step is to typecheck the
file, which parses and checks for semantic errors, such as undeclared
names and ambiguous types.  As we have already seen, typechecking may
build new files or internal structures such as {\em type correctness
conditions} ({\em \tccs}) that represent {\em proof obligations\/}
that must be discharged before the {\tt pipe} theory can be considered
typechecked.  The typechecker does not generate any \tccs\ in the
present example.  If, for example, one of the assumptions, say for
{\tt addr}, in the {\tt ASSUMING} part of the theory was missing, the
typechecker would generate the following \tcc\ to show that the {\tt
addr} type is nonempty.  The declaration of the signal {\tt src1}
forces generation of this \tcc\ because a function is nonexistent if
its range is empty.

\mbox{}

\noindent
\begin{boxedminipage}{\textwidth}
\begin{alltt}
{\smaller\smaller
% Existence TCC generated (line 17) for src1: signal[addr]
% May need to add an assuming clause to prove this.
  % unproved
src1_TCC1: OBLIGATION (EXISTS (x1: signal[addr]): TRUE);
}
\end{alltt}
\end{boxedminipage}

\mbox{}

%The \pvs\ proof checker runs as a subprocess of Emacs.
%Once invoked on a theorem to be proved, it accepts commands
%directly from the user.

By way of review, the basic objective of developing a proof in \pvs\
as in other subgoal-directed proof checkers (e.g., HOL), is to
generate a {\em proof tree\/} in which all of the leaves are trivially
true.  The nodes of the proof tree are sequents, and in the
prover you are always looking at an unproved leaf of the tree.
The {\em current\/} branch of a proof is the branch leading back to
the root from the current sequent.  When a given branch is complete
(i.e., ends in a true leaf), the prover automatically moves on to the
next unproved branch, or, if there are no more unproven branches,
notifies you that the proof is complete.

%One reason why a proof in \pvs\ differs from a HOL proof is due to
The primitive inference steps in \pvs\ are a lot more
powerful than in HOL; it is not necessary to build complex tactics to
handle tedious lower level proofs in \pvs\@.  A knowledgeable \pvs\
user can typically get proofs to go through mostly automatically by
making a few critical decisions at the start of the proof.  However,
as noted previously, \pvs\ does provide the user with the equivalent
of HOL's tacticals, called {\em strategies}, and other features to
control the desired level of automation in a proof.

The proof of the microprocessor property shown below follows a certain
general pattern that works successfully for most hardware proofs.
This general proof pattern, variants of which have been used
in other verification
exercises \cite{mephisto,HOL:super}, consists of the following
sequence of general proof tasks.
\begin{description}
\item[Quantifier elimination:] Since the decision procedures work on
ground formulas, the user must eliminate the relevant universal
quantifiers by skolemization or selecting variables on which to induct
and existential quantifiers by suitable instantiation.

\item[Unfolding definitions:] The user may have to simplify selected
expressions
and defined function symbols in the goal by rewriting using definitions,
axioms or lemmas.
The user may also have to decide the level to which the function symbols have
to rewritten.

\item[Case analysis:] The user may have to split the proof based on
selected boolean expressions in the current goal and simplify the
resulting goals further.
\end{description}

Each of the above tasks can be accomplished automatically using a short
sequence of primitive PVS proof commands.
The complete proof of the theorem is shown below.
Selected parts of the proof session are reproduced below as we describe
the proof.

\mbox{}

\noindent
\begin{boxedminipage}{\textwidth}
\begin{alltt}
{\smaller\smaller
1: ({\em then*} (skosimp)
2:         (auto-rewrite-theory ``pipe'' :always? t)
3:         (repeat (do-rewrite))
4:         (apply (then* (repeat (lift-if))
5:                       (bddsimp)
6:                       (assert))))
}
\end{alltt}
\end{boxedminipage}

\mbox{}

In the proof, the names of strategies are shown in {\em italics} and
the primitive inference steps in {\tt type-writer font}.
(We have numbered the lines in the proof for reference.)
{\tt Then*} applies
the first command in the list that follows to the current goal;
the rest of the commands in the list are then applied
to each of the subgoals generated by the first command
application.
The {\tt apply} command used in line 5 makes the application of 
a compound proof step implemented by a strategy behave as
an atomic step.

The first goal in the proof session is shown below.
It consists of a
single formula (labeled {\tt \{1\}}) under a dashed line.  This is a
{\em sequent\/}; formulas above the dashed lines are called {\em
antecedents\/} and those below are called {\em succedents\/}.  The
interpretation of a sequent is that the conjunction of the antecedents
implies the disjunction of the succedents.
\comment{
Either or both of the
antecedents and succedents may be empty.\footnote{An empty antecedent
is equivalent to {\tt true}, and an empty succedent is equivalent to
{\tt false}, so if both are empty the sequent is unprovable.}}

\mbox{}

\noindent
\begin{boxedminipage}{\textwidth}
\begin{alltt}
{\smaller\smaller
correctness :   

  |-------
\{1\}   (FORALL t: NOT (stall(t))
                  IMPLIES regfile(t + 3)(dstn(t)) =
                     aluop(opcode(t), regfile(t + 2)(src1(t)),
                           regfile(t + 2)(src2(t))))
}
\end{alltt}
\end{boxedminipage}

\mbox{}

The quantifier elimination task of the proof is accomplished
by the command {\tt skosimp}, which skolemizes all the universally quantified
variables in a formula and flattens the sequent resulting in the following
goal.  Note that {\tt stall(t!1)} has been moved to the
succedent in the sequent because \pvs\ displays every atomic formula in
its positive form.

\mbox{}

\noindent
\begin{boxedminipage}{\textwidth}
\begin{alltt}
{\smaller\smaller
Rule? (skosimp)
Skolemizing and flattening, this simplifies to: 
correctness :   

  |-------
\{1\}   (stall(t!1))
\{2\}   regfile(t!1 + 3)(dstn(t!1))
        =
        aluop(opcode(t!1), regfile(t!1 + 2)(src1(t!1)),
              regfile(t!1 + 2)(src2(t!1)))
}
\end{alltt}
\end{boxedminipage}

\mbox{}

The next task---unfolding definitions---is performed by the commands
in lines 2 through 3.  \pvs\ provides a number of ways of unfolding
definitions ranging from unfolding one step at a time to automatic
rewriting that performs unfolding in a brute-force fashion.
Brute-force rewriting usually results in larger expressions than
controlled unfolding and, hence, potentially larger number of cases to
consider.  If a system provides automatic and efficient rewriting and
case analysis facilities, then the automatic approach is viable,
as illustrated here.  In \pvs\, automatic rewriting is performed
by first entering the definitions and AXIOMs to be used
for unfolding as rewrite rules.  Once entered, the commands that
perform rewriting as part of their repertoire, such as {\tt
do-rewrite} and {\tt assert}, repeatedly apply the rewrite rules until
none of the rules is applicable.  To control the size of the
expression resulting from rewriting and the potential for looping, the
rewriter uses the following restriction for stopping a rewrite: If the
right-hand-side of a rewrite is a conditional expression, then the
rule is applied only if the condition simplifies to true or false.
\comment{ Also, application of a recursive rewrite rule, such as {\tt
regfile\_ax} is inhibited on recursive instances of a function symbol
if the function is inside a conditional expression.}

Here our aim is to unfold every signal in
the sequent so that every signal expression contains only the
start time {\tt t!1}.
So, we make a rewrite rule out of every AXIOM in the theory {\tt pipe}
by means of the command {\tt auto-rewrite-theory} on line 2.
We also force an over-ride of the default restriction for stopping rewriting by
setting the tag\footnote{Tags are one of the ways in which \pvs\ permits the user to modify
the functionality of proof commands.} {\tt always?}\ to true in the {\tt auto-rewrite-theory}
command and embed {\tt do-rewrite} inside a {\tt repeat} loop to force
maximum rewriting.
In the present example, the rewriting
is guaranteed to terminate because every feedback loop is cut by a sequential
component.

At the end of automatic rewriting, the succedent we are trying to prove
is in the form of an equation on two deeply nested conditional expressions
as shown below in an abbreviated fashion.
The various cases in conditional expression shown above
arise as a result of
the different possible conflicts between instructions
in the pipeline.  The equation we are trying to prove contains
two distinct, but equivalent conditional expressions, as in
{\tt IF a THEN b ELSE c ENDIF = IF NOT a THEN c ELSE b ENDIF}, that
can only be proved equal by performing a case-split on one or more of the
conditions.  While {\tt assert} simplifies the leaves of a conditional
expression assuming every condition along the path to the leaves holds,
it does not split propositions.
One way to perform the case-splitting task automatically is
to ``lift'' all the {\tt IF-THEN-ELSE}s to the top so that the equation
is transformed into a propositional formula with unconditional equalities
as atomic predicates.
After performing such a lifting, we can try to reduce the resulting
proposition to true using the propositional simplification command
{\tt bddsimp}.  If {\tt bddsimp} does not simplify the proposition
to true, then it is most likely the case that equations at one or more
of the leaves of the proposition need to be further simplified, e.g., by
{\tt assert}, using the conditions along the path.
If the propositional formula does not reduce to true or false,
{\tt bddsimp} produces a set of subgoals to be proved.
In the present case, each of these goals can be discharged
by {\tt assert}.
The compound proof step appearing on lines 4 through 6 of the proof
accomplishes the case-splitting task.

\mbox{}

\noindent
\begin{boxedminipage}{\textwidth}
\begin{alltt}
{\smaller\smaller
correctness :   

  |-------
[1]   (stall(t!1))
\{2\}   aluop(opcode(t!1),
            IF src1(t!1) = dstnd(t!1) & NOT stalld(t!1)
              THEN aluop(opcoded(t!1), opreg1(t!1), opreg2(t!1))
            ELSIF src1(t!1) = dstndd(t!1) & NOT stalldd(t!1)
            THEN wbreg(t!1)
            ELSE regfile(t!1)(src1(t!1)) ENDIF,
            ....
            ENDIF)
        = aluop(opcode(t!1),
              IF stalld(t!1) THEN IF stalldd(t!1) THEN regfile(t!1)
                ELSE regfile(t!1) WITH [(dstndd(t!1)) := wbreg(t!1)]
                ENDIF
              ELSE ...
              ENDIF(src1(t!1)),
              IF stalld(t!1) THEN IF stalldd(t!1) THEN regfile(t!1)
                ELSE ... ENDIF
              ELSE ...
              ENDIF(src2(t!1)))
}
\end{alltt}
\end{boxedminipage}

\mbox{}

We have found that the sequence of steps shown above works successfully
for proving safety properties of finite state machines that relate
states of the machine that are finite distance apart.  If the strategy
does not succeed, the most likely cause is that either the
property is not true or that a certain property about some of the
functions in the specification unknown to the prover needs to be
proved as a lemma.  In either case, the unproven goals remaining at the
end of the proof provide information about the probable cause.


\subsection{An N-bit Ripple-Carry Adder}

The second example we consider is the verification of a parametrized
N-bit ripple-carry adder circuit.
The theory {\tt adder}, shown in Figure~\ref{adder-spec},
specifies a ripple-carry adder
circuit and a statement of correctness for the circuit.

\pvstheory{adder-spec}{Adder Specification}{adder-spec}

The theory is parameterized with respect to the length of the bit-vectors.
It imports the theories (not shown here)
{\tt full\_adder}, which contains a
specification of a full adder circuit ({\tt fa\_cout} and {\tt fa\_sum}),
and {\tt bv}, which specifies
the bit-vector type ({\tt bvec[N]}) and functions.
An N-bit bit-vector is represented as an array, i.e., a function, from
the the type {\tt below[N]}, a subtype of {\tt nat} ranging from
{\tt 0} through {\tt N-1}, to {\tt bool}; the index {\tt 0} denotes the least
significant bit.
Note that the parameter {\tt N} is constrained to be a {\tt posnat} since
we do not permit bit vectors of length {\tt 0}.
The {\tt adder} theory contains several declarations including a set
of initial variable declarations.
%\pvs\ allows logical variables to be declared globally within a theory
%so that the variables can be used later in function
%definitions and quantified formulas.

The carry bit that ripples through the full adder is specified recursively
by means of the function {\tt nth\_cin}.
%Associated with this definition is a
%{\em measure\/} function, following the {\tt MEASURE} keyword, which
%will be explained below.
The function {\tt bv\_cout} and {\tt bv\_sum} define the carry output
and the bit-vector sum of the adder, respectively.
The theorem {\tt adder\_correct} expresses the conventional correctness
statement of an adder circuit using {\tt bvec2nat}, which returns the
natural number equivalent of an N-bit bit-vector.
Note that variables that are left free in a formula are assumed to be
universally quantified.
We state and prove a more general lemma {\tt adder\_correct\_rec} of which
{\tt adder\_correct} is an instance.
For a given {\tt n < N},
{\tt bvec2nat\_rec} returns the natural number equivalent of
the least significant n-bits of a given bit-vector and {\tt bool2bit}
converts the boolean constants {\tt TRUE} and {\tt FALSE} into the natural
numbers {\tt 1} and {\tt 0}, respectively.

\subsubsection{Typechecking}
\index{typecheck|(}

The typechecker generates several \tccs\
(shown in Figure~\ref{adder-tccs} below) for {\tt adder}.
%These \tccs\ represent
%{\em proof obligations\/} that must be discharged before the {\tt adder}
%theory can be considered typechecked.  The proofs of the \tccs\
%may be postponed until it is convenient to prove them, though it is a good
%idea to view them to see if they are provable.

\pvstheory{adder-tccs}{\tccs\ for Theory {\tt adder}}{adder-tccs}

The first \tcc\ is due to the fact that the first argument to {\tt nth\_cin}
is of type {\tt below[N]}, but the type of the argument ({\tt n-1})
in the recursive
call to {\tt nth\_cin} is integer, since {\tt below[N]} is not closed
under subtraction.
Note that the \tcc\ includes the condition {\tt NOT n = 0}, which holds
in the branch of the {\tt IF-THEN-ELSE} in which the expression
{\tt n - 1} occurs.
A \tcc\ identical to this one is generated for each
of the two other occurrences of the expression {\tt n-1} because
{\tt bv1} and {\tt bv2} also expect arguments of type {\tt below[N]}.
These \tccs\ are not retained because they are subsumed by the first one.

The second \tcc\ is generated by the expression {\tt N-1} in the
definition of the theorem {\tt adder\_correct} because the first
argument to {\tt bv\_cout} is expected to be the subtype {\tt below[N]}.

There is yet another \tcc\ that is internally generated
by \pvs\ but is not even included in the \tccs\ file because
it can be discharged trivially by the typechecker, which calls
the prover to perform simple normalizations of expressions.
This \tcc\ is generated to ensure that the recursive definition
of {\tt nth\_cin} terminates.
\pvs\ does not directly support partial
functions, although its powerful subtyping mechanism allows \pvs\ to
express many operations that are traditionally regarded as partial.
As discussed earlier, the
measure function is used to show that recursive definitions are total by
requiring the measure to decrease with each recursive call.
For the definition of {\tt nth\_cin}, this entails showing
{\tt n-1 < n}, which the typechecker trivially deduces.

In the present case, all the remaining \tccs\ are simple, and in fact can be
discharged automatically
by using the {\tt typecheck-prove} command, which attempts
to prove all \tccs\ that have been generated using a predefined
proof strategy called {\tt tcc}.
\index{TCCs@\tccs|)}\index{typecheck|)}

\subsubsection{Proof of Adder\_correct\_n}
The proof of the lemma uses the same core strategy as in the
microprocessor proof except for the quantifier elimination step.
Since the specification is recursive in the length of the bit-vector,
we need to perform induction on the variable {\tt n}. As we've seen
in earlier proofs,
the user invokes an inductive proof in \pvs\ by means of the command
{\tt induct} with the variable to induct on ({\tt n}) and the induction
scheme to be used ({\tt below\_induction[N]}) as arguments.
The induction used in this case is defined in the \pvs\ prelude and
is parameterized, as is the type {\tt below[N]}, with respect to
the upper limit of the subrange.

This command  generates two subgoals:
the subgoal corresponding to the base case, which is the first goal presented
to prove, is shown in Figure~\ref{base-step}.

\pvstheory{base-step}{Base Step}{base-step}

The goal corresponding to the inductive case is shown below.

\pvstheory{siblings}{Inductive Step}{siblings}

The base and the inductive steps can be proved automatically
using essentially the same strategy used in the microprocessor proof.
A complete proof of {\tt adder\_correct\_n} is shown in~Figure~\ref{siblings}.

\begin{alltt}
{\smaller\smaller
 1: ({\em spread} (induct ``n'' 1 ``below_induction[N]'')
 2:   ( ({\em then*}  (skosimp*)
 3:              (auto-rewrite-defs :always? t)
 4:              (do-rewrite)
 5:              ({\em repeat} (lift-if))
 6:              ({\em apply} ({\em then*} (bddsimp)(assert))))
 7:     ({\em then*} (skosimp*)
 8:             (inst?)
 9:             (auto-rewrite-defs :always? t)
10:             (do-rewrite)
11:             ({\em repeat} (lift-if))
12:             ({\em apply} ({\em then*} (bddsimp)(assert))))))
}
\end{alltt}

The strategy {\em spread} used on line 1 applies the first proof step
({\tt induct})
and then applies the $i^{th}$ element of the list of commands that follow
to the $i^{th}$ subgoal resulting from the application of the first prof step.
Thus, the proof steps listed on lines 2 through 6 prove the base case
of induction, the steps on lines 7 through 12 prove the inductive case, and
the proof step on line 13 takes care of the third \tcc\ subgoal.

%\input{adder-auto-proof-script}

We consider the base case first.
The {\tt induct} command has already instantiated the variable {\tt n}
to {\tt 0}.
The remaining variables are skolemized away by {\tt skosimp*}.
To unfold the definitions in the resulting goal, we use
the command {\tt auto-rewrite-defs}, which makes rewrite rules out
of the definition of every function either directly or indirectly
used in the given formula.
The rest of the proof proceeds exactly as for the microprocessor.

The proof of the inductive step follows exactly the same pattern except
that we need to instantiate the induction hypothesis and use it in
the process of unfolding and case-analysis.
\pvs\ provides a command {\tt inst?}\ that tries to find instantiations
for existential-strength variables in a formula by searching for possible
matches between terms involving these variables with ground terms inside
formulas in the rest of the sequent.  This command finds the desired
instantiations in the present case.  The rest of the proof proceeds
as in the basis case.

Since the inductive proof pattern shown above is applicable to any
iteratively generated hardware designs, we have packaged it into a
general proof strategy called {\tt name-induct-and-bddrewrite}.  The
strategy is parameterized with respect to an induction scheme
and the set of rewrite rules to be used for unfolding.  We have used
the strategy to prove an N-bit ALU~\cite{cantu:alu} that executes 12
microoperations by cascading N 1-bit ALU slices.


\section{Exercises}

\newtheorem{prob}{Problem}

\begin{prob}
Based on the discussion of the specification of stacks, try to specify a
PVS theory formalizing queues.  Can the PVS abstract datatype facility
be used for specifying queues?
\end{prob}

\begin{prob}
Specify binary trees with value type {\tt T} as a parametric abstract
datatype in PVS.
\end{prob}

\begin{prob}
Specify a PVS theory formalizing {\em ordered\/} binary trees with respect to a
type parameter {\tt T} and a given total-ordering relation, \ie\ define
a predicate {\tt ordered?} that checks if a given binary tree is ordered
with respect to the given total ordering.
\end{prob}

\begin{prob}
  Prove the stack axioms for the definitions stated in {\tt newstacks}.
\end{prob}

\begin{prob}
  Prove the theorems in the theory {\tt half} (Page~\pageref{half}).
\end{prob}

\begin{prob}
  Define the operation for carrying out the ordered insertion of a value
into an ordered binary tree.  Prove that the insertion operator applied
to an ordered binary tree returns an ordered binary tree.
\end{prob}

\setcounter{section}{0}
\part{PVS Reference}
\markboth{PVS Reference}{}
\cleardoublepage
\newcommand{\wiftnewpage}{\newpage}
\newcommand{\refnewpage}{}
% Document Type: LaTeX
% Master File: refcard.tex
% Reference Card for \pvs\ version 6.1; adapted from the refcard.tex
% distributed with GNU Emacs
%**start of header

%\documentstyle[alltt,relative]{article}
%\pagestyle{empty}
\newcount\columnsperpage

% This file can be printed with 1, 2, or 3 columns per page (see below).
% Specify how many you want here.  Nothing else needs to be changed.

\columnsperpage=1

% This file is intended to be processed by LaTeX
%
% The final reference card has eight columns, four on each side.
% This file can be used to produce it in any of three ways:
% 1 column per page
%    produces six separate pages, each of which needs to be reduced to 80%.
%    This gives the best resolution.
% 2 columns per page
%    produces three already-reduced pages.
%    You will still need to cut and paste.
% 3 columns per page
%    produces two pages which must be printed sideways to make a
%    ready-to-use 8.5 x 11 inch reference card.
%    For this you need a dvi device driver that can print sideways.
% Which mode to use is controlled by setting \columnsperpage above.
%
% Author:
%  Sam Owre
%  Internet: owre@csl.sri.com

\def\pvs{{PVS}}
\def\titlepvs{{PVS}}
\def\headingpvs{{PVS}}
\def\headinglatex{{\bf L\kern-.36em\raise.3ex\hbox{\smaller\smaller\bf A}\kern-.15em
    T\kern-.1667em\lower.7ex\hbox{E}\kern-.125emX}}

\def\versionnumber{2.0\alpha+}
\def\year{1995}
\def\version{February \year\ v\versionnumber}

%\def\shortcopyrightnotice{\vskip 1ex plus 2 fill
%  \centerline{\small \copyright\ \year\ Free Software Foundation, Inc.
%  Permissions on back.  v\versionnumber}}

%\def\copyrightnotice{
%\vskip 1ex plus 2 fill\begingroup\small
%\centerline{Copyright \copyright\ \year\ Free Software Foundation, Inc.}
%\centerline{designed by Stephen Gildea, \version}
%\centerline{for GNU Emacs version 18 on Unix systems}

%Permission is granted to make and distribute copies of
%this card provided the copyright notice and this permission notice
%are preserved on all copies.
%
%For copies of the GNU Emacs manual, write to the Free Software
%Foundation, Inc., 675 Massachusetts Ave, Cambridge MA 02139.
%
%\endgroup}

% make \bye not \outer so that the \def\bye in the \else clause below
% can be scanned without complaint.
%\def\bye{\par\vfill\supereject\end}

\newdimen\intercolumnskip
\newbox\columna
\newbox\columnb

\def\ncolumns{\the\columnsperpage}

\message{[\ncolumns\space 
  column\if 1\ncolumns\else s\fi\space per page]}

\def\scaledmag#1{ scaled \magstep #1}

% This multi-way format was designed by Stephen Gildea
% October 1986.
\if 1\ncolumns
  \hsize \textwidth
%  \textwidth \hsize
%  \vsize 10in
%  \textheight \vsize
%  \topmargin -.6in
%  \voffset -.7in
  \font\titlefont=cmbx10 \scaledmag3
  \font\headingfont=cmbx10 \scaledmag1
  \font\smallfont=cmr7
  \font\smallsy=cmsy7

%  \footline{\hss\folio}
%  \def\makefootline{\baselineskip10pt\hsize6.5in\line{\the\footline}}
\else
  \hsize 3.2in
  \vsize 7.95in
  \hoffset -.75in
  \voffset -.745in
  \font\titlefont=cmbx10 \scaledmag2
  \font\headingfont=cmbx10 \scaledmag1
  \font\smallfont=cmr6
  \font\smallsy=cmsy6
  \font\eightrm=cmr8
  \font\eightbf=cmbx8
  \font\eightit=cmti8
  \font\eighttt=cmtt8
  \font\eightsy=cmsy8
  \textfont0=\eightrm
  \textfont2=\eightsy
  \def\rm{\eightrm}
  \def\bf{\eightbf}
  \def\it{\eightit}
  \def\tt{\eighttt}
  \normalbaselineskip=.8\normalbaselineskip
  \normallineskip=.8\normallineskip
  \normallineskiplimit=.8\normallineskiplimit
  \normalbaselines\rm		%make definitions take effect

  \if 2\ncolumns
    \let\maxcolumn=b
%    \footline{\hss\rm\folio\hss}
    \def\makefootline{\vskip 2in \hsize=6.86in\line{\the\footline}}
  \else \if 3\ncolumns
    \let\maxcolumn=c
    %\nopagenumbers
  \else
    \errhelp{You must set \columnsperpage equal to 1, 2, or 3.}
    \errmessage{Illegal number of columns per page}
  \fi\fi

  \intercolumnskip=.46in
  \def\abc{a}
  \output={%
      % This next line is useful when designing the layout.
      %\immediate\write16{Column \folio\abc\space starts with \firstmark}
      \if \maxcolumn\abc \multicolumnformat \global\def\abc{a}
      \else\if a\abc
	\global\setbox\columna\columnbox \global\def\abc{b}
        %% in case we never use \columnb (two-column mode)
        \global\setbox\columnb\hbox to -\intercolumnskip{}
      \else
	\global\setbox\columnb\columnbox \global\def\abc{c}\fi\fi}
  \def\multicolumnformat{\shipout\vbox{\makeheadline
      \hbox{\box\columna\hskip\intercolumnskip
        \box\columnb\hskip\intercolumnskip\columnbox}
      \makefootline}\advancepageno}
  \def\columnbox{\leftline{\pagebody}}

  \def\bye{\par\vfill\supereject
    \if a\abc \else\null\vfill\eject\fi
    \if a\abc \else\null\vfill\eject\fi
    \end}  
\fi

% we won't be using math mode much, so redefine some of the characters
% we might want to talk about
%\catcode`\^=12
%\catcode`\_=12

\chardef\\=`\\
\chardef\{=`\{
\chardef\}=`\}

\hyphenation{mini-buf-fer}

\parindent 0pt
\parskip 1ex plus .5ex minus .5ex

\def\small{\smallfont\textfont2=\smallsy\baselineskip=.8\baselineskip}
\def\hyphfill{\cleaders\hbox{-}\hfill}

\outer\def\newcolumn{\vfill\eject}

\outer\def\title#1{{\titlefont\centerline{#1}}\vskip 1ex plus .5ex}

\outer\def\refsection#1{\par % \filbreak
  \vskip 2ex plus 1ex minus 1ex {\headingfont #1}%\mark{#1}%
  \vskip 1ex plus .5ex minus .75ex}
%\outer\def\subsection#1{\par\filbreak
%  \vskip 0.5ex plus .25ex minus .4ex {\mbox{}\hyphfill\fbox{#1}\hyphfill\mbox{}}\mark{#1}%
%  \vskip -1ex}

\newdimen\keyindent

\def\beginindentedkeys{\keyindent=1em}
\def\endindentedkeys{\keyindent=0em}
\endindentedkeys

\def\paralign{\vskip\parskip\halign}

\def\<#1>{$\langle${\rm #1}$\rangle$}

\def\kbd#1{{\tt#1}\null}	%\null so not an abbrev even if period follows

\def\beginexample{\par\leavevmode\begingroup
  \obeylines\obeyspaces\parskip0pt\tt}
{\obeyspaces\global\let =\ }
\def\endexample{\endgroup}

\def\key#1#2{\leavevmode\hbox to \hsize{\vtop
  {\hsize=.6\hsize\rightskip=1em
  \hskip\keyindent\relax#1}\kbd{#2}\hfil}}

\def\nkey#1#2{\leavevmode\hbox to \hsize{\vtop
  {\hsize=.4\hsize\rightskip=1em
  \hskip\keyindent\relax#1${}^\dagger$}\kbd{#2}\hfil}}

\newbox\metaxbox
\setbox\metaxbox\hbox{\kbd{M-x }}
\newdimen\metaxwidth
\metaxwidth=\wd\metaxbox

\def\metax#1#2{\leavevmode\hbox to \hsize{\hbox to .75\hsize
  {\hskip\keyindent\relax#1\hfil}%
  \hskip -\metaxwidth minus 1fil
  \kbd{#2}\hfil}}

\def\threecol#1#2#3{\hskip\keyindent\relax#1\hfil&\kbd{#2}\quad
  &\kbd{#3}\quad\cr}

\def\var#1#2{\hskip\keyindent\relax#1\hfil&\kbd{#2}\quad\cr}
%\def\var#1#2{\hskip\keyindent\relax#1\hfil&\kbd{}\quad\cr}

\def\cmdkey#1#2{\leavevmode\hbox to \hsize{\vtop
  {\hsize=.6\hsize\rightskip=1em
  \hskip\keyindent\relax#1}\kbd{#2}\hfil}}

%\outer\def\refsubsection#1{\par\filbreak
%  \vskip 0.5ex plus .25ex minus .4ex {\mbox{}\hyphfill\fbox{#1}\hyphfill\mbox{}}\mark{#1}%
%  \vskip -1ex}

% redefine refsubsection to be same as refsection

\outer\def\refsubsection#1{\par % \filbreak
  \vskip 2ex plus 1ex minus 1ex {\headingfont #1}%\mark{#1}%
  \vskip 1ex plus .5ex minus .75ex}
%\outer\def\subsection#1{\par\filbreak
%  \vskip 0.5ex plus .25ex minus .4ex {\mbox{}\hyphfill\fbox{#1}\hyphfill\mbox{}}\mark{#1}%
%  \vskip -1ex}



%**end of header

%\begin{document}
%\title{\titlepvs\ Reference Card}

\centerline{Reference to PVS Version 2.0$\alpha+$}
\vspace*{2ex}

\markright{PVS Files}
\title{\titlepvs\ Files}
\addcontentsline{toc}{section}{\numberline{1} PVS Files}

\input{../refcard/pvs-files}

\input{../refcard/latex-ref}

\newcolumn
\title{\titlepvs\ Language Summary}
\markright{PVS Language Summary}
\addcontentsline{toc}{section}{\numberline{2} PVS Language Summary}
\input{../refcard/language-ref}

\newcolumn
\title{\titlepvs\ Emacs Commands}
\markright{PVS Emacs Commands}
\addcontentsline{toc}{section}{\numberline{3} PVS Emacs Commands}

\input{../refcard/emacs-commands}


\newcolumn
%\def\vars#1{\paralign to \hsize{##\tabskip=10pt plus 1 fil&\vtop{\parindent=0pt\hsize=0.1in{\small\strut##\strut}}\cr#1}}
\def\pcmds#1{\setlength{\partopsep}{-0.1in}\begin{description}\setlength{\itemsep}{-0.05in}#1\end{description}}
%\def\pcmd#1#2{\item[{\tt #1}]\makebox[0in]{}\dotfill\makebox[0in]{#2}}
\def\pcmd#1#2{\item{\tt #1}}
\def\opt{\textmd{\textsc{{\footnotesize \&}opt }}}
\def\rest{\textmd{\textsl{{\footnotesize \&}rest }}}
\def\default#1{[{\tt #1}]}


\title{\titlepvs\ Prover Commands}
\markright{PVS Prover Commands}
\addcontentsline{toc}{section}{\numberline{4} PVS Prover Commands}

\input{../refcard/prover-cmds}

\wiftnewpage

\markright{PVS Prover Emacs Shortcuts}
\input{../refcard/prover-emacs-cmds}


%\copyrightnotice

%\end{document}


\newpage
\def\UrlFont{\tt}
\section*{References}
\markboth{}{References}
\bibliographystyle{modalpha}
\addcontentsline{toc}{section}{References}
\bibliography{/homes/rushby/jmr,/project/pvs/doc/pvs,/homes/shankar/tex}


\end{document}


