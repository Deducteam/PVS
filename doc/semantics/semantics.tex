% Document Type: LaTeX
% Master File: semantics.tex
\documentclass [12pt,twoside]{cslreport} 
\usepackage{fleqn,alltt,spacecites,relsize,url,amssymb,verbatim}
%\usepackage{doublespace}
%\setlength{\textwidth}{6.5in}
%\setlength{\textheight}{9in}
%\setlength{\topmargin}{-.6in}
%\setlength{\oddsidemargin}{0in}
%\setlength{\evensidemargin}{0in}
\setlength{\textwidth}{5.5in}
\setlength{\textheight}{8in}
\setlength{\topmargin}{.25in}
\setlength{\oddsidemargin}{.5in}
\setlength{\evensidemargin}{0in}
\setlength{\mathindent}{0.2in}
%\input amssym.def
%\input amssym
%\input commands
%\includeonly{soundness}
%%
%
%    LaTeX  Macro File  /usr2/jcm/tex/macros.tex
%
%
\tracingonline=0 % shorter error messages (on screen)

%\long\def\comment#1{} % mulit-line comments
%\newcommand{\note}[1]{\fbox{#1}}

% font changes  (as function calls, scribe style)

\renewcommand{\i}[1]{{\it #1\/}}       % italics with space correction
%\newcommand{\emex}[1]{\/{\em #1}}   % emphasis in example, exercise, theorem, etc.
\renewcommand{\c}[1]{{\sc #1}}       % small caps (eliminates \c as cedilla)
%\newcommand{\r}[1]{{\rm #1}}            % for roman font in math mode
%\newcommand{\s}[1]{{\scr #1}}          % script (for use with tatex) 
\newcommand{\s}[1]{{\cal #1}}
\renewcommand{\b}[1]{{\bf #1}}           % bold face
\newcommand{\calg}[1]{{\cal #1}}          % caligraphic 

%\newcommand{\q}[1]{``#1''}    % matching quotes for in-line quotation

% numbered environments

%\newcounter{partcounter}
%\setcounter{partcounter}{0}
%\renewcommand{\part}[1]{\newpage \addtocounter{partcounter}{1}
%\noindent{\Large \bf Part \Roman{partcounter}. #1 } \\[1ex]}

\newtheorem{thm}{Theorem}[section]
\newtheorem{theorem}[thm]{Theorem}
\newtheorem{lemma}[thm]{Lemma}
\newtheorem{cor}[thm]{Corollary}
\newtheorem{corollary}[thm]{Corollary}
\newtheorem{claim}[thm]{Claim}
\newtheorem{prop}[thm]{Proposition}
\newtheorem{conj}[thm]{Conjecture}
\newtheorem{definition}[thm]{Definition}
\newtheorem{exercise}[thm]{Exercise}
\newtheorem{example}[thm]{Example}
\newtheorem{remark}[thm]{Remark}
\newtheorem{open}[thm]{Open Problem}

%\newcommand{\proof}{\\{\bf Proof.}\ }
\newenvironment{proof}{{\bf Proof. }}{\thmbox}

% axiom and inference rule (centered, in math mode,  with name at left)

\newcommand{\axiom}[2]
{\[ \hbox to \columnwidth
    { \rlap{$#2$} \hfil {$ #1 $} \hfil }
\]}

\newcommand{\infrule}[4]
{\[ \hbox to \columnwidth %\textwidth
    { \rlap{$#4$} \hfil $
      \frac {\strut\displaystyle #1 } {\strut\displaystyle #2 } \; \rlap{$#3$} \hfil $
      \hfil }
\]}
\newcommand{\infruletw}[4]
{\[ \hbox to \textwidth
    { \rlap{$#4$} \hfil $
      \frac {\strut\displaystyle #1 } {\strut\displaystyle #2 } \; \rlap{$#3$} \hfil $
      \hfil }
\]}

%\newcommand{\infrule}[4]
%{\[ \hbox to \columnwidth
%    { \rlap{$#4$} \hfil $
%      {{\displaystyle\strut #1}\over{\displaystyle\strut #2}}\quad\makebox[0pt][l]{\it #3} $
%      \hfil }
%\]}

% axiom and inference rule macros for use in tables, etc.
% presumed in math mode #1=axiom, #2=side condition
\newcommand{\Axiom}[2]{
{\displaystyle\strut #1}\qquad\makebox[0pt][l]{\it #2}
}
% presumed in math mode #1=top, #2=bottom, #3=side condition
\newcommand{\Infrule}[3]{
{{\displaystyle\strut #1}\over{\displaystyle\strut #2}}\;\mbox{\scriptsize$\bf #3$}
}

% sequence of #1's, numbered up to #2  (e.g., seq{x}{n} for x1, ..., xn   )

\newcommand{\seq}[2]{#1_{1} \ldots #1_{#2}}  
%\newcommand{\ie}{{\it i.e.}}
%\newcommand{\eg}{{\it e.g.}}
%\newcommand{\cf}{{\it c.f.\,}}

% common symbols

\newcommand{\union}{\cup}
\newcommand{\intersect}{\cap}
\newcommand{\subs}{\subseteq}
\newcommand{\el}{\in}
\newcommand{\nel}{\not\in}
\newcommand{\ns}{\emptyset}
\newcommand{\compose}{\circ}
\newcommand{\set}[2]{ \{\, #1 \,\mid\, #2 \,\}  } % set macro
\newcommand{\infinity}{\infty}
\newcommand{\pair}[1]{\langle #1 \rangle}
\newcommand{\tuple}[1]{\langle #1 \rangle}


\newcommand{\fa}{\forall}
\newcommand{\te}{\exists}
\newcommand{\imp}{\supset}
\renewcommand{\implies}{\supset}
\newcommand{\ts}{\vdash}
\newcommand{\dts}{\models}

\newcommand{\aro}{\mathord\rightarrow} % see pages 154-155 of TeX manual
\newcommand{\paro}{\rightharpoonup} 
\newcommand{\karo}{\mathop\Rightarrow} % see pages 154-155 of TeX manual
%\newcommand{\cross}{\times}
%\newcommand{\dlb}{\lbrack\!\lbrack}
%\newcommand{\drb}{\rbrack\!\rbrack}
\newcommand{\mean}[1]{\lbrack\!\lbrack #1 \rbrack\!\rbrack}
\newcommand{\lam}{\lambda}
\newcommand{\subst}[2]{{}[#1/#2]}
\renewcommand{\dot}{\mathrel{\bullet}}
\newcommand{\Dinf}{D_{\infty}}
\newcommand{\bottom}{\perp}

\mathcode`:="603A  % treat : as punctuation instead of relation in math mode
\mathchardef\colon="303A	% relation colon

\newcommand{\eqdef}{\mathrel{:=}}
\newcommand{\Dom}{\mathop{\rm dom}}
\newcommand{\Pow}{\mathop{\rm Pow}}

\newcommand{\aequiv}{\equiv_\alpha}
\newcommand{\baro}{\buildrel \beta \over \rightarrow}
\newcommand{\earo}{\buildrel \eta \over \rightarrow}
\newcommand{\red}{\rightarrow\!\!\!\!\rightarrow}
\newcommand{\backred}{\leftarrow\!\!\!\!\leftarrow}
\newcommand{\bred}{\buildrel \beta \over \red}
\newcommand{\ered}{\buildrel \eta \over \red}
\newcommand{\conv}{\leftrightarrow}
\newcommand{\beconv}{\buildrel {\beta, \eta} \over\leftrightarrow}

% lambda calculus abbreviations
\newcommand{\letdec}[3]{\b{let\ } #1 = #2 \b{\ in\ } #3}
\newcommand{\letrec}[3]{\b{letrec\ } #1 = #2 \b{\ in\ } #3}

\newcommand{\thmbox}
   {{\ \hfill\hbox{%
      \vrule width1.0ex height1.0ex
   }\parfillskip 0pt }}
\newcommand{\qed}{\thmbox}


% make single spacing

\newcommand{\singlespace}{\renewcommand{\baselinestretch}{1}\@normalsize}
\newcommand{\etal}{{\em et al.}}

% bycase command for definition by cases (pg 49 of LaTeX)

\newcommand{\bycase}[1]
	{\left\{ \begin{array}{ll}  #1  \end{array} \right. }


% make \cite put blanks after the comma (use in alpha style)

%\def\@citex[#1]#2{\if@filesw\immediate\write\@auxout{\string\citation{#2}}\fi
%  \def\@citea{}\@cite{\@for\@citeb:=#2\do
%    {\@citea\def\@citea{,\penalty100\hskip2.5pt plus1.5pt minus.8pt}%
%       \@ifundefined{b@\@citeb}{{\bf ?}\@warning
%       {Citation `\@citeb' on page \thepage \space undefined}}
%\hbox{\csname b@\@citeb\endcsname}}}{#1}}

\newcommand{\thmbox}
   {{\ \hfill\hbox{%
      \vrule width1.0ex height1.0ex
   }\parfillskip 0pt }}
\newtheorem{thm}{Theorem}[chapter]
\newtheorem{prop}[thm]{Proposition}
\newenvironment{proof}{{\bf Proof. }}{\thmbox}
\newcommand{\Infrule}[3]{
{{\displaystyle\strut #1}\over{\displaystyle\strut #2}}\qquad\makebox[0pt][l]{\it #3}
}
\newcommand{\tuple}[1]{\langle #1 \rangle}
\newtheorem{example}[thm]{Example}
\newcommand{\aro}{\mathord\rightarrow} % see pages 154-155 of TeX manual
\newcommand{\mean}[1]{\lbrack\!\lbrack #1 \rbrack\!\rbrack}
\newcommand{\pair}[1]{\langle #1 \rangle}
\newcommand{\union}{\cup}
\newcommand{\funtype}[2]{[#1 \aro #2]}
\newcommand{\tupletype}[1]{[#1]}
\newcommand{\subtype}[3]{\{#1 : #2 | #3 \}}
\newcommand{\tauGamma}[1]{\tau(\Gamma)(#1)}
\newcommand{\Mgamma}[1]{{\mathcal M}(\Gamma\vbar\gamma)(#1)}
\newcommand{\proj}[1]{\mathtt{p}_{#1}}
\newcommand{\lis}[2]{#1_{1}, \ldots, #1_{#2}}
\newcommand{\listwo}[2]{#1_{1}, #1_{2}}
\newcommand{\ol}[1]{\overline{#1}}
%\newcommand{\ul}[1]{\underline{#1}}
\newcommand{\twob}{\mathbf{2}}
\newcommand{\oneb}{\mathbf{1}}
\newcommand{\zerob}{\mathbf{0}}
\newcommand{\reals}{\mathbf{R}}
\newcommand{\ttbool}{\mathtt{bool}}
\newcommand{\ttreal}{\mathtt{real}}
\newcommand{\tttrue}{\mathtt{TRUE}}
\newcommand{\ttfalse}{\mathtt{FALSE}}
\newcommand{\tttype}{\mathtt{TYPE}}
\newcommand{\ttcontext}{\mathtt{CONTEXT}}
\newcommand{\ttvar}{\mathtt{VAR}}
\newcommand{\ttconstant}{\mathtt{CONSTANT}}
\newcommand{\ttvariable}{\mathtt{VARIABLE}}
\newcommand{\tttheory}{\mathtt{THEORY}}
\newcommand{\ttif}{\mathtt{IF}}
\newcommand{\ttint}{\mathtt{int}}
\newcommand{\ttnat}{\mathtt{nat}}
\newcommand{\itkind}{\textit{kind}}
\newcommand{\ittype}{\textit{type}}
\newcommand{\itdef}{\textit{definition}}
\newcommand{\itformals}{\textit{formals}}
\newcommand{\vbar}{\ |\ }
\newcommand{\sima}{\stackrel{a}{\sim}} %/{\scriptsize{a}}
\newenvironment{Eg}[1]{\begin{example}[#1]\label{eg:#1}\em }{\thmbox\end{example}}
\newtheorem{definition}{Definition}
\newtheorem{lemma}{Lemma}
\newtheorem{theorem}{Theorem}
%% Derived from John Rushby's prelude.tex, modified for NFSS2
%
% define variants of the \LaTeX macro that avoid using \sc
% for use in headings
%

% Define fonts that work in math or text mode
\def\dwimrm#1{\ifmmode\mathrm{#1}\else\textrm{#1}\fi}
\def\dwimsf#1{\ifmmode\mathsf{#1}\else\textsf{#1}\fi}
\def\dwimtt#1{\ifmmode\mathtt{#1}\else\texttt{#1}\fi}
\def\dwimbf#1{\ifmmode\mathbf{#1}\else\textbf{#1}\fi}
\def\dwimit#1{\ifmmode\mathit{#1}\else\textit{#1}\fi}
\def\dwimnormal#1{\ifmmode\mathnormal{#1}\else\textnormal{#1}\fi}

\def\BigLaTeX{{\rm L\kern-.36em\raise.3ex\hbox{\small\small A}\kern-.15em
    T\kern-.1667em\lower.7ex\hbox{E}\kern-.125emX}}
\def\BoldLaTeX{{\bf L\kern-.36em\raise.3ex\hbox{\small\small\bf A}\kern-.15em
    T\kern-.1667em\lower.7ex\hbox{E}\kern-.125emX}}
%\def\labelitemi{$\bullet$}
\def\labelitemii{$\circ$}
\def\labelitemiii{$\star$}
\def\labelitemiv{$\diamond$}
\newcommand{\tcc}{{\small\small TCC}}
\newcommand{\tccs}{\tcc s}
\newcommand{\emacs}{{Emacs}}
\newcommand{\Emacs}{\emacs}
\newcommand{\ehdm}{{E{\small\small HDM}}}
\newcommand{\Ehdm}{\ehdm}
\newcommand{\tm}{$^{\mbox{\tiny TM}}$}
\newcommand{\hozline}{{\noindent\rule{\textwidth}{0.4mm}}}

\newcommand{\allclear}%
  {\mbox{\boldmath$\stackrel{\raisebox{-.2ex}[0pt][0pt]%
              {$\textstyle\oslash$}}{\displaystyle\bot}$}}

\newenvironment{private}{}{}

\newenvironment{smalltt}{\begin{alltt}\small}{\end{alltt}}

\newlength{\hsbw}

\newenvironment{session}%
  {\begin{flushleft}
   \setlength{\hsbw}{\linewidth}
   \addtolength{\hsbw}{-\arrayrulewidth}
   \addtolength{\hsbw}{-\tabcolsep}
   \begin{tabular}{@{}|c@{}|@{}}\hline 
   \begin{minipage}[b]{\hsbw}
   \begingroup\small\mbox{ }\\[-1.8\baselineskip]\begin{alltt}}
  {\end{alltt}\endgroup\end{minipage}\\ \hline 
   \end{tabular}
   \end{flushleft}}

\newenvironment{smallsession}%
  {\begin{flushleft}
   \setlength{\hsbw}{\linewidth}
   \addtolength{\hsbw}{-\arrayrulewidth}
   \addtolength{\hsbw}{-\tabcolsep}
   \begin{tabular}{@{}|c@{}|@{}}\hline 
   \begin{minipage}[b]{\hsbw}
   \begingroup\footnotesize\mbox{ }\\[-1.8\baselineskip]\begin{alltt}}%
  {\end{alltt}\endgroup\end{minipage}\\ \hline 
   \end{tabular}
   \end{flushleft}}

\newenvironment{spec}%
  {\begin{flushleft}
   \setlength{\hsbw}{\textwidth}
   \addtolength{\hsbw}{-\arrayrulewidth}
   \addtolength{\hsbw}{-\tabcolsep}
   \begin{tabular}{@{}|c@{}|@{}}\hline 
   \begin{minipage}[b]{\hsbw}
   \begingroup\small\mbox{ }\\[-0.2\baselineskip]}%
  {\endgroup\end{minipage}\\ \hline 
   \end{tabular}
   \end{flushleft}}

\newcommand{\memo}[1]%
  {\mbox{}\par\vspace{0.25in}%
   \setlength{\hsbw}{\linewidth}\addtolength{\hsbw}{-1.5ex}%
   \noindent\fbox{\parbox{\hsbw}{{\bf Memo: }#1}}\vspace{0.25in}}

\newcommand{\nb}[1]%
  {\mbox{}\par\vspace{0.25in}%
   \setlength{\hsbw}{\linewidth}\addtolength{\hsbw}{-1.5ex}%
   \noindent\fbox{\parbox{\hsbw}{{\bf Note: }#1}}\vspace{0.25in}}

\newcommand{\comment}[1]{}
\newcommand{\exfootnote}[1]{}
%\newcommand{\ifelse}[2]{#1}
\sloppy
\clubpenalty=100000
\widowpenalty=100000
%\displaywidowpenalty=100000
\setcounter{secnumdepth}{3} 
\setcounter{tocdepth}{3}
\setcounter{topnumber}{9}
\setcounter{bottomnumber}{9}
\setcounter{totalnumber}{9}
\renewcommand{\topfraction}{.99}
\renewcommand{\bottomfraction}{.99}
\renewcommand{\floatpagefraction}{.01}
\renewcommand{\textfraction}{.2}
\font\largett=cmtt10 scaled\magstep1
\font\Largett=cmtt10 scaled\magstep2
\font\hugett=cmtt10 scaled\magstep3

%\input{/homes/shankar/tex/ch-jmacros}

\newenvironment{Defn}[1]{\begin{definition}[#1]\label{defn:#1}}{
\thmbox\end{definition}}
%
\newenvironment{display}{\begin{alltt}\small\tt\vspace{0.3\baselineskip}}{\vspace{0.3\baselineskip}\end{alltt}}
%
%\newcommand{\gets}{\leftarrow}
\title{The Formal Semantics of PVS\thanks{Funded by National Aeronautics
and Space Administration Contract NAS1-18969, Task 11 and National Science
Foundation Grant CCR 9300444. 
}}
\author{
\begin{tabular}{ccc}
Sam Owre & and & Natarajan Shankar\vspace{-.05in}\\
{\small\tt owre@csl.sri.com}& & {\small\tt shankar@csl.sri.com}\\
\end{tabular}\\
URL: \url{http://www.csl.sri.com/sri-csl-fm.html}\\
SRI International\\
Computer Science Laboratory\\
Menlo Park CA 94025 USA
}
\date{Technical Report CSL-97-2R\\August 1997, Revised March 1999}
%\renewcommand{\today}{}
%\renewcommand{\is}{::=}
%\renewcommand{\baselinestretch}{2}
\begin{document}
%\bibliographystyle{alpha}
\pagenumbering{roman}
\maketitle
\setcounter{page}{2}
\mbox{}
\cleardoublepage
\begin{abstract}
\setcounter{page}{3} 
\thispagestyle{plain}

A specification language is a medium for expressing {\em what\/} is
computed rather than {\em how\/} it is computed.  Specification languages
share some features with programming languages but are also different in
several important ways.  For our purpose, a specification language is a
logic within which the behavior of computational systems can be
formalized.  Although a specification can be used to simulate the behavior
of such systems, we mainly use specifications to state and prove system
properties with mechanical assistance.

We present the formal semantics of the specification language of SRI's
Prototype Verification System (PVS)\@.  This specification language is
based on the simply typed lambda calculus.  The novelty in PVS is that it
contains very expressive language features whose static analysis (e.g.,
typechecking) requires the assistance of a theorem prover.  The formal
semantics illuminates several of the design considerations underlying PVS,
particularly the interaction between theorem proving and typechecking.
\end{abstract}
\cleardoublepage
\tableofcontents
%\listwotoffigures
\cleardoublepage
\setcounter{page}{0} 
\pagestyle{headings}
\pagenumbering{arabic}

\chapter{Introduction} \label{introduction} PVS is a system for specifying
and verifying properties of digital hardware and software systems.  The
specification language of PVS is designed to admit succinct, readable, and
logically meaningful specifications.  The PVS specification language is
designed for effective proof construction rather than efficient execution.
The design considerations underlying the language are therefore somewhat
different from those of a corresponding programming language.  For
example, the language contains constructs that can be statically
typechecked only with the assistance of a theorem prover.  This is
acceptable because the PVS specification language is intended for use in
conjunction with powerful support for automated theorem proving.  The
logic of PVS is based on a simply typed higher-order logic with function,
record, and product types, and recursive type definitions.  This type
system is extended with subtypes that are analogous to subsets, and with
dependently typed functions, records, and products.  The resulting type
system has several advantages.  It is possible, for instance, to
statically ensure that all array references are within their respective
array bounds.  PVS specifications are organized into theories that can be
parametric in types as well as individuals.  While the semantics of the
simply typed fragment is straightforward, the extensions such as
subtyping, dependent typing, and (theory-level) parametricity do pose
significant challenges.  This report presents a concise but idealized
definition of the PVS specification language and its intended formal
set-theoretic semantics.  It is neither an overview of the PVS language
nor a guide to the Prototype Verification System (see the PVS user
manuals~\cite{PVS:manuals}).


The primary purpose of the formal semantics is as a useful reference for
the developers and users of PVS\@.  The idealized core of the
specification language as presented here serves as a succinct foundation
for studying the expressive power of the language.  Pertinent questions
about PVS are answered directly by the formal semantics presented here:
\begin{enumerate}
\item\label{one} What is the semantic core of the language, and what is
just syntactic sugar?

\item\label{two} What are the rules for determining whether a given PVS
expression is well typed?

\item\label{three} How is subtyping handled, and in particular, how are
proof obligations corresponding to subtypes generated?

\item\label{four} What is the meaning, in set-theoretic terms, of a PVS expression or
assertion?

\item\label{five} Are the type rules sound with respect to the semantics? 

\item\label{six} Are the proof rules sound with respect to the semantics?

\item\label{seven} What is the form of dependent typing used by PVS, and
what kinds of type dependencies are disallowed by the language?

\item\label{eight} What is the meaning of theory-level parametricity, and
what, if any, are the semantic limits on such parameterization?

\item\label{nine} What language extensions are incompatible with the reference
semantics given here?
\end{enumerate}

Chapter~\ref{conclusion} summarizes the answers to these questions.


\section{Real versus Idealized PVS}

The semantic treatment in this report is incomplete in some important ways.
It does not treat the nonlogical parts of the language.  In particular,
it ignores arithmetic and recursive definitions.  It also omits
abstract datatypes~\cite{Shankar:ADT}.  These will be treated in a future
expanded version. 

The present semantics also makes several idealizations from the real
PVS for 
the purpose of clarity.  While the semantic treatment is not
comprehensive, the idealization of PVS used here is faithful to the
implemented form of PVS.   

\begin{enumerate}
  \item \emph{No name resolution.}  All names must be in fully resolved form
with their theory name and actual parameters.  We regard name resolution as
a convenience provided by the PVS type checker and not an operation with any
semantic relevance.  A technical description of name resolution in PVS will
be given elsewhere.


  \item \emph{No overloading.}   As with name resolution, overloading is
a syntactic convenience with no semantic import. 

\item \emph{No \texttt{IMPORTING}s.}  The importing of theories is a 
hint to  name resolution.  The semantic definition assumes that all
instances of theories declared prior to the present one are visible.   
  \item \emph{Variable declarations ignored.}  All variables must be 
locally declared.   Global variable declarations are regarded as a syntactic
convenience.  
%  \item TCCs are not kept around to be used as lemmas.
  \item \emph{No records.}  These are ignored in the semantic treatment
since product types capture all the semantically essential features of records.
\end{enumerate}


\section{Semantic Preliminaries}

The PVS specification language is based on higher-order logic.  This means
that variables can range over individuals (such as numbers) as well as
functions, functions of functions, and so on.  As is well known, some type
distinction is needed; otherwise, it is easy to obtain a
contradiction by defining the predicate $N(P)$ as $ \neg(P(P))$ so that
both $N(N)$ and $\neg N(N)$ hold.  
In the theory of types~\cite{Church:types}, the universe is stratified
into distinct types 
so that a predicate can  be applied only to a \emph{lower} type
and thus cannot be applied to itself.

Types also serve as a powerful mechanism for detecting syntactic and
semantic errors through typechecking.  This role of types is best
exemplified by their use in various programming languages such as Algol,
Ada, and ML, and is also heavily emphasized in the PVS type
system.

The desirability for strong typing in a specification
logic is not widely accepted.  Fraenkel {\em et
al\/}~\cite{Fraenkel-etal84} express
the opinion that such 
typing is repugnant in a mathematical logic since it constrains
expressiveness by not allowing individuals of differing types to be
treated uniformly.  Lamport~\cite{TLA:TOPLAS94} argues that type
correctness is like any other 
program property and should be established by means of a proof rather than
by syntactic restraints.  Lamport and Paulson~\cite{Lamport&Paulson97}
analyze the tradeoffs between typed and untyped specification languages.  
We claim that
\begin{enumerate}
\item Types impose a useful discipline on the specification.
\item Types lead to easy and early detection of a large class of
syntactic and  semantic errors.
\item Type information is useful in mechanized reasoning.
\end{enumerate}


\begin{comment}
The first typed higher-order logic was the Principia Mathematica (PM) system of
Russell and Whitehead.  They were the first to propose a stratified
hierarchy of types as a means avoiding certain paradoxes.  The PM type
system is quite complicated.  Church proposed his \emph{simple theory of
types} as a higher-order logic based on a simply typed lambda calculus.
Church's system forms the basis for most modern higher-order logics
including that of PVS.  The Automath system of de~Bruijn pioneered the use
of dependent types (analogous to the variant records of Pascal) as a basis
for intuitionistic logic.\footnote{Similar ideas were 
independently proposed by several other researchers.}  Dependent types 
form an important component of the PVS type system.  
Girard and Reynolds, independently extended the typed lambda calculus
to a second-order or \emph{polymorphic} type system so that, for example,
an identity function for each type could be obtained by specializing a
generic identity function to the required type.
\end{comment}

The semantics of a higher-order logic is given by mapping the well-formed
types of the logic to {\em sets\/}, and the well-formed terms of the logic to
{\em elements\/} of the sets representing their type.  The set
constructions we use can be formalized within Zermelo-Fraenkel set theory
with the axiom of choice (ZFC).   The intended interpretation
of a function type in higher-order logic is that it represents the set of
all functions from the set representing the domain type to the set
representing the range types.\footnote{It is only in the {\em standard\/}
model of higher-order logic that the function type is required to
represent the set of all 
functions from the domain set to the range set.  Higher-order logic can be
interpreted in {\em general models\/} where the function type can be
interpreted in any manner as long as it satisfies the various axioms
such as application, abstraction, and extensionality~\cite{Andrews:book}.
Higher-order logic 
is complete with respect to the general models interpretation so that
a statement that is valid in all models is provable.  It is, however,
incomplete with respect to the standard model.}  
PVS also has predicate subtypes that are to
be interpreted over the subsets of the set representing the parent type.

The semantics of PVS will be given by considering a sequence of
increasingly expressive fragments of PVS\@.   
The semantics of each fragment of PVS will be presented in three steps.
The first step is to define a set-theoretic universe containing
enough sets to represent the PVS types.  The second step is to define a
typechecking operation that determines whether a given PVS expression is
well typed.  The third step is to define a semantic function that assigns
a representation in the semantic universe to each well-typed PVS type and
term. 



\begin{comment}
I am not exactly sure where the axiom of choice is really needed.
Since PVS builds in a choice operator via parametric theories, it seems
that this will be needed.
\end{comment}


We first lay out the ZFC set constructions needed for defining the
semantics of PVS.  The base types in PVS consist of the Booleans
\texttt{bool} and the real numbers \texttt{real}\@.  The Booleans can be
modeled by any two-element set, say $\twob$ consisting of the elements
$\zerob$ and $\oneb$, where $\zerob$ is the empty set and the only element
of the set $\oneb$\@.  The real numbers can be captured by means of
Dedekind cuts or Cauchy sequences, and we label this set $\reals$\@.

To define the semantics, we need a universe that contains the sets
$\twob$ and 
$\reals$
and is closed under Cartesian products (written as $X\times Y$)
and power sets (written as $\wp(X)$)\@.  Note that
functions are modeled as \emph{graphs}, that is, sets of ordered pairs,
so that a function type $\funtype{A}{B}$ is represented by a subset of
the powerset $\wp(\mean{A}\times \mean{B})$
of the Cartesian product of the sets $\mean{A}$ and $\mean{B}$
representing $A$ and 
$B$, respectively\@.  A set $F$ that is a subset of $X \times Y$ is the graph of a
function with domain $X$ and range $Y$ if for every $x\in X$ there is a $y\in
Y$ such that $\pair{x, y}\in F$, and whenever $\pair{x, y}\in F$
and $\pair{x, y'}\in F$, we have $y = y'$\@.  For such a set $F$,
$ \textit{Function}(F)$ holds and $ \textit{dom}(F) = X$\@.  
The set of graphs of total functions from a set $Y$ to a set $X$
is represented as $X^Y$\@.   If $F$ is the graph of a function and $t$ an
element in its domain, then
$F(t)$ represents the result of applying the function $F$ to $t$\@.
At the semantic level,  a function $F$ will never be applied to an argument
$t$ outside $\textit{dom}(F)$, because in the PVS language,
a function application is typechecked so that the
argument expression has the same type as the domain type of the
function expression.  

We can model the entire type universe of the simply typed fragment of PVS
by the set $U$, which is defined cumulatively by starting from the base
sets $\twob$ and $\reals$, and including the Cartesian products, the
function spaces, and subsets of previously included sets, at each stage.
Cartesian products are used to model products in PVS, and function spaces
model function types.  Subsets are needed to model predicate subtypes.
It is sufficient to iterate these stages up to the ordinal $\omega$\@.
\begin{Defn}{type universe}
\begin{eqnarray*}
  U_0 & = & \{ {\twob}, {\reals}  \} \\
  U_{i+1} & = & U_i \,\union\, \{X\times Y \vbar X, Y \in U_i\} \,\union\,
                           \{X^Y \vbar X, Y \in U_i \} \,\union\,
                           \bigcup_{X\in U_i}\wp(X)\\
  U_{\omega} & = & \bigcup_{i\in\omega} U_i\\
  U & = & U_{\omega}
\end{eqnarray*}
\end{Defn}


We refer to $U$ as the \emph{basic universe}.\footnote{The inclusion of $X^Y$ in $U$ is actually redundant but aids clarity.}   The semantic definitions
below will assign a set in $U$ to each PVS type and an element
in $\bigcup U$ to each well-typed term of PVS.  The \emph{rank} of a set
$X$ in $U$ is the least $i$ such that $X\in U_i$\@.  The notion of rank
plays an important role in the semantics of dependent types and
parametric theories.

\section{Related Work}

There is a long history of work in specification languages.
Many  ideas similar to those underlying the PVS specification language
also occur in other specification languages.  

The {\em wide-spectrum\/} languages are typically based on set theory or
higher-order logic.  The language VDM is one of the earliest such
specification formalisms~\cite{Jones:VDM}.  It is based on a first-order
logic with partial functions augmented with datatype axioms.  The datatype
theories in VDM include those for finite sets, maps, sequences, and
recursive datatypes such as lists and trees.  VDM has a notion of datatype
invariants that yields a simple form of predicate subtyping.  Operations
on state are specified in terms of pre-condition/post-condition pairs.
Specifications are structured into parameterized modules.  In contrast to
VDM, the PVS language is based on strictly typed higher-order logic with a
built-in notion of predicate subtyping and dependent typing.  The
resulting PVS logic is more compact in that many of the datatypes that are
presented axiomatically in VDM can be defined within PVS.  There is no
built-in notion of state in PVS since it is possible to use the
higher-order logic of PVS to define a variety of state-based formalisms,
including various linear and branching-time temporal logics.  VDM uses a
3-valued logic for the logical connectives in order to deal
with partial functions, whereas PVS uses a classical 2-valued logic and
predicate subtyping to assign a type to a partial function as a total
function on its domain of definition.  Jones~\cite{Jones:VDM} provides
only an informal semantics for VDM.  The RAISE system is a comprehensive
toolset based on the ideas of VDM~\cite{RAISE:book}.

The Z specification language~\cite{Z-understanding} is another
wide-spectrum language based on a typed first-order set theory.  A Z
specification is a collection of schemas consisting of declarations of
types and constants accompanied with invariants.  Z schemas can either
specify datatype invariants or pre-condition/post-condition constraints.
A schema calculus is used to combine schemas using logical connectives.
Spivey~\cite{Z-understanding} presents a formal semantics for Z without
giving a proof system or a soundness proof.  Spivey's treatment of
partial functions in the Z semantics employs the commonly used convention
that $f(a)$ when $a$ is not in the domain of $a$ is some arbitrarily
chosen value.  This is fine for most purposes but can be confusing when
dealing with recursively defined partial functions.  For example, the
definition $\mathit{bad}(x) = 1 + \mathit{bad}(x)$ is everywhere undefined but
admitting it as an axiom leads to an immediate contradiction.  Z also
lacks any mechanism for conservative extensions such as definitional
principles for constants and datatypes so that the consistency of Z
specification has to be demonstrated by exhibiting a model.

Algebraic specification languages like OBJ~\cite{popl85} and
Larch~\cite{Larch:book} provide an 
equational/rewriting framework for specifying datatypes and operations on
datatypes.   OBJ has many of the same theory parameterization mechanisms
as PVS.  The subsort mechanism in OBJ is also similar except that
it is handled by introducing {\em retracts\/} or runtime checks
rather than proof obligations generated by the type checker.
The OBJ logic is quite restricted compared to PVS since it is based
on a first-order, equational framework with an {\em initial\/} semantics
where two ground terms are distinct unless they can be proved equal.
OBJ has very limited support for proof development and is primarily
intended as an executable specification language.

The specification languages that are closer to PVS are those that
accompany various automated proof checking systems.  The closest of these
is {\sc Ehdm}~\cite{EHDM:manuals}, which employs a similar higher-order
logic with subtyping and proof obligation generation.  {\sc Ehdm} lacks
many of the features of PVS: subtyping is restricted to type declarations
and there is no dependent typing.

Higher-order logic is used by other systems such as
HOL~\cite{Gordon&Melham:HOL} and TPS~\cite{andrews84}.
Both HOL and TPS employ simply typed higher-order logic 
without features such as subtyping, dependent typing, or parametric
theories.  
Andrews~\cite{Andrews:book} gives a thorough account of the semantic
aspects of higher-order logic.  The formal semantics of the HOL logic are
carefully outlined (by Pitts) in the book by Gordon and
Melham~\cite{Gordon&Melham:HOL}.

Systems like Coq~\cite{COQ} and Nuprl~\cite{Nuprl-book} are based on
intuitionistic higher-order logics.  Coq allows quantification over types,
whereas Nuprl has quantification over a hierarchy of type universes.  Both
logics admit dependent typing.  The set-theoretic semantics of dependently
typed intuitionistic type theories has been studied by
Dybjer~\cite{DybjerP:indsfm} and Howe~\cite{Howe91,Howe96}.  Not
surprisingly, their semantic treatment of dependent typing is similar to
the one given here but they do not delimit the possible dependencies as is
done with the PVS semantics.  The PVS semantics presented here clearly
specifies the kind of type dependencies that are disallowed in the logic.
Dybjer and Howe also do not address subtyping but do describe the
semantics of language features missing in PVS (type universes in the case
of Howe, and inductive families in the case of Dybjer).  Dybjer does not
identify the universe over which terms and types are interpreted.  Howe
requires an infinite sequence of inaccessible cardinals for his universe
construction.


\section{Outline}

In Chapter~\ref{simple}, we define the syntax and semantics of the simply
typed fragment of PVS.  Type definitions are also introduced in this
chapter along with the definition of definitional equivalence on types.
Chapter~\ref{subtypes} adds subtyping to the simply typed fragment and
specifies the additional type rules and semantic definitions that are
needed.  Chapter~\ref{dependent} extends the language with dependent
function and product types.  Theories and parametric theories are introduced
into the language in Chapter~\ref{theories}.  The type rules and semantics
for conditional expressions and the logical connectives defined using
conditional expressions are introduced in Chapter~\ref{conditionals}.
Chapter~\ref{proof theory} specifies the axioms and inference rules of
PVS\@.
 
\chapter{The Simple Type Theory}\label{simple}

\begin{comment}
The types of expression
are determined from the types of their subexpressions.  In the base case,
the types of atomic expressions must be given in a context.  There are
certain well-formedness constraints on contexts that we now specify.
A context is a sequence of declarations.  Declarations can either
be type declarations of the form $T\ :\ \mathtt{TYPE}$ where $T$ is a symbol,
constant declarations of the form $c\ :\ T$, or variable declarations of
the form $x\ :\ \mathtt{VAR} T$.  
The empty context $\{\}$ is a context.  If $\Gamma$ is a context,
then
\end{comment}

PVS is a strongly typed specification language.  The simply typed fragment
includes types constructed from the base types by the function and product
type constructions, and expressions constructed from the constants and
variables by means of application, abstraction, and tupling.  Expressions
are checked to be well typed under a {\em context\/}, which is a partial
function that assigns a {\em kind\/} (one of \texttt{TYPE},
\texttt{CONSTANT}, or \texttt{VARIABLE}) to each symbol, and a type to the
constant and variable symbols.  We use the metavariables $\Gamma$,
$\Delta$, and $\Theta$ to range over contexts.  The metavariables $A$,
$B$, and $T$ range over PVS type expressions, the metavariables $r$ and
$s$ range 
over symbols (identifiers), the metavariables $x$ and $y$ range over PVS
variables, and the metavariables $a$, $b$, $f$, and $g$ range over PVS
terms.  Given 
a context $\Gamma$ and a symbol $s$, we say that $\Gamma(s)$ is
undefined if $s$ is not declared in $\Gamma$\@.

The \emph{pretypes} of the simple type theory include the \emph{base}
types such as \texttt{bool} and \texttt{real}.  A \emph{function} pretype
from domain pretype $A$ to range pretype $B$ is constructed as
$\funtype{A}{B}$\@.  A \emph{product} pretype of ${\listwo{A}{n}}$ is
constructed as $\tupletype{\listwo{A}{n}}$.
\begin{comment}
We make no
distinction between a type $A$ and the singleton tuple type $[A]$.  We
also disallow a 0-tuple type.
\end{comment}
A \emph{type} is a pretype that has been
typechecked in a given context.  Types in the simple type theory are
simple enough  that the only distinction between pretypes and types is
that the symbols in a type must be appropriately declared in the given
context.

\begin{Eg}{pretypes} 
 \texttt{bool}, \texttt{real},
 $\tupletype
{\mathtt{bool}, \mathtt{real}}$,
 $\funtype{\tupletype{\mathtt{real}, \mathtt{bool}}}{ \mathtt{bool}}$.
\end{Eg}

The {\em preterms\/} of the language consist of the constants, variables,
pairs, projections, applications, and abstractions.  The metavariables $c$
and $d$ range over constants.  Pairs are of the form $(a_1, a_2)$ where
each $a_i$ is a preterm.  Applications have the form $f\ a$ where $f$ and
$a$ are preterms.  A pair projection is an expression of the form
$\proj{i}\ a$, where $i\in\{1, 2\}$\@.
\begin{comment}
As with types, we make no
distinction between the singleton tuple preterm $(a)$ and the preterm
$a$.
\end{comment}
Lambda abstractions have the form $\lambda (x : T): a$, where $T$
is a pretype and $a$ is a preterm.
\begin{comment}
$\lambda (x_1 : T_1, \ldots,
x_n : T_n): a$, where each $T_i$ is a type, each $x_i$ is a symbol so that
$x_i\not\equiv x_j$ for $i\neq j$, and $a$ is a preterm.
\end{comment}
Parentheses
are used for disambiguation.  A \emph{term} is a preterm that has been
typechecked in a given context.

\begin{Eg}{preterms}
\texttt{TRUE}, $\neg~\mathtt{TRUE}$,
$\lambda~(x~:~\mathtt{bool}):~\neg(x)$,\hfill\\
$\proj{2}~(\mathtt{TRUE}, \mathtt{FALSE})$,
$(\mathtt{TRUE},  \lambda~(x~:~\mathtt{bool})~:~\neg~(\neg~x))$.
\end{Eg}

\section{Contexts}

A context is 
 a sequence of
declarations, where each declaration is either a type declaration
$s\ :\ \mathtt{TYPE}$, a constant declaration $c\ :\ T$ where $T$ is a type,
or a variable declaration $x\ :\ \mathtt{VAR}\ T$.
Preterms and pretypes are typechecked with respect to a given context.  
The empty context is represented as $\{\}$\@.  
The well-formedness rules
for contexts are presented below.
A context can also be applied as a
partial function so that for a symbol $s$ with declaration $D$, $(\Gamma, s
: D)(s) = D$ and 
$(\Gamma, s : D)(r) = \Gamma(r)$ for $r\neq s$.
If $s$ is not declared in $\Gamma$, then $\Gamma(s)$ is undefined. 
If $\Gamma$ is a context, then
for any symbol $s$, the kind of the symbol $s$ in $\Gamma$ is given by $\textit{kind}(\Gamma(s))$.  If the kind of $s$ in $\Gamma$ is \texttt{CONSTANT} or
\texttt{VARIABLE}, then the $ \textit{type\/}(\Gamma(s))$ is the type assigned
to $s$ in $\Gamma$.  
%\memo{Variable declarations are unrelated to variable declarations in
%PVS and correspond to binding declarations.}
\newcommand{\coln}[2]{#1\ : #2}

\begin{Eg}{context}\hfill\\
$ \mathtt{bool}\ :\  \mathtt{TYPE},\hspace{.2cm} \coln{ \mathtt{TRUE}}{
\mathtt{bool}},\hspace{.2cm} \coln{ \mathtt{FALSE}}{
\mathtt{bool}},\hspace{.2cm} \coln{x}{ \mathtt{VAR}\ \funtype{\tupletype{
\mathtt{bool}, \mathtt{bool}}}{ \mathtt{bool} }}$ 
\end{Eg}


\section{Type Rules}

The type rules for the simple type theory are given by a recursively
defined partial function
$\tau$ that assigns
\begin{enumerate}
\item  A type $\tau(\Gamma)(a)$ to a preterm $a$ that
is well typed with
respect to a context $\Gamma$. 
\item  The keyword \texttt{TYPE} as the result of $\tauGamma{A}$
when $A$ is a  well-formed type under context $\Gamma$\@.

\item  The keyword \texttt{CONTEXT} as the result of
$\tauGamma{\Delta}$ when $\Delta$ is a well-formed context
under context $\Gamma$\@.  The context $\Gamma$ is empty for the
simply typed fragment so that typechecking is always invoked as
$\tau()(\Gamma)$\@. 
\end{enumerate}
Otherwise, $\tau$ is undefined in the case of an ill-typed preterm or an
ill-formed type or context. 

The type rules are given by the recursive definition for $\tau$\@.
Typechecking in PVS assigns a ``canonical'' type to a preterm.
Customarily, type rules are presented as inference rules, but
a functional presentation is more appropriate for PVS since
\begin{enumerate}
\item The type assignment is deterministic.  A term can, in
general, though not in the simply typed fragment,  be assigned
a number of types but it always has at most one canonical type.  

\item The soundness proof need only show that the meaning of the term
is an element of the meaning of its canonical type.  Thus, only
the canonical type derivation for a term has to be shown sound and not
every valid type derivation. 

\item The meaning of a term is therefore given by recursion on the term
itself and not on its typing derivation.  There is no need to show
separately that this meaning is {\em coherent\/}, that is, independent of
the typing derivation.
\end{enumerate}
  A functional presentation of the type rules also
leads to natural and straightforward soundness arguments.  
Note that the well-formedness rules for contexts and types are trivial
in the simply typed situation but become more meaningful when the type
theory is extended.  Note also that in the type rules for expressions and
types, the well-formedness of the relevant context is not explicitly
checked.  These rules do preserve the well-formedness of the context
in each recursive call so that if the initial context is well formed, then
so is every intermediate one.    
\begin{Defn}{type rules}
\begin{eqnarray*}
  \tau()(\{\}) & = & \ttcontext \\
%
  \tau()(\Gamma, s\ :\ \tttype{}) & = & \ttcontext, \mbox{ if }
  \Gamma(s) \mbox{ is undefined}\\& &
  \mbox{and } \tau()(\Gamma) = \ttcontext\\
%
  \tau()(\Gamma, c : T) & = & \ttcontext, \mbox{ if }
   \Gamma(c) \mbox{ is undefined,}\\ & & 
\tauGamma{T} =
      \tttype{},\\
   & & \mbox{and } \tau()(\Gamma) = \ttcontext\\
%
  \tau()(\Gamma, x : \ttvar{}\ T) & = & \ttcontext, \mbox{ if }
  \Gamma(x) \mbox{ is undefined,}\\ & & 
\tauGamma{T}
    = \tttype{},\\
    & & \mbox{and } \tau()(\Gamma) = \ttcontext\\ 
%
  \tauGamma{s} & = & \tttype{}, \mbox{ if } \textit{kind\/}(\Gamma(s)) =
\tttype{}\\
%
  \tauGamma{\funtype{A}{B}} & = & \tttype{}, \mbox{ if } \tauGamma{A} =
     \tauGamma{B} = \tttype{}\\
%
  \tauGamma{\tupletype{\listwo{A}{n}}} & = & \tttype{}, \mbox{ if }
     \tauGamma{A_i} = \tttype{} \mbox{ for } 1\leq i\leq 2 \\
%
  \tau(\Gamma)(s) & = &  \textit{type\/}(\Gamma(s)),\\
                    & &  \mbox{ if } 
\textit{kind}(\Gamma(s)) \in \{ \ttconstant{}, \ttvariable \} \\
%
 \tau(\Gamma)(f\ a) & = & B, \mbox{ if } \tau(\Gamma)(f) = \funtype{A}{B}
\mbox{ and } \tau(\Gamma)(a) = A \\
%
\tauGamma{\lambda (x : T): a} & = & \funtype{T}{\tau(\Gamma, x : \ttvar{}\ T)(a)},
\mbox{ if } \Gamma(x) \mbox{ is undefined}\\& & 
  \mbox{and }\tauGamma{T} = \tttype{}\\
%\tau(\Gamma)(\lambda (x_1 : T_1, \ldots, x_n : T_n): a) & = &
%         \begin{array}[t]{l}
%	   [{\tupletype{\listwo{T}{n}}} \aro \\
%   \tau(\Gamma, x_1 : \ttvar{}\ T_1, \ldots, x_n : \ttvar{}\ T_n)(a)]
%	 \end{array} \\ & &
%\mbox{if } 
%   \tauGamma{T_i} = \tttype{}, \mbox{ for } 1\leq i\leq n,\\
%& & \mbox{and  } x_i \not\equiv x_j, \Gamma(x_i) \mbox{ is undefined},\\
% & &  \mbox{ for } 1\leq i, j \leq n, i\neq j \\
%
\tau(\Gamma)((a_1, a_2)) & = &
    [\tau(\Gamma)(a_1),  \tau(\Gamma)(a_2))]\\
%
\tau(\Gamma)(\proj{i}\ a) & = & T_i, \mbox{ where }\\ & & 
 \tau(\Gamma)(a) = [T_1, T_2] %\mbox{ for some } n, n\geq i
\end{eqnarray*}
\end{Defn}

 In the type rule for lambda abstraction, the constraint that
$\Gamma(x)$ must be undefined can be satisfied by suitably renaming the
bound variable since we treat terms as equivalent modulo the renaming of
bound variables.  

\begin{comment}
The above definition suggests that projection applications should
not be treated as applications.
\end{comment}

\begin{Eg}{type rules}  Let $\Omega$ label the context
$\coln{ \mathtt{bool}}{\tttype{}}$,
            $\coln{ \mathtt{TRUE}}{\mathtt{bool}}$,
            $\coln{ \mathtt{FALSE}}{ \mathtt{bool}}$
\begin{eqnarray*}
\tau()(\{\}) & = & \ttcontext\\
\tau()(\Omega)
   & = & \ttcontext\\
\tau(\Omega)(\funtype{\tupletype{ \mathtt{bool}, \mathtt{bool}}}{
\mathtt{bool}}) & = & \tttype{}\\
\tau(\Omega)(( \mathtt{TRUE}, \mathtt{FALSE})) & = & \tupletype{ \mathtt{bool},
\mathtt{bool}}\\
\tau(\Omega)( \proj{2} ( \mathtt{TRUE}, \mathtt{FALSE})) & = & \mathtt{bool}\\
\tau(\Omega)(\lambda (\coln{x}{ \mathtt{bool}}) : \mathtt{TRUE}) & = &
\funtype{ \mathtt{bool}}{ \mathtt{bool}}
\end{eqnarray*}
\end{Eg}

\section{Semantics}

Recall that a preterm $a$ with a type assigned by $\tau$  under context $\Gamma$
is said to be a term of type $\tau(\Gamma)(a)$
in the context $\Gamma$.  If $\gamma$ 
is an \emph{assignment} for the symbols declared in context $\Gamma$, the
semantics of the simple type theory of PVS 
is given by  mapping  a type $T$ to a (possibly empty) set
$\mathcal{M}(\Gamma\vbar \gamma)(T)$, 
and a term $a$ with assigned type $T$ to an element of the set $\mathcal{
M}(\Gamma\vbar \gamma)(T)$ in the basic universe $U$\@. 
The assignment $\gamma$ is a list of bindings of the form $\{s_1\gets
t_1\}\ldots \{s_n\gets t_n \}$\@.  The application of an assignment
$\gamma$ to a symbol $s$ is such that $\gamma \{s \gets t\} (s)$ is $t$,
whereas $\gamma \{r \gets t\}(s)$ is $\gamma(s)$ when $r\not\equiv s$\@.

The  meaning function $\mathcal{M}$ returns the meaning of
a well-formed type $A$  
and a well-formed expression $a$ in the context $\Gamma$ under an
assignment $\gamma$ as $\mathcal{M}(\Gamma\vbar \gamma)(A)$ 
and 
$\mathcal{M}(\Gamma\vbar \gamma)(a)$,respectively. 
\begin{comment}
The base types 
\texttt{bool} and \texttt{real} are mapped to $\twob$ and $\reals$,
respectively.
\end{comment}

The meanings of type names, constants, and variables
declared in $\Gamma$ are obtained from the assignment $\gamma$\@.  A
function type is mapped to the 
corresponding function space.  A product type is mapped to the corresponding
Cartesian product.
\begin{comment}
an $n$-tuple is represented by an $(n+1)$-fold
Cartesian product terminated by the sole element $\zerob$ of the set \oneb\@.
This representation  distinguishes a 3-tuple
$\tupletype{\ttbool, \ttbool, \ttbool}$ from the 
2-tuple $\tupletype{\ttbool, \tupletype{\ttbool, \ttbool}}$\@.
\end{comment}
An application term is interpreted by 
means of set-theoretic function application.  A lambda abstraction yields the
graph of the corresponding function.  A pair expression is mapped
to the corresponding set-theoretic ordered pair.    
\begin{Defn}{meaning function}
\begin{eqnarray*}
%% \begin{comment}
%%   \Mgamma{ \mathtt{bool}} & = & {\twob} \\
%%   \Mgamma{ \mathtt{real}} & = & {\reals}\\
%%   \Mgamma{ \mathtt{TRUE}} & = & {\oneb}\\
%%   \Mgamma{ \mathtt{FALSE}} & = &{\zerob}\\
%% \end{comment}
  \mathcal{M}(\Gamma\vbar \gamma)(s) & = &  \gamma(s),\\
                    & &  \mbox{ if } 
\textit{kind}(\Gamma(s)) \in \{ \tttype{}, \ttconstant{}, \ttvariable \}
\\
%
  \mathcal{M}(\Gamma\vbar \gamma)(\funtype{A}{B}) & = & \Mgamma{B}^{\Mgamma{A}}\\
%
  \Mgamma{\tupletype{\listwo{T}{n}}} & = &  \Mgamma{T_1} \times %%\ldots\times
\Mgamma{T_2} %\times {\oneb}
\\
%
\mathcal{M}(\Gamma\vbar \gamma)(f\ a) & = & (\mathcal{M}(\Gamma\vbar \gamma)(f))(\mathcal{M}(\Gamma\vbar \gamma)(a)) \\
%
\Mgamma{\lambda (x : T): a} & = & \{\tuple{y, z} \vbar \begin{array}[t]{l}
						       y\in\Mgamma{T},\\
                					z = \mathcal{M}(\Gamma, x: \ttvar{}\ T\vbar \gamma\{x\gets y\})(a)\}
						       \end{array}\\
%               
%\mathcal{M}(\Gamma| \gamma)(\lambda (x_1 : T_1, \ldots, x_n : T_n): a) & = &
%  {\begin{array}[t]{l}\{ \tuple{\tuple{\listwo{y}{n}, {\zerob}}, z} | \\
%					y_1\in \Mgamma{T_1}\\
%                                        \vdots\\
%                                        y_n\in \Mgamma{T_n}\\
%                                   z = \mathcal{M}\left(
%        \begin{array}{l}
%  \Gamma, x_1 : \ttvar{}\ T_1, \ldots, x_n : \ttvar{}\ T_n;\\
%  \gamma\{x_1 \gets y_1\}\ldots \{x_n\gets y_n\}
%	\end{array}\right)(a)\} 
%				      \end{array}}\\
%
\Mgamma{(\listwo{a}{n})} & = & \tuple{\Mgamma{a_1},  \Mgamma{a_2}} \\
%
\Mgamma{\proj{i}\ a} & = & t_i, \mbox{ where } \Mgamma{a} =
\tuple{\listwo{t}{n}}
\end{eqnarray*}
\end{Defn}

\begin{Eg}{meaning function}
 Let $\omega$ be an assignment for the context $\Omega$ in
Example~\ref{eg:type rules}, of the form
$$\{\ttbool\gets \twob\}\{ \mathtt{TRUE}\gets \oneb\}\{
\mathtt{FALSE}\gets \zerob\}$$
then
\begin{eqnarray*}
\mathcal{M}(\Omega\vbar \omega)(\tupletype{\ttbool, \ttbool}) & = &
\twob\times\twob\\
%
\mathcal{M}(\Omega\vbar \omega)(( \mathtt{TRUE}, \mathtt{FALSE})) & = &
\tuple{\oneb, \zerob}\\
%
\mathcal{M}(\Omega\vbar \omega)(\lambda (x : \ttbool): \mathtt{TRUE}) & = &
\{\tuple{\zerob, \oneb}, \tuple{\oneb, \oneb}\}
\end{eqnarray*}
\end{Eg}

\begin{Defn}{satisfaction}
A context assignment $\gamma$ is said to \emph{satisfy} a context $\Gamma$
(in symbols $\gamma\models \Gamma$) 
iff
\begin{enumerate}
 \item $\gamma(\ttbool) = {\twob},$
 \item $\gamma(\tttrue) = {\oneb},$
 \item $\gamma(\ttfalse) = {\zerob},$
\item $\gamma(s)\in U$ whenever $ \itkind{}(\Gamma(s)) = \tttype$, and
 \item $\gamma(s) \in \Mgamma{ \ittype{}(\Gamma(s))}$\hfill\\\hspace*{1in}
whenever $ \itkind{}(\Gamma(s)) \in \{ \ttconstant,
\ttvariable \}$.
\end{enumerate}
\end{Defn}
\begin{Eg}{satisfaction}\hfill\\\begin{enumerate}
\item The assignment $\omega$ satisfies context $\Omega$.

\item The assignment $\omega \{ \mathtt{one}~\gets~{\oneb}\}\{
\mathtt{zero}~\gets~\zerob\}$ satisfies the context
$\Omega,~\mathtt{one~:~TYPE},~\mathtt{zero~:~one}$\@.  
\end{enumerate}
\end{Eg}

We need one useful proposition that asserts that typing judgements are not
invalidated when the context is extended.
\begin{prop}\label{context-weakening} %where is this used? NSH(9.6.95)
If $\tau()(\Gamma) = \tau()(\Gamma') = \ttcontext$ and $\Gamma$ is
a prefix of $\Gamma'$, then for all pretypes $A$, $\tauGamma{A} = \tttype$
implies $\tau(\Gamma')(A) = \tttype$, and for all preterms $a$,
$\tauGamma{a} = A$ implies $\tau(\Gamma')(a) = A$\@.  
\end{prop}

The following theorems follow from the induction suggested by the
definitions of $\tau$ and $\mathcal{M}$.  The first of these
is straightforward and is given without proof.
\begin{theorem}[type construction]  \label{type construction}
If $\tau()(\Gamma) = \ttcontext$ and $\tauGamma{a} = A$, then
$\tauGamma{A} = \tttype$.
\end{theorem}
%
\begin{theorem}[type soundness]\label{simple-type-semantics}
 If $\tau()(\Gamma) = {\ttcontext}$, 
   $\gamma$ satisfies $\Gamma$, and
   $\tauGamma{A} = {\tttype{}} $, then 
$\Mgamma{A} \in U$\@.
\end{theorem}
\begin{proof}
The proof is by induction on the structure of the pretype $A$.
Recall that if $X\in U$, then for some $i$, $X \in U_i$\@.  
This yields three cases:
\begin{enumerate}
\item $A \equiv s$: By Definition~\ref{defn:type rules},
$\Gamma(s)$ is defined and $\itkind(\Gamma(s))$ is $\tttype$\@.
Then by Definition~\ref{defn:meaning function}, $\Mgamma{s}$
is $\gamma(s)$, and by Definition~\ref{defn:satisfaction},
$\gamma(s)\in U$\@.

\item $A\equiv \funtype{B}{C}$: We then have that $\tauGamma{B} =
\tauGamma{C} = \tttype$\@.  Letting $X$ label $\Mgamma{B}$, and $Y$ label
$\Mgamma{C}$, we have by the induction hypothesis that $X\in U$ and $Y\in
U$\@.  Let $j$ be the least rank such that $\Mgamma{B}\in U_j$ and
$\Mgamma{C}\in U_j$\@.  By Definition~\ref{defn:meaning function},
$\Mgamma{A} = Y^X$, and hence $\Mgamma{A} \in U_{j+1}$ by
Definition~\ref{defn:type universe}\@.

\item $A \equiv \tupletype{\listwo{A}{n}}$: Again by
Definition~\ref{defn:type rules} and the induction hypothesis, we have
for each $i \in \{1, 2\}$, that $\Mgamma{A_i}\in U$\@.  Let $j$ be the least
rank such that for $i\in \{1, 2\}$, $\Mgamma{A_i}\in U_j$\@.  Then,
it is easy to verify from Definition~\ref{defn:type universe} that
$\Mgamma{A}\in U_{j + 1}$\@.  
\end{enumerate}
\end{proof}

\begin{theorem}[term soundness]\label{simple-term-semantics}
 If $\tau()(\Gamma) = \ttcontext$, 
   $\gamma$ satisfies $\Gamma$, and
   $\tauGamma{a}$ is defined and equal to $A$, then 
$\Mgamma{a} \in \Mgamma{A}$.  
\end{theorem}
\begin{proof}
The proof is by induction on the structure of preterms.
\begin{enumerate}
\item $a\equiv s$: By Definition~\ref{defn:type rules},
we have that $\ittype(\Gamma(s)) = A$\@.  By
Definitions~\ref{defn:meaning function}
and~\ref{defn:satisfaction}, we have that $\Mgamma{a} = \gamma(s)$
and $\gamma(s) \in
\Mgamma{A}$\@.
\item $a \equiv (f\ b)$:  By Definition~\ref{defn:type rules},
$\tauGamma{f} = \funtype{B}{A}$, and $\tauGamma{b} = B$, for some
$B$ such that $\tauGamma{B} = \tttype$\@.  
Let $\Mgamma{A}$ be $X$ and $\Mgamma{B}$ be $Y$, then 
by Definitions~\ref{defn:type rules}
and~\ref{defn:meaning function}, and the induction hypothesis, we have
$\Mgamma{f} \in X^Y$ and $\Mgamma{b}\in Y$\@.  It therefore follows
by Definition~\ref{defn:meaning function} that $\Mgamma{(f\ b)} =
(\Mgamma{f})(\Mgamma{b})$, and hence $\Mgamma{(f\ b)} \in X$\@.

\item $a \equiv (\lambda (x : C): b)$:  By
Definition~\ref{defn:type rules}, we have that
$\tauGamma{a}$ is $\funtype{C}{B}$, where
$\tau(\Gamma, x : \ttvar~C)(b)$ is $B$\@.  
Let $X$ be $\Mgamma{C}$, and $Y$ be
$\mathcal{M}(\Gamma, x: \ttvar~C \vbar \gamma\{x\gets u\}))(B)$\@.  
By the induction hypothesis, we have that for
any $u\in Y$, 
$\mathcal{M}(\Gamma, x: \ttvar~C \vbar \gamma\{x\gets u\})(b)
\in X$\@.  
Since $\Mgamma{a}$ is $\{\pair{u, v} \vbar u\in X, v = \mathcal{M}(\Gamma, x:
\ttvar~C \vbar \gamma\{x\gets u\})(b)\}$,
we have that $\Mgamma{a} \in X^Y$\@.

\item $a \equiv (a_1, a_2)$: By Definition~\ref{defn:type rules},
$\tauGamma{a} = \tupletype{\listwo{A}{n}}$, where $\tauGamma{a_i} = A_i$ for
$i\in \{1, 2\}$\@.   By the induction hypothesis, $\Mgamma{a_i}\in
\Mgamma{A_i}$ for $i\in \{1, 2\}$\@.  By
Definition~\ref{defn:meaning function}, $\Mgamma{a} =
\pair{\Mgamma{a_1},\Mgamma{a_2}}$ and hence $\Mgamma{a}$ is
an element of $\Mgamma{A}$ which is
$\Mgamma{A}\times\Mgamma{A_n}$\@.

\item $a\equiv \proj{i}~b$: In this case, we know by
Definition~\ref{defn:type rules} that $\tauGamma{b} =
\tupletype{\listwo{A}{n}}$
with $i\in \{1, 2\}$, and $\tauGamma{a} = A_i$\@.
By the induction hypothesis, $\Mgamma{b} = \pair{t_1, t_2}$,
and by Definition~\ref{defn:meaning function}, $\Mgamma{a} = t_i$
and $\Mgamma{\tauGamma{b}} =
\Mgamma{A_1}\times\Mgamma{A_2}$,
hence $\Mgamma{a}\in \Mgamma{A_i}$\@.  
\end{enumerate}
%%\memo{What about tuples/projections?}
\end{proof}

These three theorems (\ref{type construction},
\ref{simple-type-semantics}, and \ref{simple-term-semantics}) are the key
invariants that must be satisfied by the semantics when the language is
extended below with type definitions, subtypes, dependent types, and
parametric theories.

\section{Some Syntactic Operations}

We first define the operation of collecting the free variables of a
term $a$ in a given context $\Gamma$ as $FV(\Gamma)(a)$, and then
define the operation of substitution.
\begin{Defn}{free variables}
\begin{eqnarray*}
  FV(\Gamma)(s) & = & \left\{\begin{array}{l}
			      \{s \}, \mbox{ if } \textit{kind}(\Gamma(s))
= \ttvariable\\
                              \emptyset, \mbox{ otherwise}
			    \end{array}\right.\\
%
  FV(\Gamma)(f\ a) & = & FV(\Gamma)(f) \union FV(\Gamma)(a) \\
%
  FV(\Gamma)(\lambda (x : T) : a) & = &
     FV(\Gamma, x : \ttvar{}\ T)(a)
- \{x\} \\
%
 FV(\Gamma)((\listwo{a}{n})) & = & FV(\Gamma)(a_1)\union
FV(\Gamma)(a_2)\\
%
 FV(\Gamma)( \proj{i}\ a) & = & FV(\Gamma)(a)
\end{eqnarray*}
\end{Defn}
%
\begin{Defn}{substitution}
\begin{eqnarray*}
  s[a_1/x_1,\ldots, a_n/x_n] & = & \left\{\begin{array}{l}
					    a_i, \mbox{ if for some minimal } i,
s \equiv x_i\\
                                            s, \mbox{ otherwise}
					  \end{array}\right.\\
%
  (f\ a)[a_1/x_1,\ldots, a_n/x_n] & = &
     (f[a_1/x_1,\ldots, a_n/x_n]\\&&
       \hspace*{.5in} a[a_1/x_1,\ldots, a_n/x_n]) \\
%
  (\lambda (y : T): a)[a_1/x_1,\ldots, a_n/x_n] & = &
    (\lambda (y':T) :
     \ a[y'/y, a_1/x_1,\ldots, a_n/x_n]), \\ & & 
%
\mbox{where  } y' \mbox{ is a fresh variable}\\
%
 (\listwo{b}{m})[a_1/x_1,\ldots, a_n/x_n] & = &
     (b_1[a_1/x_1,\ldots, a_n/x_n],\\&&
       \hspace*{.5in}b_2[a_1/x_1,\ldots, a_n/x_n])\\
%
  ( \proj{i}\ a)[a_1/x_1,\ldots, a_n/x_n] & = &
   ( \proj{i}\ a[a_1/x_1,\ldots, a_n/x_n])
\end{eqnarray*}
\end{Defn}

Recall that terms are treated as syntactically equivalent modulo
alpha conversion.  
The above definitions must be extended as more 
features are added to the language. 

\section{Type Definitions}

%%\memo{See ADT section for recursive datatypes.}

Here we enrich contexts so that type symbols may have definitions.  PVS
does not allow recursive type definitions\footnote{For the moment, we are
not considering the PVS \texttt{DATATYPE} mechanism, which is a form of
recursive type definition~\cite{Shankar:ADT}.  Recursive datatypes in
the context of the HOL proof checking system are described by
Melham~\cite{Melham89}.}  so a type declaration/definition in a context
may use only the symbols declared in the prior part of the context.  The
main difference in the extended language is that type names can have
definitions.  In such cases, the definitions rather than the type names
are used to determine the actual type of an expression.  In other words,
two type expressions are treated as the same if they are
\emph{definitionally equivalent}\@.  Most other specification languages
tend to employ the weaker notion of \emph{name equivalence} where
syntactically different types are treated as distinct even when their
definitions coincide.

To accommodate type definitions, a context can
contain type declarations of the form $s\ :\ \mathtt{TYPE} = T$, where $T$
is a type.  If context $\Gamma$ contains such a declaration for $s$, then
$ \itdef{}(\Gamma(s))$ is $T$\@.  To extend $\tau$ to handle type
definitions under definitional equivalence, we must ensure that $\tau$
returns the canonical form of a type where all defined types have been
replaced by their definitions.  The operation $\delta(\Gamma)(T)$ returns
the expanded form of a type relative to the context $\Gamma$\@.
%
\begin{Defn}{expanded type}
\begin{eqnarray*}
  \delta(\Gamma)(s) & = & s, \mbox{ if } \textit{definition\/}(\Gamma(s))
\mbox{ is empty}\\
% 
  \delta(\Gamma)(s) & = & \delta(\Gamma)(
\textit{definition\/}(\Gamma(s))), 
\mbox{ if } \textit{definition\/}(\Gamma(s))
\mbox{ is nonempty} \\
%
 \delta(\Gamma)(\funtype{A}{B}) & = &
      \funtype{\delta(\Gamma)(A)}{\delta(\Gamma)(B)} \\
%
 \delta(\Gamma)(\tupletype{\listwo{T}{n}}) & = &
     \tupletype{\delta(\Gamma)(T_1),  \delta(\Gamma)(T_2)}
\end{eqnarray*}
\end{Defn}


The typing rules are augmented to return the type in expanded form.
The main issue here is to determine that the definition part of a type
declaration in a context is well formed relative to the preceding context.
We also need to ensure that $\tau$ returns the expanded form of the type
corresponding to a preterm.
%
\begin{Defn}{type rules with type definitions}
\begin{eqnarray*}
    \tau()(\Gamma, s\ :\ \tttype{} = T) & = & \ttcontext, \mbox{
if }  \Gamma(s) \mbox{ is undefined}, \\ & &
\tau()(\Gamma) = \ttcontext,\\ & & 
  \mbox{and }     \tauGamma{T} = \tttype{} \\
%
%   \tau()(\Gamma, c : T = a) & = & \ttcontext, \mbox{ if } \tauGamma{T} =
%      \tttype{} \mbox{ and } \tauGamma{a} = \delta(\Gamma)(T)\\
%
    \tau(\Gamma)(s) & = &  \delta(\Gamma)(\textit{type\/}(\Gamma(s))),\\
                    & &  \mbox{ if } 
\textit{kind}(\Gamma(s)) \in \{ \ttconstant{}, \ttvariable \} 
\end{eqnarray*}
\end{Defn}


Note that the $\delta$ operator is idempotent, and $\tauGamma{a}$
for a term $a$ always returns an expanded type, that is, 
$\delta(\tauGamma{a}) = \tauGamma{a}$\@.  

We do not need to update the definition of ${\mathcal M}$ from
Definition~\ref{defn:meaning function} since the syntax for terms is
unchanged, 
but we do need
to revise the notion of a satisfying context assignment (from 
Definition~\ref{defn:satisfaction}) to respect the type
definitions.
\begin{Defn}{satisfaction with type definitions}
An assignment $\gamma$ satisfies a context $\Gamma$ if
in addition to the conditions in Definition~\ref{defn:satisfaction},
whenever $ \textit{kind}(\Gamma(s)) = \tttype{}$ and
$\textit{definition}(\Gamma(s))$ (abbreviated as $T$) is nonempty, then
$\gamma(s) = \Mgamma{ T}$.
\end{Defn}

Theorems~\ref{type construction} and~\ref{simple-type-semantics}
and~\ref{simple-term-semantics} continue to hold under these extensions,
and the proofs are easily adapted to the modified definitions.  

\begin{Eg}{type definition}
Let $\Omega'$ be the context\\ $\Omega,
\mathtt{boolop}:\mathtt{TYPE}=\funtype{\tupletype{\mathtt{bool}, \mathtt{bool}}}{\ttbool},
\vee:\mathtt{boolop}$\@.  Then
\begin{eqnarray*}
\tau()(\Omega') & = & \mathtt{CONTEXT}\\
%
 \delta(\Omega')( \mathtt{boolop})  & = & \funtype{\tupletype{ \mathtt{bool},
\mathtt{bool}}}{\ttbool},\\
 \tau(\Omega')( \vee) & = & \funtype{\tupletype{ \mathtt{bool}, \mathtt{bool}}}{\ttbool}
\end{eqnarray*}
\end{Eg}

\section{Summary}

We have defined the simply typed fragment of PVS by introducing the syntax
for pretypes and preterms, the type rules and semantics for well-formed
contexts, types, and terms.  The type rules are presented in a novel
functional style where each well-formed context is assigned the label {\tt
CONTEXT}, each well-formed type is assigned the label {\tt TYPE}, and each
well-formed term is assigned a canonical type.  The semantics takes a
satisfying assignment for a context and maps a well-formed type to a set
and a well-formed term to an element of the set corresponding to its
canonical type.  We then defined the syntactic operations of collecting
the free variables in an expression and for substituting terms for
variables in an expression.

The simple type theory is then extended with type definitions.  With this
extension, two type expressions are treated as equivalent if they are
identical after all type definitions have been expanded.  The operation
$\delta$ returns the expanded form of a given type expression.


\chapter{Adding Subtypes}\label{subtypes}

Subtyping is one of the main features of the PVS specification
language.\footnote{The form of subtyping used in PVS is derived from a
suggestion of Friedrich von~Henke.}  Subtyping in PVS corresponds to the
set-theoretic notion of a subset.  It raises several delicate issues that
were absent in the language presented thus far.  In the simply typed
fragment, each type corresponds to a set of values that is somehow
structurally different from the set of values for another type so that a
term has at most one type.  Subtyping makes it possible to introduce the
natural numbers as a subtype of the reals, and to treat the primes, the
even numbers, and the odd numbers as subtypes of the natural numbers.
With subtyping, a term can obviously have several possible types, but the
type-checking function $\tau$ may return only a single type.  We constrain
$\tau$ to return a natural canonical type of an expression that is given
by the declarations of the symbols in the expression.  If the expression
is used in a context where the expected type is a supertype of its
canonical type, then the type correctness is straightforward.  If the
expected type is a subtype that is {\em compatible\/} with the canonical
type of the expression, then typechecking generates proof obligations
asserting that the expression satisfies the predicate constraints imposed
by the expected type.  Two types are compatible if they have equivalent
maximal supertypes.  Type equivalence in the presence of subtypes is not a
simple notion.  Subtyping also introduces the possibility of types being
empty.  Typed lambda calculi with possibly empty types have been studied
by Meyer, Mitchell, Moggi, and Statman~\cite{MMMS:empty}.  This chapter
introduces predicate subtypes and defines the notions of compatibility and
type equivalence prior to presenting the type rules and semantics.


We restrict our attention to contexts $\Gamma$ that extend
the declarations:
\begin{alltt}
 bool : TYPE,
 TRUE : bool,
 FALSE : bool,
 boolop : \(\funtype{\tupletype{\ttbool, \ttbool}}{\ttbool}\),
 \(\neg\) : \(\funtype{\ttbool}{\ttbool}\),
 \(\vee\) : boolop,
 \(\wedge\) : boolop,
 \(\supset\) : boolop
\end{alltt}
We will abuse PVS notation to employ the customary infix forms of
operations like $\vee$, 
$\wedge$, and $\supset$.    The pretype corresponding to a predicate
subtype has the form $\{x : T \vbar a\}$ where $x$ is a symbol, $T$ is a
pretype, and $a$ is a preterm.   
A \emph{predicate type} in PVS is a function type where the range is the
primitive type \texttt{bool}.  A predicate is a term that has a predicate
type.  If $a$ is a term of type \texttt{bool}, then 
we can define the subtype $\{ x : T \vbar a\}$ consisting of those
elements $e$ of $T$ satisfying $a[e/x]$ ($e$ substituted for $x$ in $a$).
Since the elements of the subtype $\{ x : T \vbar a\}$ satisfy the predicate $
\lambda (x  : T): a$, 
we call this type a {\em predicate subtype\/} to distinguish it from other
forms of subtyping.  
Universal quantification $\forall (x : T): a$ is just an abbreviation
for the term $(\lambda (x : T): a) = (\lambda (x : T): \mathtt{TRUE})$\@.  
Although we use the equality  predicate in the definition of universal
quantification and in the definitions below, the actual introduction of
equality is deferred to a later section following the introduction of
parametric theories.  The equality between PVS terms of function type
is to be interpreted as extensional equality.  Note that the `$=$' symbol is
used both for the formal equality symbol in the language and for
metatheoretic equality.   
\begin{comment}
We will also use the notation $\lambda (x_1 : T_1, \ldots, x_n : T_n): a$
in place of $\lambda (x : \tupletype{\seq{T}{n}}): a[
\proj{1}(x)/x_1, \ldots, \mathtt{PROJ\_n}(x)/x_n]$\@.
\end{comment}



Our first step will be to define the notion of a \emph{maximal supertype}
of a given type as $\mu(T)$.  A \emph{maximal type} $T$ is one such that
$\mu(T) = T$.  In a given context, we will apply  $\mu$ only to the
expanded form (given by $\delta$) of a type expression.  
\begin{Defn}{maximal supertype}
\begin{eqnarray*}
  \mu(s) & = & s \\
   \mu(\{x : T \vbar a\}) & = & \mu(T) \\
%
   \mu(\funtype{A}{B}) & = & \funtype{A}{\mu(B)} \\
  \mu(\tupletype{\listwo{A}{n}}) & = & \tupletype{\mu(A_1), \mu(A_2)}
\end{eqnarray*}
\end{Defn}
Note that since subtypes correspond to subsets, in taking the maximal
supertype of a function type, the domain type is held fixed.  In most type
theories with subtypes, the rule for subtyping between function types
$\funtype{A}{B}$ and $\funtype{A'}{B'}$ requires showing that $A'$ is a
subtype of $A$, and $B$ is a subtype of $B'$\@.  Subtyping between
function types is therefore said to be contravariant in the domain type
and covariant in the range type.  Subtyping on function types in PVS is
covariant in the range type but is neither covariant nor contravariant in
the domain type.  This means that the function type
$\funtype{\ttnat}{\ttnat}$ is not a supertype of the function type
$\funtype{\ttint}{\ttnat}$\@.  Such a subtyping relation would violate
{\em extensionality\/}\@.  Two functions on $\ttnat$ are extensionally
equal when they return equal values when applied to equal arguments in
$\ttnat$\@.  Consider two functions in $\funtype{\ttnat}{\ttnat}$:
$\mathit{abs}$ which returns the absolute value, and $\mathit{idnat}$
which behaves as an identity function on natural numbers and returns $0$
otherwise.  These two functions will be erroneously identified if they can
be viewed as being of type $\funtype{\ttnat}{\ttnat}$, and the subset
interpretation of subtypes would be lost.

We will also employ a weaker supertype $\mu_0(T)$ or the
{\em direct supertype\/}, that only considers supertypes of
explicitly given subtypes of the form $\{x : T \vbar a\}$.
\begin{Defn}{direct supertype}
\begin{eqnarray*}
\mu_0(\{x : T \vbar a\}) & = & \mu_0(T)\\
\mu_0(T) & = & T, \mbox{ otherwise}
\end{eqnarray*}
\end{Defn}

\begin{Eg}{maximal supertype}
Given a context containing the declarations
$$\begin{array}{l}
  \ttint : \tttype, \\
  \mathtt{0} : \ttint,\\
  \leq : \funtype{\tupletype{\ttint, \ttint}}{\ttbool},\\
  \ttnat : \tttype = \{ i : \ttint \vbar  0 \leq i\}\\
  \mathtt{natinjection} : \tttype = \{ f : \funtype{\ttnat}{\ttnat}\ \vbar\ 
\forall (i, j: \ttnat): f(i) = f(j) \supset i = j \}
\end{array}$$
we have
\begin{eqnarray*}
  \mu(\mathtt{natinjection}) & = & \mu(\funtype{\ttnat}{\ttnat})\\
 &=  & \funtype{\ttnat}{\mu(\ttnat)}\\
 &=  & \funtype{\ttnat}{\ttint}\\
\mu_0(\mathtt{natinjection}) & = & \funtype{\ttnat}{\ttnat}
\end{eqnarray*}
\end{Eg}

Note that   $\mu(\mu(A)) = \mu(A)$.
Note also that a maximal supertype is never a subtype.  
We can in fact collect the predicates that constrain a type $A$
relative to its maximal supertype $\mu(A)$ as $\pi(A)$\@.

\begin{Defn}{subtype constraints}
\begin{eqnarray*}
  \pi(s) & = & \lambda (x : s): \tttrue \\
   \pi(\{y : T \vbar a\}) & = &   \lambda (x : \mu(T)): (\pi(T)(x) \wedge a[x/y])\\
%
   \pi(\funtype{A}{B}) & = &   \lambda (x : \funtype{A}{\mu(B)}):
    (\forall (y: A): \pi(B)(x(y))) \\
%
\pi(\tupletype{\listwo{A}{n}}) & = &
    \lambda (x: \tupletype{\mu(A_1),  \mu(A_2)}): 
	(\pi(A_1)(\proj{1}~x)\wedge \pi(A_2)(\proj{2}~x))
\end{eqnarray*}
\end{Defn}
Observe that in Definition~\ref{defn:subtype constraints}, if $\tauGamma{A} =
\tttype$, then 
$\tauGamma{\pi(A)} = \funtype{\mu(A)}{
 \mathtt{bool}}$\@.\footnote{This is somewhat tricky in the case of
$\pi(\{y:T\vbar a\})$ since in $a[x/y]$, $x$ has type $\mu(T)$, whereas
$y$ has type $T$\@.  As shown in Chapter~\ref{conditionals}, the
type rules for conjunction are such that $\tau(\Gamma, x :
\ttvar~\mu(T))(\pi(T)(x) \wedge a)$ reduces to
$\tau(\Gamma, x : \ttvar~\mu(T))(\pi(T)(x))$ and
$\tau(\Gamma, x : \ttvar~\mu(T), \pi(T)(x))(a[x/y])$ where the first
conjunct is added to the context as a contextual assumption.
One can then show by induction that $\tau(\Gamma, x:\ttvar~\mu(T),
\pi(T)(x))(a[x/y]) = \tau(\Gamma, y: \ttvar~T)(a)$\@.  }

\begin{Eg}{subtype constraints}
\begin{eqnarray*}
\lefteqn{\pi(\ttnat)}\\
  & = & \lambda (j: \ttint): 0\leq j\\
\lefteqn{\pi(\funtype{\ttnat}{\ttnat})}\\
& = & \lambda (g : \funtype{\ttnat}{\ttint}): \forall (i:\ttnat): (\lambda
(j:\ttint): 0 \leq j)(g(i))\\
\lefteqn{\pi(\mathtt{natinjection})}\\
&= & \lambda (f : \funtype{\ttnat}{\ttint}) : 
\begin{array}[t]{cl}
& \pi(\funtype{\ttnat}{\ttnat})(f)\\
\wedge & (\forall (i, j: \ttnat): f(i) = f(j) \supset  i = j)
\end{array}\\
& = & 
\lambda (f : \funtype{\ttnat}{\ttint}) :\\&&\hspace*{.5in}
\begin{array}[t]{cl}
& (\lambda (g : \funtype{\ttnat}{\ttint}) : \forall (i: \ttnat):
(\lambda (j : \ttint): \mathtt{0}\leq j)(g(i)))(f) \\
\wedge & (\forall (i, j: \ttnat): f(i) = f(j) \supset  i = j)
\end{array}
\end{eqnarray*}
\end{Eg}
Observe that $\pi(\mu(A))$ is essentially equivalent to $\lambda (x : \mu(A)):
\mathtt{TRUE}$. 

Since the subtype $\{x : T \vbar p(x) \wedge q(x)\}$ can also be written
as $\{x : T \vbar q(x) \wedge p(x)\}$, we need a notion of equivalence
between types.  One way to do this is to make types ``first-class'' and to
allow explicit theorems to be proved about type equivalence and subtyping.
Since this would be a fairly drastic extension to the specification
language, we have designed the PVS type system so as to avoid any
first-class treatment of types.  It turns out that all the needed
properties about types (such as equality and subtyping) can be obtained by
generating ordinary proof obligations rather than by explicitly proving
theorems about types.

We introduce 
below a metatheoretic operation that generates the proof obligations
needed to establish that two (maximal) types are equivalent.  This
equivalence is denoted by $\simeq$ and is  applied only to maximal types
and returns a list of the 
proof obligations that must be proved.  Note the invariant in the
definition below that the arguments to $\simeq$ are always maximal.  The
definition of $\simeq$ makes use of the PVS equality predicate that will be
introduced later.  A list of formulas is represented as $a_1,\ldots, a_n$.
Given two such lists $\lis{a}{m}$ and $\lis{b}{n}$, the concatenation of
these two lists is written as $\lis{a}{m}\ ; \ \lis{b}{n}$\@.

\begin{Defn}{type equivalence proof obligations}
\begin{eqnarray*}
  (s \simeq s) & = & \tttrue \\
%
  (\funtype{A}{B} \simeq \funtype{A'}{B'}) & = & ((\mu(A)\simeq \mu(A'));
(\pi(A) = \pi(A'));
 (B \simeq B'))\footnotemark \\
%
  (\tupletype{\listwo{A}{n}} \simeq \tupletype{\listwo{B}{n}}) & = &
    ((A_1\simeq B_1); (A_2\simeq B_2)) \\
%
 (A \simeq B) & = & \ttfalse, \mbox{ otherwise}
\end{eqnarray*}
\end{Defn}
\footnotetext{The type correctness of the proof obligation $(\pi(A) =
\pi(A'))$ depends on the \emph{prior} proof obligations $\mu(A)\simeq
\mu(A')$.}

\begin{Eg}{type equivalence}  Building on the context given in
Example~\ref{eg:maximal supertype}, 
if we have the following variants of $\ttnat{}$ and
\texttt{natinjection}:
$$
\begin{array}{l}
\mathtt{NAT} :  \tttype = \{ i : \ttint \vbar  i \leq \mathtt{0}\ 
\supset\ i = 0\}\\
\mathtt{NATinjection} : \tttype = \{ f : \funtype{ \mathtt{NAT}}{
\mathtt{NAT}} \vbar 
\forall (i, j: \mathtt{NAT}): f(i) = f(j)\ \supset\ i = j \}
\end{array}$$
we get
\begin{eqnarray*}
\mu(\funtype{ \mathtt{natinjection}}{ \mathtt{natinjection}}) =
 \funtype{ \mathtt{natinjection}}{\funtype{\ttnat}{\ttint}}\\
%
\mu(\funtype{ \mathtt{NATinjection}}{ \mathtt{NATinjection}}) =
 \funtype{ \mathtt{NATinjection}}{\funtype{ \mathtt{NAT}}{ \ttint}}
\end{eqnarray*}
\begin{eqnarray*}
\lefteqn{\mu(\funtype{ \mathtt{natinjection}}{ \mathtt{natinjection}})
\simeq \mu(\funtype{ \mathtt{NATinjection}}{ \mathtt{NATinjection}})}\\
& = &  \begin{array}[t]{l}
       (\mu( \mathtt{natinjection}) \simeq \mu( \mathtt{NATinjection}));\\
       (\pi( \mathtt{natinjection}) = \pi( \mathtt{NATinjection}));\\
       (\funtype{\ttnat}{\ttint} \simeq \funtype{ \mathtt{NAT}}{ \mathtt{int}})     
       \end{array}\\
 \lefteqn{(\pi( \mathtt{natinjection}) = \pi( \mathtt{NATinjection}))}\\
& = & \begin{array}[t]{cl}
(&      \lambda (f : \funtype{\ttnat}{\ttint}) :
\begin{array}[t]{cl}
& (\lambda (g : \funtype{\ttnat}{\ttint}) : \forall (i: \ttnat):
\mathtt{0}\leq  g(i))(f) \\
\wedge & (\forall (i, j: \ttnat): f(i) = f(j) \ \supset\  i = j)
 \end{array}\\
 =  \\&  \lambda (f : \funtype{ \mathtt{NAT}}{\ttint}) :\\&\hspace*{.3in}
\begin{array}[t]{cl}
& (\lambda (g : \funtype{ \mathtt{NAT}}{\ttint}) : \forall (i: \mathtt{NAT}):
g(i) \leq \mathtt{0}\ \supset\ g(i) = 0)(f) \\
\wedge & (\forall (i, j: \mathtt{NAT}): f(i) = f(j) \ \supset\  i = j)
      \end{array} )
 \end{array}\\
\lefteqn{(\funtype{\ttnat}{\ttint} \simeq \funtype{ \mathtt{NAT}}{
\mathtt{int}})}\\
& = & (\ttint \simeq \ttint);\\&& (\lambda (i : \ttint): \mathtt{0} \leq i) =
                              (\lambda (i : \ttint): 
i\leq \mathtt{0}\  \supset\ i = 0); (\ttint\simeq\ttint)
\end{eqnarray*}
\end{Eg}

A basic question during typechecking is whether two types are
\emph{compatible}, that is,  have the same maximal supertype.  
Two types are said to be compatible if the type equivalence proof
obligations on their respective 
maximal supertypes are provable.  The provability of a formula $a$
under context $\Gamma$ is represented as $\vdash_\Gamma a$\@.  
\begin{Defn}{compatible}
\emph{Two types $A$ and $B$ are said to be compatible in context $\Gamma$
(in notation, $(A \sim B)_\Gamma$)
if $\vdash_\Gamma a$, for each $a$ in $(\mu(A)\simeq \mu(B))$.}\footnotemark
\end{Defn}
\footnotetext{ The PVS proof rules are described in 
Chapter~\ref{proof theory}.}


\begin{comment}
With subtypes, it is easy to define an empty subtype and then
declare a constant of such an empty subtype.  The context thus has to be
checked to disallow such an inconsistency.
\end{comment}
We now extend the
definition of $\delta$ to the case of subtypes so that it leaves the
predicate unchanged but expands the definition of the supertype.
%
\begin{Defn}{expanded type with subtypes}
\begin{eqnarray*}
  \delta(\Gamma)(\{x : T \vbar a\}) & = & \{x : \delta(\Gamma)(T) \vbar a \}
\end{eqnarray*}
\end{Defn}

We now extend the definition $\tau$ to the case of subtypes.  Here we
could force $\tau$ to always return a maximal supertype but this is not
done in Definition~\ref{defn:type rules with subtypes}  since it
would weaken the soundness theorem without significantly simplifying the
definition of the type rules.  The typechecking of contexts has to be
modified to generate a {\em nonemptiness\/} proof obligation for the type of
any constant declaration.  A constant of an empty type would lead to an
inconsistent context, and this would mean that constant declarations are
not conservative extensions.  This modification to
Definition~\ref{defn:type rules} is not needed for soundness since an
inconsistent context makes soundness trivial.  It is needed to show that
constant declarations and definitions are conservative extensions.  Note
that with subtypes, the type rule for an application is modified to check
that the domain type of the function is compatible with the type of its
argument, and that the argument satisfies any constraints imposed by the
domain type of the function.  The case of projection expressions is also
not straightforward since the argument type can be a subtype of a tuple
type.  In this case, we use the direct supertype (see
Definition~\ref{defn:direct supertype}) which must be a tuple type.
%
\begin{Defn}{type rules with subtypes}
\begin{eqnarray*}
  \tau()(\Gamma, c : T) & = & \ttcontext, \mbox{ if } \Gamma(c)
\mbox{ is undefined},\\& &
\tauGamma{T} =   \tttype{},\\& &
      \tau()(\Gamma) = \ttcontext,  \mbox{ and } \\& & 
   \vdash_\Gamma (\exists (x: T) : \tttrue)\\
%
  \tauGamma{\{x : T \vbar a\}} & = & \tttype{}, \mbox{ if } \Gamma(x)
\mbox{ is undefined}, \\& & 
    \tauGamma{T} = \tttype{}, \mbox{ and } \tau(\Gamma, x : \ttvar{}\ T)(a) = \ttbool \\
%
  \tauGamma{f\ a} & = &  B, \mbox{ where } \mu_0(\tauGamma{f}) =
                                                 \funtype{A}{B},\\&  & 
                   			\tauGamma{a} = A',\\ & &
					(A \sim A')_\Gamma,\\& & 
                                        \vdash_\Gamma \pi(A)(a) \\
 \tauGamma{\proj{i}~a} & = & A_i, \mbox{ where } \mu_0(\tauGamma{a}) =
\tupletype{A_1, A_2}
\end{eqnarray*}
\end{Defn}

\begin{Eg}{typechecking subtypes}
Let $\Gamma$ contain the above declarations of $\ttint$, $\ttnat$,
$\mathtt{0}$, 
$\leq$, and {\tt
natinjection}.
\begin{eqnarray*}
\lefteqn{\tauGamma{\{i : \ttint \vbar 0\leq i\}}}\\
  & = & \tttype\\
\lefteqn{\tauGamma{(\lambda (f : \mathtt{ natinjection}) : f(0))(\lambda
(i: \mathtt{nat}): i)}}\\
& = &
\delta(\Gamma)(\ttnat), \mbox{ if}\\& &
( \mathtt{natinjection} \sim \funtype{\ttnat}{\ttnat})_\Gamma,\\& & 
\vdash_\Gamma \forall (j, k : \ttnat): (\lambda (i: \ttnat): i)(j) = (\lambda
(i: \ttnat) : i)(k) \ \supset\  j = k, \\& &
(\ttint \sim \ttnat)_\Gamma, \mbox{ and}\\& & 
\vdash_\Gamma \mathtt{0}\leq \mathtt{0}
\end{eqnarray*}

\end{Eg}

Only one additional clause to Definition~\ref{defn:meaning function} is
needed to capture the 
semantics of predicate subtypes.
\begin{Defn}{meaning function with subtypes}
\begin{eqnarray*}
  \lefteqn{\Mgamma{\{x : T \vbar a\}}}\\
& = & \{ y \in \Mgamma{T} \vbar
\mathcal{M}(\Gamma, x: \ttvar~T\vbar \gamma\{x
\gets y\})(a) = {\oneb} \} 
\end{eqnarray*}
\end{Defn}

\begin{Eg}{semantics of predicate subtypes}
If we assign the usual truth table interpretation to the Boolean function
$\supset$: 
\begin{eqnarray*}
\lefteqn{\Mgamma{\{f : \funtype{\ttbool}{\ttbool} \vbar \forall (x: \ttbool): x\ \supset\ f(x)\}}}\\
& = & \{\{\pair{\zerob, \zerob}, \pair{\oneb, \oneb}\},
\{\pair{\zerob,\oneb}, \pair{\oneb, \oneb}\}\}.
\end{eqnarray*}
\end{Eg}

The following useful propositions are easily proved from the definitions
given above.   Proposition~\ref{max-type-correct} asserts that the
maximal supertype of a type is well typed.  Proposition~\ref{maximal-subset}
asserts that the denotation of a type is a subset of the denotation of its
maximal supertype.  Proposition~\ref{maximal-equality} asserts that
if all the proof obligations in $(A\simeq A')$ are valid relative to
a given assignment $\gamma$ for context $\Gamma$, then
the denotations of $A$ and $A'$ under $\gamma$ are equal. 
%
\begin{prop}\label{max-type-correct}
  If $\tau()(\Gamma) = \ttcontext$ and
  $\tauGamma{A} = \tttype{}$, then $\tauGamma{\mu(A)} = \tttype{}$.
\end{prop}
\begin{prop}\label{maximal-subset}
  If $\tau()(\Gamma) = \ttcontext$, 
    $\tauGamma{A} = \tttype{}$,
   and $\gamma$ satisfies $\Gamma$, 
then
\begin{enumerate}
\item   $\Mgamma{A} \subseteq \Mgamma{\mu(A)}$ and
\item   $\Mgamma{A} \subseteq \Mgamma{\mu_0(A)}$.
\end{enumerate}
\end{prop}
\begin{prop}\label{maximal-equality}
  If $A$ and $A'$ are maximal
types in context $\Gamma$, i.e.,
\begin{enumerate}
\item $\tau()(\Gamma) = \ttcontext$, 
\item $\tauGamma{A} = \tauGamma{A'} = \tttype{}$,
\item $\mu(A) = A$ and $\mu(A') = A'$
\end{enumerate}
and
 for each $a$  in $(A \simeq A')$, %%\memo{What about true/false?}
\begin{enumerate}
\item $a\equiv \tttrue$, or 
\item $a\equiv (a_1 = a_2)$ and $\Mgamma{a_1} = \Mgamma{a_2}$
holds,
\end{enumerate}
then $\Mgamma{A} = \Mgamma{A'}$.\footnote{ 
We remind the reader that the formulas $a$ in $(A\simeq A')$ are equalities,
but we have not yet formally introduced equality into the language.   } 
\end{prop}
\begin{prop}\label{maximal-subtype}
If $\tau()(\Gamma) = \ttcontext$ and $\tauGamma{T} = \tttype$,
then $\Mgamma{T} = \Mgamma{\{x : \mu(T) \vbar \pi(T)(x)\}}$.
\end{prop}

We can now examine the updated forms of the invariants given by
Theorems~\ref{type construction},
\ref{simple-type-semantics},
and~\ref{simple-term-semantics}\@.  
The proof of Theorem~\ref{type construction} remains straightforward.
\begin{comment}
Theorems~\ref{simple-type-semantics}
and~\ref{simple-term-semantics} must be proved with a simultaneous
induction since types now contain terms.
\end{comment}
The statement of
Theorem~\ref{simple-term-semantics} must now
be strengthened to include soundness, that is, if $\vdash_\Gamma a$
and $\gamma$ satisfies $\Gamma$, then 
$\Mgamma{a} = \oneb$\@.  For now, we assume soundness
(Theorem~\ref{proof-soundness})  since we
have not yet presented the proof rules.  
%
\begin{theorem}[type soundness]
\label{subtype-type-semantics}
If $\tau()(\Gamma) = \ttcontext$, $\gamma$ satisfies $\Gamma$, and
$\tauGamma{A} = \tttype$ then $\Mgamma{A}\in U$.
\end{theorem}
\begin{proof}
There is only one new case to add to the induction proof of
Theorem~\ref{simple-type-semantics}, namely, when 
 $A\equiv \{x : T \vbar a\}$.   In this case, by
Definition~\ref{defn:type rules with subtypes},
$\tauGamma{T} = \tttype$, so by the induction hypothesis,
$\Mgamma{T} \in U$.  Since, by
Definition~\ref{defn:meaning function with subtypes}, $\Mgamma{A} \subseteq
\Mgamma{T}$, we have $\Mgamma{A}\in U$ by
Definition~\ref{defn:type universe}\@.
\end{proof}
\begin{theorem}[term soundness]\label{subtype-term-semantics}
If $\tau()(\Gamma) = \ttcontext$, $\gamma$ satisfies $\Gamma$, and $\tauGamma{a} = A$ then $\Mgamma{a} \in
\Mgamma{A}$\@.  
\end{theorem}
\begin{proof}  There are two affected cases in the proof from that of
Theorem~\ref{simple-term-semantics}, namely, those of application
and projection.  The case of projection expressions is straightforward
given Proposition~\ref{maximal-subset}.

When 
 $a \equiv (f\ b)$, by Definition~\ref{defn:type rules with subtypes},
we have that $\tauGamma{f} = \funtype{B}{A}$ and $\tauGamma{b} = B'$.
Let $X$ be $\Mgamma{B}$, $X'$ be $\Mgamma{B'}$,  and $Y$ be $\Mgamma{A}$.
Then by Definition~\ref{defn:meaning function}, $\Mgamma{\funtype{B}{A}} =
Y^X$\@.  By the induction hypotheses, $\Mgamma{f}\in Y^X$ and
$\Mgamma{b}\in X'$\@.  
By Definition~\ref{defn:type rules with subtypes},
soundness of the proof rules (Theorem~\ref{proof-soundness}), and
Propositions~\ref{maximal-subset} and 
~\ref{maximal-equality}, there is a 
maximal supertype $\mu(B)$ of both $B$ and $B'$ such that $X$ and $X'$
are both subsets of $\Mgamma{\mu(B)}$\@.  Since, by
Definition~\ref{defn:type rules with subtypes}, $\vdash_\Gamma{}\pi(B)(b)$,
and by Proposition~\ref{maximal-subtype}, $\Mgamma{B} =
\Mgamma{\{x : \mu(B)\vbar \pi(B)(x)\}}$, we have $\Mgamma{b}\in
\Mgamma{B}$, and hence by Definition~\ref{defn:meaning function}, $\Mgamma{(f\
b)}\in  \Mgamma{A}$\@.  
\end{proof}



\section{Summary}

PVS features a form of subtyping where it is possible to form the subtype
of a type satisfying a given predicate on the type.  This kind of
subtyping introduces several delicate semantic issues into PVS.  A term
can now have several types since, for example, the term corresponding to
the number {\tt 2} can be a prime number, an even number, a natural
number, an integer, a rational number, or a real number.  When the
expected type is a subtype, the canonical type of the actual term must be
compatible with the expected type, that is, the two maximal supertypes
must be equivalent and the actual term must satisfy any subtype
constraints imposed by the expected type.  We have defined the notions of
maximal supertype, subtype constraints, type equivalence, and
compatibility.  These notions are used to define the type rules and
semantics of the simply typed fragment of PVS extended with subtypes.
Note that both type equivalence (and hence, compatibility) and type
correctness are undecidable.  Proof obligations generated during
typechecking are the only source of such undecidability.  The
modularization of the type system into a decidable part consisting of the
simply typed fragment,  and the proof obligations generated by
subtyping, is perhaps the most significant design consideration in the PVS
language.




\chapter{Dependent Types}\label{dependent}

The PVS language fragment described thus far is already quite expressive.
It employs definitional equivalence between types and contains
predicate subtypes.  It is undecidable whether an expression in this
fragment is type-correct because of the proof obligations that arise with
respect to predicate subtypes and type equivalence.  
The next step is the addition of type dependencies between the components
of a type.  This extension considerably
enhances the utility of this type system.  It is also a natural extension
given predicate subtyping which already allows types that
{\em depend\/} on free variables in the predicates.  
With dependent typing, we can
make the type of one component of a product depend on the value of another
component, or the type of the range of a function vary according to its
argument value.

\begin{comment}
Let $V$ be a metavariable that ranges over types or {\em
type bindings\/} of the form $x : T$.
\end{comment}
A dependent product type is
written as $\tupletype{x : A, B}$.  A dependent function type is written
as $\funtype{x : A}{B}$.
Any product or function type can be transformed into a dependent type by
inserting dummy type bindings.  Conversely, any dummy type bindings that
do not actually bind any variable occurrences can be
removed.  The type rules and semantics below will assume that all product and
function types are presented as dependent types.

\begin{Eg}{dependent types}
$$
\begin{array}{l}
\tupletype{i : \ttnat, \{j: \ttnat \vbar j\leq i\}}, \\
\tupletype{i : \ttnat, \funtype{\{j: \ttnat \vbar j\leq i\}}{\ttbool}},\\
\funtype{i: \ttint}{\{j : \ttint \vbar i\leq j\}}.
\end{array}$$
\end{Eg}


Before we treat dependent types, we update the definitions of the set
of free variables and substitution to account for the fact that
with subtyping and dependent typing, both free and bound variables
can occur in terms and types.  This is needed for the next step
where we try to remove type dependencies by substituting a term into
a dependent type.  
\begin{Defn}{free variables for types}
\begin{eqnarray*}
%%  FV(\Gamma)(s) & = & \emptyset, \mbox{ if } \itkind(\Gamma(s)) \neq
%%\ttvariable\\
FV(\Gamma)(\funtype{x : A}{B}) & = & FV(\Gamma)(A)\union
                   (FV(\Gamma, x: \ttvar~A)(B) - \{x\}) \\
%
FV(\Gamma)(\tupletype{x: A, B}) & = &
  FV(\Gamma)(A)\union (FV(\Gamma,
x:\ttvar~A)(B) - \{ x\}) \\
%
FV(\Gamma)(\{x : A \vbar a\}) & = & FV(\Gamma)(A) \union
(FV(\Gamma, x:\ttvar~A)(a) - \{x\})
\end{eqnarray*}
\end{Defn}

\begin{Defn}{substitution for types}
\begin{eqnarray*}
\lefteqn{\funtype{x : A}{B}[a_1/x_1,\ldots, a_n/x_n]}\\ & = & \funtype{y :
A[a_1/x_1,\ldots,a_n/x_n]}{B[y/x, a_1/x_n, \ldots, a_n/x_n]} \\
%
\lefteqn{\tupletype{x: A, B}[a_1/x_1, \ldots, a_n/x_n]}\\ & = &
\tupletype{y: A[a_1/x_1,\ldots, a_n/x_n], 
B[y/x, a_1/x_1,\ldots,a_n/x_n]} \\
%
\lefteqn{\{x : A\vbar a\}[a_1/x_n,\ldots, a_n/x_n]}\\ & = & \{y : A[a_1/x_1,\ldots,
a_n/x_n] \vbar  a[y/x, a_1/x_n, \ldots, a_n/x_n] \}
\end{eqnarray*}
where y is a fresh variable.
\end{Defn}

%\memo{Need to extend substitution to types.}

\begin{comment}
To determine whether a given expression is of a given dependent type, we
must define the notion of a substituting the expression into the type.  If
$\tauGamma{a}$ is a (possibly dependent) product type and $A$ is a
(possibly dependent) product type of the form $\tupletype{x : A_1, A_2}$, then
$a/A$ is the type obtained by ``substituting'' $a$ in $A$: if $a$ is of
the form $(\listwo{a}{n})$ then $a/A$ is $\tupletype{x : A_1,
a_2/(A_2[a_1/x])}$,
else if
$a$ is not of the form $(\listwo{a}{n})$, then we use $\proj{i}\ a$ in place
of $a_i$\@.  Similarly, if $\tauGamma{a}$ is a function type of the form
$\funtype{x : T}{S}$, then if $a$ is of the form $(\lambda (x : T): b)$,
then $(\lambda (x: T): b)/\funtype{y : A}{B}$ is $\funtype{y :
A}{b[y/x]/B}$\@.  If $a$ is not a lambda-expression but $\tauGamma{a}$ is
of the form $\funtype{z:T}{S}$, then $a/\funtype{y : A}{B}$ is just
$(\lambda (x : T): a(x))/\funtype{y : A}{B}$.  In every other case, $a/A$
is just $A$\@.
%%\memo{Need subtype case for substituting.  Put in separate definition.}
\begin{Eg}{substituting into dependent types}
\begin{eqnarray*}
\lefteqn{(5, 3)~/~\tupletype{i : \ttnat, \{j: \ttnat \vbar j\leq i\}}}\\
  & \equiv &
\tupletype{i: \ttnat, \{j: \ttnat \vbar j\leq 5\}}\\
%
\lefteqn{(3, (\lambda (k: \{j : \ttnat\vbar j < 3\}): k))~/~\tupletype{i : \ttnat,
\funtype{k : \{j: \ttnat \vbar j< i\}}{\{j: \ttnat \vbar j\leq k\}}}}\\
&
\equiv &
\tupletype{ i : \ttnat, \funtype{k : \{j: \ttnat \vbar j< 3\}}{\{j: \ttnat \vbar j\leq k\}}} 
\end{eqnarray*}
\end{Eg}
\end{comment}

\newcommand{\constrain}[2]{#1\backslash #2}
The definition of $\mu$ has to be modified slightly for dependent types.
The definition is first extended to type bindings, $\mu(x : T) = x :
\mu(T)$.  The definition for the case of dependent function types
is unchanged so that $\mu(\funtype{x : A}{B}) = \funtype{x :
A}{\mu(B)}$\@.   The definition for the product case is more delicate
since the definition $\mu(\tupletype{x : A, B}) = \tupletype{x : \mu(A),
\mu(B)}$ results in a loss of type information regarding the occurrences
of $x$ in $B$\@.\footnote{Doug Howe brought this problem to our attention.}
To ensure that type information regarding $x$ is
retained, we define a new operation $\constrain{T}{a}$
which constrains the subtype assertions in  type $T$ with an additional
assertion $a$\@.
\begin{Defn}{Adding subtype constraints}
\begin{eqnarray*}
\constrain{s}{a} & = & s \\
\constrain{\subtype{x}{T}{b}}{a} & = & \subtype{x}{T}{a \wedge b}\\
\constrain{\funtype{A}{B}}{a} & = &
\funtype{\constrain{A}{a}}{\constrain{B}{a}} \\
\constrain{\tupletype{A, B}}{a} & = & \tupletype{\constrain{A}{a}, \constrain{B}{a}}
\end{eqnarray*}
\end{Defn}

We can now define the maximal supertype operation for
dependent tuple types.
\begin{Defn}{Maximal supertype for dependent product types}
\begin{eqnarray*}
\mu(\tupletype{x : A, B}) & = & \tupletype{x : \mu(A), \constrain{B}{\pi(A)(x)}}
\end{eqnarray*}
\end{Defn}

The definition of $\pi$ for a dependent function type $\funtype{ y :
A}{B}$ is slightly different from that of an ordinary function type
since $\pi(B)$ can contain free occurrences of the variable $y$\@.
For example, $\pi(\funtype{i : \ttint}{\{j : \ttint \vbar i\leq j\}})$
must be  $\lambda (f : \funtype{i : \ttint}{\ttint}): (\forall (i:\ttint): i
\leq f(i))$\@.  The definition for dependent tuples remains essentially
unchanged from that of ordinary products.  
\begin{Defn}{constraint predicates for dependent types}
\begin{eqnarray*}
   \pi(\funtype{y:A}{B}) & = &  ( \lambda (x : \funtype{y:A}{\mu(B)}):
    (\forall (y: A): \pi(B)(x(y)))) \\
%
  \pi(\tupletype{y:A, B}) & = &
   ( \lambda (x: \tupletype{y : \mu(A), \constrain{\mu(B)}{\pi(A)(y)}}): \\&&\hspace*{.3in}
	\pi(A)(\proj{1}~x)\wedge \pi(B)(\proj{2}~x)[(\proj{1}~x)/y])
\end{eqnarray*}
\end{Defn}

\newcommand{\depfunexample}{\funtype{i : \ttint}{\{j : \ttint \vbar i\leq
j\}}}
\newcommand{\deptupexample}{\tupletype{i : \ttnat, \{j: nat \vbar j\leq i\}}}
\begin{Eg}{dependent type predicates}
\begin{eqnarray*}
\mu(\depfunexample{}) & = & \funtype{i : \ttint}{\ttint}\\
\pi(\depfunexample{}) & = & \lambda (f : \funtype{i : \ttint}{\ttint}):\\&&\hspace{.3in}
\forall (i: \ttint): (\lambda (j : \ttint): i\leq j)(f(i))
\end{eqnarray*}
\end{Eg}

The definition of $\simeq$ must also be massaged slightly for dependent
types.  Recall that $\simeq$ checks whether two maximal types are
equivalent by generating proof obligations as needed.  This is the
basic operation for checking whether the expected type of an expression
is compatible with its actual type.  
The subtlety now is that the expected type might be
a dependent type where the actual type is not.  Consider the
case of the pair $\pair{5, (\lambda (x : \{j:\ttnat \vbar j\leq 5\}): x)}$
whose type would be computed 
by $\tau$ as $\tupletype{i : \ttnat, \funtype{\{j : \ttnat\vbar j\leq 5\}}{\{j : \ttnat\vbar j\leq 5\}}}$
where the expected type might be 
$\tupletype{i : \ttnat, \funtype{\{j : \ttnat \vbar j\leq i\}}{\{j : \ttnat \vbar
j\leq i\}}}$\@.  To cope with this, 
we will allow the option of two maximal types, say $A$ and $B$, to be
compared using $\simeq$
in the context of an expression $a$\@.  This is indicated by the notation $(A\simeq B)/a$\@.  
Note that $(A\simeq B)/a$ is sensible only when $A$ and $B$
are maximal types.   The missing cases in
Definition~\ref{defn:type equivalence proof obligations}
are included in
Definition~\ref{defn:type equivalence for dependent types}.    For a list
of formulas $a_1, \ldots, a_n$, let 
$(\forall (x : T): a_1,\ldots, a_n)$ represent the list
$(\forall (x : T): a_1), \ldots, (\forall (x : T): a_n)$\@.\footnote{Note that the type-correctness of the proof
obligation $(\pi(A) = \pi(A'))$ in
Definition~\ref{defn:type equivalence for dependent types}  depends on the
\emph{prior} proof obligations 
$\mu(A)\simeq \mu(A')$.}
\begin{Defn}{type equivalence for dependent types}
\begin{eqnarray*}
  (s \simeq s)/a & = & \tttrue \\
(\funtype{x:A}{B} \simeq \funtype{x':A'}{B'}) & = & (\mu(A)\simeq
\mu(A'));\\ & & 
(\pi(A) = \pi(A'));\\ & & 
 (\forall (x : A): (B \simeq B'[x/x'])) \\
%
(\funtype{x:A}{B} \simeq \funtype{x':A'}{B'})/a & = & (\mu(A)\simeq
\mu(A'));\\ & & 
(\pi(A) = \pi(A'));\\ & & 
 (\forall (x : A): (B \simeq B'[x/x'])/a(x)) \\
%
(\tupletype{x : A_1, A_2} \simeq \tupletype{y :
B_1, B_2})
& = &
    (A_1\simeq B_1);\\ & & (\forall (x : A_1) : (A_2\simeq
B_2[x/y]))\\
%
(\tupletype{x : A_1, A_2} \simeq \tupletype{y :
B_1, B_2})/a
& = &
    (A_1\simeq B_1)/(\proj{1}~a);\\ & & (A_2[(\proj{1}~a)/x]\simeq
B_2[(\proj{1}~a)/y])/(\proj{2}~a)\\
%
 (A \simeq B)/a & = & \ttfalse, \mbox{ otherwise.}
\end{eqnarray*}
\end{Defn}
As with $(A\sim B)_\Gamma$, the notation $(A \sima 
B)_\Gamma$ indicates that  all the proof obligations $a'$ in
$(\mu(A)\simeq \mu(B))/a$ are provable, that is,  $\vdash_\Gamma a'$\@.  

With dependent types, the type rules must be modified so as to augment the
context suitably to account for any dependencies.  
We will give the definitions only for dependent type constructions.

%\memo{need to handle type dependencies in expected for application case.}
\begin{Defn}{type rules with dependent types}
\begin{eqnarray*}
\tauGamma{\tupletype{x : A, B}} & = &
\tttype{}, \mbox{ if } \Gamma(x) \mbox{ is undefined},\\
& & \tauGamma{A} = \tttype{}, \mbox{ and}\\
& & \tau(\Gamma, x : \ttvar{}\ A)(B) = \tttype{} \\
%
%\tauGamma{\tupletype{T_1, U_2, \ldots, U_n}} & = &
%\tttype{}, \mbox{ if } \tauGamma{T_1} = \tttype{}, \mbox{ and}\\
%& & \tauGamma{\tupletype{U_2,\ldots, U_n}} = \tttype{} \\
%
\tauGamma{\funtype{x: A}{B}} & = & \tttype{},
    \mbox{ if } \Gamma(x) \mbox{ is undefined},\\
& & \tauGamma{A} = \tttype{}, \mbox{ and}\\
& & \tau(\Gamma, x : \ttvar{}\ A)(B) = \tttype{}\\
%
  \tauGamma{f\ a} & = &  B', \mbox{ where } \mu_0(\tauGamma{f}) = \funtype{x :
A}{B},\\ 
 & &                   			\tauGamma{a} = A', \\ & & 
					(A \sima  A')_\Gamma, \\
& &
                                        B' \mbox{ is } B[a/x],\\ & & 
                                        \vdash_\Gamma \pi(A)(a)\\
%
 \tau(\Gamma)(\lambda (x : A): a) & = &
         \funtype{x : A}{B}, \mbox{ where}\\
%& & A  = \tupletype{x_1 : T_1, \ldots, x_n : T_n}, \\
& & B = \tau(\Gamma, x : \ttvar{}\ A)(a)\\ 
%B' = B[(\proj{1}\ x)/x_1, \ldots, (\proj{n}\
%x)/x_n] \\ & & 
%\mbox{if } \mbox{ for } 1\leq i\in \{1, 2\},\\& & 
%\tau(\Gamma, x_1 : \ttvar{}\ T_1,\ldots, x_{i-1} : \ttvar{}\ T_{i-1})(T_i) = \tttype{}, \\
%& & \mbox{and for } 1\leq i, j \in \{1, 2\}, i\neq j, x_i \not\equiv x_j \\
%
\tauGamma{\proj{1}~a} & = & A_1, \mbox{ where } \mu_0(\tauGamma{a}) =
\tupletype{x: A_1, A_2} \\
%
\tauGamma{\proj{2}~a} & = & A_2[(\proj{1}~a)/x], \mbox{ where } \mu_0(\tauGamma{a}) =
\tupletype{x: A_1, A_2} \\
\end{eqnarray*}
\end{Defn}
%%\memo{Need proj case above.}

\begin{Eg}{dependent typing}
\begin{eqnarray*}
\tauGamma{\tupletype{x : \ttbool,  \{ y : \ttbool \vbar x \supset
y\}}}
& = & \tttype\\
\tauGamma{\funtype{x : \ttbool}{\{ y : \ttbool \vbar x \supset 
y\}}} & = & \tttype
\end{eqnarray*}
\end{Eg}
Before we can assign meanings to dependent types, we must
augment our definition of the universe $U$ to contain sets corresponding
to these constructions.  If $F$ is a function with domain set $X$ and
a range $Y$, which is a set of sets, we can  define $\Sigma{F}$ to be the set 
$\{ \pair{x, y} \vbar x\in \textit{dom}(F), y\in F(x)\}$ and
$\Pi{F}$ to be the set $\{f \vbar (\forall x\in \textit{dom}(F): f(x)\in
F(x))\}$\@.   Note that $\Pi{F} \subseteq \bigcup_{X\in \Sigma{F}}\wp(X)$ but we
include $\Pi{F}$ in the universe $U$ defined below for simplicity\@.
We can drop $X\times Y$ and $X^Y$ from the universe definition since
$X\times Y$ can be obtained from $\Sigma{F}$ by defining an
$F$ with domain $X$ that always returns $Y$, and similarly,
$X^Y$ can be obtained by $\Pi{F}$ where $F$ is defined to with domain $Y$
to always return $X$\@.  The universe $U$ can then be redefined as below.  
\begin{Defn}{type universe with dependent types}
\begin{eqnarray*}
  U_0 & = & \{ {\twob}, {\reals}  \} \\
  U_{i+1} & = & U_i  \\& &
               %\union \{X\times Y \vbar X, Y \in U_i\} \\ & & 
                         %\union  \{X^Y \vbar X, Y \in U_i \} \\ & & 
                          \union  \bigcup_{X\in U_i}\wp(X)\\& & 
                           \union\; \{\Sigma{F} \vbar F \in W_i\}\\& &
                            \union\; \{\Pi{F} \vbar F \in W_i\}\\
  W_i & = & \bigcup_{X\in U_i} U_i^X \\
  U_{\omega} & = & \bigcup_{i\in\omega} U_i\\
  U & = & U_{\omega}
\end{eqnarray*}
\end{Defn}

One very important consequence of the above extension of the universe
is that all type dependencies must be bounded in the sense that
if $B$ is a type expression with a single free variable $x$ of type $A$,
then it must be the case that for any set $\mean{A}$ representing $A$,
there is a bound $n$ such that for any $z$ in $\mean{A}$, the meaning of 
$B$ under $\{x \gets z\}$ must be in $U_n$\@.  This property is easily
proved by induction on the structure of a PVS type since the parameter
$x$ can appear only in the predicate part of a subtype where the
rank of the meaning of the resulting type cannot vary with the value
of $x$\@.   In
particular, there is no way to define a type constructor $T^n$ in PVS that
returns the $n$-tuple $[\underbrace{T, [\ldots, T]}_n]$ for a given $n$
since this would entail an unbounded dependency.  
If unbounded type dependencies were allowed in PVS,
one can construct a dependent type such as $\funtype{n : nat}{T^n}$
whose representation is not in $U$ as defined above.    
  

\begin{comment}
A simpler alternative might be to define a second universe
U2 which starts from U1 and adds the dependent types.  This will work fine
in the absence of parametric theories since each type contains only a
fixed number of type dependencies.  The introduction of parametric
theories might require a third iteration since they introduce a third
form of dependency.
\end{comment}

\begin{comment}
Seeking advice on PVS semantics:  I'd like to know if the following
universe construction provides a satisfactory model for PVS types.  The
PVS simple type system consists of the base types bool and real and the
tuple [A_1,...,A_N] and function [A->B] types.  The flg. universe U1
(where (U_n)^n is the n-fold Cartesian product and (U_n)^(U_n) is the
function space) provides a model: 
 U_0     = {2, R}
 U_{n+1} = U_n UNION (U_n)^n UNION (U_n)^(U_n)
 U_{omega} = SUP_n U_n
 U1      = U_omega
%
The next step is to add subtypes {x:A \vbar p(x)}.  The above universe can be
easily modified to accommodate these by adding the subsets of the finite
simple types. Here P(X) is the power-set of X.  
%
 U_{n+1} = U_n UNION {U_n}^n UNION (U_n)^(U_n) UNION P(U_n)
%
The type system is then extended with dependent function [x:A->B(x)]
and tuple types [x_1: A, B].  The simple way to construct a universe is to
say that there are only finite length chains of dependencies and these are
captured by the universe U2 below which is constructed iteratively from
U1.  Here, if F is a function graph, #F is the set
{<x, y> \vbar x in domain(F), y in F(x)}. 
%
  U2_0 = U1
  U2_{n+1} = U_n UNION (U_n)^n UNION (U_n)^(U_n) UNION P(U_n)
             UNION {#F \vbar F in {(U2_n)^X \vbar X in U2_n}}
  U2_{omega} = SUP_n U2_n
  U2 = U2_{omega}
%
The reason this works is that it is easy to see by induction on the
type structure that we can assign a degree k to a given type,
where the degree of a nondependent type is 0, and the degree of
a type of the form [x: S, T(x)] is max(deg(S), deg(T)) + 1.
Note that if a type has degree k, then it is introduced in the
kth iteration of the definition of U2.  Dependent function types
are ignored here since they are defined as subsets of the corresponding
tuples so that they are introduced in the next iteration by the power-set
construction.
%
There really is no need for a second universe construction here since the
above argument would also work if the U2 iteration was rolled into that of
U1.
%
Parametric theories pose a serious difficulty since we now have types and
constants that can depend on types.  The context provides a
meaning function to each theory that maps the meanings of the
actual parameters to the meanings of the symbols declared in the theory.
We need to constrain the meaning function for a theory so that
the degree of each type name in the theory depends only on the
degrees of the type parameters.  This is a basic invariant in PVS that
the dependency between types and individuals only affects the
predicate part of the type and not its structure.  One cannot, for
instance, define a tuple whose first component is n:nat, and whose
second component is the n-fold tuple T^n.   Note that this
invariant is preserved by the way in which theories are actually used.
An interpretation of a theory simply attaches a definition to an
undefined type and the meaning of this definition will also have the
boundedness property.  
\end{comment}









The meaning function for dependent types is obtained by adding the
cases corresponding to 
dependent product and function types.  All the other cases are unchanged from
Definition~\ref{defn:meaning function with subtypes}\@.  Note that the
semantic definition  for dependent types is 
equivalent to the nondependent one when there are no dependencies.
\begin{Defn}{meaning function with dependent types}
\begin{eqnarray*}
  \Mgamma{\tupletype{x : A, B}} & = &
    \Sigma{F}, \mbox{ where }\\& &
     F \mbox{ maps } z \in \Mgamma{A} \mbox{ to }\\& & \mathcal{M}(\Gamma, x : \ttvar{}\
A\vbar \gamma \{x\gets z \})(B)\\
%
  \Mgamma{\funtype{x:A}{B}} & = &  \Pi{F}, \mbox{ where }\\& &
                        F \mbox{ maps } z \in \Mgamma{A} \mbox{ to }\\& &  \mathcal{M}(\Gamma,
x : \ttvar{}\  A\vbar \gamma\{x\gets z\})(B)  
%    
\end{eqnarray*}
\end{Defn}

\begin{Eg}{meaning function with dependent types}
\begin{eqnarray*}
\Mgamma{\tupletype{x : \ttbool,  \{ y : \ttbool \vbar x \supset 
y\}}}
& = & \{\pair{\zerob, \zerob}, \pair{\zerob, \oneb},
\pair{\oneb, \oneb}\}\\ 
\Mgamma{\funtype{x : \ttbool}{\{ y : \ttbool \vbar x \supset 
y\}}} & = & \{\{\pair{\zerob, \zerob}, \pair{\oneb, \oneb}\},\\&&\hspace*{1ex}
              \{\pair{\zerob, \oneb}, \pair{\oneb, \oneb}\}\}
\end{eqnarray*}
\end{Eg}

We now need to show that the extensions corresponding to dependent types
preserve the properties in Theorems~\ref{subtype-type-semantics}
and~\ref{subtype-term-semantics}, namely, 
$\Mgamma{T} \in U$ and $\Mgamma{a}\in \Mgamma{\tauGamma{a}}$\@.
For the former, we prove a stronger theorem that incorporates the
rank-boundedness of dependent types.
\begin{theorem}[rank bounded type semantics]\label{dependent-type-semantics}
If $B$ is a pretype, $x_1,\ldots, x_n$ is a list of symbols,
$A_1,\ldots, A_n$ is a list of pretypes such that  
\begin{enumerate}
\item $\tau()(\Gamma, x_1:\ttvar~A_1, \ldots, x_n:\ttvar~A_n) = \ttcontext$,
\item $\tau(\Gamma, x_1:\ttvar~A_1, \ldots, x_n:\ttvar~A_n)(B) = \tttype$,
and
\item $\gamma$ is an assignment satisfying $\Gamma$,
\end{enumerate}
then
there is an $i$ such that for any list of values
$z_1,\ldots, z_n$ where $\gamma\{x_1\gets
z_1\}\ldots\{x_n\gets z_n\}$ is a satisfying assignment for
$\Gamma, x_1:\ttvar~A_1, \ldots, x_n:\ttvar~A_n$,  we have
$$\mathcal{M}(\Gamma, x_1:\ttvar~A_1, \ldots, x_n:\ttvar~A_n\vbar
\gamma\{x_1\gets z_1\}\ldots\{x_n\gets z_n\})(B) \in U_i.$$
\end{theorem}
\begin{proof}
The proof is by structural induction on the pretype $B$\@.
Let $\Gamma'$ denote $\Gamma, x_1:\ttvar~A_1, \ldots, x_n:\ttvar~A_n$,
$\gamma'$ denote $\gamma\{x_1\gets z_1\}\ldots\{x_n\gets z_n\}$, and 
$\mean{C}$ denote $\mathcal{M}(\Gamma'\vbar
\gamma')(C)$.  
\begin{enumerate}
\item $B\equiv s$: Since $\mean{B}$ is just $\gamma(B)$ by
Definition~\ref{defn:meaning function}, we have that there is an $i$
such that $\mean{B}\in U_i$ regardless of the choice of values
$z_1,\ldots, z_n$\@.  

\item $B\equiv \{y:T\vbar a\}$: By the induction hypothesis, we know that
for some $j$, it is always the case that $\mean{T}\in U_j$\@.  By
Definition~\ref{defn:meaning function with subtypes}, we have that
$\mean{B}\subseteq \mean{T}$ so if we let $i = j+1$, then
by Definition~\ref{defn:type universe with dependent types},
it is always the case that $\mean{B}\in U_i$\@.

\item $B \equiv \funtype{y:C}{D}$: By
Definition~\ref{defn:type rules with dependent types},
$\Gamma'(y)$ is undefined, $\tau(\Gamma')(C) = \tttype$,
$\tau()(\Gamma', y:\ttvar~C) = \ttcontext$, and
$\tau(\Gamma', y:\ttvar~C)(D) = \tttype$.
By the induction hypothesis, for some $j$,
it is always the case that $\mathcal{M}(\Gamma'\vbar\gamma')(C)\in U_j$,
and for some $k$, it is always the case that for any satisfying
assignment $\gamma'\{y\gets w\}$ for $\Gamma', y:\ttvar~C$,
we have $\mathcal{M}(\Gamma', y:\ttvar~C\vbar \gamma'\{y\gets
w\})(D)\in U_k$\@.   Then the function $F$ mapping $w$ in
$\mathcal{M}(\Gamma')(C)$ to $\mathcal{M}(\Gamma', y:\ttvar~C\vbar
\gamma'\{y\gets w\})(D)$ is an element of $W_{j+k}$\@.
Letting $i$ be $j+k+1$, we have by
Definition~\ref{defn:meaning function with dependent types}
that $\mathcal{M}(\Gamma'\vbar\gamma')(B)$ is
$\Pi{F}$ and is hence an element of $U_i$ by
Definition~\ref{defn:type universe with dependent types}\@.

\item $B\equiv \tupletype{y: C, D}$:  Similar to the
previous case.  
\end{enumerate}
\end{proof}

By choosing $n$ to be $0$, the previous theorem yields the result 
that when $\tauGamma{B} = \tttype$, $\Mgamma{B}\in U$\@.

We next need to establish that for any preterm $a$,
if $\tauGamma{a} = A$, then $\Mgamma{a}\in \Mgamma{A}$\@.
The first step in this direction is the proof of the substitution lemma
below.
\begin{prop}\label{context-equivalence}
If $\tau()(\Gamma) = \tau()(\Gamma') = \ttcontext$ where
for each $s$, $\Gamma(s)$ is defined if and only if  $\Gamma'(s)$ is defined, and 
$\gamma$ is an assignment satisfying both $\Gamma$ and $\Gamma'$,
then
\begin{enumerate}
\item If  $\Gamma(s) = \ \Gamma'(s)$ (i.e., they are equal when
either $\Gamma(s)$ or $\Gamma'(s)$ is defined), then 
\begin{enumerate}
\item $\tau(\Gamma)(a) = \tau(\Gamma')(a)$, for any preterm $a$.
\item $\tau(\Gamma)(A) = \tau(\Gamma')(A)$, for any pretype $A$.
\end{enumerate}

\item $\Mgamma{A} = \mathcal{M}(\Gamma'\vbar\gamma)(A)$, when $\tauGamma{A} =
\tttype$.

\item $\Mgamma{a} = \mathcal{M}(\Gamma'\vbar\gamma)(a)$, for any preterm $a$
such that $\tauGamma{a}$ is defined.
\end{enumerate}
\end{prop}
%
%%\memo{Needs induction on preterms and pretypes.}
\begin{lemma}[substitution lemma]\label{substitution-lemma}
If $\tau()(\Gamma, x:\ttvar~A) = \ttcontext$,
$\tauGamma{a} = A$, then
\begin{enumerate}
\item If $\tau(\Gamma, x: \ttvar~A)(b) = B$, then\\
$\Mgamma{b[a/x]} = \mathcal{M}(\Gamma, x:\ttvar~A\vbar \gamma\{x\gets
\Mgamma{a}\})(b)$.

\item If $\tau(\Gamma, x: \ttvar~A)(C) = \tttype$, then\\
$\Mgamma{C[a/x]} = \mathcal{M}(\Gamma, x:\ttvar~A\vbar \gamma\{x\gets
\Mgamma{a}\})(C)$\@. 
\end{enumerate}
\end{lemma}
\begin{proof}
The proof is by simultaneous structural induction on the preterm $b$ and
the pretype $C$\@.  The following cases deal with the preterm $b$.
\begin{enumerate}
\item $b\equiv s$: If $s\equiv x$, then by
Definition~\ref{defn:meaning function with dependent types}, the left-hand
side $\Mgamma{b[a/x]}$ is 
$\Mgamma{a}$, and the right-hand side  $\mathcal{M}(\Gamma, x:\ttvar~A\vbar \gamma\{x\gets
\Mgamma{a}\})(b)$ is also $\Mgamma{a}$\@. 

If $s\not\equiv x$, then by
Definition~\ref{defn:meaning function with dependent types},
the left-hand side and the right-hand side
are both equal to $\gamma(s)$\@. 

\item $b \equiv (\lambda (y : C): d)$: Since $C$ can contain free
occurrences of $x$, we have by the induction hypothesis that
$\Mgamma{C[a/x]} = \mathcal{M}(\Gamma, x:\ttvar~A\vbar \gamma\{x\gets
\Mgamma{a}\})(C)$\@.  Also, $\Mgamma{(\lambda (y:C): d)[a/x]}$ is equal to
the set of ordered pairs $\pair{v, z}$ such that $v\in \Mgamma{C[a/x]}$
and $z = \mathcal{M}(\Gamma, y:\ttvar~C[a/x]\vbar \gamma\{y\gets
v\})(d[a/x])$\@.

By the induction hypothesis,
$\mathcal{M}(\Gamma, y:\ttvar~C[a/x]\vbar \gamma\{y\gets v\})(d[a/x])
= \mathcal{M}(\Gamma, y:\ttvar~C[a/x], x:\ttvar~A\vbar \gamma\{y\gets
v\}\{x\gets \Mgamma{a}\})(d)$\@.  Since $x$ does not occur free in
$C[a/x]$, by Proposition~\ref{context-equivalence} we can exchange the
occurrences of $y$ and $x$ so that 
$\mathcal{M}(\Gamma, y:\ttvar~C[a/x], x:\ttvar~A\vbar \gamma\{y\gets
v\}\{x\gets \Mgamma{a}\})(d) =
\mathcal{M}(\Gamma, x:\ttvar~A, y:\ttvar~C[a/x]\vbar \gamma\{x\gets \Mgamma{a}\}\{y\gets
v\})(d)$.

By Definition~\ref{defn:meaning function with dependent types},
the right-hand side is the set of ordered pairs of the form $\pair{v, z}$
such that $v\in \mathcal{M}(\Gamma, x: \ttvar~A \vbar \gamma\{x\gets
\Mgamma{a}\})(C)$
and $z = \mathcal{M}(\Gamma, x: \ttvar~A, y: \ttvar~C\vbar \gamma\{x\gets
\Mgamma{a}\}\{y\gets v\})(d)$\@.  By Proposition~\ref{context-equivalence}
and the induction hypothesis, we know that
$\mathcal{M}(\Gamma, x: \ttvar~A, y: \ttvar~C\vbar \gamma\{x\gets
\Mgamma{a}\}\{y\gets v\})(d) =
\mathcal{M}(\Gamma, x:\ttvar~A, y:\ttvar~C[a/x]\vbar \gamma\{x\gets \Mgamma{a}\}\{y\gets
v\})(d)$, and hence it follows that the two sets of ordered pairs are
equal.

\item $b\equiv (f\ c)$:  In this case, $b[a/x]\equiv (f[a/x]~c[a/x])$
and the conclusion follows easily from the induction hypothesis
and Definition~\ref{defn:meaning function with dependent types}\@.

\item $b\equiv (b_1, b_2)$: 
The conclusion follows easily from 
Definitions~\ref{defn:substitution},~\ref{defn:meaning function with
dependent types}, and the induction hypotheses.


\item $b\equiv (\proj{i}~c)$: This case is also straightforward since
$b[a/x] \equiv (\proj{i}~c[a/x])$,  
and by the induction hypothesis,
$\mathcal{M}(\Gamma, x: \ttvar~a \vbar \gamma\{x \gets \Mgamma{a}\})(c) =
\Mgamma{c[a/x]}$.  
%
\end{enumerate}

The remaining cases deal with the pretype $C$\@.
\begin{enumerate}
\item $C\equiv s$:  This case is trivial since by
Definition~\ref{defn:substitution}, $C[a/x]\equiv C$ and
the left-hand and right-hand sides both reduce to $\gamma(C)$\@.

\item $C\equiv \{y : T \vbar d\}$: The argument here follows along the
lines of the $b\equiv (\lambda (x : C): D)$ case above.
By the induction hypotheses, we know that
\begin{eqnarray*}
\lefteqn{\mathcal{M}(\Gamma, x : \ttvar~A\vbar \gamma\{x\gets
\Mgamma{a}\})(T)}\\
& = & \Mgamma{T[a/x]}\\
\lefteqn{\mathcal{M}(\Gamma, y : \ttvar~T[a/x], x : \ttvar~A\vbar \gamma\{y\gets
z\}\{x\gets \Mgamma{a}\})(d)}\\
& = & \mathcal{M}(\Gamma, y : \ttvar~T[a/x]\vbar \gamma\{y\gets z\})(d[a/x]),\\&&
\mbox{ for any } z\in \Mgamma{T[a/x]}
\end{eqnarray*}
The conclusion follows from Proposition~\ref{context-equivalence}
and Definition~\ref{defn:meaning function with dependent types}.

\item $C\equiv \funtype{y:C_1}{C_2}$: The argument here is similar
to that of the previous case.  Essentially, by the induction hypothesis
and Proposition~\ref{context-equivalence}, the function mapping
$z\in \mathcal{M}(\Gamma, x : \ttvar~A\vbar \gamma\{x\gets \Mgamma{a})(C_1)$
to $\mathcal{M}(\Gamma, y:\ttvar~C_1[a/x], x : \ttvar~A\vbar \gamma\{y\gets
z\}\{x\gets \Mgamma{a}\})(C_2)$ is the same as the function mapping
$z\in \Mgamma{C_1[a/x]}$ to $\mathcal{M}(\Gamma, y: \ttvar~C_1[a/x]\vbar
\gamma\{y\gets z\})(C_2[a/x])$\@.  

\item $C\equiv \tupletype{y:C_1, C_2}$:   Similar to the previous case.
\end{enumerate}
\end{proof}

Proposition~\ref{substitution-equivalence} is stated below without proof.
It asserts the semantic equivalence with respect to term $a$
of types $A$ and $B$ when $(A\sima B)_\Gamma$ holds.   Note that its
correctness depends on the soundness of the proof rules.  

\begin{prop}\label{substitution-equivalence}
If $\tau()(\Gamma) = \ttcontext$,  $a$ is a preterm such that
$\tauGamma{a} = B$, and $(A\sima  B)_\Gamma$,
then $\Mgamma{a}\in \Mgamma{A}$ iff $\Mgamma{a}\in \Mgamma{B}$\@.  
\begin{comment}
\begin{enumerate}
\item $\Mgamma{a}\in \Mgamma{A}$ iff $\Mgamma{a}\in \Mgamma{a/A}$
\item $\vdash \pi(a/A)(a)$ iff $\vdash \pi(A)(a)$\@.
\end{enumerate}
\end{comment}
\end{prop}
%
\begin{theorem}\label{dependent-term-semantics}
If $\tau()(\Gamma) = \ttcontext$, $\gamma$ is an assignment satisfying
$\Gamma$, and $a$ is a preterm such that $\tauGamma{a} = A$, then
$\Mgamma{a}\in \Mgamma{A}$\@.   
\end{theorem}
\begin{proof}
The proof is by  induction on the structure of the preterm
$a$\@.
\begin{enumerate}
\item $a\equiv s$: Then  by
Definition~\ref{defn:meaning function with dependent types},
$\Mgamma{a} = \gamma(a)$, and by Definition~\ref{defn:satisfaction},
we have that $\gamma(a)\in \Mgamma{A}$\@.

\item $a \equiv (\lambda (x : C): b)$:  By
Definition~\ref{defn:type rules with dependent types}, we
have $\tauGamma{a} = A = \funtype{x : C}{\tau(\Gamma, x: \ttvar~C)(b)}$\@.
Let $B$ label $\tau(\Gamma, x: \ttvar~C)(b)$\@.  
We know that $\Mgamma{A}$ is of the form $\Pi{} F$ where
$F$ maps $z\in \Mgamma{C}$ to $\mathcal{M}(\Gamma, x : \ttvar~C\vbar \gamma\{x
\gets z\})(B)$\@.

By the induction hypothesis on $b$, we know that for any $z\in \Mgamma{C}$, 
$\mathcal{M}(\Gamma, x : \ttvar~C\vbar \gamma\{x \gets z\})(b)\in
\mathcal{M}(\Gamma, x : \ttvar~C\vbar \gamma\{x
\gets z\})(B)$\@.  
Since by Definition~\ref{defn:meaning function with dependent types},
$\Mgamma{a}$ is a function mapping $z\in \Mgamma{C}$ to
$\mathcal{M}(\Gamma, x : \ttvar~C\vbar \gamma\{x \gets z\})(b)$, we have
$\Mgamma{a} \in \Pi F$ by the
definition of $\Pi$\@.  

\item $a \equiv (f~b)$: By
Definition~\ref{defn:type rules with dependent types}, we have that
$\tauGamma{f} = \funtype{x : B}{A'}$, $\tauGamma{b} = B'$,
$(B \sima  B')_\Gamma$,  $A \equiv A'[a/x]$, and
$\vdash_\Gamma\pi(B)(b)$\@.  We know by the induction hypothesis
that $\Mgamma{f}\in \Mgamma{\funtype{x : B}{A'}}$ and
$\Mgamma{b} \in \Mgamma{B'}$\@.  By
Propositions~\ref{substitution-equivalence}
and~\ref{maximal-subtype},
$\Mgamma{b}\in \Mgamma{\mu(B)}$\@.  
We therefore have 
by Proposition~\ref{maximal-subtype} that $\Mgamma{b}\in \Mgamma{B}$\@.
By Definition~\ref{defn:meaning function with dependent types},
$\Mgamma{a}\in \mathcal{M}(\Gamma, x: \ttvar~B\vbar \gamma\{x\gets
\Mgamma{b}\})(A')$,
and hence  by Lemma~\ref{substitution-lemma} it follows that
$\Mgamma{a}\in \Mgamma{A'[b/x]}$\@.  

\item $a \equiv (\listwo{a}{n})$:  The conclusion follows easily from
the induction hypothesis and
Definition~\ref{defn:type rules with dependent types}\@.

\item $a\equiv (\proj{i}~b)$: The conclusion follows easily from
Proposition~\ref{maximal-subtype}, the  induction hypothesis,
and Definition~\ref{defn:type rules with dependent types}\@.
The $(\proj{2}~b)$ case also employs Lemma~\ref{substitution-lemma}\@.  
\end{enumerate}
\end{proof}

\begin{comment}
\begin{lemma}\label{rank-boundedness}
If $\tau()(\Gamma) = \ttcontext$,
$\gamma$ is a satisfying assignment for $\Gamma$, 
$\tauGamma{A} = \tttype$,
$\Mgamma{A} \in U$, 
and $\tau(\Gamma, x: \ttvar~A)(B) = \tttype$, then
there is an $i$ such that the function $F$ mapping
$z$ in $\Mgamma{A}$ to 
$\mathcal{M}(\Gamma, x : \ttvar~A\vbar \gamma\{x\gets z\})(B)$
is in $W_i$\@.  
\end{lemma}
\begin{proof}
The proof is by induction on the structure of the pretype $B$\@.
Let the rank of $\Mgamma{A}$ be $j$\@.  
\begin{enumerate}
\item $B\equiv s$: Then $\mathcal{M}(\Gamma, x : \ttvar~A\vbar \gamma\{x\gets
z\})(B)$ is just $\gamma(s)$ and by Definition~ref{defn:satisfaction},
we have some $k$ such that $\gamma(s)\in U_k$\@.  Let $i$ be the maximum
of $j$ and $k$, we have $F\in U_i$\@.
%
\item $B\equiv \funtype{y:C}{D}$:  We have the induction hypothesis
%
\end{enumerate}
\end{proof}
%
\begin{theorem}\label{dependent-type-semantics}
If $\tau()(\Gamma) = \ttcontext$,  $\gamma$ satisfies $\Gamma$,
and $\tauGamma{A} = \tttype$, 
then 
\begin{enumerate}
%\item  $\Mgamma{A}\in U$, and
%%\item if $\tauGamma{b} = B$ then $\Mgamma{b} \in \Mgamma{B}$, and
\item if for all $n$, for $i$,  $1\leq i\leq n$,  $\mathcal{M}(\gamma,
x_1:\ttvar~A_1, \ldots x_{i-1}:\ttvar~A_{i-1}\vbar {A_i})\in U$ and 
$\tau(\Gamma, x_1 : \ttvar~A_1, \ldots, x_n:\ttvar~A_n)(B) = \tttype$,
then $\mathcal{M}(\Gamma, x_1 : \ttvar~A_1\ldots\vbar \gamma\{x\gets z\})(B)$
there exists an $i$ such that 
whenever $F$ maps $z$ in $\Mgamma{A}$ to $\mathcal{M}(\Gamma, x :
\ttvar~A\vbar \gamma\{x\gets z\})(B)$, there exists an $i$ such
that $F\in W_i$\@.  
\end{enumerate}
\end{theorem}
\begin{proof}
The proof is by simultaneous induction on the sizes of pretypes $A$ and
$B$\@.  We present the structure of the entire proof here but refer to
the proofs of Theorems~\ref{simple-type-semantics}
and~\ref{subtype-type-semantics} as needed.
\begin{enumerate}
\item $A\equiv s$: Same as in Theorem~\ref{simple-type-semantics}\@.
%
\item $A\equiv \funtype{y : C}{D}$: We have by the induction hypothesis
that $\Mgamma{C}\in U$ and for any $z\in \Mgamma{C}$, $\mathcal{M}(\Gamma,
y:\ttvar~C\vbar \gamma\{y\gets z\})(D) \in U$\@.  We also have by the
induction hypothesis with $D$ for $B$ that there is an $i$ such that
the function $F$ mapping $z\in \Mgamma{C}$ to $\mathcal{M}(\Gamma,
y:\ttvar~C\vbar \gamma\{y\gets z\})(D)$ is in $W_i$\@.
Since, by Definition~\ref{defn:meaning function with dependent types},
$\Mgamma{A} = \Pi{F}$, we have $\Mgamma{A}$ in $U_{i+1}$ and hence $U$\@.
%
\item $A\equiv \tupletype{x_1 : T_1,\ldots, x_n : T_n}$:  We have by the
induction hypothesis that $\Mgamma{T_1}\in U$ and that there is an $i$
such that 
the function $F$ mapping $z$ in $\Mgamma{T_1}$ to
$\mathcal{M}(\Gamma, x: \ttvar~T_1\vbar \gamma\{x\gets z\})(\tupletype{x_2:
T_2, \ldots x_n: T_n})$  is in $W_i$\@.  Since, by
Definition~\ref{defn:meaning function with dependent types},
$\Mgamma{A} = \Sigma{F}$,
we have by Definition~\ref{defn:type universe with dependent types},
that $\Mgamma{A}\in U_{i+1}$ and hence $\Mgamma{A}\in U$\@.
%
\item $A\equiv \{y : T | a\}$: Same as in
Theorem~\ref{subtype-type-semantics}\@.
%
\item $B \equiv s$:  Let $j$ be the rank of $\Mgamma{A}$\@.  
Then $\mathcal{M}(\Gamma, x : \ttvar~A\vbar \gamma\{x\gets
z\})(B)$ is just $\gamma(s)$ and by Definition~ref{defn:satisfaction},
we have some $k$ such that $\gamma(s)\in U_k$\@.  Let $i$ be the maximum
of $j$ and $k$, we have $F\in U_i$\@.
%
\item $B\equiv \funtype{x: C}{D}$:  
\end{enumerate}
\end{proof}
\end{comment}

\section{Summary}

Dependent typing is a significant enhancement to PVS since it adds
an important degree of flexibility and precision to the type system.
Notions  such as subtype constraints and type equivalence
that were introduced for subtyping can be extended for the case of
dependent types.  The semantic universe must be extended to include
additional sets to accommodate the semantics of dependent types.
The rank-boundedness of type dependencies is crucial in demonstrating that
dependent types can be interpreted in this extended semantic universe.
  






\chapter{Theories and Parametric Theories}\label{theories}

The next extension of the PVS language introduces theories and parametric
theories.  
The theory construct of PVS provides a way of packaging together a related
collection of declarations.  Theories can be parametric in
individual or type parameters.  Thus, PVS permits polymorphism or type
parametricity only at the theory level rather than at the declaration
level as in HOL~\cite{Gordon&Melham:HOL}.  
We first consider PVS theories without parameters.  The main change now is
that contexts are no longer simple and can contain theory declarations as
well.  A theory declaration has the form $m : \tttheory{} = \Delta$,
where $\Delta$ is a \emph{simple context} with no variable or theory
declarations. 
If $\Gamma(m)$ is the declaration $m : \tttheory{} = \Delta$, then
$ \itkind{}(\Gamma(m)) = \tttheory{}$, and
$ \itdef{}(\Gamma(m)) = \Delta$\@.
Correspondingly, constants and type names are no longer just symbols
but can be compound names of the form $m.s$ where $m$ is a symbol
naming a theory and $s$ is a symbol corresponding to the constant or type
name.

\section{Theories without Parameters}

To define the type rules for theories, we first modify the definition of
$\tau$ for simple contexts so that the context argument is not always
empty.  Here $\Delta ; \Gamma$ represents the concatenation of contexts.
\begin{Defn}{type rules for contexts}
\begin{eqnarray*}
  %
  \tau(\Theta)(\{\}) & = & \ttcontext \\
%
  \tau(\Theta)(\Gamma, s\ :\ \tttype{} = T) & = & \ttcontext, \mbox{
if } \Gamma(s) \mbox{ and } \Theta(s) \mbox{ are undefined,}\\& & 
  \tau(\Theta)(\Gamma) = \ttcontext, \mbox{and} \\& &
  \tau(\Theta;\Gamma)(T) = \tttype{}\\
%
  \tau(\Theta)(\Gamma, c : T) & = & \ttcontext, \mbox{ if }
\Gamma(c) \mbox{ and } \Theta(c)  \mbox{ are undefined,}\\ & &
\tau(\Theta)(\Gamma) = \ttcontext, \mbox{and }\\
   & &  \tau(\Theta;\Gamma)(T) =
      \tttype{}\\
%
%
  \tau(\Theta)(\Gamma, x : \ttvar{}\ T) & = & \ttcontext, \mbox{ if }
\Gamma(x) \mbox{ and } \Theta(x) \mbox{ are undefined,} \\ & &
\tau(\Theta)(\Gamma) = \ttcontext, \mbox{and }\\
    & &  \tau(\Theta;\Gamma)(T)
    = \tttype{}
%
\end{eqnarray*}
\end{Defn}
\begin{Eg}{type rules for contexts}
\begin{eqnarray*}
\tau(\Omega)(\ttreal{} : \tttype{}, \mathtt{0} : \ttreal{},
\leq : \funtype{\tupletype{\ttreal{}, \ttreal{}}}{\ttbool}) & = & \ttcontext
\end{eqnarray*}
\end{Eg}

The following rule handles theory declarations.
\begin{Defn}{type rule for contexts with theory declarations}
\begin{eqnarray*}
  \tau(\Theta)(\Gamma, m : \tttheory{} = \Delta) & = & \ttcontext
     \mbox{ if } \Theta(m), \Gamma(m) \mbox{ are undefined} \\& &
     \Delta \mbox{ only has constant and type declarations,}\\ & &
     \tau(\Theta ; \Gamma)(\Delta)  = \ttcontext,\\& &
     \tau(\Theta)(\Gamma) = \ttcontext
\end{eqnarray*}
\end{Defn}
\begin{Eg}{contexts with theory declarations}
\begin{eqnarray*}
\lefteqn{\tau(\Omega)(
\mathtt{reals} : \tttheory{} = (\ttreal{} : \tttype{}, \mathtt{0} : \ttreal{},
\leq : \funtype{\tupletype{\ttreal{},
\ttreal{}}}{\ttbool})) }\\
& = & 
\ttcontext
\end{eqnarray*}

\end{Eg}


\begin{comment}
Perhaps each declaration should be required to introduce a new
symbol so that there is no confusion between $\Gamma$ and $\gamma$.
\end{comment}

\begin{comment}
Example:
bool: TYPE,
integer : THEORY = (integer: TYPE, + :
\funtype{\tupletype{integer, integer}}{ {\tt integer}}),
natural : THEORY = (nat : TYPE)
\end{comment}

Any reference to a type name or a constant $s$ declared in a theory $m$
outside of this theory must be prefixed by the theory name, as in $m.s$\@.
Note that references to a type name or constant that is declared in the
same theory 
should not be given a theory prefix.  Before we can give the type rules, we
must update the definition of the type expansion
operation $\delta$ to prefix symbols with their theory names.
Let $\Gamma(m)(s)$  abbreviate $ \itdef{}(\Gamma(m))(s)$, which is
the declaration of the symbol $s$ in the definition of the theory $m$\@.
Let $\eta(\Gamma, m)(a)$ be the result of prefixing every unprefixed type or
constant symbol in $a$ by $m$, where $a$ is either an individual or type
expression.  We omit the definition of $\eta$ since it is straightforward.
  
%The $\delta$ operation is modified so that its first argument is now a
%pair consisting of a context and the `current' theory name.  The theory
%name component can be empty, as in Definition~\ref{defn:expanded type}\@.
We modify the definition of $\delta$ in Definition~\ref{defn:expanded
type} with the following clauses.  
\begin{Defn}{expanded type for prefixed symbols}
\begin{eqnarray*}
  \delta(\Gamma)(m.s) & = & \delta(\Gamma)(
\eta(\Gamma, m)(\itdef{}(\Gamma(m)(s)))), 
                        \mbox{ if }\\& &  \itdef{}(\Gamma(m)(s)) \mbox{ is
nonempty.}\\
  \delta(\Gamma)(m.s) & = & m.s
                        \mbox{ if } \itdef{}(\Gamma(m)(s)) \mbox{ is
empty.}
%\\
% \delta(\Gamma, m)(s) & = & m.s \mbox{ if } 
%\itdef{}(\Gamma(m)(s)) \mbox{ is
%empty.  }
\end{eqnarray*}
\end{Defn}
\begin{Eg}{expanded type for prefixed symbols}
Let $\Omega''$ be the context
$$\Omega, 
\mathtt{reals} : \tttheory{} = \begin{array}[t]{l}
				   (\ttreal{} : \tttype{},\\\
                                    \mathtt{0} : \ttreal{},\\\ 
                        \leq : \funtype{\tupletype{\ttreal{},
              \ttreal{}}}{\ttbool}, \\\ 
        \mathtt{nonneg\_real} : \tttype{} = \{x : \ttreal{} \vbar
\leq(0, x)\},\\\  \mathtt{1} : \mathtt{nonneg\_real} )
				   \end{array}$$

$$\delta(\Omega'')( \mathtt{reals.nonneg\_real}) =
\{ {x} : \mathtt{reals.real} \vbar
\mathtt{reals}.\!\leq( \mathtt{reals.0}, {x})\}$$
\end{Eg}

The type rules for prefixed symbols are given below.
\begin{Defn}{type rules for prefixed symbols}
\begin{eqnarray*}
  \tauGamma{m.s} & = & \tttype{}, \mbox{ if } \itkind{}(\Gamma(m)) =
\tttheory{} \mbox{ and }\\ & &
\itkind{}(\Gamma(m)(s)) = \tttype{}\\
%
  \tauGamma{m.s} & = & \delta(\Gamma)(\eta(\Gamma, m)(\ittype(\Gamma(m)(s)))), \\ &
&
\mbox{ if } \itkind{}(\Gamma(m)) =
\tttheory{} \mbox{ and }\\ & &
\itkind{}(\Gamma(m)(s)) = \ttconstant{}
\end{eqnarray*}
\end{Defn}

\begin{Eg}{type rules for prefixed symbols}
\begin{eqnarray*}
\tau(\Omega'')(\mathtt{reals.nonneg\_real}) & = & \tttype{}\\
\tau(\Omega'')(\mathtt{reals.1}) & = &
\{ {x} : \mathtt{reals.real} \vbar
\mathtt{reals}.\!\leq( \mathtt{reals.0}, {x})\}
\end{eqnarray*}
\end{Eg}

\begin{comment}
\begin{Eg}{contexts with theory declarations}
\begin{eqnarray*}
\tau(\Omega)(
\begin{array}[t]{l}
\mathtt{reals} : \tttheory{} = (\ttreal{} : \tttype{}, \mathtt{0} : \ttreal{},
\leq : \funtype{\tupletype{\ttreal{},
\ttreal{}}}{\ttbool}), \\
\mathtt{nonneg\_reals}: \tttheory{}  = \\\ \
   ( \mathtt{nonneg\_real} :
\tttype{} = \{ {x} : \mathtt{reals.real} \vbar
\mathtt{reals.\leq}( \mathtt{reals.0}, {x})\},\\\ \  \mathtt{sqrt}: \funtype{
\mathtt{nonneg_real}}{ \mathtt{nonneg_real}}\} )) 
\end{array}
& = & 
\ttcontext
\end{eqnarray*}
%
\end{Eg}
\end{comment}




The operations  $\pi$, and $\mu$ remain unchanged.
  An assignment $\gamma$ now
maps a theory name $m$ to an assignment $\gamma(m)$.
\begin{Defn}{meaning function for prefixed symbols}
\begin{eqnarray*}
  \Mgamma{m.s} & = & \gamma(m)(s)
\end{eqnarray*}
\end{Defn}
\begin{Eg}{meaning function for prefixed symbols}
Let $\omega''$ be a satisfying assignment for $\Omega''$ of the form
$$\ldots \{ \mathtt{reals}\gets \{\ttreal{}\gets
\reals\}\{\mathtt{0}\gets \zerob\}\ldots \} \ldots.$$

\begin{eqnarray*}
\mathcal{M}(\Omega''\vbar\omega'')(\mathtt{reals.real}) & = & \reals\\
\mathcal{M}(\Omega''\vbar\omega'')(\mathtt{reals.0}) & = & \zerob
\end{eqnarray*}
\end{Eg}



\begin{Defn}{satisfaction for contexts with theories}
An assignment~$\gamma$ satisfies a context $\Gamma$ if in addition to the 
constraints stated in
Definition~\ref{defn:satisfaction with type definitions}, $\gamma$ maps 
every theory $m$  declared in $\Gamma$ to 
a satisfying assignment for the body of the theory given by $
\itdef{}(\Gamma(m))$, that is for each declared symbol $s$ in $m$:
\begin{enumerate}
\item If $ \itkind{}(\Gamma(m)(s)) = \tttype{}$, then $\gamma(m)(s)
\in U$.


\item If $\itkind{}(\Gamma(m)(s)) = \ttconstant{}$, then $\gamma(m)(s)
\in \mathcal{M}(\Gamma\vbar\gamma)(\tau(\Gamma)(m.s))$.

\item  If  $\itdef{}(\Gamma(m)(s))$ is nonempty,
then $$\gamma(m)(s) = \mathcal{M}(\Gamma| \gamma)(
\eta(\Gamma, m)(\itdef{}(\Gamma(m)(s)))).$$
\end{enumerate}
\end{Defn}


\section{Constant Definitions}  We first extend the subset of PVS
described so far to include 
constant definitions in a manner similar to type definitions.
This extension is used in formalizing the semantics of parametric theories.
The syntax
for a constant definition is $c : T = a$ where $\itdef{}(\Gamma(c))$ is
$a$\@.  These definitions are explicit, that is, not recursive.  With this
extension, the type rule for constant declarations in contexts changes
from that of Definition~\ref{defn:type rules with subtypes}.
\begin{Defn}{type rule with constant definitions}
\begin{eqnarray*}
\tau(\Theta)(\Gamma, c : T = a) & = & T, \mbox{ if } \Gamma(c) \mbox{ is
undefined}, \\& & \Theta(c) \mbox{ is undefined}, \\& &
\tau(\Theta)(\Gamma) = \ttcontext,\\& &
%\vdash_\Gamma (\exists(x : T) : \tttrue),\\& &
\tau(\Theta;\Gamma)(a) = T',\\& & 
(T\sim T')_\Gamma,\\& &
\vdash_\Gamma \pi(T)(a)
\end{eqnarray*}
\end{Defn}

The notion of satisfaction must be extended from that of
Definition~\ref{defn:satisfaction for contexts with theories} to ensure
that an 
assignment 
for a defined constant satisfies the definition.
\begin{Defn}{satisfaction with constant definitions}
An assignment $\gamma$ satisfies a context $\Gamma$ if
in addition to the conditions in
Definition~\ref{defn:satisfaction for contexts with theories}, whenever $
\textit{kind}(\Gamma(s)) = \ttconstant$ and 
$\textit{definition}(\Gamma(s))$ is nonempty, then
$\gamma(s) = \Mgamma{ \textit{definition}(\Gamma(s))}$.
\end{Defn}


\section{Parametric Theories}

The extension to parametric theories is obtained by permitting
theories to be declared as $ m[\Pi] : \tttheory{} = \Delta$,
where $\Pi$ is a context listing the parameters and $\Delta$ is
the body of the theory.  If the above declaration of $m$ occurs
in context $\Gamma$, then $\Pi$ is $ \emph{formals}(\Gamma(m))$, and
$\Delta$ is $ \itdef{}(\Gamma(m))$.
For nonparametric theories, $ \mathit{formals}(\Gamma(m))$
is empty.   
Types or constants declared in a parametric theory are
referenced outside the theory as $m[\sigma].s$, where $\sigma$
is a list of \emph{actual} parameters consisting of types and terms.  
The type rule from the nonparametric case
must be modified to check the parameters.
\begin{Defn}{type rule for contexts with parametric theories}
\begin{eqnarray*}
  \lefteqn{\tau(\Theta)(\Gamma, m[\Pi] : \tttheory{} = \Delta)}\\ & = & \ttcontext
     \mbox{ if } \Gamma(m), \Theta(m), \Pi(m) \mbox{ are undefined}\\& &
      \tau(\Theta)(\Gamma) = \ttcontext\\& &
      \tau(\Theta ; \Gamma)(\Pi)  = \ttcontext,  \\ & & 
      \Pi \mbox{ has only constant and}\\& & \mbox{type declarations
without definitions,}\\ & &
      \tau(\Theta ; \Gamma ; \Pi)(\Delta)  = \ttcontext \\& &
      \mbox{ $\Delta$ only has type and constant declarations}
\end{eqnarray*}
\end{Defn}

  The 
 type  rules for prefixed symbols are given below.
The notation $\Pi = \sigma$, where $\Pi$ is of the form
$s_1 : \alpha_1,\ldots, s_n : \alpha_n$, and $\sigma$ is of
the form $\sigma_1,\ldots, \sigma_n$, is short for
the context $s_1 : \alpha_1 = \sigma_1, \ldots, s_n : \alpha_n =
\sigma_n$\@.  The definition of $\eta$ is now extended to substitute
actual theory parameters for formals, so that $\eta(\Gamma, m[\sigma])(a)$
prefixes every unprefixed symbol $s$ in $a$ that is declared in
$\itdef(\Gamma(m))$ by $m[\sigma]$, and replaces any $s_i$ in $a$
that is declared in $\itformals(\Gamma(m))$ by the corresponding
$\sigma_i$ in $\sigma$\@.  

\begin{Defn}{type rules for prefixed names with actuals}
Let $\Pi$ be $\mathit{formals}(\Gamma(m))$\@.  
\begin{eqnarray*}
  \tauGamma{m[\sigma].s} & = & \tttype{}, \mbox{ if }\\& & 
\itkind{}(\Gamma(m)) =  \tttheory{} \\ & &
\itkind{}(\Gamma(m))(s) = \tttype{} \mbox{ and }\\& &
\tauGamma{\Pi = \sigma} = \ttcontext\\
%
  \tauGamma{m[\sigma].s} & = & \delta(\Gamma)((\eta(\Gamma, m[\sigma])(
\ittype(\Gamma(m)(s)) )), \\ & &
\mbox{ if } \itkind{}(\Gamma(m)) = \tttheory{} \\ & &
\itkind{}(\Gamma(m)(s)) = \ttconstant{} \mbox{ and }\\& &
\tauGamma{\Pi = \sigma} = \ttcontext
\end{eqnarray*}
\end{Defn}

\begin{comment}
The definition of $\delta$ has to be modified slightly to carry along the
parameters.  The notation $a[\Pi]$ represents the result of
replacing each occurrence of a symbol $s$ in $a$ such that $\Pi(s)$ is
defined, by $\itdef{}(\Pi(s))$ when it exists.
\end{comment}
\begin{Defn}{type expansion with parametric theories}
\begin{eqnarray*}
  \delta(\Gamma)(m[\sigma].s) & = & \delta(\Gamma)(
(\eta(\Gamma, m[\sigma])(\itdef{}(\Gamma(m)(s)))) ), 
                        \mbox{ if }\\ & &  \itdef{}(\Gamma(m)(s)) \mbox{ is
nonempty.}\\
  \delta(\Gamma)(m[\sigma].s) & = & m[\sigma].s, 
                        \mbox{ if } \itdef{}(\Gamma(m)(s)) \mbox{ is
empty.}
%\\
% \delta(\Gamma, m[\sigma])(s) & = & \delta(\Gamma)( \itdef{}((\Pi
%= \sigma)(s))) \mbox{ if } \\& &
% \Pi(s) \mbox{ is defined}\\
% \delta(\Gamma, m[\sigma])(s) & = & m[\sigma].s \mbox{ if } \\& &
%\itdef{}(\Gamma(m)(s)) \mbox{ is
%empty.  }
\end{eqnarray*}
\end{Defn}

The definition of an assignment for a context with parametric theories is
a bit complicated.  
In the nonparametric case, $\gamma(m)$ simply returns an
assignment of values for the types and constants declared in the theory
$m$\@.  For the case of parametric theories $m$, $\gamma(m)$  returns a
function that maps the meaning of the given actuals $\sigma$ to
an assignment $\gamma(m)(\Mgamma{\sigma})$ for the  types and
constants declared in the theory $m$\@.   
There is an important restriction that $\gamma(m)$ must be
\emph{rank-preserving}, that is,  if $\varpi$  
and $\varpi'$ are assignments for 
$\Pi$ so that for each $i$ where $\Pi_i$ is a type parameter, the
rank of $\varpi(\Pi_i)$ equals the rank of 
$\varpi'(\Pi_i)$,  then the ranks of $\gamma(m)(\varpi)(s)$
and $\gamma(m)(\varpi')(s)$ must be the same for each type symbol $s$
declared in $m$\@.  
%\memo{This needs to be stated more directly.}

It is also important to observe that the semantics of parametric theories
makes use of the axiom of choice since the assignment corresponding to a
theory $\mathtt{m}$ of the form  $\mathtt{m [t:TYPE]: THEORY =
\{c : t\}}$ is essentially a choice function.  

Let $\gamma\{\Pi \gets \varpi\}$ represent the assignment such that
$\gamma\{\Pi \gets \varpi\}(s) = \varpi(s)$ for $s$
in the domain of the context $\Pi$, and $\gamma(s)$, otherwise\@.  
The meaning of symbols of the form $m[\sigma].s$ can then be defined
as below.
\begin{Defn}{meaning function for prefixed symbols with actuals}
\begin{eqnarray*}
  \lefteqn{\Mgamma{m[\sigma].s}}\\ & = &
\mathcal{M}(\Gamma;\Pi;\Delta \vbar
\gamma\{\Pi\gets\varpi\}\{\Delta\gets\gamma(m)(\varpi)\})(s), \mbox{ where}\\
 & & \Pi = \mathit{formals}(\Gamma(m)) \\
& &      \Delta = \mathit{definition}(\Gamma(m)) \\
& &      \varpi(r) = \Mgamma{(\Pi = \sigma)(r)}, \mbox{ for } r\in \Pi
\end{eqnarray*}
\end{Defn}

The definition of a satisfying assignment given in
Definition~\ref{defn:satisfaction for contexts with theories} 
also must be strengthened.    
Let $\Pi$ be the
formal parameters to theory $m$ in context $\Gamma$; then, an assignment
$\varpi$ is 
said to be satisfying {\em parameter assignment} for $\Pi$ under the
assignment $\gamma$ to $\Gamma$ 
iff $\gamma\{\Pi\gets \varpi\}$ is a satisfying assignment for $\Pi$\@.  
\begin{Defn}{satisfaction for contexts with parametric theories}
An assignment
$\gamma$ satisfies a context $\Gamma$ if in addition to the 
constraints stated in
Definition~\ref{defn:satisfaction for contexts with theories}, $\gamma$ maps 
every parametric theory $m$  declared in $\Gamma$ with parameters $\Pi$
and definition $\Delta$, 
to a function that maps
any satisfying parameter assignment $\varpi$ for the theory parameters $\Pi$
(namely, 
$\itformals(\Gamma(m))$) to a satisfying assignment $\gamma\{\Pi\gets
\varpi\}\{\Delta \gets \gamma(m)(\varpi)\}$ for 
$\Delta$ (given by 
$\itdef{}(\Gamma(m))$).
\begin{comment}
That is, 
for each declared symbol $s$ in
$\Delta$: 
\begin{enumerate}
\item If $ \itkind{}(\Gamma(m)(s)) = \tttype{}$, then
$\gamma(m)(\Mgamma{\sigma})(s)
\in U$
%
\item if $\itkind{}(\Gamma(m)(s)) = \ttconstant{}$, then $\gamma(m)(\Mgamma{\sigma})(s)
\in \mathcal{M}(\Gamma| \gamma;\{\Pi\gets \Mgamma{\sigma}\};\gamma(m)(\Mgamma{\sigma})(\tau(\Gamma)(m[\sigma].s)))$
%
\item if $\itdef{}(\Gamma(m)(s))$ is defined,
then $\gamma(m)(\Mgamma{\sigma})(s) = \mathcal{M}(\Gamma| \gamma;
\{\Pi\gets \Mgamma{\sigma}\}; \gamma(m)(\Mgamma{\sigma}))(
\itdef{}(\Gamma(m)(s)))$.
\end{enumerate}
\end{comment}
\end{Defn}


\begin{comment}
This map preserves the
assignments given to the parameters so that if for $s : T$ in $\Pi$,
$m[\sigma]$ is such that the formal $s$ is matched with the actual $a$,
then 
$\gamma(m)(\{\Pi\gets \Mgamma{\sigma}\})(s)$ must be $\Mgamma{a}$\@.
\end{comment}
\begin{comment}
The delicate point here is that $\gamma(m)$ is not a set but a map (a
rule) that maps a given assignment (a set) of sets to symbols, to another
assignment of sets to symbols.  This means that $\gamma$ itself is not a
set.  There is no reason why $\gamma$ should be a set since all we require
is that it map a given type name to a set, and a constant to an element
of the set representing its type.  
\end{comment}
\begin{comment}
\begin{eqnarray*}
  \Mgamma{m[\sigma].s} & = & \gamma(m)(\{\Pi\gets \Mgamma(\sigma)\})(s)
\end{eqnarray*}
\end{comment}

\section{Summary}

Theories are used to package related declarations together.  Parametric
theories can be used to package together declarations that are generic in
type and individual parameters.  The type rules for contexts must be
extended to accommodate the theories.  The type rules for simple
(nonparametric) theories are straightforward given this extension.  The
operation of expanding a type using type definitions must be enhanced so
that symbols declared in a theory are prefixed with their theory name when
referenced outside the theory.  Assignments now have the same nested
structure as contexts, and the semantic definition is easily extended to
handle prefixed symbols.  Parametric theories are more complex.  The
theory prefixes now contain actual parameters that have to be typechecked
relative to the expected formal parameters.  The assignments corresponding
to parametric theories are functions that map given assignments for the
formals to assignments for the declarations within a theory.  Such a
mapping must be constrained to be rank-preserving.  Parametric theories
can have subtype parameters, and assumptions on the parameters.  The rules
for subtype parameters and assumptions are omitted for now but will be
included in an expanded version of this report.



\chapter{Conditional Expressions and Logical Connectives}\label{conditionals}

We have, so far, introduced the core of PVS containing types, type
definitions, constant and variable declarations, subtypes, dependent
types, and theories.  In extending the language with
both explicit and recursive constant definitions and formulas,  a crucial
difference  is that the logical context under which a type-correctness
condition is generated provides additional assumptions that can be used in
proving any proof obligations.  Examples of expressions where an
extended context is needed to establish type correctness by discharging
proof obligations include
\begin{enumerate}
\item $x \neq y \supset  (x + y)/(x - y) \leq 0$.  The type of the
division operator constrains the denominator to be nonzero,
that is, $\{x : \ttreal \vbar x \neq 0\}$\@.    
In the given expression,  the denominator
can be shown to be nonzero only in the context of the antecedent $x \neq
y$\@.

\item $\mathtt{IF}(i > 0, i, -i)$ has type $\ttnat$ given integer $i$
provided the {\em then\/} and {\em else\/} parts are typechecked with the
assumptions $i > 0$ and $\neg (i > 0)$, respectively.
\end{enumerate}

PVS has a polymorphic primitive equality predicate:
\begin{display}
equality[T : TYPE] : THEORY = \{ =: [[T, T] -> bool] \}
\end{display}
Note that an equality of the form $ \mathtt{equality}[T].\mathtt{=}(a, b)$ is
informally written as $a = b$\@.  When it is relevant to indicate the type
parameter, we write the equality as $a =_{T} b$\@.  It can be deduced from
the meaning of equality that if $S$ is a subtype of $T$, then for $a$ and
$b$ in $S$, it must be the case that $a =_{S} b$ iff $a =_{T} b$\@.  Thus,
we can assume that equality is always parameterized by a maximal type.  We
assume that any relevant context $\Gamma$ contains the above declaration
of the theory {\tt equality}.  Furthermore, any satisfying assignment
$\gamma$ for such a $\Gamma$ must satisfy
\begin{eqnarray*}
\gamma( \mathtt{ equality})(X)(=) &  = & \{\pair{x, x}\vbar x\in X\}.
\end{eqnarray*}

The negation operation can be defined in terms of equality as shown below.
We assume that the context contains a declaration of the form
\begin{alltt}
   \(\neg : \funtype{\ttbool}{\ttbool} = (\lambda (x : \ttbool): x = \ttfalse)\)
\end{alltt}

As is clear, a satisfying assignment $\gamma$ for a context $\Gamma$
containing the above declaration must be such that $\gamma(\neg)$ yields
the usual truth-table semantics, that is, 
$\{ \pair{\zerob, \oneb}, \pair{\oneb, \zerob} \}$\@.  

We can then introduce the polymorphic \texttt{IF-THEN-ELSE} operation
as follows: 
\begin{display}
if_def [T: TYPE]: THEORY = \{ IF:[bool,T,T -> T] \}
\end{display}

In typechecking conditional expressions, the notion of context
has to be extended to include formulas so that the typechecking
of the subterm $b$ in $\mathtt{IF}(a, b, c)$ is done in the context of $a$,
and the typechecking of $c$ is done in the context of $\neg a$\@.
There is one new typechecking rule for contexts with formulas.
\begin{eqnarray*}
\tau()(\Gamma, a) & = & \ttcontext, \mbox{ if}\\
& & \tau()(\Gamma) = \ttcontext, \mbox{ and} \\
& & (\tau(\Gamma)(a)   \sim \ttbool)_\Gamma
\end{eqnarray*}
Note that the type rule checks that the type of $a$ is compatible with
$\ttbool$ rather than equivalent to it since it is possible that the  type
of $a$ might be a subtype of $\ttbool$\@.  


\begin{Defn}{satisfaction for contexts with formulas}
An assignment $\gamma$ satisfies context $\Gamma$ when in 
addition to the conditions in
Definition~\ref{defn:satisfaction for contexts with parametric theories},  
for each prefix $\Gamma', a$ of $\Gamma$, $\mathcal{M}(\Gamma'\vbar \gamma)(a)
= \oneb$\@.  
\end{Defn}

The typechecking of conditional expressions is different from that of
other application expressions since the \emph{test} part of the
conditional expression is introduced into the context as a contextual
assumption.
\begin{Defn}{type rule for conditional expressions}
   \begin{eqnarray*}
     \tauGamma{ \mathtt{ if\_def[\mathit{T}].IF}(a, b, c)} & = & T, \mbox{ if }
(\tauGamma{a} \sim  \mathtt{bool})_\Gamma,\\ & &
   \tau(\Gamma, a)(b) = B,\\& &  (B \sim T)_{\Gamma, a}, \\& & \vdash_{\Gamma,
a}\pi(T)(b)\\& & 
    \tau(\Gamma, \neg a)(c) = C,\\& &  (C \sim T)_{\Gamma, \neg a},\\ & & 
\vdash_{\Gamma, \neg a} \pi(T)(c)
\end{eqnarray*}
\end{Defn}

The meaning of conditional expressions must be treated in a special way
since the \emph{else} part need not denote when the \emph{test} part is
true and, correspondingly, the \emph{then} part need not denote if
the \emph{test} part is false.  We assume that any relevant contexts $\Gamma$
contain the above declaration of the \texttt{if\_def} theory.
Conditional expressions can be regarded as a
new construct in the language rather than a form of application.
However, it is conservative to regard conditional expressions as
applications since the latter introduce the additional constraint that all
the arguments must already denote, that is, applications are
\emph{strict}.
\begin{Defn}{meaning function for conditional expressions}
\begin{eqnarray*}
\Mgamma{\mathtt{if\_def}[T].\ttif(a, b, c)} & = &
 \left\{\begin{array}{ll}
	\Mgamma{b}, & \mbox{ if } \Mgamma{a} = \oneb\\
        \Mgamma{c}, & \mbox{ otherwise}
	\end{array}\right.
\end{eqnarray*}
\end{Defn}

The semantics for conditional expressions raises an important issue.
The equality
$$\mathtt{if\_def}[\mathtt{bool}].\ttif(x, y, \ttfalse) =
\mathtt{if\_def}[\mathtt{bool}].\ttif(y, x, \ttfalse)$$
is semantically valid for variables $x$ and $y$ of type $\ttbool$\@.
An expression like $\mathtt{if\_def}[\ttbool].\ttif(i \neq 0, 1/i
> 0, \ttfalse)$ can be typechecked to have the type $\ttbool$ since
it generates a valid proof obligation $i\neq 0 \supset i\neq 0$, but
the seemingly equivalent expression
$\mathtt{if\_def}[\ttbool].\ttif( 1/i > 0,i \neq 0, \ttfalse)$
generates an unverifiable proof obligation $i\neq 0$\@.
This may seem contradictory since the equality suggests a transformation
of a type correct conditional expression to a type incorrect expression.  
The resolution here is that equality cannot be instantiated with
$i\neq 0$ for $x$ and $1/i > 0$ for $y$ since the expression
$1/i > 0$  typechecks as having type $\ttbool$ only when $i\neq 0$ is known
from the context.  The same applies in the case of the other
propositional connectives, thus ensuring that each expression is type
correct in the context in which it occurs.  

We can then define the propositional connectives in terms of  conditional
expressions. 
\begin{eqnarray*}
  \wedge : \funtype{\tupletype{\ttbool, \ttbool}}{\ttbool} & = & \lambda (x : \ttbool, y
: \ttbool) : \mathtt{if\_def[\ttbool].IF}(x, y, \ttfalse)\\
   \vee : \funtype{\tupletype{\ttbool, \ttbool}}{\ttbool} & = & \lambda (x : \ttbool, y
: \ttbool) : \mathtt{if\_def[\ttbool].IF}(x, \tttrue, y)\\
   \supset : \funtype{\tupletype{\ttbool, \ttbool}}{\ttbool} & = & \lambda (x : \ttbool, y
: \ttbool) : \mathtt{if\_def[\ttbool].IF}(x, y, \tttrue)\\
%  \neg : \funtype{\ttbool}{\ttbool} & = & \lambda (x : \ttbool) :
%\mathtt{if\_def[\ttbool].IF}(x, \ttfalse, \tttrue)
\end{eqnarray*}

In the typechecking of terms of the form $a \wedge b$,
we follow the corresponding rule for the definition so that 
the term $a$ is assumed in the context when typechecking term $b$\@.
Similarly, for $a\vee  b$, the formula $\neg a$ is assumed in
the context when typechecking $b$, and for $ a \supset b$,
the formula $ a$ is assumed in the context when typechecking $b$.
The Boolean equivalence operator {\tt IFF} has no special rules
for adding formulas to contexts during typechecking.   

\section{Summary}

The use of assumption formulas  enables expressions
to be typechecked within the narrow context of their use so that
the governing assumptions can be used in discharging any
proof obligations.  The type rules for conditional expressions and the
Boolean connectives $\wedge$, $\vee$, and $\supset$ make use of
contextual assumptions.  

\begin{comment}
We will assume that the context $\Gamma$ used below contain the
theories \mathtt{equality} and \mathtt{if\_def}\@.  
The typechecking of constant definitions is straightforward when
the definitions are not recursive.  As we have already seen, the
form of an explicit definitions is $ s : T = a$, where $s$ is a symbol,
$T$ is a type, and $a$ is a term.  
When a constant is defined, no nonemptiness TCCs are generated since
there is a definition present.
%
\begin{eqnarray*}
  \tauGamma{\Delta, s : T = a} & = & \mathtt{CONTEXT}, \mbox{ if }
    \tauGamma{\Delta} = \mathtt{CONTEXT} \\ & &
   \tau(\Gamma, \Delta)(T)  = \mathtt{TYPE}\\ & &
   \tau(\Gamma, \Delta)(a) = T', \mbox{ and}\\ & &
   (T\sim_\Gamma T')_{\Gamma, \Delta}\\& &
   \vdash_{\Gamma, \Delta} \pi(T)(a)
\end{eqnarray*}
%
A recursive constant definition has the syntactic form
$ s : RECURSIVE\ T = a\ MEASURE\ f RELATION p$\@.
%
\memo{This looks slightly complicated. Postpone for now.}
%
A formula declaration has the form
$s : \ \mathtt{FORMULA} a$.
%
\memo{Need to formally defined substitution, freevars, and closure
relative to a context.}  
\end{comment}

\begin{comment}
%
\memo{Need to highlight the shape of a context: theory declarations
followed by type/constant declarations followed by variable declarations.}
%
Introduce equality, if-then-else, connectives, updates.
%
\section{Abstract Datatypes}
\end{comment}


\chapter{Proof Theory of PVS}\label{proof theory}

The final step in the presentation of the semantics is the presentation of
the proof rules for the idealized subset of PVS described thus far.
As already indicated, the proof theory is an integral part of the
semantics since typechecking and proof checking are closely intertwined.
Fortunately, the proof rules turn out to be much less complicated than the
type rules.

\newcommand{\sequent}[3]{#1\vdash_{#2}#3}

The PVS proof theory is presented in terms of  a sequent calculus.  A
sequent is of the form 
$\sequent{\Sigma}{\Gamma}{\Lambda}$, where $\Gamma$ is the context,
$\Sigma$ is a {\em set\/} of {\em antecedent\/} formulas, and $\Lambda$ 
is a {\em set\/} of {\em consequent\/} formulas.  Such a sequent should
be read as stating that the conjunction of the formulas in $\Sigma$ implies the
disjunction of formulas in $\Lambda$\@.

% presumed in math mode #1=top, #2=bottom, #3=side condition
\renewcommand{\Infrule}[3]{
{{\displaystyle\strut #1}\over{\displaystyle\strut #2}}~\makebox[0pt][l]{{$\mathbf{ #3}$}}
}

Inference rules are presented in the form
\begin{center}
$\Infrule{\mathit{premise(s)}}{\mathit{conclusion}}{\mathit{name}}\hspace{2cm}
\mathit{side\ condition}$
\end{center}


\section{PVS Proof Rules} 

\subsection{Structural Rules}

The structural rules permit the sequent to be rearranged or weakened
via the introduction of new sequent formulas into the conclusion.  
All the structural rules can be expressed in terms of the single
powerful weakening rule  shown below.
It allows a weaker statement to
be derived from a stronger one by adding either antecedent formulas or
consequent formulas.  The relation $\Sigma_1\subseteq\Sigma_2$ holds
between two lists when all the formulas in $\Sigma_1$ occur in the list
$\Sigma_2$.  

\begin{center}
$\Infrule{\sequent{\Sigma_1}{\Gamma}{\Lambda_1}}
         {\sequent{\Sigma_2}{\Gamma}{\Lambda_2}} {W}$
\hspace{1cm}\mbox{\smaller\smaller if $\Sigma_1\subseteq\Sigma_2$ and $\Lambda_1\subseteq\Lambda_2$}
\end{center}

Both the Contraction and Exchange rules shown below are absorbed by the
above  weakening rule {\bf W}\@.     The  {\em Contraction\/} rules
$\mathbf{C}\vdash$ and $\vdash\mathbf{C}$
allow multiple occurrences of the same sequent formula
to be replaced by a single occurrence. 
\begin{center}
\begin{tabular}{ccc}
$\Infrule{\sequent{a,a, \Sigma}{\Gamma}{\Lambda}}%
         {\sequent{a, \Sigma}{\Gamma}{\Lambda}} {C\vdash}$
& \hspace{1in} &
$\Infrule{\sequent{\Sigma}{\Gamma}{ a, a, \Lambda}}%
         {\sequent{\Sigma}{\Gamma}{ a, \Lambda}} {\vdash C}$
\end{tabular}
\end{center}

The {\em Exchange\/} rule asserts that the order of the formulas in the
antecedent and the consequent parts of the sequent is immaterial.  It can
be stated as

\begin{center}
\begin{tabular}{c@{\hspace{1in}}c}
$\Infrule{\sequent{\Sigma_1, b, a,  \Sigma_2}{\Gamma}{\Lambda}}
         {\sequent{\Sigma_1, a, b, \Sigma_2}{\Gamma}{\Lambda}} {X\vdash}$
&
$\Infrule{\sequent{\Sigma}{\Gamma}{\Lambda_1, b, a, \Lambda_2}}
         {\sequent{\Sigma}{\Gamma}{\Lambda_1, a, b, \Lambda_2}} {\vdash X}$
\end{tabular}
\end{center}

As seen above, 
{\em inference rules\/} have the general form
$$\Infrule{\Sigma_1\vdash\Lambda_1\quad\cdots\quad\Sigma_n\vdash\Lambda_n}
{\Sigma\vdash\Lambda}{R}.$$ This says that if we are given a leaf of a
proof tree of the form $\Sigma\vdash\Lambda$, then by applying the rule
named {\bf R}, we may obtain a tree with $n$ new leaves.

\subsection{Cut Rule}

The cut rule {\bf Cut} can be used to introduce a case split on a formula $a$
into a proof of a sequent $\sequent{\Sigma}{\Gamma}{\Lambda}$ so as to yield the
subgoals 
$\sequent{\Sigma, a}{\Gamma}{\Lambda}$ and
$\sequent{\Sigma}{\Gamma}{a, \Lambda}$, which can be seen  
as assuming $a$ along one branch and $\neg a$ along the other.
\begin{center}
$\Infrule{(\tau(\Gamma)(a) \sim \ttbool)_\Gamma \qquad \sequent{\Sigma,  a}{\Gamma}{\Lambda}\qquad\sequent{\Sigma}{\Gamma}{ a, \Lambda}}
         {\sequent{\Sigma}{\Gamma}{\Lambda}}{Cut}$
\end{center}

\subsection{Propositional Axioms}

The axioms rule \textbf{Ax} simply asserts that $a$ follows from $a$\@.  
%\begin{center}%{(\tauGamma{a}\sim \ttbool)_\Gamma}
$$\Infrule{}{\sequent{\Sigma, a}{\Gamma}{a,\Lambda}}{Ax}$$  

The next two rules assert that any sequent with either
an antecedent occurrence of $\ttfalse$ or a consequent
occurrence of $\tttrue$ is an axiom.

$$
\begin{array}{c@{\hspace{1in}}c}
\Infrule{}{\sequent{\Sigma, \mathtt{FALSE}}{\Gamma}{\Lambda}}{
\ttfalse \vdash} &
\Infrule{}{\sequent{\Sigma}{\Gamma}{ \mathtt{TRUE}, \Lambda}}{
 \vdash \tttrue}
\end{array}
$$



\subsection{Context Rules}

Certain formulas hold in a context simply because they are already
asserted in the context either as a formula or a constant definition.
$$
\begin{array}{c}
\Infrule{}{\sequent{}{\Gamma}{a}}{Context Formula} \hspace{1.5in}\mbox{if
$a$ is a formula in $\Gamma$}\hspace{.8in}\\
\Infrule{}{\sequent{}{\Gamma}{s = a}}{Context Definition} \hspace{1.5in}\mbox{if
$s : T = a$ is a constant definition in $\Gamma$}
\end{array}
$$

The context $\Gamma$ can be extended with antecedent formulas or
negations of consequent formulas using the following two rules.
$$\begin{array}{c@{\hspace{1in}}c}
\Infrule{\sequent{\Sigma, a}{\Gamma,
a}{\Lambda}}{\sequent{\Sigma, a}{\Gamma}{\Lambda}}{Context \vdash} &
\Infrule{\sequent{\Sigma}{\Gamma,
\neg a}{a, \Lambda}}{\sequent{\Sigma}{\Gamma}{a, \Lambda}}{\vdash Context}
\end{array}
$$

The following context-weakening rule is useful since it shows that
provability is monotonic with respect to the context.
$$\begin{array}{c}
\Infrule{\sequent{\Sigma}{\Gamma}{\Lambda}}{\sequent{\Sigma}{\Gamma'}{\Lambda}}{Context
W}  \hspace{1in}\mbox{if $\Gamma$ is a prefix of $\Gamma'$}
\end{array} 
$$

\begin{comment}
\subsection{Negation Rules}
%
These rules are straightforward since negation is eliminated by moving
the body of a negated antecedent formula to the consequent, and
conversely, by moving the body of a negated consequent formula to
the antecedent.
$$
\begin{array}{c}
\Infrule{\sequent{\Sigma}{\Gamma}{ a, \Lambda}}
         {\sequent{\Sigma,\neg a }{\Gamma}{\Lambda}} {\neg\vdash}\\[.5cm]
\Infrule{\sequent{\Sigma, a }{\Gamma}{\Lambda}}
         {\sequent{\Sigma}{\Gamma}{ \neg a, \Lambda}} {\vdash\neg}
\end{array}
$$
\end{comment}

\subsection{Conditional Rules}

The rules governing the elimination of \texttt{IF-THEN-ELSE} in a proof
are unusual since they augment the context with the test part or its
negation, as in the corresponding type rules.
$$
\begin{array}{c}
\Infrule{\sequent{\Sigma, a, b}{\Gamma, a}{\Lambda}\hspace{1cm}
        \sequent{\Sigma,  c}{\Gamma, \neg a}{a, \Delta}}
         {\sequent{\Sigma, \mathtt{IF}(a, b, c)}{\Gamma}{ \Lambda}}
                  {\mathtt{IF} \vdash}\\[.5cm]
\Infrule{\sequent{\Sigma, a}{\Gamma,
a}{b,\Lambda}\hspace{1cm}\sequent{\Sigma}{\Gamma,\neg a}{a, c, \Lambda}}
         {\sequent{\Sigma}{\Gamma}{{\mathtt{IF}}(a, b, c), \Lambda}}
                      {\vdash {\mathtt {IF}}}
\end{array}
$$

\subsection{Equality Rules}

The rules for equality can be stated as below.  The rules of
transitivity and symmetry for equality can be derived from these rules.
The notation $a[e]$ is used to highlight one or more 
occurrences of $e$ in the formula $a$ such that there are no free variable
occurrences in $e$\@.\footnote{We enforce an invariant on a sequent that
it must not contain any free variables.  This invariant is preserved by
each of the proof rules.}   
The notation $\Lambda[e]$ similarly highlights
occurrences of $e$ in $\Lambda$. 

\begin{center}
\begin{tabular}{c@{\hspace{1in}}c}
$\Infrule{}{\sequent{\Sigma}{\Gamma}{ a=a,\Lambda}}{Refl}$
&
$\Infrule{\sequent{a=b, \Sigma[b]}{\Gamma}{\Lambda[b]}}
         {\sequent{a=b, \Sigma[a]}{\Gamma}{\Lambda[a]}} {Repl}$
\end{tabular}
\end{center}

\subsection{Boolean Equality Rules}

The rule $\mathbf{Repl~\tttrue}$ asserts that an antecedent formula
$a$
can be treated as an antecedent equality of the form $a = \tttrue$, and
correspondingly, 
a consequent formula $a$ can be treated as an antecedent
equality of the form $a = \ttfalse$\@.

$$
\begin{array}{c@{\hspace{1in}}c}
\Infrule{\sequent{\Sigma[\tttrue], a}{\Gamma}{\Lambda[\tttrue]}}{
\sequent{\Sigma[a], a}{\Gamma}{\Lambda[a]}}{Repl~\tttrue} &
\Infrule{\sequent{\Sigma[\ttfalse], a}{\Gamma}{\Lambda[\ttfalse]}}{
\sequent{\Sigma[a]}{\Gamma}{a, \Lambda[a]}}{Repl~\ttfalse}
\end{array}
$$

The rule $\tttrue\mbox{-}\ttfalse$ asserts that $\tttrue$ and $\ttfalse$ are
distinct Boolean constants.
$$
\begin{array}{c}
\Infrule{}{\sequent{\Sigma, \tttrue = \ttfalse}{\Gamma}{\Lambda}}{\tttrue\mbox{-}\ttfalse}
\end{array}
$$
\begin{comment}
Four rules exist for simplifying equalities of the form $a =
\tttrue$ and $a = \ttfalse$ in  a sequent.
%
$$
\begin{array}{c@{\hspace{1in}}c}
\Infrule{\sequent{\Sigma, a}{\Gamma}{\Lambda}}{\sequent{\Sigma, a =
\tttrue}{\Gamma}{\Lambda}}{\tttrue\vdash} &
\Infrule{\sequent{\Sigma}{\Gamma}{a, \Lambda}}{\sequent{\Sigma}{\Gamma}{a =
\tttrue, \Lambda}}{\vdash\tttrue} \\
%
\Infrule{\sequent{\Sigma}{\Gamma}{a, \Lambda}}{\sequent{\Sigma, a =
\ttfalse}{\Gamma}{\Lambda}}{\ttfalse\vdash} &
\Infrule{\sequent{a, \Sigma}{\Gamma}{\Lambda}}{\sequent{\Sigma}{\Gamma}{a =
\ttfalse, \Lambda}}{\vdash\ttfalse} 
\end{array}
$$
\end{comment}

\subsection{Reduction Rules}

The reduction rules are equality rules (axioms) that provide the obvious
simplifications for applications involving lambda abstractions and
product projections.

$$
\begin{array}{c}
\Infrule{}{\sequent{}{\Gamma}{(\lambda (x : T):~a)(b) =
a[b/x]}}{\beta}\\[.5cm]
%
\Infrule{}{\sequent{}{\Gamma}{\proj{i}(a_1, a_2) =
a_i}}{\pi}
\end{array}
$$

\subsection{Extensionality Rules}

The extensionality rules are also equality rules for establishing equality
between two expressions of function or product type.  The extensionality
rule for functions, \textbf{FunExt}, introduces a Skolem constant $s$
to determine that two functions $f$ and $g$ are equal when the results of
applying them to an arbitrary argument $s$ are equal.  
$$
\begin{array}{c}
\Infrule{\sequent{\Sigma}{\Gamma, s : A}{(f~s) =_{B[s/x]} (g~s), \Lambda}}
{\sequent{\Sigma}{\Gamma}{f =_{\funtype{x:A}{B}} g,
\Lambda}}{FunExt} \hspace{1in} \Gamma(s) \mbox{ undefined}\\[.5cm]
\end{array}
$$

The extensionality rule for products asserts that two products are equal if
their corresponding projections are equal.
\begin{comment}
Note that the type
$\tupletype{\listwo{T'}{n}}$ is just $a/\tupletype{\listwo{T}{n}}$\@.
\end{comment}
$$
\begin{array}{c}
\Infrule{\sequent{\Sigma}{\Gamma}{\proj{1}(a) =_{T_1} \proj{1}(b),\Lambda}\qquad\sequent{\Sigma}{\Gamma}{\proj{2}(a) =_{T_2[(\proj{1}~a)/x]} \proj{2}(b),\Lambda}
}{\sequent{\Sigma}{\Gamma}{a =_{\tupletype{x : T_1 T_2}} b, \Lambda}}{TupExt}
\end{array}
$$

Recall that the quantifiers can be defined in terms of lambda abstraction
and equality so that $(\forall (x: T): a)$ is just $(\lambda (x: T): a) =
(\lambda (x : T): \tttrue)$\@.   Existential quantification $(\exists (x:
T): a)$  can easily be defined as $\neg (\forall (x: T): \neg a)$\@.
The proof rules for quantifiers can then be derived from the
rules $\beta$, $\mathbf{TupExt}$, and the equality rules\@.

\subsection{Type Constraint Rule }

We need a rule to introduce the type constraint on a term as an antecedent
formula of the given goal sequent.

$$
\Infrule{\tau(\Gamma)(a) = A\qquad\sequent{\pi(A)(a),\Sigma}{\Gamma}{\Lambda}}
{\sequent{\Sigma}{\Gamma}{\Lambda}}{Typepred}
$$

\section{Soundness of the Proof Rules}
\begin{prop}\label{meaning-preservation}
If $\Gamma$ is a prefix of $\Gamma'$,
$\tau()(\Gamma) = \tau()(\Gamma') = \ttcontext$,
$\gamma'$ is a
satisfying assignment for $\Gamma'$, and $\gamma = \gamma'\restriction\Gamma$ then
for any $a$ such that 
$\tauGamma{a} = \tau(\Gamma')(a)$,  it is the case that $\Mgamma{a} =
\mathcal{M}(\Gamma'\vbar\gamma')(a)$\@.  
\end{prop}

\begin{theorem}[soundness]\label{proof-soundness}
If $\tau()(\Gamma) = \ttcontext$ such that
%\begin{enumerate}
%\item $\Gamma$ contains no variable declarations %bogus(1.18.96)
%\item
for every formula $a$ in $\Sigma;  \Lambda$, $(\tauGamma{a} \sim
\ttbool)_\Gamma$,
%\end{enumerate}
and $\sequent{\Sigma}{\Gamma}{\Lambda}$ is provable, then
for any satisfying assignment
$\gamma$ for $\Gamma$, 
either there is a formula $b$ in $\Sigma$, such that $\Mgamma{b} = \zerob$
or a formula $c$ in
$\Lambda$, such that $\Mgamma{c} = \oneb$\@.      
\end{theorem}
\begin{proof}
The proof is by induction on the structure of the proof of
$\sequent{\Sigma}{\Gamma}{\Lambda}$.  Recall that this proof is actually
part of a simultaneous induction that includes the soundness of the type
rules relative to the semantic function, that is,
Theorems~\ref{dependent-type-semantics}
and~\ref{dependent-term-semantics}\@.   Specific invocations of the
soundness theorem occur in the proofs of
Theorem~\ref{subtype-term-semantics}
and Proposition~\ref{substitution-equivalence}\@.  
\begin{enumerate}
\item {\em Structural Rules\/}:   Since the subset
of formulas in the premise and the conclusion of these rules are the same,
the conclusion follows easily from the induction hypothesis.

\item {\em Cut\/}: By the semantic soundness of the type rules,
we have $\Mgamma{a} \in \twob$\@.   If $\Mgamma{a} = \zerob$, then
by the induction hypothesis on the second subgoal of the proof rule,
there must be some $b$ in $\Sigma$ such that $\Mgamma{b} = \zerob$
or a $c$ in $\Lambda$ such that $\Mgamma{c} = \oneb$\@.  The case
when $\Mgamma{a} = \oneb$ is symmetrical.

\item {\em Propositional Axioms\/}:  Obvious.

\item \emph{Context Rules}:
\begin{description}
\item[ContextFormula:]  If $\gamma$ satisfies $\Gamma$ and $a\in \Gamma$,
then $\Mgamma{a} = \oneb$\@.

\item[ContextDefinition:] If $\gamma$ satisfies $\Gamma$ and $s: T = a$
is a declaration in $\Gamma$, then by the definition of satisfaction,
$\Mgamma{s} = \Mgamma{a}$\@.

\item [Context $\vdash$: ] The argument is trivial when $\Mgamma{a} =
\zerob$.  Otherwise, $\gamma$ satisfies the extended context $\Gamma, a$,
and the conclusion follows from the induction hypothesis.

\item [$\vdash$ Context: ] Similar to $\textbf{Context} \vdash$ above.

\item [ContextW: ]  If $\gamma$ satisfies $\Gamma'$, then it also
satisfies $\Gamma$, and hence the proof.
\end{description}

\item \emph{Conditional Rules}: We only consider $ \mathtt{IF} \vdash$
since the $\vdash \mathtt{IF}$ proof is similar.  If $\Mgamma{
\mathtt{IF}(a, b, c)} = \zerob$, the conclusion follows  trivially.
Otherwise, If $\gamma$ satisfies
$\Gamma$, then $\Mgamma{a}\in \twob$.  If $\Mgamma{a} = \oneb$, then
$\Mgamma{b} = \oneb$.  The induction hypothesis on the subgoal
$\sequent{\Sigma, a, b}{\Gamma, a}{\Lambda}$ yields the desired
conclusion.  Similarly, if $\Mgamma{a} = \zerob$, we have $\Mgamma{c} =
\oneb$
and the induction hypothesis on the second subgoal yields the
desired conclusion.

\item \emph{Equality Rules}: The \textbf{Refl} rule is obvious. 
For the \textbf{Repl} rule, if $\Mgamma{a = b} = \zerob$, the conclusion
follows trivially.  Otherwise, $\Mgamma{a} = \Mgamma{b}$.  Hence, $\gamma$
satisfies the extended context $\Gamma, a = b$.  Then for each
$c[a]$ in $\Sigma[a]$ or $\Lambda[a]$, $\Mgamma{c[a]} = \Mgamma{c[b]}$\@.
\begin{comment}
Note: no free variables in a, b.  Need to show that
replacement preserves meaning when there is no capture.
\end{comment}

\item \emph{Boolean Equality Rules}: The $\mathbf{Repl}~\tttrue$ and
$\mathbf{Repl}~\ttfalse$ rules follow easily since when $\Mgamma{a} =
\oneb$, we have $\Mgamma{c[a]} = \Mgamma{c[\tttrue]}$\@.
A similar argument applies to $\mathbf{Repl}~\ttfalse$\@.

The soundness of $\tttrue\mbox{-}\ttfalse$ is easy since
$\Mgamma{\tttrue = \ttfalse} = \zerob$\@.

\item \emph{Reduction Rules}:  The $\beta$-reduction rule follows because
$\Mgamma{(\lambda (x : T):~a)(b)}$ is
$\mathcal{M}(\Gamma, x : \ttvar~T\vbar \gamma\{x\gets\Mgamma{b}\})(a)$
which by the Substitution Lemma~\ref{substitution-lemma} is
equal to $\Mgamma{a[b/x]}$\@.

The soundness $\pi$-reduction rule is a direct consequent of
Definition~\ref{defn:meaning function}\@.

\item \emph{Extensionality Rules}:
\begin{description}
\item [FunExt:] 
First consider the case when the domain type $\Mgamma{A}$ is empty.  Then by
Definition~\ref{defn:meaning function with dependent types},
$\Mgamma{f} = \Mgamma{g} = \emptyset$\@.  Therefore $\Mgamma{f = g} =
\oneb$ and hence the conclusion.\footnote{Since
the  subgoal sequent $\sequent{\Sigma}{\Gamma, s:A}{(f~s) =
(g~s),\Lambda}$
is valid when $\Mgamma{A} = \emptyset$ for all assignments $\gamma$,
it is natural to ask how it is actually proved.  The only
way a type $A$ can be empty under any assignment $\gamma$ is if
$\Mgamma{\pi(A)(a) = \zerob}$.  The $\mathbf{Typepred}$ rule can
therefore be used on the Skolem constant $s$ to complete the proof.}

The case when $\Mgamma{A}$ is nonempty, we have for any $\gamma$
satisfying $\Gamma$ and $s\in \Mgamma{A}$, that $\gamma'$ given by 
$\gamma\{s\gets z\}$ is a satisfying assignment for $\Gamma, s: A$\@.
By the induction hypothesis, there is either an $a$ in $\Sigma$
such that $\mathcal{M}(\Gamma, s: A\vbar\gamma')(b) = \zerob$ or
a $c$ in  ${(f~s) = (g~s),\Lambda}$ such that $\mathcal{M}(\Gamma, s:
A\vbar\gamma')(c) = \oneb$\@.  If we have such a $b$ in $\Sigma$,
by Proposition~\ref{meaning-preservation}, we also have that
$\Mgamma{b} = \zerob$.  A similar argument can be used
if we have such a $c$ in $\Lambda$\@.  If $c$ is $(f~s) = (g~s)$, then
$\Mgamma{f}(z) = \Mgamma{g}(z)$ for every $z$ in $\Mgamma{A}$\@.
By set-theoretic extensionality, this means that $\Mgamma{f}$ and
$\Mgamma{g}$ are identical elements 
of $\Pi{}F$ where $F$ maps $z$ in $\Mgamma{A}$ to an element of
$\mathcal{M}(\Gamma, x : \ttvar~A\vbar \gamma\{x \gets z\})(B)$\@.
Therefore $\Mgamma{f = g} = \oneb$ as desired.

\item [TupExt:]  If there is some $d$ in $\Sigma$ such that by applying the
induction hypothesis to any of the subgoals $\Mgamma{d} = \zerob$,
then the same holds for the conclusion sequent.  Similarly,
if the induction hypothesis on some subgoal yields a $c$ in $\Lambda$
such that $\Mgamma{c} = \oneb$, then the same holds for the
conclusion sequent.  So the remaining case is when, by the
induction hypothesis, 
$\Mgamma{\proj{i}(a)} = \Mgamma{\proj{i}(b)}$ for each $i\in \{1,2\}$.
It is therefore easy to conclude by set-theoretic extensionality that
$\Mgamma{a}$ and $\Mgamma{b}$ are identical elements of
$\Mgamma{a/\tupletype{\listwo{T}{n}}}$\@.  We can then use
Proposition~\ref{substitution-equivalence} to conclude
that $\Mgamma{a}$ and $\Mgamma{b}$ are identical
elements of $\Mgamma{\tupletype{\listwo{T}{n}}}$\@. 

\end{description}

\item \emph{Type Constraint Rule}: Recall from
Proposition~\ref{maximal-subtype}
that when $\tauGamma{a} = A$, then
$\Mgamma{\pi(A)(a)} = \oneb$.  Given this and the induction hypothesis,
it must either be the case that we have a $b$ in $\Sigma$ such that
$\Mgamma{b} = \zerob$ or a $c$ in $\Lambda$ such that $\Mgamma{c} =
\oneb$\@. 
\end{enumerate}
\end{proof}

To tie the development so far into a single simultaneous
induction as promised, we state the key theorem whose subproofs have been
given by the theorems presented thus far, namely,
Theorems~\ref{dependent-type-semantics},~\ref{dependent-term-semantics},
and~\ref{proof-soundness}\@.  
\begin{theorem}\label{big-soundness}
If $\tau()(\Gamma) = \ttcontext$, then 
\begin{enumerate}
\item If $\Sigma, \Lambda$ is a list of preterms such that for every $a$
in $\Sigma; \Lambda$, $(\tauGamma{a} \sim 
\ttbool)_\Gamma$, and $\sequent{\Sigma}{\Gamma}{\Lambda}$ is provable, then
for any satisfying assignment
$\gamma$ for $\Gamma$, 
either there is a $b$ in $\Sigma$, such that $\Mgamma{b} = \zerob$ or a $c$ in
$\Lambda$, such that $\Mgamma{c} = \oneb$\@.

\item If $A$ is a pretype such that $\tauGamma{A} = \tttype$, then for any
assignment $\gamma$ 
satisfying $\Gamma$, $\Mgamma{A}\in U$\@.

\item If $a$ is a preterm such that $\tauGamma{a} = A$, then for any
assignment $\gamma$ satisfying $\Gamma$, $\Mgamma{a}\in \Mgamma{A}$\@.
\end{enumerate}
\end{theorem}


\section{Summary}

The logical inference rules for the PVS logic have been presented in a
sequent calculus format.  The formal semantics presented in the earlier
chapters is used to establish the soundness of these proof rules.


\chapter{Conclusion}\label{conclusion}

We have presented the syntax and semantics of idealized PVS in several
stages.  In the first stage we introduced the simply typed fragment,
which was then extended with type definitions.  The third such fragment
included subtyping;  the fourth fragment introduced dependent typing.
Finally, we introduced constant definitions and parametric and nonparametric
theories.

The semantic definition was given in a novel, functional style where a
canonical type was assigned to each type correct term.  The interplay between
types and proofs in PVS introduced subtleties and complexities
into the semantic definition.  We can now answer some of the questions
raised in Chapter~\ref{introduction}:
\begin{itemize}
\item {\em What is the semantic core of the language, and what is
just syntactic sugar?}

The semantic core of the language is a typed lambda calculus with
simple function and tuple types, predicate subtypes, dependent types,
parametric theories, and conditional expressions.  Many of the other
features of the PVS language such as records and update expressions
can be explained in terms of the core language.

\item {\em What are the rules for determining whether a given PVS
expression is well typed?  }

The typechecking rules have been presented in terms of the definition of
the $\tau$ operator in Chapters~\ref{simple},  \ref{subtypes},
\ref{dependent}, \ref{theories}, and~\ref{conditionals}\@. 

\item {\em How is subtyping handled, and in particular, how are
proof obligations corresponding to subtypes generated?}

Typechecking an expression $a$ with respect to predicate subtype
constraint $\subtype{x}{T}{p(x)}$
is done by generating the proof obligation $p(a)$ under the logical
context in which $a$ is being typechecked.  This is made precise
in Definitions~\ref{defn:type rules with subtypes} and~\ref{defn:type rule
for conditional expressions}\@.  
Proof obligations are generated when typechecking contexts (for
nonemptiness),  typechecking expressions with respect to expected subtypes, 
and  comparing two types containing subtype expressions for
compatibility.

\item  {\em What is the meaning, in set-theoretic terms, of a PVS expression or
assertion?}

The set-theoretic meaning of well-formed PVS types and expressions
is given by  a meaning function
$\mathcal{M}$ that assigns a set $\Mgamma{T}$ from the universe $U$ to each type
$T$,  and an element $\Mgamma{a}$ of $\Mgamma{T}$ to a given term $a$ of type
$T$\@.

\item {\em Are the type rules sound with respect to the semantics? }

The typechecking function $\tau$ is defined to check contexts,
preterms, and pretypes for type correctness.    The type rules are shown
to be sound with respect to the given semantics in
Theorem~\ref{big-soundness}\@. 

\item {\em Are the proof rules sound with respect to the semantics?}

The proof rules are given in Chapter~\ref{proof theory} in a sequent
calculus format and proved to be sound with respect to the semantics in
Theorem~\ref{big-soundness}\@.  

\item {\em What is the form of dependent typing used by PVS, and
what kinds of type dependencies are disallowed by the language?}

The semantic analysis of dependent typing in Chapter~\ref{dependent}
revealed that type dependencies were constrained to be rank-bounded.  This
is true because the dependencies in dependent typing only constrain the
predicate part of predicate subtypes. Thus, when there is a dependent type
$T(n)$ that depends on a parameter $n$, the meaning of $T(n)$ has a fixed
rank regardless of the meaning assigned to $n$\@.  The PVS language
features used to define dependent types all preserve the rank-boundedness.
Language extensions violating rank-boundedness such as a type dependency
of the form $\funtype{n : nat}{T^n}$ are disallowed.  One can extend the
language with such dependent types, but the semantics would then be
considerably more complicated.

\item {\em What is the meaning of theory-level parametricity, and
what, if any, are the semantic limits on such parameterization?}

The semantics of parametric theories is described in
Chapter~\ref{theories}\@.  In particular, the semantics for parametric
theories is given in terms of rank-preserving maps between the meanings of
the parameters and the meanings of the identifiers declared in the theory.
These maps must be such that the rank of an assignment to a type in a theory
depends only on the ranks of the (meanings of the) type parameters.

\item {\em What language extensions are incompatible with the reference
semantics given here?}

We have already indicated that any language extension, such as an $n$-tuple
type $T^n$, that violates rank-boundedness would be incompatible with the
semantics presented here.
\end{itemize}

This report presents only the core language of PVS.  A more complete
semantic treatment would include arithmetic, recursive constant
definitions, inductive definitions, recursive datatypes, assumptions on
theory parameters, and type judgements.

\paragraph{Acknowledgments. } The advice and encouragement of John Rushby,
Rick Butler, Paul Miner, Pat Lincoln, and Mandayam Srivas are greatly
appreciated, as are the useful expert comments of Peter Dybjer, Mike
Gordon, Doug Howe, and Paul Jackson.  Bruno Dutertre, Paul Miner, and
Harald Rue{\ss} suggested numerous improvements to earlier drafts.

\begin{comment}
What is a specification language?  A programming language is meant to
express an algorithm in such a way that it can be executed on a computing
machine.  For example, a program implementing a sorting algorithm such as
quicksort can be written in a programming language like C.  Such a program
is a recipe for computing the desired outcome but does not directly
describe this outcome.  In the case of a sorting program, the desired
result is that 
the output array must be an ordered permutation of the input array.
A specification language is the medium for expressing and establishing
such properties of programs.  A specification language could itself be a
programming language but non-executable features such as quantifiers and
temporal operators  add to the expressiveness of a specification language.
%
Specification languages need to capture mathematics
as well as programming.  The use of an expressive type system lends
considerable mathematical clarity to a specification.  
ProVeS is an environment for
specification  and verification based on a typed higher-order logic.
The type system of ProVeS includes
{\em predicate subtypes}, i.e.,
the type consisting of elements of a given type satisfying a specified
predicate, in addition to the usual constructs for
function, product, record, and array types.  Automated theorem proving
procedures are employed in the process of typechecking.
The design and rules
for the type system and logic of ProVeS are presented along
with arguments for their soundness.
\end{abstract}
%
% Document Type: LaTeX
% Master File: intro.tex

\chapter{Introduction}

PVS is a \emph{P}rototype \emph{V}erification \emph{S}ystem for the
development and analysis of formal specifications.  The PVS system
consists of a specification language, a parser, a typechecker, a prover,
specification libraries, and various browsing tools.  This document
primarily describes the specification language and is meant to be used as
a reference manual.  The \emph{PVS System Guide}~\cite{PVS:userguide} is to
be consulted for information on how to use the system to develop
specifications and proofs.  The \emph{PVS Prover Guide}~\cite{PVS:prover}
is a reference manual for the commands used to construct proofs.

In this section, we provide a brief summary of the PVS specification
language, enumerate the key design principles behind the language, and
provide a brief example.

The following sections provide more details on the various features of
the language.  The lexical aspects of the language are detailed in
Section~\ref{lexical}.  Section~\ref{declarations} describes
declarations, Section~\ref{types} describes type expressions,
Section~\ref{expressions} describes expressions, and
Section~\ref{theories} describes theories, theory parameters, and
imports and exports of names.  Section~\ref{names} describes names and
name resolution, and Section~\ref{adts} describes the datatype
facility of PVS.  Finally, 
Appendix~\ref{grammar} provides the grammar of the language.

\section{Summary of the PVS Language}

A PVS specification consists of a collection of \emph{theories}.
Each theory consists of a \emph{signature} for the type names and
constants introduced in the theory, and the axioms, definitions, and
theorems associated with the signature.  For example, a typical
specification for a queue would introduce the \texttt{queue} type and the
operations of \texttt{enq}, \texttt{deq}, and \texttt{front} with their
associated types.  In such a theory, one can either define a
representation for the \texttt{queue} type and its associated operations in
terms of some more primitive types and operations, or merely axiomatize
their properties.  A theory can build on other theories: for example, a
theory for ordered binary trees could build on the theory for
binary trees.  A theory can be \emph{parametric} in certain specified
types and values: as examples, a theory of queues can be parametric in
the maximum queue length, and a theory of ordered binary trees can be
parametric in the element type as well as the ordering relation.  It is
possible to place constraints, called \emph{assumptions}, on the
parameters of a theory so that, for instance, the ordering relation
parameter of an ordered binary tree can be constrained to be a total
ordering.

The PVS specification language is based on simply typed higher-order
logic.  Within a theory, \emph{types} can be defined starting from
\emph{base} types (Booleans, numbers, etc.) using the function, record,
and tuple type constructions.  The \emph{terms} of the language can be
constructed using function application, lambda abstraction, and record and
tuple construction.

There are a few significant enhancements to the simply typed language
above that lend considerable power and sophistication to PVS.  New
uninterpreted base types may be introduced.  One can define a
\emph{predicate subtype} of a given type as the subset of individuals in a
type satisfying a given predicate: the subtype of nonzero reals is
written as \texttt{\{x:real | x /= 0\}}.  One benefit of such
subtyping is that when an operation is not defined on all the elements of
a type, the signature can directly reflect this.  For example, the
division operation on reals is given a type where the denominator is
constrained to be nonzero.  Typechecking then ensures that
division is never applied to a zero denominator.  Since the predicate used
in defining a predicate subtype is arbitrary, typechecking is undecidable
and may lead to proof obligations called \emph{type correctness
conditions} (TCCs).  The user is expected to discharge these proof
obligations with the assistance of the PVS prover.  The PVS type system
also features dependent function, record, and tuple type constructions.
There is also a facility for defining a certain class of abstract datatype
(namely well-founded trees) theories automatically.

\section{PVS Language Design Principles}

There are several basic principles that have motivated the design of
PVS which are explicated in this section.

\paragraph{Specification vs. Programming Languages.}
A specification represents requirements or a design whereas a program
text represents an implementation of a design.  A program can be seen as
a specification, but a specification need not be a program.  Typically,
a specification expresses \emph{what} is being
computed whereas a program expresses \emph{how} it is computed.  A
specification can be incomplete and still be meaningful whereas an
incomplete program will typically not be executable.  A specification
need not be executable; it may use high-level constructs, quantifiers
and the like, that need have no computational meaning.  However, there
are a number of aspects of programming languages that a specification
language should include, such as:
\begin{itemize}

\item the usual basic types: booleans, integers, and rational numbers

\item the familiar datatypes of programming languages such as arrays,
records, lists, sequences, and abstract datatypes

\item the higher-order capabilities provided by modern functional
programming languages so that extremely general-purpose operations can
be defined

\item definition by recursion

\item support for dividing large specifications into parameterized
modules

\end{itemize}

It is clearly not enough to say that a specification language shares some
important features of a programming language but need not be executable.
Any useful formal language must have a clearly defined
semantics\footnote{The PVS semantics are presented in a technical
report~\cite{PVS:semantics}.} and must be capable of being manipulated in
ways that are meaningful relative to the semantics.  A programming
language for example can be given a denotational semantics so that the
execution of the program respects its denotational meaning.  The reason
one writes a specification in a formal language is typically to ensure
that it is sensible, to derive some useful consequences from it, and to
demonstrate that one specification implements another.  All of these
activities require the notion of a justification or a proof based on the
specification, a notion that can only be captured meaningfully within the
framework of logic.

\paragraph{Untyped set theory versus higher-order logic}
\index{set theory}\index{higher-order logic}
Which logic should be chosen?  There is a wide variety of choices:
simple propositional logics, which can be classical or intuitionistic,
equational logics, quantificational logics, modal and temporal logics,
set theory, higher-order logic, etc.  Some propositional and modal
logics are appropriate for dealing with finite state machines where one
is primarily interested in efficiently deciding certain finite state
machine properties.  For a general purpose specification language,
however, only a set theory or a higher-order logic would provide the
needed expressiveness.  Higher-order logic requires strict typing to
avoid inconsistencies whereas set theory restricts the rules for forming
sets.  Set theory is inherently untyped, and grafting a typechecker onto
a language based on set theory is likely to be too strict and arbitrary.
Typechecking, however, is an extremely important and easy way of
checking whether a specification makes semantic sense (although 
for an opposing view, the reader is referred to a report by Lamport
and Paulson~\cite{Lamport&Paulson97}).  Higher-order
logic does admit effective typechecking but at the expense of an
inflexible type system.  Recent advances in type theory have made it
possible to design more flexible type systems for higher-order logic
without losing the benefits of typechecking.  We have therefore chosen
to base PVS on higher-order logic.

\paragraph{Total versus partial functions}
\index{function!total}\index{function!partial} In the PVS higher-order
logic, an individual is either a function, a tuple, a record, or the
member of a base type.  Functions are extremely important in higher-order
logic.  They are \emph{first-class} individuals, i.e., variables can range
over functions.  In general, functions can represent either \emph{total}
or \emph{partial} maps.  A total map from domain $A$ to range $B$ maps
each element of $A$ to some element of $B$, whereas a partial map only
maps some of the elements of $A$ to elements of $B$.  Most traditional
logics build in the assumption that functions represent total maps.
Partial functions arise quite naturally in specifications.  For example,
the division operation is undefined on a zero denominator and the
operation of popping a stack is undefined on an empty stack.

Some recent logics, notably those of VDM~\cite{Jones:VDM},
LUTINS~\cite{Farmer:functions}, RAISE~\cite{RAISE-tutorial},
Beeson~\cite{Beeson:book} and Scott~\cite{Scott79}, admit partial
functions.  In these logics, some terms may be \emph{undefined} by not
denoting any individuals.  Some of these logics have mechanisms for
distinguishing defined and undefined terms, while others allow
``undefined'' to propagate from terms to expressions and therefore must
employ multiple truth values.  In all these cases, the ability to
formalize partially defined functions comes at the cost of complicating
the deductive apparatus, even when the specification does not involve any
partial functions.  Though logics that allow partial functions are
extremely interesting, we have chosen to avoid partial functions in PVS.
We have instead employed the notion of a \emph{predicate
subtype}\index{predicate subtype}, a type that consists of those elements
of a given type satisfying a given predicate.  Using predicate subtypes,
the type of the division operator, for example, can be constrained to
admit only nonzero denominators.  Division then becomes a total operation
on the domain consisting of arbitrary numerators and nonzero denominators.
The domain of a \emph{pop} operation on stacks can be similarly restricted
to nonempty stacks.  PVS thus admits partial functions within the
framework of a logic of total functions by enriching the type system to
include predicate subtypes.  We find this use of predicate subtypes to be
significantly in tune with conventional mathematical practice of being
explicit about the domain over which a function is defined.

\section{An Example: \texttt{stacks}}\label{stacks-example}
\index{stacks example@\texttt{stacks} example}

In this section we discuss a specific example, the theory of
\texttt{stacks}, in order to give a feel for the various aspects of the
PVS language before going into detail.  Apart from the basic notation for
defining a theory, this example illustrates the use of type parameters at
the theory level, the general format of declarations, the use of predicate
subtyping to define the type of nonempty stacks, and the generation of
typechecking obligations.

\pvstheory{stacks-alltt}{{Theory \texttt{stacks}}}{stacks-alltt}

Figure~\ref{stacks-alltt} illustrates a theory for stacks of an arbitrary
type with corresponding stack operations.  Note that this is not the
recommended approach to specifying stacks; a more convenient and complete
specification is provided in Section~\ref{stacks-adt},
page~\pageref{stacks-adt}.

The first line introduces a theory named \texttt{stacks} that is
parameterized by a type \texttt{t} (the \emph{formal parameter} of
\texttt{stacks}).  The keyword \texttt{TYPE+} indicates that \texttt{t} is
a \emph{non-empty} type.  The uninterpreted (nonempty) type \texttt{stack}
is declared, and the constant \texttt{empty} and variable \texttt{s} are
declared to be of type \texttt{stack}.  The defined predicate
\texttt{nonemptystack?}~is then declared on elements of type
\texttt{stack}; it is \texttt{true} for a given \texttt{stack} element
iff\footnote{Iff is short for ``if and only if''.} that element is not
equal to \texttt{empty}.  The functions \texttt{push}, \texttt{pop}, and
\texttt{top} are then declared.  Note that the predicate
\texttt{nonemptystack?}~is being used as a type in specifying the
signatures of these functions; any predicate may be used as a type simply
by putting parentheses around it.

The variables \texttt{x} and \texttt{y} are then declared, followed by the
usual axioms for \texttt{push}, \texttt{pop}, and \texttt{top}, which make
\texttt{push} a stack constructor and \texttt{pop} and \texttt{top} stack
accessors.  Finally, there is the theorem \texttt{pop2push2}, that can
easily be proved by two applications of the \texttt{pop\_push} axiom.

This simple theorem has an additional facet that shows up during
typechecking.  Note that \texttt{pop} expects an element of type
\texttt{(nonemptystack?)} and returns a value of type \texttt{stack}.
This works fine for the inner \texttt{pop} because it is applied to
\texttt{push}, which returns an element of type \texttt{(nonemptystack?)};
but the outer occurrence of \texttt{pop} cannot be seen to be type correct
by such syntactic means.  In cases like these, where a subtype is expected
but not directly provided, the system generates a \emph{type-correctness
condition} (TCC).  In this case, the TCC is
\begin{pvsex}
  pop2push2_TCC1: OBLIGATION
    (FORALL (s: stack, y: t, x: t):
      nonemptystack?(pop(push(x, push(y, s)))))
\end{pvsex}
and is easily proved using the \texttt{pop\_push} axiom.  The system keeps
track of all such obligations and will flag the unproved ones during proof
chain analysis.

Parameterized theories such as \texttt{stacks} introduce theory schemas,
where the type \texttt{t} may be instantiated with any other nonempty
type.  To use the types, constants, and formulas of the \texttt{stacks}
theory from another theory, the \texttt{stacks} theory must be imported,
with \emph{actual parameters} provided for the corresponding theory
parameters.  This is done by means of an \texttt{IMPORTING} clause. For
example, the theory
\begin{pvsex}
  ustacks : THEORY
    BEGIN
    IMPORTING stacks[int], stacks[stack[int]]

    si : stack[int]
    sos : stack[stack[int]] = push(si, empty)
    END ustacks
\end{pvsex}
imports stacks of integers and stacks of stacks of integers.  The constant
\texttt{si} is then declared to be a stack of integers, and the constant
\texttt{sos} is a stack of stacks of integers whose top element is
\texttt{si}.  Note that the system is able to determine which instance of
\texttt{push} and \texttt{empty} is meant from the type of the first
argument.  In general, the typechecker infers the type of an expression
from its context.  


\include{rules}
\include{logic}
% Document Type: LaTeX
% Master File: semantics.tex
\documentclass [12pt,twoside]{cslreport} 
\usepackage{fleqn,alltt,spacecites,relsize,url,amssymb,verbatim}
%\usepackage{doublespace}
%\setlength{\textwidth}{6.5in}
%\setlength{\textheight}{9in}
%\setlength{\topmargin}{-.6in}
%\setlength{\oddsidemargin}{0in}
%\setlength{\evensidemargin}{0in}
\setlength{\textwidth}{5.5in}
\setlength{\textheight}{8in}
\setlength{\topmargin}{.25in}
\setlength{\oddsidemargin}{.5in}
\setlength{\evensidemargin}{0in}
\setlength{\mathindent}{0.2in}
%\input amssym.def
%\input amssym
%\input commands
%\includeonly{soundness}
%%
%
%    LaTeX  Macro File  /usr2/jcm/tex/macros.tex
%
%
\tracingonline=0 % shorter error messages (on screen)

%\long\def\comment#1{} % mulit-line comments
%\newcommand{\note}[1]{\fbox{#1}}

% font changes  (as function calls, scribe style)

\renewcommand{\i}[1]{{\it #1\/}}       % italics with space correction
%\newcommand{\emex}[1]{\/{\em #1}}   % emphasis in example, exercise, theorem, etc.
\renewcommand{\c}[1]{{\sc #1}}       % small caps (eliminates \c as cedilla)
%\newcommand{\r}[1]{{\rm #1}}            % for roman font in math mode
%\newcommand{\s}[1]{{\scr #1}}          % script (for use with tatex) 
\newcommand{\s}[1]{{\cal #1}}
\renewcommand{\b}[1]{{\bf #1}}           % bold face
\newcommand{\calg}[1]{{\cal #1}}          % caligraphic 

%\newcommand{\q}[1]{``#1''}    % matching quotes for in-line quotation

% numbered environments

%\newcounter{partcounter}
%\setcounter{partcounter}{0}
%\renewcommand{\part}[1]{\newpage \addtocounter{partcounter}{1}
%\noindent{\Large \bf Part \Roman{partcounter}. #1 } \\[1ex]}

\newtheorem{thm}{Theorem}[section]
\newtheorem{theorem}[thm]{Theorem}
\newtheorem{lemma}[thm]{Lemma}
\newtheorem{cor}[thm]{Corollary}
\newtheorem{corollary}[thm]{Corollary}
\newtheorem{claim}[thm]{Claim}
\newtheorem{prop}[thm]{Proposition}
\newtheorem{conj}[thm]{Conjecture}
\newtheorem{definition}[thm]{Definition}
\newtheorem{exercise}[thm]{Exercise}
\newtheorem{example}[thm]{Example}
\newtheorem{remark}[thm]{Remark}
\newtheorem{open}[thm]{Open Problem}

%\newcommand{\proof}{\\{\bf Proof.}\ }
\newenvironment{proof}{{\bf Proof. }}{\thmbox}

% axiom and inference rule (centered, in math mode,  with name at left)

\newcommand{\axiom}[2]
{\[ \hbox to \columnwidth
    { \rlap{$#2$} \hfil {$ #1 $} \hfil }
\]}

\newcommand{\infrule}[4]
{\[ \hbox to \columnwidth %\textwidth
    { \rlap{$#4$} \hfil $
      \frac {\strut\displaystyle #1 } {\strut\displaystyle #2 } \; \rlap{$#3$} \hfil $
      \hfil }
\]}
\newcommand{\infruletw}[4]
{\[ \hbox to \textwidth
    { \rlap{$#4$} \hfil $
      \frac {\strut\displaystyle #1 } {\strut\displaystyle #2 } \; \rlap{$#3$} \hfil $
      \hfil }
\]}

%\newcommand{\infrule}[4]
%{\[ \hbox to \columnwidth
%    { \rlap{$#4$} \hfil $
%      {{\displaystyle\strut #1}\over{\displaystyle\strut #2}}\quad\makebox[0pt][l]{\it #3} $
%      \hfil }
%\]}

% axiom and inference rule macros for use in tables, etc.
% presumed in math mode #1=axiom, #2=side condition
\newcommand{\Axiom}[2]{
{\displaystyle\strut #1}\qquad\makebox[0pt][l]{\it #2}
}
% presumed in math mode #1=top, #2=bottom, #3=side condition
\newcommand{\Infrule}[3]{
{{\displaystyle\strut #1}\over{\displaystyle\strut #2}}\;\mbox{\scriptsize$\bf #3$}
}

% sequence of #1's, numbered up to #2  (e.g., seq{x}{n} for x1, ..., xn   )

\newcommand{\seq}[2]{#1_{1} \ldots #1_{#2}}  
%\newcommand{\ie}{{\it i.e.}}
%\newcommand{\eg}{{\it e.g.}}
%\newcommand{\cf}{{\it c.f.\,}}

% common symbols

\newcommand{\union}{\cup}
\newcommand{\intersect}{\cap}
\newcommand{\subs}{\subseteq}
\newcommand{\el}{\in}
\newcommand{\nel}{\not\in}
\newcommand{\ns}{\emptyset}
\newcommand{\compose}{\circ}
\newcommand{\set}[2]{ \{\, #1 \,\mid\, #2 \,\}  } % set macro
\newcommand{\infinity}{\infty}
\newcommand{\pair}[1]{\langle #1 \rangle}
\newcommand{\tuple}[1]{\langle #1 \rangle}


\newcommand{\fa}{\forall}
\newcommand{\te}{\exists}
\newcommand{\imp}{\supset}
\renewcommand{\implies}{\supset}
\newcommand{\ts}{\vdash}
\newcommand{\dts}{\models}

\newcommand{\aro}{\mathord\rightarrow} % see pages 154-155 of TeX manual
\newcommand{\paro}{\rightharpoonup} 
\newcommand{\karo}{\mathop\Rightarrow} % see pages 154-155 of TeX manual
%\newcommand{\cross}{\times}
%\newcommand{\dlb}{\lbrack\!\lbrack}
%\newcommand{\drb}{\rbrack\!\rbrack}
\newcommand{\mean}[1]{\lbrack\!\lbrack #1 \rbrack\!\rbrack}
\newcommand{\lam}{\lambda}
\newcommand{\subst}[2]{{}[#1/#2]}
\renewcommand{\dot}{\mathrel{\bullet}}
\newcommand{\Dinf}{D_{\infty}}
\newcommand{\bottom}{\perp}

\mathcode`:="603A  % treat : as punctuation instead of relation in math mode
\mathchardef\colon="303A	% relation colon

\newcommand{\eqdef}{\mathrel{:=}}
\newcommand{\Dom}{\mathop{\rm dom}}
\newcommand{\Pow}{\mathop{\rm Pow}}

\newcommand{\aequiv}{\equiv_\alpha}
\newcommand{\baro}{\buildrel \beta \over \rightarrow}
\newcommand{\earo}{\buildrel \eta \over \rightarrow}
\newcommand{\red}{\rightarrow\!\!\!\!\rightarrow}
\newcommand{\backred}{\leftarrow\!\!\!\!\leftarrow}
\newcommand{\bred}{\buildrel \beta \over \red}
\newcommand{\ered}{\buildrel \eta \over \red}
\newcommand{\conv}{\leftrightarrow}
\newcommand{\beconv}{\buildrel {\beta, \eta} \over\leftrightarrow}

% lambda calculus abbreviations
\newcommand{\letdec}[3]{\b{let\ } #1 = #2 \b{\ in\ } #3}
\newcommand{\letrec}[3]{\b{letrec\ } #1 = #2 \b{\ in\ } #3}

\newcommand{\thmbox}
   {{\ \hfill\hbox{%
      \vrule width1.0ex height1.0ex
   }\parfillskip 0pt }}
\newcommand{\qed}{\thmbox}


% make single spacing

\newcommand{\singlespace}{\renewcommand{\baselinestretch}{1}\@normalsize}
\newcommand{\etal}{{\em et al.}}

% bycase command for definition by cases (pg 49 of LaTeX)

\newcommand{\bycase}[1]
	{\left\{ \begin{array}{ll}  #1  \end{array} \right. }


% make \cite put blanks after the comma (use in alpha style)

%\def\@citex[#1]#2{\if@filesw\immediate\write\@auxout{\string\citation{#2}}\fi
%  \def\@citea{}\@cite{\@for\@citeb:=#2\do
%    {\@citea\def\@citea{,\penalty100\hskip2.5pt plus1.5pt minus.8pt}%
%       \@ifundefined{b@\@citeb}{{\bf ?}\@warning
%       {Citation `\@citeb' on page \thepage \space undefined}}
%\hbox{\csname b@\@citeb\endcsname}}}{#1}}

\newcommand{\thmbox}
   {{\ \hfill\hbox{%
      \vrule width1.0ex height1.0ex
   }\parfillskip 0pt }}
\newtheorem{thm}{Theorem}[chapter]
\newtheorem{prop}[thm]{Proposition}
\newenvironment{proof}{{\bf Proof. }}{\thmbox}
\newcommand{\Infrule}[3]{
{{\displaystyle\strut #1}\over{\displaystyle\strut #2}}\qquad\makebox[0pt][l]{\it #3}
}
\newcommand{\tuple}[1]{\langle #1 \rangle}
\newtheorem{example}[thm]{Example}
\newcommand{\aro}{\mathord\rightarrow} % see pages 154-155 of TeX manual
\newcommand{\mean}[1]{\lbrack\!\lbrack #1 \rbrack\!\rbrack}
\newcommand{\pair}[1]{\langle #1 \rangle}
\newcommand{\union}{\cup}
\newcommand{\funtype}[2]{[#1 \aro #2]}
\newcommand{\tupletype}[1]{[#1]}
\newcommand{\subtype}[3]{\{#1 : #2 | #3 \}}
\newcommand{\tauGamma}[1]{\tau(\Gamma)(#1)}
\newcommand{\Mgamma}[1]{{\mathcal M}(\Gamma\vbar\gamma)(#1)}
\newcommand{\proj}[1]{\mathtt{p}_{#1}}
\newcommand{\lis}[2]{#1_{1}, \ldots, #1_{#2}}
\newcommand{\listwo}[2]{#1_{1}, #1_{2}}
\newcommand{\ol}[1]{\overline{#1}}
%\newcommand{\ul}[1]{\underline{#1}}
\newcommand{\twob}{\mathbf{2}}
\newcommand{\oneb}{\mathbf{1}}
\newcommand{\zerob}{\mathbf{0}}
\newcommand{\reals}{\mathbf{R}}
\newcommand{\ttbool}{\mathtt{bool}}
\newcommand{\ttreal}{\mathtt{real}}
\newcommand{\tttrue}{\mathtt{TRUE}}
\newcommand{\ttfalse}{\mathtt{FALSE}}
\newcommand{\tttype}{\mathtt{TYPE}}
\newcommand{\ttcontext}{\mathtt{CONTEXT}}
\newcommand{\ttvar}{\mathtt{VAR}}
\newcommand{\ttconstant}{\mathtt{CONSTANT}}
\newcommand{\ttvariable}{\mathtt{VARIABLE}}
\newcommand{\tttheory}{\mathtt{THEORY}}
\newcommand{\ttif}{\mathtt{IF}}
\newcommand{\ttint}{\mathtt{int}}
\newcommand{\ttnat}{\mathtt{nat}}
\newcommand{\itkind}{\textit{kind}}
\newcommand{\ittype}{\textit{type}}
\newcommand{\itdef}{\textit{definition}}
\newcommand{\itformals}{\textit{formals}}
\newcommand{\vbar}{\ |\ }
\newcommand{\sima}{\stackrel{a}{\sim}} %/{\scriptsize{a}}
\newenvironment{Eg}[1]{\begin{example}[#1]\label{eg:#1}\em }{\thmbox\end{example}}
\newtheorem{definition}{Definition}
\newtheorem{lemma}{Lemma}
\newtheorem{theorem}{Theorem}
%% Derived from John Rushby's prelude.tex, modified for NFSS2
%
% define variants of the \LaTeX macro that avoid using \sc
% for use in headings
%

% Define fonts that work in math or text mode
\def\dwimrm#1{\ifmmode\mathrm{#1}\else\textrm{#1}\fi}
\def\dwimsf#1{\ifmmode\mathsf{#1}\else\textsf{#1}\fi}
\def\dwimtt#1{\ifmmode\mathtt{#1}\else\texttt{#1}\fi}
\def\dwimbf#1{\ifmmode\mathbf{#1}\else\textbf{#1}\fi}
\def\dwimit#1{\ifmmode\mathit{#1}\else\textit{#1}\fi}
\def\dwimnormal#1{\ifmmode\mathnormal{#1}\else\textnormal{#1}\fi}

\def\BigLaTeX{{\rm L\kern-.36em\raise.3ex\hbox{\small\small A}\kern-.15em
    T\kern-.1667em\lower.7ex\hbox{E}\kern-.125emX}}
\def\BoldLaTeX{{\bf L\kern-.36em\raise.3ex\hbox{\small\small\bf A}\kern-.15em
    T\kern-.1667em\lower.7ex\hbox{E}\kern-.125emX}}
%\def\labelitemi{$\bullet$}
\def\labelitemii{$\circ$}
\def\labelitemiii{$\star$}
\def\labelitemiv{$\diamond$}
\newcommand{\tcc}{{\small\small TCC}}
\newcommand{\tccs}{\tcc s}
\newcommand{\emacs}{{Emacs}}
\newcommand{\Emacs}{\emacs}
\newcommand{\ehdm}{{E{\small\small HDM}}}
\newcommand{\Ehdm}{\ehdm}
\newcommand{\tm}{$^{\mbox{\tiny TM}}$}
\newcommand{\hozline}{{\noindent\rule{\textwidth}{0.4mm}}}

\newcommand{\allclear}%
  {\mbox{\boldmath$\stackrel{\raisebox{-.2ex}[0pt][0pt]%
              {$\textstyle\oslash$}}{\displaystyle\bot}$}}

\newenvironment{private}{}{}

\newenvironment{smalltt}{\begin{alltt}\small}{\end{alltt}}

\newlength{\hsbw}

\newenvironment{session}%
  {\begin{flushleft}
   \setlength{\hsbw}{\linewidth}
   \addtolength{\hsbw}{-\arrayrulewidth}
   \addtolength{\hsbw}{-\tabcolsep}
   \begin{tabular}{@{}|c@{}|@{}}\hline 
   \begin{minipage}[b]{\hsbw}
   \begingroup\small\mbox{ }\\[-1.8\baselineskip]\begin{alltt}}
  {\end{alltt}\endgroup\end{minipage}\\ \hline 
   \end{tabular}
   \end{flushleft}}

\newenvironment{smallsession}%
  {\begin{flushleft}
   \setlength{\hsbw}{\linewidth}
   \addtolength{\hsbw}{-\arrayrulewidth}
   \addtolength{\hsbw}{-\tabcolsep}
   \begin{tabular}{@{}|c@{}|@{}}\hline 
   \begin{minipage}[b]{\hsbw}
   \begingroup\footnotesize\mbox{ }\\[-1.8\baselineskip]\begin{alltt}}%
  {\end{alltt}\endgroup\end{minipage}\\ \hline 
   \end{tabular}
   \end{flushleft}}

\newenvironment{spec}%
  {\begin{flushleft}
   \setlength{\hsbw}{\textwidth}
   \addtolength{\hsbw}{-\arrayrulewidth}
   \addtolength{\hsbw}{-\tabcolsep}
   \begin{tabular}{@{}|c@{}|@{}}\hline 
   \begin{minipage}[b]{\hsbw}
   \begingroup\small\mbox{ }\\[-0.2\baselineskip]}%
  {\endgroup\end{minipage}\\ \hline 
   \end{tabular}
   \end{flushleft}}

\newcommand{\memo}[1]%
  {\mbox{}\par\vspace{0.25in}%
   \setlength{\hsbw}{\linewidth}\addtolength{\hsbw}{-1.5ex}%
   \noindent\fbox{\parbox{\hsbw}{{\bf Memo: }#1}}\vspace{0.25in}}

\newcommand{\nb}[1]%
  {\mbox{}\par\vspace{0.25in}%
   \setlength{\hsbw}{\linewidth}\addtolength{\hsbw}{-1.5ex}%
   \noindent\fbox{\parbox{\hsbw}{{\bf Note: }#1}}\vspace{0.25in}}

\newcommand{\comment}[1]{}
\newcommand{\exfootnote}[1]{}
%\newcommand{\ifelse}[2]{#1}
\sloppy
\clubpenalty=100000
\widowpenalty=100000
%\displaywidowpenalty=100000
\setcounter{secnumdepth}{3} 
\setcounter{tocdepth}{3}
\setcounter{topnumber}{9}
\setcounter{bottomnumber}{9}
\setcounter{totalnumber}{9}
\renewcommand{\topfraction}{.99}
\renewcommand{\bottomfraction}{.99}
\renewcommand{\floatpagefraction}{.01}
\renewcommand{\textfraction}{.2}
\font\largett=cmtt10 scaled\magstep1
\font\Largett=cmtt10 scaled\magstep2
\font\hugett=cmtt10 scaled\magstep3

%\input{/homes/shankar/tex/ch-jmacros}

\newenvironment{Defn}[1]{\begin{definition}[#1]\label{defn:#1}}{
\thmbox\end{definition}}
%
\newenvironment{display}{\begin{alltt}\small\tt\vspace{0.3\baselineskip}}{\vspace{0.3\baselineskip}\end{alltt}}
%
%\newcommand{\gets}{\leftarrow}
\title{The Formal Semantics of PVS\thanks{Funded by National Aeronautics
and Space Administration Contract NAS1-18969, Task 11 and National Science
Foundation Grant CCR 9300444. 
}}
\author{
\begin{tabular}{ccc}
Sam Owre & and & Natarajan Shankar\vspace{-.05in}\\
{\small\tt owre@csl.sri.com}& & {\small\tt shankar@csl.sri.com}\\
\end{tabular}\\
URL: \url{http://www.csl.sri.com/sri-csl-fm.html}\\
SRI International\\
Computer Science Laboratory\\
Menlo Park CA 94025 USA
}
\date{Technical Report CSL-97-2R\\August 1997, Revised March 1999}
%\renewcommand{\today}{}
%\renewcommand{\is}{::=}
%\renewcommand{\baselinestretch}{2}
\begin{document}
%\bibliographystyle{alpha}
\pagenumbering{roman}
\maketitle
\setcounter{page}{2}
\mbox{}
\cleardoublepage
\begin{abstract}
\setcounter{page}{3} 
\thispagestyle{plain}

A specification language is a medium for expressing {\em what\/} is
computed rather than {\em how\/} it is computed.  Specification languages
share some features with programming languages but are also different in
several important ways.  For our purpose, a specification language is a
logic within which the behavior of computational systems can be
formalized.  Although a specification can be used to simulate the behavior
of such systems, we mainly use specifications to state and prove system
properties with mechanical assistance.

We present the formal semantics of the specification language of SRI's
Prototype Verification System (PVS)\@.  This specification language is
based on the simply typed lambda calculus.  The novelty in PVS is that it
contains very expressive language features whose static analysis (e.g.,
typechecking) requires the assistance of a theorem prover.  The formal
semantics illuminates several of the design considerations underlying PVS,
particularly the interaction between theorem proving and typechecking.
\end{abstract}
\cleardoublepage
\tableofcontents
%\listwotoffigures
\cleardoublepage
\setcounter{page}{0} 
\pagestyle{headings}
\pagenumbering{arabic}

\chapter{Introduction} \label{introduction} PVS is a system for specifying
and verifying properties of digital hardware and software systems.  The
specification language of PVS is designed to admit succinct, readable, and
logically meaningful specifications.  The PVS specification language is
designed for effective proof construction rather than efficient execution.
The design considerations underlying the language are therefore somewhat
different from those of a corresponding programming language.  For
example, the language contains constructs that can be statically
typechecked only with the assistance of a theorem prover.  This is
acceptable because the PVS specification language is intended for use in
conjunction with powerful support for automated theorem proving.  The
logic of PVS is based on a simply typed higher-order logic with function,
record, and product types, and recursive type definitions.  This type
system is extended with subtypes that are analogous to subsets, and with
dependently typed functions, records, and products.  The resulting type
system has several advantages.  It is possible, for instance, to
statically ensure that all array references are within their respective
array bounds.  PVS specifications are organized into theories that can be
parametric in types as well as individuals.  While the semantics of the
simply typed fragment is straightforward, the extensions such as
subtyping, dependent typing, and (theory-level) parametricity do pose
significant challenges.  This report presents a concise but idealized
definition of the PVS specification language and its intended formal
set-theoretic semantics.  It is neither an overview of the PVS language
nor a guide to the Prototype Verification System (see the PVS user
manuals~\cite{PVS:manuals}).


The primary purpose of the formal semantics is as a useful reference for
the developers and users of PVS\@.  The idealized core of the
specification language as presented here serves as a succinct foundation
for studying the expressive power of the language.  Pertinent questions
about PVS are answered directly by the formal semantics presented here:
\begin{enumerate}
\item\label{one} What is the semantic core of the language, and what is
just syntactic sugar?

\item\label{two} What are the rules for determining whether a given PVS
expression is well typed?

\item\label{three} How is subtyping handled, and in particular, how are
proof obligations corresponding to subtypes generated?

\item\label{four} What is the meaning, in set-theoretic terms, of a PVS expression or
assertion?

\item\label{five} Are the type rules sound with respect to the semantics? 

\item\label{six} Are the proof rules sound with respect to the semantics?

\item\label{seven} What is the form of dependent typing used by PVS, and
what kinds of type dependencies are disallowed by the language?

\item\label{eight} What is the meaning of theory-level parametricity, and
what, if any, are the semantic limits on such parameterization?

\item\label{nine} What language extensions are incompatible with the reference
semantics given here?
\end{enumerate}

Chapter~\ref{conclusion} summarizes the answers to these questions.


\section{Real versus Idealized PVS}

The semantic treatment in this report is incomplete in some important ways.
It does not treat the nonlogical parts of the language.  In particular,
it ignores arithmetic and recursive definitions.  It also omits
abstract datatypes~\cite{Shankar:ADT97}.  These will be treated in a future
expanded version. 

The present semantics also makes several idealizations from the real
PVS for 
the purpose of clarity.  While the semantic treatment is not
comprehensive, the idealization of PVS used here is faithful to the
implemented form of PVS.   

\begin{enumerate}
  \item \emph{No name resolution.}  All names must be in fully resolved form
with their theory name and actual parameters.  We regard name resolution as
a convenience provided by the PVS type checker and not an operation with any
semantic relevance.  A technical description of name resolution in PVS will
be given elsewhere.


  \item \emph{No overloading.}   As with name resolution, overloading is
a syntactic convenience with no semantic import. 

\item \emph{No \texttt{IMPORTING}s.}  The importing of theories is a 
hint to  name resolution.  The semantic definition assumes that all
instances of theories declared prior to the present one are visible.   
  \item \emph{Variable declarations ignored.}  All variables must be 
locally declared.   Global variable declarations are regarded as a syntactic
convenience.  
%  \item TCCs are not kept around to be used as lemmas.
  \item \emph{No records.}  These are ignored in the semantic treatment
since product types capture all the semantically essential features of records.
\end{enumerate}


\section{Semantic Preliminaries}

The PVS specification language is based on higher-order logic.  This means
that variables can range over individuals (such as numbers) as well as
functions, functions of functions, and so on.  As is well known, some type
distinction is needed; otherwise, it is easy to obtain a
contradiction by defining the predicate $N(P)$ as $ \neg(P(P))$ so that
both $N(N)$ and $\neg N(N)$ hold.  
In the theory of types~\cite{Church:types}, the universe is stratified
into distinct types 
so that a predicate can  be applied only to a \emph{lower} type
and thus cannot be applied to itself.

Types also serve as a powerful mechanism for detecting syntactic and
semantic errors through typechecking.  This role of types is best
exemplified by their use in various programming languages such as Algol,
Ada, and ML, and is also heavily emphasized in the PVS type
system.

The desirability for strong typing in a specification
logic is not widely accepted.  Fraenkel {\em et
al\/}~\cite{Fraenkel-etal84} express
the opinion that such 
typing is repugnant in a mathematical logic since it constrains
expressiveness by not allowing individuals of differing types to be
treated uniformly.  Lamport~\cite{TLA:TOPLAS94} argues that type
correctness is like any other 
program property and should be established by means of a proof rather than
by syntactic restraints.  Lamport and Paulson~\cite{Lamport&Paulson97}
analyze the tradeoffs between typed and untyped specification languages.  
We claim that
\begin{enumerate}
\item Types impose a useful discipline on the specification.
\item Types lead to easy and early detection of a large class of
syntactic and  semantic errors.
\item Type information is useful in mechanized reasoning.
\end{enumerate}


\begin{comment}
The first typed higher-order logic was the Principia Mathematica (PM) system of
Russell and Whitehead.  They were the first to propose a stratified
hierarchy of types as a means avoiding certain paradoxes.  The PM type
system is quite complicated.  Church proposed his \emph{simple theory of
types} as a higher-order logic based on a simply typed lambda calculus.
Church's system forms the basis for most modern higher-order logics
including that of PVS.  The Automath system of de~Bruijn pioneered the use
of dependent types (analogous to the variant records of Pascal) as a basis
for intuitionistic logic.\footnote{Similar ideas were 
independently proposed by several other researchers.}  Dependent types 
form an important component of the PVS type system.  
Girard and Reynolds, independently extended the typed lambda calculus
to a second-order or \emph{polymorphic} type system so that, for example,
an identity function for each type could be obtained by specializing a
generic identity function to the required type.
\end{comment}

The semantics of a higher-order logic is given by mapping the well-formed
types of the logic to {\em sets\/}, and the well-formed terms of the logic to
{\em elements\/} of the sets representing their type.  The set
constructions we use can be formalized within Zermelo-Fraenkel set theory
with the axiom of choice (ZFC).   The intended interpretation
of a function type in higher-order logic is that it represents the set of
all functions from the set representing the domain type to the set
representing the range types.\footnote{It is only in the {\em standard\/}
model of higher-order logic that the function type is required to
represent the set of all 
functions from the domain set to the range set.  Higher-order logic can be
interpreted in {\em general models\/} where the function type can be
interpreted in any manner as long as it satisfies the various axioms
such as application, abstraction, and extensionality~\cite{Andrews:book}.
Higher-order logic 
is complete with respect to the general models interpretation so that
a statement that is valid in all models is provable.  It is, however,
incomplete with respect to the standard model.}  
PVS also has predicate subtypes that are to
be interpreted over the subsets of the set representing the parent type.

The semantics of PVS will be given by considering a sequence of
increasingly expressive fragments of PVS\@.   
The semantics of each fragment of PVS will be presented in three steps.
The first step is to define a set-theoretic universe containing
enough sets to represent the PVS types.  The second step is to define a
typechecking operation that determines whether a given PVS expression is
well typed.  The third step is to define a semantic function that assigns
a representation in the semantic universe to each well-typed PVS type and
term. 



\begin{comment}
I am not exactly sure where the axiom of choice is really needed.
Since PVS builds in a choice operator via parametric theories, it seems
that this will be needed.
\end{comment}


We first lay out the ZFC set constructions needed for defining the
semantics of PVS.  The base types in PVS consist of the Booleans
\texttt{bool} and the real numbers \texttt{real}\@.  The Booleans can be
modeled by any two-element set, say $\twob$ consisting of the elements
$\zerob$ and $\oneb$, where $\zerob$ is the empty set and the only element
of the set $\oneb$\@.  The real numbers can be captured by means of
Dedekind cuts or Cauchy sequences, and we label this set $\reals$\@.

To define the semantics, we need a universe that contains the sets
$\twob$ and 
$\reals$
and is closed under Cartesian products (written as $X\times Y$)
and power sets (written as $\wp(X)$)\@.  Note that
functions are modeled as \emph{graphs}, that is, sets of ordered pairs,
so that a function type $\funtype{A}{B}$ is represented by a subset of
the powerset $\wp(\mean{A}\times \mean{B})$
of the Cartesian product of the sets $\mean{A}$ and $\mean{B}$
representing $A$ and 
$B$, respectively\@.  A set $F$ that is a subset of $X \times Y$ is the graph of a
function with domain $X$ and range $Y$ if for every $x\in X$ there is a $y\in
Y$ such that $\pair{x, y}\in F$, and whenever $\pair{x, y}\in F$
and $\pair{x, y'}\in F$, we have $y = y'$\@.  For such a set $F$,
$ \textit{Function}(F)$ holds and $ \textit{dom}(F) = X$\@.  
The set of graphs of total functions from a set $Y$ to a set $X$
is represented as $X^Y$\@.   If $F$ is the graph of a function and $t$ an
element in its domain, then
$F(t)$ represents the result of applying the function $F$ to $t$\@.
At the semantic level,  a function $F$ will never be applied to an argument
$t$ outside $\textit{dom}(F)$, because in the PVS language,
a function application is typechecked so that the
argument expression has the same type as the domain type of the
function expression.  

We can model the entire type universe of the simply typed fragment of PVS
by the set $U$, which is defined cumulatively by starting from the base
sets $\twob$ and $\reals$, and including the Cartesian products, the
function spaces, and subsets of previously included sets, at each stage.
Cartesian products are used to model products in PVS, and function spaces
model function types.  Subsets are needed to model predicate subtypes.
It is sufficient to iterate these stages up to the ordinal $\omega$\@.
\begin{Defn}{type universe}
\begin{eqnarray*}
  U_0 & = & \{ {\twob}, {\reals}  \} \\
  U_{i+1} & = & U_i \,\union\, \{X\times Y \vbar X, Y \in U_i\} \,\union\,
                           \{X^Y \vbar X, Y \in U_i \} \,\union\,
                           \bigcup_{X\in U_i}\wp(X)\\
  U_{\omega} & = & \bigcup_{i\in\omega} U_i\\
  U & = & U_{\omega}
\end{eqnarray*}
\end{Defn}


We refer to $U$ as the \emph{basic universe}.\footnote{The inclusion of $X^Y$ in $U$ is actually redundant but aids clarity.}   The semantic definitions
below will assign a set in $U$ to each PVS type and an element
in $\bigcup U$ to each well-typed term of PVS.  The \emph{rank} of a set
$X$ in $U$ is the least $i$ such that $X\in U_i$\@.  The notion of rank
plays an important role in the semantics of dependent types and
parametric theories.

\section{Related Work}

There is a long history of work in specification languages.
Many  ideas similar to those underlying the PVS specification language
also occur in other specification languages.  

The {\em wide-spectrum\/} languages are typically based on set theory or
higher-order logic.  The language VDM is one of the earliest such
specification formalisms~\cite{Jones:VDM}.  It is based on a first-order
logic with partial functions augmented with datatype axioms.  The datatype
theories in VDM include those for finite sets, maps, sequences, and
recursive datatypes such as lists and trees.  VDM has a notion of datatype
invariants that yields a simple form of predicate subtyping.  Operations
on state are specified in terms of pre-condition/post-condition pairs.
Specifications are structured into parameterized modules.  In contrast to
VDM, the PVS language is based on strictly typed higher-order logic with a
built-in notion of predicate subtyping and dependent typing.  The
resulting PVS logic is more compact in that many of the datatypes that are
presented axiomatically in VDM can be defined within PVS.  There is no
built-in notion of state in PVS since it is possible to use the
higher-order logic of PVS to define a variety of state-based formalisms,
including various linear and branching-time temporal logics.  VDM uses a
3-valued logic for the logical connectives in order to deal
with partial functions, whereas PVS uses a classical 2-valued logic and
predicate subtyping to assign a type to a partial function as a total
function on its domain of definition.  Jones~\cite{Jones:VDM} provides
only an informal semantics for VDM.  The RAISE system is a comprehensive
toolset based on the ideas of VDM~\cite{RAISE:book}.

The Z specification language~\cite{Z-understanding} is another
wide-spectrum language based on a typed first-order set theory.  A Z
specification is a collection of schemas consisting of declarations of
types and constants accompanied with invariants.  Z schemas can either
specify datatype invariants or pre-condition/post-condition constraints.
A schema calculus is used to combine schemas using logical connectives.
Spivey~\cite{Z-understanding} presents a formal semantics for Z without
giving a proof system or a soundness proof.  Spivey's treatment of
partial functions in the Z semantics employs the commonly used convention
that $f(a)$ when $a$ is not in the domain of $a$ is some arbitrarily
chosen value.  This is fine for most purposes but can be confusing when
dealing with recursively defined partial functions.  For example, the
definition $\mathit{bad}(x) = 1 + \mathit{bad}(x)$ is everywhere undefined but
admitting it as an axiom leads to an immediate contradiction.  Z also
lacks any mechanism for conservative extensions such as definitional
principles for constants and datatypes so that the consistency of Z
specification has to be demonstrated by exhibiting a model.

Algebraic specification languages like OBJ~\cite{popl85} and
Larch~\cite{Larch:book} provide an 
equational/rewriting framework for specifying datatypes and operations on
datatypes.   OBJ has many of the same theory parameterization mechanisms
as PVS.  The subsort mechanism in OBJ is also similar except that
it is handled by introducing {\em retracts\/} or runtime checks
rather than proof obligations generated by the type checker.
The OBJ logic is quite restricted compared to PVS since it is based
on a first-order, equational framework with an {\em initial\/} semantics
where two ground terms are distinct unless they can be proved equal.
OBJ has very limited support for proof development and is primarily
intended as an executable specification language.

The specification languages that are closer to PVS are those that
accompany various automated proof checking systems.  The closest of these
is {\sc Ehdm}~\cite{EHDM:manuals}, which employs a similar higher-order
logic with subtyping and proof obligation generation.  {\sc Ehdm} lacks
many of the features of PVS: subtyping is restricted to type declarations
and there is no dependent typing.

Higher-order logic is used by other systems such as
HOL~\cite{Gordon&Melham:HOL} and TPS~\cite{andrews84}.
Both HOL and TPS employ simply typed higher-order logic 
without features such as subtyping, dependent typing, or parametric
theories.  
Andrews~\cite{Andrews:book} gives a thorough account of the semantic
aspects of higher-order logic.  The formal semantics of the HOL logic are
carefully outlined (by Pitts) in the book by Gordon and
Melham~\cite{Gordon&Melham:HOL}.

Systems like Coq~\cite{COQ} and Nuprl~\cite{Nuprl-book} are based on
intuitionistic higher-order logics.  Coq allows quantification over types,
whereas Nuprl has quantification over a hierarchy of type universes.  Both
logics admit dependent typing.  The set-theoretic semantics of dependently
typed intuitionistic type theories has been studied by
Dybjer~\cite{DybjerP:indsfm} and Howe~\cite{Howe91,Howe96}.  Not
surprisingly, their semantic treatment of dependent typing is similar to
the one given here but they do not delimit the possible dependencies as is
done with the PVS semantics.  The PVS semantics presented here clearly
specifies the kind of type dependencies that are disallowed in the logic.
Dybjer and Howe also do not address subtyping but do describe the
semantics of language features missing in PVS (type universes in the case
of Howe, and inductive families in the case of Dybjer).  Dybjer does not
identify the universe over which terms and types are interpreted.  Howe
requires an infinite sequence of inaccessible cardinals for his universe
construction.


\section{Outline}

In Chapter~\ref{simple}, we define the syntax and semantics of the simply
typed fragment of PVS.  Type definitions are also introduced in this
chapter along with the definition of definitional equivalence on types.
Chapter~\ref{subtypes} adds subtyping to the simply typed fragment and
specifies the additional type rules and semantic definitions that are
needed.  Chapter~\ref{dependent} extends the language with dependent
function and product types.  Theories and parametric theories are introduced
into the language in Chapter~\ref{theories}.  The type rules and semantics
for conditional expressions and the logical connectives defined using
conditional expressions are introduced in Chapter~\ref{conditionals}.
Chapter~\ref{proof theory} specifies the axioms and inference rules of
PVS\@.
 
\chapter{The Simple Type Theory}\label{simple}

\begin{comment}
The types of expression
are determined from the types of their subexpressions.  In the base case,
the types of atomic expressions must be given in a context.  There are
certain well-formedness constraints on contexts that we now specify.
A context is a sequence of declarations.  Declarations can either
be type declarations of the form $T\ :\ \mathtt{TYPE}$ where $T$ is a symbol,
constant declarations of the form $c\ :\ T$, or variable declarations of
the form $x\ :\ \mathtt{VAR} T$.  
The empty context $\{\}$ is a context.  If $\Gamma$ is a context,
then
\end{comment}

PVS is a strongly typed specification language.  The simply typed fragment
includes types constructed from the base types by the function and product
type constructions, and expressions constructed from the constants and
variables by means of application, abstraction, and tupling.  Expressions
are checked to be well typed under a {\em context\/}, which is a partial
function that assigns a {\em kind\/} (one of \texttt{TYPE},
\texttt{CONSTANT}, or \texttt{VARIABLE}) to each symbol, and a type to the
constant and variable symbols.  We use the metavariables $\Gamma$,
$\Delta$, and $\Theta$ to range over contexts.  The metavariables $A$,
$B$, and $T$ range over PVS type expressions, the metavariables $r$ and
$s$ range 
over symbols (identifiers), the metavariables $x$ and $y$ range over PVS
variables, and the metavariables $a$, $b$, $f$, and $g$ range over PVS
terms.  Given 
a context $\Gamma$ and a symbol $s$, we say that $\Gamma(s)$ is
undefined if $s$ is not declared in $\Gamma$\@.

The \emph{pretypes} of the simple type theory include the \emph{base}
types such as \texttt{bool} and \texttt{real}.  A \emph{function} pretype
from domain pretype $A$ to range pretype $B$ is constructed as
$\funtype{A}{B}$\@.  A \emph{product} pretype of ${\listwo{A}{n}}$ is
constructed as $\tupletype{\listwo{A}{n}}$.
\begin{comment}
We make no
distinction between a type $A$ and the singleton tuple type $[A]$.  We
also disallow a 0-tuple type.
\end{comment}
A \emph{type} is a pretype that has been
typechecked in a given context.  Types in the simple type theory are
simple enough  that the only distinction between pretypes and types is
that the symbols in a type must be appropriately declared in the given
context.

\begin{Eg}{pretypes} 
 \texttt{bool}, \texttt{real},
 $\tupletype
{\mathtt{bool}, \mathtt{real}}$,
 $\funtype{\tupletype{\mathtt{real}, \mathtt{bool}}}{ \mathtt{bool}}$.
\end{Eg}

The {\em preterms\/} of the language consist of the constants, variables,
pairs, projections, applications, and abstractions.  The metavariables $c$
and $d$ range over constants.  Pairs are of the form $(a_1, a_2)$ where
each $a_i$ is a preterm.  Applications have the form $f\ a$ where $f$ and
$a$ are preterms.  A pair projection is an expression of the form
$\proj{i}\ a$, where $i\in\{1, 2\}$\@.
\begin{comment}
As with types, we make no
distinction between the singleton tuple preterm $(a)$ and the preterm
$a$.
\end{comment}
Lambda abstractions have the form $\lambda (x : T): a$, where $T$
is a pretype and $a$ is a preterm.
\begin{comment}
$\lambda (x_1 : T_1, \ldots,
x_n : T_n): a$, where each $T_i$ is a type, each $x_i$ is a symbol so that
$x_i\not\equiv x_j$ for $i\neq j$, and $a$ is a preterm.
\end{comment}
Parentheses
are used for disambiguation.  A \emph{term} is a preterm that has been
typechecked in a given context.

\begin{Eg}{preterms}
\texttt{TRUE}, $\neg~\mathtt{TRUE}$,
$\lambda~(x~:~\mathtt{bool}):~\neg(x)$,\hfill\\
$\proj{2}~(\mathtt{TRUE}, \mathtt{FALSE})$,
$(\mathtt{TRUE},  \lambda~(x~:~\mathtt{bool})~:~\neg~(\neg~x))$.
\end{Eg}

\section{Contexts}

A context is 
 a sequence of
declarations, where each declaration is either a type declaration
$s\ :\ \mathtt{TYPE}$, a constant declaration $c\ :\ T$ where $T$ is a type,
or a variable declaration $x\ :\ \mathtt{VAR}\ T$.
Preterms and pretypes are typechecked with respect to a given context.  
The empty context is represented as $\{\}$\@.  
The well-formedness rules
for contexts are presented below.
A context can also be applied as a
partial function so that for a symbol $s$ with declaration $D$, $(\Gamma, s
: D)(s) = D$ and 
$(\Gamma, s : D)(r) = \Gamma(r)$ for $r\neq s$.
If $s$ is not declared in $\Gamma$, then $\Gamma(s)$ is undefined. 
If $\Gamma$ is a context, then
for any symbol $s$, the kind of the symbol $s$ in $\Gamma$ is given by $\textit{kind}(\Gamma(s))$.  If the kind of $s$ in $\Gamma$ is \texttt{CONSTANT} or
\texttt{VARIABLE}, then the $ \textit{type\/}(\Gamma(s))$ is the type assigned
to $s$ in $\Gamma$.  
%\memo{Variable declarations are unrelated to variable declarations in
%PVS and correspond to binding declarations.}
\newcommand{\coln}[2]{#1\ : #2}

\begin{Eg}{context}\hfill\\
$ \mathtt{bool}\ :\  \mathtt{TYPE},\hspace{.2cm} \coln{ \mathtt{TRUE}}{
\mathtt{bool}},\hspace{.2cm} \coln{ \mathtt{FALSE}}{
\mathtt{bool}},\hspace{.2cm} \coln{x}{ \mathtt{VAR}\ \funtype{\tupletype{
\mathtt{bool}, \mathtt{bool}}}{ \mathtt{bool} }}$ 
\end{Eg}


\section{Type Rules}

The type rules for the simple type theory are given by a recursively
defined partial function
$\tau$ that assigns
\begin{enumerate}
\item  A type $\tau(\Gamma)(a)$ to a preterm $a$ that
is well typed with
respect to a context $\Gamma$. 
\item  The keyword \texttt{TYPE} as the result of $\tauGamma{A}$
when $A$ is a  well-formed type under context $\Gamma$\@.

\item  The keyword \texttt{CONTEXT} as the result of
$\tauGamma{\Delta}$ when $\Delta$ is a well-formed context
under context $\Gamma$\@.  The context $\Gamma$ is empty for the
simply typed fragment so that typechecking is always invoked as
$\tau()(\Gamma)$\@. 
\end{enumerate}
Otherwise, $\tau$ is undefined in the case of an ill-typed preterm or an
ill-formed type or context. 

The type rules are given by the recursive definition for $\tau$\@.
Typechecking in PVS assigns a ``canonical'' type to a preterm.
Customarily, type rules are presented as inference rules, but
a functional presentation is more appropriate for PVS since
\begin{enumerate}
\item The type assignment is deterministic.  A term can, in
general, though not in the simply typed fragment,  be assigned
a number of types but it always has at most one canonical type.  

\item The soundness proof need only show that the meaning of the term
is an element of the meaning of its canonical type.  Thus, only
the canonical type derivation for a term has to be shown sound and not
every valid type derivation. 

\item The meaning of a term is therefore given by recursion on the term
itself and not on its typing derivation.  There is no need to show
separately that this meaning is {\em coherent\/}, that is, independent of
the typing derivation.
\end{enumerate}
  A functional presentation of the type rules also
leads to natural and straightforward soundness arguments.  
Note that the well-formedness rules for contexts and types are trivial
in the simply typed situation but become more meaningful when the type
theory is extended.  Note also that in the type rules for expressions and
types, the well-formedness of the relevant context is not explicitly
checked.  These rules do preserve the well-formedness of the context
in each recursive call so that if the initial context is well formed, then
so is every intermediate one.    
\begin{Defn}{type rules}
\begin{eqnarray*}
  \tau()(\{\}) & = & \ttcontext \\
%
  \tau()(\Gamma, s\ :\ \tttype{}) & = & \ttcontext, \mbox{ if }
  \Gamma(s) \mbox{ is undefined}\\& &
  \mbox{and } \tau()(\Gamma) = \ttcontext\\
%
  \tau()(\Gamma, c : T) & = & \ttcontext, \mbox{ if }
   \Gamma(c) \mbox{ is undefined,}\\ & & 
\tauGamma{T} =
      \tttype{},\\
   & & \mbox{and } \tau()(\Gamma) = \ttcontext\\
%
  \tau()(\Gamma, x : \ttvar{}\ T) & = & \ttcontext, \mbox{ if }
  \Gamma(x) \mbox{ is undefined,}\\ & & 
\tauGamma{T}
    = \tttype{},\\
    & & \mbox{and } \tau()(\Gamma) = \ttcontext\\ 
%
  \tauGamma{s} & = & \tttype{}, \mbox{ if } \textit{kind\/}(\Gamma(s)) =
\tttype{}\\
%
  \tauGamma{\funtype{A}{B}} & = & \tttype{}, \mbox{ if } \tauGamma{A} =
     \tauGamma{B} = \tttype{}\\
%
  \tauGamma{\tupletype{\listwo{A}{n}}} & = & \tttype{}, \mbox{ if }
     \tauGamma{A_i} = \tttype{} \mbox{ for } 1\leq i\leq 2 \\
%
  \tau(\Gamma)(s) & = &  \textit{type\/}(\Gamma(s)),\\
                    & &  \mbox{ if } 
\textit{kind}(\Gamma(s)) \in \{ \ttconstant{}, \ttvariable \} \\
%
 \tau(\Gamma)(f\ a) & = & B, \mbox{ if } \tau(\Gamma)(f) = \funtype{A}{B}
\mbox{ and } \tau(\Gamma)(a) = A \\
%
\tauGamma{\lambda (x : T): a} & = & \funtype{T}{\tau(\Gamma, x : \ttvar{}\ T)(a)},
\mbox{ if } \Gamma(x) \mbox{ is undefined}\\& & 
  \mbox{and }\tauGamma{T} = \tttype{}\\
%\tau(\Gamma)(\lambda (x_1 : T_1, \ldots, x_n : T_n): a) & = &
%         \begin{array}[t]{l}
%	   [{\tupletype{\listwo{T}{n}}} \aro \\
%   \tau(\Gamma, x_1 : \ttvar{}\ T_1, \ldots, x_n : \ttvar{}\ T_n)(a)]
%	 \end{array} \\ & &
%\mbox{if } 
%   \tauGamma{T_i} = \tttype{}, \mbox{ for } 1\leq i\leq n,\\
%& & \mbox{and  } x_i \not\equiv x_j, \Gamma(x_i) \mbox{ is undefined},\\
% & &  \mbox{ for } 1\leq i, j \leq n, i\neq j \\
%
\tau(\Gamma)((a_1, a_2)) & = &
    [\tau(\Gamma)(a_1),  \tau(\Gamma)(a_2))]\\
%
\tau(\Gamma)(\proj{i}\ a) & = & T_i, \mbox{ where }\\ & & 
 \tau(\Gamma)(a) = [T_1, T_2] %\mbox{ for some } n, n\geq i
\end{eqnarray*}
\end{Defn}

 In the type rule for lambda abstraction, the constraint that
$\Gamma(x)$ must be undefined can be satisfied by suitably renaming the
bound variable since we treat terms as equivalent modulo the renaming of
bound variables.  

\begin{comment}
The above definition suggests that projection applications should
not be treated as applications.
\end{comment}

\begin{Eg}{type rules}  Let $\Omega$ label the context
$\coln{ \mathtt{bool}}{\tttype{}}$,
            $\coln{ \mathtt{TRUE}}{\mathtt{bool}}$,
            $\coln{ \mathtt{FALSE}}{ \mathtt{bool}}$
\begin{eqnarray*}
\tau()(\{\}) & = & \ttcontext\\
\tau()(\Omega)
   & = & \ttcontext\\
\tau(\Omega)(\funtype{\tupletype{ \mathtt{bool}, \mathtt{bool}}}{
\mathtt{bool}}) & = & \tttype{}\\
\tau(\Omega)(( \mathtt{TRUE}, \mathtt{FALSE})) & = & \tupletype{ \mathtt{bool},
\mathtt{bool}}\\
\tau(\Omega)( \proj{2} ( \mathtt{TRUE}, \mathtt{FALSE})) & = & \mathtt{bool}\\
\tau(\Omega)(\lambda (\coln{x}{ \mathtt{bool}}) : \mathtt{TRUE}) & = &
\funtype{ \mathtt{bool}}{ \mathtt{bool}}
\end{eqnarray*}
\end{Eg}

\section{Semantics}

Recall that a preterm $a$ with a type assigned by $\tau$  under context $\Gamma$
is said to be a term of type $\tau(\Gamma)(a)$
in the context $\Gamma$.  If $\gamma$ 
is an \emph{assignment} for the symbols declared in context $\Gamma$, the
semantics of the simple type theory of PVS 
is given by  mapping  a type $T$ to a (possibly empty) set
$\mathcal{M}(\Gamma\vbar \gamma)(T)$, 
and a term $a$ with assigned type $T$ to an element of the set $\mathcal{
M}(\Gamma\vbar \gamma)(T)$ in the basic universe $U$\@. 
The assignment $\gamma$ is a list of bindings of the form $\{s_1\gets
t_1\}\ldots \{s_n\gets t_n \}$\@.  The application of an assignment
$\gamma$ to a symbol $s$ is such that $\gamma \{s \gets t\} (s)$ is $t$,
whereas $\gamma \{r \gets t\}(s)$ is $\gamma(s)$ when $r\not\equiv s$\@.

The  meaning function $\mathcal{M}$ returns the meaning of
a well-formed type $A$  
and a well-formed expression $a$ in the context $\Gamma$ under an
assignment $\gamma$ as $\mathcal{M}(\Gamma\vbar \gamma)(A)$ 
and 
$\mathcal{M}(\Gamma\vbar \gamma)(a)$,respectively. 
\begin{comment}
The base types 
\texttt{bool} and \texttt{real} are mapped to $\twob$ and $\reals$,
respectively.
\end{comment}

The meanings of type names, constants, and variables
declared in $\Gamma$ are obtained from the assignment $\gamma$\@.  A
function type is mapped to the 
corresponding function space.  A product type is mapped to the corresponding
Cartesian product.
\begin{comment}
an $n$-tuple is represented by an $(n+1)$-fold
Cartesian product terminated by the sole element $\zerob$ of the set \oneb\@.
This representation  distinguishes a 3-tuple
$\tupletype{\ttbool, \ttbool, \ttbool}$ from the 
2-tuple $\tupletype{\ttbool, \tupletype{\ttbool, \ttbool}}$\@.
\end{comment}
An application term is interpreted by 
means of set-theoretic function application.  A lambda abstraction yields the
graph of the corresponding function.  A pair expression is mapped
to the corresponding set-theoretic ordered pair.    
\begin{Defn}{meaning function}
\begin{eqnarray*}
%% \begin{comment}
%%   \Mgamma{ \mathtt{bool}} & = & {\twob} \\
%%   \Mgamma{ \mathtt{real}} & = & {\reals}\\
%%   \Mgamma{ \mathtt{TRUE}} & = & {\oneb}\\
%%   \Mgamma{ \mathtt{FALSE}} & = &{\zerob}\\
%% \end{comment}
  \mathcal{M}(\Gamma\vbar \gamma)(s) & = &  \gamma(s),\\
                    & &  \mbox{ if } 
\textit{kind}(\Gamma(s)) \in \{ \tttype{}, \ttconstant{}, \ttvariable \}
\\
%
  \mathcal{M}(\Gamma\vbar \gamma)(\funtype{A}{B}) & = & \Mgamma{B}^{\Mgamma{A}}\\
%
  \Mgamma{\tupletype{\listwo{T}{n}}} & = &  \Mgamma{T_1} \times %%\ldots\times
\Mgamma{T_2} %\times {\oneb}
\\
%
\mathcal{M}(\Gamma\vbar \gamma)(f\ a) & = & (\mathcal{M}(\Gamma\vbar \gamma)(f))(\mathcal{M}(\Gamma\vbar \gamma)(a)) \\
%
\Mgamma{\lambda (x : T): a} & = & \{\tuple{y, z} \vbar \begin{array}[t]{l}
						       y\in\Mgamma{T},\\
                					z = \mathcal{M}(\Gamma, x: \ttvar{}\ T\vbar \gamma\{x\gets y\})(a)\}
						       \end{array}\\
%               
%\mathcal{M}(\Gamma| \gamma)(\lambda (x_1 : T_1, \ldots, x_n : T_n): a) & = &
%  {\begin{array}[t]{l}\{ \tuple{\tuple{\listwo{y}{n}, {\zerob}}, z} | \\
%					y_1\in \Mgamma{T_1}\\
%                                        \vdots\\
%                                        y_n\in \Mgamma{T_n}\\
%                                   z = \mathcal{M}\left(
%        \begin{array}{l}
%  \Gamma, x_1 : \ttvar{}\ T_1, \ldots, x_n : \ttvar{}\ T_n;\\
%  \gamma\{x_1 \gets y_1\}\ldots \{x_n\gets y_n\}
%	\end{array}\right)(a)\} 
%				      \end{array}}\\
%
\Mgamma{(\listwo{a}{n})} & = & \tuple{\Mgamma{a_1},  \Mgamma{a_2}} \\
%
\Mgamma{\proj{i}\ a} & = & t_i, \mbox{ where } \Mgamma{a} =
\tuple{\listwo{t}{n}}
\end{eqnarray*}
\end{Defn}

\begin{Eg}{meaning function}
 Let $\omega$ be an assignment for the context $\Omega$ in
Example~\ref{eg:type rules}, of the form
$$\{\ttbool\gets \twob\}\{ \mathtt{TRUE}\gets \oneb\}\{
\mathtt{FALSE}\gets \zerob\}$$
then
\begin{eqnarray*}
\mathcal{M}(\Omega\vbar \omega)(\tupletype{\ttbool, \ttbool}) & = &
\twob\times\twob\\
%
\mathcal{M}(\Omega\vbar \omega)(( \mathtt{TRUE}, \mathtt{FALSE})) & = &
\tuple{\oneb, \zerob}\\
%
\mathcal{M}(\Omega\vbar \omega)(\lambda (x : \ttbool): \mathtt{TRUE}) & = &
\{\tuple{\zerob, \oneb}, \tuple{\oneb, \oneb}\}
\end{eqnarray*}
\end{Eg}

\begin{Defn}{satisfaction}
A context assignment $\gamma$ is said to \emph{satisfy} a context $\Gamma$
(in symbols $\gamma\models \Gamma$) 
iff
\begin{enumerate}
 \item $\gamma(\ttbool) = {\twob},$
 \item $\gamma(\tttrue) = {\oneb},$
 \item $\gamma(\ttfalse) = {\zerob},$
\item $\gamma(s)\in U$ whenever $ \itkind{}(\Gamma(s)) = \tttype$, and
 \item $\gamma(s) \in \Mgamma{ \ittype{}(\Gamma(s))}$\hfill\\\hspace*{1in}
whenever $ \itkind{}(\Gamma(s)) \in \{ \ttconstant,
\ttvariable \}$.
\end{enumerate}
\end{Defn}
\begin{Eg}{satisfaction}\hfill\\\begin{enumerate}
\item The assignment $\omega$ satisfies context $\Omega$.

\item The assignment $\omega \{ \mathtt{one}~\gets~{\oneb}\}\{
\mathtt{zero}~\gets~\zerob\}$ satisfies the context
$\Omega,~\mathtt{one~:~TYPE},~\mathtt{zero~:~one}$\@.  
\end{enumerate}
\end{Eg}

We need one useful proposition that asserts that typing judgements are not
invalidated when the context is extended.
\begin{prop}\label{context-weakening} %where is this used? NSH(9.6.95)
If $\tau()(\Gamma) = \tau()(\Gamma') = \ttcontext$ and $\Gamma$ is
a prefix of $\Gamma'$, then for all pretypes $A$, $\tauGamma{A} = \tttype$
implies $\tau(\Gamma')(A) = \tttype$, and for all preterms $a$,
$\tauGamma{a} = A$ implies $\tau(\Gamma')(a) = A$\@.  
\end{prop}

The following theorems follow from the induction suggested by the
definitions of $\tau$ and $\mathcal{M}$.  The first of these
is straightforward and is given without proof.
\begin{theorem}[type construction]  \label{type construction}
If $\tau()(\Gamma) = \ttcontext$ and $\tauGamma{a} = A$, then
$\tauGamma{A} = \tttype$.
\end{theorem}
%
\begin{theorem}[type soundness]\label{simple-type-semantics}
 If $\tau()(\Gamma) = {\ttcontext}$, 
   $\gamma$ satisfies $\Gamma$, and
   $\tauGamma{A} = {\tttype{}} $, then 
$\Mgamma{A} \in U$\@.
\end{theorem}
\begin{proof}
The proof is by induction on the structure of the pretype $A$.
Recall that if $X\in U$, then for some $i$, $X \in U_i$\@.  
This yields three cases:
\begin{enumerate}
\item $A \equiv s$: By Definition~\ref{defn:type rules},
$\Gamma(s)$ is defined and $\itkind(\Gamma(s))$ is $\tttype$\@.
Then by Definition~\ref{defn:meaning function}, $\Mgamma{s}$
is $\gamma(s)$, and by Definition~\ref{defn:satisfaction},
$\gamma(s)\in U$\@.

\item $A\equiv \funtype{B}{C}$: We then have that $\tauGamma{B} =
\tauGamma{C} = \tttype$\@.  Letting $X$ label $\Mgamma{B}$, and $Y$ label
$\Mgamma{C}$, we have by the induction hypothesis that $X\in U$ and $Y\in
U$\@.  Let $j$ be the least rank such that $\Mgamma{B}\in U_j$ and
$\Mgamma{C}\in U_j$\@.  By Definition~\ref{defn:meaning function},
$\Mgamma{A} = Y^X$, and hence $\Mgamma{A} \in U_{j+1}$ by
Definition~\ref{defn:type universe}\@.

\item $A \equiv \tupletype{\listwo{A}{n}}$: Again by
Definition~\ref{defn:type rules} and the induction hypothesis, we have
for each $i \in \{1, 2\}$, that $\Mgamma{A_i}\in U$\@.  Let $j$ be the least
rank such that for $i\in \{1, 2\}$, $\Mgamma{A_i}\in U_j$\@.  Then,
it is easy to verify from Definition~\ref{defn:type universe} that
$\Mgamma{A}\in U_{j + 1}$\@.  
\end{enumerate}
\end{proof}

\begin{theorem}[term soundness]\label{simple-term-semantics}
 If $\tau()(\Gamma) = \ttcontext$, 
   $\gamma$ satisfies $\Gamma$, and
   $\tauGamma{a}$ is defined and equal to $A$, then 
$\Mgamma{a} \in \Mgamma{A}$.  
\end{theorem}
\begin{proof}
The proof is by induction on the structure of preterms.
\begin{enumerate}
\item $a\equiv s$: By Definition~\ref{defn:type rules},
we have that $\ittype(\Gamma(s)) = A$\@.  By
Definitions~\ref{defn:meaning function}
and~\ref{defn:satisfaction}, we have that $\Mgamma{a} = \gamma(s)$
and $\gamma(s) \in
\Mgamma{A}$\@.
\item $a \equiv (f\ b)$:  By Definition~\ref{defn:type rules},
$\tauGamma{f} = \funtype{B}{A}$, and $\tauGamma{b} = B$, for some
$B$ such that $\tauGamma{B} = \tttype$\@.  
Let $\Mgamma{A}$ be $X$ and $\Mgamma{B}$ be $Y$, then 
by Definitions~\ref{defn:type rules}
and~\ref{defn:meaning function}, and the induction hypothesis, we have
$\Mgamma{f} \in X^Y$ and $\Mgamma{b}\in Y$\@.  It therefore follows
by Definition~\ref{defn:meaning function} that $\Mgamma{(f\ b)} =
(\Mgamma{f})(\Mgamma{b})$, and hence $\Mgamma{(f\ b)} \in X$\@.

\item $a \equiv (\lambda (x : C): b)$:  By
Definition~\ref{defn:type rules}, we have that
$\tauGamma{a}$ is $\funtype{C}{B}$, where
$\tau(\Gamma, x : \ttvar~C)(b)$ is $B$\@.  
Let $X$ be $\Mgamma{C}$, and $Y$ be
$\mathcal{M}(\Gamma, x: \ttvar~C \vbar \gamma\{x\gets u\}))(B)$\@.  
By the induction hypothesis, we have that for
any $u\in Y$, 
$\mathcal{M}(\Gamma, x: \ttvar~C \vbar \gamma\{x\gets u\})(b)
\in X$\@.  
Since $\Mgamma{a}$ is $\{\pair{u, v} \vbar u\in X, v = \mathcal{M}(\Gamma, x:
\ttvar~C \vbar \gamma\{x\gets u\})(b)\}$,
we have that $\Mgamma{a} \in X^Y$\@.

\item $a \equiv (a_1, a_2)$: By Definition~\ref{defn:type rules},
$\tauGamma{a} = \tupletype{\listwo{A}{n}}$, where $\tauGamma{a_i} = A_i$ for
$i\in \{1, 2\}$\@.   By the induction hypothesis, $\Mgamma{a_i}\in
\Mgamma{A_i}$ for $i\in \{1, 2\}$\@.  By
Definition~\ref{defn:meaning function}, $\Mgamma{a} =
\pair{\Mgamma{a_1},\Mgamma{a_2}}$ and hence $\Mgamma{a}$ is
an element of $\Mgamma{A}$ which is
$\Mgamma{A}\times\Mgamma{A_n}$\@.

\item $a\equiv \proj{i}~b$: In this case, we know by
Definition~\ref{defn:type rules} that $\tauGamma{b} =
\tupletype{\listwo{A}{n}}$
with $i\in \{1, 2\}$, and $\tauGamma{a} = A_i$\@.
By the induction hypothesis, $\Mgamma{b} = \pair{t_1, t_2}$,
and by Definition~\ref{defn:meaning function}, $\Mgamma{a} = t_i$
and $\Mgamma{\tauGamma{b}} =
\Mgamma{A_1}\times\Mgamma{A_2}$,
hence $\Mgamma{a}\in \Mgamma{A_i}$\@.  
\end{enumerate}
%%\memo{What about tuples/projections?}
\end{proof}

These three theorems (\ref{type construction},
\ref{simple-type-semantics}, and \ref{simple-term-semantics}) are the key
invariants that must be satisfied by the semantics when the language is
extended below with type definitions, subtypes, dependent types, and
parametric theories.

\section{Some Syntactic Operations}

We first define the operation of collecting the free variables of a
term $a$ in a given context $\Gamma$ as $FV(\Gamma)(a)$, and then
define the operation of substitution.
\begin{Defn}{free variables}
\begin{eqnarray*}
  FV(\Gamma)(s) & = & \left\{\begin{array}{l}
			      \{s \}, \mbox{ if } \textit{kind}(\Gamma(s))
= \ttvariable\\
                              \emptyset, \mbox{ otherwise}
			    \end{array}\right.\\
%
  FV(\Gamma)(f\ a) & = & FV(\Gamma)(f) \union FV(\Gamma)(a) \\
%
  FV(\Gamma)(\lambda (x : T) : a) & = &
     FV(\Gamma, x : \ttvar{}\ T)(a)
- \{x\} \\
%
 FV(\Gamma)((\listwo{a}{n})) & = & FV(\Gamma)(a_1)\union
FV(\Gamma)(a_2)\\
%
 FV(\Gamma)( \proj{i}\ a) & = & FV(\Gamma)(a)
\end{eqnarray*}
\end{Defn}
%
\begin{Defn}{substitution}
\begin{eqnarray*}
  s[a_1/x_1,\ldots, a_n/x_n] & = & \left\{\begin{array}{l}
					    a_i, \mbox{ if for some minimal } i,
s \equiv x_i\\
                                            s, \mbox{ otherwise}
					  \end{array}\right.\\
%
  (f\ a)[a_1/x_1,\ldots, a_n/x_n] & = &
     (f[a_1/x_1,\ldots, a_n/x_n]\\&&
       \hspace*{.5in} a[a_1/x_1,\ldots, a_n/x_n]) \\
%
  (\lambda (y : T): a)[a_1/x_1,\ldots, a_n/x_n] & = &
    (\lambda (y':T) :
     \ a[y'/y, a_1/x_1,\ldots, a_n/x_n]), \\ & & 
%
\mbox{where  } y' \mbox{ is a fresh variable}\\
%
 (\listwo{b}{m})[a_1/x_1,\ldots, a_n/x_n] & = &
     (b_1[a_1/x_1,\ldots, a_n/x_n],\\&&
       \hspace*{.5in}b_2[a_1/x_1,\ldots, a_n/x_n])\\
%
  ( \proj{i}\ a)[a_1/x_1,\ldots, a_n/x_n] & = &
   ( \proj{i}\ a[a_1/x_1,\ldots, a_n/x_n])
\end{eqnarray*}
\end{Defn}

Recall that terms are treated as syntactically equivalent modulo
alpha conversion.  
The above definitions must be extended as more 
features are added to the language. 

\section{Type Definitions}

%%\memo{See ADT section for recursive datatypes.}

Here we enrich contexts so that type symbols may have definitions.  PVS
does not allow recursive type definitions\footnote{For the moment, we are
not considering the PVS \texttt{DATATYPE} mechanism, which is a form of
recursive type definition~\cite{Shankar:ADT97}.  Recursive datatypes in
the context of the HOL proof checking system are described by
Melham~\cite{Melham89}.}  so a type declaration/definition in a context
may use only the symbols declared in the prior part of the context.  The
main difference in the extended language is that type names can have
definitions.  In such cases, the definitions rather than the type names
are used to determine the actual type of an expression.  In other words,
two type expressions are treated as the same if they are
\emph{definitionally equivalent}\@.  Most other specification languages
tend to employ the weaker notion of \emph{name equivalence} where
syntactically different types are treated as distinct even when their
definitions coincide.

To accommodate type definitions, a context can
contain type declarations of the form $s\ :\ \mathtt{TYPE} = T$, where $T$
is a type.  If context $\Gamma$ contains such a declaration for $s$, then
$ \itdef{}(\Gamma(s))$ is $T$\@.  To extend $\tau$ to handle type
definitions under definitional equivalence, we must ensure that $\tau$
returns the canonical form of a type where all defined types have been
replaced by their definitions.  The operation $\delta(\Gamma)(T)$ returns
the expanded form of a type relative to the context $\Gamma$\@.
%
\begin{Defn}{expanded type}
\begin{eqnarray*}
  \delta(\Gamma)(s) & = & s, \mbox{ if } \textit{definition\/}(\Gamma(s))
\mbox{ is empty}\\
% 
  \delta(\Gamma)(s) & = & \delta(\Gamma)(
\textit{definition\/}(\Gamma(s))), 
\mbox{ if } \textit{definition\/}(\Gamma(s))
\mbox{ is nonempty} \\
%
 \delta(\Gamma)(\funtype{A}{B}) & = &
      \funtype{\delta(\Gamma)(A)}{\delta(\Gamma)(B)} \\
%
 \delta(\Gamma)(\tupletype{\listwo{T}{n}}) & = &
     \tupletype{\delta(\Gamma)(T_1),  \delta(\Gamma)(T_2)}
\end{eqnarray*}
\end{Defn}


The typing rules are augmented to return the type in expanded form.
The main issue here is to determine that the definition part of a type
declaration in a context is well formed relative to the preceding context.
We also need to ensure that $\tau$ returns the expanded form of the type
corresponding to a preterm.
%
\begin{Defn}{type rules with type definitions}
\begin{eqnarray*}
    \tau()(\Gamma, s\ :\ \tttype{} = T) & = & \ttcontext, \mbox{
if }  \Gamma(s) \mbox{ is undefined}, \\ & &
\tau()(\Gamma) = \ttcontext,\\ & & 
  \mbox{and }     \tauGamma{T} = \tttype{} \\
%
%   \tau()(\Gamma, c : T = a) & = & \ttcontext, \mbox{ if } \tauGamma{T} =
%      \tttype{} \mbox{ and } \tauGamma{a} = \delta(\Gamma)(T)\\
%
    \tau(\Gamma)(s) & = &  \delta(\Gamma)(\textit{type\/}(\Gamma(s))),\\
                    & &  \mbox{ if } 
\textit{kind}(\Gamma(s)) \in \{ \ttconstant{}, \ttvariable \} 
\end{eqnarray*}
\end{Defn}


Note that the $\delta$ operator is idempotent, and $\tauGamma{a}$
for a term $a$ always returns an expanded type, that is, 
$\delta(\tauGamma{a}) = \tauGamma{a}$\@.  

We do not need to update the definition of ${\mathcal M}$ from
Definition~\ref{defn:meaning function} since the syntax for terms is
unchanged, 
but we do need
to revise the notion of a satisfying context assignment (from 
Definition~\ref{defn:satisfaction}) to respect the type
definitions.
\begin{Defn}{satisfaction with type definitions}
An assignment $\gamma$ satisfies a context $\Gamma$ if
in addition to the conditions in Definition~\ref{defn:satisfaction},
whenever $ \textit{kind}(\Gamma(s)) = \tttype{}$ and
$\textit{definition}(\Gamma(s))$ (abbreviated as $T$) is nonempty, then
$\gamma(s) = \Mgamma{ T}$.
\end{Defn}

Theorems~\ref{type construction} and~\ref{simple-type-semantics}
and~\ref{simple-term-semantics} continue to hold under these extensions,
and the proofs are easily adapted to the modified definitions.  

\begin{Eg}{type definition}
Let $\Omega'$ be the context\\ $\Omega,
\mathtt{boolop}:\mathtt{TYPE}=\funtype{\tupletype{\mathtt{bool}, \mathtt{bool}}}{\ttbool},
\vee:\mathtt{boolop}$\@.  Then
\begin{eqnarray*}
\tau()(\Omega') & = & \mathtt{CONTEXT}\\
%
 \delta(\Omega')( \mathtt{boolop})  & = & \funtype{\tupletype{ \mathtt{bool},
\mathtt{bool}}}{\ttbool},\\
 \tau(\Omega')( \vee) & = & \funtype{\tupletype{ \mathtt{bool}, \mathtt{bool}}}{\ttbool}
\end{eqnarray*}
\end{Eg}

\section{Summary}

We have defined the simply typed fragment of PVS by introducing the syntax
for pretypes and preterms, the type rules and semantics for well-formed
contexts, types, and terms.  The type rules are presented in a novel
functional style where each well-formed context is assigned the label {\tt
CONTEXT}, each well-formed type is assigned the label {\tt TYPE}, and each
well-formed term is assigned a canonical type.  The semantics takes a
satisfying assignment for a context and maps a well-formed type to a set
and a well-formed term to an element of the set corresponding to its
canonical type.  We then defined the syntactic operations of collecting
the free variables in an expression and for substituting terms for
variables in an expression.

The simple type theory is then extended with type definitions.  With this
extension, two type expressions are treated as equivalent if they are
identical after all type definitions have been expanded.  The operation
$\delta$ returns the expanded form of a given type expression.


\chapter{Adding Subtypes}\label{subtypes}

Subtyping is one of the main features of the PVS specification
language.\footnote{The form of subtyping used in PVS is derived from a
suggestion of Friedrich von~Henke.}  Subtyping in PVS corresponds to the
set-theoretic notion of a subset.  It raises several delicate issues that
were absent in the language presented thus far.  In the simply typed
fragment, each type corresponds to a set of values that is somehow
structurally different from the set of values for another type so that a
term has at most one type.  Subtyping makes it possible to introduce the
natural numbers as a subtype of the reals, and to treat the primes, the
even numbers, and the odd numbers as subtypes of the natural numbers.
With subtyping, a term can obviously have several possible types, but the
type-checking function $\tau$ may return only a single type.  We constrain
$\tau$ to return a natural canonical type of an expression that is given
by the declarations of the symbols in the expression.  If the expression
is used in a context where the expected type is a supertype of its
canonical type, then the type correctness is straightforward.  If the
expected type is a subtype that is {\em compatible\/} with the canonical
type of the expression, then typechecking generates proof obligations
asserting that the expression satisfies the predicate constraints imposed
by the expected type.  Two types are compatible if they have equivalent
maximal supertypes.  Type equivalence in the presence of subtypes is not a
simple notion.  Subtyping also introduces the possibility of types being
empty.  Typed lambda calculi with possibly empty types have been studied
by Meyer, Mitchell, Moggi, and Statman~\cite{MMMS:empty}.  This chapter
introduces predicate subtypes and defines the notions of compatibility and
type equivalence prior to presenting the type rules and semantics.


We restrict our attention to contexts $\Gamma$ that extend
the declarations:
\begin{alltt}
 bool : TYPE,
 TRUE : bool,
 FALSE : bool,
 boolop : \(\funtype{\tupletype{\ttbool, \ttbool}}{\ttbool}\),
 \(\neg\) : \(\funtype{\ttbool}{\ttbool}\),
 \(\vee\) : boolop,
 \(\wedge\) : boolop,
 \(\supset\) : boolop
\end{alltt}
We will abuse PVS notation to employ the customary infix forms of
operations like $\vee$, 
$\wedge$, and $\supset$.    The pretype corresponding to a predicate
subtype has the form $\{x : T \vbar a\}$ where $x$ is a symbol, $T$ is a
pretype, and $a$ is a preterm.   
A \emph{predicate type} in PVS is a function type where the range is the
primitive type \texttt{bool}.  A predicate is a term that has a predicate
type.  If $a$ is a term of type \texttt{bool}, then 
we can define the subtype $\{ x : T \vbar a\}$ consisting of those
elements $e$ of $T$ satisfying $a[e/x]$ ($e$ substituted for $x$ in $a$).
Since the elements of the subtype $\{ x : T \vbar a\}$ satisfy the predicate $
\lambda (x  : T): a$, 
we call this type a {\em predicate subtype\/} to distinguish it from other
forms of subtyping.  
Universal quantification $\forall (x : T): a$ is just an abbreviation
for the term $(\lambda (x : T): a) = (\lambda (x : T): \mathtt{TRUE})$\@.  
Although we use the equality  predicate in the definition of universal
quantification and in the definitions below, the actual introduction of
equality is deferred to a later section following the introduction of
parametric theories.  The equality between PVS terms of function type
is to be interpreted as extensional equality.  Note that the `$=$' symbol is
used both for the formal equality symbol in the language and for
metatheoretic equality.   
\begin{comment}
We will also use the notation $\lambda (x_1 : T_1, \ldots, x_n : T_n): a$
in place of $\lambda (x : \tupletype{\seq{T}{n}}): a[
\proj{1}(x)/x_1, \ldots, \mathtt{PROJ\_n}(x)/x_n]$\@.
\end{comment}



Our first step will be to define the notion of a \emph{maximal supertype}
of a given type as $\mu(T)$.  A \emph{maximal type} $T$ is one such that
$\mu(T) = T$.  In a given context, we will apply  $\mu$ only to the
expanded form (given by $\delta$) of a type expression.  
\begin{Defn}{maximal supertype}
\begin{eqnarray*}
  \mu(s) & = & s \\
   \mu(\{x : T \vbar a\}) & = & \mu(T) \\
%
   \mu(\funtype{A}{B}) & = & \funtype{A}{\mu(B)} \\
  \mu(\tupletype{\listwo{A}{n}}) & = & \tupletype{\mu(A_1), \mu(A_2)}
\end{eqnarray*}
\end{Defn}
Note that since subtypes correspond to subsets, in taking the maximal
supertype of a function type, the domain type is held fixed.  In most type
theories with subtypes, the rule for subtyping between function types
$\funtype{A}{B}$ and $\funtype{A'}{B'}$ requires showing that $A'$ is a
subtype of $A$, and $B$ is a subtype of $B'$\@.  Subtyping between
function types is therefore said to be contravariant in the domain type
and covariant in the range type.  Subtyping on function types in PVS is
covariant in the range type but is neither covariant nor contravariant in
the domain type.  This means that the function type
$\funtype{\ttnat}{\ttnat}$ is not a supertype of the function type
$\funtype{\ttint}{\ttnat}$\@.  Such a subtyping relation would violate
{\em extensionality\/}\@.  Two functions on $\ttnat$ are extensionally
equal when they return equal values when applied to equal arguments in
$\ttnat$\@.  Consider two functions in $\funtype{\ttnat}{\ttnat}$:
$\mathit{abs}$ which returns the absolute value, and $\mathit{idnat}$
which behaves as an identity function on natural numbers and returns $0$
otherwise.  These two functions will be erroneously identified if they can
be viewed as being of type $\funtype{\ttnat}{\ttnat}$, and the subset
interpretation of subtypes would be lost.

We will also employ a weaker supertype $\mu_0(T)$ or the
{\em direct supertype\/}, that only considers supertypes of
explicitly given subtypes of the form $\{x : T \vbar a\}$.
\begin{Defn}{direct supertype}
\begin{eqnarray*}
\mu_0(\{x : T \vbar a\}) & = & \mu_0(T)\\
\mu_0(T) & = & T, \mbox{ otherwise}
\end{eqnarray*}
\end{Defn}

\begin{Eg}{maximal supertype}
Given a context containing the declarations
$$\begin{array}{l}
  \ttint : \tttype, \\
  \mathtt{0} : \ttint,\\
  \leq : \funtype{\tupletype{\ttint, \ttint}}{\ttbool},\\
  \ttnat : \tttype = \{ i : \ttint \vbar  0 \leq i\}\\
  \mathtt{natinjection} : \tttype = \{ f : \funtype{\ttnat}{\ttnat}\ \vbar\ 
\forall (i, j: \ttnat): f(i) = f(j) \supset i = j \}
\end{array}$$
we have
\begin{eqnarray*}
  \mu(\mathtt{natinjection}) & = & \mu(\funtype{\ttnat}{\ttnat})\\
 &=  & \funtype{\ttnat}{\mu(\ttnat)}\\
 &=  & \funtype{\ttnat}{\ttint}\\
\mu_0(\mathtt{natinjection}) & = & \funtype{\ttnat}{\ttnat}
\end{eqnarray*}
\end{Eg}

Note that   $\mu(\mu(A)) = \mu(A)$.
Note also that a maximal supertype is never a subtype.  
We can in fact collect the predicates that constrain a type $A$
relative to its maximal supertype $\mu(A)$ as $\pi(A)$\@.

\begin{Defn}{subtype constraints}
\begin{eqnarray*}
  \pi(s) & = & \lambda (x : s): \tttrue \\
   \pi(\{y : T \vbar a\}) & = &   \lambda (x : \mu(T)): (\pi(T)(x) \wedge a[x/y])\\
%
   \pi(\funtype{A}{B}) & = &   \lambda (x : \funtype{A}{\mu(B)}):
    (\forall (y: A): \pi(B)(x(y))) \\
%
\pi(\tupletype{\listwo{A}{n}}) & = &
    \lambda (x: \tupletype{\mu(A_1),  \mu(A_2)}): 
	(\pi(A_1)(\proj{1}~x)\wedge \pi(A_2)(\proj{2}~x))
\end{eqnarray*}
\end{Defn}
Observe that in Definition~\ref{defn:subtype constraints}, if $\tauGamma{A} =
\tttype$, then 
$\tauGamma{\pi(A)} = \funtype{\mu(A)}{
 \mathtt{bool}}$\@.\footnote{This is somewhat tricky in the case of
$\pi(\{y:T\vbar a\})$ since in $a[x/y]$, $x$ has type $\mu(T)$, whereas
$y$ has type $T$\@.  As shown in Chapter~\ref{conditionals}, the
type rules for conjunction are such that $\tau(\Gamma, x :
\ttvar~\mu(T))(\pi(T)(x) \wedge a)$ reduces to
$\tau(\Gamma, x : \ttvar~\mu(T))(\pi(T)(x))$ and
$\tau(\Gamma, x : \ttvar~\mu(T), \pi(T)(x))(a[x/y])$ where the first
conjunct is added to the context as a contextual assumption.
One can then show by induction that $\tau(\Gamma, x:\ttvar~\mu(T),
\pi(T)(x))(a[x/y]) = \tau(\Gamma, y: \ttvar~T)(a)$\@.  }

\begin{Eg}{subtype constraints}
\begin{eqnarray*}
\lefteqn{\pi(\ttnat)}\\
  & = & \lambda (j: \ttint): 0\leq j\\
\lefteqn{\pi(\funtype{\ttnat}{\ttnat})}\\
& = & \lambda (g : \funtype{\ttnat}{\ttint}): \forall (i:\ttnat): (\lambda
(j:\ttint): 0 \leq j)(g(i))\\
\lefteqn{\pi(\mathtt{natinjection})}\\
&= & \lambda (f : \funtype{\ttnat}{\ttint}) : 
\begin{array}[t]{cl}
& \pi(\funtype{\ttnat}{\ttnat})(f)\\
\wedge & (\forall (i, j: \ttnat): f(i) = f(j) \supset  i = j)
\end{array}\\
& = & 
\lambda (f : \funtype{\ttnat}{\ttint}) :\\&&\hspace*{.5in}
\begin{array}[t]{cl}
& (\lambda (g : \funtype{\ttnat}{\ttint}) : \forall (i: \ttnat):
(\lambda (j : \ttint): \mathtt{0}\leq j)(g(i)))(f) \\
\wedge & (\forall (i, j: \ttnat): f(i) = f(j) \supset  i = j)
\end{array}
\end{eqnarray*}
\end{Eg}
Observe that $\pi(\mu(A))$ is essentially equivalent to $\lambda (x : \mu(A)):
\mathtt{TRUE}$. 

Since the subtype $\{x : T \vbar p(x) \wedge q(x)\}$ can also be written
as $\{x : T \vbar q(x) \wedge p(x)\}$, we need a notion of equivalence
between types.  One way to do this is to make types ``first-class'' and to
allow explicit theorems to be proved about type equivalence and subtyping.
Since this would be a fairly drastic extension to the specification
language, we have designed the PVS type system so as to avoid any
first-class treatment of types.  It turns out that all the needed
properties about types (such as equality and subtyping) can be obtained by
generating ordinary proof obligations rather than by explicitly proving
theorems about types.

We introduce 
below a metatheoretic operation that generates the proof obligations
needed to establish that two (maximal) types are equivalent.  This
equivalence is denoted by $\simeq$ and is  applied only to maximal types
and returns a list of the 
proof obligations that must be proved.  Note the invariant in the
definition below that the arguments to $\simeq$ are always maximal.  The
definition of $\simeq$ makes use of the PVS equality predicate that will be
introduced later.  A list of formulas is represented as $a_1,\ldots, a_n$.
Given two such lists $\lis{a}{m}$ and $\lis{b}{n}$, the concatenation of
these two lists is written as $\lis{a}{m}\ ; \ \lis{b}{n}$\@.

\begin{Defn}{type equivalence proof obligations}
\begin{eqnarray*}
  (s \simeq s) & = & \tttrue \\
%
  (\funtype{A}{B} \simeq \funtype{A'}{B'}) & = & ((\mu(A)\simeq \mu(A'));
(\pi(A) = \pi(A'));
 (B \simeq B'))\footnotemark \\
%
  (\tupletype{\listwo{A}{n}} \simeq \tupletype{\listwo{B}{n}}) & = &
    ((A_1\simeq B_1); (A_2\simeq B_2)) \\
%
 (A \simeq B) & = & \ttfalse, \mbox{ otherwise}
\end{eqnarray*}
\end{Defn}
\footnotetext{The type correctness of the proof obligation $(\pi(A) =
\pi(A'))$ depends on the \emph{prior} proof obligations $\mu(A)\simeq
\mu(A')$.}

\begin{Eg}{type equivalence}  Building on the context given in
Example~\ref{eg:maximal supertype}, 
if we have the following variants of $\ttnat{}$ and
\texttt{natinjection}:
$$
\begin{array}{l}
\mathtt{NAT} :  \tttype = \{ i : \ttint \vbar  i \leq \mathtt{0}\ 
\supset\ i = 0\}\\
\mathtt{NATinjection} : \tttype = \{ f : \funtype{ \mathtt{NAT}}{
\mathtt{NAT}} \vbar 
\forall (i, j: \mathtt{NAT}): f(i) = f(j)\ \supset\ i = j \}
\end{array}$$
we get
\begin{eqnarray*}
\mu(\funtype{ \mathtt{natinjection}}{ \mathtt{natinjection}}) =
 \funtype{ \mathtt{natinjection}}{\funtype{\ttnat}{\ttint}}\\
%
\mu(\funtype{ \mathtt{NATinjection}}{ \mathtt{NATinjection}}) =
 \funtype{ \mathtt{NATinjection}}{\funtype{ \mathtt{NAT}}{ \ttint}}
\end{eqnarray*}
\begin{eqnarray*}
\lefteqn{\mu(\funtype{ \mathtt{natinjection}}{ \mathtt{natinjection}})
\simeq \mu(\funtype{ \mathtt{NATinjection}}{ \mathtt{NATinjection}})}\\
& = &  \begin{array}[t]{l}
       (\mu( \mathtt{natinjection}) \simeq \mu( \mathtt{NATinjection}));\\
       (\pi( \mathtt{natinjection}) = \pi( \mathtt{NATinjection}));\\
       (\funtype{\ttnat}{\ttint} \simeq \funtype{ \mathtt{NAT}}{ \mathtt{int}})     
       \end{array}\\
 \lefteqn{(\pi( \mathtt{natinjection}) = \pi( \mathtt{NATinjection}))}\\
& = & \begin{array}[t]{cl}
(&      \lambda (f : \funtype{\ttnat}{\ttint}) :
\begin{array}[t]{cl}
& (\lambda (g : \funtype{\ttnat}{\ttint}) : \forall (i: \ttnat):
\mathtt{0}\leq  g(i))(f) \\
\wedge & (\forall (i, j: \ttnat): f(i) = f(j) \ \supset\  i = j)
 \end{array}\\
 =  \\&  \lambda (f : \funtype{ \mathtt{NAT}}{\ttint}) :\\&\hspace*{.3in}
\begin{array}[t]{cl}
& (\lambda (g : \funtype{ \mathtt{NAT}}{\ttint}) : \forall (i: \mathtt{NAT}):
g(i) \leq \mathtt{0}\ \supset\ g(i) = 0)(f) \\
\wedge & (\forall (i, j: \mathtt{NAT}): f(i) = f(j) \ \supset\  i = j)
      \end{array} )
 \end{array}\\
\lefteqn{(\funtype{\ttnat}{\ttint} \simeq \funtype{ \mathtt{NAT}}{
\mathtt{int}})}\\
& = & (\ttint \simeq \ttint);\\&& (\lambda (i : \ttint): \mathtt{0} \leq i) =
                              (\lambda (i : \ttint): 
i\leq \mathtt{0}\  \supset\ i = 0); (\ttint\simeq\ttint)
\end{eqnarray*}
\end{Eg}

A basic question during typechecking is whether two types are
\emph{compatible}, that is,  have the same maximal supertype.  
Two types are said to be compatible if the type equivalence proof
obligations on their respective 
maximal supertypes are provable.  The provability of a formula $a$
under context $\Gamma$ is represented as $\vdash_\Gamma a$\@.  
\begin{Defn}{compatible}
\emph{Two types $A$ and $B$ are said to be compatible in context $\Gamma$
(in notation, $(A \sim B)_\Gamma$)
if $\vdash_\Gamma a$, for each $a$ in $(\mu(A)\simeq \mu(B))$.}\footnotemark
\end{Defn}
\footnotetext{ The PVS proof rules are described in 
Chapter~\ref{proof theory}.}


\begin{comment}
With subtypes, it is easy to define an empty subtype and then
declare a constant of such an empty subtype.  The context thus has to be
checked to disallow such an inconsistency.
\end{comment}
We now extend the
definition of $\delta$ to the case of subtypes so that it leaves the
predicate unchanged but expands the definition of the supertype.
%
\begin{Defn}{expanded type with subtypes}
\begin{eqnarray*}
  \delta(\Gamma)(\{x : T \vbar a\}) & = & \{x : \delta(\Gamma)(T) \vbar a \}
\end{eqnarray*}
\end{Defn}

We now extend the definition $\tau$ to the case of subtypes.  Here we
could force $\tau$ to always return a maximal supertype but this is not
done in Definition~\ref{defn:type rules with subtypes}  since it
would weaken the soundness theorem without significantly simplifying the
definition of the type rules.  The typechecking of contexts has to be
modified to generate a {\em nonemptiness\/} proof obligation for the type of
any constant declaration.  A constant of an empty type would lead to an
inconsistent context, and this would mean that constant declarations are
not conservative extensions.  This modification to
Definition~\ref{defn:type rules} is not needed for soundness since an
inconsistent context makes soundness trivial.  It is needed to show that
constant declarations and definitions are conservative extensions.  Note
that with subtypes, the type rule for an application is modified to check
that the domain type of the function is compatible with the type of its
argument, and that the argument satisfies any constraints imposed by the
domain type of the function.  The case of projection expressions is also
not straightforward since the argument type can be a subtype of a tuple
type.  In this case, we use the direct supertype (see
Definition~\ref{defn:direct supertype}) which must be a tuple type.
%
\begin{Defn}{type rules with subtypes}
\begin{eqnarray*}
  \tau()(\Gamma, c : T) & = & \ttcontext, \mbox{ if } \Gamma(c)
\mbox{ is undefined},\\& &
\tauGamma{T} =   \tttype{},\\& &
      \tau()(\Gamma) = \ttcontext,  \mbox{ and } \\& & 
   \vdash_\Gamma (\exists (x: T) : \tttrue)\\
%
  \tauGamma{\{x : T \vbar a\}} & = & \tttype{}, \mbox{ if } \Gamma(x)
\mbox{ is undefined}, \\& & 
    \tauGamma{T} = \tttype{}, \mbox{ and } \tau(\Gamma, x : \ttvar{}\ T)(a) = \ttbool \\
%
  \tauGamma{f\ a} & = &  B, \mbox{ where } \mu_0(\tauGamma{f}) =
                                                 \funtype{A}{B},\\&  & 
                   			\tauGamma{a} = A',\\ & &
					(A \sim A')_\Gamma,\\& & 
                                        \vdash_\Gamma \pi(A)(a) \\
 \tauGamma{\proj{i}~a} & = & A_i, \mbox{ where } \mu_0(\tauGamma{a}) =
\tupletype{A_1, A_2}
\end{eqnarray*}
\end{Defn}

\begin{Eg}{typechecking subtypes}
Let $\Gamma$ contain the above declarations of $\ttint$, $\ttnat$,
$\mathtt{0}$, 
$\leq$, and {\tt
natinjection}.
\begin{eqnarray*}
\lefteqn{\tauGamma{\{i : \ttint \vbar 0\leq i\}}}\\
  & = & \tttype\\
\lefteqn{\tauGamma{(\lambda (f : \mathtt{ natinjection}) : f(0))(\lambda
(i: \mathtt{nat}): i)}}\\
& = &
\delta(\Gamma)(\ttnat), \mbox{ if}\\& &
( \mathtt{natinjection} \sim \funtype{\ttnat}{\ttnat})_\Gamma,\\& & 
\vdash_\Gamma \forall (j, k : \ttnat): (\lambda (i: \ttnat): i)(j) = (\lambda
(i: \ttnat) : i)(k) \ \supset\  j = k, \\& &
(\ttint \sim \ttnat)_\Gamma, \mbox{ and}\\& & 
\vdash_\Gamma \mathtt{0}\leq \mathtt{0}
\end{eqnarray*}

\end{Eg}

Only one additional clause to Definition~\ref{defn:meaning function} is
needed to capture the 
semantics of predicate subtypes.
\begin{Defn}{meaning function with subtypes}
\begin{eqnarray*}
  \lefteqn{\Mgamma{\{x : T \vbar a\}}}\\
& = & \{ y \in \Mgamma{T} \vbar
\mathcal{M}(\Gamma, x: \ttvar~T\vbar \gamma\{x
\gets y\})(a) = {\oneb} \} 
\end{eqnarray*}
\end{Defn}

\begin{Eg}{semantics of predicate subtypes}
If we assign the usual truth table interpretation to the Boolean function
$\supset$: 
\begin{eqnarray*}
\lefteqn{\Mgamma{\{f : \funtype{\ttbool}{\ttbool} \vbar \forall (x: \ttbool): x\ \supset\ f(x)\}}}\\
& = & \{\{\pair{\zerob, \zerob}, \pair{\oneb, \oneb}\},
\{\pair{\zerob,\oneb}, \pair{\oneb, \oneb}\}\}.
\end{eqnarray*}
\end{Eg}

The following useful propositions are easily proved from the definitions
given above.   Proposition~\ref{max-type-correct} asserts that the
maximal supertype of a type is well typed.  Proposition~\ref{maximal-subset}
asserts that the denotation of a type is a subset of the denotation of its
maximal supertype.  Proposition~\ref{maximal-equality} asserts that
if all the proof obligations in $(A\simeq A')$ are valid relative to
a given assignment $\gamma$ for context $\Gamma$, then
the denotations of $A$ and $A'$ under $\gamma$ are equal. 
%
\begin{prop}\label{max-type-correct}
  If $\tau()(\Gamma) = \ttcontext$ and
  $\tauGamma{A} = \tttype{}$, then $\tauGamma{\mu(A)} = \tttype{}$.
\end{prop}
\begin{prop}\label{maximal-subset}
  If $\tau()(\Gamma) = \ttcontext$, 
    $\tauGamma{A} = \tttype{}$,
   and $\gamma$ satisfies $\Gamma$, 
then
\begin{enumerate}
\item   $\Mgamma{A} \subseteq \Mgamma{\mu(A)}$ and
\item   $\Mgamma{A} \subseteq \Mgamma{\mu_0(A)}$.
\end{enumerate}
\end{prop}
\begin{prop}\label{maximal-equality}
  If $A$ and $A'$ are maximal
types in context $\Gamma$, i.e.,
\begin{enumerate}
\item $\tau()(\Gamma) = \ttcontext$, 
\item $\tauGamma{A} = \tauGamma{A'} = \tttype{}$,
\item $\mu(A) = A$ and $\mu(A') = A'$
\end{enumerate}
and
 for each $a$  in $(A \simeq A')$, %%\memo{What about true/false?}
\begin{enumerate}
\item $a\equiv \tttrue$, or 
\item $a\equiv (a_1 = a_2)$ and $\Mgamma{a_1} = \Mgamma{a_2}$
holds,
\end{enumerate}
then $\Mgamma{A} = \Mgamma{A'}$.\footnote{ 
We remind the reader that the formulas $a$ in $(A\simeq A')$ are equalities,
but we have not yet formally introduced equality into the language.   } 
\end{prop}
\begin{prop}\label{maximal-subtype}
If $\tau()(\Gamma) = \ttcontext$ and $\tauGamma{T} = \tttype$,
then $\Mgamma{T} = \Mgamma{\{x : \mu(T) \vbar \pi(T)(x)\}}$.
\end{prop}

We can now examine the updated forms of the invariants given by
Theorems~\ref{type construction},
\ref{simple-type-semantics},
and~\ref{simple-term-semantics}\@.  
The proof of Theorem~\ref{type construction} remains straightforward.
\begin{comment}
Theorems~\ref{simple-type-semantics}
and~\ref{simple-term-semantics} must be proved with a simultaneous
induction since types now contain terms.
\end{comment}
The statement of
Theorem~\ref{simple-term-semantics} must now
be strengthened to include soundness, that is, if $\vdash_\Gamma a$
and $\gamma$ satisfies $\Gamma$, then 
$\Mgamma{a} = \oneb$\@.  For now, we assume soundness
(Theorem~\ref{proof-soundness})  since we
have not yet presented the proof rules.  
%
\begin{theorem}[type soundness]
\label{subtype-type-semantics}
If $\tau()(\Gamma) = \ttcontext$, $\gamma$ satisfies $\Gamma$, and
$\tauGamma{A} = \tttype$ then $\Mgamma{A}\in U$.
\end{theorem}
\begin{proof}
There is only one new case to add to the induction proof of
Theorem~\ref{simple-type-semantics}, namely, when 
 $A\equiv \{x : T \vbar a\}$.   In this case, by
Definition~\ref{defn:type rules with subtypes},
$\tauGamma{T} = \tttype$, so by the induction hypothesis,
$\Mgamma{T} \in U$.  Since, by
Definition~\ref{defn:meaning function with subtypes}, $\Mgamma{A} \subseteq
\Mgamma{T}$, we have $\Mgamma{A}\in U$ by
Definition~\ref{defn:type universe}\@.
\end{proof}
\begin{theorem}[term soundness]\label{subtype-term-semantics}
If $\tau()(\Gamma) = \ttcontext$, $\gamma$ satisfies $\Gamma$, and $\tauGamma{a} = A$ then $\Mgamma{a} \in
\Mgamma{A}$\@.  
\end{theorem}
\begin{proof}  There are two affected cases in the proof from that of
Theorem~\ref{simple-term-semantics}, namely, those of application
and projection.  The case of projection expressions is straightforward
given Proposition~\ref{maximal-subset}.

When 
 $a \equiv (f\ b)$, by Definition~\ref{defn:type rules with subtypes},
we have that $\tauGamma{f} = \funtype{B}{A}$ and $\tauGamma{b} = B'$.
Let $X$ be $\Mgamma{B}$, $X'$ be $\Mgamma{B'}$,  and $Y$ be $\Mgamma{A}$.
Then by Definition~\ref{defn:meaning function}, $\Mgamma{\funtype{B}{A}} =
Y^X$\@.  By the induction hypotheses, $\Mgamma{f}\in Y^X$ and
$\Mgamma{b}\in X'$\@.  
By Definition~\ref{defn:type rules with subtypes},
soundness of the proof rules (Theorem~\ref{proof-soundness}), and
Propositions~\ref{maximal-subset} and 
~\ref{maximal-equality}, there is a 
maximal supertype $\mu(B)$ of both $B$ and $B'$ such that $X$ and $X'$
are both subsets of $\Mgamma{\mu(B)}$\@.  Since, by
Definition~\ref{defn:type rules with subtypes}, $\vdash_\Gamma{}\pi(B)(b)$,
and by Proposition~\ref{maximal-subtype}, $\Mgamma{B} =
\Mgamma{\{x : \mu(B)\vbar \pi(B)(x)\}}$, we have $\Mgamma{b}\in
\Mgamma{B}$, and hence by Definition~\ref{defn:meaning function}, $\Mgamma{(f\
b)}\in  \Mgamma{A}$\@.  
\end{proof}



\section{Summary}

PVS features a form of subtyping where it is possible to form the subtype
of a type satisfying a given predicate on the type.  This kind of
subtyping introduces several delicate semantic issues into PVS.  A term
can now have several types since, for example, the term corresponding to
the number {\tt 2} can be a prime number, an even number, a natural
number, an integer, a rational number, or a real number.  When the
expected type is a subtype, the canonical type of the actual term must be
compatible with the expected type, that is, the two maximal supertypes
must be equivalent and the actual term must satisfy any subtype
constraints imposed by the expected type.  We have defined the notions of
maximal supertype, subtype constraints, type equivalence, and
compatibility.  These notions are used to define the type rules and
semantics of the simply typed fragment of PVS extended with subtypes.
Note that both type equivalence (and hence, compatibility) and type
correctness are undecidable.  Proof obligations generated during
typechecking are the only source of such undecidability.  The
modularization of the type system into a decidable part consisting of the
simply typed fragment,  and the proof obligations generated by
subtyping, is perhaps the most significant design consideration in the PVS
language.




\chapter{Dependent Types}\label{dependent}

The PVS language fragment described thus far is already quite expressive.
It employs definitional equivalence between types and contains
predicate subtypes.  It is undecidable whether an expression in this
fragment is type-correct because of the proof obligations that arise with
respect to predicate subtypes and type equivalence.  
The next step is the addition of type dependencies between the components
of a type.  This extension considerably
enhances the utility of this type system.  It is also a natural extension
given predicate subtyping which already allows types that
{\em depend\/} on free variables in the predicates.  
With dependent typing, we can
make the type of one component of a product depend on the value of another
component, or the type of the range of a function vary according to its
argument value.

\begin{comment}
Let $V$ be a metavariable that ranges over types or {\em
type bindings\/} of the form $x : T$.
\end{comment}
A dependent product type is
written as $\tupletype{x : A, B}$.  A dependent function type is written
as $\funtype{x : A}{B}$.
Any product or function type can be transformed into a dependent type by
inserting dummy type bindings.  Conversely, any dummy type bindings that
do not actually bind any variable occurrences can be
removed.  The type rules and semantics below will assume that all product and
function types are presented as dependent types.

\begin{Eg}{dependent types}
$$
\begin{array}{l}
\tupletype{i : \ttnat, \{j: \ttnat \vbar j\leq i\}}, \\
\tupletype{i : \ttnat, \funtype{\{j: \ttnat \vbar j\leq i\}}{\ttbool}},\\
\funtype{i: \ttint}{\{j : \ttint \vbar i\leq j\}}.
\end{array}$$
\end{Eg}


Before we treat dependent types, we update the definitions of the set
of free variables and substitution to account for the fact that
with subtyping and dependent typing, both free and bound variables
can occur in terms and types.  This is needed for the next step
where we try to remove type dependencies by substituting a term into
a dependent type.  
\begin{Defn}{free variables for types}
\begin{eqnarray*}
%%  FV(\Gamma)(s) & = & \emptyset, \mbox{ if } \itkind(\Gamma(s)) \neq
%%\ttvariable\\
FV(\Gamma)(\funtype{x : A}{B}) & = & FV(\Gamma)(A)\union
                   (FV(\Gamma, x: \ttvar~A)(B) - \{x\}) \\
%
FV(\Gamma)(\tupletype{x: A, B}) & = &
  FV(\Gamma)(A)\union (FV(\Gamma,
x:\ttvar~A)(B) - \{ x\}) \\
%
FV(\Gamma)(\{x : A \vbar a\}) & = & FV(\Gamma)(A) \union
(FV(\Gamma, x:\ttvar~A)(a) - \{x\})
\end{eqnarray*}
\end{Defn}

\begin{Defn}{substitution for types}
\begin{eqnarray*}
\lefteqn{\funtype{x : A}{B}[a_1/x_1,\ldots, a_n/x_n]}\\ & = & \funtype{y :
A[a_1/x_1,\ldots,a_n/x_n]}{B[y/x, a_1/x_n, \ldots, a_n/x_n]} \\
%
\lefteqn{\tupletype{x: A, B}[a_1/x_1, \ldots, a_n/x_n]}\\ & = &
\tupletype{y: A[a_1/x_1,\ldots, a_n/x_n], 
B[y/x, a_1/x_1,\ldots,a_n/x_n]} \\
%
\lefteqn{\{x : A\vbar a\}[a_1/x_n,\ldots, a_n/x_n]}\\ & = & \{y : A[a_1/x_1,\ldots,
a_n/x_n] \vbar  a[y/x, a_1/x_n, \ldots, a_n/x_n] \}
\end{eqnarray*}
where y is a fresh variable.
\end{Defn}

%\memo{Need to extend substitution to types.}

\begin{comment}
To determine whether a given expression is of a given dependent type, we
must define the notion of a substituting the expression into the type.  If
$\tauGamma{a}$ is a (possibly dependent) product type and $A$ is a
(possibly dependent) product type of the form $\tupletype{x : A_1, A_2}$, then
$a/A$ is the type obtained by ``substituting'' $a$ in $A$: if $a$ is of
the form $(\listwo{a}{n})$ then $a/A$ is $\tupletype{x : A_1,
a_2/(A_2[a_1/x])}$,
else if
$a$ is not of the form $(\listwo{a}{n})$, then we use $\proj{i}\ a$ in place
of $a_i$\@.  Similarly, if $\tauGamma{a}$ is a function type of the form
$\funtype{x : T}{S}$, then if $a$ is of the form $(\lambda (x : T): b)$,
then $(\lambda (x: T): b)/\funtype{y : A}{B}$ is $\funtype{y :
A}{b[y/x]/B}$\@.  If $a$ is not a lambda-expression but $\tauGamma{a}$ is
of the form $\funtype{z:T}{S}$, then $a/\funtype{y : A}{B}$ is just
$(\lambda (x : T): a(x))/\funtype{y : A}{B}$.  In every other case, $a/A$
is just $A$\@.
%%\memo{Need subtype case for substituting.  Put in separate definition.}
\begin{Eg}{substituting into dependent types}
\begin{eqnarray*}
\lefteqn{(5, 3)~/~\tupletype{i : \ttnat, \{j: \ttnat \vbar j\leq i\}}}\\
  & \equiv &
\tupletype{i: \ttnat, \{j: \ttnat \vbar j\leq 5\}}\\
%
\lefteqn{(3, (\lambda (k: \{j : \ttnat\vbar j < 3\}): k))~/~\tupletype{i : \ttnat,
\funtype{k : \{j: \ttnat \vbar j< i\}}{\{j: \ttnat \vbar j\leq k\}}}}\\
&
\equiv &
\tupletype{ i : \ttnat, \funtype{k : \{j: \ttnat \vbar j< 3\}}{\{j: \ttnat \vbar j\leq k\}}} 
\end{eqnarray*}
\end{Eg}
\end{comment}

\newcommand{\constrain}[2]{#1\backslash #2}
The definition of $\mu$ has to be modified slightly for dependent types.
The definition is first extended to type bindings, $\mu(x : T) = x :
\mu(T)$.  The definition for the case of dependent function types
is unchanged so that $\mu(\funtype{x : A}{B}) = \funtype{x :
A}{\mu(B)}$\@.   The definition for the product case is more delicate
since the definition $\mu(\tupletype{x : A, B}) = \tupletype{x : \mu(A),
\mu(B)}$ results in a loss of type information regarding the occurrences
of $x$ in $B$\@.\footnote{Doug Howe brought this problem to our attention.}
To ensure that type information regarding $x$ is
retained, we define a new operation $\constrain{T}{a}$
which constrains the subtype assertions in  type $T$ with an additional
assertion $a$\@.
\begin{Defn}{Adding subtype constraints}
\begin{eqnarray*}
\constrain{s}{a} & = & s \\
\constrain{\subtype{x}{T}{b}}{a} & = & \subtype{x}{T}{a \wedge b}\\
\constrain{\funtype{A}{B}}{a} & = &
\funtype{\constrain{A}{a}}{\constrain{B}{a}} \\
\constrain{\tupletype{A, B}}{a} & = & \tupletype{\constrain{A}{a}, \constrain{B}{a}}
\end{eqnarray*}
\end{Defn}

We can now define the maximal supertype operation for
dependent tuple types.
\begin{Defn}{Maximal supertype for dependent product types}
\begin{eqnarray*}
\mu(\tupletype{x : A, B}) & = & \tupletype{x : \mu(A), \constrain{B}{\pi(A)(x)}}
\end{eqnarray*}
\end{Defn}

The definition of $\pi$ for a dependent function type $\funtype{ y :
A}{B}$ is slightly different from that of an ordinary function type
since $\pi(B)$ can contain free occurrences of the variable $y$\@.
For example, $\pi(\funtype{i : \ttint}{\{j : \ttint \vbar i\leq j\}})$
must be  $\lambda (f : \funtype{i : \ttint}{\ttint}): (\forall (i:\ttint): i
\leq f(i))$\@.  The definition for dependent tuples remains essentially
unchanged from that of ordinary products.  
\begin{Defn}{constraint predicates for dependent types}
\begin{eqnarray*}
   \pi(\funtype{y:A}{B}) & = &  ( \lambda (x : \funtype{y:A}{\mu(B)}):
    (\forall (y: A): \pi(B)(x(y)))) \\
%
  \pi(\tupletype{y:A, B}) & = &
   ( \lambda (x: \tupletype{y : \mu(A), \constrain{\mu(B)}{\pi(A)(y)}}): \\&&\hspace*{.3in}
	\pi(A)(\proj{1}~x)\wedge \pi(B)(\proj{2}~x)[(\proj{1}~x)/y])
\end{eqnarray*}
\end{Defn}

\newcommand{\depfunexample}{\funtype{i : \ttint}{\{j : \ttint \vbar i\leq
j\}}}
\newcommand{\deptupexample}{\tupletype{i : \ttnat, \{j: nat \vbar j\leq i\}}}
\begin{Eg}{dependent type predicates}
\begin{eqnarray*}
\mu(\depfunexample{}) & = & \funtype{i : \ttint}{\ttint}\\
\pi(\depfunexample{}) & = & \lambda (f : \funtype{i : \ttint}{\ttint}):\\&&\hspace{.3in}
\forall (i: \ttint): (\lambda (j : \ttint): i\leq j)(f(i))
\end{eqnarray*}
\end{Eg}

The definition of $\simeq$ must also be massaged slightly for dependent
types.  Recall that $\simeq$ checks whether two maximal types are
equivalent by generating proof obligations as needed.  This is the
basic operation for checking whether the expected type of an expression
is compatible with its actual type.  
The subtlety now is that the expected type might be
a dependent type where the actual type is not.  Consider the
case of the pair $\pair{5, (\lambda (x : \{j:\ttnat \vbar j\leq 5\}): x)}$
whose type would be computed 
by $\tau$ as $\tupletype{i : \ttnat, \funtype{\{j : \ttnat\vbar j\leq 5\}}{\{j : \ttnat\vbar j\leq 5\}}}$
where the expected type might be 
$\tupletype{i : \ttnat, \funtype{\{j : \ttnat \vbar j\leq i\}}{\{j : \ttnat \vbar
j\leq i\}}}$\@.  To cope with this, 
we will allow the option of two maximal types, say $A$ and $B$, to be
compared using $\simeq$
in the context of an expression $a$\@.  This is indicated by the notation $(A\simeq B)/a$\@.  
Note that $(A\simeq B)/a$ is sensible only when $A$ and $B$
are maximal types.   The missing cases in
Definition~\ref{defn:type equivalence proof obligations}
are included in
Definition~\ref{defn:type equivalence for dependent types}.    For a list
of formulas $a_1, \ldots, a_n$, let 
$(\forall (x : T): a_1,\ldots, a_n)$ represent the list
$(\forall (x : T): a_1), \ldots, (\forall (x : T): a_n)$\@.\footnote{Note that the type-correctness of the proof
obligation $(\pi(A) = \pi(A'))$ in
Definition~\ref{defn:type equivalence for dependent types}  depends on the
\emph{prior} proof obligations 
$\mu(A)\simeq \mu(A')$.}
\begin{Defn}{type equivalence for dependent types}
\begin{eqnarray*}
  (s \simeq s)/a & = & \tttrue \\
(\funtype{x:A}{B} \simeq \funtype{x':A'}{B'}) & = & (\mu(A)\simeq
\mu(A'));\\ & & 
(\pi(A) = \pi(A'));\\ & & 
 (\forall (x : A): (B \simeq B'[x/x'])) \\
%
(\funtype{x:A}{B} \simeq \funtype{x':A'}{B'})/a & = & (\mu(A)\simeq
\mu(A'));\\ & & 
(\pi(A) = \pi(A'));\\ & & 
 (\forall (x : A): (B \simeq B'[x/x'])/a(x)) \\
%
(\tupletype{x : A_1, A_2} \simeq \tupletype{y :
B_1, B_2})
& = &
    (A_1\simeq B_1);\\ & & (\forall (x : A_1) : (A_2\simeq
B_2[x/y]))\\
%
(\tupletype{x : A_1, A_2} \simeq \tupletype{y :
B_1, B_2})/a
& = &
    (A_1\simeq B_1)/(\proj{1}~a);\\ & & (A_2[(\proj{1}~a)/x]\simeq
B_2[(\proj{1}~a)/y])/(\proj{2}~a)\\
%
 (A \simeq B)/a & = & \ttfalse, \mbox{ otherwise.}
\end{eqnarray*}
\end{Defn}
As with $(A\sim B)_\Gamma$, the notation $(A \sima 
B)_\Gamma$ indicates that  all the proof obligations $a'$ in
$(\mu(A)\simeq \mu(B))/a$ are provable, that is,  $\vdash_\Gamma a'$\@.  

With dependent types, the type rules must be modified so as to augment the
context suitably to account for any dependencies.  
We will give the definitions only for dependent type constructions.

%\memo{need to handle type dependencies in expected for application case.}
\begin{Defn}{type rules with dependent types}
\begin{eqnarray*}
\tauGamma{\tupletype{x : A, B}} & = &
\tttype{}, \mbox{ if } \Gamma(x) \mbox{ is undefined},\\
& & \tauGamma{A} = \tttype{}, \mbox{ and}\\
& & \tau(\Gamma, x : \ttvar{}\ A)(B) = \tttype{} \\
%
%\tauGamma{\tupletype{T_1, U_2, \ldots, U_n}} & = &
%\tttype{}, \mbox{ if } \tauGamma{T_1} = \tttype{}, \mbox{ and}\\
%& & \tauGamma{\tupletype{U_2,\ldots, U_n}} = \tttype{} \\
%
\tauGamma{\funtype{x: A}{B}} & = & \tttype{},
    \mbox{ if } \Gamma(x) \mbox{ is undefined},\\
& & \tauGamma{A} = \tttype{}, \mbox{ and}\\
& & \tau(\Gamma, x : \ttvar{}\ A)(B) = \tttype{}\\
%
  \tauGamma{f\ a} & = &  B', \mbox{ where } \mu_0(\tauGamma{f}) = \funtype{x :
A}{B},\\ 
 & &                   			\tauGamma{a} = A', \\ & & 
					(A \sima  A')_\Gamma, \\
& &
                                        B' \mbox{ is } B[a/x],\\ & & 
                                        \vdash_\Gamma \pi(A)(a)\\
%
 \tau(\Gamma)(\lambda (x : A): a) & = &
         \funtype{x : A}{B}, \mbox{ where}\\
%& & A  = \tupletype{x_1 : T_1, \ldots, x_n : T_n}, \\
& & B = \tau(\Gamma, x : \ttvar{}\ A)(a)\\ 
%B' = B[(\proj{1}\ x)/x_1, \ldots, (\proj{n}\
%x)/x_n] \\ & & 
%\mbox{if } \mbox{ for } 1\leq i\in \{1, 2\},\\& & 
%\tau(\Gamma, x_1 : \ttvar{}\ T_1,\ldots, x_{i-1} : \ttvar{}\ T_{i-1})(T_i) = \tttype{}, \\
%& & \mbox{and for } 1\leq i, j \in \{1, 2\}, i\neq j, x_i \not\equiv x_j \\
%
\tauGamma{\proj{1}~a} & = & A_1, \mbox{ where } \mu_0(\tauGamma{a}) =
\tupletype{x: A_1, A_2} \\
%
\tauGamma{\proj{2}~a} & = & A_2[(\proj{1}~a)/x], \mbox{ where } \mu_0(\tauGamma{a}) =
\tupletype{x: A_1, A_2} \\
\end{eqnarray*}
\end{Defn}
%%\memo{Need proj case above.}

\begin{Eg}{dependent typing}
\begin{eqnarray*}
\tauGamma{\tupletype{x : \ttbool,  \{ y : \ttbool \vbar x \supset
y\}}}
& = & \tttype\\
\tauGamma{\funtype{x : \ttbool}{\{ y : \ttbool \vbar x \supset 
y\}}} & = & \tttype
\end{eqnarray*}
\end{Eg}
Before we can assign meanings to dependent types, we must
augment our definition of the universe $U$ to contain sets corresponding
to these constructions.  If $F$ is a function with domain set $X$ and
a range $Y$, which is a set of sets, we can  define $\Sigma{F}$ to be the set 
$\{ \pair{x, y} \vbar x\in \textit{dom}(F), y\in F(x)\}$ and
$\Pi{F}$ to be the set $\{f \vbar (\forall x\in \textit{dom}(F): f(x)\in
F(x))\}$\@.   Note that $\Pi{F} \subseteq \bigcup_{X\in \Sigma{F}}\wp(X)$ but we
include $\Pi{F}$ in the universe $U$ defined below for simplicity\@.
We can drop $X\times Y$ and $X^Y$ from the universe definition since
$X\times Y$ can be obtained from $\Sigma{F}$ by defining an
$F$ with domain $X$ that always returns $Y$, and similarly,
$X^Y$ can be obtained by $\Pi{F}$ where $F$ is defined to with domain $Y$
to always return $X$\@.  The universe $U$ can then be redefined as below.  
\begin{Defn}{type universe with dependent types}
\begin{eqnarray*}
  U_0 & = & \{ {\twob}, {\reals}  \} \\
  U_{i+1} & = & U_i  \\& &
               %\union \{X\times Y \vbar X, Y \in U_i\} \\ & & 
                         %\union  \{X^Y \vbar X, Y \in U_i \} \\ & & 
                          \union  \bigcup_{X\in U_i}\wp(X)\\& & 
                           \union\; \{\Sigma{F} \vbar F \in W_i\}\\& &
                            \union\; \{\Pi{F} \vbar F \in W_i\}\\
  W_i & = & \bigcup_{X\in U_i} U_i^X \\
  U_{\omega} & = & \bigcup_{i\in\omega} U_i\\
  U & = & U_{\omega}
\end{eqnarray*}
\end{Defn}

One very important consequence of the above extension of the universe
is that all type dependencies must be bounded in the sense that
if $B$ is a type expression with a single free variable $x$ of type $A$,
then it must be the case that for any set $\mean{A}$ representing $A$,
there is a bound $n$ such that for any $z$ in $\mean{A}$, the meaning of 
$B$ under $\{x \gets z\}$ must be in $U_n$\@.  This property is easily
proved by induction on the structure of a PVS type since the parameter
$x$ can appear only in the predicate part of a subtype where the
rank of the meaning of the resulting type cannot vary with the value
of $x$\@.   In
particular, there is no way to define a type constructor $T^n$ in PVS that
returns the $n$-tuple $[\underbrace{T, [\ldots, T]}_n]$ for a given $n$
since this would entail an unbounded dependency.  
If unbounded type dependencies were allowed in PVS,
one can construct a dependent type such as $\funtype{n : nat}{T^n}$
whose representation is not in $U$ as defined above.    
  

\begin{comment}
A simpler alternative might be to define a second universe
U2 which starts from U1 and adds the dependent types.  This will work fine
in the absence of parametric theories since each type contains only a
fixed number of type dependencies.  The introduction of parametric
theories might require a third iteration since they introduce a third
form of dependency.
\end{comment}

\begin{comment}
Seeking advice on PVS semantics:  I'd like to know if the following
universe construction provides a satisfactory model for PVS types.  The
PVS simple type system consists of the base types bool and real and the
tuple [A_1,...,A_N] and function [A->B] types.  The flg. universe U1
(where (U_n)^n is the n-fold Cartesian product and (U_n)^(U_n) is the
function space) provides a model: 
 U_0     = {2, R}
 U_{n+1} = U_n UNION (U_n)^n UNION (U_n)^(U_n)
 U_{omega} = SUP_n U_n
 U1      = U_omega
%
The next step is to add subtypes {x:A \vbar p(x)}.  The above universe can be
easily modified to accommodate these by adding the subsets of the finite
simple types. Here P(X) is the power-set of X.  
%
 U_{n+1} = U_n UNION {U_n}^n UNION (U_n)^(U_n) UNION P(U_n)
%
The type system is then extended with dependent function [x:A->B(x)]
and tuple types [x_1: A, B].  The simple way to construct a universe is to
say that there are only finite length chains of dependencies and these are
captured by the universe U2 below which is constructed iteratively from
U1.  Here, if F is a function graph, #F is the set
{<x, y> \vbar x in domain(F), y in F(x)}. 
%
  U2_0 = U1
  U2_{n+1} = U_n UNION (U_n)^n UNION (U_n)^(U_n) UNION P(U_n)
             UNION {#F \vbar F in {(U2_n)^X \vbar X in U2_n}}
  U2_{omega} = SUP_n U2_n
  U2 = U2_{omega}
%
The reason this works is that it is easy to see by induction on the
type structure that we can assign a degree k to a given type,
where the degree of a nondependent type is 0, and the degree of
a type of the form [x: S, T(x)] is max(deg(S), deg(T)) + 1.
Note that if a type has degree k, then it is introduced in the
kth iteration of the definition of U2.  Dependent function types
are ignored here since they are defined as subsets of the corresponding
tuples so that they are introduced in the next iteration by the power-set
construction.
%
There really is no need for a second universe construction here since the
above argument would also work if the U2 iteration was rolled into that of
U1.
%
Parametric theories pose a serious difficulty since we now have types and
constants that can depend on types.  The context provides a
meaning function to each theory that maps the meanings of the
actual parameters to the meanings of the symbols declared in the theory.
We need to constrain the meaning function for a theory so that
the degree of each type name in the theory depends only on the
degrees of the type parameters.  This is a basic invariant in PVS that
the dependency between types and individuals only affects the
predicate part of the type and not its structure.  One cannot, for
instance, define a tuple whose first component is n:nat, and whose
second component is the n-fold tuple T^n.   Note that this
invariant is preserved by the way in which theories are actually used.
An interpretation of a theory simply attaches a definition to an
undefined type and the meaning of this definition will also have the
boundedness property.  
\end{comment}









The meaning function for dependent types is obtained by adding the
cases corresponding to 
dependent product and function types.  All the other cases are unchanged from
Definition~\ref{defn:meaning function with subtypes}\@.  Note that the
semantic definition  for dependent types is 
equivalent to the nondependent one when there are no dependencies.
\begin{Defn}{meaning function with dependent types}
\begin{eqnarray*}
  \Mgamma{\tupletype{x : A, B}} & = &
    \Sigma{F}, \mbox{ where }\\& &
     F \mbox{ maps } z \in \Mgamma{A} \mbox{ to }\\& & \mathcal{M}(\Gamma, x : \ttvar{}\
A\vbar \gamma \{x\gets z \})(B)\\
%
  \Mgamma{\funtype{x:A}{B}} & = &  \Pi{F}, \mbox{ where }\\& &
                        F \mbox{ maps } z \in \Mgamma{A} \mbox{ to }\\& &  \mathcal{M}(\Gamma,
x : \ttvar{}\  A\vbar \gamma\{x\gets z\})(B)  
%    
\end{eqnarray*}
\end{Defn}

\begin{Eg}{meaning function with dependent types}
\begin{eqnarray*}
\Mgamma{\tupletype{x : \ttbool,  \{ y : \ttbool \vbar x \supset 
y\}}}
& = & \{\pair{\zerob, \zerob}, \pair{\zerob, \oneb},
\pair{\oneb, \oneb}\}\\ 
\Mgamma{\funtype{x : \ttbool}{\{ y : \ttbool \vbar x \supset 
y\}}} & = & \{\{\pair{\zerob, \zerob}, \pair{\oneb, \oneb}\},\\&&\hspace*{1ex}
              \{\pair{\zerob, \oneb}, \pair{\oneb, \oneb}\}\}
\end{eqnarray*}
\end{Eg}

We now need to show that the extensions corresponding to dependent types
preserve the properties in Theorems~\ref{subtype-type-semantics}
and~\ref{subtype-term-semantics}, namely, 
$\Mgamma{T} \in U$ and $\Mgamma{a}\in \Mgamma{\tauGamma{a}}$\@.
For the former, we prove a stronger theorem that incorporates the
rank-boundedness of dependent types.
\begin{theorem}[rank bounded type semantics]\label{dependent-type-semantics}
If $B$ is a pretype, $x_1,\ldots, x_n$ is a list of symbols,
$A_1,\ldots, A_n$ is a list of pretypes such that  
\begin{enumerate}
\item $\tau()(\Gamma, x_1:\ttvar~A_1, \ldots, x_n:\ttvar~A_n) = \ttcontext$,
\item $\tau(\Gamma, x_1:\ttvar~A_1, \ldots, x_n:\ttvar~A_n)(B) = \tttype$,
and
\item $\gamma$ is an assignment satisfying $\Gamma$,
\end{enumerate}
then
there is an $i$ such that for any list of values
$z_1,\ldots, z_n$ where $\gamma\{x_1\gets
z_1\}\ldots\{x_n\gets z_n\}$ is a satisfying assignment for
$\Gamma, x_1:\ttvar~A_1, \ldots, x_n:\ttvar~A_n$,  we have
$$\mathcal{M}(\Gamma, x_1:\ttvar~A_1, \ldots, x_n:\ttvar~A_n\vbar
\gamma\{x_1\gets z_1\}\ldots\{x_n\gets z_n\})(B) \in U_i.$$
\end{theorem}
\begin{proof}
The proof is by structural induction on the pretype $B$\@.
Let $\Gamma'$ denote $\Gamma, x_1:\ttvar~A_1, \ldots, x_n:\ttvar~A_n$,
$\gamma'$ denote $\gamma\{x_1\gets z_1\}\ldots\{x_n\gets z_n\}$, and 
$\mean{C}$ denote $\mathcal{M}(\Gamma'\vbar
\gamma')(C)$.  
\begin{enumerate}
\item $B\equiv s$: Since $\mean{B}$ is just $\gamma(B)$ by
Definition~\ref{defn:meaning function}, we have that there is an $i$
such that $\mean{B}\in U_i$ regardless of the choice of values
$z_1,\ldots, z_n$\@.  

\item $B\equiv \{y:T\vbar a\}$: By the induction hypothesis, we know that
for some $j$, it is always the case that $\mean{T}\in U_j$\@.  By
Definition~\ref{defn:meaning function with subtypes}, we have that
$\mean{B}\subseteq \mean{T}$ so if we let $i = j+1$, then
by Definition~\ref{defn:type universe with dependent types},
it is always the case that $\mean{B}\in U_i$\@.

\item $B \equiv \funtype{y:C}{D}$: By
Definition~\ref{defn:type rules with dependent types},
$\Gamma'(y)$ is undefined, $\tau(\Gamma')(C) = \tttype$,
$\tau()(\Gamma', y:\ttvar~C) = \ttcontext$, and
$\tau(\Gamma', y:\ttvar~C)(D) = \tttype$.
By the induction hypothesis, for some $j$,
it is always the case that $\mathcal{M}(\Gamma'\vbar\gamma')(C)\in U_j$,
and for some $k$, it is always the case that for any satisfying
assignment $\gamma'\{y\gets w\}$ for $\Gamma', y:\ttvar~C$,
we have $\mathcal{M}(\Gamma', y:\ttvar~C\vbar \gamma'\{y\gets
w\})(D)\in U_k$\@.   Then the function $F$ mapping $w$ in
$\mathcal{M}(\Gamma')(C)$ to $\mathcal{M}(\Gamma', y:\ttvar~C\vbar
\gamma'\{y\gets w\})(D)$ is an element of $W_{j+k}$\@.
Letting $i$ be $j+k+1$, we have by
Definition~\ref{defn:meaning function with dependent types}
that $\mathcal{M}(\Gamma'\vbar\gamma')(B)$ is
$\Pi{F}$ and is hence an element of $U_i$ by
Definition~\ref{defn:type universe with dependent types}\@.

\item $B\equiv \tupletype{y: C, D}$:  Similar to the
previous case.  
\end{enumerate}
\end{proof}

By choosing $n$ to be $0$, the previous theorem yields the result 
that when $\tauGamma{B} = \tttype$, $\Mgamma{B}\in U$\@.

We next need to establish that for any preterm $a$,
if $\tauGamma{a} = A$, then $\Mgamma{a}\in \Mgamma{A}$\@.
The first step in this direction is the proof of the substitution lemma
below.
\begin{prop}\label{context-equivalence}
If $\tau()(\Gamma) = \tau()(\Gamma') = \ttcontext$ where
for each $s$, $\Gamma(s)$ is defined if and only if  $\Gamma'(s)$ is defined, and 
$\gamma$ is an assignment satisfying both $\Gamma$ and $\Gamma'$,
then
\begin{enumerate}
\item If  $\Gamma(s) = \ \Gamma'(s)$ (i.e., they are equal when
either $\Gamma(s)$ or $\Gamma'(s)$ is defined), then 
\begin{enumerate}
\item $\tau(\Gamma)(a) = \tau(\Gamma')(a)$, for any preterm $a$.
\item $\tau(\Gamma)(A) = \tau(\Gamma')(A)$, for any pretype $A$.
\end{enumerate}

\item $\Mgamma{A} = \mathcal{M}(\Gamma'\vbar\gamma)(A)$, when $\tauGamma{A} =
\tttype$.

\item $\Mgamma{a} = \mathcal{M}(\Gamma'\vbar\gamma)(a)$, for any preterm $a$
such that $\tauGamma{a}$ is defined.
\end{enumerate}
\end{prop}
%
%%\memo{Needs induction on preterms and pretypes.}
\begin{lemma}[substitution lemma]\label{substitution-lemma}
If $\tau()(\Gamma, x:\ttvar~A) = \ttcontext$,
$\tauGamma{a} = A$, then
\begin{enumerate}
\item If $\tau(\Gamma, x: \ttvar~A)(b) = B$, then\\
$\Mgamma{b[a/x]} = \mathcal{M}(\Gamma, x:\ttvar~A\vbar \gamma\{x\gets
\Mgamma{a}\})(b)$.

\item If $\tau(\Gamma, x: \ttvar~A)(C) = \tttype$, then\\
$\Mgamma{C[a/x]} = \mathcal{M}(\Gamma, x:\ttvar~A\vbar \gamma\{x\gets
\Mgamma{a}\})(C)$\@. 
\end{enumerate}
\end{lemma}
\begin{proof}
The proof is by simultaneous structural induction on the preterm $b$ and
the pretype $C$\@.  The following cases deal with the preterm $b$.
\begin{enumerate}
\item $b\equiv s$: If $s\equiv x$, then by
Definition~\ref{defn:meaning function with dependent types}, the left-hand
side $\Mgamma{b[a/x]}$ is 
$\Mgamma{a}$, and the right-hand side  $\mathcal{M}(\Gamma, x:\ttvar~A\vbar \gamma\{x\gets
\Mgamma{a}\})(b)$ is also $\Mgamma{a}$\@. 

If $s\not\equiv x$, then by
Definition~\ref{defn:meaning function with dependent types},
the left-hand side and the right-hand side
are both equal to $\gamma(s)$\@. 

\item $b \equiv (\lambda (y : C): d)$: Since $C$ can contain free
occurrences of $x$, we have by the induction hypothesis that
$\Mgamma{C[a/x]} = \mathcal{M}(\Gamma, x:\ttvar~A\vbar \gamma\{x\gets
\Mgamma{a}\})(C)$\@.  Also, $\Mgamma{(\lambda (y:C): d)[a/x]}$ is equal to
the set of ordered pairs $\pair{v, z}$ such that $v\in \Mgamma{C[a/x]}$
and $z = \mathcal{M}(\Gamma, y:\ttvar~C[a/x]\vbar \gamma\{y\gets
v\})(d[a/x])$\@.

By the induction hypothesis,
$\mathcal{M}(\Gamma, y:\ttvar~C[a/x]\vbar \gamma\{y\gets v\})(d[a/x])
= \mathcal{M}(\Gamma, y:\ttvar~C[a/x], x:\ttvar~A\vbar \gamma\{y\gets
v\}\{x\gets \Mgamma{a}\})(d)$\@.  Since $x$ does not occur free in
$C[a/x]$, by Proposition~\ref{context-equivalence} we can exchange the
occurrences of $y$ and $x$ so that 
$\mathcal{M}(\Gamma, y:\ttvar~C[a/x], x:\ttvar~A\vbar \gamma\{y\gets
v\}\{x\gets \Mgamma{a}\})(d) =
\mathcal{M}(\Gamma, x:\ttvar~A, y:\ttvar~C[a/x]\vbar \gamma\{x\gets \Mgamma{a}\}\{y\gets
v\})(d)$.

By Definition~\ref{defn:meaning function with dependent types},
the right-hand side is the set of ordered pairs of the form $\pair{v, z}$
such that $v\in \mathcal{M}(\Gamma, x: \ttvar~A \vbar \gamma\{x\gets
\Mgamma{a}\})(C)$
and $z = \mathcal{M}(\Gamma, x: \ttvar~A, y: \ttvar~C\vbar \gamma\{x\gets
\Mgamma{a}\}\{y\gets v\})(d)$\@.  By Proposition~\ref{context-equivalence}
and the induction hypothesis, we know that
$\mathcal{M}(\Gamma, x: \ttvar~A, y: \ttvar~C\vbar \gamma\{x\gets
\Mgamma{a}\}\{y\gets v\})(d) =
\mathcal{M}(\Gamma, x:\ttvar~A, y:\ttvar~C[a/x]\vbar \gamma\{x\gets \Mgamma{a}\}\{y\gets
v\})(d)$, and hence it follows that the two sets of ordered pairs are
equal.

\item $b\equiv (f\ c)$:  In this case, $b[a/x]\equiv (f[a/x]~c[a/x])$
and the conclusion follows easily from the induction hypothesis
and Definition~\ref{defn:meaning function with dependent types}\@.

\item $b\equiv (b_1, b_2)$: 
The conclusion follows easily from 
Definitions~\ref{defn:substitution},~\ref{defn:meaning function with
dependent types}, and the induction hypotheses.


\item $b\equiv (\proj{i}~c)$: This case is also straightforward since
$b[a/x] \equiv (\proj{i}~c[a/x])$,  
and by the induction hypothesis,
$\mathcal{M}(\Gamma, x: \ttvar~a \vbar \gamma\{x \gets \Mgamma{a}\})(c) =
\Mgamma{c[a/x]}$.  
%
\end{enumerate}

The remaining cases deal with the pretype $C$\@.
\begin{enumerate}
\item $C\equiv s$:  This case is trivial since by
Definition~\ref{defn:substitution}, $C[a/x]\equiv C$ and
the left-hand and right-hand sides both reduce to $\gamma(C)$\@.

\item $C\equiv \{y : T \vbar d\}$: The argument here follows along the
lines of the $b\equiv (\lambda (x : C): D)$ case above.
By the induction hypotheses, we know that
\begin{eqnarray*}
\lefteqn{\mathcal{M}(\Gamma, x : \ttvar~A\vbar \gamma\{x\gets
\Mgamma{a}\})(T)}\\
& = & \Mgamma{T[a/x]}\\
\lefteqn{\mathcal{M}(\Gamma, y : \ttvar~T[a/x], x : \ttvar~A\vbar \gamma\{y\gets
z\}\{x\gets \Mgamma{a}\})(d)}\\
& = & \mathcal{M}(\Gamma, y : \ttvar~T[a/x]\vbar \gamma\{y\gets z\})(d[a/x]),\\&&
\mbox{ for any } z\in \Mgamma{T[a/x]}
\end{eqnarray*}
The conclusion follows from Proposition~\ref{context-equivalence}
and Definition~\ref{defn:meaning function with dependent types}.

\item $C\equiv \funtype{y:C_1}{C_2}$: The argument here is similar
to that of the previous case.  Essentially, by the induction hypothesis
and Proposition~\ref{context-equivalence}, the function mapping
$z\in \mathcal{M}(\Gamma, x : \ttvar~A\vbar \gamma\{x\gets \Mgamma{a})(C_1)$
to $\mathcal{M}(\Gamma, y:\ttvar~C_1[a/x], x : \ttvar~A\vbar \gamma\{y\gets
z\}\{x\gets \Mgamma{a}\})(C_2)$ is the same as the function mapping
$z\in \Mgamma{C_1[a/x]}$ to $\mathcal{M}(\Gamma, y: \ttvar~C_1[a/x]\vbar
\gamma\{y\gets z\})(C_2[a/x])$\@.  

\item $C\equiv \tupletype{y:C_1, C_2}$:   Similar to the previous case.
\end{enumerate}
\end{proof}

Proposition~\ref{substitution-equivalence} is stated below without proof.
It asserts the semantic equivalence with respect to term $a$
of types $A$ and $B$ when $(A\sima B)_\Gamma$ holds.   Note that its
correctness depends on the soundness of the proof rules.  

\begin{prop}\label{substitution-equivalence}
If $\tau()(\Gamma) = \ttcontext$,  $a$ is a preterm such that
$\tauGamma{a} = B$, and $(A\sima  B)_\Gamma$,
then $\Mgamma{a}\in \Mgamma{A}$ iff $\Mgamma{a}\in \Mgamma{B}$\@.  
\begin{comment}
\begin{enumerate}
\item $\Mgamma{a}\in \Mgamma{A}$ iff $\Mgamma{a}\in \Mgamma{a/A}$
\item $\vdash \pi(a/A)(a)$ iff $\vdash \pi(A)(a)$\@.
\end{enumerate}
\end{comment}
\end{prop}
%
\begin{theorem}\label{dependent-term-semantics}
If $\tau()(\Gamma) = \ttcontext$, $\gamma$ is an assignment satisfying
$\Gamma$, and $a$ is a preterm such that $\tauGamma{a} = A$, then
$\Mgamma{a}\in \Mgamma{A}$\@.   
\end{theorem}
\begin{proof}
The proof is by  induction on the structure of the preterm
$a$\@.
\begin{enumerate}
\item $a\equiv s$: Then  by
Definition~\ref{defn:meaning function with dependent types},
$\Mgamma{a} = \gamma(a)$, and by Definition~\ref{defn:satisfaction},
we have that $\gamma(a)\in \Mgamma{A}$\@.

\item $a \equiv (\lambda (x : C): b)$:  By
Definition~\ref{defn:type rules with dependent types}, we
have $\tauGamma{a} = A = \funtype{x : C}{\tau(\Gamma, x: \ttvar~C)(b)}$\@.
Let $B$ label $\tau(\Gamma, x: \ttvar~C)(b)$\@.  
We know that $\Mgamma{A}$ is of the form $\Pi{} F$ where
$F$ maps $z\in \Mgamma{C}$ to $\mathcal{M}(\Gamma, x : \ttvar~C\vbar \gamma\{x
\gets z\})(B)$\@.

By the induction hypothesis on $b$, we know that for any $z\in \Mgamma{C}$, 
$\mathcal{M}(\Gamma, x : \ttvar~C\vbar \gamma\{x \gets z\})(b)\in
\mathcal{M}(\Gamma, x : \ttvar~C\vbar \gamma\{x
\gets z\})(B)$\@.  
Since by Definition~\ref{defn:meaning function with dependent types},
$\Mgamma{a}$ is a function mapping $z\in \Mgamma{C}$ to
$\mathcal{M}(\Gamma, x : \ttvar~C\vbar \gamma\{x \gets z\})(b)$, we have
$\Mgamma{a} \in \Pi F$ by the
definition of $\Pi$\@.  

\item $a \equiv (f~b)$: By
Definition~\ref{defn:type rules with dependent types}, we have that
$\tauGamma{f} = \funtype{x : B}{A'}$, $\tauGamma{b} = B'$,
$(B \sima  B')_\Gamma$,  $A \equiv A'[a/x]$, and
$\vdash_\Gamma\pi(B)(b)$\@.  We know by the induction hypothesis
that $\Mgamma{f}\in \Mgamma{\funtype{x : B}{A'}}$ and
$\Mgamma{b} \in \Mgamma{B'}$\@.  By
Propositions~\ref{substitution-equivalence}
and~\ref{maximal-subtype},
$\Mgamma{b}\in \Mgamma{\mu(B)}$\@.  
We therefore have 
by Proposition~\ref{maximal-subtype} that $\Mgamma{b}\in \Mgamma{B}$\@.
By Definition~\ref{defn:meaning function with dependent types},
$\Mgamma{a}\in \mathcal{M}(\Gamma, x: \ttvar~B\vbar \gamma\{x\gets
\Mgamma{b}\})(A')$,
and hence  by Lemma~\ref{substitution-lemma} it follows that
$\Mgamma{a}\in \Mgamma{A'[b/x]}$\@.  

\item $a \equiv (\listwo{a}{n})$:  The conclusion follows easily from
the induction hypothesis and
Definition~\ref{defn:type rules with dependent types}\@.

\item $a\equiv (\proj{i}~b)$: The conclusion follows easily from
Proposition~\ref{maximal-subtype}, the  induction hypothesis,
and Definition~\ref{defn:type rules with dependent types}\@.
The $(\proj{2}~b)$ case also employs Lemma~\ref{substitution-lemma}\@.  
\end{enumerate}
\end{proof}

\begin{comment}
\begin{lemma}\label{rank-boundedness}
If $\tau()(\Gamma) = \ttcontext$,
$\gamma$ is a satisfying assignment for $\Gamma$, 
$\tauGamma{A} = \tttype$,
$\Mgamma{A} \in U$, 
and $\tau(\Gamma, x: \ttvar~A)(B) = \tttype$, then
there is an $i$ such that the function $F$ mapping
$z$ in $\Mgamma{A}$ to 
$\mathcal{M}(\Gamma, x : \ttvar~A\vbar \gamma\{x\gets z\})(B)$
is in $W_i$\@.  
\end{lemma}
\begin{proof}
The proof is by induction on the structure of the pretype $B$\@.
Let the rank of $\Mgamma{A}$ be $j$\@.  
\begin{enumerate}
\item $B\equiv s$: Then $\mathcal{M}(\Gamma, x : \ttvar~A\vbar \gamma\{x\gets
z\})(B)$ is just $\gamma(s)$ and by Definition~ref{defn:satisfaction},
we have some $k$ such that $\gamma(s)\in U_k$\@.  Let $i$ be the maximum
of $j$ and $k$, we have $F\in U_i$\@.
%
\item $B\equiv \funtype{y:C}{D}$:  We have the induction hypothesis
%
\end{enumerate}
\end{proof}
%
\begin{theorem}\label{dependent-type-semantics}
If $\tau()(\Gamma) = \ttcontext$,  $\gamma$ satisfies $\Gamma$,
and $\tauGamma{A} = \tttype$, 
then 
\begin{enumerate}
%\item  $\Mgamma{A}\in U$, and
%%\item if $\tauGamma{b} = B$ then $\Mgamma{b} \in \Mgamma{B}$, and
\item if for all $n$, for $i$,  $1\leq i\leq n$,  $\mathcal{M}(\gamma,
x_1:\ttvar~A_1, \ldots x_{i-1}:\ttvar~A_{i-1}\vbar {A_i})\in U$ and 
$\tau(\Gamma, x_1 : \ttvar~A_1, \ldots, x_n:\ttvar~A_n)(B) = \tttype$,
then $\mathcal{M}(\Gamma, x_1 : \ttvar~A_1\ldots\vbar \gamma\{x\gets z\})(B)$
there exists an $i$ such that 
whenever $F$ maps $z$ in $\Mgamma{A}$ to $\mathcal{M}(\Gamma, x :
\ttvar~A\vbar \gamma\{x\gets z\})(B)$, there exists an $i$ such
that $F\in W_i$\@.  
\end{enumerate}
\end{theorem}
\begin{proof}
The proof is by simultaneous induction on the sizes of pretypes $A$ and
$B$\@.  We present the structure of the entire proof here but refer to
the proofs of Theorems~\ref{simple-type-semantics}
and~\ref{subtype-type-semantics} as needed.
\begin{enumerate}
\item $A\equiv s$: Same as in Theorem~\ref{simple-type-semantics}\@.
%
\item $A\equiv \funtype{y : C}{D}$: We have by the induction hypothesis
that $\Mgamma{C}\in U$ and for any $z\in \Mgamma{C}$, $\mathcal{M}(\Gamma,
y:\ttvar~C\vbar \gamma\{y\gets z\})(D) \in U$\@.  We also have by the
induction hypothesis with $D$ for $B$ that there is an $i$ such that
the function $F$ mapping $z\in \Mgamma{C}$ to $\mathcal{M}(\Gamma,
y:\ttvar~C\vbar \gamma\{y\gets z\})(D)$ is in $W_i$\@.
Since, by Definition~\ref{defn:meaning function with dependent types},
$\Mgamma{A} = \Pi{F}$, we have $\Mgamma{A}$ in $U_{i+1}$ and hence $U$\@.
%
\item $A\equiv \tupletype{x_1 : T_1,\ldots, x_n : T_n}$:  We have by the
induction hypothesis that $\Mgamma{T_1}\in U$ and that there is an $i$
such that 
the function $F$ mapping $z$ in $\Mgamma{T_1}$ to
$\mathcal{M}(\Gamma, x: \ttvar~T_1\vbar \gamma\{x\gets z\})(\tupletype{x_2:
T_2, \ldots x_n: T_n})$  is in $W_i$\@.  Since, by
Definition~\ref{defn:meaning function with dependent types},
$\Mgamma{A} = \Sigma{F}$,
we have by Definition~\ref{defn:type universe with dependent types},
that $\Mgamma{A}\in U_{i+1}$ and hence $\Mgamma{A}\in U$\@.
%
\item $A\equiv \{y : T | a\}$: Same as in
Theorem~\ref{subtype-type-semantics}\@.
%
\item $B \equiv s$:  Let $j$ be the rank of $\Mgamma{A}$\@.  
Then $\mathcal{M}(\Gamma, x : \ttvar~A\vbar \gamma\{x\gets
z\})(B)$ is just $\gamma(s)$ and by Definition~ref{defn:satisfaction},
we have some $k$ such that $\gamma(s)\in U_k$\@.  Let $i$ be the maximum
of $j$ and $k$, we have $F\in U_i$\@.
%
\item $B\equiv \funtype{x: C}{D}$:  
\end{enumerate}
\end{proof}
\end{comment}

\section{Summary}

Dependent typing is a significant enhancement to PVS since it adds
an important degree of flexibility and precision to the type system.
Notions  such as subtype constraints and type equivalence
that were introduced for subtyping can be extended for the case of
dependent types.  The semantic universe must be extended to include
additional sets to accommodate the semantics of dependent types.
The rank-boundedness of type dependencies is crucial in demonstrating that
dependent types can be interpreted in this extended semantic universe.
  






\chapter{Theories and Parametric Theories}\label{theories}

The next extension of the PVS language introduces theories and parametric
theories.  
The theory construct of PVS provides a way of packaging together a related
collection of declarations.  Theories can be parametric in
individual or type parameters.  Thus, PVS permits polymorphism or type
parametricity only at the theory level rather than at the declaration
level as in HOL~\cite{Gordon&Melham:HOL}.  
We first consider PVS theories without parameters.  The main change now is
that contexts are no longer simple and can contain theory declarations as
well.  A theory declaration has the form $m : \tttheory{} = \Delta$,
where $\Delta$ is a \emph{simple context} with no variable or theory
declarations. 
If $\Gamma(m)$ is the declaration $m : \tttheory{} = \Delta$, then
$ \itkind{}(\Gamma(m)) = \tttheory{}$, and
$ \itdef{}(\Gamma(m)) = \Delta$\@.
Correspondingly, constants and type names are no longer just symbols
but can be compound names of the form $m.s$ where $m$ is a symbol
naming a theory and $s$ is a symbol corresponding to the constant or type
name.

\section{Theories without Parameters}

To define the type rules for theories, we first modify the definition of
$\tau$ for simple contexts so that the context argument is not always
empty.  Here $\Delta ; \Gamma$ represents the concatenation of contexts.
\begin{Defn}{type rules for contexts}
\begin{eqnarray*}
  %
  \tau(\Theta)(\{\}) & = & \ttcontext \\
%
  \tau(\Theta)(\Gamma, s\ :\ \tttype{} = T) & = & \ttcontext, \mbox{
if } \Gamma(s) \mbox{ and } \Theta(s) \mbox{ are undefined,}\\& & 
  \tau(\Theta)(\Gamma) = \ttcontext, \mbox{and} \\& &
  \tau(\Theta;\Gamma)(T) = \tttype{}\\
%
  \tau(\Theta)(\Gamma, c : T) & = & \ttcontext, \mbox{ if }
\Gamma(c) \mbox{ and } \Theta(c)  \mbox{ are undefined,}\\ & &
\tau(\Theta)(\Gamma) = \ttcontext, \mbox{and }\\
   & &  \tau(\Theta;\Gamma)(T) =
      \tttype{}\\
%
%
  \tau(\Theta)(\Gamma, x : \ttvar{}\ T) & = & \ttcontext, \mbox{ if }
\Gamma(x) \mbox{ and } \Theta(x) \mbox{ are undefined,} \\ & &
\tau(\Theta)(\Gamma) = \ttcontext, \mbox{and }\\
    & &  \tau(\Theta;\Gamma)(T)
    = \tttype{}
%
\end{eqnarray*}
\end{Defn}
\begin{Eg}{type rules for contexts}
\begin{eqnarray*}
\tau(\Omega)(\ttreal{} : \tttype{}, \mathtt{0} : \ttreal{},
\leq : \funtype{\tupletype{\ttreal{}, \ttreal{}}}{\ttbool}) & = & \ttcontext
\end{eqnarray*}
\end{Eg}

The following rule handles theory declarations.
\begin{Defn}{type rule for contexts with theory declarations}
\begin{eqnarray*}
  \tau(\Theta)(\Gamma, m : \tttheory{} = \Delta) & = & \ttcontext
     \mbox{ if } \Theta(m), \Gamma(m) \mbox{ are undefined} \\& &
     \Delta \mbox{ only has constant and type declarations,}\\ & &
     \tau(\Theta ; \Gamma)(\Delta)  = \ttcontext,\\& &
     \tau(\Theta)(\Gamma) = \ttcontext
\end{eqnarray*}
\end{Defn}
\begin{Eg}{contexts with theory declarations}
\begin{eqnarray*}
\lefteqn{\tau(\Omega)(
\mathtt{reals} : \tttheory{} = (\ttreal{} : \tttype{}, \mathtt{0} : \ttreal{},
\leq : \funtype{\tupletype{\ttreal{},
\ttreal{}}}{\ttbool})) }\\
& = & 
\ttcontext
\end{eqnarray*}

\end{Eg}


\begin{comment}
Perhaps each declaration should be required to introduce a new
symbol so that there is no confusion between $\Gamma$ and $\gamma$.
\end{comment}

\begin{comment}
Example:
bool: TYPE,
integer : THEORY = (integer: TYPE, + :
\funtype{\tupletype{integer, integer}}{ {\tt integer}}),
natural : THEORY = (nat : TYPE)
\end{comment}

Any reference to a type name or a constant $s$ declared in a theory $m$
outside of this theory must be prefixed by the theory name, as in $m.s$\@.
Note that references to a type name or constant that is declared in the
same theory 
should not be given a theory prefix.  Before we can give the type rules, we
must update the definition of the type expansion
operation $\delta$ to prefix symbols with their theory names.
Let $\Gamma(m)(s)$  abbreviate $ \itdef{}(\Gamma(m))(s)$, which is
the declaration of the symbol $s$ in the definition of the theory $m$\@.
Let $\eta(\Gamma, m)(a)$ be the result of prefixing every unprefixed type or
constant symbol in $a$ by $m$, where $a$ is either an individual or type
expression.  We omit the definition of $\eta$ since it is straightforward.
  
%The $\delta$ operation is modified so that its first argument is now a
%pair consisting of a context and the `current' theory name.  The theory
%name component can be empty, as in Definition~\ref{defn:expanded type}\@.
We modify the definition of $\delta$ in Definition~\ref{defn:expanded
type} with the following clauses.  
\begin{Defn}{expanded type for prefixed symbols}
\begin{eqnarray*}
  \delta(\Gamma)(m.s) & = & \delta(\Gamma)(
\eta(\Gamma, m)(\itdef{}(\Gamma(m)(s)))), 
                        \mbox{ if }\\& &  \itdef{}(\Gamma(m)(s)) \mbox{ is
nonempty.}\\
  \delta(\Gamma)(m.s) & = & m.s
                        \mbox{ if } \itdef{}(\Gamma(m)(s)) \mbox{ is
empty.}
%\\
% \delta(\Gamma, m)(s) & = & m.s \mbox{ if } 
%\itdef{}(\Gamma(m)(s)) \mbox{ is
%empty.  }
\end{eqnarray*}
\end{Defn}
\begin{Eg}{expanded type for prefixed symbols}
Let $\Omega''$ be the context
$$\Omega, 
\mathtt{reals} : \tttheory{} = \begin{array}[t]{l}
				   (\ttreal{} : \tttype{},\\\
                                    \mathtt{0} : \ttreal{},\\\ 
                        \leq : \funtype{\tupletype{\ttreal{},
              \ttreal{}}}{\ttbool}, \\\ 
        \mathtt{nonneg\_real} : \tttype{} = \{x : \ttreal{} \vbar
\leq(0, x)\},\\\  \mathtt{1} : \mathtt{nonneg\_real} )
				   \end{array}$$

$$\delta(\Omega'')( \mathtt{reals.nonneg\_real}) =
\{ {x} : \mathtt{reals.real} \vbar
\mathtt{reals}.\!\leq( \mathtt{reals.0}, {x})\}$$
\end{Eg}

The type rules for prefixed symbols are given below.
\begin{Defn}{type rules for prefixed symbols}
\begin{eqnarray*}
  \tauGamma{m.s} & = & \tttype{}, \mbox{ if } \itkind{}(\Gamma(m)) =
\tttheory{} \mbox{ and }\\ & &
\itkind{}(\Gamma(m)(s)) = \tttype{}\\
%
  \tauGamma{m.s} & = & \delta(\Gamma)(\eta(\Gamma, m)(\ittype(\Gamma(m)(s)))), \\ &
&
\mbox{ if } \itkind{}(\Gamma(m)) =
\tttheory{} \mbox{ and }\\ & &
\itkind{}(\Gamma(m)(s)) = \ttconstant{}
\end{eqnarray*}
\end{Defn}

\begin{Eg}{type rules for prefixed symbols}
\begin{eqnarray*}
\tau(\Omega'')(\mathtt{reals.nonneg\_real}) & = & \tttype{}\\
\tau(\Omega'')(\mathtt{reals.1}) & = &
\{ {x} : \mathtt{reals.real} \vbar
\mathtt{reals}.\!\leq( \mathtt{reals.0}, {x})\}
\end{eqnarray*}
\end{Eg}

\begin{comment}
\begin{Eg}{contexts with theory declarations}
\begin{eqnarray*}
\tau(\Omega)(
\begin{array}[t]{l}
\mathtt{reals} : \tttheory{} = (\ttreal{} : \tttype{}, \mathtt{0} : \ttreal{},
\leq : \funtype{\tupletype{\ttreal{},
\ttreal{}}}{\ttbool}), \\
\mathtt{nonneg\_reals}: \tttheory{}  = \\\ \
   ( \mathtt{nonneg\_real} :
\tttype{} = \{ {x} : \mathtt{reals.real} \vbar
\mathtt{reals.\leq}( \mathtt{reals.0}, {x})\},\\\ \  \mathtt{sqrt}: \funtype{
\mathtt{nonneg_real}}{ \mathtt{nonneg_real}}\} )) 
\end{array}
& = & 
\ttcontext
\end{eqnarray*}
%
\end{Eg}
\end{comment}




The operations  $\pi$, and $\mu$ remain unchanged.
  An assignment $\gamma$ now
maps a theory name $m$ to an assignment $\gamma(m)$.
\begin{Defn}{meaning function for prefixed symbols}
\begin{eqnarray*}
  \Mgamma{m.s} & = & \gamma(m)(s)
\end{eqnarray*}
\end{Defn}
\begin{Eg}{meaning function for prefixed symbols}
Let $\omega''$ be a satisfying assignment for $\Omega''$ of the form
$$\ldots \{ \mathtt{reals}\gets \{\ttreal{}\gets
\reals\}\{\mathtt{0}\gets \zerob\}\ldots \} \ldots.$$

\begin{eqnarray*}
\mathcal{M}(\Omega''\vbar\omega'')(\mathtt{reals.real}) & = & \reals\\
\mathcal{M}(\Omega''\vbar\omega'')(\mathtt{reals.0}) & = & \zerob
\end{eqnarray*}
\end{Eg}



\begin{Defn}{satisfaction for contexts with theories}
An assignment~$\gamma$ satisfies a context $\Gamma$ if in addition to the 
constraints stated in
Definition~\ref{defn:satisfaction with type definitions}, $\gamma$ maps 
every theory $m$  declared in $\Gamma$ to 
a satisfying assignment for the body of the theory given by $
\itdef{}(\Gamma(m))$, that is for each declared symbol $s$ in $m$:
\begin{enumerate}
\item If $ \itkind{}(\Gamma(m)(s)) = \tttype{}$, then $\gamma(m)(s)
\in U$.


\item If $\itkind{}(\Gamma(m)(s)) = \ttconstant{}$, then $\gamma(m)(s)
\in \mathcal{M}(\Gamma\vbar\gamma)(\tau(\Gamma)(m.s))$.

\item  If  $\itdef{}(\Gamma(m)(s))$ is nonempty,
then $$\gamma(m)(s) = \mathcal{M}(\Gamma| \gamma)(
\eta(\Gamma, m)(\itdef{}(\Gamma(m)(s)))).$$
\end{enumerate}
\end{Defn}


\section{Constant Definitions}  We first extend the subset of PVS
described so far to include 
constant definitions in a manner similar to type definitions.
This extension is used in formalizing the semantics of parametric theories.
The syntax
for a constant definition is $c : T = a$ where $\itdef{}(\Gamma(c))$ is
$a$\@.  These definitions are explicit, that is, not recursive.  With this
extension, the type rule for constant declarations in contexts changes
from that of Definition~\ref{defn:type rules with subtypes}.
\begin{Defn}{type rule with constant definitions}
\begin{eqnarray*}
\tau(\Theta)(\Gamma, c : T = a) & = & T, \mbox{ if } \Gamma(c) \mbox{ is
undefined}, \\& & \Theta(c) \mbox{ is undefined}, \\& &
\tau(\Theta)(\Gamma) = \ttcontext,\\& &
%\vdash_\Gamma (\exists(x : T) : \tttrue),\\& &
\tau(\Theta;\Gamma)(a) = T',\\& & 
(T\sim T')_\Gamma,\\& &
\vdash_\Gamma \pi(T)(a)
\end{eqnarray*}
\end{Defn}

The notion of satisfaction must be extended from that of
Definition~\ref{defn:satisfaction for contexts with theories} to ensure
that an 
assignment 
for a defined constant satisfies the definition.
\begin{Defn}{satisfaction with constant definitions}
An assignment $\gamma$ satisfies a context $\Gamma$ if
in addition to the conditions in
Definition~\ref{defn:satisfaction for contexts with theories}, whenever $
\textit{kind}(\Gamma(s)) = \ttconstant$ and 
$\textit{definition}(\Gamma(s))$ is nonempty, then
$\gamma(s) = \Mgamma{ \textit{definition}(\Gamma(s))}$.
\end{Defn}


\section{Parametric Theories}

The extension to parametric theories is obtained by permitting
theories to be declared as $ m[\Pi] : \tttheory{} = \Delta$,
where $\Pi$ is a context listing the parameters and $\Delta$ is
the body of the theory.  If the above declaration of $m$ occurs
in context $\Gamma$, then $\Pi$ is $ \emph{formals}(\Gamma(m))$, and
$\Delta$ is $ \itdef{}(\Gamma(m))$.
For nonparametric theories, $ \mathit{formals}(\Gamma(m))$
is empty.   
Types or constants declared in a parametric theory are
referenced outside the theory as $m[\sigma].s$, where $\sigma$
is a list of \emph{actual} parameters consisting of types and terms.  
The type rule from the nonparametric case
must be modified to check the parameters.
\begin{Defn}{type rule for contexts with parametric theories}
\begin{eqnarray*}
  \lefteqn{\tau(\Theta)(\Gamma, m[\Pi] : \tttheory{} = \Delta)}\\ & = & \ttcontext
     \mbox{ if } \Gamma(m), \Theta(m), \Pi(m) \mbox{ are undefined}\\& &
      \tau(\Theta)(\Gamma) = \ttcontext\\& &
      \tau(\Theta ; \Gamma)(\Pi)  = \ttcontext,  \\ & & 
      \Pi \mbox{ has only constant and}\\& & \mbox{type declarations
without definitions,}\\ & &
      \tau(\Theta ; \Gamma ; \Pi)(\Delta)  = \ttcontext \\& &
      \mbox{ $\Delta$ only has type and constant declarations}
\end{eqnarray*}
\end{Defn}

  The 
 type  rules for prefixed symbols are given below.
The notation $\Pi = \sigma$, where $\Pi$ is of the form
$s_1 : \alpha_1,\ldots, s_n : \alpha_n$, and $\sigma$ is of
the form $\sigma_1,\ldots, \sigma_n$, is short for
the context $s_1 : \alpha_1 = \sigma_1, \ldots, s_n : \alpha_n =
\sigma_n$\@.  The definition of $\eta$ is now extended to substitute
actual theory parameters for formals, so that $\eta(\Gamma, m[\sigma])(a)$
prefixes every unprefixed symbol $s$ in $a$ that is declared in
$\itdef(\Gamma(m))$ by $m[\sigma]$, and replaces any $s_i$ in $a$
that is declared in $\itformals(\Gamma(m))$ by the corresponding
$\sigma_i$ in $\sigma$\@.  

\begin{Defn}{type rules for prefixed names with actuals}
Let $\Pi$ be $\mathit{formals}(\Gamma(m))$\@.  
\begin{eqnarray*}
  \tauGamma{m[\sigma].s} & = & \tttype{}, \mbox{ if }\\& & 
\itkind{}(\Gamma(m)) =  \tttheory{} \\ & &
\itkind{}(\Gamma(m))(s) = \tttype{} \mbox{ and }\\& &
\tauGamma{\Pi = \sigma} = \ttcontext\\
%
  \tauGamma{m[\sigma].s} & = & \delta(\Gamma)((\eta(\Gamma, m[\sigma])(
\ittype(\Gamma(m)(s)) )), \\ & &
\mbox{ if } \itkind{}(\Gamma(m)) = \tttheory{} \\ & &
\itkind{}(\Gamma(m)(s)) = \ttconstant{} \mbox{ and }\\& &
\tauGamma{\Pi = \sigma} = \ttcontext
\end{eqnarray*}
\end{Defn}

\begin{comment}
The definition of $\delta$ has to be modified slightly to carry along the
parameters.  The notation $a[\Pi]$ represents the result of
replacing each occurrence of a symbol $s$ in $a$ such that $\Pi(s)$ is
defined, by $\itdef{}(\Pi(s))$ when it exists.
\end{comment}
\begin{Defn}{type expansion with parametric theories}
\begin{eqnarray*}
  \delta(\Gamma)(m[\sigma].s) & = & \delta(\Gamma)(
(\eta(\Gamma, m[\sigma])(\itdef{}(\Gamma(m)(s)))) ), 
                        \mbox{ if }\\ & &  \itdef{}(\Gamma(m)(s)) \mbox{ is
nonempty.}\\
  \delta(\Gamma)(m[\sigma].s) & = & m[\sigma].s, 
                        \mbox{ if } \itdef{}(\Gamma(m)(s)) \mbox{ is
empty.}
%\\
% \delta(\Gamma, m[\sigma])(s) & = & \delta(\Gamma)( \itdef{}((\Pi
%= \sigma)(s))) \mbox{ if } \\& &
% \Pi(s) \mbox{ is defined}\\
% \delta(\Gamma, m[\sigma])(s) & = & m[\sigma].s \mbox{ if } \\& &
%\itdef{}(\Gamma(m)(s)) \mbox{ is
%empty.  }
\end{eqnarray*}
\end{Defn}

The definition of an assignment for a context with parametric theories is
a bit complicated.  
In the nonparametric case, $\gamma(m)$ simply returns an
assignment of values for the types and constants declared in the theory
$m$\@.  For the case of parametric theories $m$, $\gamma(m)$  returns a
function that maps the meaning of the given actuals $\sigma$ to
an assignment $\gamma(m)(\Mgamma{\sigma})$ for the  types and
constants declared in the theory $m$\@.   
There is an important restriction that $\gamma(m)$ must be
\emph{rank-preserving}, that is,  if $\varpi$  
and $\varpi'$ are assignments for 
$\Pi$ so that for each $i$ where $\Pi_i$ is a type parameter, the
rank of $\varpi(\Pi_i)$ equals the rank of 
$\varpi'(\Pi_i)$,  then the ranks of $\gamma(m)(\varpi)(s)$
and $\gamma(m)(\varpi')(s)$ must be the same for each type symbol $s$
declared in $m$\@.  
%\memo{This needs to be stated more directly.}

It is also important to observe that the semantics of parametric theories
makes use of the axiom of choice since the assignment corresponding to a
theory $\mathtt{m}$ of the form  $\mathtt{m [t:TYPE]: THEORY =
\{c : t\}}$ is essentially a choice function.  

Let $\gamma\{\Pi \gets \varpi\}$ represent the assignment such that
$\gamma\{\Pi \gets \varpi\}(s) = \varpi(s)$ for $s$
in the domain of the context $\Pi$, and $\gamma(s)$, otherwise\@.  
The meaning of symbols of the form $m[\sigma].s$ can then be defined
as below.
\begin{Defn}{meaning function for prefixed symbols with actuals}
\begin{eqnarray*}
  \lefteqn{\Mgamma{m[\sigma].s}}\\ & = &
\mathcal{M}(\Gamma;\Pi;\Delta \vbar
\gamma\{\Pi\gets\varpi\}\{\Delta\gets\gamma(m)(\varpi)\})(s), \mbox{ where}\\
 & & \Pi = \mathit{formals}(\Gamma(m)) \\
& &      \Delta = \mathit{definition}(\Gamma(m)) \\
& &      \varpi(r) = \Mgamma{(\Pi = \sigma)(r)}, \mbox{ for } r\in \Pi
\end{eqnarray*}
\end{Defn}

The definition of a satisfying assignment given in
Definition~\ref{defn:satisfaction for contexts with theories} 
also must be strengthened.    
Let $\Pi$ be the
formal parameters to theory $m$ in context $\Gamma$; then, an assignment
$\varpi$ is 
said to be satisfying {\em parameter assignment} for $\Pi$ under the
assignment $\gamma$ to $\Gamma$ 
iff $\gamma\{\Pi\gets \varpi\}$ is a satisfying assignment for $\Pi$\@.  
\begin{Defn}{satisfaction for contexts with parametric theories}
An assignment
$\gamma$ satisfies a context $\Gamma$ if in addition to the 
constraints stated in
Definition~\ref{defn:satisfaction for contexts with theories}, $\gamma$ maps 
every parametric theory $m$  declared in $\Gamma$ with parameters $\Pi$
and definition $\Delta$, 
to a function that maps
any satisfying parameter assignment $\varpi$ for the theory parameters $\Pi$
(namely, 
$\itformals(\Gamma(m))$) to a satisfying assignment $\gamma\{\Pi\gets
\varpi\}\{\Delta \gets \gamma(m)(\varpi)\}$ for 
$\Delta$ (given by 
$\itdef{}(\Gamma(m))$).
\begin{comment}
That is, 
for each declared symbol $s$ in
$\Delta$: 
\begin{enumerate}
\item If $ \itkind{}(\Gamma(m)(s)) = \tttype{}$, then
$\gamma(m)(\Mgamma{\sigma})(s)
\in U$
%
\item if $\itkind{}(\Gamma(m)(s)) = \ttconstant{}$, then $\gamma(m)(\Mgamma{\sigma})(s)
\in \mathcal{M}(\Gamma| \gamma;\{\Pi\gets \Mgamma{\sigma}\};\gamma(m)(\Mgamma{\sigma})(\tau(\Gamma)(m[\sigma].s)))$
%
\item if $\itdef{}(\Gamma(m)(s))$ is defined,
then $\gamma(m)(\Mgamma{\sigma})(s) = \mathcal{M}(\Gamma| \gamma;
\{\Pi\gets \Mgamma{\sigma}\}; \gamma(m)(\Mgamma{\sigma}))(
\itdef{}(\Gamma(m)(s)))$.
\end{enumerate}
\end{comment}
\end{Defn}


\begin{comment}
This map preserves the
assignments given to the parameters so that if for $s : T$ in $\Pi$,
$m[\sigma]$ is such that the formal $s$ is matched with the actual $a$,
then 
$\gamma(m)(\{\Pi\gets \Mgamma{\sigma}\})(s)$ must be $\Mgamma{a}$\@.
\end{comment}
\begin{comment}
The delicate point here is that $\gamma(m)$ is not a set but a map (a
rule) that maps a given assignment (a set) of sets to symbols, to another
assignment of sets to symbols.  This means that $\gamma$ itself is not a
set.  There is no reason why $\gamma$ should be a set since all we require
is that it map a given type name to a set, and a constant to an element
of the set representing its type.  
\end{comment}
\begin{comment}
\begin{eqnarray*}
  \Mgamma{m[\sigma].s} & = & \gamma(m)(\{\Pi\gets \Mgamma(\sigma)\})(s)
\end{eqnarray*}
\end{comment}

\section{Summary}

Theories are used to package related declarations together.  Parametric
theories can be used to package together declarations that are generic in
type and individual parameters.  The type rules for contexts must be
extended to accommodate the theories.  The type rules for simple
(nonparametric) theories are straightforward given this extension.  The
operation of expanding a type using type definitions must be enhanced so
that symbols declared in a theory are prefixed with their theory name when
referenced outside the theory.  Assignments now have the same nested
structure as contexts, and the semantic definition is easily extended to
handle prefixed symbols.  Parametric theories are more complex.  The
theory prefixes now contain actual parameters that have to be typechecked
relative to the expected formal parameters.  The assignments corresponding
to parametric theories are functions that map given assignments for the
formals to assignments for the declarations within a theory.  Such a
mapping must be constrained to be rank-preserving.  Parametric theories
can have subtype parameters, and assumptions on the parameters.  The rules
for subtype parameters and assumptions are omitted for now but will be
included in an expanded version of this report.



\chapter{Conditional Expressions and Logical Connectives}\label{conditionals}

We have, so far, introduced the core of PVS containing types, type
definitions, constant and variable declarations, subtypes, dependent
types, and theories.  In extending the language with
both explicit and recursive constant definitions and formulas,  a crucial
difference  is that the logical context under which a type-correctness
condition is generated provides additional assumptions that can be used in
proving any proof obligations.  Examples of expressions where an
extended context is needed to establish type correctness by discharging
proof obligations include
\begin{enumerate}
\item $x \neq y \supset  (x + y)/(x - y) \leq 0$.  The type of the
division operator constrains the denominator to be nonzero,
that is, $\{x : \ttreal \vbar x \neq 0\}$\@.    
In the given expression,  the denominator
can be shown to be nonzero only in the context of the antecedent $x \neq
y$\@.

\item $\mathtt{IF}(i > 0, i, -i)$ has type $\ttnat$ given integer $i$
provided the {\em then\/} and {\em else\/} parts are typechecked with the
assumptions $i > 0$ and $\neg (i > 0)$, respectively.
\end{enumerate}

PVS has a polymorphic primitive equality predicate:
\begin{display}
equality[T : TYPE] : THEORY = \{ =: [[T, T] -> bool] \}
\end{display}
Note that an equality of the form $ \mathtt{equality}[T].\mathtt{=}(a, b)$ is
informally written as $a = b$\@.  When it is relevant to indicate the type
parameter, we write the equality as $a =_{T} b$\@.  It can be deduced from
the meaning of equality that if $S$ is a subtype of $T$, then for $a$ and
$b$ in $S$, it must be the case that $a =_{S} b$ iff $a =_{T} b$\@.  Thus,
we can assume that equality is always parameterized by a maximal type.  We
assume that any relevant context $\Gamma$ contains the above declaration
of the theory {\tt equality}.  Furthermore, any satisfying assignment
$\gamma$ for such a $\Gamma$ must satisfy
\begin{eqnarray*}
\gamma( \mathtt{ equality})(X)(=) &  = & \{\pair{x, x}\vbar x\in X\}.
\end{eqnarray*}

The negation operation can be defined in terms of equality as shown below.
We assume that the context contains a declaration of the form
\begin{alltt}
   \(\neg : \funtype{\ttbool}{\ttbool} = (\lambda (x : \ttbool): x = \ttfalse)\)
\end{alltt}

As is clear, a satisfying assignment $\gamma$ for a context $\Gamma$
containing the above declaration must be such that $\gamma(\neg)$ yields
the usual truth-table semantics, that is, 
$\{ \pair{\zerob, \oneb}, \pair{\oneb, \zerob} \}$\@.  

We can then introduce the polymorphic \texttt{IF-THEN-ELSE} operation
as follows: 
\begin{display}
if_def [T: TYPE]: THEORY = \{ IF:[bool,T,T -> T] \}
\end{display}

In typechecking conditional expressions, the notion of context
has to be extended to include formulas so that the typechecking
of the subterm $b$ in $\mathtt{IF}(a, b, c)$ is done in the context of $a$,
and the typechecking of $c$ is done in the context of $\neg a$\@.
There is one new typechecking rule for contexts with formulas.
\begin{eqnarray*}
\tau()(\Gamma, a) & = & \ttcontext, \mbox{ if}\\
& & \tau()(\Gamma) = \ttcontext, \mbox{ and} \\
& & (\tau(\Gamma)(a)   \sim \ttbool)_\Gamma
\end{eqnarray*}
Note that the type rule checks that the type of $a$ is compatible with
$\ttbool$ rather than equivalent to it since it is possible that the  type
of $a$ might be a subtype of $\ttbool$\@.  


\begin{Defn}{satisfaction for contexts with formulas}
An assignment $\gamma$ satisfies context $\Gamma$ when in 
addition to the conditions in
Definition~\ref{defn:satisfaction for contexts with parametric theories},  
for each prefix $\Gamma', a$ of $\Gamma$, $\mathcal{M}(\Gamma'\vbar \gamma)(a)
= \oneb$\@.  
\end{Defn}

The typechecking of conditional expressions is different from that of
other application expressions since the \emph{test} part of the
conditional expression is introduced into the context as a contextual
assumption.
\begin{Defn}{type rule for conditional expressions}
   \begin{eqnarray*}
     \tauGamma{ \mathtt{ if\_def[\mathit{T}].IF}(a, b, c)} & = & T, \mbox{ if }
(\tauGamma{a} \sim  \mathtt{bool})_\Gamma,\\ & &
   \tau(\Gamma, a)(b) = B,\\& &  (B \sim T)_{\Gamma, a}, \\& & \vdash_{\Gamma,
a}\pi(T)(b)\\& & 
    \tau(\Gamma, \neg a)(c) = C,\\& &  (C \sim T)_{\Gamma, \neg a},\\ & & 
\vdash_{\Gamma, \neg a} \pi(T)(c)
\end{eqnarray*}
\end{Defn}

The meaning of conditional expressions must be treated in a special way
since the \emph{else} part need not denote when the \emph{test} part is
true and, correspondingly, the \emph{then} part need not denote if
the \emph{test} part is false.  We assume that any relevant contexts $\Gamma$
contain the above declaration of the \texttt{if\_def} theory.
Conditional expressions can be regarded as a
new construct in the language rather than a form of application.
However, it is conservative to regard conditional expressions as
applications since the latter introduce the additional constraint that all
the arguments must already denote, that is, applications are
\emph{strict}.
\begin{Defn}{meaning function for conditional expressions}
\begin{eqnarray*}
\Mgamma{\mathtt{if\_def}[T].\ttif(a, b, c)} & = &
 \left\{\begin{array}{ll}
	\Mgamma{b}, & \mbox{ if } \Mgamma{a} = \oneb\\
        \Mgamma{c}, & \mbox{ otherwise}
	\end{array}\right.
\end{eqnarray*}
\end{Defn}

The semantics for conditional expressions raises an important issue.
The equality
$$\mathtt{if\_def}[\mathtt{bool}].\ttif(x, y, \ttfalse) =
\mathtt{if\_def}[\mathtt{bool}].\ttif(y, x, \ttfalse)$$
is semantically valid for variables $x$ and $y$ of type $\ttbool$\@.
An expression like $\mathtt{if\_def}[\ttbool].\ttif(i \neq 0, 1/i
> 0, \ttfalse)$ can be typechecked to have the type $\ttbool$ since
it generates a valid proof obligation $i\neq 0 \supset i\neq 0$, but
the seemingly equivalent expression
$\mathtt{if\_def}[\ttbool].\ttif( 1/i > 0,i \neq 0, \ttfalse)$
generates an unverifiable proof obligation $i\neq 0$\@.
This may seem contradictory since the equality suggests a transformation
of a type correct conditional expression to a type incorrect expression.  
The resolution here is that equality cannot be instantiated with
$i\neq 0$ for $x$ and $1/i > 0$ for $y$ since the expression
$1/i > 0$  typechecks as having type $\ttbool$ only when $i\neq 0$ is known
from the context.  The same applies in the case of the other
propositional connectives, thus ensuring that each expression is type
correct in the context in which it occurs.  

We can then define the propositional connectives in terms of  conditional
expressions. 
\begin{eqnarray*}
  \wedge : \funtype{\tupletype{\ttbool, \ttbool}}{\ttbool} & = & \lambda (x : \ttbool, y
: \ttbool) : \mathtt{if\_def[\ttbool].IF}(x, y, \ttfalse)\\
   \vee : \funtype{\tupletype{\ttbool, \ttbool}}{\ttbool} & = & \lambda (x : \ttbool, y
: \ttbool) : \mathtt{if\_def[\ttbool].IF}(x, \tttrue, y)\\
   \supset : \funtype{\tupletype{\ttbool, \ttbool}}{\ttbool} & = & \lambda (x : \ttbool, y
: \ttbool) : \mathtt{if\_def[\ttbool].IF}(x, y, \tttrue)\\
%  \neg : \funtype{\ttbool}{\ttbool} & = & \lambda (x : \ttbool) :
%\mathtt{if\_def[\ttbool].IF}(x, \ttfalse, \tttrue)
\end{eqnarray*}

In the typechecking of terms of the form $a \wedge b$,
we follow the corresponding rule for the definition so that 
the term $a$ is assumed in the context when typechecking term $b$\@.
Similarly, for $a\vee  b$, the formula $\neg a$ is assumed in
the context when typechecking $b$, and for $ a \supset b$,
the formula $ a$ is assumed in the context when typechecking $b$.
The Boolean equivalence operator {\tt IFF} has no special rules
for adding formulas to contexts during typechecking.   

\section{Summary}

The use of assumption formulas  enables expressions
to be typechecked within the narrow context of their use so that
the governing assumptions can be used in discharging any
proof obligations.  The type rules for conditional expressions and the
Boolean connectives $\wedge$, $\vee$, and $\supset$ make use of
contextual assumptions.  

\begin{comment}
We will assume that the context $\Gamma$ used below contain the
theories \mathtt{equality} and \mathtt{if\_def}\@.  
The typechecking of constant definitions is straightforward when
the definitions are not recursive.  As we have already seen, the
form of an explicit definitions is $ s : T = a$, where $s$ is a symbol,
$T$ is a type, and $a$ is a term.  
When a constant is defined, no nonemptiness TCCs are generated since
there is a definition present.
%
\begin{eqnarray*}
  \tauGamma{\Delta, s : T = a} & = & \mathtt{CONTEXT}, \mbox{ if }
    \tauGamma{\Delta} = \mathtt{CONTEXT} \\ & &
   \tau(\Gamma, \Delta)(T)  = \mathtt{TYPE}\\ & &
   \tau(\Gamma, \Delta)(a) = T', \mbox{ and}\\ & &
   (T\sim_\Gamma T')_{\Gamma, \Delta}\\& &
   \vdash_{\Gamma, \Delta} \pi(T)(a)
\end{eqnarray*}
%
A recursive constant definition has the syntactic form
$ s : RECURSIVE\ T = a\ MEASURE\ f RELATION p$\@.
%
\memo{This looks slightly complicated. Postpone for now.}
%
A formula declaration has the form
$s : \ \mathtt{FORMULA} a$.
%
\memo{Need to formally defined substitution, freevars, and closure
relative to a context.}  
\end{comment}

\begin{comment}
%
\memo{Need to highlight the shape of a context: theory declarations
followed by type/constant declarations followed by variable declarations.}
%
Introduce equality, if-then-else, connectives, updates.
%
\section{Abstract Datatypes}
\end{comment}


\chapter{Proof Theory of PVS}\label{proof theory}

The final step in the presentation of the semantics is the presentation of
the proof rules for the idealized subset of PVS described thus far.
As already indicated, the proof theory is an integral part of the
semantics since typechecking and proof checking are closely intertwined.
Fortunately, the proof rules turn out to be much less complicated than the
type rules.

\newcommand{\sequent}[3]{#1\vdash_{#2}#3}

The PVS proof theory is presented in terms of  a sequent calculus.  A
sequent is of the form 
$\sequent{\Sigma}{\Gamma}{\Lambda}$, where $\Gamma$ is the context,
$\Sigma$ is a {\em set\/} of {\em antecedent\/} formulas, and $\Lambda$ 
is a {\em set\/} of {\em consequent\/} formulas.  Such a sequent should
be read as stating that the conjunction of the formulas in $\Sigma$ implies the
disjunction of formulas in $\Lambda$\@.

% presumed in math mode #1=top, #2=bottom, #3=side condition
\renewcommand{\Infrule}[3]{
{{\displaystyle\strut #1}\over{\displaystyle\strut #2}}~\makebox[0pt][l]{{$\mathbf{ #3}$}}
}

Inference rules are presented in the form
\begin{center}
$\Infrule{\mathit{premise(s)}}{\mathit{conclusion}}{\mathit{name}}\hspace{2cm}
\mathit{side\ condition}$
\end{center}


\section{PVS Proof Rules} 

\subsection{Structural Rules}

The structural rules permit the sequent to be rearranged or weakened
via the introduction of new sequent formulas into the conclusion.  
All the structural rules can be expressed in terms of the single
powerful weakening rule  shown below.
It allows a weaker statement to
be derived from a stronger one by adding either antecedent formulas or
consequent formulas.  The relation $\Sigma_1\subseteq\Sigma_2$ holds
between two lists when all the formulas in $\Sigma_1$ occur in the list
$\Sigma_2$.  

\begin{center}
$\Infrule{\sequent{\Sigma_1}{\Gamma}{\Lambda_1}}
         {\sequent{\Sigma_2}{\Gamma}{\Lambda_2}} {W}$
\hspace{1cm}\mbox{\smaller\smaller if $\Sigma_1\subseteq\Sigma_2$ and $\Lambda_1\subseteq\Lambda_2$}
\end{center}

Both the Contraction and Exchange rules shown below are absorbed by the
above  weakening rule {\bf W}\@.     The  {\em Contraction\/} rules
$\mathbf{C}\vdash$ and $\vdash\mathbf{C}$
allow multiple occurrences of the same sequent formula
to be replaced by a single occurrence. 
\begin{center}
\begin{tabular}{ccc}
$\Infrule{\sequent{a,a, \Sigma}{\Gamma}{\Lambda}}%
         {\sequent{a, \Sigma}{\Gamma}{\Lambda}} {C\vdash}$
& \hspace{1in} &
$\Infrule{\sequent{\Sigma}{\Gamma}{ a, a, \Lambda}}%
         {\sequent{\Sigma}{\Gamma}{ a, \Lambda}} {\vdash C}$
\end{tabular}
\end{center}

The {\em Exchange\/} rule asserts that the order of the formulas in the
antecedent and the consequent parts of the sequent is immaterial.  It can
be stated as

\begin{center}
\begin{tabular}{c@{\hspace{1in}}c}
$\Infrule{\sequent{\Sigma_1, b, a,  \Sigma_2}{\Gamma}{\Lambda}}
         {\sequent{\Sigma_1, a, b, \Sigma_2}{\Gamma}{\Lambda}} {X\vdash}$
&
$\Infrule{\sequent{\Sigma}{\Gamma}{\Lambda_1, b, a, \Lambda_2}}
         {\sequent{\Sigma}{\Gamma}{\Lambda_1, a, b, \Lambda_2}} {\vdash X}$
\end{tabular}
\end{center}

As seen above, 
{\em inference rules\/} have the general form
$$\Infrule{\Sigma_1\vdash\Lambda_1\quad\cdots\quad\Sigma_n\vdash\Lambda_n}
{\Sigma\vdash\Lambda}{R}.$$ This says that if we are given a leaf of a
proof tree of the form $\Sigma\vdash\Lambda$, then by applying the rule
named {\bf R}, we may obtain a tree with $n$ new leaves.

\subsection{Cut Rule}

The cut rule {\bf Cut} can be used to introduce a case split on a formula $a$
into a proof of a sequent $\sequent{\Sigma}{\Gamma}{\Lambda}$ so as to yield the
subgoals 
$\sequent{\Sigma, a}{\Gamma}{\Lambda}$ and
$\sequent{\Sigma}{\Gamma}{a, \Lambda}$, which can be seen  
as assuming $a$ along one branch and $\neg a$ along the other.
\begin{center}
$\Infrule{(\tau(\Gamma)(a) \sim \ttbool)_\Gamma \qquad \sequent{\Sigma,  a}{\Gamma}{\Lambda}\qquad\sequent{\Sigma}{\Gamma}{ a, \Lambda}}
         {\sequent{\Sigma}{\Gamma}{\Lambda}}{Cut}$
\end{center}

\subsection{Propositional Axioms}

The axioms rule \textbf{Ax} simply asserts that $a$ follows from $a$\@.  
%\begin{center}%{(\tauGamma{a}\sim \ttbool)_\Gamma}
$$\Infrule{}{\sequent{\Sigma, a}{\Gamma}{a,\Lambda}}{Ax}$$  

The next two rules assert that any sequent with either
an antecedent occurrence of $\ttfalse$ or a consequent
occurrence of $\tttrue$ is an axiom.

$$
\begin{array}{c@{\hspace{1in}}c}
\Infrule{}{\sequent{\Sigma, \mathtt{FALSE}}{\Gamma}{\Lambda}}{
\ttfalse \vdash} &
\Infrule{}{\sequent{\Sigma}{\Gamma}{ \mathtt{TRUE}, \Lambda}}{
 \vdash \tttrue}
\end{array}
$$



\subsection{Context Rules}

Certain formulas hold in a context simply because they are already
asserted in the context either as a formula or a constant definition.
$$
\begin{array}{c}
\Infrule{}{\sequent{}{\Gamma}{a}}{Context Formula} \hspace{1.5in}\mbox{if
$a$ is a formula in $\Gamma$}\hspace{.8in}\\
\Infrule{}{\sequent{}{\Gamma}{s = a}}{Context Definition} \hspace{1.5in}\mbox{if
$s : T = a$ is a constant definition in $\Gamma$}
\end{array}
$$

The context $\Gamma$ can be extended with antecedent formulas or
negations of consequent formulas using the following two rules.
$$\begin{array}{c@{\hspace{1in}}c}
\Infrule{\sequent{\Sigma, a}{\Gamma,
a}{\Lambda}}{\sequent{\Sigma, a}{\Gamma}{\Lambda}}{Context \vdash} &
\Infrule{\sequent{\Sigma}{\Gamma,
\neg a}{a, \Lambda}}{\sequent{\Sigma}{\Gamma}{a, \Lambda}}{\vdash Context}
\end{array}
$$

The following context-weakening rule is useful since it shows that
provability is monotonic with respect to the context.
$$\begin{array}{c}
\Infrule{\sequent{\Sigma}{\Gamma}{\Lambda}}{\sequent{\Sigma}{\Gamma'}{\Lambda}}{Context
W}  \hspace{1in}\mbox{if $\Gamma$ is a prefix of $\Gamma'$}
\end{array} 
$$

\begin{comment}
\subsection{Negation Rules}
%
These rules are straightforward since negation is eliminated by moving
the body of a negated antecedent formula to the consequent, and
conversely, by moving the body of a negated consequent formula to
the antecedent.
$$
\begin{array}{c}
\Infrule{\sequent{\Sigma}{\Gamma}{ a, \Lambda}}
         {\sequent{\Sigma,\neg a }{\Gamma}{\Lambda}} {\neg\vdash}\\[.5cm]
\Infrule{\sequent{\Sigma, a }{\Gamma}{\Lambda}}
         {\sequent{\Sigma}{\Gamma}{ \neg a, \Lambda}} {\vdash\neg}
\end{array}
$$
\end{comment}

\subsection{Conditional Rules}

The rules governing the elimination of \texttt{IF-THEN-ELSE} in a proof
are unusual since they augment the context with the test part or its
negation, as in the corresponding type rules.
$$
\begin{array}{c}
\Infrule{\sequent{\Sigma, a, b}{\Gamma, a}{\Lambda}\hspace{1cm}
        \sequent{\Sigma,  c}{\Gamma, \neg a}{a, \Delta}}
         {\sequent{\Sigma, \mathtt{IF}(a, b, c)}{\Gamma}{ \Lambda}}
                  {\mathtt{IF} \vdash}\\[.5cm]
\Infrule{\sequent{\Sigma, a}{\Gamma,
a}{b,\Lambda}\hspace{1cm}\sequent{\Sigma}{\Gamma,\neg a}{a, c, \Lambda}}
         {\sequent{\Sigma}{\Gamma}{{\mathtt{IF}}(a, b, c), \Lambda}}
                      {\vdash {\mathtt {IF}}}
\end{array}
$$

\subsection{Equality Rules}

The rules for equality can be stated as below.  The rules of
transitivity and symmetry for equality can be derived from these rules.
The notation $a[e]$ is used to highlight one or more 
occurrences of $e$ in the formula $a$ such that there are no free variable
occurrences in $e$\@.\footnote{We enforce an invariant on a sequent that
it must not contain any free variables.  This invariant is preserved by
each of the proof rules.}   
The notation $\Lambda[e]$ similarly highlights
occurrences of $e$ in $\Lambda$. 

\begin{center}
\begin{tabular}{c@{\hspace{1in}}c}
$\Infrule{}{\sequent{\Sigma}{\Gamma}{ a=a,\Lambda}}{Refl}$
&
$\Infrule{\sequent{a=b, \Sigma[b]}{\Gamma}{\Lambda[b]}}
         {\sequent{a=b, \Sigma[a]}{\Gamma}{\Lambda[a]}} {Repl}$
\end{tabular}
\end{center}

\subsection{Boolean Equality Rules}

The rule $\mathbf{Repl~\tttrue}$ asserts that an antecedent formula
$a$
can be treated as an antecedent equality of the form $a = \tttrue$, and
correspondingly, 
a consequent formula $a$ can be treated as an antecedent
equality of the form $a = \ttfalse$\@.

$$
\begin{array}{c@{\hspace{1in}}c}
\Infrule{\sequent{\Sigma[\tttrue], a}{\Gamma}{\Lambda[\tttrue]}}{
\sequent{\Sigma[a], a}{\Gamma}{\Lambda[a]}}{Repl~\tttrue} &
\Infrule{\sequent{\Sigma[\ttfalse], a}{\Gamma}{\Lambda[\ttfalse]}}{
\sequent{\Sigma[a]}{\Gamma}{a, \Lambda[a]}}{Repl~\ttfalse}
\end{array}
$$

The rule $\tttrue\mbox{-}\ttfalse$ asserts that $\tttrue$ and $\ttfalse$ are
distinct Boolean constants.
$$
\begin{array}{c}
\Infrule{}{\sequent{\Sigma, \tttrue = \ttfalse}{\Gamma}{\Lambda}}{\tttrue\mbox{-}\ttfalse}
\end{array}
$$
\begin{comment}
Four rules exist for simplifying equalities of the form $a =
\tttrue$ and $a = \ttfalse$ in  a sequent.
%
$$
\begin{array}{c@{\hspace{1in}}c}
\Infrule{\sequent{\Sigma, a}{\Gamma}{\Lambda}}{\sequent{\Sigma, a =
\tttrue}{\Gamma}{\Lambda}}{\tttrue\vdash} &
\Infrule{\sequent{\Sigma}{\Gamma}{a, \Lambda}}{\sequent{\Sigma}{\Gamma}{a =
\tttrue, \Lambda}}{\vdash\tttrue} \\
%
\Infrule{\sequent{\Sigma}{\Gamma}{a, \Lambda}}{\sequent{\Sigma, a =
\ttfalse}{\Gamma}{\Lambda}}{\ttfalse\vdash} &
\Infrule{\sequent{a, \Sigma}{\Gamma}{\Lambda}}{\sequent{\Sigma}{\Gamma}{a =
\ttfalse, \Lambda}}{\vdash\ttfalse} 
\end{array}
$$
\end{comment}

\subsection{Reduction Rules}

The reduction rules are equality rules (axioms) that provide the obvious
simplifications for applications involving lambda abstractions and
product projections.

$$
\begin{array}{c}
\Infrule{}{\sequent{}{\Gamma}{(\lambda (x : T):~a)(b) =
a[b/x]}}{\beta}\\[.5cm]
%
\Infrule{}{\sequent{}{\Gamma}{\proj{i}(a_1, a_2) =
a_i}}{\pi}
\end{array}
$$

\subsection{Extensionality Rules}

The extensionality rules are also equality rules for establishing equality
between two expressions of function or product type.  The extensionality
rule for functions, \textbf{FunExt}, introduces a Skolem constant $s$
to determine that two functions $f$ and $g$ are equal when the results of
applying them to an arbitrary argument $s$ are equal.  
$$
\begin{array}{c}
\Infrule{\sequent{\Sigma}{\Gamma, s : A}{(f~s) =_{B[s/x]} (g~s), \Lambda}}
{\sequent{\Sigma}{\Gamma}{f =_{\funtype{x:A}{B}} g,
\Lambda}}{FunExt} \hspace{1in} \Gamma(s) \mbox{ undefined}\\[.5cm]
\end{array}
$$

The extensionality rule for products asserts that two products are equal if
their corresponding projections are equal.
\begin{comment}
Note that the type
$\tupletype{\listwo{T'}{n}}$ is just $a/\tupletype{\listwo{T}{n}}$\@.
\end{comment}
$$
\begin{array}{c}
\Infrule{\sequent{\Sigma}{\Gamma}{\proj{1}(a) =_{T_1} \proj{1}(b),\Lambda}\qquad\sequent{\Sigma}{\Gamma}{\proj{2}(a) =_{T_2[(\proj{1}~a)/x]} \proj{2}(b),\Lambda}
}{\sequent{\Sigma}{\Gamma}{a =_{\tupletype{x : T_1 T_2}} b, \Lambda}}{TupExt}
\end{array}
$$

Recall that the quantifiers can be defined in terms of lambda abstraction
and equality so that $(\forall (x: T): a)$ is just $(\lambda (x: T): a) =
(\lambda (x : T): \tttrue)$\@.   Existential quantification $(\exists (x:
T): a)$  can easily be defined as $\neg (\forall (x: T): \neg a)$\@.
The proof rules for quantifiers can then be derived from the
rules $\beta$, $\mathbf{TupExt}$, and the equality rules\@.

\subsection{Type Constraint Rule }

We need a rule to introduce the type constraint on a term as an antecedent
formula of the given goal sequent.

$$
\Infrule{\tau(\Gamma)(a) = A\qquad\sequent{\pi(A)(a),\Sigma}{\Gamma}{\Lambda}}
{\sequent{\Sigma}{\Gamma}{\Lambda}}{Typepred}
$$

\section{Soundness of the Proof Rules}
\begin{prop}\label{meaning-preservation}
If $\Gamma$ is a prefix of $\Gamma'$,
$\tau()(\Gamma) = \tau()(\Gamma') = \ttcontext$,
$\gamma'$ is a
satisfying assignment for $\Gamma'$, and $\gamma = \gamma'\restriction\Gamma$ then
for any $a$ such that 
$\tauGamma{a} = \tau(\Gamma')(a)$,  it is the case that $\Mgamma{a} =
\mathcal{M}(\Gamma'\vbar\gamma')(a)$\@.  
\end{prop}

\begin{theorem}[soundness]\label{proof-soundness}
If $\tau()(\Gamma) = \ttcontext$ such that
%\begin{enumerate}
%\item $\Gamma$ contains no variable declarations %bogus(1.18.96)
%\item
for every formula $a$ in $\Sigma;  \Lambda$, $(\tauGamma{a} \sim
\ttbool)_\Gamma$,
%\end{enumerate}
and $\sequent{\Sigma}{\Gamma}{\Lambda}$ is provable, then
for any satisfying assignment
$\gamma$ for $\Gamma$, 
either there is a formula $b$ in $\Sigma$, such that $\Mgamma{b} = \zerob$
or a formula $c$ in
$\Lambda$, such that $\Mgamma{c} = \oneb$\@.      
\end{theorem}
\begin{proof}
The proof is by induction on the structure of the proof of
$\sequent{\Sigma}{\Gamma}{\Lambda}$.  Recall that this proof is actually
part of a simultaneous induction that includes the soundness of the type
rules relative to the semantic function, that is,
Theorems~\ref{dependent-type-semantics}
and~\ref{dependent-term-semantics}\@.   Specific invocations of the
soundness theorem occur in the proofs of
Theorem~\ref{subtype-term-semantics}
and Proposition~\ref{substitution-equivalence}\@.  
\begin{enumerate}
\item {\em Structural Rules\/}:   Since the subset
of formulas in the premise and the conclusion of these rules are the same,
the conclusion follows easily from the induction hypothesis.

\item {\em Cut\/}: By the semantic soundness of the type rules,
we have $\Mgamma{a} \in \twob$\@.   If $\Mgamma{a} = \zerob$, then
by the induction hypothesis on the second subgoal of the proof rule,
there must be some $b$ in $\Sigma$ such that $\Mgamma{b} = \zerob$
or a $c$ in $\Lambda$ such that $\Mgamma{c} = \oneb$\@.  The case
when $\Mgamma{a} = \oneb$ is symmetrical.

\item {\em Propositional Axioms\/}:  Obvious.

\item \emph{Context Rules}:
\begin{description}
\item[ContextFormula:]  If $\gamma$ satisfies $\Gamma$ and $a\in \Gamma$,
then $\Mgamma{a} = \oneb$\@.

\item[ContextDefinition:] If $\gamma$ satisfies $\Gamma$ and $s: T = a$
is a declaration in $\Gamma$, then by the definition of satisfaction,
$\Mgamma{s} = \Mgamma{a}$\@.

\item [Context $\vdash$: ] The argument is trivial when $\Mgamma{a} =
\zerob$.  Otherwise, $\gamma$ satisfies the extended context $\Gamma, a$,
and the conclusion follows from the induction hypothesis.

\item [$\vdash$ Context: ] Similar to $\textbf{Context} \vdash$ above.

\item [ContextW: ]  If $\gamma$ satisfies $\Gamma'$, then it also
satisfies $\Gamma$, and hence the proof.
\end{description}

\item \emph{Conditional Rules}: We only consider $ \mathtt{IF} \vdash$
since the $\vdash \mathtt{IF}$ proof is similar.  If $\Mgamma{
\mathtt{IF}(a, b, c)} = \zerob$, the conclusion follows  trivially.
Otherwise, If $\gamma$ satisfies
$\Gamma$, then $\Mgamma{a}\in \twob$.  If $\Mgamma{a} = \oneb$, then
$\Mgamma{b} = \oneb$.  The induction hypothesis on the subgoal
$\sequent{\Sigma, a, b}{\Gamma, a}{\Lambda}$ yields the desired
conclusion.  Similarly, if $\Mgamma{a} = \zerob$, we have $\Mgamma{c} =
\oneb$
and the induction hypothesis on the second subgoal yields the
desired conclusion.

\item \emph{Equality Rules}: The \textbf{Refl} rule is obvious. 
For the \textbf{Repl} rule, if $\Mgamma{a = b} = \zerob$, the conclusion
follows trivially.  Otherwise, $\Mgamma{a} = \Mgamma{b}$.  Hence, $\gamma$
satisfies the extended context $\Gamma, a = b$.  Then for each
$c[a]$ in $\Sigma[a]$ or $\Lambda[a]$, $\Mgamma{c[a]} = \Mgamma{c[b]}$\@.
\begin{comment}
Note: no free variables in a, b.  Need to show that
replacement preserves meaning when there is no capture.
\end{comment}

\item \emph{Boolean Equality Rules}: The $\mathbf{Repl}~\tttrue$ and
$\mathbf{Repl}~\ttfalse$ rules follow easily since when $\Mgamma{a} =
\oneb$, we have $\Mgamma{c[a]} = \Mgamma{c[\tttrue]}$\@.
A similar argument applies to $\mathbf{Repl}~\ttfalse$\@.

The soundness of $\tttrue\mbox{-}\ttfalse$ is easy since
$\Mgamma{\tttrue = \ttfalse} = \zerob$\@.

\item \emph{Reduction Rules}:  The $\beta$-reduction rule follows because
$\Mgamma{(\lambda (x : T):~a)(b)}$ is
$\mathcal{M}(\Gamma, x : \ttvar~T\vbar \gamma\{x\gets\Mgamma{b}\})(a)$
which by the Substitution Lemma~\ref{substitution-lemma} is
equal to $\Mgamma{a[b/x]}$\@.

The soundness $\pi$-reduction rule is a direct consequent of
Definition~\ref{defn:meaning function}\@.

\item \emph{Extensionality Rules}:
\begin{description}
\item [FunExt:] 
First consider the case when the domain type $\Mgamma{A}$ is empty.  Then by
Definition~\ref{defn:meaning function with dependent types},
$\Mgamma{f} = \Mgamma{g} = \emptyset$\@.  Therefore $\Mgamma{f = g} =
\oneb$ and hence the conclusion.\footnote{Since
the  subgoal sequent $\sequent{\Sigma}{\Gamma, s:A}{(f~s) =
(g~s),\Lambda}$
is valid when $\Mgamma{A} = \emptyset$ for all assignments $\gamma$,
it is natural to ask how it is actually proved.  The only
way a type $A$ can be empty under any assignment $\gamma$ is if
$\Mgamma{\pi(A)(a) = \zerob}$.  The $\mathbf{Typepred}$ rule can
therefore be used on the Skolem constant $s$ to complete the proof.}

The case when $\Mgamma{A}$ is nonempty, we have for any $\gamma$
satisfying $\Gamma$ and $s\in \Mgamma{A}$, that $\gamma'$ given by 
$\gamma\{s\gets z\}$ is a satisfying assignment for $\Gamma, s: A$\@.
By the induction hypothesis, there is either an $a$ in $\Sigma$
such that $\mathcal{M}(\Gamma, s: A\vbar\gamma')(b) = \zerob$ or
a $c$ in  ${(f~s) = (g~s),\Lambda}$ such that $\mathcal{M}(\Gamma, s:
A\vbar\gamma')(c) = \oneb$\@.  If we have such a $b$ in $\Sigma$,
by Proposition~\ref{meaning-preservation}, we also have that
$\Mgamma{b} = \zerob$.  A similar argument can be used
if we have such a $c$ in $\Lambda$\@.  If $c$ is $(f~s) = (g~s)$, then
$\Mgamma{f}(z) = \Mgamma{g}(z)$ for every $z$ in $\Mgamma{A}$\@.
By set-theoretic extensionality, this means that $\Mgamma{f}$ and
$\Mgamma{g}$ are identical elements 
of $\Pi{}F$ where $F$ maps $z$ in $\Mgamma{A}$ to an element of
$\mathcal{M}(\Gamma, x : \ttvar~A\vbar \gamma\{x \gets z\})(B)$\@.
Therefore $\Mgamma{f = g} = \oneb$ as desired.

\item [TupExt:]  If there is some $d$ in $\Sigma$ such that by applying the
induction hypothesis to any of the subgoals $\Mgamma{d} = \zerob$,
then the same holds for the conclusion sequent.  Similarly,
if the induction hypothesis on some subgoal yields a $c$ in $\Lambda$
such that $\Mgamma{c} = \oneb$, then the same holds for the
conclusion sequent.  So the remaining case is when, by the
induction hypothesis, 
$\Mgamma{\proj{i}(a)} = \Mgamma{\proj{i}(b)}$ for each $i\in \{1,2\}$.
It is therefore easy to conclude by set-theoretic extensionality that
$\Mgamma{a}$ and $\Mgamma{b}$ are identical elements of
$\Mgamma{a/\tupletype{\listwo{T}{n}}}$\@.  We can then use
Proposition~\ref{substitution-equivalence} to conclude
that $\Mgamma{a}$ and $\Mgamma{b}$ are identical
elements of $\Mgamma{\tupletype{\listwo{T}{n}}}$\@. 

\end{description}

\item \emph{Type Constraint Rule}: Recall from
Proposition~\ref{maximal-subtype}
that when $\tauGamma{a} = A$, then
$\Mgamma{\pi(A)(a)} = \oneb$.  Given this and the induction hypothesis,
it must either be the case that we have a $b$ in $\Sigma$ such that
$\Mgamma{b} = \zerob$ or a $c$ in $\Lambda$ such that $\Mgamma{c} =
\oneb$\@. 
\end{enumerate}
\end{proof}

To tie the development so far into a single simultaneous
induction as promised, we state the key theorem whose subproofs have been
given by the theorems presented thus far, namely,
Theorems~\ref{dependent-type-semantics},~\ref{dependent-term-semantics},
and~\ref{proof-soundness}\@.  
\begin{theorem}\label{big-soundness}
If $\tau()(\Gamma) = \ttcontext$, then 
\begin{enumerate}
\item If $\Sigma, \Lambda$ is a list of preterms such that for every $a$
in $\Sigma; \Lambda$, $(\tauGamma{a} \sim 
\ttbool)_\Gamma$, and $\sequent{\Sigma}{\Gamma}{\Lambda}$ is provable, then
for any satisfying assignment
$\gamma$ for $\Gamma$, 
either there is a $b$ in $\Sigma$, such that $\Mgamma{b} = \zerob$ or a $c$ in
$\Lambda$, such that $\Mgamma{c} = \oneb$\@.

\item If $A$ is a pretype such that $\tauGamma{A} = \tttype$, then for any
assignment $\gamma$ 
satisfying $\Gamma$, $\Mgamma{A}\in U$\@.

\item If $a$ is a preterm such that $\tauGamma{a} = A$, then for any
assignment $\gamma$ satisfying $\Gamma$, $\Mgamma{a}\in \Mgamma{A}$\@.
\end{enumerate}
\end{theorem}


\section{Summary}

The logical inference rules for the PVS logic have been presented in a
sequent calculus format.  The formal semantics presented in the earlier
chapters is used to establish the soundness of these proof rules.


\chapter{Conclusion}\label{conclusion}

We have presented the syntax and semantics of idealized PVS in several
stages.  In the first stage we introduced the simply typed fragment,
which was then extended with type definitions.  The third such fragment
included subtyping;  the fourth fragment introduced dependent typing.
Finally, we introduced constant definitions and parametric and nonparametric
theories.

The semantic definition was given in a novel, functional style where a
canonical type was assigned to each type correct term.  The interplay between
types and proofs in PVS introduced subtleties and complexities
into the semantic definition.  We can now answer some of the questions
raised in Chapter~\ref{introduction}:
\begin{itemize}
\item {\em What is the semantic core of the language, and what is
just syntactic sugar?}

The semantic core of the language is a typed lambda calculus with
simple function and tuple types, predicate subtypes, dependent types,
parametric theories, and conditional expressions.  Many of the other
features of the PVS language such as records and update expressions
can be explained in terms of the core language.

\item {\em What are the rules for determining whether a given PVS
expression is well typed?  }

The typechecking rules have been presented in terms of the definition of
the $\tau$ operator in Chapters~\ref{simple},  \ref{subtypes},
\ref{dependent}, \ref{theories}, and~\ref{conditionals}\@. 

\item {\em How is subtyping handled, and in particular, how are
proof obligations corresponding to subtypes generated?}

Typechecking an expression $a$ with respect to predicate subtype
constraint $\subtype{x}{T}{p(x)}$
is done by generating the proof obligation $p(a)$ under the logical
context in which $a$ is being typechecked.  This is made precise
in Definitions~\ref{defn:type rules with subtypes} and~\ref{defn:type rule
for conditional expressions}\@.  
Proof obligations are generated when typechecking contexts (for
nonemptiness),  typechecking expressions with respect to expected subtypes, 
and  comparing two types containing subtype expressions for
compatibility.

\item  {\em What is the meaning, in set-theoretic terms, of a PVS expression or
assertion?}

The set-theoretic meaning of well-formed PVS types and expressions
is given by  a meaning function
$\mathcal{M}$ that assigns a set $\Mgamma{T}$ from the universe $U$ to each type
$T$,  and an element $\Mgamma{a}$ of $\Mgamma{T}$ to a given term $a$ of type
$T$\@.

\item {\em Are the type rules sound with respect to the semantics? }

The typechecking function $\tau$ is defined to check contexts,
preterms, and pretypes for type correctness.    The type rules are shown
to be sound with respect to the given semantics in
Theorem~\ref{big-soundness}\@. 

\item {\em Are the proof rules sound with respect to the semantics?}

The proof rules are given in Chapter~\ref{proof theory} in a sequent
calculus format and proved to be sound with respect to the semantics in
Theorem~\ref{big-soundness}\@.  

\item {\em What is the form of dependent typing used by PVS, and
what kinds of type dependencies are disallowed by the language?}

The semantic analysis of dependent typing in Chapter~\ref{dependent}
revealed that type dependencies were constrained to be rank-bounded.  This
is true because the dependencies in dependent typing only constrain the
predicate part of predicate subtypes. Thus, when there is a dependent type
$T(n)$ that depends on a parameter $n$, the meaning of $T(n)$ has a fixed
rank regardless of the meaning assigned to $n$\@.  The PVS language
features used to define dependent types all preserve the rank-boundedness.
Language extensions violating rank-boundedness such as a type dependency
of the form $\funtype{n : nat}{T^n}$ are disallowed.  One can extend the
language with such dependent types, but the semantics would then be
considerably more complicated.

\item {\em What is the meaning of theory-level parametricity, and
what, if any, are the semantic limits on such parameterization?}

The semantics of parametric theories is described in
Chapter~\ref{theories}\@.  In particular, the semantics for parametric
theories is given in terms of rank-preserving maps between the meanings of
the parameters and the meanings of the identifiers declared in the theory.
These maps must be such that the rank of an assignment to a type in a theory
depends only on the ranks of the (meanings of the) type parameters.

\item {\em What language extensions are incompatible with the reference
semantics given here?}

We have already indicated that any language extension, such as an $n$-tuple
type $T^n$, that violates rank-boundedness would be incompatible with the
semantics presented here.
\end{itemize}

This report presents only the core language of PVS.  A more complete
semantic treatment would include arithmetic, recursive constant
definitions, inductive definitions, recursive datatypes, assumptions on
theory parameters, and type judgements.

\paragraph{Acknowledgments. } The advice and encouragement of John Rushby,
Rick Butler, Paul Miner, Pat Lincoln, and Mandayam Srivas are greatly
appreciated, as are the useful expert comments of Peter Dybjer, Mike
Gordon, Doug Howe, and Paul Jackson.  Bruno Dutertre, Paul Miner, and
Harald Rue{\ss} suggested numerous improvements to earlier drafts.

\begin{comment}
What is a specification language?  A programming language is meant to
express an algorithm in such a way that it can be executed on a computing
machine.  For example, a program implementing a sorting algorithm such as
quicksort can be written in a programming language like C.  Such a program
is a recipe for computing the desired outcome but does not directly
describe this outcome.  In the case of a sorting program, the desired
result is that 
the output array must be an ordered permutation of the input array.
A specification language is the medium for expressing and establishing
such properties of programs.  A specification language could itself be a
programming language but non-executable features such as quantifiers and
temporal operators  add to the expressiveness of a specification language.
%
Specification languages need to capture mathematics
as well as programming.  The use of an expressive type system lends
considerable mathematical clarity to a specification.  
ProVeS is an environment for
specification  and verification based on a typed higher-order logic.
The type system of ProVeS includes
{\em predicate subtypes}, i.e.,
the type consisting of elements of a given type satisfying a specified
predicate, in addition to the usual constructs for
function, product, record, and array types.  Automated theorem proving
procedures are employed in the process of typechecking.
The design and rules
for the type system and logic of ProVeS are presented along
with arguments for their soundness.
\end{abstract}
%
% Document Type: LaTeX
% Master File: intro.tex

\chapter{Introduction}

PVS is a \emph{P}rototype \emph{V}erification \emph{S}ystem for the
development and analysis of formal specifications.  The PVS system
consists of a specification language, a parser, a typechecker, a prover,
specification libraries, and various browsing tools.  This document
primarily describes the specification language and is meant to be used as
a reference manual.  The \emph{PVS System Guide}~\cite{PVS:userguide} is to
be consulted for information on how to use the system to develop
specifications and proofs.  The \emph{PVS Prover Guide}~\cite{PVS:prover}
is a reference manual for the commands used to construct proofs.

In this section, we provide a brief summary of the PVS specification
language, enumerate the key design principles behind the language, and
provide a brief example.

The following sections provide more details on the various features of
the language.  The lexical aspects of the language are detailed in
Section~\ref{lexical}.  Section~\ref{declarations} describes
declarations, Section~\ref{types} describes type expressions,
Section~\ref{expressions} describes expressions, and
Section~\ref{theories} describes theories, theory parameters, and
imports and exports of names.  Section~\ref{names} describes names and
name resolution, and Section~\ref{adts} describes the datatype
facility of PVS.  Finally, 
Appendix~\ref{grammar} provides the grammar of the language.

\section{Summary of the PVS Language}

A PVS specification consists of a collection of \emph{theories}.
Each theory consists of a \emph{signature} for the type names and
constants introduced in the theory, and the axioms, definitions, and
theorems associated with the signature.  For example, a typical
specification for a queue would introduce the \texttt{queue} type and the
operations of \texttt{enq}, \texttt{deq}, and \texttt{front} with their
associated types.  In such a theory, one can either define a
representation for the \texttt{queue} type and its associated operations in
terms of some more primitive types and operations, or merely axiomatize
their properties.  A theory can build on other theories: for example, a
theory for ordered binary trees could build on the theory for
binary trees.  A theory can be \emph{parametric} in certain specified
types and values: as examples, a theory of queues can be parametric in
the maximum queue length, and a theory of ordered binary trees can be
parametric in the element type as well as the ordering relation.  It is
possible to place constraints, called \emph{assumptions}, on the
parameters of a theory so that, for instance, the ordering relation
parameter of an ordered binary tree can be constrained to be a total
ordering.

The PVS specification language is based on simply typed higher-order
logic.  Within a theory, \emph{types} can be defined starting from
\emph{base} types (Booleans, numbers, etc.) using the function, record,
and tuple type constructions.  The \emph{terms} of the language can be
constructed using function application, lambda abstraction, and record and
tuple construction.

There are a few significant enhancements to the simply typed language
above that lend considerable power and sophistication to PVS.  New
uninterpreted base types may be introduced.  One can define a
\emph{predicate subtype} of a given type as the subset of individuals in a
type satisfying a given predicate: the subtype of nonzero reals is
written as \texttt{\{x:real | x /= 0\}}.  One benefit of such
subtyping is that when an operation is not defined on all the elements of
a type, the signature can directly reflect this.  For example, the
division operation on reals is given a type where the denominator is
constrained to be nonzero.  Typechecking then ensures that
division is never applied to a zero denominator.  Since the predicate used
in defining a predicate subtype is arbitrary, typechecking is undecidable
and may lead to proof obligations called \emph{type correctness
conditions} (TCCs).  The user is expected to discharge these proof
obligations with the assistance of the PVS prover.  The PVS type system
also features dependent function, record, and tuple type constructions.
There is also a facility for defining a certain class of abstract datatype
(namely well-founded trees) theories automatically.

\section{PVS Language Design Principles}

There are several basic principles that have motivated the design of
PVS which are explicated in this section.

\paragraph{Specification vs. Programming Languages.}
A specification represents requirements or a design whereas a program
text represents an implementation of a design.  A program can be seen as
a specification, but a specification need not be a program.  Typically,
a specification expresses \emph{what} is being
computed whereas a program expresses \emph{how} it is computed.  A
specification can be incomplete and still be meaningful whereas an
incomplete program will typically not be executable.  A specification
need not be executable; it may use high-level constructs, quantifiers
and the like, that need have no computational meaning.  However, there
are a number of aspects of programming languages that a specification
language should include, such as:
\begin{itemize}

\item the usual basic types: booleans, integers, and rational numbers

\item the familiar datatypes of programming languages such as arrays,
records, lists, sequences, and abstract datatypes

\item the higher-order capabilities provided by modern functional
programming languages so that extremely general-purpose operations can
be defined

\item definition by recursion

\item support for dividing large specifications into parameterized
modules

\end{itemize}

It is clearly not enough to say that a specification language shares some
important features of a programming language but need not be executable.
Any useful formal language must have a clearly defined
semantics\footnote{The PVS semantics are presented in a technical
report~\cite{PVS:semantics}.} and must be capable of being manipulated in
ways that are meaningful relative to the semantics.  A programming
language for example can be given a denotational semantics so that the
execution of the program respects its denotational meaning.  The reason
one writes a specification in a formal language is typically to ensure
that it is sensible, to derive some useful consequences from it, and to
demonstrate that one specification implements another.  All of these
activities require the notion of a justification or a proof based on the
specification, a notion that can only be captured meaningfully within the
framework of logic.

\paragraph{Untyped set theory versus higher-order logic}
\index{set theory}\index{higher-order logic}
Which logic should be chosen?  There is a wide variety of choices:
simple propositional logics, which can be classical or intuitionistic,
equational logics, quantificational logics, modal and temporal logics,
set theory, higher-order logic, etc.  Some propositional and modal
logics are appropriate for dealing with finite state machines where one
is primarily interested in efficiently deciding certain finite state
machine properties.  For a general purpose specification language,
however, only a set theory or a higher-order logic would provide the
needed expressiveness.  Higher-order logic requires strict typing to
avoid inconsistencies whereas set theory restricts the rules for forming
sets.  Set theory is inherently untyped, and grafting a typechecker onto
a language based on set theory is likely to be too strict and arbitrary.
Typechecking, however, is an extremely important and easy way of
checking whether a specification makes semantic sense (although 
for an opposing view, the reader is referred to a report by Lamport
and Paulson~\cite{Lamport&Paulson97}).  Higher-order
logic does admit effective typechecking but at the expense of an
inflexible type system.  Recent advances in type theory have made it
possible to design more flexible type systems for higher-order logic
without losing the benefits of typechecking.  We have therefore chosen
to base PVS on higher-order logic.

\paragraph{Total versus partial functions}
\index{function!total}\index{function!partial} In the PVS higher-order
logic, an individual is either a function, a tuple, a record, or the
member of a base type.  Functions are extremely important in higher-order
logic.  They are \emph{first-class} individuals, i.e., variables can range
over functions.  In general, functions can represent either \emph{total}
or \emph{partial} maps.  A total map from domain $A$ to range $B$ maps
each element of $A$ to some element of $B$, whereas a partial map only
maps some of the elements of $A$ to elements of $B$.  Most traditional
logics build in the assumption that functions represent total maps.
Partial functions arise quite naturally in specifications.  For example,
the division operation is undefined on a zero denominator and the
operation of popping a stack is undefined on an empty stack.

Some recent logics, notably those of VDM~\cite{Jones:VDM},
LUTINS~\cite{Farmer:functions}, RAISE~\cite{RAISE-tutorial},
Beeson~\cite{Beeson:book} and Scott~\cite{Scott79}, admit partial
functions.  In these logics, some terms may be \emph{undefined} by not
denoting any individuals.  Some of these logics have mechanisms for
distinguishing defined and undefined terms, while others allow
``undefined'' to propagate from terms to expressions and therefore must
employ multiple truth values.  In all these cases, the ability to
formalize partially defined functions comes at the cost of complicating
the deductive apparatus, even when the specification does not involve any
partial functions.  Though logics that allow partial functions are
extremely interesting, we have chosen to avoid partial functions in PVS.
We have instead employed the notion of a \emph{predicate
subtype}\index{predicate subtype}, a type that consists of those elements
of a given type satisfying a given predicate.  Using predicate subtypes,
the type of the division operator, for example, can be constrained to
admit only nonzero denominators.  Division then becomes a total operation
on the domain consisting of arbitrary numerators and nonzero denominators.
The domain of a \emph{pop} operation on stacks can be similarly restricted
to nonempty stacks.  PVS thus admits partial functions within the
framework of a logic of total functions by enriching the type system to
include predicate subtypes.  We find this use of predicate subtypes to be
significantly in tune with conventional mathematical practice of being
explicit about the domain over which a function is defined.

\section{An Example: \texttt{stacks}}\label{stacks-example}
\index{stacks example@\texttt{stacks} example}

In this section we discuss a specific example, the theory of
\texttt{stacks}, in order to give a feel for the various aspects of the
PVS language before going into detail.  Apart from the basic notation for
defining a theory, this example illustrates the use of type parameters at
the theory level, the general format of declarations, the use of predicate
subtyping to define the type of nonempty stacks, and the generation of
typechecking obligations.

\pvstheory{stacks-alltt}{{Theory \texttt{stacks}}}{stacks-alltt}

Figure~\ref{stacks-alltt} illustrates a theory for stacks of an arbitrary
type with corresponding stack operations.  Note that this is not the
recommended approach to specifying stacks; a more convenient and complete
specification is provided in Section~\ref{stacks-adt},
page~\pageref{stacks-adt}.

The first line introduces a theory named \texttt{stacks} that is
parameterized by a type \texttt{t} (the \emph{formal parameter} of
\texttt{stacks}).  The keyword \texttt{TYPE+} indicates that \texttt{t} is
a \emph{non-empty} type.  The uninterpreted (nonempty) type \texttt{stack}
is declared, and the constant \texttt{empty} and variable \texttt{s} are
declared to be of type \texttt{stack}.  The defined predicate
\texttt{nonemptystack?}~is then declared on elements of type
\texttt{stack}; it is \texttt{true} for a given \texttt{stack} element
iff\footnote{Iff is short for ``if and only if''.} that element is not
equal to \texttt{empty}.  The functions \texttt{push}, \texttt{pop}, and
\texttt{top} are then declared.  Note that the predicate
\texttt{nonemptystack?}~is being used as a type in specifying the
signatures of these functions; any predicate may be used as a type simply
by putting parentheses around it.

The variables \texttt{x} and \texttt{y} are then declared, followed by the
usual axioms for \texttt{push}, \texttt{pop}, and \texttt{top}, which make
\texttt{push} a stack constructor and \texttt{pop} and \texttt{top} stack
accessors.  Finally, there is the theorem \texttt{pop2push2}, that can
easily be proved by two applications of the \texttt{pop\_push} axiom.

This simple theorem has an additional facet that shows up during
typechecking.  Note that \texttt{pop} expects an element of type
\texttt{(nonemptystack?)} and returns a value of type \texttt{stack}.
This works fine for the inner \texttt{pop} because it is applied to
\texttt{push}, which returns an element of type \texttt{(nonemptystack?)};
but the outer occurrence of \texttt{pop} cannot be seen to be type correct
by such syntactic means.  In cases like these, where a subtype is expected
but not directly provided, the system generates a \emph{type-correctness
condition} (TCC).  In this case, the TCC is
\begin{pvsex}
  pop2push2_TCC1: OBLIGATION
    (FORALL (s: stack, y: t, x: t):
      nonemptystack?(pop(push(x, push(y, s)))))
\end{pvsex}
and is easily proved using the \texttt{pop\_push} axiom.  The system keeps
track of all such obligations and will flag the unproved ones during proof
chain analysis.

Parameterized theories such as \texttt{stacks} introduce theory schemas,
where the type \texttt{t} may be instantiated with any other nonempty
type.  To use the types, constants, and formulas of the \texttt{stacks}
theory from another theory, the \texttt{stacks} theory must be imported,
with \emph{actual parameters} provided for the corresponding theory
parameters.  This is done by means of an \texttt{IMPORTING} clause. For
example, the theory
\begin{pvsex}
  ustacks : THEORY
    BEGIN
    IMPORTING stacks[int], stacks[stack[int]]

    si : stack[int]
    sos : stack[stack[int]] = push(si, empty)
    END ustacks
\end{pvsex}
imports stacks of integers and stacks of stacks of integers.  The constant
\texttt{si} is then declared to be a stack of integers, and the constant
\texttt{sos} is a stack of stacks of integers whose top element is
\texttt{si}.  Note that the system is able to determine which instance of
\texttt{push} and \texttt{empty} is meant from the type of the first
argument.  In general, the typechecker infers the type of an expression
from its context.  


\include{rules}
\include{logic}
% Document Type: LaTeX
% Master File: semantics.tex
\documentclass [12pt,twoside]{cslreport} 
\usepackage{fleqn,alltt,spacecites,relsize,url,amssymb,verbatim}
%\usepackage{doublespace}
%\setlength{\textwidth}{6.5in}
%\setlength{\textheight}{9in}
%\setlength{\topmargin}{-.6in}
%\setlength{\oddsidemargin}{0in}
%\setlength{\evensidemargin}{0in}
\setlength{\textwidth}{5.5in}
\setlength{\textheight}{8in}
\setlength{\topmargin}{.25in}
\setlength{\oddsidemargin}{.5in}
\setlength{\evensidemargin}{0in}
\setlength{\mathindent}{0.2in}
%\input amssym.def
%\input amssym
%\input commands
%\includeonly{soundness}
%%
%
%    LaTeX  Macro File  /usr2/jcm/tex/macros.tex
%
%
\tracingonline=0 % shorter error messages (on screen)

%\long\def\comment#1{} % mulit-line comments
%\newcommand{\note}[1]{\fbox{#1}}

% font changes  (as function calls, scribe style)

\renewcommand{\i}[1]{{\it #1\/}}       % italics with space correction
%\newcommand{\emex}[1]{\/{\em #1}}   % emphasis in example, exercise, theorem, etc.
\renewcommand{\c}[1]{{\sc #1}}       % small caps (eliminates \c as cedilla)
%\newcommand{\r}[1]{{\rm #1}}            % for roman font in math mode
%\newcommand{\s}[1]{{\scr #1}}          % script (for use with tatex) 
\newcommand{\s}[1]{{\cal #1}}
\renewcommand{\b}[1]{{\bf #1}}           % bold face
\newcommand{\calg}[1]{{\cal #1}}          % caligraphic 

%\newcommand{\q}[1]{``#1''}    % matching quotes for in-line quotation

% numbered environments

%\newcounter{partcounter}
%\setcounter{partcounter}{0}
%\renewcommand{\part}[1]{\newpage \addtocounter{partcounter}{1}
%\noindent{\Large \bf Part \Roman{partcounter}. #1 } \\[1ex]}

\newtheorem{thm}{Theorem}[section]
\newtheorem{theorem}[thm]{Theorem}
\newtheorem{lemma}[thm]{Lemma}
\newtheorem{cor}[thm]{Corollary}
\newtheorem{corollary}[thm]{Corollary}
\newtheorem{claim}[thm]{Claim}
\newtheorem{prop}[thm]{Proposition}
\newtheorem{conj}[thm]{Conjecture}
\newtheorem{definition}[thm]{Definition}
\newtheorem{exercise}[thm]{Exercise}
\newtheorem{example}[thm]{Example}
\newtheorem{remark}[thm]{Remark}
\newtheorem{open}[thm]{Open Problem}

%\newcommand{\proof}{\\{\bf Proof.}\ }
\newenvironment{proof}{{\bf Proof. }}{\thmbox}

% axiom and inference rule (centered, in math mode,  with name at left)

\newcommand{\axiom}[2]
{\[ \hbox to \columnwidth
    { \rlap{$#2$} \hfil {$ #1 $} \hfil }
\]}

\newcommand{\infrule}[4]
{\[ \hbox to \columnwidth %\textwidth
    { \rlap{$#4$} \hfil $
      \frac {\strut\displaystyle #1 } {\strut\displaystyle #2 } \; \rlap{$#3$} \hfil $
      \hfil }
\]}
\newcommand{\infruletw}[4]
{\[ \hbox to \textwidth
    { \rlap{$#4$} \hfil $
      \frac {\strut\displaystyle #1 } {\strut\displaystyle #2 } \; \rlap{$#3$} \hfil $
      \hfil }
\]}

%\newcommand{\infrule}[4]
%{\[ \hbox to \columnwidth
%    { \rlap{$#4$} \hfil $
%      {{\displaystyle\strut #1}\over{\displaystyle\strut #2}}\quad\makebox[0pt][l]{\it #3} $
%      \hfil }
%\]}

% axiom and inference rule macros for use in tables, etc.
% presumed in math mode #1=axiom, #2=side condition
\newcommand{\Axiom}[2]{
{\displaystyle\strut #1}\qquad\makebox[0pt][l]{\it #2}
}
% presumed in math mode #1=top, #2=bottom, #3=side condition
\newcommand{\Infrule}[3]{
{{\displaystyle\strut #1}\over{\displaystyle\strut #2}}\;\mbox{\scriptsize$\bf #3$}
}

% sequence of #1's, numbered up to #2  (e.g., seq{x}{n} for x1, ..., xn   )

\newcommand{\seq}[2]{#1_{1} \ldots #1_{#2}}  
%\newcommand{\ie}{{\it i.e.}}
%\newcommand{\eg}{{\it e.g.}}
%\newcommand{\cf}{{\it c.f.\,}}

% common symbols

\newcommand{\union}{\cup}
\newcommand{\intersect}{\cap}
\newcommand{\subs}{\subseteq}
\newcommand{\el}{\in}
\newcommand{\nel}{\not\in}
\newcommand{\ns}{\emptyset}
\newcommand{\compose}{\circ}
\newcommand{\set}[2]{ \{\, #1 \,\mid\, #2 \,\}  } % set macro
\newcommand{\infinity}{\infty}
\newcommand{\pair}[1]{\langle #1 \rangle}
\newcommand{\tuple}[1]{\langle #1 \rangle}


\newcommand{\fa}{\forall}
\newcommand{\te}{\exists}
\newcommand{\imp}{\supset}
\renewcommand{\implies}{\supset}
\newcommand{\ts}{\vdash}
\newcommand{\dts}{\models}

\newcommand{\aro}{\mathord\rightarrow} % see pages 154-155 of TeX manual
\newcommand{\paro}{\rightharpoonup} 
\newcommand{\karo}{\mathop\Rightarrow} % see pages 154-155 of TeX manual
%\newcommand{\cross}{\times}
%\newcommand{\dlb}{\lbrack\!\lbrack}
%\newcommand{\drb}{\rbrack\!\rbrack}
\newcommand{\mean}[1]{\lbrack\!\lbrack #1 \rbrack\!\rbrack}
\newcommand{\lam}{\lambda}
\newcommand{\subst}[2]{{}[#1/#2]}
\renewcommand{\dot}{\mathrel{\bullet}}
\newcommand{\Dinf}{D_{\infty}}
\newcommand{\bottom}{\perp}

\mathcode`:="603A  % treat : as punctuation instead of relation in math mode
\mathchardef\colon="303A	% relation colon

\newcommand{\eqdef}{\mathrel{:=}}
\newcommand{\Dom}{\mathop{\rm dom}}
\newcommand{\Pow}{\mathop{\rm Pow}}

\newcommand{\aequiv}{\equiv_\alpha}
\newcommand{\baro}{\buildrel \beta \over \rightarrow}
\newcommand{\earo}{\buildrel \eta \over \rightarrow}
\newcommand{\red}{\rightarrow\!\!\!\!\rightarrow}
\newcommand{\backred}{\leftarrow\!\!\!\!\leftarrow}
\newcommand{\bred}{\buildrel \beta \over \red}
\newcommand{\ered}{\buildrel \eta \over \red}
\newcommand{\conv}{\leftrightarrow}
\newcommand{\beconv}{\buildrel {\beta, \eta} \over\leftrightarrow}

% lambda calculus abbreviations
\newcommand{\letdec}[3]{\b{let\ } #1 = #2 \b{\ in\ } #3}
\newcommand{\letrec}[3]{\b{letrec\ } #1 = #2 \b{\ in\ } #3}

\newcommand{\thmbox}
   {{\ \hfill\hbox{%
      \vrule width1.0ex height1.0ex
   }\parfillskip 0pt }}
\newcommand{\qed}{\thmbox}


% make single spacing

\newcommand{\singlespace}{\renewcommand{\baselinestretch}{1}\@normalsize}
\newcommand{\etal}{{\em et al.}}

% bycase command for definition by cases (pg 49 of LaTeX)

\newcommand{\bycase}[1]
	{\left\{ \begin{array}{ll}  #1  \end{array} \right. }


% make \cite put blanks after the comma (use in alpha style)

%\def\@citex[#1]#2{\if@filesw\immediate\write\@auxout{\string\citation{#2}}\fi
%  \def\@citea{}\@cite{\@for\@citeb:=#2\do
%    {\@citea\def\@citea{,\penalty100\hskip2.5pt plus1.5pt minus.8pt}%
%       \@ifundefined{b@\@citeb}{{\bf ?}\@warning
%       {Citation `\@citeb' on page \thepage \space undefined}}
%\hbox{\csname b@\@citeb\endcsname}}}{#1}}

\newcommand{\thmbox}
   {{\ \hfill\hbox{%
      \vrule width1.0ex height1.0ex
   }\parfillskip 0pt }}
\newtheorem{thm}{Theorem}[chapter]
\newtheorem{prop}[thm]{Proposition}
\newenvironment{proof}{{\bf Proof. }}{\thmbox}
\newcommand{\Infrule}[3]{
{{\displaystyle\strut #1}\over{\displaystyle\strut #2}}\qquad\makebox[0pt][l]{\it #3}
}
\newcommand{\tuple}[1]{\langle #1 \rangle}
\newtheorem{example}[thm]{Example}
\newcommand{\aro}{\mathord\rightarrow} % see pages 154-155 of TeX manual
\newcommand{\mean}[1]{\lbrack\!\lbrack #1 \rbrack\!\rbrack}
\newcommand{\pair}[1]{\langle #1 \rangle}
\newcommand{\union}{\cup}
\newcommand{\funtype}[2]{[#1 \aro #2]}
\newcommand{\tupletype}[1]{[#1]}
\newcommand{\subtype}[3]{\{#1 : #2 | #3 \}}
\newcommand{\tauGamma}[1]{\tau(\Gamma)(#1)}
\newcommand{\Mgamma}[1]{{\mathcal M}(\Gamma\vbar\gamma)(#1)}
\newcommand{\proj}[1]{\mathtt{p}_{#1}}
\newcommand{\lis}[2]{#1_{1}, \ldots, #1_{#2}}
\newcommand{\listwo}[2]{#1_{1}, #1_{2}}
\newcommand{\ol}[1]{\overline{#1}}
%\newcommand{\ul}[1]{\underline{#1}}
\newcommand{\twob}{\mathbf{2}}
\newcommand{\oneb}{\mathbf{1}}
\newcommand{\zerob}{\mathbf{0}}
\newcommand{\reals}{\mathbf{R}}
\newcommand{\ttbool}{\mathtt{bool}}
\newcommand{\ttreal}{\mathtt{real}}
\newcommand{\tttrue}{\mathtt{TRUE}}
\newcommand{\ttfalse}{\mathtt{FALSE}}
\newcommand{\tttype}{\mathtt{TYPE}}
\newcommand{\ttcontext}{\mathtt{CONTEXT}}
\newcommand{\ttvar}{\mathtt{VAR}}
\newcommand{\ttconstant}{\mathtt{CONSTANT}}
\newcommand{\ttvariable}{\mathtt{VARIABLE}}
\newcommand{\tttheory}{\mathtt{THEORY}}
\newcommand{\ttif}{\mathtt{IF}}
\newcommand{\ttint}{\mathtt{int}}
\newcommand{\ttnat}{\mathtt{nat}}
\newcommand{\itkind}{\textit{kind}}
\newcommand{\ittype}{\textit{type}}
\newcommand{\itdef}{\textit{definition}}
\newcommand{\itformals}{\textit{formals}}
\newcommand{\vbar}{\ |\ }
\newcommand{\sima}{\stackrel{a}{\sim}} %/{\scriptsize{a}}
\newenvironment{Eg}[1]{\begin{example}[#1]\label{eg:#1}\em }{\thmbox\end{example}}
\newtheorem{definition}{Definition}
\newtheorem{lemma}{Lemma}
\newtheorem{theorem}{Theorem}
%% Derived from John Rushby's prelude.tex, modified for NFSS2
%
% define variants of the \LaTeX macro that avoid using \sc
% for use in headings
%

% Define fonts that work in math or text mode
\def\dwimrm#1{\ifmmode\mathrm{#1}\else\textrm{#1}\fi}
\def\dwimsf#1{\ifmmode\mathsf{#1}\else\textsf{#1}\fi}
\def\dwimtt#1{\ifmmode\mathtt{#1}\else\texttt{#1}\fi}
\def\dwimbf#1{\ifmmode\mathbf{#1}\else\textbf{#1}\fi}
\def\dwimit#1{\ifmmode\mathit{#1}\else\textit{#1}\fi}
\def\dwimnormal#1{\ifmmode\mathnormal{#1}\else\textnormal{#1}\fi}

\def\BigLaTeX{{\rm L\kern-.36em\raise.3ex\hbox{\small\small A}\kern-.15em
    T\kern-.1667em\lower.7ex\hbox{E}\kern-.125emX}}
\def\BoldLaTeX{{\bf L\kern-.36em\raise.3ex\hbox{\small\small\bf A}\kern-.15em
    T\kern-.1667em\lower.7ex\hbox{E}\kern-.125emX}}
%\def\labelitemi{$\bullet$}
\def\labelitemii{$\circ$}
\def\labelitemiii{$\star$}
\def\labelitemiv{$\diamond$}
\newcommand{\tcc}{{\small\small TCC}}
\newcommand{\tccs}{\tcc s}
\newcommand{\emacs}{{Emacs}}
\newcommand{\Emacs}{\emacs}
\newcommand{\ehdm}{{E{\small\small HDM}}}
\newcommand{\Ehdm}{\ehdm}
\newcommand{\tm}{$^{\mbox{\tiny TM}}$}
\newcommand{\hozline}{{\noindent\rule{\textwidth}{0.4mm}}}

\newcommand{\allclear}%
  {\mbox{\boldmath$\stackrel{\raisebox{-.2ex}[0pt][0pt]%
              {$\textstyle\oslash$}}{\displaystyle\bot}$}}

\newenvironment{private}{}{}

\newenvironment{smalltt}{\begin{alltt}\small}{\end{alltt}}

\newlength{\hsbw}

\newenvironment{session}%
  {\begin{flushleft}
   \setlength{\hsbw}{\linewidth}
   \addtolength{\hsbw}{-\arrayrulewidth}
   \addtolength{\hsbw}{-\tabcolsep}
   \begin{tabular}{@{}|c@{}|@{}}\hline 
   \begin{minipage}[b]{\hsbw}
   \begingroup\small\mbox{ }\\[-1.8\baselineskip]\begin{alltt}}
  {\end{alltt}\endgroup\end{minipage}\\ \hline 
   \end{tabular}
   \end{flushleft}}

\newenvironment{smallsession}%
  {\begin{flushleft}
   \setlength{\hsbw}{\linewidth}
   \addtolength{\hsbw}{-\arrayrulewidth}
   \addtolength{\hsbw}{-\tabcolsep}
   \begin{tabular}{@{}|c@{}|@{}}\hline 
   \begin{minipage}[b]{\hsbw}
   \begingroup\footnotesize\mbox{ }\\[-1.8\baselineskip]\begin{alltt}}%
  {\end{alltt}\endgroup\end{minipage}\\ \hline 
   \end{tabular}
   \end{flushleft}}

\newenvironment{spec}%
  {\begin{flushleft}
   \setlength{\hsbw}{\textwidth}
   \addtolength{\hsbw}{-\arrayrulewidth}
   \addtolength{\hsbw}{-\tabcolsep}
   \begin{tabular}{@{}|c@{}|@{}}\hline 
   \begin{minipage}[b]{\hsbw}
   \begingroup\small\mbox{ }\\[-0.2\baselineskip]}%
  {\endgroup\end{minipage}\\ \hline 
   \end{tabular}
   \end{flushleft}}

\newcommand{\memo}[1]%
  {\mbox{}\par\vspace{0.25in}%
   \setlength{\hsbw}{\linewidth}\addtolength{\hsbw}{-1.5ex}%
   \noindent\fbox{\parbox{\hsbw}{{\bf Memo: }#1}}\vspace{0.25in}}

\newcommand{\nb}[1]%
  {\mbox{}\par\vspace{0.25in}%
   \setlength{\hsbw}{\linewidth}\addtolength{\hsbw}{-1.5ex}%
   \noindent\fbox{\parbox{\hsbw}{{\bf Note: }#1}}\vspace{0.25in}}

\newcommand{\comment}[1]{}
\newcommand{\exfootnote}[1]{}
%\newcommand{\ifelse}[2]{#1}
\sloppy
\clubpenalty=100000
\widowpenalty=100000
%\displaywidowpenalty=100000
\setcounter{secnumdepth}{3} 
\setcounter{tocdepth}{3}
\setcounter{topnumber}{9}
\setcounter{bottomnumber}{9}
\setcounter{totalnumber}{9}
\renewcommand{\topfraction}{.99}
\renewcommand{\bottomfraction}{.99}
\renewcommand{\floatpagefraction}{.01}
\renewcommand{\textfraction}{.2}
\font\largett=cmtt10 scaled\magstep1
\font\Largett=cmtt10 scaled\magstep2
\font\hugett=cmtt10 scaled\magstep3

%\input{/homes/shankar/tex/ch-jmacros}

\newenvironment{Defn}[1]{\begin{definition}[#1]\label{defn:#1}}{
\thmbox\end{definition}}
%
\newenvironment{display}{\begin{alltt}\small\tt\vspace{0.3\baselineskip}}{\vspace{0.3\baselineskip}\end{alltt}}
%
%\newcommand{\gets}{\leftarrow}
\title{The Formal Semantics of PVS\thanks{Funded by National Aeronautics
and Space Administration Contract NAS1-18969, Task 11 and National Science
Foundation Grant CCR 9300444. 
}}
\author{
\begin{tabular}{ccc}
Sam Owre & and & Natarajan Shankar\vspace{-.05in}\\
{\small\tt owre@csl.sri.com}& & {\small\tt shankar@csl.sri.com}\\
\end{tabular}\\
URL: \url{http://www.csl.sri.com/sri-csl-fm.html}\\
SRI International\\
Computer Science Laboratory\\
Menlo Park CA 94025 USA
}
\date{Technical Report CSL-97-2R\\August 1997, Revised March 1999}
%\renewcommand{\today}{}
%\renewcommand{\is}{::=}
%\renewcommand{\baselinestretch}{2}
\begin{document}
%\bibliographystyle{alpha}
\pagenumbering{roman}
\maketitle
\setcounter{page}{2}
\mbox{}
\cleardoublepage
\begin{abstract}
\setcounter{page}{3} 
\thispagestyle{plain}

A specification language is a medium for expressing {\em what\/} is
computed rather than {\em how\/} it is computed.  Specification languages
share some features with programming languages but are also different in
several important ways.  For our purpose, a specification language is a
logic within which the behavior of computational systems can be
formalized.  Although a specification can be used to simulate the behavior
of such systems, we mainly use specifications to state and prove system
properties with mechanical assistance.

We present the formal semantics of the specification language of SRI's
Prototype Verification System (PVS)\@.  This specification language is
based on the simply typed lambda calculus.  The novelty in PVS is that it
contains very expressive language features whose static analysis (e.g.,
typechecking) requires the assistance of a theorem prover.  The formal
semantics illuminates several of the design considerations underlying PVS,
particularly the interaction between theorem proving and typechecking.
\end{abstract}
\cleardoublepage
\tableofcontents
%\listwotoffigures
\cleardoublepage
\setcounter{page}{0} 
\pagestyle{headings}
\pagenumbering{arabic}

\chapter{Introduction} \label{introduction} PVS is a system for specifying
and verifying properties of digital hardware and software systems.  The
specification language of PVS is designed to admit succinct, readable, and
logically meaningful specifications.  The PVS specification language is
designed for effective proof construction rather than efficient execution.
The design considerations underlying the language are therefore somewhat
different from those of a corresponding programming language.  For
example, the language contains constructs that can be statically
typechecked only with the assistance of a theorem prover.  This is
acceptable because the PVS specification language is intended for use in
conjunction with powerful support for automated theorem proving.  The
logic of PVS is based on a simply typed higher-order logic with function,
record, and product types, and recursive type definitions.  This type
system is extended with subtypes that are analogous to subsets, and with
dependently typed functions, records, and products.  The resulting type
system has several advantages.  It is possible, for instance, to
statically ensure that all array references are within their respective
array bounds.  PVS specifications are organized into theories that can be
parametric in types as well as individuals.  While the semantics of the
simply typed fragment is straightforward, the extensions such as
subtyping, dependent typing, and (theory-level) parametricity do pose
significant challenges.  This report presents a concise but idealized
definition of the PVS specification language and its intended formal
set-theoretic semantics.  It is neither an overview of the PVS language
nor a guide to the Prototype Verification System (see the PVS user
manuals~\cite{PVS:manuals}).


The primary purpose of the formal semantics is as a useful reference for
the developers and users of PVS\@.  The idealized core of the
specification language as presented here serves as a succinct foundation
for studying the expressive power of the language.  Pertinent questions
about PVS are answered directly by the formal semantics presented here:
\begin{enumerate}
\item\label{one} What is the semantic core of the language, and what is
just syntactic sugar?

\item\label{two} What are the rules for determining whether a given PVS
expression is well typed?

\item\label{three} How is subtyping handled, and in particular, how are
proof obligations corresponding to subtypes generated?

\item\label{four} What is the meaning, in set-theoretic terms, of a PVS expression or
assertion?

\item\label{five} Are the type rules sound with respect to the semantics? 

\item\label{six} Are the proof rules sound with respect to the semantics?

\item\label{seven} What is the form of dependent typing used by PVS, and
what kinds of type dependencies are disallowed by the language?

\item\label{eight} What is the meaning of theory-level parametricity, and
what, if any, are the semantic limits on such parameterization?

\item\label{nine} What language extensions are incompatible with the reference
semantics given here?
\end{enumerate}

Chapter~\ref{conclusion} summarizes the answers to these questions.


\section{Real versus Idealized PVS}

The semantic treatment in this report is incomplete in some important ways.
It does not treat the nonlogical parts of the language.  In particular,
it ignores arithmetic and recursive definitions.  It also omits
abstract datatypes~\cite{Shankar:ADT97}.  These will be treated in a future
expanded version. 

The present semantics also makes several idealizations from the real
PVS for 
the purpose of clarity.  While the semantic treatment is not
comprehensive, the idealization of PVS used here is faithful to the
implemented form of PVS.   

\begin{enumerate}
  \item \emph{No name resolution.}  All names must be in fully resolved form
with their theory name and actual parameters.  We regard name resolution as
a convenience provided by the PVS type checker and not an operation with any
semantic relevance.  A technical description of name resolution in PVS will
be given elsewhere.


  \item \emph{No overloading.}   As with name resolution, overloading is
a syntactic convenience with no semantic import. 

\item \emph{No \texttt{IMPORTING}s.}  The importing of theories is a 
hint to  name resolution.  The semantic definition assumes that all
instances of theories declared prior to the present one are visible.   
  \item \emph{Variable declarations ignored.}  All variables must be 
locally declared.   Global variable declarations are regarded as a syntactic
convenience.  
%  \item TCCs are not kept around to be used as lemmas.
  \item \emph{No records.}  These are ignored in the semantic treatment
since product types capture all the semantically essential features of records.
\end{enumerate}


\section{Semantic Preliminaries}

The PVS specification language is based on higher-order logic.  This means
that variables can range over individuals (such as numbers) as well as
functions, functions of functions, and so on.  As is well known, some type
distinction is needed; otherwise, it is easy to obtain a
contradiction by defining the predicate $N(P)$ as $ \neg(P(P))$ so that
both $N(N)$ and $\neg N(N)$ hold.  
In the theory of types~\cite{Church:types}, the universe is stratified
into distinct types 
so that a predicate can  be applied only to a \emph{lower} type
and thus cannot be applied to itself.

Types also serve as a powerful mechanism for detecting syntactic and
semantic errors through typechecking.  This role of types is best
exemplified by their use in various programming languages such as Algol,
Ada, and ML, and is also heavily emphasized in the PVS type
system.

The desirability for strong typing in a specification
logic is not widely accepted.  Fraenkel {\em et
al\/}~\cite{Fraenkel-etal84} express
the opinion that such 
typing is repugnant in a mathematical logic since it constrains
expressiveness by not allowing individuals of differing types to be
treated uniformly.  Lamport~\cite{TLA:TOPLAS94} argues that type
correctness is like any other 
program property and should be established by means of a proof rather than
by syntactic restraints.  Lamport and Paulson~\cite{Lamport&Paulson97}
analyze the tradeoffs between typed and untyped specification languages.  
We claim that
\begin{enumerate}
\item Types impose a useful discipline on the specification.
\item Types lead to easy and early detection of a large class of
syntactic and  semantic errors.
\item Type information is useful in mechanized reasoning.
\end{enumerate}


\begin{comment}
The first typed higher-order logic was the Principia Mathematica (PM) system of
Russell and Whitehead.  They were the first to propose a stratified
hierarchy of types as a means avoiding certain paradoxes.  The PM type
system is quite complicated.  Church proposed his \emph{simple theory of
types} as a higher-order logic based on a simply typed lambda calculus.
Church's system forms the basis for most modern higher-order logics
including that of PVS.  The Automath system of de~Bruijn pioneered the use
of dependent types (analogous to the variant records of Pascal) as a basis
for intuitionistic logic.\footnote{Similar ideas were 
independently proposed by several other researchers.}  Dependent types 
form an important component of the PVS type system.  
Girard and Reynolds, independently extended the typed lambda calculus
to a second-order or \emph{polymorphic} type system so that, for example,
an identity function for each type could be obtained by specializing a
generic identity function to the required type.
\end{comment}

The semantics of a higher-order logic is given by mapping the well-formed
types of the logic to {\em sets\/}, and the well-formed terms of the logic to
{\em elements\/} of the sets representing their type.  The set
constructions we use can be formalized within Zermelo-Fraenkel set theory
with the axiom of choice (ZFC).   The intended interpretation
of a function type in higher-order logic is that it represents the set of
all functions from the set representing the domain type to the set
representing the range types.\footnote{It is only in the {\em standard\/}
model of higher-order logic that the function type is required to
represent the set of all 
functions from the domain set to the range set.  Higher-order logic can be
interpreted in {\em general models\/} where the function type can be
interpreted in any manner as long as it satisfies the various axioms
such as application, abstraction, and extensionality~\cite{Andrews:book}.
Higher-order logic 
is complete with respect to the general models interpretation so that
a statement that is valid in all models is provable.  It is, however,
incomplete with respect to the standard model.}  
PVS also has predicate subtypes that are to
be interpreted over the subsets of the set representing the parent type.

The semantics of PVS will be given by considering a sequence of
increasingly expressive fragments of PVS\@.   
The semantics of each fragment of PVS will be presented in three steps.
The first step is to define a set-theoretic universe containing
enough sets to represent the PVS types.  The second step is to define a
typechecking operation that determines whether a given PVS expression is
well typed.  The third step is to define a semantic function that assigns
a representation in the semantic universe to each well-typed PVS type and
term. 



\begin{comment}
I am not exactly sure where the axiom of choice is really needed.
Since PVS builds in a choice operator via parametric theories, it seems
that this will be needed.
\end{comment}


We first lay out the ZFC set constructions needed for defining the
semantics of PVS.  The base types in PVS consist of the Booleans
\texttt{bool} and the real numbers \texttt{real}\@.  The Booleans can be
modeled by any two-element set, say $\twob$ consisting of the elements
$\zerob$ and $\oneb$, where $\zerob$ is the empty set and the only element
of the set $\oneb$\@.  The real numbers can be captured by means of
Dedekind cuts or Cauchy sequences, and we label this set $\reals$\@.

To define the semantics, we need a universe that contains the sets
$\twob$ and 
$\reals$
and is closed under Cartesian products (written as $X\times Y$)
and power sets (written as $\wp(X)$)\@.  Note that
functions are modeled as \emph{graphs}, that is, sets of ordered pairs,
so that a function type $\funtype{A}{B}$ is represented by a subset of
the powerset $\wp(\mean{A}\times \mean{B})$
of the Cartesian product of the sets $\mean{A}$ and $\mean{B}$
representing $A$ and 
$B$, respectively\@.  A set $F$ that is a subset of $X \times Y$ is the graph of a
function with domain $X$ and range $Y$ if for every $x\in X$ there is a $y\in
Y$ such that $\pair{x, y}\in F$, and whenever $\pair{x, y}\in F$
and $\pair{x, y'}\in F$, we have $y = y'$\@.  For such a set $F$,
$ \textit{Function}(F)$ holds and $ \textit{dom}(F) = X$\@.  
The set of graphs of total functions from a set $Y$ to a set $X$
is represented as $X^Y$\@.   If $F$ is the graph of a function and $t$ an
element in its domain, then
$F(t)$ represents the result of applying the function $F$ to $t$\@.
At the semantic level,  a function $F$ will never be applied to an argument
$t$ outside $\textit{dom}(F)$, because in the PVS language,
a function application is typechecked so that the
argument expression has the same type as the domain type of the
function expression.  

We can model the entire type universe of the simply typed fragment of PVS
by the set $U$, which is defined cumulatively by starting from the base
sets $\twob$ and $\reals$, and including the Cartesian products, the
function spaces, and subsets of previously included sets, at each stage.
Cartesian products are used to model products in PVS, and function spaces
model function types.  Subsets are needed to model predicate subtypes.
It is sufficient to iterate these stages up to the ordinal $\omega$\@.
\begin{Defn}{type universe}
\begin{eqnarray*}
  U_0 & = & \{ {\twob}, {\reals}  \} \\
  U_{i+1} & = & U_i \,\union\, \{X\times Y \vbar X, Y \in U_i\} \,\union\,
                           \{X^Y \vbar X, Y \in U_i \} \,\union\,
                           \bigcup_{X\in U_i}\wp(X)\\
  U_{\omega} & = & \bigcup_{i\in\omega} U_i\\
  U & = & U_{\omega}
\end{eqnarray*}
\end{Defn}


We refer to $U$ as the \emph{basic universe}.\footnote{The inclusion of $X^Y$ in $U$ is actually redundant but aids clarity.}   The semantic definitions
below will assign a set in $U$ to each PVS type and an element
in $\bigcup U$ to each well-typed term of PVS.  The \emph{rank} of a set
$X$ in $U$ is the least $i$ such that $X\in U_i$\@.  The notion of rank
plays an important role in the semantics of dependent types and
parametric theories.

\section{Related Work}

There is a long history of work in specification languages.
Many  ideas similar to those underlying the PVS specification language
also occur in other specification languages.  

The {\em wide-spectrum\/} languages are typically based on set theory or
higher-order logic.  The language VDM is one of the earliest such
specification formalisms~\cite{Jones:VDM}.  It is based on a first-order
logic with partial functions augmented with datatype axioms.  The datatype
theories in VDM include those for finite sets, maps, sequences, and
recursive datatypes such as lists and trees.  VDM has a notion of datatype
invariants that yields a simple form of predicate subtyping.  Operations
on state are specified in terms of pre-condition/post-condition pairs.
Specifications are structured into parameterized modules.  In contrast to
VDM, the PVS language is based on strictly typed higher-order logic with a
built-in notion of predicate subtyping and dependent typing.  The
resulting PVS logic is more compact in that many of the datatypes that are
presented axiomatically in VDM can be defined within PVS.  There is no
built-in notion of state in PVS since it is possible to use the
higher-order logic of PVS to define a variety of state-based formalisms,
including various linear and branching-time temporal logics.  VDM uses a
3-valued logic for the logical connectives in order to deal
with partial functions, whereas PVS uses a classical 2-valued logic and
predicate subtyping to assign a type to a partial function as a total
function on its domain of definition.  Jones~\cite{Jones:VDM} provides
only an informal semantics for VDM.  The RAISE system is a comprehensive
toolset based on the ideas of VDM~\cite{RAISE:book}.

The Z specification language~\cite{Z-understanding} is another
wide-spectrum language based on a typed first-order set theory.  A Z
specification is a collection of schemas consisting of declarations of
types and constants accompanied with invariants.  Z schemas can either
specify datatype invariants or pre-condition/post-condition constraints.
A schema calculus is used to combine schemas using logical connectives.
Spivey~\cite{Z-understanding} presents a formal semantics for Z without
giving a proof system or a soundness proof.  Spivey's treatment of
partial functions in the Z semantics employs the commonly used convention
that $f(a)$ when $a$ is not in the domain of $a$ is some arbitrarily
chosen value.  This is fine for most purposes but can be confusing when
dealing with recursively defined partial functions.  For example, the
definition $\mathit{bad}(x) = 1 + \mathit{bad}(x)$ is everywhere undefined but
admitting it as an axiom leads to an immediate contradiction.  Z also
lacks any mechanism for conservative extensions such as definitional
principles for constants and datatypes so that the consistency of Z
specification has to be demonstrated by exhibiting a model.

Algebraic specification languages like OBJ~\cite{popl85} and
Larch~\cite{Larch:book} provide an 
equational/rewriting framework for specifying datatypes and operations on
datatypes.   OBJ has many of the same theory parameterization mechanisms
as PVS.  The subsort mechanism in OBJ is also similar except that
it is handled by introducing {\em retracts\/} or runtime checks
rather than proof obligations generated by the type checker.
The OBJ logic is quite restricted compared to PVS since it is based
on a first-order, equational framework with an {\em initial\/} semantics
where two ground terms are distinct unless they can be proved equal.
OBJ has very limited support for proof development and is primarily
intended as an executable specification language.

The specification languages that are closer to PVS are those that
accompany various automated proof checking systems.  The closest of these
is {\sc Ehdm}~\cite{EHDM:manuals}, which employs a similar higher-order
logic with subtyping and proof obligation generation.  {\sc Ehdm} lacks
many of the features of PVS: subtyping is restricted to type declarations
and there is no dependent typing.

Higher-order logic is used by other systems such as
HOL~\cite{Gordon&Melham:HOL} and TPS~\cite{andrews84}.
Both HOL and TPS employ simply typed higher-order logic 
without features such as subtyping, dependent typing, or parametric
theories.  
Andrews~\cite{Andrews:book} gives a thorough account of the semantic
aspects of higher-order logic.  The formal semantics of the HOL logic are
carefully outlined (by Pitts) in the book by Gordon and
Melham~\cite{Gordon&Melham:HOL}.

Systems like Coq~\cite{COQ} and Nuprl~\cite{Nuprl-book} are based on
intuitionistic higher-order logics.  Coq allows quantification over types,
whereas Nuprl has quantification over a hierarchy of type universes.  Both
logics admit dependent typing.  The set-theoretic semantics of dependently
typed intuitionistic type theories has been studied by
Dybjer~\cite{DybjerP:indsfm} and Howe~\cite{Howe91,Howe96}.  Not
surprisingly, their semantic treatment of dependent typing is similar to
the one given here but they do not delimit the possible dependencies as is
done with the PVS semantics.  The PVS semantics presented here clearly
specifies the kind of type dependencies that are disallowed in the logic.
Dybjer and Howe also do not address subtyping but do describe the
semantics of language features missing in PVS (type universes in the case
of Howe, and inductive families in the case of Dybjer).  Dybjer does not
identify the universe over which terms and types are interpreted.  Howe
requires an infinite sequence of inaccessible cardinals for his universe
construction.


\section{Outline}

In Chapter~\ref{simple}, we define the syntax and semantics of the simply
typed fragment of PVS.  Type definitions are also introduced in this
chapter along with the definition of definitional equivalence on types.
Chapter~\ref{subtypes} adds subtyping to the simply typed fragment and
specifies the additional type rules and semantic definitions that are
needed.  Chapter~\ref{dependent} extends the language with dependent
function and product types.  Theories and parametric theories are introduced
into the language in Chapter~\ref{theories}.  The type rules and semantics
for conditional expressions and the logical connectives defined using
conditional expressions are introduced in Chapter~\ref{conditionals}.
Chapter~\ref{proof theory} specifies the axioms and inference rules of
PVS\@.
 
\chapter{The Simple Type Theory}\label{simple}

\begin{comment}
The types of expression
are determined from the types of their subexpressions.  In the base case,
the types of atomic expressions must be given in a context.  There are
certain well-formedness constraints on contexts that we now specify.
A context is a sequence of declarations.  Declarations can either
be type declarations of the form $T\ :\ \mathtt{TYPE}$ where $T$ is a symbol,
constant declarations of the form $c\ :\ T$, or variable declarations of
the form $x\ :\ \mathtt{VAR} T$.  
The empty context $\{\}$ is a context.  If $\Gamma$ is a context,
then
\end{comment}

PVS is a strongly typed specification language.  The simply typed fragment
includes types constructed from the base types by the function and product
type constructions, and expressions constructed from the constants and
variables by means of application, abstraction, and tupling.  Expressions
are checked to be well typed under a {\em context\/}, which is a partial
function that assigns a {\em kind\/} (one of \texttt{TYPE},
\texttt{CONSTANT}, or \texttt{VARIABLE}) to each symbol, and a type to the
constant and variable symbols.  We use the metavariables $\Gamma$,
$\Delta$, and $\Theta$ to range over contexts.  The metavariables $A$,
$B$, and $T$ range over PVS type expressions, the metavariables $r$ and
$s$ range 
over symbols (identifiers), the metavariables $x$ and $y$ range over PVS
variables, and the metavariables $a$, $b$, $f$, and $g$ range over PVS
terms.  Given 
a context $\Gamma$ and a symbol $s$, we say that $\Gamma(s)$ is
undefined if $s$ is not declared in $\Gamma$\@.

The \emph{pretypes} of the simple type theory include the \emph{base}
types such as \texttt{bool} and \texttt{real}.  A \emph{function} pretype
from domain pretype $A$ to range pretype $B$ is constructed as
$\funtype{A}{B}$\@.  A \emph{product} pretype of ${\listwo{A}{n}}$ is
constructed as $\tupletype{\listwo{A}{n}}$.
\begin{comment}
We make no
distinction between a type $A$ and the singleton tuple type $[A]$.  We
also disallow a 0-tuple type.
\end{comment}
A \emph{type} is a pretype that has been
typechecked in a given context.  Types in the simple type theory are
simple enough  that the only distinction between pretypes and types is
that the symbols in a type must be appropriately declared in the given
context.

\begin{Eg}{pretypes} 
 \texttt{bool}, \texttt{real},
 $\tupletype
{\mathtt{bool}, \mathtt{real}}$,
 $\funtype{\tupletype{\mathtt{real}, \mathtt{bool}}}{ \mathtt{bool}}$.
\end{Eg}

The {\em preterms\/} of the language consist of the constants, variables,
pairs, projections, applications, and abstractions.  The metavariables $c$
and $d$ range over constants.  Pairs are of the form $(a_1, a_2)$ where
each $a_i$ is a preterm.  Applications have the form $f\ a$ where $f$ and
$a$ are preterms.  A pair projection is an expression of the form
$\proj{i}\ a$, where $i\in\{1, 2\}$\@.
\begin{comment}
As with types, we make no
distinction between the singleton tuple preterm $(a)$ and the preterm
$a$.
\end{comment}
Lambda abstractions have the form $\lambda (x : T): a$, where $T$
is a pretype and $a$ is a preterm.
\begin{comment}
$\lambda (x_1 : T_1, \ldots,
x_n : T_n): a$, where each $T_i$ is a type, each $x_i$ is a symbol so that
$x_i\not\equiv x_j$ for $i\neq j$, and $a$ is a preterm.
\end{comment}
Parentheses
are used for disambiguation.  A \emph{term} is a preterm that has been
typechecked in a given context.

\begin{Eg}{preterms}
\texttt{TRUE}, $\neg~\mathtt{TRUE}$,
$\lambda~(x~:~\mathtt{bool}):~\neg(x)$,\hfill\\
$\proj{2}~(\mathtt{TRUE}, \mathtt{FALSE})$,
$(\mathtt{TRUE},  \lambda~(x~:~\mathtt{bool})~:~\neg~(\neg~x))$.
\end{Eg}

\section{Contexts}

A context is 
 a sequence of
declarations, where each declaration is either a type declaration
$s\ :\ \mathtt{TYPE}$, a constant declaration $c\ :\ T$ where $T$ is a type,
or a variable declaration $x\ :\ \mathtt{VAR}\ T$.
Preterms and pretypes are typechecked with respect to a given context.  
The empty context is represented as $\{\}$\@.  
The well-formedness rules
for contexts are presented below.
A context can also be applied as a
partial function so that for a symbol $s$ with declaration $D$, $(\Gamma, s
: D)(s) = D$ and 
$(\Gamma, s : D)(r) = \Gamma(r)$ for $r\neq s$.
If $s$ is not declared in $\Gamma$, then $\Gamma(s)$ is undefined. 
If $\Gamma$ is a context, then
for any symbol $s$, the kind of the symbol $s$ in $\Gamma$ is given by $\textit{kind}(\Gamma(s))$.  If the kind of $s$ in $\Gamma$ is \texttt{CONSTANT} or
\texttt{VARIABLE}, then the $ \textit{type\/}(\Gamma(s))$ is the type assigned
to $s$ in $\Gamma$.  
%\memo{Variable declarations are unrelated to variable declarations in
%PVS and correspond to binding declarations.}
\newcommand{\coln}[2]{#1\ : #2}

\begin{Eg}{context}\hfill\\
$ \mathtt{bool}\ :\  \mathtt{TYPE},\hspace{.2cm} \coln{ \mathtt{TRUE}}{
\mathtt{bool}},\hspace{.2cm} \coln{ \mathtt{FALSE}}{
\mathtt{bool}},\hspace{.2cm} \coln{x}{ \mathtt{VAR}\ \funtype{\tupletype{
\mathtt{bool}, \mathtt{bool}}}{ \mathtt{bool} }}$ 
\end{Eg}


\section{Type Rules}

The type rules for the simple type theory are given by a recursively
defined partial function
$\tau$ that assigns
\begin{enumerate}
\item  A type $\tau(\Gamma)(a)$ to a preterm $a$ that
is well typed with
respect to a context $\Gamma$. 
\item  The keyword \texttt{TYPE} as the result of $\tauGamma{A}$
when $A$ is a  well-formed type under context $\Gamma$\@.

\item  The keyword \texttt{CONTEXT} as the result of
$\tauGamma{\Delta}$ when $\Delta$ is a well-formed context
under context $\Gamma$\@.  The context $\Gamma$ is empty for the
simply typed fragment so that typechecking is always invoked as
$\tau()(\Gamma)$\@. 
\end{enumerate}
Otherwise, $\tau$ is undefined in the case of an ill-typed preterm or an
ill-formed type or context. 

The type rules are given by the recursive definition for $\tau$\@.
Typechecking in PVS assigns a ``canonical'' type to a preterm.
Customarily, type rules are presented as inference rules, but
a functional presentation is more appropriate for PVS since
\begin{enumerate}
\item The type assignment is deterministic.  A term can, in
general, though not in the simply typed fragment,  be assigned
a number of types but it always has at most one canonical type.  

\item The soundness proof need only show that the meaning of the term
is an element of the meaning of its canonical type.  Thus, only
the canonical type derivation for a term has to be shown sound and not
every valid type derivation. 

\item The meaning of a term is therefore given by recursion on the term
itself and not on its typing derivation.  There is no need to show
separately that this meaning is {\em coherent\/}, that is, independent of
the typing derivation.
\end{enumerate}
  A functional presentation of the type rules also
leads to natural and straightforward soundness arguments.  
Note that the well-formedness rules for contexts and types are trivial
in the simply typed situation but become more meaningful when the type
theory is extended.  Note also that in the type rules for expressions and
types, the well-formedness of the relevant context is not explicitly
checked.  These rules do preserve the well-formedness of the context
in each recursive call so that if the initial context is well formed, then
so is every intermediate one.    
\begin{Defn}{type rules}
\begin{eqnarray*}
  \tau()(\{\}) & = & \ttcontext \\
%
  \tau()(\Gamma, s\ :\ \tttype{}) & = & \ttcontext, \mbox{ if }
  \Gamma(s) \mbox{ is undefined}\\& &
  \mbox{and } \tau()(\Gamma) = \ttcontext\\
%
  \tau()(\Gamma, c : T) & = & \ttcontext, \mbox{ if }
   \Gamma(c) \mbox{ is undefined,}\\ & & 
\tauGamma{T} =
      \tttype{},\\
   & & \mbox{and } \tau()(\Gamma) = \ttcontext\\
%
  \tau()(\Gamma, x : \ttvar{}\ T) & = & \ttcontext, \mbox{ if }
  \Gamma(x) \mbox{ is undefined,}\\ & & 
\tauGamma{T}
    = \tttype{},\\
    & & \mbox{and } \tau()(\Gamma) = \ttcontext\\ 
%
  \tauGamma{s} & = & \tttype{}, \mbox{ if } \textit{kind\/}(\Gamma(s)) =
\tttype{}\\
%
  \tauGamma{\funtype{A}{B}} & = & \tttype{}, \mbox{ if } \tauGamma{A} =
     \tauGamma{B} = \tttype{}\\
%
  \tauGamma{\tupletype{\listwo{A}{n}}} & = & \tttype{}, \mbox{ if }
     \tauGamma{A_i} = \tttype{} \mbox{ for } 1\leq i\leq 2 \\
%
  \tau(\Gamma)(s) & = &  \textit{type\/}(\Gamma(s)),\\
                    & &  \mbox{ if } 
\textit{kind}(\Gamma(s)) \in \{ \ttconstant{}, \ttvariable \} \\
%
 \tau(\Gamma)(f\ a) & = & B, \mbox{ if } \tau(\Gamma)(f) = \funtype{A}{B}
\mbox{ and } \tau(\Gamma)(a) = A \\
%
\tauGamma{\lambda (x : T): a} & = & \funtype{T}{\tau(\Gamma, x : \ttvar{}\ T)(a)},
\mbox{ if } \Gamma(x) \mbox{ is undefined}\\& & 
  \mbox{and }\tauGamma{T} = \tttype{}\\
%\tau(\Gamma)(\lambda (x_1 : T_1, \ldots, x_n : T_n): a) & = &
%         \begin{array}[t]{l}
%	   [{\tupletype{\listwo{T}{n}}} \aro \\
%   \tau(\Gamma, x_1 : \ttvar{}\ T_1, \ldots, x_n : \ttvar{}\ T_n)(a)]
%	 \end{array} \\ & &
%\mbox{if } 
%   \tauGamma{T_i} = \tttype{}, \mbox{ for } 1\leq i\leq n,\\
%& & \mbox{and  } x_i \not\equiv x_j, \Gamma(x_i) \mbox{ is undefined},\\
% & &  \mbox{ for } 1\leq i, j \leq n, i\neq j \\
%
\tau(\Gamma)((a_1, a_2)) & = &
    [\tau(\Gamma)(a_1),  \tau(\Gamma)(a_2))]\\
%
\tau(\Gamma)(\proj{i}\ a) & = & T_i, \mbox{ where }\\ & & 
 \tau(\Gamma)(a) = [T_1, T_2] %\mbox{ for some } n, n\geq i
\end{eqnarray*}
\end{Defn}

 In the type rule for lambda abstraction, the constraint that
$\Gamma(x)$ must be undefined can be satisfied by suitably renaming the
bound variable since we treat terms as equivalent modulo the renaming of
bound variables.  

\begin{comment}
The above definition suggests that projection applications should
not be treated as applications.
\end{comment}

\begin{Eg}{type rules}  Let $\Omega$ label the context
$\coln{ \mathtt{bool}}{\tttype{}}$,
            $\coln{ \mathtt{TRUE}}{\mathtt{bool}}$,
            $\coln{ \mathtt{FALSE}}{ \mathtt{bool}}$
\begin{eqnarray*}
\tau()(\{\}) & = & \ttcontext\\
\tau()(\Omega)
   & = & \ttcontext\\
\tau(\Omega)(\funtype{\tupletype{ \mathtt{bool}, \mathtt{bool}}}{
\mathtt{bool}}) & = & \tttype{}\\
\tau(\Omega)(( \mathtt{TRUE}, \mathtt{FALSE})) & = & \tupletype{ \mathtt{bool},
\mathtt{bool}}\\
\tau(\Omega)( \proj{2} ( \mathtt{TRUE}, \mathtt{FALSE})) & = & \mathtt{bool}\\
\tau(\Omega)(\lambda (\coln{x}{ \mathtt{bool}}) : \mathtt{TRUE}) & = &
\funtype{ \mathtt{bool}}{ \mathtt{bool}}
\end{eqnarray*}
\end{Eg}

\section{Semantics}

Recall that a preterm $a$ with a type assigned by $\tau$  under context $\Gamma$
is said to be a term of type $\tau(\Gamma)(a)$
in the context $\Gamma$.  If $\gamma$ 
is an \emph{assignment} for the symbols declared in context $\Gamma$, the
semantics of the simple type theory of PVS 
is given by  mapping  a type $T$ to a (possibly empty) set
$\mathcal{M}(\Gamma\vbar \gamma)(T)$, 
and a term $a$ with assigned type $T$ to an element of the set $\mathcal{
M}(\Gamma\vbar \gamma)(T)$ in the basic universe $U$\@. 
The assignment $\gamma$ is a list of bindings of the form $\{s_1\gets
t_1\}\ldots \{s_n\gets t_n \}$\@.  The application of an assignment
$\gamma$ to a symbol $s$ is such that $\gamma \{s \gets t\} (s)$ is $t$,
whereas $\gamma \{r \gets t\}(s)$ is $\gamma(s)$ when $r\not\equiv s$\@.

The  meaning function $\mathcal{M}$ returns the meaning of
a well-formed type $A$  
and a well-formed expression $a$ in the context $\Gamma$ under an
assignment $\gamma$ as $\mathcal{M}(\Gamma\vbar \gamma)(A)$ 
and 
$\mathcal{M}(\Gamma\vbar \gamma)(a)$,respectively. 
\begin{comment}
The base types 
\texttt{bool} and \texttt{real} are mapped to $\twob$ and $\reals$,
respectively.
\end{comment}

The meanings of type names, constants, and variables
declared in $\Gamma$ are obtained from the assignment $\gamma$\@.  A
function type is mapped to the 
corresponding function space.  A product type is mapped to the corresponding
Cartesian product.
\begin{comment}
an $n$-tuple is represented by an $(n+1)$-fold
Cartesian product terminated by the sole element $\zerob$ of the set \oneb\@.
This representation  distinguishes a 3-tuple
$\tupletype{\ttbool, \ttbool, \ttbool}$ from the 
2-tuple $\tupletype{\ttbool, \tupletype{\ttbool, \ttbool}}$\@.
\end{comment}
An application term is interpreted by 
means of set-theoretic function application.  A lambda abstraction yields the
graph of the corresponding function.  A pair expression is mapped
to the corresponding set-theoretic ordered pair.    
\begin{Defn}{meaning function}
\begin{eqnarray*}
%% \begin{comment}
%%   \Mgamma{ \mathtt{bool}} & = & {\twob} \\
%%   \Mgamma{ \mathtt{real}} & = & {\reals}\\
%%   \Mgamma{ \mathtt{TRUE}} & = & {\oneb}\\
%%   \Mgamma{ \mathtt{FALSE}} & = &{\zerob}\\
%% \end{comment}
  \mathcal{M}(\Gamma\vbar \gamma)(s) & = &  \gamma(s),\\
                    & &  \mbox{ if } 
\textit{kind}(\Gamma(s)) \in \{ \tttype{}, \ttconstant{}, \ttvariable \}
\\
%
  \mathcal{M}(\Gamma\vbar \gamma)(\funtype{A}{B}) & = & \Mgamma{B}^{\Mgamma{A}}\\
%
  \Mgamma{\tupletype{\listwo{T}{n}}} & = &  \Mgamma{T_1} \times %%\ldots\times
\Mgamma{T_2} %\times {\oneb}
\\
%
\mathcal{M}(\Gamma\vbar \gamma)(f\ a) & = & (\mathcal{M}(\Gamma\vbar \gamma)(f))(\mathcal{M}(\Gamma\vbar \gamma)(a)) \\
%
\Mgamma{\lambda (x : T): a} & = & \{\tuple{y, z} \vbar \begin{array}[t]{l}
						       y\in\Mgamma{T},\\
                					z = \mathcal{M}(\Gamma, x: \ttvar{}\ T\vbar \gamma\{x\gets y\})(a)\}
						       \end{array}\\
%               
%\mathcal{M}(\Gamma| \gamma)(\lambda (x_1 : T_1, \ldots, x_n : T_n): a) & = &
%  {\begin{array}[t]{l}\{ \tuple{\tuple{\listwo{y}{n}, {\zerob}}, z} | \\
%					y_1\in \Mgamma{T_1}\\
%                                        \vdots\\
%                                        y_n\in \Mgamma{T_n}\\
%                                   z = \mathcal{M}\left(
%        \begin{array}{l}
%  \Gamma, x_1 : \ttvar{}\ T_1, \ldots, x_n : \ttvar{}\ T_n;\\
%  \gamma\{x_1 \gets y_1\}\ldots \{x_n\gets y_n\}
%	\end{array}\right)(a)\} 
%				      \end{array}}\\
%
\Mgamma{(\listwo{a}{n})} & = & \tuple{\Mgamma{a_1},  \Mgamma{a_2}} \\
%
\Mgamma{\proj{i}\ a} & = & t_i, \mbox{ where } \Mgamma{a} =
\tuple{\listwo{t}{n}}
\end{eqnarray*}
\end{Defn}

\begin{Eg}{meaning function}
 Let $\omega$ be an assignment for the context $\Omega$ in
Example~\ref{eg:type rules}, of the form
$$\{\ttbool\gets \twob\}\{ \mathtt{TRUE}\gets \oneb\}\{
\mathtt{FALSE}\gets \zerob\}$$
then
\begin{eqnarray*}
\mathcal{M}(\Omega\vbar \omega)(\tupletype{\ttbool, \ttbool}) & = &
\twob\times\twob\\
%
\mathcal{M}(\Omega\vbar \omega)(( \mathtt{TRUE}, \mathtt{FALSE})) & = &
\tuple{\oneb, \zerob}\\
%
\mathcal{M}(\Omega\vbar \omega)(\lambda (x : \ttbool): \mathtt{TRUE}) & = &
\{\tuple{\zerob, \oneb}, \tuple{\oneb, \oneb}\}
\end{eqnarray*}
\end{Eg}

\begin{Defn}{satisfaction}
A context assignment $\gamma$ is said to \emph{satisfy} a context $\Gamma$
(in symbols $\gamma\models \Gamma$) 
iff
\begin{enumerate}
 \item $\gamma(\ttbool) = {\twob},$
 \item $\gamma(\tttrue) = {\oneb},$
 \item $\gamma(\ttfalse) = {\zerob},$
\item $\gamma(s)\in U$ whenever $ \itkind{}(\Gamma(s)) = \tttype$, and
 \item $\gamma(s) \in \Mgamma{ \ittype{}(\Gamma(s))}$\hfill\\\hspace*{1in}
whenever $ \itkind{}(\Gamma(s)) \in \{ \ttconstant,
\ttvariable \}$.
\end{enumerate}
\end{Defn}
\begin{Eg}{satisfaction}\hfill\\\begin{enumerate}
\item The assignment $\omega$ satisfies context $\Omega$.

\item The assignment $\omega \{ \mathtt{one}~\gets~{\oneb}\}\{
\mathtt{zero}~\gets~\zerob\}$ satisfies the context
$\Omega,~\mathtt{one~:~TYPE},~\mathtt{zero~:~one}$\@.  
\end{enumerate}
\end{Eg}

We need one useful proposition that asserts that typing judgements are not
invalidated when the context is extended.
\begin{prop}\label{context-weakening} %where is this used? NSH(9.6.95)
If $\tau()(\Gamma) = \tau()(\Gamma') = \ttcontext$ and $\Gamma$ is
a prefix of $\Gamma'$, then for all pretypes $A$, $\tauGamma{A} = \tttype$
implies $\tau(\Gamma')(A) = \tttype$, and for all preterms $a$,
$\tauGamma{a} = A$ implies $\tau(\Gamma')(a) = A$\@.  
\end{prop}

The following theorems follow from the induction suggested by the
definitions of $\tau$ and $\mathcal{M}$.  The first of these
is straightforward and is given without proof.
\begin{theorem}[type construction]  \label{type construction}
If $\tau()(\Gamma) = \ttcontext$ and $\tauGamma{a} = A$, then
$\tauGamma{A} = \tttype$.
\end{theorem}
%
\begin{theorem}[type soundness]\label{simple-type-semantics}
 If $\tau()(\Gamma) = {\ttcontext}$, 
   $\gamma$ satisfies $\Gamma$, and
   $\tauGamma{A} = {\tttype{}} $, then 
$\Mgamma{A} \in U$\@.
\end{theorem}
\begin{proof}
The proof is by induction on the structure of the pretype $A$.
Recall that if $X\in U$, then for some $i$, $X \in U_i$\@.  
This yields three cases:
\begin{enumerate}
\item $A \equiv s$: By Definition~\ref{defn:type rules},
$\Gamma(s)$ is defined and $\itkind(\Gamma(s))$ is $\tttype$\@.
Then by Definition~\ref{defn:meaning function}, $\Mgamma{s}$
is $\gamma(s)$, and by Definition~\ref{defn:satisfaction},
$\gamma(s)\in U$\@.

\item $A\equiv \funtype{B}{C}$: We then have that $\tauGamma{B} =
\tauGamma{C} = \tttype$\@.  Letting $X$ label $\Mgamma{B}$, and $Y$ label
$\Mgamma{C}$, we have by the induction hypothesis that $X\in U$ and $Y\in
U$\@.  Let $j$ be the least rank such that $\Mgamma{B}\in U_j$ and
$\Mgamma{C}\in U_j$\@.  By Definition~\ref{defn:meaning function},
$\Mgamma{A} = Y^X$, and hence $\Mgamma{A} \in U_{j+1}$ by
Definition~\ref{defn:type universe}\@.

\item $A \equiv \tupletype{\listwo{A}{n}}$: Again by
Definition~\ref{defn:type rules} and the induction hypothesis, we have
for each $i \in \{1, 2\}$, that $\Mgamma{A_i}\in U$\@.  Let $j$ be the least
rank such that for $i\in \{1, 2\}$, $\Mgamma{A_i}\in U_j$\@.  Then,
it is easy to verify from Definition~\ref{defn:type universe} that
$\Mgamma{A}\in U_{j + 1}$\@.  
\end{enumerate}
\end{proof}

\begin{theorem}[term soundness]\label{simple-term-semantics}
 If $\tau()(\Gamma) = \ttcontext$, 
   $\gamma$ satisfies $\Gamma$, and
   $\tauGamma{a}$ is defined and equal to $A$, then 
$\Mgamma{a} \in \Mgamma{A}$.  
\end{theorem}
\begin{proof}
The proof is by induction on the structure of preterms.
\begin{enumerate}
\item $a\equiv s$: By Definition~\ref{defn:type rules},
we have that $\ittype(\Gamma(s)) = A$\@.  By
Definitions~\ref{defn:meaning function}
and~\ref{defn:satisfaction}, we have that $\Mgamma{a} = \gamma(s)$
and $\gamma(s) \in
\Mgamma{A}$\@.
\item $a \equiv (f\ b)$:  By Definition~\ref{defn:type rules},
$\tauGamma{f} = \funtype{B}{A}$, and $\tauGamma{b} = B$, for some
$B$ such that $\tauGamma{B} = \tttype$\@.  
Let $\Mgamma{A}$ be $X$ and $\Mgamma{B}$ be $Y$, then 
by Definitions~\ref{defn:type rules}
and~\ref{defn:meaning function}, and the induction hypothesis, we have
$\Mgamma{f} \in X^Y$ and $\Mgamma{b}\in Y$\@.  It therefore follows
by Definition~\ref{defn:meaning function} that $\Mgamma{(f\ b)} =
(\Mgamma{f})(\Mgamma{b})$, and hence $\Mgamma{(f\ b)} \in X$\@.

\item $a \equiv (\lambda (x : C): b)$:  By
Definition~\ref{defn:type rules}, we have that
$\tauGamma{a}$ is $\funtype{C}{B}$, where
$\tau(\Gamma, x : \ttvar~C)(b)$ is $B$\@.  
Let $X$ be $\Mgamma{C}$, and $Y$ be
$\mathcal{M}(\Gamma, x: \ttvar~C \vbar \gamma\{x\gets u\}))(B)$\@.  
By the induction hypothesis, we have that for
any $u\in Y$, 
$\mathcal{M}(\Gamma, x: \ttvar~C \vbar \gamma\{x\gets u\})(b)
\in X$\@.  
Since $\Mgamma{a}$ is $\{\pair{u, v} \vbar u\in X, v = \mathcal{M}(\Gamma, x:
\ttvar~C \vbar \gamma\{x\gets u\})(b)\}$,
we have that $\Mgamma{a} \in X^Y$\@.

\item $a \equiv (a_1, a_2)$: By Definition~\ref{defn:type rules},
$\tauGamma{a} = \tupletype{\listwo{A}{n}}$, where $\tauGamma{a_i} = A_i$ for
$i\in \{1, 2\}$\@.   By the induction hypothesis, $\Mgamma{a_i}\in
\Mgamma{A_i}$ for $i\in \{1, 2\}$\@.  By
Definition~\ref{defn:meaning function}, $\Mgamma{a} =
\pair{\Mgamma{a_1},\Mgamma{a_2}}$ and hence $\Mgamma{a}$ is
an element of $\Mgamma{A}$ which is
$\Mgamma{A}\times\Mgamma{A_n}$\@.

\item $a\equiv \proj{i}~b$: In this case, we know by
Definition~\ref{defn:type rules} that $\tauGamma{b} =
\tupletype{\listwo{A}{n}}$
with $i\in \{1, 2\}$, and $\tauGamma{a} = A_i$\@.
By the induction hypothesis, $\Mgamma{b} = \pair{t_1, t_2}$,
and by Definition~\ref{defn:meaning function}, $\Mgamma{a} = t_i$
and $\Mgamma{\tauGamma{b}} =
\Mgamma{A_1}\times\Mgamma{A_2}$,
hence $\Mgamma{a}\in \Mgamma{A_i}$\@.  
\end{enumerate}
%%\memo{What about tuples/projections?}
\end{proof}

These three theorems (\ref{type construction},
\ref{simple-type-semantics}, and \ref{simple-term-semantics}) are the key
invariants that must be satisfied by the semantics when the language is
extended below with type definitions, subtypes, dependent types, and
parametric theories.

\section{Some Syntactic Operations}

We first define the operation of collecting the free variables of a
term $a$ in a given context $\Gamma$ as $FV(\Gamma)(a)$, and then
define the operation of substitution.
\begin{Defn}{free variables}
\begin{eqnarray*}
  FV(\Gamma)(s) & = & \left\{\begin{array}{l}
			      \{s \}, \mbox{ if } \textit{kind}(\Gamma(s))
= \ttvariable\\
                              \emptyset, \mbox{ otherwise}
			    \end{array}\right.\\
%
  FV(\Gamma)(f\ a) & = & FV(\Gamma)(f) \union FV(\Gamma)(a) \\
%
  FV(\Gamma)(\lambda (x : T) : a) & = &
     FV(\Gamma, x : \ttvar{}\ T)(a)
- \{x\} \\
%
 FV(\Gamma)((\listwo{a}{n})) & = & FV(\Gamma)(a_1)\union
FV(\Gamma)(a_2)\\
%
 FV(\Gamma)( \proj{i}\ a) & = & FV(\Gamma)(a)
\end{eqnarray*}
\end{Defn}
%
\begin{Defn}{substitution}
\begin{eqnarray*}
  s[a_1/x_1,\ldots, a_n/x_n] & = & \left\{\begin{array}{l}
					    a_i, \mbox{ if for some minimal } i,
s \equiv x_i\\
                                            s, \mbox{ otherwise}
					  \end{array}\right.\\
%
  (f\ a)[a_1/x_1,\ldots, a_n/x_n] & = &
     (f[a_1/x_1,\ldots, a_n/x_n]\\&&
       \hspace*{.5in} a[a_1/x_1,\ldots, a_n/x_n]) \\
%
  (\lambda (y : T): a)[a_1/x_1,\ldots, a_n/x_n] & = &
    (\lambda (y':T) :
     \ a[y'/y, a_1/x_1,\ldots, a_n/x_n]), \\ & & 
%
\mbox{where  } y' \mbox{ is a fresh variable}\\
%
 (\listwo{b}{m})[a_1/x_1,\ldots, a_n/x_n] & = &
     (b_1[a_1/x_1,\ldots, a_n/x_n],\\&&
       \hspace*{.5in}b_2[a_1/x_1,\ldots, a_n/x_n])\\
%
  ( \proj{i}\ a)[a_1/x_1,\ldots, a_n/x_n] & = &
   ( \proj{i}\ a[a_1/x_1,\ldots, a_n/x_n])
\end{eqnarray*}
\end{Defn}

Recall that terms are treated as syntactically equivalent modulo
alpha conversion.  
The above definitions must be extended as more 
features are added to the language. 

\section{Type Definitions}

%%\memo{See ADT section for recursive datatypes.}

Here we enrich contexts so that type symbols may have definitions.  PVS
does not allow recursive type definitions\footnote{For the moment, we are
not considering the PVS \texttt{DATATYPE} mechanism, which is a form of
recursive type definition~\cite{Shankar:ADT97}.  Recursive datatypes in
the context of the HOL proof checking system are described by
Melham~\cite{Melham89}.}  so a type declaration/definition in a context
may use only the symbols declared in the prior part of the context.  The
main difference in the extended language is that type names can have
definitions.  In such cases, the definitions rather than the type names
are used to determine the actual type of an expression.  In other words,
two type expressions are treated as the same if they are
\emph{definitionally equivalent}\@.  Most other specification languages
tend to employ the weaker notion of \emph{name equivalence} where
syntactically different types are treated as distinct even when their
definitions coincide.

To accommodate type definitions, a context can
contain type declarations of the form $s\ :\ \mathtt{TYPE} = T$, where $T$
is a type.  If context $\Gamma$ contains such a declaration for $s$, then
$ \itdef{}(\Gamma(s))$ is $T$\@.  To extend $\tau$ to handle type
definitions under definitional equivalence, we must ensure that $\tau$
returns the canonical form of a type where all defined types have been
replaced by their definitions.  The operation $\delta(\Gamma)(T)$ returns
the expanded form of a type relative to the context $\Gamma$\@.
%
\begin{Defn}{expanded type}
\begin{eqnarray*}
  \delta(\Gamma)(s) & = & s, \mbox{ if } \textit{definition\/}(\Gamma(s))
\mbox{ is empty}\\
% 
  \delta(\Gamma)(s) & = & \delta(\Gamma)(
\textit{definition\/}(\Gamma(s))), 
\mbox{ if } \textit{definition\/}(\Gamma(s))
\mbox{ is nonempty} \\
%
 \delta(\Gamma)(\funtype{A}{B}) & = &
      \funtype{\delta(\Gamma)(A)}{\delta(\Gamma)(B)} \\
%
 \delta(\Gamma)(\tupletype{\listwo{T}{n}}) & = &
     \tupletype{\delta(\Gamma)(T_1),  \delta(\Gamma)(T_2)}
\end{eqnarray*}
\end{Defn}


The typing rules are augmented to return the type in expanded form.
The main issue here is to determine that the definition part of a type
declaration in a context is well formed relative to the preceding context.
We also need to ensure that $\tau$ returns the expanded form of the type
corresponding to a preterm.
%
\begin{Defn}{type rules with type definitions}
\begin{eqnarray*}
    \tau()(\Gamma, s\ :\ \tttype{} = T) & = & \ttcontext, \mbox{
if }  \Gamma(s) \mbox{ is undefined}, \\ & &
\tau()(\Gamma) = \ttcontext,\\ & & 
  \mbox{and }     \tauGamma{T} = \tttype{} \\
%
%   \tau()(\Gamma, c : T = a) & = & \ttcontext, \mbox{ if } \tauGamma{T} =
%      \tttype{} \mbox{ and } \tauGamma{a} = \delta(\Gamma)(T)\\
%
    \tau(\Gamma)(s) & = &  \delta(\Gamma)(\textit{type\/}(\Gamma(s))),\\
                    & &  \mbox{ if } 
\textit{kind}(\Gamma(s)) \in \{ \ttconstant{}, \ttvariable \} 
\end{eqnarray*}
\end{Defn}


Note that the $\delta$ operator is idempotent, and $\tauGamma{a}$
for a term $a$ always returns an expanded type, that is, 
$\delta(\tauGamma{a}) = \tauGamma{a}$\@.  

We do not need to update the definition of ${\mathcal M}$ from
Definition~\ref{defn:meaning function} since the syntax for terms is
unchanged, 
but we do need
to revise the notion of a satisfying context assignment (from 
Definition~\ref{defn:satisfaction}) to respect the type
definitions.
\begin{Defn}{satisfaction with type definitions}
An assignment $\gamma$ satisfies a context $\Gamma$ if
in addition to the conditions in Definition~\ref{defn:satisfaction},
whenever $ \textit{kind}(\Gamma(s)) = \tttype{}$ and
$\textit{definition}(\Gamma(s))$ (abbreviated as $T$) is nonempty, then
$\gamma(s) = \Mgamma{ T}$.
\end{Defn}

Theorems~\ref{type construction} and~\ref{simple-type-semantics}
and~\ref{simple-term-semantics} continue to hold under these extensions,
and the proofs are easily adapted to the modified definitions.  

\begin{Eg}{type definition}
Let $\Omega'$ be the context\\ $\Omega,
\mathtt{boolop}:\mathtt{TYPE}=\funtype{\tupletype{\mathtt{bool}, \mathtt{bool}}}{\ttbool},
\vee:\mathtt{boolop}$\@.  Then
\begin{eqnarray*}
\tau()(\Omega') & = & \mathtt{CONTEXT}\\
%
 \delta(\Omega')( \mathtt{boolop})  & = & \funtype{\tupletype{ \mathtt{bool},
\mathtt{bool}}}{\ttbool},\\
 \tau(\Omega')( \vee) & = & \funtype{\tupletype{ \mathtt{bool}, \mathtt{bool}}}{\ttbool}
\end{eqnarray*}
\end{Eg}

\section{Summary}

We have defined the simply typed fragment of PVS by introducing the syntax
for pretypes and preterms, the type rules and semantics for well-formed
contexts, types, and terms.  The type rules are presented in a novel
functional style where each well-formed context is assigned the label {\tt
CONTEXT}, each well-formed type is assigned the label {\tt TYPE}, and each
well-formed term is assigned a canonical type.  The semantics takes a
satisfying assignment for a context and maps a well-formed type to a set
and a well-formed term to an element of the set corresponding to its
canonical type.  We then defined the syntactic operations of collecting
the free variables in an expression and for substituting terms for
variables in an expression.

The simple type theory is then extended with type definitions.  With this
extension, two type expressions are treated as equivalent if they are
identical after all type definitions have been expanded.  The operation
$\delta$ returns the expanded form of a given type expression.


\chapter{Adding Subtypes}\label{subtypes}

Subtyping is one of the main features of the PVS specification
language.\footnote{The form of subtyping used in PVS is derived from a
suggestion of Friedrich von~Henke.}  Subtyping in PVS corresponds to the
set-theoretic notion of a subset.  It raises several delicate issues that
were absent in the language presented thus far.  In the simply typed
fragment, each type corresponds to a set of values that is somehow
structurally different from the set of values for another type so that a
term has at most one type.  Subtyping makes it possible to introduce the
natural numbers as a subtype of the reals, and to treat the primes, the
even numbers, and the odd numbers as subtypes of the natural numbers.
With subtyping, a term can obviously have several possible types, but the
type-checking function $\tau$ may return only a single type.  We constrain
$\tau$ to return a natural canonical type of an expression that is given
by the declarations of the symbols in the expression.  If the expression
is used in a context where the expected type is a supertype of its
canonical type, then the type correctness is straightforward.  If the
expected type is a subtype that is {\em compatible\/} with the canonical
type of the expression, then typechecking generates proof obligations
asserting that the expression satisfies the predicate constraints imposed
by the expected type.  Two types are compatible if they have equivalent
maximal supertypes.  Type equivalence in the presence of subtypes is not a
simple notion.  Subtyping also introduces the possibility of types being
empty.  Typed lambda calculi with possibly empty types have been studied
by Meyer, Mitchell, Moggi, and Statman~\cite{MMMS:empty}.  This chapter
introduces predicate subtypes and defines the notions of compatibility and
type equivalence prior to presenting the type rules and semantics.


We restrict our attention to contexts $\Gamma$ that extend
the declarations:
\begin{alltt}
 bool : TYPE,
 TRUE : bool,
 FALSE : bool,
 boolop : \(\funtype{\tupletype{\ttbool, \ttbool}}{\ttbool}\),
 \(\neg\) : \(\funtype{\ttbool}{\ttbool}\),
 \(\vee\) : boolop,
 \(\wedge\) : boolop,
 \(\supset\) : boolop
\end{alltt}
We will abuse PVS notation to employ the customary infix forms of
operations like $\vee$, 
$\wedge$, and $\supset$.    The pretype corresponding to a predicate
subtype has the form $\{x : T \vbar a\}$ where $x$ is a symbol, $T$ is a
pretype, and $a$ is a preterm.   
A \emph{predicate type} in PVS is a function type where the range is the
primitive type \texttt{bool}.  A predicate is a term that has a predicate
type.  If $a$ is a term of type \texttt{bool}, then 
we can define the subtype $\{ x : T \vbar a\}$ consisting of those
elements $e$ of $T$ satisfying $a[e/x]$ ($e$ substituted for $x$ in $a$).
Since the elements of the subtype $\{ x : T \vbar a\}$ satisfy the predicate $
\lambda (x  : T): a$, 
we call this type a {\em predicate subtype\/} to distinguish it from other
forms of subtyping.  
Universal quantification $\forall (x : T): a$ is just an abbreviation
for the term $(\lambda (x : T): a) = (\lambda (x : T): \mathtt{TRUE})$\@.  
Although we use the equality  predicate in the definition of universal
quantification and in the definitions below, the actual introduction of
equality is deferred to a later section following the introduction of
parametric theories.  The equality between PVS terms of function type
is to be interpreted as extensional equality.  Note that the `$=$' symbol is
used both for the formal equality symbol in the language and for
metatheoretic equality.   
\begin{comment}
We will also use the notation $\lambda (x_1 : T_1, \ldots, x_n : T_n): a$
in place of $\lambda (x : \tupletype{\seq{T}{n}}): a[
\proj{1}(x)/x_1, \ldots, \mathtt{PROJ\_n}(x)/x_n]$\@.
\end{comment}



Our first step will be to define the notion of a \emph{maximal supertype}
of a given type as $\mu(T)$.  A \emph{maximal type} $T$ is one such that
$\mu(T) = T$.  In a given context, we will apply  $\mu$ only to the
expanded form (given by $\delta$) of a type expression.  
\begin{Defn}{maximal supertype}
\begin{eqnarray*}
  \mu(s) & = & s \\
   \mu(\{x : T \vbar a\}) & = & \mu(T) \\
%
   \mu(\funtype{A}{B}) & = & \funtype{A}{\mu(B)} \\
  \mu(\tupletype{\listwo{A}{n}}) & = & \tupletype{\mu(A_1), \mu(A_2)}
\end{eqnarray*}
\end{Defn}
Note that since subtypes correspond to subsets, in taking the maximal
supertype of a function type, the domain type is held fixed.  In most type
theories with subtypes, the rule for subtyping between function types
$\funtype{A}{B}$ and $\funtype{A'}{B'}$ requires showing that $A'$ is a
subtype of $A$, and $B$ is a subtype of $B'$\@.  Subtyping between
function types is therefore said to be contravariant in the domain type
and covariant in the range type.  Subtyping on function types in PVS is
covariant in the range type but is neither covariant nor contravariant in
the domain type.  This means that the function type
$\funtype{\ttnat}{\ttnat}$ is not a supertype of the function type
$\funtype{\ttint}{\ttnat}$\@.  Such a subtyping relation would violate
{\em extensionality\/}\@.  Two functions on $\ttnat$ are extensionally
equal when they return equal values when applied to equal arguments in
$\ttnat$\@.  Consider two functions in $\funtype{\ttnat}{\ttnat}$:
$\mathit{abs}$ which returns the absolute value, and $\mathit{idnat}$
which behaves as an identity function on natural numbers and returns $0$
otherwise.  These two functions will be erroneously identified if they can
be viewed as being of type $\funtype{\ttnat}{\ttnat}$, and the subset
interpretation of subtypes would be lost.

We will also employ a weaker supertype $\mu_0(T)$ or the
{\em direct supertype\/}, that only considers supertypes of
explicitly given subtypes of the form $\{x : T \vbar a\}$.
\begin{Defn}{direct supertype}
\begin{eqnarray*}
\mu_0(\{x : T \vbar a\}) & = & \mu_0(T)\\
\mu_0(T) & = & T, \mbox{ otherwise}
\end{eqnarray*}
\end{Defn}

\begin{Eg}{maximal supertype}
Given a context containing the declarations
$$\begin{array}{l}
  \ttint : \tttype, \\
  \mathtt{0} : \ttint,\\
  \leq : \funtype{\tupletype{\ttint, \ttint}}{\ttbool},\\
  \ttnat : \tttype = \{ i : \ttint \vbar  0 \leq i\}\\
  \mathtt{natinjection} : \tttype = \{ f : \funtype{\ttnat}{\ttnat}\ \vbar\ 
\forall (i, j: \ttnat): f(i) = f(j) \supset i = j \}
\end{array}$$
we have
\begin{eqnarray*}
  \mu(\mathtt{natinjection}) & = & \mu(\funtype{\ttnat}{\ttnat})\\
 &=  & \funtype{\ttnat}{\mu(\ttnat)}\\
 &=  & \funtype{\ttnat}{\ttint}\\
\mu_0(\mathtt{natinjection}) & = & \funtype{\ttnat}{\ttnat}
\end{eqnarray*}
\end{Eg}

Note that   $\mu(\mu(A)) = \mu(A)$.
Note also that a maximal supertype is never a subtype.  
We can in fact collect the predicates that constrain a type $A$
relative to its maximal supertype $\mu(A)$ as $\pi(A)$\@.

\begin{Defn}{subtype constraints}
\begin{eqnarray*}
  \pi(s) & = & \lambda (x : s): \tttrue \\
   \pi(\{y : T \vbar a\}) & = &   \lambda (x : \mu(T)): (\pi(T)(x) \wedge a[x/y])\\
%
   \pi(\funtype{A}{B}) & = &   \lambda (x : \funtype{A}{\mu(B)}):
    (\forall (y: A): \pi(B)(x(y))) \\
%
\pi(\tupletype{\listwo{A}{n}}) & = &
    \lambda (x: \tupletype{\mu(A_1),  \mu(A_2)}): 
	(\pi(A_1)(\proj{1}~x)\wedge \pi(A_2)(\proj{2}~x))
\end{eqnarray*}
\end{Defn}
Observe that in Definition~\ref{defn:subtype constraints}, if $\tauGamma{A} =
\tttype$, then 
$\tauGamma{\pi(A)} = \funtype{\mu(A)}{
 \mathtt{bool}}$\@.\footnote{This is somewhat tricky in the case of
$\pi(\{y:T\vbar a\})$ since in $a[x/y]$, $x$ has type $\mu(T)$, whereas
$y$ has type $T$\@.  As shown in Chapter~\ref{conditionals}, the
type rules for conjunction are such that $\tau(\Gamma, x :
\ttvar~\mu(T))(\pi(T)(x) \wedge a)$ reduces to
$\tau(\Gamma, x : \ttvar~\mu(T))(\pi(T)(x))$ and
$\tau(\Gamma, x : \ttvar~\mu(T), \pi(T)(x))(a[x/y])$ where the first
conjunct is added to the context as a contextual assumption.
One can then show by induction that $\tau(\Gamma, x:\ttvar~\mu(T),
\pi(T)(x))(a[x/y]) = \tau(\Gamma, y: \ttvar~T)(a)$\@.  }

\begin{Eg}{subtype constraints}
\begin{eqnarray*}
\lefteqn{\pi(\ttnat)}\\
  & = & \lambda (j: \ttint): 0\leq j\\
\lefteqn{\pi(\funtype{\ttnat}{\ttnat})}\\
& = & \lambda (g : \funtype{\ttnat}{\ttint}): \forall (i:\ttnat): (\lambda
(j:\ttint): 0 \leq j)(g(i))\\
\lefteqn{\pi(\mathtt{natinjection})}\\
&= & \lambda (f : \funtype{\ttnat}{\ttint}) : 
\begin{array}[t]{cl}
& \pi(\funtype{\ttnat}{\ttnat})(f)\\
\wedge & (\forall (i, j: \ttnat): f(i) = f(j) \supset  i = j)
\end{array}\\
& = & 
\lambda (f : \funtype{\ttnat}{\ttint}) :\\&&\hspace*{.5in}
\begin{array}[t]{cl}
& (\lambda (g : \funtype{\ttnat}{\ttint}) : \forall (i: \ttnat):
(\lambda (j : \ttint): \mathtt{0}\leq j)(g(i)))(f) \\
\wedge & (\forall (i, j: \ttnat): f(i) = f(j) \supset  i = j)
\end{array}
\end{eqnarray*}
\end{Eg}
Observe that $\pi(\mu(A))$ is essentially equivalent to $\lambda (x : \mu(A)):
\mathtt{TRUE}$. 

Since the subtype $\{x : T \vbar p(x) \wedge q(x)\}$ can also be written
as $\{x : T \vbar q(x) \wedge p(x)\}$, we need a notion of equivalence
between types.  One way to do this is to make types ``first-class'' and to
allow explicit theorems to be proved about type equivalence and subtyping.
Since this would be a fairly drastic extension to the specification
language, we have designed the PVS type system so as to avoid any
first-class treatment of types.  It turns out that all the needed
properties about types (such as equality and subtyping) can be obtained by
generating ordinary proof obligations rather than by explicitly proving
theorems about types.

We introduce 
below a metatheoretic operation that generates the proof obligations
needed to establish that two (maximal) types are equivalent.  This
equivalence is denoted by $\simeq$ and is  applied only to maximal types
and returns a list of the 
proof obligations that must be proved.  Note the invariant in the
definition below that the arguments to $\simeq$ are always maximal.  The
definition of $\simeq$ makes use of the PVS equality predicate that will be
introduced later.  A list of formulas is represented as $a_1,\ldots, a_n$.
Given two such lists $\lis{a}{m}$ and $\lis{b}{n}$, the concatenation of
these two lists is written as $\lis{a}{m}\ ; \ \lis{b}{n}$\@.

\begin{Defn}{type equivalence proof obligations}
\begin{eqnarray*}
  (s \simeq s) & = & \tttrue \\
%
  (\funtype{A}{B} \simeq \funtype{A'}{B'}) & = & ((\mu(A)\simeq \mu(A'));
(\pi(A) = \pi(A'));
 (B \simeq B'))\footnotemark \\
%
  (\tupletype{\listwo{A}{n}} \simeq \tupletype{\listwo{B}{n}}) & = &
    ((A_1\simeq B_1); (A_2\simeq B_2)) \\
%
 (A \simeq B) & = & \ttfalse, \mbox{ otherwise}
\end{eqnarray*}
\end{Defn}
\footnotetext{The type correctness of the proof obligation $(\pi(A) =
\pi(A'))$ depends on the \emph{prior} proof obligations $\mu(A)\simeq
\mu(A')$.}

\begin{Eg}{type equivalence}  Building on the context given in
Example~\ref{eg:maximal supertype}, 
if we have the following variants of $\ttnat{}$ and
\texttt{natinjection}:
$$
\begin{array}{l}
\mathtt{NAT} :  \tttype = \{ i : \ttint \vbar  i \leq \mathtt{0}\ 
\supset\ i = 0\}\\
\mathtt{NATinjection} : \tttype = \{ f : \funtype{ \mathtt{NAT}}{
\mathtt{NAT}} \vbar 
\forall (i, j: \mathtt{NAT}): f(i) = f(j)\ \supset\ i = j \}
\end{array}$$
we get
\begin{eqnarray*}
\mu(\funtype{ \mathtt{natinjection}}{ \mathtt{natinjection}}) =
 \funtype{ \mathtt{natinjection}}{\funtype{\ttnat}{\ttint}}\\
%
\mu(\funtype{ \mathtt{NATinjection}}{ \mathtt{NATinjection}}) =
 \funtype{ \mathtt{NATinjection}}{\funtype{ \mathtt{NAT}}{ \ttint}}
\end{eqnarray*}
\begin{eqnarray*}
\lefteqn{\mu(\funtype{ \mathtt{natinjection}}{ \mathtt{natinjection}})
\simeq \mu(\funtype{ \mathtt{NATinjection}}{ \mathtt{NATinjection}})}\\
& = &  \begin{array}[t]{l}
       (\mu( \mathtt{natinjection}) \simeq \mu( \mathtt{NATinjection}));\\
       (\pi( \mathtt{natinjection}) = \pi( \mathtt{NATinjection}));\\
       (\funtype{\ttnat}{\ttint} \simeq \funtype{ \mathtt{NAT}}{ \mathtt{int}})     
       \end{array}\\
 \lefteqn{(\pi( \mathtt{natinjection}) = \pi( \mathtt{NATinjection}))}\\
& = & \begin{array}[t]{cl}
(&      \lambda (f : \funtype{\ttnat}{\ttint}) :
\begin{array}[t]{cl}
& (\lambda (g : \funtype{\ttnat}{\ttint}) : \forall (i: \ttnat):
\mathtt{0}\leq  g(i))(f) \\
\wedge & (\forall (i, j: \ttnat): f(i) = f(j) \ \supset\  i = j)
 \end{array}\\
 =  \\&  \lambda (f : \funtype{ \mathtt{NAT}}{\ttint}) :\\&\hspace*{.3in}
\begin{array}[t]{cl}
& (\lambda (g : \funtype{ \mathtt{NAT}}{\ttint}) : \forall (i: \mathtt{NAT}):
g(i) \leq \mathtt{0}\ \supset\ g(i) = 0)(f) \\
\wedge & (\forall (i, j: \mathtt{NAT}): f(i) = f(j) \ \supset\  i = j)
      \end{array} )
 \end{array}\\
\lefteqn{(\funtype{\ttnat}{\ttint} \simeq \funtype{ \mathtt{NAT}}{
\mathtt{int}})}\\
& = & (\ttint \simeq \ttint);\\&& (\lambda (i : \ttint): \mathtt{0} \leq i) =
                              (\lambda (i : \ttint): 
i\leq \mathtt{0}\  \supset\ i = 0); (\ttint\simeq\ttint)
\end{eqnarray*}
\end{Eg}

A basic question during typechecking is whether two types are
\emph{compatible}, that is,  have the same maximal supertype.  
Two types are said to be compatible if the type equivalence proof
obligations on their respective 
maximal supertypes are provable.  The provability of a formula $a$
under context $\Gamma$ is represented as $\vdash_\Gamma a$\@.  
\begin{Defn}{compatible}
\emph{Two types $A$ and $B$ are said to be compatible in context $\Gamma$
(in notation, $(A \sim B)_\Gamma$)
if $\vdash_\Gamma a$, for each $a$ in $(\mu(A)\simeq \mu(B))$.}\footnotemark
\end{Defn}
\footnotetext{ The PVS proof rules are described in 
Chapter~\ref{proof theory}.}


\begin{comment}
With subtypes, it is easy to define an empty subtype and then
declare a constant of such an empty subtype.  The context thus has to be
checked to disallow such an inconsistency.
\end{comment}
We now extend the
definition of $\delta$ to the case of subtypes so that it leaves the
predicate unchanged but expands the definition of the supertype.
%
\begin{Defn}{expanded type with subtypes}
\begin{eqnarray*}
  \delta(\Gamma)(\{x : T \vbar a\}) & = & \{x : \delta(\Gamma)(T) \vbar a \}
\end{eqnarray*}
\end{Defn}

We now extend the definition $\tau$ to the case of subtypes.  Here we
could force $\tau$ to always return a maximal supertype but this is not
done in Definition~\ref{defn:type rules with subtypes}  since it
would weaken the soundness theorem without significantly simplifying the
definition of the type rules.  The typechecking of contexts has to be
modified to generate a {\em nonemptiness\/} proof obligation for the type of
any constant declaration.  A constant of an empty type would lead to an
inconsistent context, and this would mean that constant declarations are
not conservative extensions.  This modification to
Definition~\ref{defn:type rules} is not needed for soundness since an
inconsistent context makes soundness trivial.  It is needed to show that
constant declarations and definitions are conservative extensions.  Note
that with subtypes, the type rule for an application is modified to check
that the domain type of the function is compatible with the type of its
argument, and that the argument satisfies any constraints imposed by the
domain type of the function.  The case of projection expressions is also
not straightforward since the argument type can be a subtype of a tuple
type.  In this case, we use the direct supertype (see
Definition~\ref{defn:direct supertype}) which must be a tuple type.
%
\begin{Defn}{type rules with subtypes}
\begin{eqnarray*}
  \tau()(\Gamma, c : T) & = & \ttcontext, \mbox{ if } \Gamma(c)
\mbox{ is undefined},\\& &
\tauGamma{T} =   \tttype{},\\& &
      \tau()(\Gamma) = \ttcontext,  \mbox{ and } \\& & 
   \vdash_\Gamma (\exists (x: T) : \tttrue)\\
%
  \tauGamma{\{x : T \vbar a\}} & = & \tttype{}, \mbox{ if } \Gamma(x)
\mbox{ is undefined}, \\& & 
    \tauGamma{T} = \tttype{}, \mbox{ and } \tau(\Gamma, x : \ttvar{}\ T)(a) = \ttbool \\
%
  \tauGamma{f\ a} & = &  B, \mbox{ where } \mu_0(\tauGamma{f}) =
                                                 \funtype{A}{B},\\&  & 
                   			\tauGamma{a} = A',\\ & &
					(A \sim A')_\Gamma,\\& & 
                                        \vdash_\Gamma \pi(A)(a) \\
 \tauGamma{\proj{i}~a} & = & A_i, \mbox{ where } \mu_0(\tauGamma{a}) =
\tupletype{A_1, A_2}
\end{eqnarray*}
\end{Defn}

\begin{Eg}{typechecking subtypes}
Let $\Gamma$ contain the above declarations of $\ttint$, $\ttnat$,
$\mathtt{0}$, 
$\leq$, and {\tt
natinjection}.
\begin{eqnarray*}
\lefteqn{\tauGamma{\{i : \ttint \vbar 0\leq i\}}}\\
  & = & \tttype\\
\lefteqn{\tauGamma{(\lambda (f : \mathtt{ natinjection}) : f(0))(\lambda
(i: \mathtt{nat}): i)}}\\
& = &
\delta(\Gamma)(\ttnat), \mbox{ if}\\& &
( \mathtt{natinjection} \sim \funtype{\ttnat}{\ttnat})_\Gamma,\\& & 
\vdash_\Gamma \forall (j, k : \ttnat): (\lambda (i: \ttnat): i)(j) = (\lambda
(i: \ttnat) : i)(k) \ \supset\  j = k, \\& &
(\ttint \sim \ttnat)_\Gamma, \mbox{ and}\\& & 
\vdash_\Gamma \mathtt{0}\leq \mathtt{0}
\end{eqnarray*}

\end{Eg}

Only one additional clause to Definition~\ref{defn:meaning function} is
needed to capture the 
semantics of predicate subtypes.
\begin{Defn}{meaning function with subtypes}
\begin{eqnarray*}
  \lefteqn{\Mgamma{\{x : T \vbar a\}}}\\
& = & \{ y \in \Mgamma{T} \vbar
\mathcal{M}(\Gamma, x: \ttvar~T\vbar \gamma\{x
\gets y\})(a) = {\oneb} \} 
\end{eqnarray*}
\end{Defn}

\begin{Eg}{semantics of predicate subtypes}
If we assign the usual truth table interpretation to the Boolean function
$\supset$: 
\begin{eqnarray*}
\lefteqn{\Mgamma{\{f : \funtype{\ttbool}{\ttbool} \vbar \forall (x: \ttbool): x\ \supset\ f(x)\}}}\\
& = & \{\{\pair{\zerob, \zerob}, \pair{\oneb, \oneb}\},
\{\pair{\zerob,\oneb}, \pair{\oneb, \oneb}\}\}.
\end{eqnarray*}
\end{Eg}

The following useful propositions are easily proved from the definitions
given above.   Proposition~\ref{max-type-correct} asserts that the
maximal supertype of a type is well typed.  Proposition~\ref{maximal-subset}
asserts that the denotation of a type is a subset of the denotation of its
maximal supertype.  Proposition~\ref{maximal-equality} asserts that
if all the proof obligations in $(A\simeq A')$ are valid relative to
a given assignment $\gamma$ for context $\Gamma$, then
the denotations of $A$ and $A'$ under $\gamma$ are equal. 
%
\begin{prop}\label{max-type-correct}
  If $\tau()(\Gamma) = \ttcontext$ and
  $\tauGamma{A} = \tttype{}$, then $\tauGamma{\mu(A)} = \tttype{}$.
\end{prop}
\begin{prop}\label{maximal-subset}
  If $\tau()(\Gamma) = \ttcontext$, 
    $\tauGamma{A} = \tttype{}$,
   and $\gamma$ satisfies $\Gamma$, 
then
\begin{enumerate}
\item   $\Mgamma{A} \subseteq \Mgamma{\mu(A)}$ and
\item   $\Mgamma{A} \subseteq \Mgamma{\mu_0(A)}$.
\end{enumerate}
\end{prop}
\begin{prop}\label{maximal-equality}
  If $A$ and $A'$ are maximal
types in context $\Gamma$, i.e.,
\begin{enumerate}
\item $\tau()(\Gamma) = \ttcontext$, 
\item $\tauGamma{A} = \tauGamma{A'} = \tttype{}$,
\item $\mu(A) = A$ and $\mu(A') = A'$
\end{enumerate}
and
 for each $a$  in $(A \simeq A')$, %%\memo{What about true/false?}
\begin{enumerate}
\item $a\equiv \tttrue$, or 
\item $a\equiv (a_1 = a_2)$ and $\Mgamma{a_1} = \Mgamma{a_2}$
holds,
\end{enumerate}
then $\Mgamma{A} = \Mgamma{A'}$.\footnote{ 
We remind the reader that the formulas $a$ in $(A\simeq A')$ are equalities,
but we have not yet formally introduced equality into the language.   } 
\end{prop}
\begin{prop}\label{maximal-subtype}
If $\tau()(\Gamma) = \ttcontext$ and $\tauGamma{T} = \tttype$,
then $\Mgamma{T} = \Mgamma{\{x : \mu(T) \vbar \pi(T)(x)\}}$.
\end{prop}

We can now examine the updated forms of the invariants given by
Theorems~\ref{type construction},
\ref{simple-type-semantics},
and~\ref{simple-term-semantics}\@.  
The proof of Theorem~\ref{type construction} remains straightforward.
\begin{comment}
Theorems~\ref{simple-type-semantics}
and~\ref{simple-term-semantics} must be proved with a simultaneous
induction since types now contain terms.
\end{comment}
The statement of
Theorem~\ref{simple-term-semantics} must now
be strengthened to include soundness, that is, if $\vdash_\Gamma a$
and $\gamma$ satisfies $\Gamma$, then 
$\Mgamma{a} = \oneb$\@.  For now, we assume soundness
(Theorem~\ref{proof-soundness})  since we
have not yet presented the proof rules.  
%
\begin{theorem}[type soundness]
\label{subtype-type-semantics}
If $\tau()(\Gamma) = \ttcontext$, $\gamma$ satisfies $\Gamma$, and
$\tauGamma{A} = \tttype$ then $\Mgamma{A}\in U$.
\end{theorem}
\begin{proof}
There is only one new case to add to the induction proof of
Theorem~\ref{simple-type-semantics}, namely, when 
 $A\equiv \{x : T \vbar a\}$.   In this case, by
Definition~\ref{defn:type rules with subtypes},
$\tauGamma{T} = \tttype$, so by the induction hypothesis,
$\Mgamma{T} \in U$.  Since, by
Definition~\ref{defn:meaning function with subtypes}, $\Mgamma{A} \subseteq
\Mgamma{T}$, we have $\Mgamma{A}\in U$ by
Definition~\ref{defn:type universe}\@.
\end{proof}
\begin{theorem}[term soundness]\label{subtype-term-semantics}
If $\tau()(\Gamma) = \ttcontext$, $\gamma$ satisfies $\Gamma$, and $\tauGamma{a} = A$ then $\Mgamma{a} \in
\Mgamma{A}$\@.  
\end{theorem}
\begin{proof}  There are two affected cases in the proof from that of
Theorem~\ref{simple-term-semantics}, namely, those of application
and projection.  The case of projection expressions is straightforward
given Proposition~\ref{maximal-subset}.

When 
 $a \equiv (f\ b)$, by Definition~\ref{defn:type rules with subtypes},
we have that $\tauGamma{f} = \funtype{B}{A}$ and $\tauGamma{b} = B'$.
Let $X$ be $\Mgamma{B}$, $X'$ be $\Mgamma{B'}$,  and $Y$ be $\Mgamma{A}$.
Then by Definition~\ref{defn:meaning function}, $\Mgamma{\funtype{B}{A}} =
Y^X$\@.  By the induction hypotheses, $\Mgamma{f}\in Y^X$ and
$\Mgamma{b}\in X'$\@.  
By Definition~\ref{defn:type rules with subtypes},
soundness of the proof rules (Theorem~\ref{proof-soundness}), and
Propositions~\ref{maximal-subset} and 
~\ref{maximal-equality}, there is a 
maximal supertype $\mu(B)$ of both $B$ and $B'$ such that $X$ and $X'$
are both subsets of $\Mgamma{\mu(B)}$\@.  Since, by
Definition~\ref{defn:type rules with subtypes}, $\vdash_\Gamma{}\pi(B)(b)$,
and by Proposition~\ref{maximal-subtype}, $\Mgamma{B} =
\Mgamma{\{x : \mu(B)\vbar \pi(B)(x)\}}$, we have $\Mgamma{b}\in
\Mgamma{B}$, and hence by Definition~\ref{defn:meaning function}, $\Mgamma{(f\
b)}\in  \Mgamma{A}$\@.  
\end{proof}



\section{Summary}

PVS features a form of subtyping where it is possible to form the subtype
of a type satisfying a given predicate on the type.  This kind of
subtyping introduces several delicate semantic issues into PVS.  A term
can now have several types since, for example, the term corresponding to
the number {\tt 2} can be a prime number, an even number, a natural
number, an integer, a rational number, or a real number.  When the
expected type is a subtype, the canonical type of the actual term must be
compatible with the expected type, that is, the two maximal supertypes
must be equivalent and the actual term must satisfy any subtype
constraints imposed by the expected type.  We have defined the notions of
maximal supertype, subtype constraints, type equivalence, and
compatibility.  These notions are used to define the type rules and
semantics of the simply typed fragment of PVS extended with subtypes.
Note that both type equivalence (and hence, compatibility) and type
correctness are undecidable.  Proof obligations generated during
typechecking are the only source of such undecidability.  The
modularization of the type system into a decidable part consisting of the
simply typed fragment,  and the proof obligations generated by
subtyping, is perhaps the most significant design consideration in the PVS
language.




\chapter{Dependent Types}\label{dependent}

The PVS language fragment described thus far is already quite expressive.
It employs definitional equivalence between types and contains
predicate subtypes.  It is undecidable whether an expression in this
fragment is type-correct because of the proof obligations that arise with
respect to predicate subtypes and type equivalence.  
The next step is the addition of type dependencies between the components
of a type.  This extension considerably
enhances the utility of this type system.  It is also a natural extension
given predicate subtyping which already allows types that
{\em depend\/} on free variables in the predicates.  
With dependent typing, we can
make the type of one component of a product depend on the value of another
component, or the type of the range of a function vary according to its
argument value.

\begin{comment}
Let $V$ be a metavariable that ranges over types or {\em
type bindings\/} of the form $x : T$.
\end{comment}
A dependent product type is
written as $\tupletype{x : A, B}$.  A dependent function type is written
as $\funtype{x : A}{B}$.
Any product or function type can be transformed into a dependent type by
inserting dummy type bindings.  Conversely, any dummy type bindings that
do not actually bind any variable occurrences can be
removed.  The type rules and semantics below will assume that all product and
function types are presented as dependent types.

\begin{Eg}{dependent types}
$$
\begin{array}{l}
\tupletype{i : \ttnat, \{j: \ttnat \vbar j\leq i\}}, \\
\tupletype{i : \ttnat, \funtype{\{j: \ttnat \vbar j\leq i\}}{\ttbool}},\\
\funtype{i: \ttint}{\{j : \ttint \vbar i\leq j\}}.
\end{array}$$
\end{Eg}


Before we treat dependent types, we update the definitions of the set
of free variables and substitution to account for the fact that
with subtyping and dependent typing, both free and bound variables
can occur in terms and types.  This is needed for the next step
where we try to remove type dependencies by substituting a term into
a dependent type.  
\begin{Defn}{free variables for types}
\begin{eqnarray*}
%%  FV(\Gamma)(s) & = & \emptyset, \mbox{ if } \itkind(\Gamma(s)) \neq
%%\ttvariable\\
FV(\Gamma)(\funtype{x : A}{B}) & = & FV(\Gamma)(A)\union
                   (FV(\Gamma, x: \ttvar~A)(B) - \{x\}) \\
%
FV(\Gamma)(\tupletype{x: A, B}) & = &
  FV(\Gamma)(A)\union (FV(\Gamma,
x:\ttvar~A)(B) - \{ x\}) \\
%
FV(\Gamma)(\{x : A \vbar a\}) & = & FV(\Gamma)(A) \union
(FV(\Gamma, x:\ttvar~A)(a) - \{x\})
\end{eqnarray*}
\end{Defn}

\begin{Defn}{substitution for types}
\begin{eqnarray*}
\lefteqn{\funtype{x : A}{B}[a_1/x_1,\ldots, a_n/x_n]}\\ & = & \funtype{y :
A[a_1/x_1,\ldots,a_n/x_n]}{B[y/x, a_1/x_n, \ldots, a_n/x_n]} \\
%
\lefteqn{\tupletype{x: A, B}[a_1/x_1, \ldots, a_n/x_n]}\\ & = &
\tupletype{y: A[a_1/x_1,\ldots, a_n/x_n], 
B[y/x, a_1/x_1,\ldots,a_n/x_n]} \\
%
\lefteqn{\{x : A\vbar a\}[a_1/x_n,\ldots, a_n/x_n]}\\ & = & \{y : A[a_1/x_1,\ldots,
a_n/x_n] \vbar  a[y/x, a_1/x_n, \ldots, a_n/x_n] \}
\end{eqnarray*}
where y is a fresh variable.
\end{Defn}

%\memo{Need to extend substitution to types.}

\begin{comment}
To determine whether a given expression is of a given dependent type, we
must define the notion of a substituting the expression into the type.  If
$\tauGamma{a}$ is a (possibly dependent) product type and $A$ is a
(possibly dependent) product type of the form $\tupletype{x : A_1, A_2}$, then
$a/A$ is the type obtained by ``substituting'' $a$ in $A$: if $a$ is of
the form $(\listwo{a}{n})$ then $a/A$ is $\tupletype{x : A_1,
a_2/(A_2[a_1/x])}$,
else if
$a$ is not of the form $(\listwo{a}{n})$, then we use $\proj{i}\ a$ in place
of $a_i$\@.  Similarly, if $\tauGamma{a}$ is a function type of the form
$\funtype{x : T}{S}$, then if $a$ is of the form $(\lambda (x : T): b)$,
then $(\lambda (x: T): b)/\funtype{y : A}{B}$ is $\funtype{y :
A}{b[y/x]/B}$\@.  If $a$ is not a lambda-expression but $\tauGamma{a}$ is
of the form $\funtype{z:T}{S}$, then $a/\funtype{y : A}{B}$ is just
$(\lambda (x : T): a(x))/\funtype{y : A}{B}$.  In every other case, $a/A$
is just $A$\@.
%%\memo{Need subtype case for substituting.  Put in separate definition.}
\begin{Eg}{substituting into dependent types}
\begin{eqnarray*}
\lefteqn{(5, 3)~/~\tupletype{i : \ttnat, \{j: \ttnat \vbar j\leq i\}}}\\
  & \equiv &
\tupletype{i: \ttnat, \{j: \ttnat \vbar j\leq 5\}}\\
%
\lefteqn{(3, (\lambda (k: \{j : \ttnat\vbar j < 3\}): k))~/~\tupletype{i : \ttnat,
\funtype{k : \{j: \ttnat \vbar j< i\}}{\{j: \ttnat \vbar j\leq k\}}}}\\
&
\equiv &
\tupletype{ i : \ttnat, \funtype{k : \{j: \ttnat \vbar j< 3\}}{\{j: \ttnat \vbar j\leq k\}}} 
\end{eqnarray*}
\end{Eg}
\end{comment}

\newcommand{\constrain}[2]{#1\backslash #2}
The definition of $\mu$ has to be modified slightly for dependent types.
The definition is first extended to type bindings, $\mu(x : T) = x :
\mu(T)$.  The definition for the case of dependent function types
is unchanged so that $\mu(\funtype{x : A}{B}) = \funtype{x :
A}{\mu(B)}$\@.   The definition for the product case is more delicate
since the definition $\mu(\tupletype{x : A, B}) = \tupletype{x : \mu(A),
\mu(B)}$ results in a loss of type information regarding the occurrences
of $x$ in $B$\@.\footnote{Doug Howe brought this problem to our attention.}
To ensure that type information regarding $x$ is
retained, we define a new operation $\constrain{T}{a}$
which constrains the subtype assertions in  type $T$ with an additional
assertion $a$\@.
\begin{Defn}{Adding subtype constraints}
\begin{eqnarray*}
\constrain{s}{a} & = & s \\
\constrain{\subtype{x}{T}{b}}{a} & = & \subtype{x}{T}{a \wedge b}\\
\constrain{\funtype{A}{B}}{a} & = &
\funtype{\constrain{A}{a}}{\constrain{B}{a}} \\
\constrain{\tupletype{A, B}}{a} & = & \tupletype{\constrain{A}{a}, \constrain{B}{a}}
\end{eqnarray*}
\end{Defn}

We can now define the maximal supertype operation for
dependent tuple types.
\begin{Defn}{Maximal supertype for dependent product types}
\begin{eqnarray*}
\mu(\tupletype{x : A, B}) & = & \tupletype{x : \mu(A), \constrain{B}{\pi(A)(x)}}
\end{eqnarray*}
\end{Defn}

The definition of $\pi$ for a dependent function type $\funtype{ y :
A}{B}$ is slightly different from that of an ordinary function type
since $\pi(B)$ can contain free occurrences of the variable $y$\@.
For example, $\pi(\funtype{i : \ttint}{\{j : \ttint \vbar i\leq j\}})$
must be  $\lambda (f : \funtype{i : \ttint}{\ttint}): (\forall (i:\ttint): i
\leq f(i))$\@.  The definition for dependent tuples remains essentially
unchanged from that of ordinary products.  
\begin{Defn}{constraint predicates for dependent types}
\begin{eqnarray*}
   \pi(\funtype{y:A}{B}) & = &  ( \lambda (x : \funtype{y:A}{\mu(B)}):
    (\forall (y: A): \pi(B)(x(y)))) \\
%
  \pi(\tupletype{y:A, B}) & = &
   ( \lambda (x: \tupletype{y : \mu(A), \constrain{\mu(B)}{\pi(A)(y)}}): \\&&\hspace*{.3in}
	\pi(A)(\proj{1}~x)\wedge \pi(B)(\proj{2}~x)[(\proj{1}~x)/y])
\end{eqnarray*}
\end{Defn}

\newcommand{\depfunexample}{\funtype{i : \ttint}{\{j : \ttint \vbar i\leq
j\}}}
\newcommand{\deptupexample}{\tupletype{i : \ttnat, \{j: nat \vbar j\leq i\}}}
\begin{Eg}{dependent type predicates}
\begin{eqnarray*}
\mu(\depfunexample{}) & = & \funtype{i : \ttint}{\ttint}\\
\pi(\depfunexample{}) & = & \lambda (f : \funtype{i : \ttint}{\ttint}):\\&&\hspace{.3in}
\forall (i: \ttint): (\lambda (j : \ttint): i\leq j)(f(i))
\end{eqnarray*}
\end{Eg}

The definition of $\simeq$ must also be massaged slightly for dependent
types.  Recall that $\simeq$ checks whether two maximal types are
equivalent by generating proof obligations as needed.  This is the
basic operation for checking whether the expected type of an expression
is compatible with its actual type.  
The subtlety now is that the expected type might be
a dependent type where the actual type is not.  Consider the
case of the pair $\pair{5, (\lambda (x : \{j:\ttnat \vbar j\leq 5\}): x)}$
whose type would be computed 
by $\tau$ as $\tupletype{i : \ttnat, \funtype{\{j : \ttnat\vbar j\leq 5\}}{\{j : \ttnat\vbar j\leq 5\}}}$
where the expected type might be 
$\tupletype{i : \ttnat, \funtype{\{j : \ttnat \vbar j\leq i\}}{\{j : \ttnat \vbar
j\leq i\}}}$\@.  To cope with this, 
we will allow the option of two maximal types, say $A$ and $B$, to be
compared using $\simeq$
in the context of an expression $a$\@.  This is indicated by the notation $(A\simeq B)/a$\@.  
Note that $(A\simeq B)/a$ is sensible only when $A$ and $B$
are maximal types.   The missing cases in
Definition~\ref{defn:type equivalence proof obligations}
are included in
Definition~\ref{defn:type equivalence for dependent types}.    For a list
of formulas $a_1, \ldots, a_n$, let 
$(\forall (x : T): a_1,\ldots, a_n)$ represent the list
$(\forall (x : T): a_1), \ldots, (\forall (x : T): a_n)$\@.\footnote{Note that the type-correctness of the proof
obligation $(\pi(A) = \pi(A'))$ in
Definition~\ref{defn:type equivalence for dependent types}  depends on the
\emph{prior} proof obligations 
$\mu(A)\simeq \mu(A')$.}
\begin{Defn}{type equivalence for dependent types}
\begin{eqnarray*}
  (s \simeq s)/a & = & \tttrue \\
(\funtype{x:A}{B} \simeq \funtype{x':A'}{B'}) & = & (\mu(A)\simeq
\mu(A'));\\ & & 
(\pi(A) = \pi(A'));\\ & & 
 (\forall (x : A): (B \simeq B'[x/x'])) \\
%
(\funtype{x:A}{B} \simeq \funtype{x':A'}{B'})/a & = & (\mu(A)\simeq
\mu(A'));\\ & & 
(\pi(A) = \pi(A'));\\ & & 
 (\forall (x : A): (B \simeq B'[x/x'])/a(x)) \\
%
(\tupletype{x : A_1, A_2} \simeq \tupletype{y :
B_1, B_2})
& = &
    (A_1\simeq B_1);\\ & & (\forall (x : A_1) : (A_2\simeq
B_2[x/y]))\\
%
(\tupletype{x : A_1, A_2} \simeq \tupletype{y :
B_1, B_2})/a
& = &
    (A_1\simeq B_1)/(\proj{1}~a);\\ & & (A_2[(\proj{1}~a)/x]\simeq
B_2[(\proj{1}~a)/y])/(\proj{2}~a)\\
%
 (A \simeq B)/a & = & \ttfalse, \mbox{ otherwise.}
\end{eqnarray*}
\end{Defn}
As with $(A\sim B)_\Gamma$, the notation $(A \sima 
B)_\Gamma$ indicates that  all the proof obligations $a'$ in
$(\mu(A)\simeq \mu(B))/a$ are provable, that is,  $\vdash_\Gamma a'$\@.  

With dependent types, the type rules must be modified so as to augment the
context suitably to account for any dependencies.  
We will give the definitions only for dependent type constructions.

%\memo{need to handle type dependencies in expected for application case.}
\begin{Defn}{type rules with dependent types}
\begin{eqnarray*}
\tauGamma{\tupletype{x : A, B}} & = &
\tttype{}, \mbox{ if } \Gamma(x) \mbox{ is undefined},\\
& & \tauGamma{A} = \tttype{}, \mbox{ and}\\
& & \tau(\Gamma, x : \ttvar{}\ A)(B) = \tttype{} \\
%
%\tauGamma{\tupletype{T_1, U_2, \ldots, U_n}} & = &
%\tttype{}, \mbox{ if } \tauGamma{T_1} = \tttype{}, \mbox{ and}\\
%& & \tauGamma{\tupletype{U_2,\ldots, U_n}} = \tttype{} \\
%
\tauGamma{\funtype{x: A}{B}} & = & \tttype{},
    \mbox{ if } \Gamma(x) \mbox{ is undefined},\\
& & \tauGamma{A} = \tttype{}, \mbox{ and}\\
& & \tau(\Gamma, x : \ttvar{}\ A)(B) = \tttype{}\\
%
  \tauGamma{f\ a} & = &  B', \mbox{ where } \mu_0(\tauGamma{f}) = \funtype{x :
A}{B},\\ 
 & &                   			\tauGamma{a} = A', \\ & & 
					(A \sima  A')_\Gamma, \\
& &
                                        B' \mbox{ is } B[a/x],\\ & & 
                                        \vdash_\Gamma \pi(A)(a)\\
%
 \tau(\Gamma)(\lambda (x : A): a) & = &
         \funtype{x : A}{B}, \mbox{ where}\\
%& & A  = \tupletype{x_1 : T_1, \ldots, x_n : T_n}, \\
& & B = \tau(\Gamma, x : \ttvar{}\ A)(a)\\ 
%B' = B[(\proj{1}\ x)/x_1, \ldots, (\proj{n}\
%x)/x_n] \\ & & 
%\mbox{if } \mbox{ for } 1\leq i\in \{1, 2\},\\& & 
%\tau(\Gamma, x_1 : \ttvar{}\ T_1,\ldots, x_{i-1} : \ttvar{}\ T_{i-1})(T_i) = \tttype{}, \\
%& & \mbox{and for } 1\leq i, j \in \{1, 2\}, i\neq j, x_i \not\equiv x_j \\
%
\tauGamma{\proj{1}~a} & = & A_1, \mbox{ where } \mu_0(\tauGamma{a}) =
\tupletype{x: A_1, A_2} \\
%
\tauGamma{\proj{2}~a} & = & A_2[(\proj{1}~a)/x], \mbox{ where } \mu_0(\tauGamma{a}) =
\tupletype{x: A_1, A_2} \\
\end{eqnarray*}
\end{Defn}
%%\memo{Need proj case above.}

\begin{Eg}{dependent typing}
\begin{eqnarray*}
\tauGamma{\tupletype{x : \ttbool,  \{ y : \ttbool \vbar x \supset
y\}}}
& = & \tttype\\
\tauGamma{\funtype{x : \ttbool}{\{ y : \ttbool \vbar x \supset 
y\}}} & = & \tttype
\end{eqnarray*}
\end{Eg}
Before we can assign meanings to dependent types, we must
augment our definition of the universe $U$ to contain sets corresponding
to these constructions.  If $F$ is a function with domain set $X$ and
a range $Y$, which is a set of sets, we can  define $\Sigma{F}$ to be the set 
$\{ \pair{x, y} \vbar x\in \textit{dom}(F), y\in F(x)\}$ and
$\Pi{F}$ to be the set $\{f \vbar (\forall x\in \textit{dom}(F): f(x)\in
F(x))\}$\@.   Note that $\Pi{F} \subseteq \bigcup_{X\in \Sigma{F}}\wp(X)$ but we
include $\Pi{F}$ in the universe $U$ defined below for simplicity\@.
We can drop $X\times Y$ and $X^Y$ from the universe definition since
$X\times Y$ can be obtained from $\Sigma{F}$ by defining an
$F$ with domain $X$ that always returns $Y$, and similarly,
$X^Y$ can be obtained by $\Pi{F}$ where $F$ is defined to with domain $Y$
to always return $X$\@.  The universe $U$ can then be redefined as below.  
\begin{Defn}{type universe with dependent types}
\begin{eqnarray*}
  U_0 & = & \{ {\twob}, {\reals}  \} \\
  U_{i+1} & = & U_i  \\& &
               %\union \{X\times Y \vbar X, Y \in U_i\} \\ & & 
                         %\union  \{X^Y \vbar X, Y \in U_i \} \\ & & 
                          \union  \bigcup_{X\in U_i}\wp(X)\\& & 
                           \union\; \{\Sigma{F} \vbar F \in W_i\}\\& &
                            \union\; \{\Pi{F} \vbar F \in W_i\}\\
  W_i & = & \bigcup_{X\in U_i} U_i^X \\
  U_{\omega} & = & \bigcup_{i\in\omega} U_i\\
  U & = & U_{\omega}
\end{eqnarray*}
\end{Defn}

One very important consequence of the above extension of the universe
is that all type dependencies must be bounded in the sense that
if $B$ is a type expression with a single free variable $x$ of type $A$,
then it must be the case that for any set $\mean{A}$ representing $A$,
there is a bound $n$ such that for any $z$ in $\mean{A}$, the meaning of 
$B$ under $\{x \gets z\}$ must be in $U_n$\@.  This property is easily
proved by induction on the structure of a PVS type since the parameter
$x$ can appear only in the predicate part of a subtype where the
rank of the meaning of the resulting type cannot vary with the value
of $x$\@.   In
particular, there is no way to define a type constructor $T^n$ in PVS that
returns the $n$-tuple $[\underbrace{T, [\ldots, T]}_n]$ for a given $n$
since this would entail an unbounded dependency.  
If unbounded type dependencies were allowed in PVS,
one can construct a dependent type such as $\funtype{n : nat}{T^n}$
whose representation is not in $U$ as defined above.    
  

\begin{comment}
A simpler alternative might be to define a second universe
U2 which starts from U1 and adds the dependent types.  This will work fine
in the absence of parametric theories since each type contains only a
fixed number of type dependencies.  The introduction of parametric
theories might require a third iteration since they introduce a third
form of dependency.
\end{comment}

\begin{comment}
Seeking advice on PVS semantics:  I'd like to know if the following
universe construction provides a satisfactory model for PVS types.  The
PVS simple type system consists of the base types bool and real and the
tuple [A_1,...,A_N] and function [A->B] types.  The flg. universe U1
(where (U_n)^n is the n-fold Cartesian product and (U_n)^(U_n) is the
function space) provides a model: 
 U_0     = {2, R}
 U_{n+1} = U_n UNION (U_n)^n UNION (U_n)^(U_n)
 U_{omega} = SUP_n U_n
 U1      = U_omega
%
The next step is to add subtypes {x:A \vbar p(x)}.  The above universe can be
easily modified to accommodate these by adding the subsets of the finite
simple types. Here P(X) is the power-set of X.  
%
 U_{n+1} = U_n UNION {U_n}^n UNION (U_n)^(U_n) UNION P(U_n)
%
The type system is then extended with dependent function [x:A->B(x)]
and tuple types [x_1: A, B].  The simple way to construct a universe is to
say that there are only finite length chains of dependencies and these are
captured by the universe U2 below which is constructed iteratively from
U1.  Here, if F is a function graph, #F is the set
{<x, y> \vbar x in domain(F), y in F(x)}. 
%
  U2_0 = U1
  U2_{n+1} = U_n UNION (U_n)^n UNION (U_n)^(U_n) UNION P(U_n)
             UNION {#F \vbar F in {(U2_n)^X \vbar X in U2_n}}
  U2_{omega} = SUP_n U2_n
  U2 = U2_{omega}
%
The reason this works is that it is easy to see by induction on the
type structure that we can assign a degree k to a given type,
where the degree of a nondependent type is 0, and the degree of
a type of the form [x: S, T(x)] is max(deg(S), deg(T)) + 1.
Note that if a type has degree k, then it is introduced in the
kth iteration of the definition of U2.  Dependent function types
are ignored here since they are defined as subsets of the corresponding
tuples so that they are introduced in the next iteration by the power-set
construction.
%
There really is no need for a second universe construction here since the
above argument would also work if the U2 iteration was rolled into that of
U1.
%
Parametric theories pose a serious difficulty since we now have types and
constants that can depend on types.  The context provides a
meaning function to each theory that maps the meanings of the
actual parameters to the meanings of the symbols declared in the theory.
We need to constrain the meaning function for a theory so that
the degree of each type name in the theory depends only on the
degrees of the type parameters.  This is a basic invariant in PVS that
the dependency between types and individuals only affects the
predicate part of the type and not its structure.  One cannot, for
instance, define a tuple whose first component is n:nat, and whose
second component is the n-fold tuple T^n.   Note that this
invariant is preserved by the way in which theories are actually used.
An interpretation of a theory simply attaches a definition to an
undefined type and the meaning of this definition will also have the
boundedness property.  
\end{comment}









The meaning function for dependent types is obtained by adding the
cases corresponding to 
dependent product and function types.  All the other cases are unchanged from
Definition~\ref{defn:meaning function with subtypes}\@.  Note that the
semantic definition  for dependent types is 
equivalent to the nondependent one when there are no dependencies.
\begin{Defn}{meaning function with dependent types}
\begin{eqnarray*}
  \Mgamma{\tupletype{x : A, B}} & = &
    \Sigma{F}, \mbox{ where }\\& &
     F \mbox{ maps } z \in \Mgamma{A} \mbox{ to }\\& & \mathcal{M}(\Gamma, x : \ttvar{}\
A\vbar \gamma \{x\gets z \})(B)\\
%
  \Mgamma{\funtype{x:A}{B}} & = &  \Pi{F}, \mbox{ where }\\& &
                        F \mbox{ maps } z \in \Mgamma{A} \mbox{ to }\\& &  \mathcal{M}(\Gamma,
x : \ttvar{}\  A\vbar \gamma\{x\gets z\})(B)  
%    
\end{eqnarray*}
\end{Defn}

\begin{Eg}{meaning function with dependent types}
\begin{eqnarray*}
\Mgamma{\tupletype{x : \ttbool,  \{ y : \ttbool \vbar x \supset 
y\}}}
& = & \{\pair{\zerob, \zerob}, \pair{\zerob, \oneb},
\pair{\oneb, \oneb}\}\\ 
\Mgamma{\funtype{x : \ttbool}{\{ y : \ttbool \vbar x \supset 
y\}}} & = & \{\{\pair{\zerob, \zerob}, \pair{\oneb, \oneb}\},\\&&\hspace*{1ex}
              \{\pair{\zerob, \oneb}, \pair{\oneb, \oneb}\}\}
\end{eqnarray*}
\end{Eg}

We now need to show that the extensions corresponding to dependent types
preserve the properties in Theorems~\ref{subtype-type-semantics}
and~\ref{subtype-term-semantics}, namely, 
$\Mgamma{T} \in U$ and $\Mgamma{a}\in \Mgamma{\tauGamma{a}}$\@.
For the former, we prove a stronger theorem that incorporates the
rank-boundedness of dependent types.
\begin{theorem}[rank bounded type semantics]\label{dependent-type-semantics}
If $B$ is a pretype, $x_1,\ldots, x_n$ is a list of symbols,
$A_1,\ldots, A_n$ is a list of pretypes such that  
\begin{enumerate}
\item $\tau()(\Gamma, x_1:\ttvar~A_1, \ldots, x_n:\ttvar~A_n) = \ttcontext$,
\item $\tau(\Gamma, x_1:\ttvar~A_1, \ldots, x_n:\ttvar~A_n)(B) = \tttype$,
and
\item $\gamma$ is an assignment satisfying $\Gamma$,
\end{enumerate}
then
there is an $i$ such that for any list of values
$z_1,\ldots, z_n$ where $\gamma\{x_1\gets
z_1\}\ldots\{x_n\gets z_n\}$ is a satisfying assignment for
$\Gamma, x_1:\ttvar~A_1, \ldots, x_n:\ttvar~A_n$,  we have
$$\mathcal{M}(\Gamma, x_1:\ttvar~A_1, \ldots, x_n:\ttvar~A_n\vbar
\gamma\{x_1\gets z_1\}\ldots\{x_n\gets z_n\})(B) \in U_i.$$
\end{theorem}
\begin{proof}
The proof is by structural induction on the pretype $B$\@.
Let $\Gamma'$ denote $\Gamma, x_1:\ttvar~A_1, \ldots, x_n:\ttvar~A_n$,
$\gamma'$ denote $\gamma\{x_1\gets z_1\}\ldots\{x_n\gets z_n\}$, and 
$\mean{C}$ denote $\mathcal{M}(\Gamma'\vbar
\gamma')(C)$.  
\begin{enumerate}
\item $B\equiv s$: Since $\mean{B}$ is just $\gamma(B)$ by
Definition~\ref{defn:meaning function}, we have that there is an $i$
such that $\mean{B}\in U_i$ regardless of the choice of values
$z_1,\ldots, z_n$\@.  

\item $B\equiv \{y:T\vbar a\}$: By the induction hypothesis, we know that
for some $j$, it is always the case that $\mean{T}\in U_j$\@.  By
Definition~\ref{defn:meaning function with subtypes}, we have that
$\mean{B}\subseteq \mean{T}$ so if we let $i = j+1$, then
by Definition~\ref{defn:type universe with dependent types},
it is always the case that $\mean{B}\in U_i$\@.

\item $B \equiv \funtype{y:C}{D}$: By
Definition~\ref{defn:type rules with dependent types},
$\Gamma'(y)$ is undefined, $\tau(\Gamma')(C) = \tttype$,
$\tau()(\Gamma', y:\ttvar~C) = \ttcontext$, and
$\tau(\Gamma', y:\ttvar~C)(D) = \tttype$.
By the induction hypothesis, for some $j$,
it is always the case that $\mathcal{M}(\Gamma'\vbar\gamma')(C)\in U_j$,
and for some $k$, it is always the case that for any satisfying
assignment $\gamma'\{y\gets w\}$ for $\Gamma', y:\ttvar~C$,
we have $\mathcal{M}(\Gamma', y:\ttvar~C\vbar \gamma'\{y\gets
w\})(D)\in U_k$\@.   Then the function $F$ mapping $w$ in
$\mathcal{M}(\Gamma')(C)$ to $\mathcal{M}(\Gamma', y:\ttvar~C\vbar
\gamma'\{y\gets w\})(D)$ is an element of $W_{j+k}$\@.
Letting $i$ be $j+k+1$, we have by
Definition~\ref{defn:meaning function with dependent types}
that $\mathcal{M}(\Gamma'\vbar\gamma')(B)$ is
$\Pi{F}$ and is hence an element of $U_i$ by
Definition~\ref{defn:type universe with dependent types}\@.

\item $B\equiv \tupletype{y: C, D}$:  Similar to the
previous case.  
\end{enumerate}
\end{proof}

By choosing $n$ to be $0$, the previous theorem yields the result 
that when $\tauGamma{B} = \tttype$, $\Mgamma{B}\in U$\@.

We next need to establish that for any preterm $a$,
if $\tauGamma{a} = A$, then $\Mgamma{a}\in \Mgamma{A}$\@.
The first step in this direction is the proof of the substitution lemma
below.
\begin{prop}\label{context-equivalence}
If $\tau()(\Gamma) = \tau()(\Gamma') = \ttcontext$ where
for each $s$, $\Gamma(s)$ is defined if and only if  $\Gamma'(s)$ is defined, and 
$\gamma$ is an assignment satisfying both $\Gamma$ and $\Gamma'$,
then
\begin{enumerate}
\item If  $\Gamma(s) = \ \Gamma'(s)$ (i.e., they are equal when
either $\Gamma(s)$ or $\Gamma'(s)$ is defined), then 
\begin{enumerate}
\item $\tau(\Gamma)(a) = \tau(\Gamma')(a)$, for any preterm $a$.
\item $\tau(\Gamma)(A) = \tau(\Gamma')(A)$, for any pretype $A$.
\end{enumerate}

\item $\Mgamma{A} = \mathcal{M}(\Gamma'\vbar\gamma)(A)$, when $\tauGamma{A} =
\tttype$.

\item $\Mgamma{a} = \mathcal{M}(\Gamma'\vbar\gamma)(a)$, for any preterm $a$
such that $\tauGamma{a}$ is defined.
\end{enumerate}
\end{prop}
%
%%\memo{Needs induction on preterms and pretypes.}
\begin{lemma}[substitution lemma]\label{substitution-lemma}
If $\tau()(\Gamma, x:\ttvar~A) = \ttcontext$,
$\tauGamma{a} = A$, then
\begin{enumerate}
\item If $\tau(\Gamma, x: \ttvar~A)(b) = B$, then\\
$\Mgamma{b[a/x]} = \mathcal{M}(\Gamma, x:\ttvar~A\vbar \gamma\{x\gets
\Mgamma{a}\})(b)$.

\item If $\tau(\Gamma, x: \ttvar~A)(C) = \tttype$, then\\
$\Mgamma{C[a/x]} = \mathcal{M}(\Gamma, x:\ttvar~A\vbar \gamma\{x\gets
\Mgamma{a}\})(C)$\@. 
\end{enumerate}
\end{lemma}
\begin{proof}
The proof is by simultaneous structural induction on the preterm $b$ and
the pretype $C$\@.  The following cases deal with the preterm $b$.
\begin{enumerate}
\item $b\equiv s$: If $s\equiv x$, then by
Definition~\ref{defn:meaning function with dependent types}, the left-hand
side $\Mgamma{b[a/x]}$ is 
$\Mgamma{a}$, and the right-hand side  $\mathcal{M}(\Gamma, x:\ttvar~A\vbar \gamma\{x\gets
\Mgamma{a}\})(b)$ is also $\Mgamma{a}$\@. 

If $s\not\equiv x$, then by
Definition~\ref{defn:meaning function with dependent types},
the left-hand side and the right-hand side
are both equal to $\gamma(s)$\@. 

\item $b \equiv (\lambda (y : C): d)$: Since $C$ can contain free
occurrences of $x$, we have by the induction hypothesis that
$\Mgamma{C[a/x]} = \mathcal{M}(\Gamma, x:\ttvar~A\vbar \gamma\{x\gets
\Mgamma{a}\})(C)$\@.  Also, $\Mgamma{(\lambda (y:C): d)[a/x]}$ is equal to
the set of ordered pairs $\pair{v, z}$ such that $v\in \Mgamma{C[a/x]}$
and $z = \mathcal{M}(\Gamma, y:\ttvar~C[a/x]\vbar \gamma\{y\gets
v\})(d[a/x])$\@.

By the induction hypothesis,
$\mathcal{M}(\Gamma, y:\ttvar~C[a/x]\vbar \gamma\{y\gets v\})(d[a/x])
= \mathcal{M}(\Gamma, y:\ttvar~C[a/x], x:\ttvar~A\vbar \gamma\{y\gets
v\}\{x\gets \Mgamma{a}\})(d)$\@.  Since $x$ does not occur free in
$C[a/x]$, by Proposition~\ref{context-equivalence} we can exchange the
occurrences of $y$ and $x$ so that 
$\mathcal{M}(\Gamma, y:\ttvar~C[a/x], x:\ttvar~A\vbar \gamma\{y\gets
v\}\{x\gets \Mgamma{a}\})(d) =
\mathcal{M}(\Gamma, x:\ttvar~A, y:\ttvar~C[a/x]\vbar \gamma\{x\gets \Mgamma{a}\}\{y\gets
v\})(d)$.

By Definition~\ref{defn:meaning function with dependent types},
the right-hand side is the set of ordered pairs of the form $\pair{v, z}$
such that $v\in \mathcal{M}(\Gamma, x: \ttvar~A \vbar \gamma\{x\gets
\Mgamma{a}\})(C)$
and $z = \mathcal{M}(\Gamma, x: \ttvar~A, y: \ttvar~C\vbar \gamma\{x\gets
\Mgamma{a}\}\{y\gets v\})(d)$\@.  By Proposition~\ref{context-equivalence}
and the induction hypothesis, we know that
$\mathcal{M}(\Gamma, x: \ttvar~A, y: \ttvar~C\vbar \gamma\{x\gets
\Mgamma{a}\}\{y\gets v\})(d) =
\mathcal{M}(\Gamma, x:\ttvar~A, y:\ttvar~C[a/x]\vbar \gamma\{x\gets \Mgamma{a}\}\{y\gets
v\})(d)$, and hence it follows that the two sets of ordered pairs are
equal.

\item $b\equiv (f\ c)$:  In this case, $b[a/x]\equiv (f[a/x]~c[a/x])$
and the conclusion follows easily from the induction hypothesis
and Definition~\ref{defn:meaning function with dependent types}\@.

\item $b\equiv (b_1, b_2)$: 
The conclusion follows easily from 
Definitions~\ref{defn:substitution},~\ref{defn:meaning function with
dependent types}, and the induction hypotheses.


\item $b\equiv (\proj{i}~c)$: This case is also straightforward since
$b[a/x] \equiv (\proj{i}~c[a/x])$,  
and by the induction hypothesis,
$\mathcal{M}(\Gamma, x: \ttvar~a \vbar \gamma\{x \gets \Mgamma{a}\})(c) =
\Mgamma{c[a/x]}$.  
%
\end{enumerate}

The remaining cases deal with the pretype $C$\@.
\begin{enumerate}
\item $C\equiv s$:  This case is trivial since by
Definition~\ref{defn:substitution}, $C[a/x]\equiv C$ and
the left-hand and right-hand sides both reduce to $\gamma(C)$\@.

\item $C\equiv \{y : T \vbar d\}$: The argument here follows along the
lines of the $b\equiv (\lambda (x : C): D)$ case above.
By the induction hypotheses, we know that
\begin{eqnarray*}
\lefteqn{\mathcal{M}(\Gamma, x : \ttvar~A\vbar \gamma\{x\gets
\Mgamma{a}\})(T)}\\
& = & \Mgamma{T[a/x]}\\
\lefteqn{\mathcal{M}(\Gamma, y : \ttvar~T[a/x], x : \ttvar~A\vbar \gamma\{y\gets
z\}\{x\gets \Mgamma{a}\})(d)}\\
& = & \mathcal{M}(\Gamma, y : \ttvar~T[a/x]\vbar \gamma\{y\gets z\})(d[a/x]),\\&&
\mbox{ for any } z\in \Mgamma{T[a/x]}
\end{eqnarray*}
The conclusion follows from Proposition~\ref{context-equivalence}
and Definition~\ref{defn:meaning function with dependent types}.

\item $C\equiv \funtype{y:C_1}{C_2}$: The argument here is similar
to that of the previous case.  Essentially, by the induction hypothesis
and Proposition~\ref{context-equivalence}, the function mapping
$z\in \mathcal{M}(\Gamma, x : \ttvar~A\vbar \gamma\{x\gets \Mgamma{a})(C_1)$
to $\mathcal{M}(\Gamma, y:\ttvar~C_1[a/x], x : \ttvar~A\vbar \gamma\{y\gets
z\}\{x\gets \Mgamma{a}\})(C_2)$ is the same as the function mapping
$z\in \Mgamma{C_1[a/x]}$ to $\mathcal{M}(\Gamma, y: \ttvar~C_1[a/x]\vbar
\gamma\{y\gets z\})(C_2[a/x])$\@.  

\item $C\equiv \tupletype{y:C_1, C_2}$:   Similar to the previous case.
\end{enumerate}
\end{proof}

Proposition~\ref{substitution-equivalence} is stated below without proof.
It asserts the semantic equivalence with respect to term $a$
of types $A$ and $B$ when $(A\sima B)_\Gamma$ holds.   Note that its
correctness depends on the soundness of the proof rules.  

\begin{prop}\label{substitution-equivalence}
If $\tau()(\Gamma) = \ttcontext$,  $a$ is a preterm such that
$\tauGamma{a} = B$, and $(A\sima  B)_\Gamma$,
then $\Mgamma{a}\in \Mgamma{A}$ iff $\Mgamma{a}\in \Mgamma{B}$\@.  
\begin{comment}
\begin{enumerate}
\item $\Mgamma{a}\in \Mgamma{A}$ iff $\Mgamma{a}\in \Mgamma{a/A}$
\item $\vdash \pi(a/A)(a)$ iff $\vdash \pi(A)(a)$\@.
\end{enumerate}
\end{comment}
\end{prop}
%
\begin{theorem}\label{dependent-term-semantics}
If $\tau()(\Gamma) = \ttcontext$, $\gamma$ is an assignment satisfying
$\Gamma$, and $a$ is a preterm such that $\tauGamma{a} = A$, then
$\Mgamma{a}\in \Mgamma{A}$\@.   
\end{theorem}
\begin{proof}
The proof is by  induction on the structure of the preterm
$a$\@.
\begin{enumerate}
\item $a\equiv s$: Then  by
Definition~\ref{defn:meaning function with dependent types},
$\Mgamma{a} = \gamma(a)$, and by Definition~\ref{defn:satisfaction},
we have that $\gamma(a)\in \Mgamma{A}$\@.

\item $a \equiv (\lambda (x : C): b)$:  By
Definition~\ref{defn:type rules with dependent types}, we
have $\tauGamma{a} = A = \funtype{x : C}{\tau(\Gamma, x: \ttvar~C)(b)}$\@.
Let $B$ label $\tau(\Gamma, x: \ttvar~C)(b)$\@.  
We know that $\Mgamma{A}$ is of the form $\Pi{} F$ where
$F$ maps $z\in \Mgamma{C}$ to $\mathcal{M}(\Gamma, x : \ttvar~C\vbar \gamma\{x
\gets z\})(B)$\@.

By the induction hypothesis on $b$, we know that for any $z\in \Mgamma{C}$, 
$\mathcal{M}(\Gamma, x : \ttvar~C\vbar \gamma\{x \gets z\})(b)\in
\mathcal{M}(\Gamma, x : \ttvar~C\vbar \gamma\{x
\gets z\})(B)$\@.  
Since by Definition~\ref{defn:meaning function with dependent types},
$\Mgamma{a}$ is a function mapping $z\in \Mgamma{C}$ to
$\mathcal{M}(\Gamma, x : \ttvar~C\vbar \gamma\{x \gets z\})(b)$, we have
$\Mgamma{a} \in \Pi F$ by the
definition of $\Pi$\@.  

\item $a \equiv (f~b)$: By
Definition~\ref{defn:type rules with dependent types}, we have that
$\tauGamma{f} = \funtype{x : B}{A'}$, $\tauGamma{b} = B'$,
$(B \sima  B')_\Gamma$,  $A \equiv A'[a/x]$, and
$\vdash_\Gamma\pi(B)(b)$\@.  We know by the induction hypothesis
that $\Mgamma{f}\in \Mgamma{\funtype{x : B}{A'}}$ and
$\Mgamma{b} \in \Mgamma{B'}$\@.  By
Propositions~\ref{substitution-equivalence}
and~\ref{maximal-subtype},
$\Mgamma{b}\in \Mgamma{\mu(B)}$\@.  
We therefore have 
by Proposition~\ref{maximal-subtype} that $\Mgamma{b}\in \Mgamma{B}$\@.
By Definition~\ref{defn:meaning function with dependent types},
$\Mgamma{a}\in \mathcal{M}(\Gamma, x: \ttvar~B\vbar \gamma\{x\gets
\Mgamma{b}\})(A')$,
and hence  by Lemma~\ref{substitution-lemma} it follows that
$\Mgamma{a}\in \Mgamma{A'[b/x]}$\@.  

\item $a \equiv (\listwo{a}{n})$:  The conclusion follows easily from
the induction hypothesis and
Definition~\ref{defn:type rules with dependent types}\@.

\item $a\equiv (\proj{i}~b)$: The conclusion follows easily from
Proposition~\ref{maximal-subtype}, the  induction hypothesis,
and Definition~\ref{defn:type rules with dependent types}\@.
The $(\proj{2}~b)$ case also employs Lemma~\ref{substitution-lemma}\@.  
\end{enumerate}
\end{proof}

\begin{comment}
\begin{lemma}\label{rank-boundedness}
If $\tau()(\Gamma) = \ttcontext$,
$\gamma$ is a satisfying assignment for $\Gamma$, 
$\tauGamma{A} = \tttype$,
$\Mgamma{A} \in U$, 
and $\tau(\Gamma, x: \ttvar~A)(B) = \tttype$, then
there is an $i$ such that the function $F$ mapping
$z$ in $\Mgamma{A}$ to 
$\mathcal{M}(\Gamma, x : \ttvar~A\vbar \gamma\{x\gets z\})(B)$
is in $W_i$\@.  
\end{lemma}
\begin{proof}
The proof is by induction on the structure of the pretype $B$\@.
Let the rank of $\Mgamma{A}$ be $j$\@.  
\begin{enumerate}
\item $B\equiv s$: Then $\mathcal{M}(\Gamma, x : \ttvar~A\vbar \gamma\{x\gets
z\})(B)$ is just $\gamma(s)$ and by Definition~ref{defn:satisfaction},
we have some $k$ such that $\gamma(s)\in U_k$\@.  Let $i$ be the maximum
of $j$ and $k$, we have $F\in U_i$\@.
%
\item $B\equiv \funtype{y:C}{D}$:  We have the induction hypothesis
%
\end{enumerate}
\end{proof}
%
\begin{theorem}\label{dependent-type-semantics}
If $\tau()(\Gamma) = \ttcontext$,  $\gamma$ satisfies $\Gamma$,
and $\tauGamma{A} = \tttype$, 
then 
\begin{enumerate}
%\item  $\Mgamma{A}\in U$, and
%%\item if $\tauGamma{b} = B$ then $\Mgamma{b} \in \Mgamma{B}$, and
\item if for all $n$, for $i$,  $1\leq i\leq n$,  $\mathcal{M}(\gamma,
x_1:\ttvar~A_1, \ldots x_{i-1}:\ttvar~A_{i-1}\vbar {A_i})\in U$ and 
$\tau(\Gamma, x_1 : \ttvar~A_1, \ldots, x_n:\ttvar~A_n)(B) = \tttype$,
then $\mathcal{M}(\Gamma, x_1 : \ttvar~A_1\ldots\vbar \gamma\{x\gets z\})(B)$
there exists an $i$ such that 
whenever $F$ maps $z$ in $\Mgamma{A}$ to $\mathcal{M}(\Gamma, x :
\ttvar~A\vbar \gamma\{x\gets z\})(B)$, there exists an $i$ such
that $F\in W_i$\@.  
\end{enumerate}
\end{theorem}
\begin{proof}
The proof is by simultaneous induction on the sizes of pretypes $A$ and
$B$\@.  We present the structure of the entire proof here but refer to
the proofs of Theorems~\ref{simple-type-semantics}
and~\ref{subtype-type-semantics} as needed.
\begin{enumerate}
\item $A\equiv s$: Same as in Theorem~\ref{simple-type-semantics}\@.
%
\item $A\equiv \funtype{y : C}{D}$: We have by the induction hypothesis
that $\Mgamma{C}\in U$ and for any $z\in \Mgamma{C}$, $\mathcal{M}(\Gamma,
y:\ttvar~C\vbar \gamma\{y\gets z\})(D) \in U$\@.  We also have by the
induction hypothesis with $D$ for $B$ that there is an $i$ such that
the function $F$ mapping $z\in \Mgamma{C}$ to $\mathcal{M}(\Gamma,
y:\ttvar~C\vbar \gamma\{y\gets z\})(D)$ is in $W_i$\@.
Since, by Definition~\ref{defn:meaning function with dependent types},
$\Mgamma{A} = \Pi{F}$, we have $\Mgamma{A}$ in $U_{i+1}$ and hence $U$\@.
%
\item $A\equiv \tupletype{x_1 : T_1,\ldots, x_n : T_n}$:  We have by the
induction hypothesis that $\Mgamma{T_1}\in U$ and that there is an $i$
such that 
the function $F$ mapping $z$ in $\Mgamma{T_1}$ to
$\mathcal{M}(\Gamma, x: \ttvar~T_1\vbar \gamma\{x\gets z\})(\tupletype{x_2:
T_2, \ldots x_n: T_n})$  is in $W_i$\@.  Since, by
Definition~\ref{defn:meaning function with dependent types},
$\Mgamma{A} = \Sigma{F}$,
we have by Definition~\ref{defn:type universe with dependent types},
that $\Mgamma{A}\in U_{i+1}$ and hence $\Mgamma{A}\in U$\@.
%
\item $A\equiv \{y : T | a\}$: Same as in
Theorem~\ref{subtype-type-semantics}\@.
%
\item $B \equiv s$:  Let $j$ be the rank of $\Mgamma{A}$\@.  
Then $\mathcal{M}(\Gamma, x : \ttvar~A\vbar \gamma\{x\gets
z\})(B)$ is just $\gamma(s)$ and by Definition~ref{defn:satisfaction},
we have some $k$ such that $\gamma(s)\in U_k$\@.  Let $i$ be the maximum
of $j$ and $k$, we have $F\in U_i$\@.
%
\item $B\equiv \funtype{x: C}{D}$:  
\end{enumerate}
\end{proof}
\end{comment}

\section{Summary}

Dependent typing is a significant enhancement to PVS since it adds
an important degree of flexibility and precision to the type system.
Notions  such as subtype constraints and type equivalence
that were introduced for subtyping can be extended for the case of
dependent types.  The semantic universe must be extended to include
additional sets to accommodate the semantics of dependent types.
The rank-boundedness of type dependencies is crucial in demonstrating that
dependent types can be interpreted in this extended semantic universe.
  






\chapter{Theories and Parametric Theories}\label{theories}

The next extension of the PVS language introduces theories and parametric
theories.  
The theory construct of PVS provides a way of packaging together a related
collection of declarations.  Theories can be parametric in
individual or type parameters.  Thus, PVS permits polymorphism or type
parametricity only at the theory level rather than at the declaration
level as in HOL~\cite{Gordon&Melham:HOL}.  
We first consider PVS theories without parameters.  The main change now is
that contexts are no longer simple and can contain theory declarations as
well.  A theory declaration has the form $m : \tttheory{} = \Delta$,
where $\Delta$ is a \emph{simple context} with no variable or theory
declarations. 
If $\Gamma(m)$ is the declaration $m : \tttheory{} = \Delta$, then
$ \itkind{}(\Gamma(m)) = \tttheory{}$, and
$ \itdef{}(\Gamma(m)) = \Delta$\@.
Correspondingly, constants and type names are no longer just symbols
but can be compound names of the form $m.s$ where $m$ is a symbol
naming a theory and $s$ is a symbol corresponding to the constant or type
name.

\section{Theories without Parameters}

To define the type rules for theories, we first modify the definition of
$\tau$ for simple contexts so that the context argument is not always
empty.  Here $\Delta ; \Gamma$ represents the concatenation of contexts.
\begin{Defn}{type rules for contexts}
\begin{eqnarray*}
  %
  \tau(\Theta)(\{\}) & = & \ttcontext \\
%
  \tau(\Theta)(\Gamma, s\ :\ \tttype{} = T) & = & \ttcontext, \mbox{
if } \Gamma(s) \mbox{ and } \Theta(s) \mbox{ are undefined,}\\& & 
  \tau(\Theta)(\Gamma) = \ttcontext, \mbox{and} \\& &
  \tau(\Theta;\Gamma)(T) = \tttype{}\\
%
  \tau(\Theta)(\Gamma, c : T) & = & \ttcontext, \mbox{ if }
\Gamma(c) \mbox{ and } \Theta(c)  \mbox{ are undefined,}\\ & &
\tau(\Theta)(\Gamma) = \ttcontext, \mbox{and }\\
   & &  \tau(\Theta;\Gamma)(T) =
      \tttype{}\\
%
%
  \tau(\Theta)(\Gamma, x : \ttvar{}\ T) & = & \ttcontext, \mbox{ if }
\Gamma(x) \mbox{ and } \Theta(x) \mbox{ are undefined,} \\ & &
\tau(\Theta)(\Gamma) = \ttcontext, \mbox{and }\\
    & &  \tau(\Theta;\Gamma)(T)
    = \tttype{}
%
\end{eqnarray*}
\end{Defn}
\begin{Eg}{type rules for contexts}
\begin{eqnarray*}
\tau(\Omega)(\ttreal{} : \tttype{}, \mathtt{0} : \ttreal{},
\leq : \funtype{\tupletype{\ttreal{}, \ttreal{}}}{\ttbool}) & = & \ttcontext
\end{eqnarray*}
\end{Eg}

The following rule handles theory declarations.
\begin{Defn}{type rule for contexts with theory declarations}
\begin{eqnarray*}
  \tau(\Theta)(\Gamma, m : \tttheory{} = \Delta) & = & \ttcontext
     \mbox{ if } \Theta(m), \Gamma(m) \mbox{ are undefined} \\& &
     \Delta \mbox{ only has constant and type declarations,}\\ & &
     \tau(\Theta ; \Gamma)(\Delta)  = \ttcontext,\\& &
     \tau(\Theta)(\Gamma) = \ttcontext
\end{eqnarray*}
\end{Defn}
\begin{Eg}{contexts with theory declarations}
\begin{eqnarray*}
\lefteqn{\tau(\Omega)(
\mathtt{reals} : \tttheory{} = (\ttreal{} : \tttype{}, \mathtt{0} : \ttreal{},
\leq : \funtype{\tupletype{\ttreal{},
\ttreal{}}}{\ttbool})) }\\
& = & 
\ttcontext
\end{eqnarray*}

\end{Eg}


\begin{comment}
Perhaps each declaration should be required to introduce a new
symbol so that there is no confusion between $\Gamma$ and $\gamma$.
\end{comment}

\begin{comment}
Example:
bool: TYPE,
integer : THEORY = (integer: TYPE, + :
\funtype{\tupletype{integer, integer}}{ {\tt integer}}),
natural : THEORY = (nat : TYPE)
\end{comment}

Any reference to a type name or a constant $s$ declared in a theory $m$
outside of this theory must be prefixed by the theory name, as in $m.s$\@.
Note that references to a type name or constant that is declared in the
same theory 
should not be given a theory prefix.  Before we can give the type rules, we
must update the definition of the type expansion
operation $\delta$ to prefix symbols with their theory names.
Let $\Gamma(m)(s)$  abbreviate $ \itdef{}(\Gamma(m))(s)$, which is
the declaration of the symbol $s$ in the definition of the theory $m$\@.
Let $\eta(\Gamma, m)(a)$ be the result of prefixing every unprefixed type or
constant symbol in $a$ by $m$, where $a$ is either an individual or type
expression.  We omit the definition of $\eta$ since it is straightforward.
  
%The $\delta$ operation is modified so that its first argument is now a
%pair consisting of a context and the `current' theory name.  The theory
%name component can be empty, as in Definition~\ref{defn:expanded type}\@.
We modify the definition of $\delta$ in Definition~\ref{defn:expanded
type} with the following clauses.  
\begin{Defn}{expanded type for prefixed symbols}
\begin{eqnarray*}
  \delta(\Gamma)(m.s) & = & \delta(\Gamma)(
\eta(\Gamma, m)(\itdef{}(\Gamma(m)(s)))), 
                        \mbox{ if }\\& &  \itdef{}(\Gamma(m)(s)) \mbox{ is
nonempty.}\\
  \delta(\Gamma)(m.s) & = & m.s
                        \mbox{ if } \itdef{}(\Gamma(m)(s)) \mbox{ is
empty.}
%\\
% \delta(\Gamma, m)(s) & = & m.s \mbox{ if } 
%\itdef{}(\Gamma(m)(s)) \mbox{ is
%empty.  }
\end{eqnarray*}
\end{Defn}
\begin{Eg}{expanded type for prefixed symbols}
Let $\Omega''$ be the context
$$\Omega, 
\mathtt{reals} : \tttheory{} = \begin{array}[t]{l}
				   (\ttreal{} : \tttype{},\\\
                                    \mathtt{0} : \ttreal{},\\\ 
                        \leq : \funtype{\tupletype{\ttreal{},
              \ttreal{}}}{\ttbool}, \\\ 
        \mathtt{nonneg\_real} : \tttype{} = \{x : \ttreal{} \vbar
\leq(0, x)\},\\\  \mathtt{1} : \mathtt{nonneg\_real} )
				   \end{array}$$

$$\delta(\Omega'')( \mathtt{reals.nonneg\_real}) =
\{ {x} : \mathtt{reals.real} \vbar
\mathtt{reals}.\!\leq( \mathtt{reals.0}, {x})\}$$
\end{Eg}

The type rules for prefixed symbols are given below.
\begin{Defn}{type rules for prefixed symbols}
\begin{eqnarray*}
  \tauGamma{m.s} & = & \tttype{}, \mbox{ if } \itkind{}(\Gamma(m)) =
\tttheory{} \mbox{ and }\\ & &
\itkind{}(\Gamma(m)(s)) = \tttype{}\\
%
  \tauGamma{m.s} & = & \delta(\Gamma)(\eta(\Gamma, m)(\ittype(\Gamma(m)(s)))), \\ &
&
\mbox{ if } \itkind{}(\Gamma(m)) =
\tttheory{} \mbox{ and }\\ & &
\itkind{}(\Gamma(m)(s)) = \ttconstant{}
\end{eqnarray*}
\end{Defn}

\begin{Eg}{type rules for prefixed symbols}
\begin{eqnarray*}
\tau(\Omega'')(\mathtt{reals.nonneg\_real}) & = & \tttype{}\\
\tau(\Omega'')(\mathtt{reals.1}) & = &
\{ {x} : \mathtt{reals.real} \vbar
\mathtt{reals}.\!\leq( \mathtt{reals.0}, {x})\}
\end{eqnarray*}
\end{Eg}

\begin{comment}
\begin{Eg}{contexts with theory declarations}
\begin{eqnarray*}
\tau(\Omega)(
\begin{array}[t]{l}
\mathtt{reals} : \tttheory{} = (\ttreal{} : \tttype{}, \mathtt{0} : \ttreal{},
\leq : \funtype{\tupletype{\ttreal{},
\ttreal{}}}{\ttbool}), \\
\mathtt{nonneg\_reals}: \tttheory{}  = \\\ \
   ( \mathtt{nonneg\_real} :
\tttype{} = \{ {x} : \mathtt{reals.real} \vbar
\mathtt{reals.\leq}( \mathtt{reals.0}, {x})\},\\\ \  \mathtt{sqrt}: \funtype{
\mathtt{nonneg_real}}{ \mathtt{nonneg_real}}\} )) 
\end{array}
& = & 
\ttcontext
\end{eqnarray*}
%
\end{Eg}
\end{comment}




The operations  $\pi$, and $\mu$ remain unchanged.
  An assignment $\gamma$ now
maps a theory name $m$ to an assignment $\gamma(m)$.
\begin{Defn}{meaning function for prefixed symbols}
\begin{eqnarray*}
  \Mgamma{m.s} & = & \gamma(m)(s)
\end{eqnarray*}
\end{Defn}
\begin{Eg}{meaning function for prefixed symbols}
Let $\omega''$ be a satisfying assignment for $\Omega''$ of the form
$$\ldots \{ \mathtt{reals}\gets \{\ttreal{}\gets
\reals\}\{\mathtt{0}\gets \zerob\}\ldots \} \ldots.$$

\begin{eqnarray*}
\mathcal{M}(\Omega''\vbar\omega'')(\mathtt{reals.real}) & = & \reals\\
\mathcal{M}(\Omega''\vbar\omega'')(\mathtt{reals.0}) & = & \zerob
\end{eqnarray*}
\end{Eg}



\begin{Defn}{satisfaction for contexts with theories}
An assignment~$\gamma$ satisfies a context $\Gamma$ if in addition to the 
constraints stated in
Definition~\ref{defn:satisfaction with type definitions}, $\gamma$ maps 
every theory $m$  declared in $\Gamma$ to 
a satisfying assignment for the body of the theory given by $
\itdef{}(\Gamma(m))$, that is for each declared symbol $s$ in $m$:
\begin{enumerate}
\item If $ \itkind{}(\Gamma(m)(s)) = \tttype{}$, then $\gamma(m)(s)
\in U$.


\item If $\itkind{}(\Gamma(m)(s)) = \ttconstant{}$, then $\gamma(m)(s)
\in \mathcal{M}(\Gamma\vbar\gamma)(\tau(\Gamma)(m.s))$.

\item  If  $\itdef{}(\Gamma(m)(s))$ is nonempty,
then $$\gamma(m)(s) = \mathcal{M}(\Gamma| \gamma)(
\eta(\Gamma, m)(\itdef{}(\Gamma(m)(s)))).$$
\end{enumerate}
\end{Defn}


\section{Constant Definitions}  We first extend the subset of PVS
described so far to include 
constant definitions in a manner similar to type definitions.
This extension is used in formalizing the semantics of parametric theories.
The syntax
for a constant definition is $c : T = a$ where $\itdef{}(\Gamma(c))$ is
$a$\@.  These definitions are explicit, that is, not recursive.  With this
extension, the type rule for constant declarations in contexts changes
from that of Definition~\ref{defn:type rules with subtypes}.
\begin{Defn}{type rule with constant definitions}
\begin{eqnarray*}
\tau(\Theta)(\Gamma, c : T = a) & = & T, \mbox{ if } \Gamma(c) \mbox{ is
undefined}, \\& & \Theta(c) \mbox{ is undefined}, \\& &
\tau(\Theta)(\Gamma) = \ttcontext,\\& &
%\vdash_\Gamma (\exists(x : T) : \tttrue),\\& &
\tau(\Theta;\Gamma)(a) = T',\\& & 
(T\sim T')_\Gamma,\\& &
\vdash_\Gamma \pi(T)(a)
\end{eqnarray*}
\end{Defn}

The notion of satisfaction must be extended from that of
Definition~\ref{defn:satisfaction for contexts with theories} to ensure
that an 
assignment 
for a defined constant satisfies the definition.
\begin{Defn}{satisfaction with constant definitions}
An assignment $\gamma$ satisfies a context $\Gamma$ if
in addition to the conditions in
Definition~\ref{defn:satisfaction for contexts with theories}, whenever $
\textit{kind}(\Gamma(s)) = \ttconstant$ and 
$\textit{definition}(\Gamma(s))$ is nonempty, then
$\gamma(s) = \Mgamma{ \textit{definition}(\Gamma(s))}$.
\end{Defn}


\section{Parametric Theories}

The extension to parametric theories is obtained by permitting
theories to be declared as $ m[\Pi] : \tttheory{} = \Delta$,
where $\Pi$ is a context listing the parameters and $\Delta$ is
the body of the theory.  If the above declaration of $m$ occurs
in context $\Gamma$, then $\Pi$ is $ \emph{formals}(\Gamma(m))$, and
$\Delta$ is $ \itdef{}(\Gamma(m))$.
For nonparametric theories, $ \mathit{formals}(\Gamma(m))$
is empty.   
Types or constants declared in a parametric theory are
referenced outside the theory as $m[\sigma].s$, where $\sigma$
is a list of \emph{actual} parameters consisting of types and terms.  
The type rule from the nonparametric case
must be modified to check the parameters.
\begin{Defn}{type rule for contexts with parametric theories}
\begin{eqnarray*}
  \lefteqn{\tau(\Theta)(\Gamma, m[\Pi] : \tttheory{} = \Delta)}\\ & = & \ttcontext
     \mbox{ if } \Gamma(m), \Theta(m), \Pi(m) \mbox{ are undefined}\\& &
      \tau(\Theta)(\Gamma) = \ttcontext\\& &
      \tau(\Theta ; \Gamma)(\Pi)  = \ttcontext,  \\ & & 
      \Pi \mbox{ has only constant and}\\& & \mbox{type declarations
without definitions,}\\ & &
      \tau(\Theta ; \Gamma ; \Pi)(\Delta)  = \ttcontext \\& &
      \mbox{ $\Delta$ only has type and constant declarations}
\end{eqnarray*}
\end{Defn}

  The 
 type  rules for prefixed symbols are given below.
The notation $\Pi = \sigma$, where $\Pi$ is of the form
$s_1 : \alpha_1,\ldots, s_n : \alpha_n$, and $\sigma$ is of
the form $\sigma_1,\ldots, \sigma_n$, is short for
the context $s_1 : \alpha_1 = \sigma_1, \ldots, s_n : \alpha_n =
\sigma_n$\@.  The definition of $\eta$ is now extended to substitute
actual theory parameters for formals, so that $\eta(\Gamma, m[\sigma])(a)$
prefixes every unprefixed symbol $s$ in $a$ that is declared in
$\itdef(\Gamma(m))$ by $m[\sigma]$, and replaces any $s_i$ in $a$
that is declared in $\itformals(\Gamma(m))$ by the corresponding
$\sigma_i$ in $\sigma$\@.  

\begin{Defn}{type rules for prefixed names with actuals}
Let $\Pi$ be $\mathit{formals}(\Gamma(m))$\@.  
\begin{eqnarray*}
  \tauGamma{m[\sigma].s} & = & \tttype{}, \mbox{ if }\\& & 
\itkind{}(\Gamma(m)) =  \tttheory{} \\ & &
\itkind{}(\Gamma(m))(s) = \tttype{} \mbox{ and }\\& &
\tauGamma{\Pi = \sigma} = \ttcontext\\
%
  \tauGamma{m[\sigma].s} & = & \delta(\Gamma)((\eta(\Gamma, m[\sigma])(
\ittype(\Gamma(m)(s)) )), \\ & &
\mbox{ if } \itkind{}(\Gamma(m)) = \tttheory{} \\ & &
\itkind{}(\Gamma(m)(s)) = \ttconstant{} \mbox{ and }\\& &
\tauGamma{\Pi = \sigma} = \ttcontext
\end{eqnarray*}
\end{Defn}

\begin{comment}
The definition of $\delta$ has to be modified slightly to carry along the
parameters.  The notation $a[\Pi]$ represents the result of
replacing each occurrence of a symbol $s$ in $a$ such that $\Pi(s)$ is
defined, by $\itdef{}(\Pi(s))$ when it exists.
\end{comment}
\begin{Defn}{type expansion with parametric theories}
\begin{eqnarray*}
  \delta(\Gamma)(m[\sigma].s) & = & \delta(\Gamma)(
(\eta(\Gamma, m[\sigma])(\itdef{}(\Gamma(m)(s)))) ), 
                        \mbox{ if }\\ & &  \itdef{}(\Gamma(m)(s)) \mbox{ is
nonempty.}\\
  \delta(\Gamma)(m[\sigma].s) & = & m[\sigma].s, 
                        \mbox{ if } \itdef{}(\Gamma(m)(s)) \mbox{ is
empty.}
%\\
% \delta(\Gamma, m[\sigma])(s) & = & \delta(\Gamma)( \itdef{}((\Pi
%= \sigma)(s))) \mbox{ if } \\& &
% \Pi(s) \mbox{ is defined}\\
% \delta(\Gamma, m[\sigma])(s) & = & m[\sigma].s \mbox{ if } \\& &
%\itdef{}(\Gamma(m)(s)) \mbox{ is
%empty.  }
\end{eqnarray*}
\end{Defn}

The definition of an assignment for a context with parametric theories is
a bit complicated.  
In the nonparametric case, $\gamma(m)$ simply returns an
assignment of values for the types and constants declared in the theory
$m$\@.  For the case of parametric theories $m$, $\gamma(m)$  returns a
function that maps the meaning of the given actuals $\sigma$ to
an assignment $\gamma(m)(\Mgamma{\sigma})$ for the  types and
constants declared in the theory $m$\@.   
There is an important restriction that $\gamma(m)$ must be
\emph{rank-preserving}, that is,  if $\varpi$  
and $\varpi'$ are assignments for 
$\Pi$ so that for each $i$ where $\Pi_i$ is a type parameter, the
rank of $\varpi(\Pi_i)$ equals the rank of 
$\varpi'(\Pi_i)$,  then the ranks of $\gamma(m)(\varpi)(s)$
and $\gamma(m)(\varpi')(s)$ must be the same for each type symbol $s$
declared in $m$\@.  
%\memo{This needs to be stated more directly.}

It is also important to observe that the semantics of parametric theories
makes use of the axiom of choice since the assignment corresponding to a
theory $\mathtt{m}$ of the form  $\mathtt{m [t:TYPE]: THEORY =
\{c : t\}}$ is essentially a choice function.  

Let $\gamma\{\Pi \gets \varpi\}$ represent the assignment such that
$\gamma\{\Pi \gets \varpi\}(s) = \varpi(s)$ for $s$
in the domain of the context $\Pi$, and $\gamma(s)$, otherwise\@.  
The meaning of symbols of the form $m[\sigma].s$ can then be defined
as below.
\begin{Defn}{meaning function for prefixed symbols with actuals}
\begin{eqnarray*}
  \lefteqn{\Mgamma{m[\sigma].s}}\\ & = &
\mathcal{M}(\Gamma;\Pi;\Delta \vbar
\gamma\{\Pi\gets\varpi\}\{\Delta\gets\gamma(m)(\varpi)\})(s), \mbox{ where}\\
 & & \Pi = \mathit{formals}(\Gamma(m)) \\
& &      \Delta = \mathit{definition}(\Gamma(m)) \\
& &      \varpi(r) = \Mgamma{(\Pi = \sigma)(r)}, \mbox{ for } r\in \Pi
\end{eqnarray*}
\end{Defn}

The definition of a satisfying assignment given in
Definition~\ref{defn:satisfaction for contexts with theories} 
also must be strengthened.    
Let $\Pi$ be the
formal parameters to theory $m$ in context $\Gamma$; then, an assignment
$\varpi$ is 
said to be satisfying {\em parameter assignment} for $\Pi$ under the
assignment $\gamma$ to $\Gamma$ 
iff $\gamma\{\Pi\gets \varpi\}$ is a satisfying assignment for $\Pi$\@.  
\begin{Defn}{satisfaction for contexts with parametric theories}
An assignment
$\gamma$ satisfies a context $\Gamma$ if in addition to the 
constraints stated in
Definition~\ref{defn:satisfaction for contexts with theories}, $\gamma$ maps 
every parametric theory $m$  declared in $\Gamma$ with parameters $\Pi$
and definition $\Delta$, 
to a function that maps
any satisfying parameter assignment $\varpi$ for the theory parameters $\Pi$
(namely, 
$\itformals(\Gamma(m))$) to a satisfying assignment $\gamma\{\Pi\gets
\varpi\}\{\Delta \gets \gamma(m)(\varpi)\}$ for 
$\Delta$ (given by 
$\itdef{}(\Gamma(m))$).
\begin{comment}
That is, 
for each declared symbol $s$ in
$\Delta$: 
\begin{enumerate}
\item If $ \itkind{}(\Gamma(m)(s)) = \tttype{}$, then
$\gamma(m)(\Mgamma{\sigma})(s)
\in U$
%
\item if $\itkind{}(\Gamma(m)(s)) = \ttconstant{}$, then $\gamma(m)(\Mgamma{\sigma})(s)
\in \mathcal{M}(\Gamma| \gamma;\{\Pi\gets \Mgamma{\sigma}\};\gamma(m)(\Mgamma{\sigma})(\tau(\Gamma)(m[\sigma].s)))$
%
\item if $\itdef{}(\Gamma(m)(s))$ is defined,
then $\gamma(m)(\Mgamma{\sigma})(s) = \mathcal{M}(\Gamma| \gamma;
\{\Pi\gets \Mgamma{\sigma}\}; \gamma(m)(\Mgamma{\sigma}))(
\itdef{}(\Gamma(m)(s)))$.
\end{enumerate}
\end{comment}
\end{Defn}


\begin{comment}
This map preserves the
assignments given to the parameters so that if for $s : T$ in $\Pi$,
$m[\sigma]$ is such that the formal $s$ is matched with the actual $a$,
then 
$\gamma(m)(\{\Pi\gets \Mgamma{\sigma}\})(s)$ must be $\Mgamma{a}$\@.
\end{comment}
\begin{comment}
The delicate point here is that $\gamma(m)$ is not a set but a map (a
rule) that maps a given assignment (a set) of sets to symbols, to another
assignment of sets to symbols.  This means that $\gamma$ itself is not a
set.  There is no reason why $\gamma$ should be a set since all we require
is that it map a given type name to a set, and a constant to an element
of the set representing its type.  
\end{comment}
\begin{comment}
\begin{eqnarray*}
  \Mgamma{m[\sigma].s} & = & \gamma(m)(\{\Pi\gets \Mgamma(\sigma)\})(s)
\end{eqnarray*}
\end{comment}

\section{Summary}

Theories are used to package related declarations together.  Parametric
theories can be used to package together declarations that are generic in
type and individual parameters.  The type rules for contexts must be
extended to accommodate the theories.  The type rules for simple
(nonparametric) theories are straightforward given this extension.  The
operation of expanding a type using type definitions must be enhanced so
that symbols declared in a theory are prefixed with their theory name when
referenced outside the theory.  Assignments now have the same nested
structure as contexts, and the semantic definition is easily extended to
handle prefixed symbols.  Parametric theories are more complex.  The
theory prefixes now contain actual parameters that have to be typechecked
relative to the expected formal parameters.  The assignments corresponding
to parametric theories are functions that map given assignments for the
formals to assignments for the declarations within a theory.  Such a
mapping must be constrained to be rank-preserving.  Parametric theories
can have subtype parameters, and assumptions on the parameters.  The rules
for subtype parameters and assumptions are omitted for now but will be
included in an expanded version of this report.



\chapter{Conditional Expressions and Logical Connectives}\label{conditionals}

We have, so far, introduced the core of PVS containing types, type
definitions, constant and variable declarations, subtypes, dependent
types, and theories.  In extending the language with
both explicit and recursive constant definitions and formulas,  a crucial
difference  is that the logical context under which a type-correctness
condition is generated provides additional assumptions that can be used in
proving any proof obligations.  Examples of expressions where an
extended context is needed to establish type correctness by discharging
proof obligations include
\begin{enumerate}
\item $x \neq y \supset  (x + y)/(x - y) \leq 0$.  The type of the
division operator constrains the denominator to be nonzero,
that is, $\{x : \ttreal \vbar x \neq 0\}$\@.    
In the given expression,  the denominator
can be shown to be nonzero only in the context of the antecedent $x \neq
y$\@.

\item $\mathtt{IF}(i > 0, i, -i)$ has type $\ttnat$ given integer $i$
provided the {\em then\/} and {\em else\/} parts are typechecked with the
assumptions $i > 0$ and $\neg (i > 0)$, respectively.
\end{enumerate}

PVS has a polymorphic primitive equality predicate:
\begin{display}
equality[T : TYPE] : THEORY = \{ =: [[T, T] -> bool] \}
\end{display}
Note that an equality of the form $ \mathtt{equality}[T].\mathtt{=}(a, b)$ is
informally written as $a = b$\@.  When it is relevant to indicate the type
parameter, we write the equality as $a =_{T} b$\@.  It can be deduced from
the meaning of equality that if $S$ is a subtype of $T$, then for $a$ and
$b$ in $S$, it must be the case that $a =_{S} b$ iff $a =_{T} b$\@.  Thus,
we can assume that equality is always parameterized by a maximal type.  We
assume that any relevant context $\Gamma$ contains the above declaration
of the theory {\tt equality}.  Furthermore, any satisfying assignment
$\gamma$ for such a $\Gamma$ must satisfy
\begin{eqnarray*}
\gamma( \mathtt{ equality})(X)(=) &  = & \{\pair{x, x}\vbar x\in X\}.
\end{eqnarray*}

The negation operation can be defined in terms of equality as shown below.
We assume that the context contains a declaration of the form
\begin{alltt}
   \(\neg : \funtype{\ttbool}{\ttbool} = (\lambda (x : \ttbool): x = \ttfalse)\)
\end{alltt}

As is clear, a satisfying assignment $\gamma$ for a context $\Gamma$
containing the above declaration must be such that $\gamma(\neg)$ yields
the usual truth-table semantics, that is, 
$\{ \pair{\zerob, \oneb}, \pair{\oneb, \zerob} \}$\@.  

We can then introduce the polymorphic \texttt{IF-THEN-ELSE} operation
as follows: 
\begin{display}
if_def [T: TYPE]: THEORY = \{ IF:[bool,T,T -> T] \}
\end{display}

In typechecking conditional expressions, the notion of context
has to be extended to include formulas so that the typechecking
of the subterm $b$ in $\mathtt{IF}(a, b, c)$ is done in the context of $a$,
and the typechecking of $c$ is done in the context of $\neg a$\@.
There is one new typechecking rule for contexts with formulas.
\begin{eqnarray*}
\tau()(\Gamma, a) & = & \ttcontext, \mbox{ if}\\
& & \tau()(\Gamma) = \ttcontext, \mbox{ and} \\
& & (\tau(\Gamma)(a)   \sim \ttbool)_\Gamma
\end{eqnarray*}
Note that the type rule checks that the type of $a$ is compatible with
$\ttbool$ rather than equivalent to it since it is possible that the  type
of $a$ might be a subtype of $\ttbool$\@.  


\begin{Defn}{satisfaction for contexts with formulas}
An assignment $\gamma$ satisfies context $\Gamma$ when in 
addition to the conditions in
Definition~\ref{defn:satisfaction for contexts with parametric theories},  
for each prefix $\Gamma', a$ of $\Gamma$, $\mathcal{M}(\Gamma'\vbar \gamma)(a)
= \oneb$\@.  
\end{Defn}

The typechecking of conditional expressions is different from that of
other application expressions since the \emph{test} part of the
conditional expression is introduced into the context as a contextual
assumption.
\begin{Defn}{type rule for conditional expressions}
   \begin{eqnarray*}
     \tauGamma{ \mathtt{ if\_def[\mathit{T}].IF}(a, b, c)} & = & T, \mbox{ if }
(\tauGamma{a} \sim  \mathtt{bool})_\Gamma,\\ & &
   \tau(\Gamma, a)(b) = B,\\& &  (B \sim T)_{\Gamma, a}, \\& & \vdash_{\Gamma,
a}\pi(T)(b)\\& & 
    \tau(\Gamma, \neg a)(c) = C,\\& &  (C \sim T)_{\Gamma, \neg a},\\ & & 
\vdash_{\Gamma, \neg a} \pi(T)(c)
\end{eqnarray*}
\end{Defn}

The meaning of conditional expressions must be treated in a special way
since the \emph{else} part need not denote when the \emph{test} part is
true and, correspondingly, the \emph{then} part need not denote if
the \emph{test} part is false.  We assume that any relevant contexts $\Gamma$
contain the above declaration of the \texttt{if\_def} theory.
Conditional expressions can be regarded as a
new construct in the language rather than a form of application.
However, it is conservative to regard conditional expressions as
applications since the latter introduce the additional constraint that all
the arguments must already denote, that is, applications are
\emph{strict}.
\begin{Defn}{meaning function for conditional expressions}
\begin{eqnarray*}
\Mgamma{\mathtt{if\_def}[T].\ttif(a, b, c)} & = &
 \left\{\begin{array}{ll}
	\Mgamma{b}, & \mbox{ if } \Mgamma{a} = \oneb\\
        \Mgamma{c}, & \mbox{ otherwise}
	\end{array}\right.
\end{eqnarray*}
\end{Defn}

The semantics for conditional expressions raises an important issue.
The equality
$$\mathtt{if\_def}[\mathtt{bool}].\ttif(x, y, \ttfalse) =
\mathtt{if\_def}[\mathtt{bool}].\ttif(y, x, \ttfalse)$$
is semantically valid for variables $x$ and $y$ of type $\ttbool$\@.
An expression like $\mathtt{if\_def}[\ttbool].\ttif(i \neq 0, 1/i
> 0, \ttfalse)$ can be typechecked to have the type $\ttbool$ since
it generates a valid proof obligation $i\neq 0 \supset i\neq 0$, but
the seemingly equivalent expression
$\mathtt{if\_def}[\ttbool].\ttif( 1/i > 0,i \neq 0, \ttfalse)$
generates an unverifiable proof obligation $i\neq 0$\@.
This may seem contradictory since the equality suggests a transformation
of a type correct conditional expression to a type incorrect expression.  
The resolution here is that equality cannot be instantiated with
$i\neq 0$ for $x$ and $1/i > 0$ for $y$ since the expression
$1/i > 0$  typechecks as having type $\ttbool$ only when $i\neq 0$ is known
from the context.  The same applies in the case of the other
propositional connectives, thus ensuring that each expression is type
correct in the context in which it occurs.  

We can then define the propositional connectives in terms of  conditional
expressions. 
\begin{eqnarray*}
  \wedge : \funtype{\tupletype{\ttbool, \ttbool}}{\ttbool} & = & \lambda (x : \ttbool, y
: \ttbool) : \mathtt{if\_def[\ttbool].IF}(x, y, \ttfalse)\\
   \vee : \funtype{\tupletype{\ttbool, \ttbool}}{\ttbool} & = & \lambda (x : \ttbool, y
: \ttbool) : \mathtt{if\_def[\ttbool].IF}(x, \tttrue, y)\\
   \supset : \funtype{\tupletype{\ttbool, \ttbool}}{\ttbool} & = & \lambda (x : \ttbool, y
: \ttbool) : \mathtt{if\_def[\ttbool].IF}(x, y, \tttrue)\\
%  \neg : \funtype{\ttbool}{\ttbool} & = & \lambda (x : \ttbool) :
%\mathtt{if\_def[\ttbool].IF}(x, \ttfalse, \tttrue)
\end{eqnarray*}

In the typechecking of terms of the form $a \wedge b$,
we follow the corresponding rule for the definition so that 
the term $a$ is assumed in the context when typechecking term $b$\@.
Similarly, for $a\vee  b$, the formula $\neg a$ is assumed in
the context when typechecking $b$, and for $ a \supset b$,
the formula $ a$ is assumed in the context when typechecking $b$.
The Boolean equivalence operator {\tt IFF} has no special rules
for adding formulas to contexts during typechecking.   

\section{Summary}

The use of assumption formulas  enables expressions
to be typechecked within the narrow context of their use so that
the governing assumptions can be used in discharging any
proof obligations.  The type rules for conditional expressions and the
Boolean connectives $\wedge$, $\vee$, and $\supset$ make use of
contextual assumptions.  

\begin{comment}
We will assume that the context $\Gamma$ used below contain the
theories \mathtt{equality} and \mathtt{if\_def}\@.  
The typechecking of constant definitions is straightforward when
the definitions are not recursive.  As we have already seen, the
form of an explicit definitions is $ s : T = a$, where $s$ is a symbol,
$T$ is a type, and $a$ is a term.  
When a constant is defined, no nonemptiness TCCs are generated since
there is a definition present.
%
\begin{eqnarray*}
  \tauGamma{\Delta, s : T = a} & = & \mathtt{CONTEXT}, \mbox{ if }
    \tauGamma{\Delta} = \mathtt{CONTEXT} \\ & &
   \tau(\Gamma, \Delta)(T)  = \mathtt{TYPE}\\ & &
   \tau(\Gamma, \Delta)(a) = T', \mbox{ and}\\ & &
   (T\sim_\Gamma T')_{\Gamma, \Delta}\\& &
   \vdash_{\Gamma, \Delta} \pi(T)(a)
\end{eqnarray*}
%
A recursive constant definition has the syntactic form
$ s : RECURSIVE\ T = a\ MEASURE\ f RELATION p$\@.
%
\memo{This looks slightly complicated. Postpone for now.}
%
A formula declaration has the form
$s : \ \mathtt{FORMULA} a$.
%
\memo{Need to formally defined substitution, freevars, and closure
relative to a context.}  
\end{comment}

\begin{comment}
%
\memo{Need to highlight the shape of a context: theory declarations
followed by type/constant declarations followed by variable declarations.}
%
Introduce equality, if-then-else, connectives, updates.
%
\section{Abstract Datatypes}
\end{comment}


\chapter{Proof Theory of PVS}\label{proof theory}

The final step in the presentation of the semantics is the presentation of
the proof rules for the idealized subset of PVS described thus far.
As already indicated, the proof theory is an integral part of the
semantics since typechecking and proof checking are closely intertwined.
Fortunately, the proof rules turn out to be much less complicated than the
type rules.

\newcommand{\sequent}[3]{#1\vdash_{#2}#3}

The PVS proof theory is presented in terms of  a sequent calculus.  A
sequent is of the form 
$\sequent{\Sigma}{\Gamma}{\Lambda}$, where $\Gamma$ is the context,
$\Sigma$ is a {\em set\/} of {\em antecedent\/} formulas, and $\Lambda$ 
is a {\em set\/} of {\em consequent\/} formulas.  Such a sequent should
be read as stating that the conjunction of the formulas in $\Sigma$ implies the
disjunction of formulas in $\Lambda$\@.

% presumed in math mode #1=top, #2=bottom, #3=side condition
\renewcommand{\Infrule}[3]{
{{\displaystyle\strut #1}\over{\displaystyle\strut #2}}~\makebox[0pt][l]{{$\mathbf{ #3}$}}
}

Inference rules are presented in the form
\begin{center}
$\Infrule{\mathit{premise(s)}}{\mathit{conclusion}}{\mathit{name}}\hspace{2cm}
\mathit{side\ condition}$
\end{center}


\section{PVS Proof Rules} 

\subsection{Structural Rules}

The structural rules permit the sequent to be rearranged or weakened
via the introduction of new sequent formulas into the conclusion.  
All the structural rules can be expressed in terms of the single
powerful weakening rule  shown below.
It allows a weaker statement to
be derived from a stronger one by adding either antecedent formulas or
consequent formulas.  The relation $\Sigma_1\subseteq\Sigma_2$ holds
between two lists when all the formulas in $\Sigma_1$ occur in the list
$\Sigma_2$.  

\begin{center}
$\Infrule{\sequent{\Sigma_1}{\Gamma}{\Lambda_1}}
         {\sequent{\Sigma_2}{\Gamma}{\Lambda_2}} {W}$
\hspace{1cm}\mbox{\smaller\smaller if $\Sigma_1\subseteq\Sigma_2$ and $\Lambda_1\subseteq\Lambda_2$}
\end{center}

Both the Contraction and Exchange rules shown below are absorbed by the
above  weakening rule {\bf W}\@.     The  {\em Contraction\/} rules
$\mathbf{C}\vdash$ and $\vdash\mathbf{C}$
allow multiple occurrences of the same sequent formula
to be replaced by a single occurrence. 
\begin{center}
\begin{tabular}{ccc}
$\Infrule{\sequent{a,a, \Sigma}{\Gamma}{\Lambda}}%
         {\sequent{a, \Sigma}{\Gamma}{\Lambda}} {C\vdash}$
& \hspace{1in} &
$\Infrule{\sequent{\Sigma}{\Gamma}{ a, a, \Lambda}}%
         {\sequent{\Sigma}{\Gamma}{ a, \Lambda}} {\vdash C}$
\end{tabular}
\end{center}

The {\em Exchange\/} rule asserts that the order of the formulas in the
antecedent and the consequent parts of the sequent is immaterial.  It can
be stated as

\begin{center}
\begin{tabular}{c@{\hspace{1in}}c}
$\Infrule{\sequent{\Sigma_1, b, a,  \Sigma_2}{\Gamma}{\Lambda}}
         {\sequent{\Sigma_1, a, b, \Sigma_2}{\Gamma}{\Lambda}} {X\vdash}$
&
$\Infrule{\sequent{\Sigma}{\Gamma}{\Lambda_1, b, a, \Lambda_2}}
         {\sequent{\Sigma}{\Gamma}{\Lambda_1, a, b, \Lambda_2}} {\vdash X}$
\end{tabular}
\end{center}

As seen above, 
{\em inference rules\/} have the general form
$$\Infrule{\Sigma_1\vdash\Lambda_1\quad\cdots\quad\Sigma_n\vdash\Lambda_n}
{\Sigma\vdash\Lambda}{R}.$$ This says that if we are given a leaf of a
proof tree of the form $\Sigma\vdash\Lambda$, then by applying the rule
named {\bf R}, we may obtain a tree with $n$ new leaves.

\subsection{Cut Rule}

The cut rule {\bf Cut} can be used to introduce a case split on a formula $a$
into a proof of a sequent $\sequent{\Sigma}{\Gamma}{\Lambda}$ so as to yield the
subgoals 
$\sequent{\Sigma, a}{\Gamma}{\Lambda}$ and
$\sequent{\Sigma}{\Gamma}{a, \Lambda}$, which can be seen  
as assuming $a$ along one branch and $\neg a$ along the other.
\begin{center}
$\Infrule{(\tau(\Gamma)(a) \sim \ttbool)_\Gamma \qquad \sequent{\Sigma,  a}{\Gamma}{\Lambda}\qquad\sequent{\Sigma}{\Gamma}{ a, \Lambda}}
         {\sequent{\Sigma}{\Gamma}{\Lambda}}{Cut}$
\end{center}

\subsection{Propositional Axioms}

The axioms rule \textbf{Ax} simply asserts that $a$ follows from $a$\@.  
%\begin{center}%{(\tauGamma{a}\sim \ttbool)_\Gamma}
$$\Infrule{}{\sequent{\Sigma, a}{\Gamma}{a,\Lambda}}{Ax}$$  

The next two rules assert that any sequent with either
an antecedent occurrence of $\ttfalse$ or a consequent
occurrence of $\tttrue$ is an axiom.

$$
\begin{array}{c@{\hspace{1in}}c}
\Infrule{}{\sequent{\Sigma, \mathtt{FALSE}}{\Gamma}{\Lambda}}{
\ttfalse \vdash} &
\Infrule{}{\sequent{\Sigma}{\Gamma}{ \mathtt{TRUE}, \Lambda}}{
 \vdash \tttrue}
\end{array}
$$



\subsection{Context Rules}

Certain formulas hold in a context simply because they are already
asserted in the context either as a formula or a constant definition.
$$
\begin{array}{c}
\Infrule{}{\sequent{}{\Gamma}{a}}{Context Formula} \hspace{1.5in}\mbox{if
$a$ is a formula in $\Gamma$}\hspace{.8in}\\
\Infrule{}{\sequent{}{\Gamma}{s = a}}{Context Definition} \hspace{1.5in}\mbox{if
$s : T = a$ is a constant definition in $\Gamma$}
\end{array}
$$

The context $\Gamma$ can be extended with antecedent formulas or
negations of consequent formulas using the following two rules.
$$\begin{array}{c@{\hspace{1in}}c}
\Infrule{\sequent{\Sigma, a}{\Gamma,
a}{\Lambda}}{\sequent{\Sigma, a}{\Gamma}{\Lambda}}{Context \vdash} &
\Infrule{\sequent{\Sigma}{\Gamma,
\neg a}{a, \Lambda}}{\sequent{\Sigma}{\Gamma}{a, \Lambda}}{\vdash Context}
\end{array}
$$

The following context-weakening rule is useful since it shows that
provability is monotonic with respect to the context.
$$\begin{array}{c}
\Infrule{\sequent{\Sigma}{\Gamma}{\Lambda}}{\sequent{\Sigma}{\Gamma'}{\Lambda}}{Context
W}  \hspace{1in}\mbox{if $\Gamma$ is a prefix of $\Gamma'$}
\end{array} 
$$

\begin{comment}
\subsection{Negation Rules}
%
These rules are straightforward since negation is eliminated by moving
the body of a negated antecedent formula to the consequent, and
conversely, by moving the body of a negated consequent formula to
the antecedent.
$$
\begin{array}{c}
\Infrule{\sequent{\Sigma}{\Gamma}{ a, \Lambda}}
         {\sequent{\Sigma,\neg a }{\Gamma}{\Lambda}} {\neg\vdash}\\[.5cm]
\Infrule{\sequent{\Sigma, a }{\Gamma}{\Lambda}}
         {\sequent{\Sigma}{\Gamma}{ \neg a, \Lambda}} {\vdash\neg}
\end{array}
$$
\end{comment}

\subsection{Conditional Rules}

The rules governing the elimination of \texttt{IF-THEN-ELSE} in a proof
are unusual since they augment the context with the test part or its
negation, as in the corresponding type rules.
$$
\begin{array}{c}
\Infrule{\sequent{\Sigma, a, b}{\Gamma, a}{\Lambda}\hspace{1cm}
        \sequent{\Sigma,  c}{\Gamma, \neg a}{a, \Delta}}
         {\sequent{\Sigma, \mathtt{IF}(a, b, c)}{\Gamma}{ \Lambda}}
                  {\mathtt{IF} \vdash}\\[.5cm]
\Infrule{\sequent{\Sigma, a}{\Gamma,
a}{b,\Lambda}\hspace{1cm}\sequent{\Sigma}{\Gamma,\neg a}{a, c, \Lambda}}
         {\sequent{\Sigma}{\Gamma}{{\mathtt{IF}}(a, b, c), \Lambda}}
                      {\vdash {\mathtt {IF}}}
\end{array}
$$

\subsection{Equality Rules}

The rules for equality can be stated as below.  The rules of
transitivity and symmetry for equality can be derived from these rules.
The notation $a[e]$ is used to highlight one or more 
occurrences of $e$ in the formula $a$ such that there are no free variable
occurrences in $e$\@.\footnote{We enforce an invariant on a sequent that
it must not contain any free variables.  This invariant is preserved by
each of the proof rules.}   
The notation $\Lambda[e]$ similarly highlights
occurrences of $e$ in $\Lambda$. 

\begin{center}
\begin{tabular}{c@{\hspace{1in}}c}
$\Infrule{}{\sequent{\Sigma}{\Gamma}{ a=a,\Lambda}}{Refl}$
&
$\Infrule{\sequent{a=b, \Sigma[b]}{\Gamma}{\Lambda[b]}}
         {\sequent{a=b, \Sigma[a]}{\Gamma}{\Lambda[a]}} {Repl}$
\end{tabular}
\end{center}

\subsection{Boolean Equality Rules}

The rule $\mathbf{Repl~\tttrue}$ asserts that an antecedent formula
$a$
can be treated as an antecedent equality of the form $a = \tttrue$, and
correspondingly, 
a consequent formula $a$ can be treated as an antecedent
equality of the form $a = \ttfalse$\@.

$$
\begin{array}{c@{\hspace{1in}}c}
\Infrule{\sequent{\Sigma[\tttrue], a}{\Gamma}{\Lambda[\tttrue]}}{
\sequent{\Sigma[a], a}{\Gamma}{\Lambda[a]}}{Repl~\tttrue} &
\Infrule{\sequent{\Sigma[\ttfalse], a}{\Gamma}{\Lambda[\ttfalse]}}{
\sequent{\Sigma[a]}{\Gamma}{a, \Lambda[a]}}{Repl~\ttfalse}
\end{array}
$$

The rule $\tttrue\mbox{-}\ttfalse$ asserts that $\tttrue$ and $\ttfalse$ are
distinct Boolean constants.
$$
\begin{array}{c}
\Infrule{}{\sequent{\Sigma, \tttrue = \ttfalse}{\Gamma}{\Lambda}}{\tttrue\mbox{-}\ttfalse}
\end{array}
$$
\begin{comment}
Four rules exist for simplifying equalities of the form $a =
\tttrue$ and $a = \ttfalse$ in  a sequent.
%
$$
\begin{array}{c@{\hspace{1in}}c}
\Infrule{\sequent{\Sigma, a}{\Gamma}{\Lambda}}{\sequent{\Sigma, a =
\tttrue}{\Gamma}{\Lambda}}{\tttrue\vdash} &
\Infrule{\sequent{\Sigma}{\Gamma}{a, \Lambda}}{\sequent{\Sigma}{\Gamma}{a =
\tttrue, \Lambda}}{\vdash\tttrue} \\
%
\Infrule{\sequent{\Sigma}{\Gamma}{a, \Lambda}}{\sequent{\Sigma, a =
\ttfalse}{\Gamma}{\Lambda}}{\ttfalse\vdash} &
\Infrule{\sequent{a, \Sigma}{\Gamma}{\Lambda}}{\sequent{\Sigma}{\Gamma}{a =
\ttfalse, \Lambda}}{\vdash\ttfalse} 
\end{array}
$$
\end{comment}

\subsection{Reduction Rules}

The reduction rules are equality rules (axioms) that provide the obvious
simplifications for applications involving lambda abstractions and
product projections.

$$
\begin{array}{c}
\Infrule{}{\sequent{}{\Gamma}{(\lambda (x : T):~a)(b) =
a[b/x]}}{\beta}\\[.5cm]
%
\Infrule{}{\sequent{}{\Gamma}{\proj{i}(a_1, a_2) =
a_i}}{\pi}
\end{array}
$$

\subsection{Extensionality Rules}

The extensionality rules are also equality rules for establishing equality
between two expressions of function or product type.  The extensionality
rule for functions, \textbf{FunExt}, introduces a Skolem constant $s$
to determine that two functions $f$ and $g$ are equal when the results of
applying them to an arbitrary argument $s$ are equal.  
$$
\begin{array}{c}
\Infrule{\sequent{\Sigma}{\Gamma, s : A}{(f~s) =_{B[s/x]} (g~s), \Lambda}}
{\sequent{\Sigma}{\Gamma}{f =_{\funtype{x:A}{B}} g,
\Lambda}}{FunExt} \hspace{1in} \Gamma(s) \mbox{ undefined}\\[.5cm]
\end{array}
$$

The extensionality rule for products asserts that two products are equal if
their corresponding projections are equal.
\begin{comment}
Note that the type
$\tupletype{\listwo{T'}{n}}$ is just $a/\tupletype{\listwo{T}{n}}$\@.
\end{comment}
$$
\begin{array}{c}
\Infrule{\sequent{\Sigma}{\Gamma}{\proj{1}(a) =_{T_1} \proj{1}(b),\Lambda}\qquad\sequent{\Sigma}{\Gamma}{\proj{2}(a) =_{T_2[(\proj{1}~a)/x]} \proj{2}(b),\Lambda}
}{\sequent{\Sigma}{\Gamma}{a =_{\tupletype{x : T_1 T_2}} b, \Lambda}}{TupExt}
\end{array}
$$

Recall that the quantifiers can be defined in terms of lambda abstraction
and equality so that $(\forall (x: T): a)$ is just $(\lambda (x: T): a) =
(\lambda (x : T): \tttrue)$\@.   Existential quantification $(\exists (x:
T): a)$  can easily be defined as $\neg (\forall (x: T): \neg a)$\@.
The proof rules for quantifiers can then be derived from the
rules $\beta$, $\mathbf{TupExt}$, and the equality rules\@.

\subsection{Type Constraint Rule }

We need a rule to introduce the type constraint on a term as an antecedent
formula of the given goal sequent.

$$
\Infrule{\tau(\Gamma)(a) = A\qquad\sequent{\pi(A)(a),\Sigma}{\Gamma}{\Lambda}}
{\sequent{\Sigma}{\Gamma}{\Lambda}}{Typepred}
$$

\section{Soundness of the Proof Rules}
\begin{prop}\label{meaning-preservation}
If $\Gamma$ is a prefix of $\Gamma'$,
$\tau()(\Gamma) = \tau()(\Gamma') = \ttcontext$,
$\gamma'$ is a
satisfying assignment for $\Gamma'$, and $\gamma = \gamma'\restriction\Gamma$ then
for any $a$ such that 
$\tauGamma{a} = \tau(\Gamma')(a)$,  it is the case that $\Mgamma{a} =
\mathcal{M}(\Gamma'\vbar\gamma')(a)$\@.  
\end{prop}

\begin{theorem}[soundness]\label{proof-soundness}
If $\tau()(\Gamma) = \ttcontext$ such that
%\begin{enumerate}
%\item $\Gamma$ contains no variable declarations %bogus(1.18.96)
%\item
for every formula $a$ in $\Sigma;  \Lambda$, $(\tauGamma{a} \sim
\ttbool)_\Gamma$,
%\end{enumerate}
and $\sequent{\Sigma}{\Gamma}{\Lambda}$ is provable, then
for any satisfying assignment
$\gamma$ for $\Gamma$, 
either there is a formula $b$ in $\Sigma$, such that $\Mgamma{b} = \zerob$
or a formula $c$ in
$\Lambda$, such that $\Mgamma{c} = \oneb$\@.      
\end{theorem}
\begin{proof}
The proof is by induction on the structure of the proof of
$\sequent{\Sigma}{\Gamma}{\Lambda}$.  Recall that this proof is actually
part of a simultaneous induction that includes the soundness of the type
rules relative to the semantic function, that is,
Theorems~\ref{dependent-type-semantics}
and~\ref{dependent-term-semantics}\@.   Specific invocations of the
soundness theorem occur in the proofs of
Theorem~\ref{subtype-term-semantics}
and Proposition~\ref{substitution-equivalence}\@.  
\begin{enumerate}
\item {\em Structural Rules\/}:   Since the subset
of formulas in the premise and the conclusion of these rules are the same,
the conclusion follows easily from the induction hypothesis.

\item {\em Cut\/}: By the semantic soundness of the type rules,
we have $\Mgamma{a} \in \twob$\@.   If $\Mgamma{a} = \zerob$, then
by the induction hypothesis on the second subgoal of the proof rule,
there must be some $b$ in $\Sigma$ such that $\Mgamma{b} = \zerob$
or a $c$ in $\Lambda$ such that $\Mgamma{c} = \oneb$\@.  The case
when $\Mgamma{a} = \oneb$ is symmetrical.

\item {\em Propositional Axioms\/}:  Obvious.

\item \emph{Context Rules}:
\begin{description}
\item[ContextFormula:]  If $\gamma$ satisfies $\Gamma$ and $a\in \Gamma$,
then $\Mgamma{a} = \oneb$\@.

\item[ContextDefinition:] If $\gamma$ satisfies $\Gamma$ and $s: T = a$
is a declaration in $\Gamma$, then by the definition of satisfaction,
$\Mgamma{s} = \Mgamma{a}$\@.

\item [Context $\vdash$: ] The argument is trivial when $\Mgamma{a} =
\zerob$.  Otherwise, $\gamma$ satisfies the extended context $\Gamma, a$,
and the conclusion follows from the induction hypothesis.

\item [$\vdash$ Context: ] Similar to $\textbf{Context} \vdash$ above.

\item [ContextW: ]  If $\gamma$ satisfies $\Gamma'$, then it also
satisfies $\Gamma$, and hence the proof.
\end{description}

\item \emph{Conditional Rules}: We only consider $ \mathtt{IF} \vdash$
since the $\vdash \mathtt{IF}$ proof is similar.  If $\Mgamma{
\mathtt{IF}(a, b, c)} = \zerob$, the conclusion follows  trivially.
Otherwise, If $\gamma$ satisfies
$\Gamma$, then $\Mgamma{a}\in \twob$.  If $\Mgamma{a} = \oneb$, then
$\Mgamma{b} = \oneb$.  The induction hypothesis on the subgoal
$\sequent{\Sigma, a, b}{\Gamma, a}{\Lambda}$ yields the desired
conclusion.  Similarly, if $\Mgamma{a} = \zerob$, we have $\Mgamma{c} =
\oneb$
and the induction hypothesis on the second subgoal yields the
desired conclusion.

\item \emph{Equality Rules}: The \textbf{Refl} rule is obvious. 
For the \textbf{Repl} rule, if $\Mgamma{a = b} = \zerob$, the conclusion
follows trivially.  Otherwise, $\Mgamma{a} = \Mgamma{b}$.  Hence, $\gamma$
satisfies the extended context $\Gamma, a = b$.  Then for each
$c[a]$ in $\Sigma[a]$ or $\Lambda[a]$, $\Mgamma{c[a]} = \Mgamma{c[b]}$\@.
\begin{comment}
Note: no free variables in a, b.  Need to show that
replacement preserves meaning when there is no capture.
\end{comment}

\item \emph{Boolean Equality Rules}: The $\mathbf{Repl}~\tttrue$ and
$\mathbf{Repl}~\ttfalse$ rules follow easily since when $\Mgamma{a} =
\oneb$, we have $\Mgamma{c[a]} = \Mgamma{c[\tttrue]}$\@.
A similar argument applies to $\mathbf{Repl}~\ttfalse$\@.

The soundness of $\tttrue\mbox{-}\ttfalse$ is easy since
$\Mgamma{\tttrue = \ttfalse} = \zerob$\@.

\item \emph{Reduction Rules}:  The $\beta$-reduction rule follows because
$\Mgamma{(\lambda (x : T):~a)(b)}$ is
$\mathcal{M}(\Gamma, x : \ttvar~T\vbar \gamma\{x\gets\Mgamma{b}\})(a)$
which by the Substitution Lemma~\ref{substitution-lemma} is
equal to $\Mgamma{a[b/x]}$\@.

The soundness $\pi$-reduction rule is a direct consequent of
Definition~\ref{defn:meaning function}\@.

\item \emph{Extensionality Rules}:
\begin{description}
\item [FunExt:] 
First consider the case when the domain type $\Mgamma{A}$ is empty.  Then by
Definition~\ref{defn:meaning function with dependent types},
$\Mgamma{f} = \Mgamma{g} = \emptyset$\@.  Therefore $\Mgamma{f = g} =
\oneb$ and hence the conclusion.\footnote{Since
the  subgoal sequent $\sequent{\Sigma}{\Gamma, s:A}{(f~s) =
(g~s),\Lambda}$
is valid when $\Mgamma{A} = \emptyset$ for all assignments $\gamma$,
it is natural to ask how it is actually proved.  The only
way a type $A$ can be empty under any assignment $\gamma$ is if
$\Mgamma{\pi(A)(a) = \zerob}$.  The $\mathbf{Typepred}$ rule can
therefore be used on the Skolem constant $s$ to complete the proof.}

The case when $\Mgamma{A}$ is nonempty, we have for any $\gamma$
satisfying $\Gamma$ and $s\in \Mgamma{A}$, that $\gamma'$ given by 
$\gamma\{s\gets z\}$ is a satisfying assignment for $\Gamma, s: A$\@.
By the induction hypothesis, there is either an $a$ in $\Sigma$
such that $\mathcal{M}(\Gamma, s: A\vbar\gamma')(b) = \zerob$ or
a $c$ in  ${(f~s) = (g~s),\Lambda}$ such that $\mathcal{M}(\Gamma, s:
A\vbar\gamma')(c) = \oneb$\@.  If we have such a $b$ in $\Sigma$,
by Proposition~\ref{meaning-preservation}, we also have that
$\Mgamma{b} = \zerob$.  A similar argument can be used
if we have such a $c$ in $\Lambda$\@.  If $c$ is $(f~s) = (g~s)$, then
$\Mgamma{f}(z) = \Mgamma{g}(z)$ for every $z$ in $\Mgamma{A}$\@.
By set-theoretic extensionality, this means that $\Mgamma{f}$ and
$\Mgamma{g}$ are identical elements 
of $\Pi{}F$ where $F$ maps $z$ in $\Mgamma{A}$ to an element of
$\mathcal{M}(\Gamma, x : \ttvar~A\vbar \gamma\{x \gets z\})(B)$\@.
Therefore $\Mgamma{f = g} = \oneb$ as desired.

\item [TupExt:]  If there is some $d$ in $\Sigma$ such that by applying the
induction hypothesis to any of the subgoals $\Mgamma{d} = \zerob$,
then the same holds for the conclusion sequent.  Similarly,
if the induction hypothesis on some subgoal yields a $c$ in $\Lambda$
such that $\Mgamma{c} = \oneb$, then the same holds for the
conclusion sequent.  So the remaining case is when, by the
induction hypothesis, 
$\Mgamma{\proj{i}(a)} = \Mgamma{\proj{i}(b)}$ for each $i\in \{1,2\}$.
It is therefore easy to conclude by set-theoretic extensionality that
$\Mgamma{a}$ and $\Mgamma{b}$ are identical elements of
$\Mgamma{a/\tupletype{\listwo{T}{n}}}$\@.  We can then use
Proposition~\ref{substitution-equivalence} to conclude
that $\Mgamma{a}$ and $\Mgamma{b}$ are identical
elements of $\Mgamma{\tupletype{\listwo{T}{n}}}$\@. 

\end{description}

\item \emph{Type Constraint Rule}: Recall from
Proposition~\ref{maximal-subtype}
that when $\tauGamma{a} = A$, then
$\Mgamma{\pi(A)(a)} = \oneb$.  Given this and the induction hypothesis,
it must either be the case that we have a $b$ in $\Sigma$ such that
$\Mgamma{b} = \zerob$ or a $c$ in $\Lambda$ such that $\Mgamma{c} =
\oneb$\@. 
\end{enumerate}
\end{proof}

To tie the development so far into a single simultaneous
induction as promised, we state the key theorem whose subproofs have been
given by the theorems presented thus far, namely,
Theorems~\ref{dependent-type-semantics},~\ref{dependent-term-semantics},
and~\ref{proof-soundness}\@.  
\begin{theorem}\label{big-soundness}
If $\tau()(\Gamma) = \ttcontext$, then 
\begin{enumerate}
\item If $\Sigma, \Lambda$ is a list of preterms such that for every $a$
in $\Sigma; \Lambda$, $(\tauGamma{a} \sim 
\ttbool)_\Gamma$, and $\sequent{\Sigma}{\Gamma}{\Lambda}$ is provable, then
for any satisfying assignment
$\gamma$ for $\Gamma$, 
either there is a $b$ in $\Sigma$, such that $\Mgamma{b} = \zerob$ or a $c$ in
$\Lambda$, such that $\Mgamma{c} = \oneb$\@.

\item If $A$ is a pretype such that $\tauGamma{A} = \tttype$, then for any
assignment $\gamma$ 
satisfying $\Gamma$, $\Mgamma{A}\in U$\@.

\item If $a$ is a preterm such that $\tauGamma{a} = A$, then for any
assignment $\gamma$ satisfying $\Gamma$, $\Mgamma{a}\in \Mgamma{A}$\@.
\end{enumerate}
\end{theorem}


\section{Summary}

The logical inference rules for the PVS logic have been presented in a
sequent calculus format.  The formal semantics presented in the earlier
chapters is used to establish the soundness of these proof rules.


\chapter{Conclusion}\label{conclusion}

We have presented the syntax and semantics of idealized PVS in several
stages.  In the first stage we introduced the simply typed fragment,
which was then extended with type definitions.  The third such fragment
included subtyping;  the fourth fragment introduced dependent typing.
Finally, we introduced constant definitions and parametric and nonparametric
theories.

The semantic definition was given in a novel, functional style where a
canonical type was assigned to each type correct term.  The interplay between
types and proofs in PVS introduced subtleties and complexities
into the semantic definition.  We can now answer some of the questions
raised in Chapter~\ref{introduction}:
\begin{itemize}
\item {\em What is the semantic core of the language, and what is
just syntactic sugar?}

The semantic core of the language is a typed lambda calculus with
simple function and tuple types, predicate subtypes, dependent types,
parametric theories, and conditional expressions.  Many of the other
features of the PVS language such as records and update expressions
can be explained in terms of the core language.

\item {\em What are the rules for determining whether a given PVS
expression is well typed?  }

The typechecking rules have been presented in terms of the definition of
the $\tau$ operator in Chapters~\ref{simple},  \ref{subtypes},
\ref{dependent}, \ref{theories}, and~\ref{conditionals}\@. 

\item {\em How is subtyping handled, and in particular, how are
proof obligations corresponding to subtypes generated?}

Typechecking an expression $a$ with respect to predicate subtype
constraint $\subtype{x}{T}{p(x)}$
is done by generating the proof obligation $p(a)$ under the logical
context in which $a$ is being typechecked.  This is made precise
in Definitions~\ref{defn:type rules with subtypes} and~\ref{defn:type rule
for conditional expressions}\@.  
Proof obligations are generated when typechecking contexts (for
nonemptiness),  typechecking expressions with respect to expected subtypes, 
and  comparing two types containing subtype expressions for
compatibility.

\item  {\em What is the meaning, in set-theoretic terms, of a PVS expression or
assertion?}

The set-theoretic meaning of well-formed PVS types and expressions
is given by  a meaning function
$\mathcal{M}$ that assigns a set $\Mgamma{T}$ from the universe $U$ to each type
$T$,  and an element $\Mgamma{a}$ of $\Mgamma{T}$ to a given term $a$ of type
$T$\@.

\item {\em Are the type rules sound with respect to the semantics? }

The typechecking function $\tau$ is defined to check contexts,
preterms, and pretypes for type correctness.    The type rules are shown
to be sound with respect to the given semantics in
Theorem~\ref{big-soundness}\@. 

\item {\em Are the proof rules sound with respect to the semantics?}

The proof rules are given in Chapter~\ref{proof theory} in a sequent
calculus format and proved to be sound with respect to the semantics in
Theorem~\ref{big-soundness}\@.  

\item {\em What is the form of dependent typing used by PVS, and
what kinds of type dependencies are disallowed by the language?}

The semantic analysis of dependent typing in Chapter~\ref{dependent}
revealed that type dependencies were constrained to be rank-bounded.  This
is true because the dependencies in dependent typing only constrain the
predicate part of predicate subtypes. Thus, when there is a dependent type
$T(n)$ that depends on a parameter $n$, the meaning of $T(n)$ has a fixed
rank regardless of the meaning assigned to $n$\@.  The PVS language
features used to define dependent types all preserve the rank-boundedness.
Language extensions violating rank-boundedness such as a type dependency
of the form $\funtype{n : nat}{T^n}$ are disallowed.  One can extend the
language with such dependent types, but the semantics would then be
considerably more complicated.

\item {\em What is the meaning of theory-level parametricity, and
what, if any, are the semantic limits on such parameterization?}

The semantics of parametric theories is described in
Chapter~\ref{theories}\@.  In particular, the semantics for parametric
theories is given in terms of rank-preserving maps between the meanings of
the parameters and the meanings of the identifiers declared in the theory.
These maps must be such that the rank of an assignment to a type in a theory
depends only on the ranks of the (meanings of the) type parameters.

\item {\em What language extensions are incompatible with the reference
semantics given here?}

We have already indicated that any language extension, such as an $n$-tuple
type $T^n$, that violates rank-boundedness would be incompatible with the
semantics presented here.
\end{itemize}

This report presents only the core language of PVS.  A more complete
semantic treatment would include arithmetic, recursive constant
definitions, inductive definitions, recursive datatypes, assumptions on
theory parameters, and type judgements.

\paragraph{Acknowledgments. } The advice and encouragement of John Rushby,
Rick Butler, Paul Miner, Pat Lincoln, and Mandayam Srivas are greatly
appreciated, as are the useful expert comments of Peter Dybjer, Mike
Gordon, Doug Howe, and Paul Jackson.  Bruno Dutertre, Paul Miner, and
Harald Rue{\ss} suggested numerous improvements to earlier drafts.

\begin{comment}
What is a specification language?  A programming language is meant to
express an algorithm in such a way that it can be executed on a computing
machine.  For example, a program implementing a sorting algorithm such as
quicksort can be written in a programming language like C.  Such a program
is a recipe for computing the desired outcome but does not directly
describe this outcome.  In the case of a sorting program, the desired
result is that 
the output array must be an ordered permutation of the input array.
A specification language is the medium for expressing and establishing
such properties of programs.  A specification language could itself be a
programming language but non-executable features such as quantifiers and
temporal operators  add to the expressiveness of a specification language.
%
Specification languages need to capture mathematics
as well as programming.  The use of an expressive type system lends
considerable mathematical clarity to a specification.  
ProVeS is an environment for
specification  and verification based on a typed higher-order logic.
The type system of ProVeS includes
{\em predicate subtypes}, i.e.,
the type consisting of elements of a given type satisfying a specified
predicate, in addition to the usual constructs for
function, product, record, and array types.  Automated theorem proving
procedures are employed in the process of typechecking.
The design and rules
for the type system and logic of ProVeS are presented along
with arguments for their soundness.
\end{abstract}
%
% Document Type: LaTeX
% Master File: intro.tex

\chapter{Introduction}

PVS is a \emph{P}rototype \emph{V}erification \emph{S}ystem for the
development and analysis of formal specifications.  The PVS system
consists of a specification language, a parser, a typechecker, a prover,
specification libraries, and various browsing tools.  This document
primarily describes the specification language and is meant to be used as
a reference manual.  The \emph{PVS System Guide}~\cite{PVS:userguide} is to
be consulted for information on how to use the system to develop
specifications and proofs.  The \emph{PVS Prover Guide}~\cite{PVS:prover}
is a reference manual for the commands used to construct proofs.

In this section, we provide a brief summary of the PVS specification
language, enumerate the key design principles behind the language, and
provide a brief example.

The following sections provide more details on the various features of
the language.  The lexical aspects of the language are detailed in
Section~\ref{lexical}.  Section~\ref{declarations} describes
declarations, Section~\ref{types} describes type expressions,
Section~\ref{expressions} describes expressions, and
Section~\ref{theories} describes theories, theory parameters, and
imports and exports of names.  Section~\ref{names} describes names and
name resolution, and Section~\ref{adts} describes the datatype
facility of PVS.  Finally, 
Appendix~\ref{grammar} provides the grammar of the language.

\section{Summary of the PVS Language}

A PVS specification consists of a collection of \emph{theories}.
Each theory consists of a \emph{signature} for the type names and
constants introduced in the theory, and the axioms, definitions, and
theorems associated with the signature.  For example, a typical
specification for a queue would introduce the \texttt{queue} type and the
operations of \texttt{enq}, \texttt{deq}, and \texttt{front} with their
associated types.  In such a theory, one can either define a
representation for the \texttt{queue} type and its associated operations in
terms of some more primitive types and operations, or merely axiomatize
their properties.  A theory can build on other theories: for example, a
theory for ordered binary trees could build on the theory for
binary trees.  A theory can be \emph{parametric} in certain specified
types and values: as examples, a theory of queues can be parametric in
the maximum queue length, and a theory of ordered binary trees can be
parametric in the element type as well as the ordering relation.  It is
possible to place constraints, called \emph{assumptions}, on the
parameters of a theory so that, for instance, the ordering relation
parameter of an ordered binary tree can be constrained to be a total
ordering.

The PVS specification language is based on simply typed higher-order
logic.  Within a theory, \emph{types} can be defined starting from
\emph{base} types (Booleans, numbers, etc.) using the function, record,
and tuple type constructions.  The \emph{terms} of the language can be
constructed using function application, lambda abstraction, and record and
tuple construction.

There are a few significant enhancements to the simply typed language
above that lend considerable power and sophistication to PVS.  New
uninterpreted base types may be introduced.  One can define a
\emph{predicate subtype} of a given type as the subset of individuals in a
type satisfying a given predicate: the subtype of nonzero reals is
written as \texttt{\{x:real | x /= 0\}}.  One benefit of such
subtyping is that when an operation is not defined on all the elements of
a type, the signature can directly reflect this.  For example, the
division operation on reals is given a type where the denominator is
constrained to be nonzero.  Typechecking then ensures that
division is never applied to a zero denominator.  Since the predicate used
in defining a predicate subtype is arbitrary, typechecking is undecidable
and may lead to proof obligations called \emph{type correctness
conditions} (TCCs).  The user is expected to discharge these proof
obligations with the assistance of the PVS prover.  The PVS type system
also features dependent function, record, and tuple type constructions.
There is also a facility for defining a certain class of abstract datatype
(namely well-founded trees) theories automatically.

\section{PVS Language Design Principles}

There are several basic principles that have motivated the design of
PVS which are explicated in this section.

\paragraph{Specification vs. Programming Languages.}
A specification represents requirements or a design whereas a program
text represents an implementation of a design.  A program can be seen as
a specification, but a specification need not be a program.  Typically,
a specification expresses \emph{what} is being
computed whereas a program expresses \emph{how} it is computed.  A
specification can be incomplete and still be meaningful whereas an
incomplete program will typically not be executable.  A specification
need not be executable; it may use high-level constructs, quantifiers
and the like, that need have no computational meaning.  However, there
are a number of aspects of programming languages that a specification
language should include, such as:
\begin{itemize}

\item the usual basic types: booleans, integers, and rational numbers

\item the familiar datatypes of programming languages such as arrays,
records, lists, sequences, and abstract datatypes

\item the higher-order capabilities provided by modern functional
programming languages so that extremely general-purpose operations can
be defined

\item definition by recursion

\item support for dividing large specifications into parameterized
modules

\end{itemize}

It is clearly not enough to say that a specification language shares some
important features of a programming language but need not be executable.
Any useful formal language must have a clearly defined
semantics\footnote{The PVS semantics are presented in a technical
report~\cite{PVS:semantics}.} and must be capable of being manipulated in
ways that are meaningful relative to the semantics.  A programming
language for example can be given a denotational semantics so that the
execution of the program respects its denotational meaning.  The reason
one writes a specification in a formal language is typically to ensure
that it is sensible, to derive some useful consequences from it, and to
demonstrate that one specification implements another.  All of these
activities require the notion of a justification or a proof based on the
specification, a notion that can only be captured meaningfully within the
framework of logic.

\paragraph{Untyped set theory versus higher-order logic}
\index{set theory}\index{higher-order logic}
Which logic should be chosen?  There is a wide variety of choices:
simple propositional logics, which can be classical or intuitionistic,
equational logics, quantificational logics, modal and temporal logics,
set theory, higher-order logic, etc.  Some propositional and modal
logics are appropriate for dealing with finite state machines where one
is primarily interested in efficiently deciding certain finite state
machine properties.  For a general purpose specification language,
however, only a set theory or a higher-order logic would provide the
needed expressiveness.  Higher-order logic requires strict typing to
avoid inconsistencies whereas set theory restricts the rules for forming
sets.  Set theory is inherently untyped, and grafting a typechecker onto
a language based on set theory is likely to be too strict and arbitrary.
Typechecking, however, is an extremely important and easy way of
checking whether a specification makes semantic sense (although 
for an opposing view, the reader is referred to a report by Lamport
and Paulson~\cite{Lamport&Paulson97}).  Higher-order
logic does admit effective typechecking but at the expense of an
inflexible type system.  Recent advances in type theory have made it
possible to design more flexible type systems for higher-order logic
without losing the benefits of typechecking.  We have therefore chosen
to base PVS on higher-order logic.

\paragraph{Total versus partial functions}
\index{function!total}\index{function!partial} In the PVS higher-order
logic, an individual is either a function, a tuple, a record, or the
member of a base type.  Functions are extremely important in higher-order
logic.  They are \emph{first-class} individuals, i.e., variables can range
over functions.  In general, functions can represent either \emph{total}
or \emph{partial} maps.  A total map from domain $A$ to range $B$ maps
each element of $A$ to some element of $B$, whereas a partial map only
maps some of the elements of $A$ to elements of $B$.  Most traditional
logics build in the assumption that functions represent total maps.
Partial functions arise quite naturally in specifications.  For example,
the division operation is undefined on a zero denominator and the
operation of popping a stack is undefined on an empty stack.

Some recent logics, notably those of VDM~\cite{Jones:VDM},
LUTINS~\cite{Farmer:functions}, RAISE~\cite{RAISE-tutorial},
Beeson~\cite{Beeson:book} and Scott~\cite{Scott79}, admit partial
functions.  In these logics, some terms may be \emph{undefined} by not
denoting any individuals.  Some of these logics have mechanisms for
distinguishing defined and undefined terms, while others allow
``undefined'' to propagate from terms to expressions and therefore must
employ multiple truth values.  In all these cases, the ability to
formalize partially defined functions comes at the cost of complicating
the deductive apparatus, even when the specification does not involve any
partial functions.  Though logics that allow partial functions are
extremely interesting, we have chosen to avoid partial functions in PVS.
We have instead employed the notion of a \emph{predicate
subtype}\index{predicate subtype}, a type that consists of those elements
of a given type satisfying a given predicate.  Using predicate subtypes,
the type of the division operator, for example, can be constrained to
admit only nonzero denominators.  Division then becomes a total operation
on the domain consisting of arbitrary numerators and nonzero denominators.
The domain of a \emph{pop} operation on stacks can be similarly restricted
to nonempty stacks.  PVS thus admits partial functions within the
framework of a logic of total functions by enriching the type system to
include predicate subtypes.  We find this use of predicate subtypes to be
significantly in tune with conventional mathematical practice of being
explicit about the domain over which a function is defined.

\section{An Example: \texttt{stacks}}\label{stacks-example}
\index{stacks example@\texttt{stacks} example}

In this section we discuss a specific example, the theory of
\texttt{stacks}, in order to give a feel for the various aspects of the
PVS language before going into detail.  Apart from the basic notation for
defining a theory, this example illustrates the use of type parameters at
the theory level, the general format of declarations, the use of predicate
subtyping to define the type of nonempty stacks, and the generation of
typechecking obligations.

\pvstheory{stacks-alltt}{{Theory \texttt{stacks}}}{stacks-alltt}

Figure~\ref{stacks-alltt} illustrates a theory for stacks of an arbitrary
type with corresponding stack operations.  Note that this is not the
recommended approach to specifying stacks; a more convenient and complete
specification is provided in Section~\ref{stacks-adt},
page~\pageref{stacks-adt}.

The first line introduces a theory named \texttt{stacks} that is
parameterized by a type \texttt{t} (the \emph{formal parameter} of
\texttt{stacks}).  The keyword \texttt{TYPE+} indicates that \texttt{t} is
a \emph{non-empty} type.  The uninterpreted (nonempty) type \texttt{stack}
is declared, and the constant \texttt{empty} and variable \texttt{s} are
declared to be of type \texttt{stack}.  The defined predicate
\texttt{nonemptystack?}~is then declared on elements of type
\texttt{stack}; it is \texttt{true} for a given \texttt{stack} element
iff\footnote{Iff is short for ``if and only if''.} that element is not
equal to \texttt{empty}.  The functions \texttt{push}, \texttt{pop}, and
\texttt{top} are then declared.  Note that the predicate
\texttt{nonemptystack?}~is being used as a type in specifying the
signatures of these functions; any predicate may be used as a type simply
by putting parentheses around it.

The variables \texttt{x} and \texttt{y} are then declared, followed by the
usual axioms for \texttt{push}, \texttt{pop}, and \texttt{top}, which make
\texttt{push} a stack constructor and \texttt{pop} and \texttt{top} stack
accessors.  Finally, there is the theorem \texttt{pop2push2}, that can
easily be proved by two applications of the \texttt{pop\_push} axiom.

This simple theorem has an additional facet that shows up during
typechecking.  Note that \texttt{pop} expects an element of type
\texttt{(nonemptystack?)} and returns a value of type \texttt{stack}.
This works fine for the inner \texttt{pop} because it is applied to
\texttt{push}, which returns an element of type \texttt{(nonemptystack?)};
but the outer occurrence of \texttt{pop} cannot be seen to be type correct
by such syntactic means.  In cases like these, where a subtype is expected
but not directly provided, the system generates a \emph{type-correctness
condition} (TCC).  In this case, the TCC is
\begin{pvsex}
  pop2push2_TCC1: OBLIGATION
    (FORALL (s: stack, y: t, x: t):
      nonemptystack?(pop(push(x, push(y, s)))))
\end{pvsex}
and is easily proved using the \texttt{pop\_push} axiom.  The system keeps
track of all such obligations and will flag the unproved ones during proof
chain analysis.

Parameterized theories such as \texttt{stacks} introduce theory schemas,
where the type \texttt{t} may be instantiated with any other nonempty
type.  To use the types, constants, and formulas of the \texttt{stacks}
theory from another theory, the \texttt{stacks} theory must be imported,
with \emph{actual parameters} provided for the corresponding theory
parameters.  This is done by means of an \texttt{IMPORTING} clause. For
example, the theory
\begin{pvsex}
  ustacks : THEORY
    BEGIN
    IMPORTING stacks[int], stacks[stack[int]]

    si : stack[int]
    sos : stack[stack[int]] = push(si, empty)
    END ustacks
\end{pvsex}
imports stacks of integers and stacks of stacks of integers.  The constant
\texttt{si} is then declared to be a stack of integers, and the constant
\texttt{sos} is a stack of stacks of integers whose top element is
\texttt{si}.  Note that the system is able to determine which instance of
\texttt{push} and \texttt{empty} is meant from the type of the first
argument.  In general, the typechecker infers the type of an expression
from its context.  


\include{rules}
\include{logic}
% Document Type: LaTeX
% Master File: semantics.tex
\documentclass [12pt,twoside]{cslreport} 
\usepackage{fleqn,alltt,spacecites,relsize,url,amssymb,verbatim}
%\usepackage{doublespace}
%\setlength{\textwidth}{6.5in}
%\setlength{\textheight}{9in}
%\setlength{\topmargin}{-.6in}
%\setlength{\oddsidemargin}{0in}
%\setlength{\evensidemargin}{0in}
\setlength{\textwidth}{5.5in}
\setlength{\textheight}{8in}
\setlength{\topmargin}{.25in}
\setlength{\oddsidemargin}{.5in}
\setlength{\evensidemargin}{0in}
\setlength{\mathindent}{0.2in}
%\input amssym.def
%\input amssym
%\input commands
%\includeonly{soundness}
%\input{/homes/rushby/tex/jmacros}
\newcommand{\thmbox}
   {{\ \hfill\hbox{%
      \vrule width1.0ex height1.0ex
   }\parfillskip 0pt }}
\newtheorem{thm}{Theorem}[chapter]
\newtheorem{prop}[thm]{Proposition}
\newenvironment{proof}{{\bf Proof. }}{\thmbox}
\newcommand{\Infrule}[3]{
{{\displaystyle\strut #1}\over{\displaystyle\strut #2}}\qquad\makebox[0pt][l]{\it #3}
}
\newcommand{\tuple}[1]{\langle #1 \rangle}
\newtheorem{example}[thm]{Example}
\newcommand{\aro}{\mathord\rightarrow} % see pages 154-155 of TeX manual
\newcommand{\mean}[1]{\lbrack\!\lbrack #1 \rbrack\!\rbrack}
\newcommand{\pair}[1]{\langle #1 \rangle}
\newcommand{\union}{\cup}
\newcommand{\funtype}[2]{[#1 \aro #2]}
\newcommand{\tupletype}[1]{[#1]}
\newcommand{\subtype}[3]{\{#1 : #2 | #3 \}}
\newcommand{\tauGamma}[1]{\tau(\Gamma)(#1)}
\newcommand{\Mgamma}[1]{{\mathcal M}(\Gamma\vbar\gamma)(#1)}
\newcommand{\proj}[1]{\mathtt{p}_{#1}}
\newcommand{\lis}[2]{#1_{1}, \ldots, #1_{#2}}
\newcommand{\listwo}[2]{#1_{1}, #1_{2}}
\newcommand{\ol}[1]{\overline{#1}}
%\newcommand{\ul}[1]{\underline{#1}}
\newcommand{\twob}{\mathbf{2}}
\newcommand{\oneb}{\mathbf{1}}
\newcommand{\zerob}{\mathbf{0}}
\newcommand{\reals}{\mathbf{R}}
\newcommand{\ttbool}{\mathtt{bool}}
\newcommand{\ttreal}{\mathtt{real}}
\newcommand{\tttrue}{\mathtt{TRUE}}
\newcommand{\ttfalse}{\mathtt{FALSE}}
\newcommand{\tttype}{\mathtt{TYPE}}
\newcommand{\ttcontext}{\mathtt{CONTEXT}}
\newcommand{\ttvar}{\mathtt{VAR}}
\newcommand{\ttconstant}{\mathtt{CONSTANT}}
\newcommand{\ttvariable}{\mathtt{VARIABLE}}
\newcommand{\tttheory}{\mathtt{THEORY}}
\newcommand{\ttif}{\mathtt{IF}}
\newcommand{\ttint}{\mathtt{int}}
\newcommand{\ttnat}{\mathtt{nat}}
\newcommand{\itkind}{\textit{kind}}
\newcommand{\ittype}{\textit{type}}
\newcommand{\itdef}{\textit{definition}}
\newcommand{\itformals}{\textit{formals}}
\newcommand{\vbar}{\ |\ }
\newcommand{\sima}{\stackrel{a}{\sim}} %/{\scriptsize{a}}
\newenvironment{Eg}[1]{\begin{example}[#1]\label{eg:#1}\em }{\thmbox\end{example}}
\newtheorem{definition}{Definition}
\newtheorem{lemma}{Lemma}
\newtheorem{theorem}{Theorem}
%\input{/homes/rushby/tex/prelude}
%\input{/homes/shankar/tex/ch-jmacros}

\newenvironment{Defn}[1]{\begin{definition}[#1]\label{defn:#1}}{
\thmbox\end{definition}}
%
\newenvironment{display}{\begin{alltt}\small\tt\vspace{0.3\baselineskip}}{\vspace{0.3\baselineskip}\end{alltt}}
%
%\newcommand{\gets}{\leftarrow}
\title{The Formal Semantics of PVS\thanks{Funded by National Aeronautics
and Space Administration Contract NAS1-18969, Task 11 and National Science
Foundation Grant CCR 9300444. 
}}
\author{
\begin{tabular}{ccc}
Sam Owre & and & Natarajan Shankar\vspace{-.05in}\\
{\small\tt owre@csl.sri.com}& & {\small\tt shankar@csl.sri.com}\\
\end{tabular}\\
URL: \url{http://www.csl.sri.com/sri-csl-fm.html}\\
SRI International\\
Computer Science Laboratory\\
Menlo Park CA 94025 USA
}
\date{Technical Report CSL-97-2R\\August 1997, Revised March 1999}
%\renewcommand{\today}{}
%\renewcommand{\is}{::=}
%\renewcommand{\baselinestretch}{2}
\begin{document}
%\bibliographystyle{alpha}
\pagenumbering{roman}
\maketitle
\setcounter{page}{2}
\mbox{}
\cleardoublepage
\begin{abstract}
\setcounter{page}{3} 
\thispagestyle{plain}

A specification language is a medium for expressing {\em what\/} is
computed rather than {\em how\/} it is computed.  Specification languages
share some features with programming languages but are also different in
several important ways.  For our purpose, a specification language is a
logic within which the behavior of computational systems can be
formalized.  Although a specification can be used to simulate the behavior
of such systems, we mainly use specifications to state and prove system
properties with mechanical assistance.

We present the formal semantics of the specification language of SRI's
Prototype Verification System (PVS)\@.  This specification language is
based on the simply typed lambda calculus.  The novelty in PVS is that it
contains very expressive language features whose static analysis (e.g.,
typechecking) requires the assistance of a theorem prover.  The formal
semantics illuminates several of the design considerations underlying PVS,
particularly the interaction between theorem proving and typechecking.
\end{abstract}
\cleardoublepage
\tableofcontents
%\listwotoffigures
\cleardoublepage
\setcounter{page}{0} 
\pagestyle{headings}
\pagenumbering{arabic}

\chapter{Introduction} \label{introduction} PVS is a system for specifying
and verifying properties of digital hardware and software systems.  The
specification language of PVS is designed to admit succinct, readable, and
logically meaningful specifications.  The PVS specification language is
designed for effective proof construction rather than efficient execution.
The design considerations underlying the language are therefore somewhat
different from those of a corresponding programming language.  For
example, the language contains constructs that can be statically
typechecked only with the assistance of a theorem prover.  This is
acceptable because the PVS specification language is intended for use in
conjunction with powerful support for automated theorem proving.  The
logic of PVS is based on a simply typed higher-order logic with function,
record, and product types, and recursive type definitions.  This type
system is extended with subtypes that are analogous to subsets, and with
dependently typed functions, records, and products.  The resulting type
system has several advantages.  It is possible, for instance, to
statically ensure that all array references are within their respective
array bounds.  PVS specifications are organized into theories that can be
parametric in types as well as individuals.  While the semantics of the
simply typed fragment is straightforward, the extensions such as
subtyping, dependent typing, and (theory-level) parametricity do pose
significant challenges.  This report presents a concise but idealized
definition of the PVS specification language and its intended formal
set-theoretic semantics.  It is neither an overview of the PVS language
nor a guide to the Prototype Verification System (see the PVS user
manuals~\cite{PVS:manuals}).


The primary purpose of the formal semantics is as a useful reference for
the developers and users of PVS\@.  The idealized core of the
specification language as presented here serves as a succinct foundation
for studying the expressive power of the language.  Pertinent questions
about PVS are answered directly by the formal semantics presented here:
\begin{enumerate}
\item\label{one} What is the semantic core of the language, and what is
just syntactic sugar?

\item\label{two} What are the rules for determining whether a given PVS
expression is well typed?

\item\label{three} How is subtyping handled, and in particular, how are
proof obligations corresponding to subtypes generated?

\item\label{four} What is the meaning, in set-theoretic terms, of a PVS expression or
assertion?

\item\label{five} Are the type rules sound with respect to the semantics? 

\item\label{six} Are the proof rules sound with respect to the semantics?

\item\label{seven} What is the form of dependent typing used by PVS, and
what kinds of type dependencies are disallowed by the language?

\item\label{eight} What is the meaning of theory-level parametricity, and
what, if any, are the semantic limits on such parameterization?

\item\label{nine} What language extensions are incompatible with the reference
semantics given here?
\end{enumerate}

Chapter~\ref{conclusion} summarizes the answers to these questions.


\section{Real versus Idealized PVS}

The semantic treatment in this report is incomplete in some important ways.
It does not treat the nonlogical parts of the language.  In particular,
it ignores arithmetic and recursive definitions.  It also omits
abstract datatypes~\cite{Shankar:ADT97}.  These will be treated in a future
expanded version. 

The present semantics also makes several idealizations from the real
PVS for 
the purpose of clarity.  While the semantic treatment is not
comprehensive, the idealization of PVS used here is faithful to the
implemented form of PVS.   

\begin{enumerate}
  \item \emph{No name resolution.}  All names must be in fully resolved form
with their theory name and actual parameters.  We regard name resolution as
a convenience provided by the PVS type checker and not an operation with any
semantic relevance.  A technical description of name resolution in PVS will
be given elsewhere.


  \item \emph{No overloading.}   As with name resolution, overloading is
a syntactic convenience with no semantic import. 

\item \emph{No \texttt{IMPORTING}s.}  The importing of theories is a 
hint to  name resolution.  The semantic definition assumes that all
instances of theories declared prior to the present one are visible.   
  \item \emph{Variable declarations ignored.}  All variables must be 
locally declared.   Global variable declarations are regarded as a syntactic
convenience.  
%  \item TCCs are not kept around to be used as lemmas.
  \item \emph{No records.}  These are ignored in the semantic treatment
since product types capture all the semantically essential features of records.
\end{enumerate}


\section{Semantic Preliminaries}

The PVS specification language is based on higher-order logic.  This means
that variables can range over individuals (such as numbers) as well as
functions, functions of functions, and so on.  As is well known, some type
distinction is needed; otherwise, it is easy to obtain a
contradiction by defining the predicate $N(P)$ as $ \neg(P(P))$ so that
both $N(N)$ and $\neg N(N)$ hold.  
In the theory of types~\cite{Church:types}, the universe is stratified
into distinct types 
so that a predicate can  be applied only to a \emph{lower} type
and thus cannot be applied to itself.

Types also serve as a powerful mechanism for detecting syntactic and
semantic errors through typechecking.  This role of types is best
exemplified by their use in various programming languages such as Algol,
Ada, and ML, and is also heavily emphasized in the PVS type
system.

The desirability for strong typing in a specification
logic is not widely accepted.  Fraenkel {\em et
al\/}~\cite{Fraenkel-etal84} express
the opinion that such 
typing is repugnant in a mathematical logic since it constrains
expressiveness by not allowing individuals of differing types to be
treated uniformly.  Lamport~\cite{TLA:TOPLAS94} argues that type
correctness is like any other 
program property and should be established by means of a proof rather than
by syntactic restraints.  Lamport and Paulson~\cite{Lamport&Paulson97}
analyze the tradeoffs between typed and untyped specification languages.  
We claim that
\begin{enumerate}
\item Types impose a useful discipline on the specification.
\item Types lead to easy and early detection of a large class of
syntactic and  semantic errors.
\item Type information is useful in mechanized reasoning.
\end{enumerate}


\begin{comment}
The first typed higher-order logic was the Principia Mathematica (PM) system of
Russell and Whitehead.  They were the first to propose a stratified
hierarchy of types as a means avoiding certain paradoxes.  The PM type
system is quite complicated.  Church proposed his \emph{simple theory of
types} as a higher-order logic based on a simply typed lambda calculus.
Church's system forms the basis for most modern higher-order logics
including that of PVS.  The Automath system of de~Bruijn pioneered the use
of dependent types (analogous to the variant records of Pascal) as a basis
for intuitionistic logic.\footnote{Similar ideas were 
independently proposed by several other researchers.}  Dependent types 
form an important component of the PVS type system.  
Girard and Reynolds, independently extended the typed lambda calculus
to a second-order or \emph{polymorphic} type system so that, for example,
an identity function for each type could be obtained by specializing a
generic identity function to the required type.
\end{comment}

The semantics of a higher-order logic is given by mapping the well-formed
types of the logic to {\em sets\/}, and the well-formed terms of the logic to
{\em elements\/} of the sets representing their type.  The set
constructions we use can be formalized within Zermelo-Fraenkel set theory
with the axiom of choice (ZFC).   The intended interpretation
of a function type in higher-order logic is that it represents the set of
all functions from the set representing the domain type to the set
representing the range types.\footnote{It is only in the {\em standard\/}
model of higher-order logic that the function type is required to
represent the set of all 
functions from the domain set to the range set.  Higher-order logic can be
interpreted in {\em general models\/} where the function type can be
interpreted in any manner as long as it satisfies the various axioms
such as application, abstraction, and extensionality~\cite{Andrews:book}.
Higher-order logic 
is complete with respect to the general models interpretation so that
a statement that is valid in all models is provable.  It is, however,
incomplete with respect to the standard model.}  
PVS also has predicate subtypes that are to
be interpreted over the subsets of the set representing the parent type.

The semantics of PVS will be given by considering a sequence of
increasingly expressive fragments of PVS\@.   
The semantics of each fragment of PVS will be presented in three steps.
The first step is to define a set-theoretic universe containing
enough sets to represent the PVS types.  The second step is to define a
typechecking operation that determines whether a given PVS expression is
well typed.  The third step is to define a semantic function that assigns
a representation in the semantic universe to each well-typed PVS type and
term. 



\begin{comment}
I am not exactly sure where the axiom of choice is really needed.
Since PVS builds in a choice operator via parametric theories, it seems
that this will be needed.
\end{comment}


We first lay out the ZFC set constructions needed for defining the
semantics of PVS.  The base types in PVS consist of the Booleans
\texttt{bool} and the real numbers \texttt{real}\@.  The Booleans can be
modeled by any two-element set, say $\twob$ consisting of the elements
$\zerob$ and $\oneb$, where $\zerob$ is the empty set and the only element
of the set $\oneb$\@.  The real numbers can be captured by means of
Dedekind cuts or Cauchy sequences, and we label this set $\reals$\@.

To define the semantics, we need a universe that contains the sets
$\twob$ and 
$\reals$
and is closed under Cartesian products (written as $X\times Y$)
and power sets (written as $\wp(X)$)\@.  Note that
functions are modeled as \emph{graphs}, that is, sets of ordered pairs,
so that a function type $\funtype{A}{B}$ is represented by a subset of
the powerset $\wp(\mean{A}\times \mean{B})$
of the Cartesian product of the sets $\mean{A}$ and $\mean{B}$
representing $A$ and 
$B$, respectively\@.  A set $F$ that is a subset of $X \times Y$ is the graph of a
function with domain $X$ and range $Y$ if for every $x\in X$ there is a $y\in
Y$ such that $\pair{x, y}\in F$, and whenever $\pair{x, y}\in F$
and $\pair{x, y'}\in F$, we have $y = y'$\@.  For such a set $F$,
$ \textit{Function}(F)$ holds and $ \textit{dom}(F) = X$\@.  
The set of graphs of total functions from a set $Y$ to a set $X$
is represented as $X^Y$\@.   If $F$ is the graph of a function and $t$ an
element in its domain, then
$F(t)$ represents the result of applying the function $F$ to $t$\@.
At the semantic level,  a function $F$ will never be applied to an argument
$t$ outside $\textit{dom}(F)$, because in the PVS language,
a function application is typechecked so that the
argument expression has the same type as the domain type of the
function expression.  

We can model the entire type universe of the simply typed fragment of PVS
by the set $U$, which is defined cumulatively by starting from the base
sets $\twob$ and $\reals$, and including the Cartesian products, the
function spaces, and subsets of previously included sets, at each stage.
Cartesian products are used to model products in PVS, and function spaces
model function types.  Subsets are needed to model predicate subtypes.
It is sufficient to iterate these stages up to the ordinal $\omega$\@.
\begin{Defn}{type universe}
\begin{eqnarray*}
  U_0 & = & \{ {\twob}, {\reals}  \} \\
  U_{i+1} & = & U_i \,\union\, \{X\times Y \vbar X, Y \in U_i\} \,\union\,
                           \{X^Y \vbar X, Y \in U_i \} \,\union\,
                           \bigcup_{X\in U_i}\wp(X)\\
  U_{\omega} & = & \bigcup_{i\in\omega} U_i\\
  U & = & U_{\omega}
\end{eqnarray*}
\end{Defn}


We refer to $U$ as the \emph{basic universe}.\footnote{The inclusion of $X^Y$ in $U$ is actually redundant but aids clarity.}   The semantic definitions
below will assign a set in $U$ to each PVS type and an element
in $\bigcup U$ to each well-typed term of PVS.  The \emph{rank} of a set
$X$ in $U$ is the least $i$ such that $X\in U_i$\@.  The notion of rank
plays an important role in the semantics of dependent types and
parametric theories.

\section{Related Work}

There is a long history of work in specification languages.
Many  ideas similar to those underlying the PVS specification language
also occur in other specification languages.  

The {\em wide-spectrum\/} languages are typically based on set theory or
higher-order logic.  The language VDM is one of the earliest such
specification formalisms~\cite{Jones:VDM}.  It is based on a first-order
logic with partial functions augmented with datatype axioms.  The datatype
theories in VDM include those for finite sets, maps, sequences, and
recursive datatypes such as lists and trees.  VDM has a notion of datatype
invariants that yields a simple form of predicate subtyping.  Operations
on state are specified in terms of pre-condition/post-condition pairs.
Specifications are structured into parameterized modules.  In contrast to
VDM, the PVS language is based on strictly typed higher-order logic with a
built-in notion of predicate subtyping and dependent typing.  The
resulting PVS logic is more compact in that many of the datatypes that are
presented axiomatically in VDM can be defined within PVS.  There is no
built-in notion of state in PVS since it is possible to use the
higher-order logic of PVS to define a variety of state-based formalisms,
including various linear and branching-time temporal logics.  VDM uses a
3-valued logic for the logical connectives in order to deal
with partial functions, whereas PVS uses a classical 2-valued logic and
predicate subtyping to assign a type to a partial function as a total
function on its domain of definition.  Jones~\cite{Jones:VDM} provides
only an informal semantics for VDM.  The RAISE system is a comprehensive
toolset based on the ideas of VDM~\cite{RAISE:book}.

The Z specification language~\cite{Z-understanding} is another
wide-spectrum language based on a typed first-order set theory.  A Z
specification is a collection of schemas consisting of declarations of
types and constants accompanied with invariants.  Z schemas can either
specify datatype invariants or pre-condition/post-condition constraints.
A schema calculus is used to combine schemas using logical connectives.
Spivey~\cite{Z-understanding} presents a formal semantics for Z without
giving a proof system or a soundness proof.  Spivey's treatment of
partial functions in the Z semantics employs the commonly used convention
that $f(a)$ when $a$ is not in the domain of $a$ is some arbitrarily
chosen value.  This is fine for most purposes but can be confusing when
dealing with recursively defined partial functions.  For example, the
definition $\mathit{bad}(x) = 1 + \mathit{bad}(x)$ is everywhere undefined but
admitting it as an axiom leads to an immediate contradiction.  Z also
lacks any mechanism for conservative extensions such as definitional
principles for constants and datatypes so that the consistency of Z
specification has to be demonstrated by exhibiting a model.

Algebraic specification languages like OBJ~\cite{popl85} and
Larch~\cite{Larch:book} provide an 
equational/rewriting framework for specifying datatypes and operations on
datatypes.   OBJ has many of the same theory parameterization mechanisms
as PVS.  The subsort mechanism in OBJ is also similar except that
it is handled by introducing {\em retracts\/} or runtime checks
rather than proof obligations generated by the type checker.
The OBJ logic is quite restricted compared to PVS since it is based
on a first-order, equational framework with an {\em initial\/} semantics
where two ground terms are distinct unless they can be proved equal.
OBJ has very limited support for proof development and is primarily
intended as an executable specification language.

The specification languages that are closer to PVS are those that
accompany various automated proof checking systems.  The closest of these
is {\sc Ehdm}~\cite{EHDM:manuals}, which employs a similar higher-order
logic with subtyping and proof obligation generation.  {\sc Ehdm} lacks
many of the features of PVS: subtyping is restricted to type declarations
and there is no dependent typing.

Higher-order logic is used by other systems such as
HOL~\cite{Gordon&Melham:HOL} and TPS~\cite{andrews84}.
Both HOL and TPS employ simply typed higher-order logic 
without features such as subtyping, dependent typing, or parametric
theories.  
Andrews~\cite{Andrews:book} gives a thorough account of the semantic
aspects of higher-order logic.  The formal semantics of the HOL logic are
carefully outlined (by Pitts) in the book by Gordon and
Melham~\cite{Gordon&Melham:HOL}.

Systems like Coq~\cite{COQ} and Nuprl~\cite{Nuprl-book} are based on
intuitionistic higher-order logics.  Coq allows quantification over types,
whereas Nuprl has quantification over a hierarchy of type universes.  Both
logics admit dependent typing.  The set-theoretic semantics of dependently
typed intuitionistic type theories has been studied by
Dybjer~\cite{DybjerP:indsfm} and Howe~\cite{Howe91,Howe96}.  Not
surprisingly, their semantic treatment of dependent typing is similar to
the one given here but they do not delimit the possible dependencies as is
done with the PVS semantics.  The PVS semantics presented here clearly
specifies the kind of type dependencies that are disallowed in the logic.
Dybjer and Howe also do not address subtyping but do describe the
semantics of language features missing in PVS (type universes in the case
of Howe, and inductive families in the case of Dybjer).  Dybjer does not
identify the universe over which terms and types are interpreted.  Howe
requires an infinite sequence of inaccessible cardinals for his universe
construction.


\section{Outline}

In Chapter~\ref{simple}, we define the syntax and semantics of the simply
typed fragment of PVS.  Type definitions are also introduced in this
chapter along with the definition of definitional equivalence on types.
Chapter~\ref{subtypes} adds subtyping to the simply typed fragment and
specifies the additional type rules and semantic definitions that are
needed.  Chapter~\ref{dependent} extends the language with dependent
function and product types.  Theories and parametric theories are introduced
into the language in Chapter~\ref{theories}.  The type rules and semantics
for conditional expressions and the logical connectives defined using
conditional expressions are introduced in Chapter~\ref{conditionals}.
Chapter~\ref{proof theory} specifies the axioms and inference rules of
PVS\@.
 
\chapter{The Simple Type Theory}\label{simple}

\begin{comment}
The types of expression
are determined from the types of their subexpressions.  In the base case,
the types of atomic expressions must be given in a context.  There are
certain well-formedness constraints on contexts that we now specify.
A context is a sequence of declarations.  Declarations can either
be type declarations of the form $T\ :\ \mathtt{TYPE}$ where $T$ is a symbol,
constant declarations of the form $c\ :\ T$, or variable declarations of
the form $x\ :\ \mathtt{VAR} T$.  
The empty context $\{\}$ is a context.  If $\Gamma$ is a context,
then
\end{comment}

PVS is a strongly typed specification language.  The simply typed fragment
includes types constructed from the base types by the function and product
type constructions, and expressions constructed from the constants and
variables by means of application, abstraction, and tupling.  Expressions
are checked to be well typed under a {\em context\/}, which is a partial
function that assigns a {\em kind\/} (one of \texttt{TYPE},
\texttt{CONSTANT}, or \texttt{VARIABLE}) to each symbol, and a type to the
constant and variable symbols.  We use the metavariables $\Gamma$,
$\Delta$, and $\Theta$ to range over contexts.  The metavariables $A$,
$B$, and $T$ range over PVS type expressions, the metavariables $r$ and
$s$ range 
over symbols (identifiers), the metavariables $x$ and $y$ range over PVS
variables, and the metavariables $a$, $b$, $f$, and $g$ range over PVS
terms.  Given 
a context $\Gamma$ and a symbol $s$, we say that $\Gamma(s)$ is
undefined if $s$ is not declared in $\Gamma$\@.

The \emph{pretypes} of the simple type theory include the \emph{base}
types such as \texttt{bool} and \texttt{real}.  A \emph{function} pretype
from domain pretype $A$ to range pretype $B$ is constructed as
$\funtype{A}{B}$\@.  A \emph{product} pretype of ${\listwo{A}{n}}$ is
constructed as $\tupletype{\listwo{A}{n}}$.
\begin{comment}
We make no
distinction between a type $A$ and the singleton tuple type $[A]$.  We
also disallow a 0-tuple type.
\end{comment}
A \emph{type} is a pretype that has been
typechecked in a given context.  Types in the simple type theory are
simple enough  that the only distinction between pretypes and types is
that the symbols in a type must be appropriately declared in the given
context.

\begin{Eg}{pretypes} 
 \texttt{bool}, \texttt{real},
 $\tupletype
{\mathtt{bool}, \mathtt{real}}$,
 $\funtype{\tupletype{\mathtt{real}, \mathtt{bool}}}{ \mathtt{bool}}$.
\end{Eg}

The {\em preterms\/} of the language consist of the constants, variables,
pairs, projections, applications, and abstractions.  The metavariables $c$
and $d$ range over constants.  Pairs are of the form $(a_1, a_2)$ where
each $a_i$ is a preterm.  Applications have the form $f\ a$ where $f$ and
$a$ are preterms.  A pair projection is an expression of the form
$\proj{i}\ a$, where $i\in\{1, 2\}$\@.
\begin{comment}
As with types, we make no
distinction between the singleton tuple preterm $(a)$ and the preterm
$a$.
\end{comment}
Lambda abstractions have the form $\lambda (x : T): a$, where $T$
is a pretype and $a$ is a preterm.
\begin{comment}
$\lambda (x_1 : T_1, \ldots,
x_n : T_n): a$, where each $T_i$ is a type, each $x_i$ is a symbol so that
$x_i\not\equiv x_j$ for $i\neq j$, and $a$ is a preterm.
\end{comment}
Parentheses
are used for disambiguation.  A \emph{term} is a preterm that has been
typechecked in a given context.

\begin{Eg}{preterms}
\texttt{TRUE}, $\neg~\mathtt{TRUE}$,
$\lambda~(x~:~\mathtt{bool}):~\neg(x)$,\hfill\\
$\proj{2}~(\mathtt{TRUE}, \mathtt{FALSE})$,
$(\mathtt{TRUE},  \lambda~(x~:~\mathtt{bool})~:~\neg~(\neg~x))$.
\end{Eg}

\section{Contexts}

A context is 
 a sequence of
declarations, where each declaration is either a type declaration
$s\ :\ \mathtt{TYPE}$, a constant declaration $c\ :\ T$ where $T$ is a type,
or a variable declaration $x\ :\ \mathtt{VAR}\ T$.
Preterms and pretypes are typechecked with respect to a given context.  
The empty context is represented as $\{\}$\@.  
The well-formedness rules
for contexts are presented below.
A context can also be applied as a
partial function so that for a symbol $s$ with declaration $D$, $(\Gamma, s
: D)(s) = D$ and 
$(\Gamma, s : D)(r) = \Gamma(r)$ for $r\neq s$.
If $s$ is not declared in $\Gamma$, then $\Gamma(s)$ is undefined. 
If $\Gamma$ is a context, then
for any symbol $s$, the kind of the symbol $s$ in $\Gamma$ is given by $\textit{kind}(\Gamma(s))$.  If the kind of $s$ in $\Gamma$ is \texttt{CONSTANT} or
\texttt{VARIABLE}, then the $ \textit{type\/}(\Gamma(s))$ is the type assigned
to $s$ in $\Gamma$.  
%\memo{Variable declarations are unrelated to variable declarations in
%PVS and correspond to binding declarations.}
\newcommand{\coln}[2]{#1\ : #2}

\begin{Eg}{context}\hfill\\
$ \mathtt{bool}\ :\  \mathtt{TYPE},\hspace{.2cm} \coln{ \mathtt{TRUE}}{
\mathtt{bool}},\hspace{.2cm} \coln{ \mathtt{FALSE}}{
\mathtt{bool}},\hspace{.2cm} \coln{x}{ \mathtt{VAR}\ \funtype{\tupletype{
\mathtt{bool}, \mathtt{bool}}}{ \mathtt{bool} }}$ 
\end{Eg}


\section{Type Rules}

The type rules for the simple type theory are given by a recursively
defined partial function
$\tau$ that assigns
\begin{enumerate}
\item  A type $\tau(\Gamma)(a)$ to a preterm $a$ that
is well typed with
respect to a context $\Gamma$. 
\item  The keyword \texttt{TYPE} as the result of $\tauGamma{A}$
when $A$ is a  well-formed type under context $\Gamma$\@.

\item  The keyword \texttt{CONTEXT} as the result of
$\tauGamma{\Delta}$ when $\Delta$ is a well-formed context
under context $\Gamma$\@.  The context $\Gamma$ is empty for the
simply typed fragment so that typechecking is always invoked as
$\tau()(\Gamma)$\@. 
\end{enumerate}
Otherwise, $\tau$ is undefined in the case of an ill-typed preterm or an
ill-formed type or context. 

The type rules are given by the recursive definition for $\tau$\@.
Typechecking in PVS assigns a ``canonical'' type to a preterm.
Customarily, type rules are presented as inference rules, but
a functional presentation is more appropriate for PVS since
\begin{enumerate}
\item The type assignment is deterministic.  A term can, in
general, though not in the simply typed fragment,  be assigned
a number of types but it always has at most one canonical type.  

\item The soundness proof need only show that the meaning of the term
is an element of the meaning of its canonical type.  Thus, only
the canonical type derivation for a term has to be shown sound and not
every valid type derivation. 

\item The meaning of a term is therefore given by recursion on the term
itself and not on its typing derivation.  There is no need to show
separately that this meaning is {\em coherent\/}, that is, independent of
the typing derivation.
\end{enumerate}
  A functional presentation of the type rules also
leads to natural and straightforward soundness arguments.  
Note that the well-formedness rules for contexts and types are trivial
in the simply typed situation but become more meaningful when the type
theory is extended.  Note also that in the type rules for expressions and
types, the well-formedness of the relevant context is not explicitly
checked.  These rules do preserve the well-formedness of the context
in each recursive call so that if the initial context is well formed, then
so is every intermediate one.    
\begin{Defn}{type rules}
\begin{eqnarray*}
  \tau()(\{\}) & = & \ttcontext \\
%
  \tau()(\Gamma, s\ :\ \tttype{}) & = & \ttcontext, \mbox{ if }
  \Gamma(s) \mbox{ is undefined}\\& &
  \mbox{and } \tau()(\Gamma) = \ttcontext\\
%
  \tau()(\Gamma, c : T) & = & \ttcontext, \mbox{ if }
   \Gamma(c) \mbox{ is undefined,}\\ & & 
\tauGamma{T} =
      \tttype{},\\
   & & \mbox{and } \tau()(\Gamma) = \ttcontext\\
%
  \tau()(\Gamma, x : \ttvar{}\ T) & = & \ttcontext, \mbox{ if }
  \Gamma(x) \mbox{ is undefined,}\\ & & 
\tauGamma{T}
    = \tttype{},\\
    & & \mbox{and } \tau()(\Gamma) = \ttcontext\\ 
%
  \tauGamma{s} & = & \tttype{}, \mbox{ if } \textit{kind\/}(\Gamma(s)) =
\tttype{}\\
%
  \tauGamma{\funtype{A}{B}} & = & \tttype{}, \mbox{ if } \tauGamma{A} =
     \tauGamma{B} = \tttype{}\\
%
  \tauGamma{\tupletype{\listwo{A}{n}}} & = & \tttype{}, \mbox{ if }
     \tauGamma{A_i} = \tttype{} \mbox{ for } 1\leq i\leq 2 \\
%
  \tau(\Gamma)(s) & = &  \textit{type\/}(\Gamma(s)),\\
                    & &  \mbox{ if } 
\textit{kind}(\Gamma(s)) \in \{ \ttconstant{}, \ttvariable \} \\
%
 \tau(\Gamma)(f\ a) & = & B, \mbox{ if } \tau(\Gamma)(f) = \funtype{A}{B}
\mbox{ and } \tau(\Gamma)(a) = A \\
%
\tauGamma{\lambda (x : T): a} & = & \funtype{T}{\tau(\Gamma, x : \ttvar{}\ T)(a)},
\mbox{ if } \Gamma(x) \mbox{ is undefined}\\& & 
  \mbox{and }\tauGamma{T} = \tttype{}\\
%\tau(\Gamma)(\lambda (x_1 : T_1, \ldots, x_n : T_n): a) & = &
%         \begin{array}[t]{l}
%	   [{\tupletype{\listwo{T}{n}}} \aro \\
%   \tau(\Gamma, x_1 : \ttvar{}\ T_1, \ldots, x_n : \ttvar{}\ T_n)(a)]
%	 \end{array} \\ & &
%\mbox{if } 
%   \tauGamma{T_i} = \tttype{}, \mbox{ for } 1\leq i\leq n,\\
%& & \mbox{and  } x_i \not\equiv x_j, \Gamma(x_i) \mbox{ is undefined},\\
% & &  \mbox{ for } 1\leq i, j \leq n, i\neq j \\
%
\tau(\Gamma)((a_1, a_2)) & = &
    [\tau(\Gamma)(a_1),  \tau(\Gamma)(a_2))]\\
%
\tau(\Gamma)(\proj{i}\ a) & = & T_i, \mbox{ where }\\ & & 
 \tau(\Gamma)(a) = [T_1, T_2] %\mbox{ for some } n, n\geq i
\end{eqnarray*}
\end{Defn}

 In the type rule for lambda abstraction, the constraint that
$\Gamma(x)$ must be undefined can be satisfied by suitably renaming the
bound variable since we treat terms as equivalent modulo the renaming of
bound variables.  

\begin{comment}
The above definition suggests that projection applications should
not be treated as applications.
\end{comment}

\begin{Eg}{type rules}  Let $\Omega$ label the context
$\coln{ \mathtt{bool}}{\tttype{}}$,
            $\coln{ \mathtt{TRUE}}{\mathtt{bool}}$,
            $\coln{ \mathtt{FALSE}}{ \mathtt{bool}}$
\begin{eqnarray*}
\tau()(\{\}) & = & \ttcontext\\
\tau()(\Omega)
   & = & \ttcontext\\
\tau(\Omega)(\funtype{\tupletype{ \mathtt{bool}, \mathtt{bool}}}{
\mathtt{bool}}) & = & \tttype{}\\
\tau(\Omega)(( \mathtt{TRUE}, \mathtt{FALSE})) & = & \tupletype{ \mathtt{bool},
\mathtt{bool}}\\
\tau(\Omega)( \proj{2} ( \mathtt{TRUE}, \mathtt{FALSE})) & = & \mathtt{bool}\\
\tau(\Omega)(\lambda (\coln{x}{ \mathtt{bool}}) : \mathtt{TRUE}) & = &
\funtype{ \mathtt{bool}}{ \mathtt{bool}}
\end{eqnarray*}
\end{Eg}

\section{Semantics}

Recall that a preterm $a$ with a type assigned by $\tau$  under context $\Gamma$
is said to be a term of type $\tau(\Gamma)(a)$
in the context $\Gamma$.  If $\gamma$ 
is an \emph{assignment} for the symbols declared in context $\Gamma$, the
semantics of the simple type theory of PVS 
is given by  mapping  a type $T$ to a (possibly empty) set
$\mathcal{M}(\Gamma\vbar \gamma)(T)$, 
and a term $a$ with assigned type $T$ to an element of the set $\mathcal{
M}(\Gamma\vbar \gamma)(T)$ in the basic universe $U$\@. 
The assignment $\gamma$ is a list of bindings of the form $\{s_1\gets
t_1\}\ldots \{s_n\gets t_n \}$\@.  The application of an assignment
$\gamma$ to a symbol $s$ is such that $\gamma \{s \gets t\} (s)$ is $t$,
whereas $\gamma \{r \gets t\}(s)$ is $\gamma(s)$ when $r\not\equiv s$\@.

The  meaning function $\mathcal{M}$ returns the meaning of
a well-formed type $A$  
and a well-formed expression $a$ in the context $\Gamma$ under an
assignment $\gamma$ as $\mathcal{M}(\Gamma\vbar \gamma)(A)$ 
and 
$\mathcal{M}(\Gamma\vbar \gamma)(a)$,respectively. 
\begin{comment}
The base types 
\texttt{bool} and \texttt{real} are mapped to $\twob$ and $\reals$,
respectively.
\end{comment}

The meanings of type names, constants, and variables
declared in $\Gamma$ are obtained from the assignment $\gamma$\@.  A
function type is mapped to the 
corresponding function space.  A product type is mapped to the corresponding
Cartesian product.
\begin{comment}
an $n$-tuple is represented by an $(n+1)$-fold
Cartesian product terminated by the sole element $\zerob$ of the set \oneb\@.
This representation  distinguishes a 3-tuple
$\tupletype{\ttbool, \ttbool, \ttbool}$ from the 
2-tuple $\tupletype{\ttbool, \tupletype{\ttbool, \ttbool}}$\@.
\end{comment}
An application term is interpreted by 
means of set-theoretic function application.  A lambda abstraction yields the
graph of the corresponding function.  A pair expression is mapped
to the corresponding set-theoretic ordered pair.    
\begin{Defn}{meaning function}
\begin{eqnarray*}
%% \begin{comment}
%%   \Mgamma{ \mathtt{bool}} & = & {\twob} \\
%%   \Mgamma{ \mathtt{real}} & = & {\reals}\\
%%   \Mgamma{ \mathtt{TRUE}} & = & {\oneb}\\
%%   \Mgamma{ \mathtt{FALSE}} & = &{\zerob}\\
%% \end{comment}
  \mathcal{M}(\Gamma\vbar \gamma)(s) & = &  \gamma(s),\\
                    & &  \mbox{ if } 
\textit{kind}(\Gamma(s)) \in \{ \tttype{}, \ttconstant{}, \ttvariable \}
\\
%
  \mathcal{M}(\Gamma\vbar \gamma)(\funtype{A}{B}) & = & \Mgamma{B}^{\Mgamma{A}}\\
%
  \Mgamma{\tupletype{\listwo{T}{n}}} & = &  \Mgamma{T_1} \times %%\ldots\times
\Mgamma{T_2} %\times {\oneb}
\\
%
\mathcal{M}(\Gamma\vbar \gamma)(f\ a) & = & (\mathcal{M}(\Gamma\vbar \gamma)(f))(\mathcal{M}(\Gamma\vbar \gamma)(a)) \\
%
\Mgamma{\lambda (x : T): a} & = & \{\tuple{y, z} \vbar \begin{array}[t]{l}
						       y\in\Mgamma{T},\\
                					z = \mathcal{M}(\Gamma, x: \ttvar{}\ T\vbar \gamma\{x\gets y\})(a)\}
						       \end{array}\\
%               
%\mathcal{M}(\Gamma| \gamma)(\lambda (x_1 : T_1, \ldots, x_n : T_n): a) & = &
%  {\begin{array}[t]{l}\{ \tuple{\tuple{\listwo{y}{n}, {\zerob}}, z} | \\
%					y_1\in \Mgamma{T_1}\\
%                                        \vdots\\
%                                        y_n\in \Mgamma{T_n}\\
%                                   z = \mathcal{M}\left(
%        \begin{array}{l}
%  \Gamma, x_1 : \ttvar{}\ T_1, \ldots, x_n : \ttvar{}\ T_n;\\
%  \gamma\{x_1 \gets y_1\}\ldots \{x_n\gets y_n\}
%	\end{array}\right)(a)\} 
%				      \end{array}}\\
%
\Mgamma{(\listwo{a}{n})} & = & \tuple{\Mgamma{a_1},  \Mgamma{a_2}} \\
%
\Mgamma{\proj{i}\ a} & = & t_i, \mbox{ where } \Mgamma{a} =
\tuple{\listwo{t}{n}}
\end{eqnarray*}
\end{Defn}

\begin{Eg}{meaning function}
 Let $\omega$ be an assignment for the context $\Omega$ in
Example~\ref{eg:type rules}, of the form
$$\{\ttbool\gets \twob\}\{ \mathtt{TRUE}\gets \oneb\}\{
\mathtt{FALSE}\gets \zerob\}$$
then
\begin{eqnarray*}
\mathcal{M}(\Omega\vbar \omega)(\tupletype{\ttbool, \ttbool}) & = &
\twob\times\twob\\
%
\mathcal{M}(\Omega\vbar \omega)(( \mathtt{TRUE}, \mathtt{FALSE})) & = &
\tuple{\oneb, \zerob}\\
%
\mathcal{M}(\Omega\vbar \omega)(\lambda (x : \ttbool): \mathtt{TRUE}) & = &
\{\tuple{\zerob, \oneb}, \tuple{\oneb, \oneb}\}
\end{eqnarray*}
\end{Eg}

\begin{Defn}{satisfaction}
A context assignment $\gamma$ is said to \emph{satisfy} a context $\Gamma$
(in symbols $\gamma\models \Gamma$) 
iff
\begin{enumerate}
 \item $\gamma(\ttbool) = {\twob},$
 \item $\gamma(\tttrue) = {\oneb},$
 \item $\gamma(\ttfalse) = {\zerob},$
\item $\gamma(s)\in U$ whenever $ \itkind{}(\Gamma(s)) = \tttype$, and
 \item $\gamma(s) \in \Mgamma{ \ittype{}(\Gamma(s))}$\hfill\\\hspace*{1in}
whenever $ \itkind{}(\Gamma(s)) \in \{ \ttconstant,
\ttvariable \}$.
\end{enumerate}
\end{Defn}
\begin{Eg}{satisfaction}\hfill\\\begin{enumerate}
\item The assignment $\omega$ satisfies context $\Omega$.

\item The assignment $\omega \{ \mathtt{one}~\gets~{\oneb}\}\{
\mathtt{zero}~\gets~\zerob\}$ satisfies the context
$\Omega,~\mathtt{one~:~TYPE},~\mathtt{zero~:~one}$\@.  
\end{enumerate}
\end{Eg}

We need one useful proposition that asserts that typing judgements are not
invalidated when the context is extended.
\begin{prop}\label{context-weakening} %where is this used? NSH(9.6.95)
If $\tau()(\Gamma) = \tau()(\Gamma') = \ttcontext$ and $\Gamma$ is
a prefix of $\Gamma'$, then for all pretypes $A$, $\tauGamma{A} = \tttype$
implies $\tau(\Gamma')(A) = \tttype$, and for all preterms $a$,
$\tauGamma{a} = A$ implies $\tau(\Gamma')(a) = A$\@.  
\end{prop}

The following theorems follow from the induction suggested by the
definitions of $\tau$ and $\mathcal{M}$.  The first of these
is straightforward and is given without proof.
\begin{theorem}[type construction]  \label{type construction}
If $\tau()(\Gamma) = \ttcontext$ and $\tauGamma{a} = A$, then
$\tauGamma{A} = \tttype$.
\end{theorem}
%
\begin{theorem}[type soundness]\label{simple-type-semantics}
 If $\tau()(\Gamma) = {\ttcontext}$, 
   $\gamma$ satisfies $\Gamma$, and
   $\tauGamma{A} = {\tttype{}} $, then 
$\Mgamma{A} \in U$\@.
\end{theorem}
\begin{proof}
The proof is by induction on the structure of the pretype $A$.
Recall that if $X\in U$, then for some $i$, $X \in U_i$\@.  
This yields three cases:
\begin{enumerate}
\item $A \equiv s$: By Definition~\ref{defn:type rules},
$\Gamma(s)$ is defined and $\itkind(\Gamma(s))$ is $\tttype$\@.
Then by Definition~\ref{defn:meaning function}, $\Mgamma{s}$
is $\gamma(s)$, and by Definition~\ref{defn:satisfaction},
$\gamma(s)\in U$\@.

\item $A\equiv \funtype{B}{C}$: We then have that $\tauGamma{B} =
\tauGamma{C} = \tttype$\@.  Letting $X$ label $\Mgamma{B}$, and $Y$ label
$\Mgamma{C}$, we have by the induction hypothesis that $X\in U$ and $Y\in
U$\@.  Let $j$ be the least rank such that $\Mgamma{B}\in U_j$ and
$\Mgamma{C}\in U_j$\@.  By Definition~\ref{defn:meaning function},
$\Mgamma{A} = Y^X$, and hence $\Mgamma{A} \in U_{j+1}$ by
Definition~\ref{defn:type universe}\@.

\item $A \equiv \tupletype{\listwo{A}{n}}$: Again by
Definition~\ref{defn:type rules} and the induction hypothesis, we have
for each $i \in \{1, 2\}$, that $\Mgamma{A_i}\in U$\@.  Let $j$ be the least
rank such that for $i\in \{1, 2\}$, $\Mgamma{A_i}\in U_j$\@.  Then,
it is easy to verify from Definition~\ref{defn:type universe} that
$\Mgamma{A}\in U_{j + 1}$\@.  
\end{enumerate}
\end{proof}

\begin{theorem}[term soundness]\label{simple-term-semantics}
 If $\tau()(\Gamma) = \ttcontext$, 
   $\gamma$ satisfies $\Gamma$, and
   $\tauGamma{a}$ is defined and equal to $A$, then 
$\Mgamma{a} \in \Mgamma{A}$.  
\end{theorem}
\begin{proof}
The proof is by induction on the structure of preterms.
\begin{enumerate}
\item $a\equiv s$: By Definition~\ref{defn:type rules},
we have that $\ittype(\Gamma(s)) = A$\@.  By
Definitions~\ref{defn:meaning function}
and~\ref{defn:satisfaction}, we have that $\Mgamma{a} = \gamma(s)$
and $\gamma(s) \in
\Mgamma{A}$\@.
\item $a \equiv (f\ b)$:  By Definition~\ref{defn:type rules},
$\tauGamma{f} = \funtype{B}{A}$, and $\tauGamma{b} = B$, for some
$B$ such that $\tauGamma{B} = \tttype$\@.  
Let $\Mgamma{A}$ be $X$ and $\Mgamma{B}$ be $Y$, then 
by Definitions~\ref{defn:type rules}
and~\ref{defn:meaning function}, and the induction hypothesis, we have
$\Mgamma{f} \in X^Y$ and $\Mgamma{b}\in Y$\@.  It therefore follows
by Definition~\ref{defn:meaning function} that $\Mgamma{(f\ b)} =
(\Mgamma{f})(\Mgamma{b})$, and hence $\Mgamma{(f\ b)} \in X$\@.

\item $a \equiv (\lambda (x : C): b)$:  By
Definition~\ref{defn:type rules}, we have that
$\tauGamma{a}$ is $\funtype{C}{B}$, where
$\tau(\Gamma, x : \ttvar~C)(b)$ is $B$\@.  
Let $X$ be $\Mgamma{C}$, and $Y$ be
$\mathcal{M}(\Gamma, x: \ttvar~C \vbar \gamma\{x\gets u\}))(B)$\@.  
By the induction hypothesis, we have that for
any $u\in Y$, 
$\mathcal{M}(\Gamma, x: \ttvar~C \vbar \gamma\{x\gets u\})(b)
\in X$\@.  
Since $\Mgamma{a}$ is $\{\pair{u, v} \vbar u\in X, v = \mathcal{M}(\Gamma, x:
\ttvar~C \vbar \gamma\{x\gets u\})(b)\}$,
we have that $\Mgamma{a} \in X^Y$\@.

\item $a \equiv (a_1, a_2)$: By Definition~\ref{defn:type rules},
$\tauGamma{a} = \tupletype{\listwo{A}{n}}$, where $\tauGamma{a_i} = A_i$ for
$i\in \{1, 2\}$\@.   By the induction hypothesis, $\Mgamma{a_i}\in
\Mgamma{A_i}$ for $i\in \{1, 2\}$\@.  By
Definition~\ref{defn:meaning function}, $\Mgamma{a} =
\pair{\Mgamma{a_1},\Mgamma{a_2}}$ and hence $\Mgamma{a}$ is
an element of $\Mgamma{A}$ which is
$\Mgamma{A}\times\Mgamma{A_n}$\@.

\item $a\equiv \proj{i}~b$: In this case, we know by
Definition~\ref{defn:type rules} that $\tauGamma{b} =
\tupletype{\listwo{A}{n}}$
with $i\in \{1, 2\}$, and $\tauGamma{a} = A_i$\@.
By the induction hypothesis, $\Mgamma{b} = \pair{t_1, t_2}$,
and by Definition~\ref{defn:meaning function}, $\Mgamma{a} = t_i$
and $\Mgamma{\tauGamma{b}} =
\Mgamma{A_1}\times\Mgamma{A_2}$,
hence $\Mgamma{a}\in \Mgamma{A_i}$\@.  
\end{enumerate}
%%\memo{What about tuples/projections?}
\end{proof}

These three theorems (\ref{type construction},
\ref{simple-type-semantics}, and \ref{simple-term-semantics}) are the key
invariants that must be satisfied by the semantics when the language is
extended below with type definitions, subtypes, dependent types, and
parametric theories.

\section{Some Syntactic Operations}

We first define the operation of collecting the free variables of a
term $a$ in a given context $\Gamma$ as $FV(\Gamma)(a)$, and then
define the operation of substitution.
\begin{Defn}{free variables}
\begin{eqnarray*}
  FV(\Gamma)(s) & = & \left\{\begin{array}{l}
			      \{s \}, \mbox{ if } \textit{kind}(\Gamma(s))
= \ttvariable\\
                              \emptyset, \mbox{ otherwise}
			    \end{array}\right.\\
%
  FV(\Gamma)(f\ a) & = & FV(\Gamma)(f) \union FV(\Gamma)(a) \\
%
  FV(\Gamma)(\lambda (x : T) : a) & = &
     FV(\Gamma, x : \ttvar{}\ T)(a)
- \{x\} \\
%
 FV(\Gamma)((\listwo{a}{n})) & = & FV(\Gamma)(a_1)\union
FV(\Gamma)(a_2)\\
%
 FV(\Gamma)( \proj{i}\ a) & = & FV(\Gamma)(a)
\end{eqnarray*}
\end{Defn}
%
\begin{Defn}{substitution}
\begin{eqnarray*}
  s[a_1/x_1,\ldots, a_n/x_n] & = & \left\{\begin{array}{l}
					    a_i, \mbox{ if for some minimal } i,
s \equiv x_i\\
                                            s, \mbox{ otherwise}
					  \end{array}\right.\\
%
  (f\ a)[a_1/x_1,\ldots, a_n/x_n] & = &
     (f[a_1/x_1,\ldots, a_n/x_n]\\&&
       \hspace*{.5in} a[a_1/x_1,\ldots, a_n/x_n]) \\
%
  (\lambda (y : T): a)[a_1/x_1,\ldots, a_n/x_n] & = &
    (\lambda (y':T) :
     \ a[y'/y, a_1/x_1,\ldots, a_n/x_n]), \\ & & 
%
\mbox{where  } y' \mbox{ is a fresh variable}\\
%
 (\listwo{b}{m})[a_1/x_1,\ldots, a_n/x_n] & = &
     (b_1[a_1/x_1,\ldots, a_n/x_n],\\&&
       \hspace*{.5in}b_2[a_1/x_1,\ldots, a_n/x_n])\\
%
  ( \proj{i}\ a)[a_1/x_1,\ldots, a_n/x_n] & = &
   ( \proj{i}\ a[a_1/x_1,\ldots, a_n/x_n])
\end{eqnarray*}
\end{Defn}

Recall that terms are treated as syntactically equivalent modulo
alpha conversion.  
The above definitions must be extended as more 
features are added to the language. 

\section{Type Definitions}

%%\memo{See ADT section for recursive datatypes.}

Here we enrich contexts so that type symbols may have definitions.  PVS
does not allow recursive type definitions\footnote{For the moment, we are
not considering the PVS \texttt{DATATYPE} mechanism, which is a form of
recursive type definition~\cite{Shankar:ADT97}.  Recursive datatypes in
the context of the HOL proof checking system are described by
Melham~\cite{Melham89}.}  so a type declaration/definition in a context
may use only the symbols declared in the prior part of the context.  The
main difference in the extended language is that type names can have
definitions.  In such cases, the definitions rather than the type names
are used to determine the actual type of an expression.  In other words,
two type expressions are treated as the same if they are
\emph{definitionally equivalent}\@.  Most other specification languages
tend to employ the weaker notion of \emph{name equivalence} where
syntactically different types are treated as distinct even when their
definitions coincide.

To accommodate type definitions, a context can
contain type declarations of the form $s\ :\ \mathtt{TYPE} = T$, where $T$
is a type.  If context $\Gamma$ contains such a declaration for $s$, then
$ \itdef{}(\Gamma(s))$ is $T$\@.  To extend $\tau$ to handle type
definitions under definitional equivalence, we must ensure that $\tau$
returns the canonical form of a type where all defined types have been
replaced by their definitions.  The operation $\delta(\Gamma)(T)$ returns
the expanded form of a type relative to the context $\Gamma$\@.
%
\begin{Defn}{expanded type}
\begin{eqnarray*}
  \delta(\Gamma)(s) & = & s, \mbox{ if } \textit{definition\/}(\Gamma(s))
\mbox{ is empty}\\
% 
  \delta(\Gamma)(s) & = & \delta(\Gamma)(
\textit{definition\/}(\Gamma(s))), 
\mbox{ if } \textit{definition\/}(\Gamma(s))
\mbox{ is nonempty} \\
%
 \delta(\Gamma)(\funtype{A}{B}) & = &
      \funtype{\delta(\Gamma)(A)}{\delta(\Gamma)(B)} \\
%
 \delta(\Gamma)(\tupletype{\listwo{T}{n}}) & = &
     \tupletype{\delta(\Gamma)(T_1),  \delta(\Gamma)(T_2)}
\end{eqnarray*}
\end{Defn}


The typing rules are augmented to return the type in expanded form.
The main issue here is to determine that the definition part of a type
declaration in a context is well formed relative to the preceding context.
We also need to ensure that $\tau$ returns the expanded form of the type
corresponding to a preterm.
%
\begin{Defn}{type rules with type definitions}
\begin{eqnarray*}
    \tau()(\Gamma, s\ :\ \tttype{} = T) & = & \ttcontext, \mbox{
if }  \Gamma(s) \mbox{ is undefined}, \\ & &
\tau()(\Gamma) = \ttcontext,\\ & & 
  \mbox{and }     \tauGamma{T} = \tttype{} \\
%
%   \tau()(\Gamma, c : T = a) & = & \ttcontext, \mbox{ if } \tauGamma{T} =
%      \tttype{} \mbox{ and } \tauGamma{a} = \delta(\Gamma)(T)\\
%
    \tau(\Gamma)(s) & = &  \delta(\Gamma)(\textit{type\/}(\Gamma(s))),\\
                    & &  \mbox{ if } 
\textit{kind}(\Gamma(s)) \in \{ \ttconstant{}, \ttvariable \} 
\end{eqnarray*}
\end{Defn}


Note that the $\delta$ operator is idempotent, and $\tauGamma{a}$
for a term $a$ always returns an expanded type, that is, 
$\delta(\tauGamma{a}) = \tauGamma{a}$\@.  

We do not need to update the definition of ${\mathcal M}$ from
Definition~\ref{defn:meaning function} since the syntax for terms is
unchanged, 
but we do need
to revise the notion of a satisfying context assignment (from 
Definition~\ref{defn:satisfaction}) to respect the type
definitions.
\begin{Defn}{satisfaction with type definitions}
An assignment $\gamma$ satisfies a context $\Gamma$ if
in addition to the conditions in Definition~\ref{defn:satisfaction},
whenever $ \textit{kind}(\Gamma(s)) = \tttype{}$ and
$\textit{definition}(\Gamma(s))$ (abbreviated as $T$) is nonempty, then
$\gamma(s) = \Mgamma{ T}$.
\end{Defn}

Theorems~\ref{type construction} and~\ref{simple-type-semantics}
and~\ref{simple-term-semantics} continue to hold under these extensions,
and the proofs are easily adapted to the modified definitions.  

\begin{Eg}{type definition}
Let $\Omega'$ be the context\\ $\Omega,
\mathtt{boolop}:\mathtt{TYPE}=\funtype{\tupletype{\mathtt{bool}, \mathtt{bool}}}{\ttbool},
\vee:\mathtt{boolop}$\@.  Then
\begin{eqnarray*}
\tau()(\Omega') & = & \mathtt{CONTEXT}\\
%
 \delta(\Omega')( \mathtt{boolop})  & = & \funtype{\tupletype{ \mathtt{bool},
\mathtt{bool}}}{\ttbool},\\
 \tau(\Omega')( \vee) & = & \funtype{\tupletype{ \mathtt{bool}, \mathtt{bool}}}{\ttbool}
\end{eqnarray*}
\end{Eg}

\section{Summary}

We have defined the simply typed fragment of PVS by introducing the syntax
for pretypes and preterms, the type rules and semantics for well-formed
contexts, types, and terms.  The type rules are presented in a novel
functional style where each well-formed context is assigned the label {\tt
CONTEXT}, each well-formed type is assigned the label {\tt TYPE}, and each
well-formed term is assigned a canonical type.  The semantics takes a
satisfying assignment for a context and maps a well-formed type to a set
and a well-formed term to an element of the set corresponding to its
canonical type.  We then defined the syntactic operations of collecting
the free variables in an expression and for substituting terms for
variables in an expression.

The simple type theory is then extended with type definitions.  With this
extension, two type expressions are treated as equivalent if they are
identical after all type definitions have been expanded.  The operation
$\delta$ returns the expanded form of a given type expression.


\chapter{Adding Subtypes}\label{subtypes}

Subtyping is one of the main features of the PVS specification
language.\footnote{The form of subtyping used in PVS is derived from a
suggestion of Friedrich von~Henke.}  Subtyping in PVS corresponds to the
set-theoretic notion of a subset.  It raises several delicate issues that
were absent in the language presented thus far.  In the simply typed
fragment, each type corresponds to a set of values that is somehow
structurally different from the set of values for another type so that a
term has at most one type.  Subtyping makes it possible to introduce the
natural numbers as a subtype of the reals, and to treat the primes, the
even numbers, and the odd numbers as subtypes of the natural numbers.
With subtyping, a term can obviously have several possible types, but the
type-checking function $\tau$ may return only a single type.  We constrain
$\tau$ to return a natural canonical type of an expression that is given
by the declarations of the symbols in the expression.  If the expression
is used in a context where the expected type is a supertype of its
canonical type, then the type correctness is straightforward.  If the
expected type is a subtype that is {\em compatible\/} with the canonical
type of the expression, then typechecking generates proof obligations
asserting that the expression satisfies the predicate constraints imposed
by the expected type.  Two types are compatible if they have equivalent
maximal supertypes.  Type equivalence in the presence of subtypes is not a
simple notion.  Subtyping also introduces the possibility of types being
empty.  Typed lambda calculi with possibly empty types have been studied
by Meyer, Mitchell, Moggi, and Statman~\cite{MMMS:empty}.  This chapter
introduces predicate subtypes and defines the notions of compatibility and
type equivalence prior to presenting the type rules and semantics.


We restrict our attention to contexts $\Gamma$ that extend
the declarations:
\begin{alltt}
 bool : TYPE,
 TRUE : bool,
 FALSE : bool,
 boolop : \(\funtype{\tupletype{\ttbool, \ttbool}}{\ttbool}\),
 \(\neg\) : \(\funtype{\ttbool}{\ttbool}\),
 \(\vee\) : boolop,
 \(\wedge\) : boolop,
 \(\supset\) : boolop
\end{alltt}
We will abuse PVS notation to employ the customary infix forms of
operations like $\vee$, 
$\wedge$, and $\supset$.    The pretype corresponding to a predicate
subtype has the form $\{x : T \vbar a\}$ where $x$ is a symbol, $T$ is a
pretype, and $a$ is a preterm.   
A \emph{predicate type} in PVS is a function type where the range is the
primitive type \texttt{bool}.  A predicate is a term that has a predicate
type.  If $a$ is a term of type \texttt{bool}, then 
we can define the subtype $\{ x : T \vbar a\}$ consisting of those
elements $e$ of $T$ satisfying $a[e/x]$ ($e$ substituted for $x$ in $a$).
Since the elements of the subtype $\{ x : T \vbar a\}$ satisfy the predicate $
\lambda (x  : T): a$, 
we call this type a {\em predicate subtype\/} to distinguish it from other
forms of subtyping.  
Universal quantification $\forall (x : T): a$ is just an abbreviation
for the term $(\lambda (x : T): a) = (\lambda (x : T): \mathtt{TRUE})$\@.  
Although we use the equality  predicate in the definition of universal
quantification and in the definitions below, the actual introduction of
equality is deferred to a later section following the introduction of
parametric theories.  The equality between PVS terms of function type
is to be interpreted as extensional equality.  Note that the `$=$' symbol is
used both for the formal equality symbol in the language and for
metatheoretic equality.   
\begin{comment}
We will also use the notation $\lambda (x_1 : T_1, \ldots, x_n : T_n): a$
in place of $\lambda (x : \tupletype{\seq{T}{n}}): a[
\proj{1}(x)/x_1, \ldots, \mathtt{PROJ\_n}(x)/x_n]$\@.
\end{comment}



Our first step will be to define the notion of a \emph{maximal supertype}
of a given type as $\mu(T)$.  A \emph{maximal type} $T$ is one such that
$\mu(T) = T$.  In a given context, we will apply  $\mu$ only to the
expanded form (given by $\delta$) of a type expression.  
\begin{Defn}{maximal supertype}
\begin{eqnarray*}
  \mu(s) & = & s \\
   \mu(\{x : T \vbar a\}) & = & \mu(T) \\
%
   \mu(\funtype{A}{B}) & = & \funtype{A}{\mu(B)} \\
  \mu(\tupletype{\listwo{A}{n}}) & = & \tupletype{\mu(A_1), \mu(A_2)}
\end{eqnarray*}
\end{Defn}
Note that since subtypes correspond to subsets, in taking the maximal
supertype of a function type, the domain type is held fixed.  In most type
theories with subtypes, the rule for subtyping between function types
$\funtype{A}{B}$ and $\funtype{A'}{B'}$ requires showing that $A'$ is a
subtype of $A$, and $B$ is a subtype of $B'$\@.  Subtyping between
function types is therefore said to be contravariant in the domain type
and covariant in the range type.  Subtyping on function types in PVS is
covariant in the range type but is neither covariant nor contravariant in
the domain type.  This means that the function type
$\funtype{\ttnat}{\ttnat}$ is not a supertype of the function type
$\funtype{\ttint}{\ttnat}$\@.  Such a subtyping relation would violate
{\em extensionality\/}\@.  Two functions on $\ttnat$ are extensionally
equal when they return equal values when applied to equal arguments in
$\ttnat$\@.  Consider two functions in $\funtype{\ttnat}{\ttnat}$:
$\mathit{abs}$ which returns the absolute value, and $\mathit{idnat}$
which behaves as an identity function on natural numbers and returns $0$
otherwise.  These two functions will be erroneously identified if they can
be viewed as being of type $\funtype{\ttnat}{\ttnat}$, and the subset
interpretation of subtypes would be lost.

We will also employ a weaker supertype $\mu_0(T)$ or the
{\em direct supertype\/}, that only considers supertypes of
explicitly given subtypes of the form $\{x : T \vbar a\}$.
\begin{Defn}{direct supertype}
\begin{eqnarray*}
\mu_0(\{x : T \vbar a\}) & = & \mu_0(T)\\
\mu_0(T) & = & T, \mbox{ otherwise}
\end{eqnarray*}
\end{Defn}

\begin{Eg}{maximal supertype}
Given a context containing the declarations
$$\begin{array}{l}
  \ttint : \tttype, \\
  \mathtt{0} : \ttint,\\
  \leq : \funtype{\tupletype{\ttint, \ttint}}{\ttbool},\\
  \ttnat : \tttype = \{ i : \ttint \vbar  0 \leq i\}\\
  \mathtt{natinjection} : \tttype = \{ f : \funtype{\ttnat}{\ttnat}\ \vbar\ 
\forall (i, j: \ttnat): f(i) = f(j) \supset i = j \}
\end{array}$$
we have
\begin{eqnarray*}
  \mu(\mathtt{natinjection}) & = & \mu(\funtype{\ttnat}{\ttnat})\\
 &=  & \funtype{\ttnat}{\mu(\ttnat)}\\
 &=  & \funtype{\ttnat}{\ttint}\\
\mu_0(\mathtt{natinjection}) & = & \funtype{\ttnat}{\ttnat}
\end{eqnarray*}
\end{Eg}

Note that   $\mu(\mu(A)) = \mu(A)$.
Note also that a maximal supertype is never a subtype.  
We can in fact collect the predicates that constrain a type $A$
relative to its maximal supertype $\mu(A)$ as $\pi(A)$\@.

\begin{Defn}{subtype constraints}
\begin{eqnarray*}
  \pi(s) & = & \lambda (x : s): \tttrue \\
   \pi(\{y : T \vbar a\}) & = &   \lambda (x : \mu(T)): (\pi(T)(x) \wedge a[x/y])\\
%
   \pi(\funtype{A}{B}) & = &   \lambda (x : \funtype{A}{\mu(B)}):
    (\forall (y: A): \pi(B)(x(y))) \\
%
\pi(\tupletype{\listwo{A}{n}}) & = &
    \lambda (x: \tupletype{\mu(A_1),  \mu(A_2)}): 
	(\pi(A_1)(\proj{1}~x)\wedge \pi(A_2)(\proj{2}~x))
\end{eqnarray*}
\end{Defn}
Observe that in Definition~\ref{defn:subtype constraints}, if $\tauGamma{A} =
\tttype$, then 
$\tauGamma{\pi(A)} = \funtype{\mu(A)}{
 \mathtt{bool}}$\@.\footnote{This is somewhat tricky in the case of
$\pi(\{y:T\vbar a\})$ since in $a[x/y]$, $x$ has type $\mu(T)$, whereas
$y$ has type $T$\@.  As shown in Chapter~\ref{conditionals}, the
type rules for conjunction are such that $\tau(\Gamma, x :
\ttvar~\mu(T))(\pi(T)(x) \wedge a)$ reduces to
$\tau(\Gamma, x : \ttvar~\mu(T))(\pi(T)(x))$ and
$\tau(\Gamma, x : \ttvar~\mu(T), \pi(T)(x))(a[x/y])$ where the first
conjunct is added to the context as a contextual assumption.
One can then show by induction that $\tau(\Gamma, x:\ttvar~\mu(T),
\pi(T)(x))(a[x/y]) = \tau(\Gamma, y: \ttvar~T)(a)$\@.  }

\begin{Eg}{subtype constraints}
\begin{eqnarray*}
\lefteqn{\pi(\ttnat)}\\
  & = & \lambda (j: \ttint): 0\leq j\\
\lefteqn{\pi(\funtype{\ttnat}{\ttnat})}\\
& = & \lambda (g : \funtype{\ttnat}{\ttint}): \forall (i:\ttnat): (\lambda
(j:\ttint): 0 \leq j)(g(i))\\
\lefteqn{\pi(\mathtt{natinjection})}\\
&= & \lambda (f : \funtype{\ttnat}{\ttint}) : 
\begin{array}[t]{cl}
& \pi(\funtype{\ttnat}{\ttnat})(f)\\
\wedge & (\forall (i, j: \ttnat): f(i) = f(j) \supset  i = j)
\end{array}\\
& = & 
\lambda (f : \funtype{\ttnat}{\ttint}) :\\&&\hspace*{.5in}
\begin{array}[t]{cl}
& (\lambda (g : \funtype{\ttnat}{\ttint}) : \forall (i: \ttnat):
(\lambda (j : \ttint): \mathtt{0}\leq j)(g(i)))(f) \\
\wedge & (\forall (i, j: \ttnat): f(i) = f(j) \supset  i = j)
\end{array}
\end{eqnarray*}
\end{Eg}
Observe that $\pi(\mu(A))$ is essentially equivalent to $\lambda (x : \mu(A)):
\mathtt{TRUE}$. 

Since the subtype $\{x : T \vbar p(x) \wedge q(x)\}$ can also be written
as $\{x : T \vbar q(x) \wedge p(x)\}$, we need a notion of equivalence
between types.  One way to do this is to make types ``first-class'' and to
allow explicit theorems to be proved about type equivalence and subtyping.
Since this would be a fairly drastic extension to the specification
language, we have designed the PVS type system so as to avoid any
first-class treatment of types.  It turns out that all the needed
properties about types (such as equality and subtyping) can be obtained by
generating ordinary proof obligations rather than by explicitly proving
theorems about types.

We introduce 
below a metatheoretic operation that generates the proof obligations
needed to establish that two (maximal) types are equivalent.  This
equivalence is denoted by $\simeq$ and is  applied only to maximal types
and returns a list of the 
proof obligations that must be proved.  Note the invariant in the
definition below that the arguments to $\simeq$ are always maximal.  The
definition of $\simeq$ makes use of the PVS equality predicate that will be
introduced later.  A list of formulas is represented as $a_1,\ldots, a_n$.
Given two such lists $\lis{a}{m}$ and $\lis{b}{n}$, the concatenation of
these two lists is written as $\lis{a}{m}\ ; \ \lis{b}{n}$\@.

\begin{Defn}{type equivalence proof obligations}
\begin{eqnarray*}
  (s \simeq s) & = & \tttrue \\
%
  (\funtype{A}{B} \simeq \funtype{A'}{B'}) & = & ((\mu(A)\simeq \mu(A'));
(\pi(A) = \pi(A'));
 (B \simeq B'))\footnotemark \\
%
  (\tupletype{\listwo{A}{n}} \simeq \tupletype{\listwo{B}{n}}) & = &
    ((A_1\simeq B_1); (A_2\simeq B_2)) \\
%
 (A \simeq B) & = & \ttfalse, \mbox{ otherwise}
\end{eqnarray*}
\end{Defn}
\footnotetext{The type correctness of the proof obligation $(\pi(A) =
\pi(A'))$ depends on the \emph{prior} proof obligations $\mu(A)\simeq
\mu(A')$.}

\begin{Eg}{type equivalence}  Building on the context given in
Example~\ref{eg:maximal supertype}, 
if we have the following variants of $\ttnat{}$ and
\texttt{natinjection}:
$$
\begin{array}{l}
\mathtt{NAT} :  \tttype = \{ i : \ttint \vbar  i \leq \mathtt{0}\ 
\supset\ i = 0\}\\
\mathtt{NATinjection} : \tttype = \{ f : \funtype{ \mathtt{NAT}}{
\mathtt{NAT}} \vbar 
\forall (i, j: \mathtt{NAT}): f(i) = f(j)\ \supset\ i = j \}
\end{array}$$
we get
\begin{eqnarray*}
\mu(\funtype{ \mathtt{natinjection}}{ \mathtt{natinjection}}) =
 \funtype{ \mathtt{natinjection}}{\funtype{\ttnat}{\ttint}}\\
%
\mu(\funtype{ \mathtt{NATinjection}}{ \mathtt{NATinjection}}) =
 \funtype{ \mathtt{NATinjection}}{\funtype{ \mathtt{NAT}}{ \ttint}}
\end{eqnarray*}
\begin{eqnarray*}
\lefteqn{\mu(\funtype{ \mathtt{natinjection}}{ \mathtt{natinjection}})
\simeq \mu(\funtype{ \mathtt{NATinjection}}{ \mathtt{NATinjection}})}\\
& = &  \begin{array}[t]{l}
       (\mu( \mathtt{natinjection}) \simeq \mu( \mathtt{NATinjection}));\\
       (\pi( \mathtt{natinjection}) = \pi( \mathtt{NATinjection}));\\
       (\funtype{\ttnat}{\ttint} \simeq \funtype{ \mathtt{NAT}}{ \mathtt{int}})     
       \end{array}\\
 \lefteqn{(\pi( \mathtt{natinjection}) = \pi( \mathtt{NATinjection}))}\\
& = & \begin{array}[t]{cl}
(&      \lambda (f : \funtype{\ttnat}{\ttint}) :
\begin{array}[t]{cl}
& (\lambda (g : \funtype{\ttnat}{\ttint}) : \forall (i: \ttnat):
\mathtt{0}\leq  g(i))(f) \\
\wedge & (\forall (i, j: \ttnat): f(i) = f(j) \ \supset\  i = j)
 \end{array}\\
 =  \\&  \lambda (f : \funtype{ \mathtt{NAT}}{\ttint}) :\\&\hspace*{.3in}
\begin{array}[t]{cl}
& (\lambda (g : \funtype{ \mathtt{NAT}}{\ttint}) : \forall (i: \mathtt{NAT}):
g(i) \leq \mathtt{0}\ \supset\ g(i) = 0)(f) \\
\wedge & (\forall (i, j: \mathtt{NAT}): f(i) = f(j) \ \supset\  i = j)
      \end{array} )
 \end{array}\\
\lefteqn{(\funtype{\ttnat}{\ttint} \simeq \funtype{ \mathtt{NAT}}{
\mathtt{int}})}\\
& = & (\ttint \simeq \ttint);\\&& (\lambda (i : \ttint): \mathtt{0} \leq i) =
                              (\lambda (i : \ttint): 
i\leq \mathtt{0}\  \supset\ i = 0); (\ttint\simeq\ttint)
\end{eqnarray*}
\end{Eg}

A basic question during typechecking is whether two types are
\emph{compatible}, that is,  have the same maximal supertype.  
Two types are said to be compatible if the type equivalence proof
obligations on their respective 
maximal supertypes are provable.  The provability of a formula $a$
under context $\Gamma$ is represented as $\vdash_\Gamma a$\@.  
\begin{Defn}{compatible}
\emph{Two types $A$ and $B$ are said to be compatible in context $\Gamma$
(in notation, $(A \sim B)_\Gamma$)
if $\vdash_\Gamma a$, for each $a$ in $(\mu(A)\simeq \mu(B))$.}\footnotemark
\end{Defn}
\footnotetext{ The PVS proof rules are described in 
Chapter~\ref{proof theory}.}


\begin{comment}
With subtypes, it is easy to define an empty subtype and then
declare a constant of such an empty subtype.  The context thus has to be
checked to disallow such an inconsistency.
\end{comment}
We now extend the
definition of $\delta$ to the case of subtypes so that it leaves the
predicate unchanged but expands the definition of the supertype.
%
\begin{Defn}{expanded type with subtypes}
\begin{eqnarray*}
  \delta(\Gamma)(\{x : T \vbar a\}) & = & \{x : \delta(\Gamma)(T) \vbar a \}
\end{eqnarray*}
\end{Defn}

We now extend the definition $\tau$ to the case of subtypes.  Here we
could force $\tau$ to always return a maximal supertype but this is not
done in Definition~\ref{defn:type rules with subtypes}  since it
would weaken the soundness theorem without significantly simplifying the
definition of the type rules.  The typechecking of contexts has to be
modified to generate a {\em nonemptiness\/} proof obligation for the type of
any constant declaration.  A constant of an empty type would lead to an
inconsistent context, and this would mean that constant declarations are
not conservative extensions.  This modification to
Definition~\ref{defn:type rules} is not needed for soundness since an
inconsistent context makes soundness trivial.  It is needed to show that
constant declarations and definitions are conservative extensions.  Note
that with subtypes, the type rule for an application is modified to check
that the domain type of the function is compatible with the type of its
argument, and that the argument satisfies any constraints imposed by the
domain type of the function.  The case of projection expressions is also
not straightforward since the argument type can be a subtype of a tuple
type.  In this case, we use the direct supertype (see
Definition~\ref{defn:direct supertype}) which must be a tuple type.
%
\begin{Defn}{type rules with subtypes}
\begin{eqnarray*}
  \tau()(\Gamma, c : T) & = & \ttcontext, \mbox{ if } \Gamma(c)
\mbox{ is undefined},\\& &
\tauGamma{T} =   \tttype{},\\& &
      \tau()(\Gamma) = \ttcontext,  \mbox{ and } \\& & 
   \vdash_\Gamma (\exists (x: T) : \tttrue)\\
%
  \tauGamma{\{x : T \vbar a\}} & = & \tttype{}, \mbox{ if } \Gamma(x)
\mbox{ is undefined}, \\& & 
    \tauGamma{T} = \tttype{}, \mbox{ and } \tau(\Gamma, x : \ttvar{}\ T)(a) = \ttbool \\
%
  \tauGamma{f\ a} & = &  B, \mbox{ where } \mu_0(\tauGamma{f}) =
                                                 \funtype{A}{B},\\&  & 
                   			\tauGamma{a} = A',\\ & &
					(A \sim A')_\Gamma,\\& & 
                                        \vdash_\Gamma \pi(A)(a) \\
 \tauGamma{\proj{i}~a} & = & A_i, \mbox{ where } \mu_0(\tauGamma{a}) =
\tupletype{A_1, A_2}
\end{eqnarray*}
\end{Defn}

\begin{Eg}{typechecking subtypes}
Let $\Gamma$ contain the above declarations of $\ttint$, $\ttnat$,
$\mathtt{0}$, 
$\leq$, and {\tt
natinjection}.
\begin{eqnarray*}
\lefteqn{\tauGamma{\{i : \ttint \vbar 0\leq i\}}}\\
  & = & \tttype\\
\lefteqn{\tauGamma{(\lambda (f : \mathtt{ natinjection}) : f(0))(\lambda
(i: \mathtt{nat}): i)}}\\
& = &
\delta(\Gamma)(\ttnat), \mbox{ if}\\& &
( \mathtt{natinjection} \sim \funtype{\ttnat}{\ttnat})_\Gamma,\\& & 
\vdash_\Gamma \forall (j, k : \ttnat): (\lambda (i: \ttnat): i)(j) = (\lambda
(i: \ttnat) : i)(k) \ \supset\  j = k, \\& &
(\ttint \sim \ttnat)_\Gamma, \mbox{ and}\\& & 
\vdash_\Gamma \mathtt{0}\leq \mathtt{0}
\end{eqnarray*}

\end{Eg}

Only one additional clause to Definition~\ref{defn:meaning function} is
needed to capture the 
semantics of predicate subtypes.
\begin{Defn}{meaning function with subtypes}
\begin{eqnarray*}
  \lefteqn{\Mgamma{\{x : T \vbar a\}}}\\
& = & \{ y \in \Mgamma{T} \vbar
\mathcal{M}(\Gamma, x: \ttvar~T\vbar \gamma\{x
\gets y\})(a) = {\oneb} \} 
\end{eqnarray*}
\end{Defn}

\begin{Eg}{semantics of predicate subtypes}
If we assign the usual truth table interpretation to the Boolean function
$\supset$: 
\begin{eqnarray*}
\lefteqn{\Mgamma{\{f : \funtype{\ttbool}{\ttbool} \vbar \forall (x: \ttbool): x\ \supset\ f(x)\}}}\\
& = & \{\{\pair{\zerob, \zerob}, \pair{\oneb, \oneb}\},
\{\pair{\zerob,\oneb}, \pair{\oneb, \oneb}\}\}.
\end{eqnarray*}
\end{Eg}

The following useful propositions are easily proved from the definitions
given above.   Proposition~\ref{max-type-correct} asserts that the
maximal supertype of a type is well typed.  Proposition~\ref{maximal-subset}
asserts that the denotation of a type is a subset of the denotation of its
maximal supertype.  Proposition~\ref{maximal-equality} asserts that
if all the proof obligations in $(A\simeq A')$ are valid relative to
a given assignment $\gamma$ for context $\Gamma$, then
the denotations of $A$ and $A'$ under $\gamma$ are equal. 
%
\begin{prop}\label{max-type-correct}
  If $\tau()(\Gamma) = \ttcontext$ and
  $\tauGamma{A} = \tttype{}$, then $\tauGamma{\mu(A)} = \tttype{}$.
\end{prop}
\begin{prop}\label{maximal-subset}
  If $\tau()(\Gamma) = \ttcontext$, 
    $\tauGamma{A} = \tttype{}$,
   and $\gamma$ satisfies $\Gamma$, 
then
\begin{enumerate}
\item   $\Mgamma{A} \subseteq \Mgamma{\mu(A)}$ and
\item   $\Mgamma{A} \subseteq \Mgamma{\mu_0(A)}$.
\end{enumerate}
\end{prop}
\begin{prop}\label{maximal-equality}
  If $A$ and $A'$ are maximal
types in context $\Gamma$, i.e.,
\begin{enumerate}
\item $\tau()(\Gamma) = \ttcontext$, 
\item $\tauGamma{A} = \tauGamma{A'} = \tttype{}$,
\item $\mu(A) = A$ and $\mu(A') = A'$
\end{enumerate}
and
 for each $a$  in $(A \simeq A')$, %%\memo{What about true/false?}
\begin{enumerate}
\item $a\equiv \tttrue$, or 
\item $a\equiv (a_1 = a_2)$ and $\Mgamma{a_1} = \Mgamma{a_2}$
holds,
\end{enumerate}
then $\Mgamma{A} = \Mgamma{A'}$.\footnote{ 
We remind the reader that the formulas $a$ in $(A\simeq A')$ are equalities,
but we have not yet formally introduced equality into the language.   } 
\end{prop}
\begin{prop}\label{maximal-subtype}
If $\tau()(\Gamma) = \ttcontext$ and $\tauGamma{T} = \tttype$,
then $\Mgamma{T} = \Mgamma{\{x : \mu(T) \vbar \pi(T)(x)\}}$.
\end{prop}

We can now examine the updated forms of the invariants given by
Theorems~\ref{type construction},
\ref{simple-type-semantics},
and~\ref{simple-term-semantics}\@.  
The proof of Theorem~\ref{type construction} remains straightforward.
\begin{comment}
Theorems~\ref{simple-type-semantics}
and~\ref{simple-term-semantics} must be proved with a simultaneous
induction since types now contain terms.
\end{comment}
The statement of
Theorem~\ref{simple-term-semantics} must now
be strengthened to include soundness, that is, if $\vdash_\Gamma a$
and $\gamma$ satisfies $\Gamma$, then 
$\Mgamma{a} = \oneb$\@.  For now, we assume soundness
(Theorem~\ref{proof-soundness})  since we
have not yet presented the proof rules.  
%
\begin{theorem}[type soundness]
\label{subtype-type-semantics}
If $\tau()(\Gamma) = \ttcontext$, $\gamma$ satisfies $\Gamma$, and
$\tauGamma{A} = \tttype$ then $\Mgamma{A}\in U$.
\end{theorem}
\begin{proof}
There is only one new case to add to the induction proof of
Theorem~\ref{simple-type-semantics}, namely, when 
 $A\equiv \{x : T \vbar a\}$.   In this case, by
Definition~\ref{defn:type rules with subtypes},
$\tauGamma{T} = \tttype$, so by the induction hypothesis,
$\Mgamma{T} \in U$.  Since, by
Definition~\ref{defn:meaning function with subtypes}, $\Mgamma{A} \subseteq
\Mgamma{T}$, we have $\Mgamma{A}\in U$ by
Definition~\ref{defn:type universe}\@.
\end{proof}
\begin{theorem}[term soundness]\label{subtype-term-semantics}
If $\tau()(\Gamma) = \ttcontext$, $\gamma$ satisfies $\Gamma$, and $\tauGamma{a} = A$ then $\Mgamma{a} \in
\Mgamma{A}$\@.  
\end{theorem}
\begin{proof}  There are two affected cases in the proof from that of
Theorem~\ref{simple-term-semantics}, namely, those of application
and projection.  The case of projection expressions is straightforward
given Proposition~\ref{maximal-subset}.

When 
 $a \equiv (f\ b)$, by Definition~\ref{defn:type rules with subtypes},
we have that $\tauGamma{f} = \funtype{B}{A}$ and $\tauGamma{b} = B'$.
Let $X$ be $\Mgamma{B}$, $X'$ be $\Mgamma{B'}$,  and $Y$ be $\Mgamma{A}$.
Then by Definition~\ref{defn:meaning function}, $\Mgamma{\funtype{B}{A}} =
Y^X$\@.  By the induction hypotheses, $\Mgamma{f}\in Y^X$ and
$\Mgamma{b}\in X'$\@.  
By Definition~\ref{defn:type rules with subtypes},
soundness of the proof rules (Theorem~\ref{proof-soundness}), and
Propositions~\ref{maximal-subset} and 
~\ref{maximal-equality}, there is a 
maximal supertype $\mu(B)$ of both $B$ and $B'$ such that $X$ and $X'$
are both subsets of $\Mgamma{\mu(B)}$\@.  Since, by
Definition~\ref{defn:type rules with subtypes}, $\vdash_\Gamma{}\pi(B)(b)$,
and by Proposition~\ref{maximal-subtype}, $\Mgamma{B} =
\Mgamma{\{x : \mu(B)\vbar \pi(B)(x)\}}$, we have $\Mgamma{b}\in
\Mgamma{B}$, and hence by Definition~\ref{defn:meaning function}, $\Mgamma{(f\
b)}\in  \Mgamma{A}$\@.  
\end{proof}



\section{Summary}

PVS features a form of subtyping where it is possible to form the subtype
of a type satisfying a given predicate on the type.  This kind of
subtyping introduces several delicate semantic issues into PVS.  A term
can now have several types since, for example, the term corresponding to
the number {\tt 2} can be a prime number, an even number, a natural
number, an integer, a rational number, or a real number.  When the
expected type is a subtype, the canonical type of the actual term must be
compatible with the expected type, that is, the two maximal supertypes
must be equivalent and the actual term must satisfy any subtype
constraints imposed by the expected type.  We have defined the notions of
maximal supertype, subtype constraints, type equivalence, and
compatibility.  These notions are used to define the type rules and
semantics of the simply typed fragment of PVS extended with subtypes.
Note that both type equivalence (and hence, compatibility) and type
correctness are undecidable.  Proof obligations generated during
typechecking are the only source of such undecidability.  The
modularization of the type system into a decidable part consisting of the
simply typed fragment,  and the proof obligations generated by
subtyping, is perhaps the most significant design consideration in the PVS
language.




\chapter{Dependent Types}\label{dependent}

The PVS language fragment described thus far is already quite expressive.
It employs definitional equivalence between types and contains
predicate subtypes.  It is undecidable whether an expression in this
fragment is type-correct because of the proof obligations that arise with
respect to predicate subtypes and type equivalence.  
The next step is the addition of type dependencies between the components
of a type.  This extension considerably
enhances the utility of this type system.  It is also a natural extension
given predicate subtyping which already allows types that
{\em depend\/} on free variables in the predicates.  
With dependent typing, we can
make the type of one component of a product depend on the value of another
component, or the type of the range of a function vary according to its
argument value.

\begin{comment}
Let $V$ be a metavariable that ranges over types or {\em
type bindings\/} of the form $x : T$.
\end{comment}
A dependent product type is
written as $\tupletype{x : A, B}$.  A dependent function type is written
as $\funtype{x : A}{B}$.
Any product or function type can be transformed into a dependent type by
inserting dummy type bindings.  Conversely, any dummy type bindings that
do not actually bind any variable occurrences can be
removed.  The type rules and semantics below will assume that all product and
function types are presented as dependent types.

\begin{Eg}{dependent types}
$$
\begin{array}{l}
\tupletype{i : \ttnat, \{j: \ttnat \vbar j\leq i\}}, \\
\tupletype{i : \ttnat, \funtype{\{j: \ttnat \vbar j\leq i\}}{\ttbool}},\\
\funtype{i: \ttint}{\{j : \ttint \vbar i\leq j\}}.
\end{array}$$
\end{Eg}


Before we treat dependent types, we update the definitions of the set
of free variables and substitution to account for the fact that
with subtyping and dependent typing, both free and bound variables
can occur in terms and types.  This is needed for the next step
where we try to remove type dependencies by substituting a term into
a dependent type.  
\begin{Defn}{free variables for types}
\begin{eqnarray*}
%%  FV(\Gamma)(s) & = & \emptyset, \mbox{ if } \itkind(\Gamma(s)) \neq
%%\ttvariable\\
FV(\Gamma)(\funtype{x : A}{B}) & = & FV(\Gamma)(A)\union
                   (FV(\Gamma, x: \ttvar~A)(B) - \{x\}) \\
%
FV(\Gamma)(\tupletype{x: A, B}) & = &
  FV(\Gamma)(A)\union (FV(\Gamma,
x:\ttvar~A)(B) - \{ x\}) \\
%
FV(\Gamma)(\{x : A \vbar a\}) & = & FV(\Gamma)(A) \union
(FV(\Gamma, x:\ttvar~A)(a) - \{x\})
\end{eqnarray*}
\end{Defn}

\begin{Defn}{substitution for types}
\begin{eqnarray*}
\lefteqn{\funtype{x : A}{B}[a_1/x_1,\ldots, a_n/x_n]}\\ & = & \funtype{y :
A[a_1/x_1,\ldots,a_n/x_n]}{B[y/x, a_1/x_n, \ldots, a_n/x_n]} \\
%
\lefteqn{\tupletype{x: A, B}[a_1/x_1, \ldots, a_n/x_n]}\\ & = &
\tupletype{y: A[a_1/x_1,\ldots, a_n/x_n], 
B[y/x, a_1/x_1,\ldots,a_n/x_n]} \\
%
\lefteqn{\{x : A\vbar a\}[a_1/x_n,\ldots, a_n/x_n]}\\ & = & \{y : A[a_1/x_1,\ldots,
a_n/x_n] \vbar  a[y/x, a_1/x_n, \ldots, a_n/x_n] \}
\end{eqnarray*}
where y is a fresh variable.
\end{Defn}

%\memo{Need to extend substitution to types.}

\begin{comment}
To determine whether a given expression is of a given dependent type, we
must define the notion of a substituting the expression into the type.  If
$\tauGamma{a}$ is a (possibly dependent) product type and $A$ is a
(possibly dependent) product type of the form $\tupletype{x : A_1, A_2}$, then
$a/A$ is the type obtained by ``substituting'' $a$ in $A$: if $a$ is of
the form $(\listwo{a}{n})$ then $a/A$ is $\tupletype{x : A_1,
a_2/(A_2[a_1/x])}$,
else if
$a$ is not of the form $(\listwo{a}{n})$, then we use $\proj{i}\ a$ in place
of $a_i$\@.  Similarly, if $\tauGamma{a}$ is a function type of the form
$\funtype{x : T}{S}$, then if $a$ is of the form $(\lambda (x : T): b)$,
then $(\lambda (x: T): b)/\funtype{y : A}{B}$ is $\funtype{y :
A}{b[y/x]/B}$\@.  If $a$ is not a lambda-expression but $\tauGamma{a}$ is
of the form $\funtype{z:T}{S}$, then $a/\funtype{y : A}{B}$ is just
$(\lambda (x : T): a(x))/\funtype{y : A}{B}$.  In every other case, $a/A$
is just $A$\@.
%%\memo{Need subtype case for substituting.  Put in separate definition.}
\begin{Eg}{substituting into dependent types}
\begin{eqnarray*}
\lefteqn{(5, 3)~/~\tupletype{i : \ttnat, \{j: \ttnat \vbar j\leq i\}}}\\
  & \equiv &
\tupletype{i: \ttnat, \{j: \ttnat \vbar j\leq 5\}}\\
%
\lefteqn{(3, (\lambda (k: \{j : \ttnat\vbar j < 3\}): k))~/~\tupletype{i : \ttnat,
\funtype{k : \{j: \ttnat \vbar j< i\}}{\{j: \ttnat \vbar j\leq k\}}}}\\
&
\equiv &
\tupletype{ i : \ttnat, \funtype{k : \{j: \ttnat \vbar j< 3\}}{\{j: \ttnat \vbar j\leq k\}}} 
\end{eqnarray*}
\end{Eg}
\end{comment}

\newcommand{\constrain}[2]{#1\backslash #2}
The definition of $\mu$ has to be modified slightly for dependent types.
The definition is first extended to type bindings, $\mu(x : T) = x :
\mu(T)$.  The definition for the case of dependent function types
is unchanged so that $\mu(\funtype{x : A}{B}) = \funtype{x :
A}{\mu(B)}$\@.   The definition for the product case is more delicate
since the definition $\mu(\tupletype{x : A, B}) = \tupletype{x : \mu(A),
\mu(B)}$ results in a loss of type information regarding the occurrences
of $x$ in $B$\@.\footnote{Doug Howe brought this problem to our attention.}
To ensure that type information regarding $x$ is
retained, we define a new operation $\constrain{T}{a}$
which constrains the subtype assertions in  type $T$ with an additional
assertion $a$\@.
\begin{Defn}{Adding subtype constraints}
\begin{eqnarray*}
\constrain{s}{a} & = & s \\
\constrain{\subtype{x}{T}{b}}{a} & = & \subtype{x}{T}{a \wedge b}\\
\constrain{\funtype{A}{B}}{a} & = &
\funtype{\constrain{A}{a}}{\constrain{B}{a}} \\
\constrain{\tupletype{A, B}}{a} & = & \tupletype{\constrain{A}{a}, \constrain{B}{a}}
\end{eqnarray*}
\end{Defn}

We can now define the maximal supertype operation for
dependent tuple types.
\begin{Defn}{Maximal supertype for dependent product types}
\begin{eqnarray*}
\mu(\tupletype{x : A, B}) & = & \tupletype{x : \mu(A), \constrain{B}{\pi(A)(x)}}
\end{eqnarray*}
\end{Defn}

The definition of $\pi$ for a dependent function type $\funtype{ y :
A}{B}$ is slightly different from that of an ordinary function type
since $\pi(B)$ can contain free occurrences of the variable $y$\@.
For example, $\pi(\funtype{i : \ttint}{\{j : \ttint \vbar i\leq j\}})$
must be  $\lambda (f : \funtype{i : \ttint}{\ttint}): (\forall (i:\ttint): i
\leq f(i))$\@.  The definition for dependent tuples remains essentially
unchanged from that of ordinary products.  
\begin{Defn}{constraint predicates for dependent types}
\begin{eqnarray*}
   \pi(\funtype{y:A}{B}) & = &  ( \lambda (x : \funtype{y:A}{\mu(B)}):
    (\forall (y: A): \pi(B)(x(y)))) \\
%
  \pi(\tupletype{y:A, B}) & = &
   ( \lambda (x: \tupletype{y : \mu(A), \constrain{\mu(B)}{\pi(A)(y)}}): \\&&\hspace*{.3in}
	\pi(A)(\proj{1}~x)\wedge \pi(B)(\proj{2}~x)[(\proj{1}~x)/y])
\end{eqnarray*}
\end{Defn}

\newcommand{\depfunexample}{\funtype{i : \ttint}{\{j : \ttint \vbar i\leq
j\}}}
\newcommand{\deptupexample}{\tupletype{i : \ttnat, \{j: nat \vbar j\leq i\}}}
\begin{Eg}{dependent type predicates}
\begin{eqnarray*}
\mu(\depfunexample{}) & = & \funtype{i : \ttint}{\ttint}\\
\pi(\depfunexample{}) & = & \lambda (f : \funtype{i : \ttint}{\ttint}):\\&&\hspace{.3in}
\forall (i: \ttint): (\lambda (j : \ttint): i\leq j)(f(i))
\end{eqnarray*}
\end{Eg}

The definition of $\simeq$ must also be massaged slightly for dependent
types.  Recall that $\simeq$ checks whether two maximal types are
equivalent by generating proof obligations as needed.  This is the
basic operation for checking whether the expected type of an expression
is compatible with its actual type.  
The subtlety now is that the expected type might be
a dependent type where the actual type is not.  Consider the
case of the pair $\pair{5, (\lambda (x : \{j:\ttnat \vbar j\leq 5\}): x)}$
whose type would be computed 
by $\tau$ as $\tupletype{i : \ttnat, \funtype{\{j : \ttnat\vbar j\leq 5\}}{\{j : \ttnat\vbar j\leq 5\}}}$
where the expected type might be 
$\tupletype{i : \ttnat, \funtype{\{j : \ttnat \vbar j\leq i\}}{\{j : \ttnat \vbar
j\leq i\}}}$\@.  To cope with this, 
we will allow the option of two maximal types, say $A$ and $B$, to be
compared using $\simeq$
in the context of an expression $a$\@.  This is indicated by the notation $(A\simeq B)/a$\@.  
Note that $(A\simeq B)/a$ is sensible only when $A$ and $B$
are maximal types.   The missing cases in
Definition~\ref{defn:type equivalence proof obligations}
are included in
Definition~\ref{defn:type equivalence for dependent types}.    For a list
of formulas $a_1, \ldots, a_n$, let 
$(\forall (x : T): a_1,\ldots, a_n)$ represent the list
$(\forall (x : T): a_1), \ldots, (\forall (x : T): a_n)$\@.\footnote{Note that the type-correctness of the proof
obligation $(\pi(A) = \pi(A'))$ in
Definition~\ref{defn:type equivalence for dependent types}  depends on the
\emph{prior} proof obligations 
$\mu(A)\simeq \mu(A')$.}
\begin{Defn}{type equivalence for dependent types}
\begin{eqnarray*}
  (s \simeq s)/a & = & \tttrue \\
(\funtype{x:A}{B} \simeq \funtype{x':A'}{B'}) & = & (\mu(A)\simeq
\mu(A'));\\ & & 
(\pi(A) = \pi(A'));\\ & & 
 (\forall (x : A): (B \simeq B'[x/x'])) \\
%
(\funtype{x:A}{B} \simeq \funtype{x':A'}{B'})/a & = & (\mu(A)\simeq
\mu(A'));\\ & & 
(\pi(A) = \pi(A'));\\ & & 
 (\forall (x : A): (B \simeq B'[x/x'])/a(x)) \\
%
(\tupletype{x : A_1, A_2} \simeq \tupletype{y :
B_1, B_2})
& = &
    (A_1\simeq B_1);\\ & & (\forall (x : A_1) : (A_2\simeq
B_2[x/y]))\\
%
(\tupletype{x : A_1, A_2} \simeq \tupletype{y :
B_1, B_2})/a
& = &
    (A_1\simeq B_1)/(\proj{1}~a);\\ & & (A_2[(\proj{1}~a)/x]\simeq
B_2[(\proj{1}~a)/y])/(\proj{2}~a)\\
%
 (A \simeq B)/a & = & \ttfalse, \mbox{ otherwise.}
\end{eqnarray*}
\end{Defn}
As with $(A\sim B)_\Gamma$, the notation $(A \sima 
B)_\Gamma$ indicates that  all the proof obligations $a'$ in
$(\mu(A)\simeq \mu(B))/a$ are provable, that is,  $\vdash_\Gamma a'$\@.  

With dependent types, the type rules must be modified so as to augment the
context suitably to account for any dependencies.  
We will give the definitions only for dependent type constructions.

%\memo{need to handle type dependencies in expected for application case.}
\begin{Defn}{type rules with dependent types}
\begin{eqnarray*}
\tauGamma{\tupletype{x : A, B}} & = &
\tttype{}, \mbox{ if } \Gamma(x) \mbox{ is undefined},\\
& & \tauGamma{A} = \tttype{}, \mbox{ and}\\
& & \tau(\Gamma, x : \ttvar{}\ A)(B) = \tttype{} \\
%
%\tauGamma{\tupletype{T_1, U_2, \ldots, U_n}} & = &
%\tttype{}, \mbox{ if } \tauGamma{T_1} = \tttype{}, \mbox{ and}\\
%& & \tauGamma{\tupletype{U_2,\ldots, U_n}} = \tttype{} \\
%
\tauGamma{\funtype{x: A}{B}} & = & \tttype{},
    \mbox{ if } \Gamma(x) \mbox{ is undefined},\\
& & \tauGamma{A} = \tttype{}, \mbox{ and}\\
& & \tau(\Gamma, x : \ttvar{}\ A)(B) = \tttype{}\\
%
  \tauGamma{f\ a} & = &  B', \mbox{ where } \mu_0(\tauGamma{f}) = \funtype{x :
A}{B},\\ 
 & &                   			\tauGamma{a} = A', \\ & & 
					(A \sima  A')_\Gamma, \\
& &
                                        B' \mbox{ is } B[a/x],\\ & & 
                                        \vdash_\Gamma \pi(A)(a)\\
%
 \tau(\Gamma)(\lambda (x : A): a) & = &
         \funtype{x : A}{B}, \mbox{ where}\\
%& & A  = \tupletype{x_1 : T_1, \ldots, x_n : T_n}, \\
& & B = \tau(\Gamma, x : \ttvar{}\ A)(a)\\ 
%B' = B[(\proj{1}\ x)/x_1, \ldots, (\proj{n}\
%x)/x_n] \\ & & 
%\mbox{if } \mbox{ for } 1\leq i\in \{1, 2\},\\& & 
%\tau(\Gamma, x_1 : \ttvar{}\ T_1,\ldots, x_{i-1} : \ttvar{}\ T_{i-1})(T_i) = \tttype{}, \\
%& & \mbox{and for } 1\leq i, j \in \{1, 2\}, i\neq j, x_i \not\equiv x_j \\
%
\tauGamma{\proj{1}~a} & = & A_1, \mbox{ where } \mu_0(\tauGamma{a}) =
\tupletype{x: A_1, A_2} \\
%
\tauGamma{\proj{2}~a} & = & A_2[(\proj{1}~a)/x], \mbox{ where } \mu_0(\tauGamma{a}) =
\tupletype{x: A_1, A_2} \\
\end{eqnarray*}
\end{Defn}
%%\memo{Need proj case above.}

\begin{Eg}{dependent typing}
\begin{eqnarray*}
\tauGamma{\tupletype{x : \ttbool,  \{ y : \ttbool \vbar x \supset
y\}}}
& = & \tttype\\
\tauGamma{\funtype{x : \ttbool}{\{ y : \ttbool \vbar x \supset 
y\}}} & = & \tttype
\end{eqnarray*}
\end{Eg}
Before we can assign meanings to dependent types, we must
augment our definition of the universe $U$ to contain sets corresponding
to these constructions.  If $F$ is a function with domain set $X$ and
a range $Y$, which is a set of sets, we can  define $\Sigma{F}$ to be the set 
$\{ \pair{x, y} \vbar x\in \textit{dom}(F), y\in F(x)\}$ and
$\Pi{F}$ to be the set $\{f \vbar (\forall x\in \textit{dom}(F): f(x)\in
F(x))\}$\@.   Note that $\Pi{F} \subseteq \bigcup_{X\in \Sigma{F}}\wp(X)$ but we
include $\Pi{F}$ in the universe $U$ defined below for simplicity\@.
We can drop $X\times Y$ and $X^Y$ from the universe definition since
$X\times Y$ can be obtained from $\Sigma{F}$ by defining an
$F$ with domain $X$ that always returns $Y$, and similarly,
$X^Y$ can be obtained by $\Pi{F}$ where $F$ is defined to with domain $Y$
to always return $X$\@.  The universe $U$ can then be redefined as below.  
\begin{Defn}{type universe with dependent types}
\begin{eqnarray*}
  U_0 & = & \{ {\twob}, {\reals}  \} \\
  U_{i+1} & = & U_i  \\& &
               %\union \{X\times Y \vbar X, Y \in U_i\} \\ & & 
                         %\union  \{X^Y \vbar X, Y \in U_i \} \\ & & 
                          \union  \bigcup_{X\in U_i}\wp(X)\\& & 
                           \union\; \{\Sigma{F} \vbar F \in W_i\}\\& &
                            \union\; \{\Pi{F} \vbar F \in W_i\}\\
  W_i & = & \bigcup_{X\in U_i} U_i^X \\
  U_{\omega} & = & \bigcup_{i\in\omega} U_i\\
  U & = & U_{\omega}
\end{eqnarray*}
\end{Defn}

One very important consequence of the above extension of the universe
is that all type dependencies must be bounded in the sense that
if $B$ is a type expression with a single free variable $x$ of type $A$,
then it must be the case that for any set $\mean{A}$ representing $A$,
there is a bound $n$ such that for any $z$ in $\mean{A}$, the meaning of 
$B$ under $\{x \gets z\}$ must be in $U_n$\@.  This property is easily
proved by induction on the structure of a PVS type since the parameter
$x$ can appear only in the predicate part of a subtype where the
rank of the meaning of the resulting type cannot vary with the value
of $x$\@.   In
particular, there is no way to define a type constructor $T^n$ in PVS that
returns the $n$-tuple $[\underbrace{T, [\ldots, T]}_n]$ for a given $n$
since this would entail an unbounded dependency.  
If unbounded type dependencies were allowed in PVS,
one can construct a dependent type such as $\funtype{n : nat}{T^n}$
whose representation is not in $U$ as defined above.    
  

\begin{comment}
A simpler alternative might be to define a second universe
U2 which starts from U1 and adds the dependent types.  This will work fine
in the absence of parametric theories since each type contains only a
fixed number of type dependencies.  The introduction of parametric
theories might require a third iteration since they introduce a third
form of dependency.
\end{comment}

\begin{comment}
Seeking advice on PVS semantics:  I'd like to know if the following
universe construction provides a satisfactory model for PVS types.  The
PVS simple type system consists of the base types bool and real and the
tuple [A_1,...,A_N] and function [A->B] types.  The flg. universe U1
(where (U_n)^n is the n-fold Cartesian product and (U_n)^(U_n) is the
function space) provides a model: 
 U_0     = {2, R}
 U_{n+1} = U_n UNION (U_n)^n UNION (U_n)^(U_n)
 U_{omega} = SUP_n U_n
 U1      = U_omega
%
The next step is to add subtypes {x:A \vbar p(x)}.  The above universe can be
easily modified to accommodate these by adding the subsets of the finite
simple types. Here P(X) is the power-set of X.  
%
 U_{n+1} = U_n UNION {U_n}^n UNION (U_n)^(U_n) UNION P(U_n)
%
The type system is then extended with dependent function [x:A->B(x)]
and tuple types [x_1: A, B].  The simple way to construct a universe is to
say that there are only finite length chains of dependencies and these are
captured by the universe U2 below which is constructed iteratively from
U1.  Here, if F is a function graph, #F is the set
{<x, y> \vbar x in domain(F), y in F(x)}. 
%
  U2_0 = U1
  U2_{n+1} = U_n UNION (U_n)^n UNION (U_n)^(U_n) UNION P(U_n)
             UNION {#F \vbar F in {(U2_n)^X \vbar X in U2_n}}
  U2_{omega} = SUP_n U2_n
  U2 = U2_{omega}
%
The reason this works is that it is easy to see by induction on the
type structure that we can assign a degree k to a given type,
where the degree of a nondependent type is 0, and the degree of
a type of the form [x: S, T(x)] is max(deg(S), deg(T)) + 1.
Note that if a type has degree k, then it is introduced in the
kth iteration of the definition of U2.  Dependent function types
are ignored here since they are defined as subsets of the corresponding
tuples so that they are introduced in the next iteration by the power-set
construction.
%
There really is no need for a second universe construction here since the
above argument would also work if the U2 iteration was rolled into that of
U1.
%
Parametric theories pose a serious difficulty since we now have types and
constants that can depend on types.  The context provides a
meaning function to each theory that maps the meanings of the
actual parameters to the meanings of the symbols declared in the theory.
We need to constrain the meaning function for a theory so that
the degree of each type name in the theory depends only on the
degrees of the type parameters.  This is a basic invariant in PVS that
the dependency between types and individuals only affects the
predicate part of the type and not its structure.  One cannot, for
instance, define a tuple whose first component is n:nat, and whose
second component is the n-fold tuple T^n.   Note that this
invariant is preserved by the way in which theories are actually used.
An interpretation of a theory simply attaches a definition to an
undefined type and the meaning of this definition will also have the
boundedness property.  
\end{comment}









The meaning function for dependent types is obtained by adding the
cases corresponding to 
dependent product and function types.  All the other cases are unchanged from
Definition~\ref{defn:meaning function with subtypes}\@.  Note that the
semantic definition  for dependent types is 
equivalent to the nondependent one when there are no dependencies.
\begin{Defn}{meaning function with dependent types}
\begin{eqnarray*}
  \Mgamma{\tupletype{x : A, B}} & = &
    \Sigma{F}, \mbox{ where }\\& &
     F \mbox{ maps } z \in \Mgamma{A} \mbox{ to }\\& & \mathcal{M}(\Gamma, x : \ttvar{}\
A\vbar \gamma \{x\gets z \})(B)\\
%
  \Mgamma{\funtype{x:A}{B}} & = &  \Pi{F}, \mbox{ where }\\& &
                        F \mbox{ maps } z \in \Mgamma{A} \mbox{ to }\\& &  \mathcal{M}(\Gamma,
x : \ttvar{}\  A\vbar \gamma\{x\gets z\})(B)  
%    
\end{eqnarray*}
\end{Defn}

\begin{Eg}{meaning function with dependent types}
\begin{eqnarray*}
\Mgamma{\tupletype{x : \ttbool,  \{ y : \ttbool \vbar x \supset 
y\}}}
& = & \{\pair{\zerob, \zerob}, \pair{\zerob, \oneb},
\pair{\oneb, \oneb}\}\\ 
\Mgamma{\funtype{x : \ttbool}{\{ y : \ttbool \vbar x \supset 
y\}}} & = & \{\{\pair{\zerob, \zerob}, \pair{\oneb, \oneb}\},\\&&\hspace*{1ex}
              \{\pair{\zerob, \oneb}, \pair{\oneb, \oneb}\}\}
\end{eqnarray*}
\end{Eg}

We now need to show that the extensions corresponding to dependent types
preserve the properties in Theorems~\ref{subtype-type-semantics}
and~\ref{subtype-term-semantics}, namely, 
$\Mgamma{T} \in U$ and $\Mgamma{a}\in \Mgamma{\tauGamma{a}}$\@.
For the former, we prove a stronger theorem that incorporates the
rank-boundedness of dependent types.
\begin{theorem}[rank bounded type semantics]\label{dependent-type-semantics}
If $B$ is a pretype, $x_1,\ldots, x_n$ is a list of symbols,
$A_1,\ldots, A_n$ is a list of pretypes such that  
\begin{enumerate}
\item $\tau()(\Gamma, x_1:\ttvar~A_1, \ldots, x_n:\ttvar~A_n) = \ttcontext$,
\item $\tau(\Gamma, x_1:\ttvar~A_1, \ldots, x_n:\ttvar~A_n)(B) = \tttype$,
and
\item $\gamma$ is an assignment satisfying $\Gamma$,
\end{enumerate}
then
there is an $i$ such that for any list of values
$z_1,\ldots, z_n$ where $\gamma\{x_1\gets
z_1\}\ldots\{x_n\gets z_n\}$ is a satisfying assignment for
$\Gamma, x_1:\ttvar~A_1, \ldots, x_n:\ttvar~A_n$,  we have
$$\mathcal{M}(\Gamma, x_1:\ttvar~A_1, \ldots, x_n:\ttvar~A_n\vbar
\gamma\{x_1\gets z_1\}\ldots\{x_n\gets z_n\})(B) \in U_i.$$
\end{theorem}
\begin{proof}
The proof is by structural induction on the pretype $B$\@.
Let $\Gamma'$ denote $\Gamma, x_1:\ttvar~A_1, \ldots, x_n:\ttvar~A_n$,
$\gamma'$ denote $\gamma\{x_1\gets z_1\}\ldots\{x_n\gets z_n\}$, and 
$\mean{C}$ denote $\mathcal{M}(\Gamma'\vbar
\gamma')(C)$.  
\begin{enumerate}
\item $B\equiv s$: Since $\mean{B}$ is just $\gamma(B)$ by
Definition~\ref{defn:meaning function}, we have that there is an $i$
such that $\mean{B}\in U_i$ regardless of the choice of values
$z_1,\ldots, z_n$\@.  

\item $B\equiv \{y:T\vbar a\}$: By the induction hypothesis, we know that
for some $j$, it is always the case that $\mean{T}\in U_j$\@.  By
Definition~\ref{defn:meaning function with subtypes}, we have that
$\mean{B}\subseteq \mean{T}$ so if we let $i = j+1$, then
by Definition~\ref{defn:type universe with dependent types},
it is always the case that $\mean{B}\in U_i$\@.

\item $B \equiv \funtype{y:C}{D}$: By
Definition~\ref{defn:type rules with dependent types},
$\Gamma'(y)$ is undefined, $\tau(\Gamma')(C) = \tttype$,
$\tau()(\Gamma', y:\ttvar~C) = \ttcontext$, and
$\tau(\Gamma', y:\ttvar~C)(D) = \tttype$.
By the induction hypothesis, for some $j$,
it is always the case that $\mathcal{M}(\Gamma'\vbar\gamma')(C)\in U_j$,
and for some $k$, it is always the case that for any satisfying
assignment $\gamma'\{y\gets w\}$ for $\Gamma', y:\ttvar~C$,
we have $\mathcal{M}(\Gamma', y:\ttvar~C\vbar \gamma'\{y\gets
w\})(D)\in U_k$\@.   Then the function $F$ mapping $w$ in
$\mathcal{M}(\Gamma')(C)$ to $\mathcal{M}(\Gamma', y:\ttvar~C\vbar
\gamma'\{y\gets w\})(D)$ is an element of $W_{j+k}$\@.
Letting $i$ be $j+k+1$, we have by
Definition~\ref{defn:meaning function with dependent types}
that $\mathcal{M}(\Gamma'\vbar\gamma')(B)$ is
$\Pi{F}$ and is hence an element of $U_i$ by
Definition~\ref{defn:type universe with dependent types}\@.

\item $B\equiv \tupletype{y: C, D}$:  Similar to the
previous case.  
\end{enumerate}
\end{proof}

By choosing $n$ to be $0$, the previous theorem yields the result 
that when $\tauGamma{B} = \tttype$, $\Mgamma{B}\in U$\@.

We next need to establish that for any preterm $a$,
if $\tauGamma{a} = A$, then $\Mgamma{a}\in \Mgamma{A}$\@.
The first step in this direction is the proof of the substitution lemma
below.
\begin{prop}\label{context-equivalence}
If $\tau()(\Gamma) = \tau()(\Gamma') = \ttcontext$ where
for each $s$, $\Gamma(s)$ is defined if and only if  $\Gamma'(s)$ is defined, and 
$\gamma$ is an assignment satisfying both $\Gamma$ and $\Gamma'$,
then
\begin{enumerate}
\item If  $\Gamma(s) = \ \Gamma'(s)$ (i.e., they are equal when
either $\Gamma(s)$ or $\Gamma'(s)$ is defined), then 
\begin{enumerate}
\item $\tau(\Gamma)(a) = \tau(\Gamma')(a)$, for any preterm $a$.
\item $\tau(\Gamma)(A) = \tau(\Gamma')(A)$, for any pretype $A$.
\end{enumerate}

\item $\Mgamma{A} = \mathcal{M}(\Gamma'\vbar\gamma)(A)$, when $\tauGamma{A} =
\tttype$.

\item $\Mgamma{a} = \mathcal{M}(\Gamma'\vbar\gamma)(a)$, for any preterm $a$
such that $\tauGamma{a}$ is defined.
\end{enumerate}
\end{prop}
%
%%\memo{Needs induction on preterms and pretypes.}
\begin{lemma}[substitution lemma]\label{substitution-lemma}
If $\tau()(\Gamma, x:\ttvar~A) = \ttcontext$,
$\tauGamma{a} = A$, then
\begin{enumerate}
\item If $\tau(\Gamma, x: \ttvar~A)(b) = B$, then\\
$\Mgamma{b[a/x]} = \mathcal{M}(\Gamma, x:\ttvar~A\vbar \gamma\{x\gets
\Mgamma{a}\})(b)$.

\item If $\tau(\Gamma, x: \ttvar~A)(C) = \tttype$, then\\
$\Mgamma{C[a/x]} = \mathcal{M}(\Gamma, x:\ttvar~A\vbar \gamma\{x\gets
\Mgamma{a}\})(C)$\@. 
\end{enumerate}
\end{lemma}
\begin{proof}
The proof is by simultaneous structural induction on the preterm $b$ and
the pretype $C$\@.  The following cases deal with the preterm $b$.
\begin{enumerate}
\item $b\equiv s$: If $s\equiv x$, then by
Definition~\ref{defn:meaning function with dependent types}, the left-hand
side $\Mgamma{b[a/x]}$ is 
$\Mgamma{a}$, and the right-hand side  $\mathcal{M}(\Gamma, x:\ttvar~A\vbar \gamma\{x\gets
\Mgamma{a}\})(b)$ is also $\Mgamma{a}$\@. 

If $s\not\equiv x$, then by
Definition~\ref{defn:meaning function with dependent types},
the left-hand side and the right-hand side
are both equal to $\gamma(s)$\@. 

\item $b \equiv (\lambda (y : C): d)$: Since $C$ can contain free
occurrences of $x$, we have by the induction hypothesis that
$\Mgamma{C[a/x]} = \mathcal{M}(\Gamma, x:\ttvar~A\vbar \gamma\{x\gets
\Mgamma{a}\})(C)$\@.  Also, $\Mgamma{(\lambda (y:C): d)[a/x]}$ is equal to
the set of ordered pairs $\pair{v, z}$ such that $v\in \Mgamma{C[a/x]}$
and $z = \mathcal{M}(\Gamma, y:\ttvar~C[a/x]\vbar \gamma\{y\gets
v\})(d[a/x])$\@.

By the induction hypothesis,
$\mathcal{M}(\Gamma, y:\ttvar~C[a/x]\vbar \gamma\{y\gets v\})(d[a/x])
= \mathcal{M}(\Gamma, y:\ttvar~C[a/x], x:\ttvar~A\vbar \gamma\{y\gets
v\}\{x\gets \Mgamma{a}\})(d)$\@.  Since $x$ does not occur free in
$C[a/x]$, by Proposition~\ref{context-equivalence} we can exchange the
occurrences of $y$ and $x$ so that 
$\mathcal{M}(\Gamma, y:\ttvar~C[a/x], x:\ttvar~A\vbar \gamma\{y\gets
v\}\{x\gets \Mgamma{a}\})(d) =
\mathcal{M}(\Gamma, x:\ttvar~A, y:\ttvar~C[a/x]\vbar \gamma\{x\gets \Mgamma{a}\}\{y\gets
v\})(d)$.

By Definition~\ref{defn:meaning function with dependent types},
the right-hand side is the set of ordered pairs of the form $\pair{v, z}$
such that $v\in \mathcal{M}(\Gamma, x: \ttvar~A \vbar \gamma\{x\gets
\Mgamma{a}\})(C)$
and $z = \mathcal{M}(\Gamma, x: \ttvar~A, y: \ttvar~C\vbar \gamma\{x\gets
\Mgamma{a}\}\{y\gets v\})(d)$\@.  By Proposition~\ref{context-equivalence}
and the induction hypothesis, we know that
$\mathcal{M}(\Gamma, x: \ttvar~A, y: \ttvar~C\vbar \gamma\{x\gets
\Mgamma{a}\}\{y\gets v\})(d) =
\mathcal{M}(\Gamma, x:\ttvar~A, y:\ttvar~C[a/x]\vbar \gamma\{x\gets \Mgamma{a}\}\{y\gets
v\})(d)$, and hence it follows that the two sets of ordered pairs are
equal.

\item $b\equiv (f\ c)$:  In this case, $b[a/x]\equiv (f[a/x]~c[a/x])$
and the conclusion follows easily from the induction hypothesis
and Definition~\ref{defn:meaning function with dependent types}\@.

\item $b\equiv (b_1, b_2)$: 
The conclusion follows easily from 
Definitions~\ref{defn:substitution},~\ref{defn:meaning function with
dependent types}, and the induction hypotheses.


\item $b\equiv (\proj{i}~c)$: This case is also straightforward since
$b[a/x] \equiv (\proj{i}~c[a/x])$,  
and by the induction hypothesis,
$\mathcal{M}(\Gamma, x: \ttvar~a \vbar \gamma\{x \gets \Mgamma{a}\})(c) =
\Mgamma{c[a/x]}$.  
%
\end{enumerate}

The remaining cases deal with the pretype $C$\@.
\begin{enumerate}
\item $C\equiv s$:  This case is trivial since by
Definition~\ref{defn:substitution}, $C[a/x]\equiv C$ and
the left-hand and right-hand sides both reduce to $\gamma(C)$\@.

\item $C\equiv \{y : T \vbar d\}$: The argument here follows along the
lines of the $b\equiv (\lambda (x : C): D)$ case above.
By the induction hypotheses, we know that
\begin{eqnarray*}
\lefteqn{\mathcal{M}(\Gamma, x : \ttvar~A\vbar \gamma\{x\gets
\Mgamma{a}\})(T)}\\
& = & \Mgamma{T[a/x]}\\
\lefteqn{\mathcal{M}(\Gamma, y : \ttvar~T[a/x], x : \ttvar~A\vbar \gamma\{y\gets
z\}\{x\gets \Mgamma{a}\})(d)}\\
& = & \mathcal{M}(\Gamma, y : \ttvar~T[a/x]\vbar \gamma\{y\gets z\})(d[a/x]),\\&&
\mbox{ for any } z\in \Mgamma{T[a/x]}
\end{eqnarray*}
The conclusion follows from Proposition~\ref{context-equivalence}
and Definition~\ref{defn:meaning function with dependent types}.

\item $C\equiv \funtype{y:C_1}{C_2}$: The argument here is similar
to that of the previous case.  Essentially, by the induction hypothesis
and Proposition~\ref{context-equivalence}, the function mapping
$z\in \mathcal{M}(\Gamma, x : \ttvar~A\vbar \gamma\{x\gets \Mgamma{a})(C_1)$
to $\mathcal{M}(\Gamma, y:\ttvar~C_1[a/x], x : \ttvar~A\vbar \gamma\{y\gets
z\}\{x\gets \Mgamma{a}\})(C_2)$ is the same as the function mapping
$z\in \Mgamma{C_1[a/x]}$ to $\mathcal{M}(\Gamma, y: \ttvar~C_1[a/x]\vbar
\gamma\{y\gets z\})(C_2[a/x])$\@.  

\item $C\equiv \tupletype{y:C_1, C_2}$:   Similar to the previous case.
\end{enumerate}
\end{proof}

Proposition~\ref{substitution-equivalence} is stated below without proof.
It asserts the semantic equivalence with respect to term $a$
of types $A$ and $B$ when $(A\sima B)_\Gamma$ holds.   Note that its
correctness depends on the soundness of the proof rules.  

\begin{prop}\label{substitution-equivalence}
If $\tau()(\Gamma) = \ttcontext$,  $a$ is a preterm such that
$\tauGamma{a} = B$, and $(A\sima  B)_\Gamma$,
then $\Mgamma{a}\in \Mgamma{A}$ iff $\Mgamma{a}\in \Mgamma{B}$\@.  
\begin{comment}
\begin{enumerate}
\item $\Mgamma{a}\in \Mgamma{A}$ iff $\Mgamma{a}\in \Mgamma{a/A}$
\item $\vdash \pi(a/A)(a)$ iff $\vdash \pi(A)(a)$\@.
\end{enumerate}
\end{comment}
\end{prop}
%
\begin{theorem}\label{dependent-term-semantics}
If $\tau()(\Gamma) = \ttcontext$, $\gamma$ is an assignment satisfying
$\Gamma$, and $a$ is a preterm such that $\tauGamma{a} = A$, then
$\Mgamma{a}\in \Mgamma{A}$\@.   
\end{theorem}
\begin{proof}
The proof is by  induction on the structure of the preterm
$a$\@.
\begin{enumerate}
\item $a\equiv s$: Then  by
Definition~\ref{defn:meaning function with dependent types},
$\Mgamma{a} = \gamma(a)$, and by Definition~\ref{defn:satisfaction},
we have that $\gamma(a)\in \Mgamma{A}$\@.

\item $a \equiv (\lambda (x : C): b)$:  By
Definition~\ref{defn:type rules with dependent types}, we
have $\tauGamma{a} = A = \funtype{x : C}{\tau(\Gamma, x: \ttvar~C)(b)}$\@.
Let $B$ label $\tau(\Gamma, x: \ttvar~C)(b)$\@.  
We know that $\Mgamma{A}$ is of the form $\Pi{} F$ where
$F$ maps $z\in \Mgamma{C}$ to $\mathcal{M}(\Gamma, x : \ttvar~C\vbar \gamma\{x
\gets z\})(B)$\@.

By the induction hypothesis on $b$, we know that for any $z\in \Mgamma{C}$, 
$\mathcal{M}(\Gamma, x : \ttvar~C\vbar \gamma\{x \gets z\})(b)\in
\mathcal{M}(\Gamma, x : \ttvar~C\vbar \gamma\{x
\gets z\})(B)$\@.  
Since by Definition~\ref{defn:meaning function with dependent types},
$\Mgamma{a}$ is a function mapping $z\in \Mgamma{C}$ to
$\mathcal{M}(\Gamma, x : \ttvar~C\vbar \gamma\{x \gets z\})(b)$, we have
$\Mgamma{a} \in \Pi F$ by the
definition of $\Pi$\@.  

\item $a \equiv (f~b)$: By
Definition~\ref{defn:type rules with dependent types}, we have that
$\tauGamma{f} = \funtype{x : B}{A'}$, $\tauGamma{b} = B'$,
$(B \sima  B')_\Gamma$,  $A \equiv A'[a/x]$, and
$\vdash_\Gamma\pi(B)(b)$\@.  We know by the induction hypothesis
that $\Mgamma{f}\in \Mgamma{\funtype{x : B}{A'}}$ and
$\Mgamma{b} \in \Mgamma{B'}$\@.  By
Propositions~\ref{substitution-equivalence}
and~\ref{maximal-subtype},
$\Mgamma{b}\in \Mgamma{\mu(B)}$\@.  
We therefore have 
by Proposition~\ref{maximal-subtype} that $\Mgamma{b}\in \Mgamma{B}$\@.
By Definition~\ref{defn:meaning function with dependent types},
$\Mgamma{a}\in \mathcal{M}(\Gamma, x: \ttvar~B\vbar \gamma\{x\gets
\Mgamma{b}\})(A')$,
and hence  by Lemma~\ref{substitution-lemma} it follows that
$\Mgamma{a}\in \Mgamma{A'[b/x]}$\@.  

\item $a \equiv (\listwo{a}{n})$:  The conclusion follows easily from
the induction hypothesis and
Definition~\ref{defn:type rules with dependent types}\@.

\item $a\equiv (\proj{i}~b)$: The conclusion follows easily from
Proposition~\ref{maximal-subtype}, the  induction hypothesis,
and Definition~\ref{defn:type rules with dependent types}\@.
The $(\proj{2}~b)$ case also employs Lemma~\ref{substitution-lemma}\@.  
\end{enumerate}
\end{proof}

\begin{comment}
\begin{lemma}\label{rank-boundedness}
If $\tau()(\Gamma) = \ttcontext$,
$\gamma$ is a satisfying assignment for $\Gamma$, 
$\tauGamma{A} = \tttype$,
$\Mgamma{A} \in U$, 
and $\tau(\Gamma, x: \ttvar~A)(B) = \tttype$, then
there is an $i$ such that the function $F$ mapping
$z$ in $\Mgamma{A}$ to 
$\mathcal{M}(\Gamma, x : \ttvar~A\vbar \gamma\{x\gets z\})(B)$
is in $W_i$\@.  
\end{lemma}
\begin{proof}
The proof is by induction on the structure of the pretype $B$\@.
Let the rank of $\Mgamma{A}$ be $j$\@.  
\begin{enumerate}
\item $B\equiv s$: Then $\mathcal{M}(\Gamma, x : \ttvar~A\vbar \gamma\{x\gets
z\})(B)$ is just $\gamma(s)$ and by Definition~ref{defn:satisfaction},
we have some $k$ such that $\gamma(s)\in U_k$\@.  Let $i$ be the maximum
of $j$ and $k$, we have $F\in U_i$\@.
%
\item $B\equiv \funtype{y:C}{D}$:  We have the induction hypothesis
%
\end{enumerate}
\end{proof}
%
\begin{theorem}\label{dependent-type-semantics}
If $\tau()(\Gamma) = \ttcontext$,  $\gamma$ satisfies $\Gamma$,
and $\tauGamma{A} = \tttype$, 
then 
\begin{enumerate}
%\item  $\Mgamma{A}\in U$, and
%%\item if $\tauGamma{b} = B$ then $\Mgamma{b} \in \Mgamma{B}$, and
\item if for all $n$, for $i$,  $1\leq i\leq n$,  $\mathcal{M}(\gamma,
x_1:\ttvar~A_1, \ldots x_{i-1}:\ttvar~A_{i-1}\vbar {A_i})\in U$ and 
$\tau(\Gamma, x_1 : \ttvar~A_1, \ldots, x_n:\ttvar~A_n)(B) = \tttype$,
then $\mathcal{M}(\Gamma, x_1 : \ttvar~A_1\ldots\vbar \gamma\{x\gets z\})(B)$
there exists an $i$ such that 
whenever $F$ maps $z$ in $\Mgamma{A}$ to $\mathcal{M}(\Gamma, x :
\ttvar~A\vbar \gamma\{x\gets z\})(B)$, there exists an $i$ such
that $F\in W_i$\@.  
\end{enumerate}
\end{theorem}
\begin{proof}
The proof is by simultaneous induction on the sizes of pretypes $A$ and
$B$\@.  We present the structure of the entire proof here but refer to
the proofs of Theorems~\ref{simple-type-semantics}
and~\ref{subtype-type-semantics} as needed.
\begin{enumerate}
\item $A\equiv s$: Same as in Theorem~\ref{simple-type-semantics}\@.
%
\item $A\equiv \funtype{y : C}{D}$: We have by the induction hypothesis
that $\Mgamma{C}\in U$ and for any $z\in \Mgamma{C}$, $\mathcal{M}(\Gamma,
y:\ttvar~C\vbar \gamma\{y\gets z\})(D) \in U$\@.  We also have by the
induction hypothesis with $D$ for $B$ that there is an $i$ such that
the function $F$ mapping $z\in \Mgamma{C}$ to $\mathcal{M}(\Gamma,
y:\ttvar~C\vbar \gamma\{y\gets z\})(D)$ is in $W_i$\@.
Since, by Definition~\ref{defn:meaning function with dependent types},
$\Mgamma{A} = \Pi{F}$, we have $\Mgamma{A}$ in $U_{i+1}$ and hence $U$\@.
%
\item $A\equiv \tupletype{x_1 : T_1,\ldots, x_n : T_n}$:  We have by the
induction hypothesis that $\Mgamma{T_1}\in U$ and that there is an $i$
such that 
the function $F$ mapping $z$ in $\Mgamma{T_1}$ to
$\mathcal{M}(\Gamma, x: \ttvar~T_1\vbar \gamma\{x\gets z\})(\tupletype{x_2:
T_2, \ldots x_n: T_n})$  is in $W_i$\@.  Since, by
Definition~\ref{defn:meaning function with dependent types},
$\Mgamma{A} = \Sigma{F}$,
we have by Definition~\ref{defn:type universe with dependent types},
that $\Mgamma{A}\in U_{i+1}$ and hence $\Mgamma{A}\in U$\@.
%
\item $A\equiv \{y : T | a\}$: Same as in
Theorem~\ref{subtype-type-semantics}\@.
%
\item $B \equiv s$:  Let $j$ be the rank of $\Mgamma{A}$\@.  
Then $\mathcal{M}(\Gamma, x : \ttvar~A\vbar \gamma\{x\gets
z\})(B)$ is just $\gamma(s)$ and by Definition~ref{defn:satisfaction},
we have some $k$ such that $\gamma(s)\in U_k$\@.  Let $i$ be the maximum
of $j$ and $k$, we have $F\in U_i$\@.
%
\item $B\equiv \funtype{x: C}{D}$:  
\end{enumerate}
\end{proof}
\end{comment}

\section{Summary}

Dependent typing is a significant enhancement to PVS since it adds
an important degree of flexibility and precision to the type system.
Notions  such as subtype constraints and type equivalence
that were introduced for subtyping can be extended for the case of
dependent types.  The semantic universe must be extended to include
additional sets to accommodate the semantics of dependent types.
The rank-boundedness of type dependencies is crucial in demonstrating that
dependent types can be interpreted in this extended semantic universe.
  






\chapter{Theories and Parametric Theories}\label{theories}

The next extension of the PVS language introduces theories and parametric
theories.  
The theory construct of PVS provides a way of packaging together a related
collection of declarations.  Theories can be parametric in
individual or type parameters.  Thus, PVS permits polymorphism or type
parametricity only at the theory level rather than at the declaration
level as in HOL~\cite{Gordon&Melham:HOL}.  
We first consider PVS theories without parameters.  The main change now is
that contexts are no longer simple and can contain theory declarations as
well.  A theory declaration has the form $m : \tttheory{} = \Delta$,
where $\Delta$ is a \emph{simple context} with no variable or theory
declarations. 
If $\Gamma(m)$ is the declaration $m : \tttheory{} = \Delta$, then
$ \itkind{}(\Gamma(m)) = \tttheory{}$, and
$ \itdef{}(\Gamma(m)) = \Delta$\@.
Correspondingly, constants and type names are no longer just symbols
but can be compound names of the form $m.s$ where $m$ is a symbol
naming a theory and $s$ is a symbol corresponding to the constant or type
name.

\section{Theories without Parameters}

To define the type rules for theories, we first modify the definition of
$\tau$ for simple contexts so that the context argument is not always
empty.  Here $\Delta ; \Gamma$ represents the concatenation of contexts.
\begin{Defn}{type rules for contexts}
\begin{eqnarray*}
  %
  \tau(\Theta)(\{\}) & = & \ttcontext \\
%
  \tau(\Theta)(\Gamma, s\ :\ \tttype{} = T) & = & \ttcontext, \mbox{
if } \Gamma(s) \mbox{ and } \Theta(s) \mbox{ are undefined,}\\& & 
  \tau(\Theta)(\Gamma) = \ttcontext, \mbox{and} \\& &
  \tau(\Theta;\Gamma)(T) = \tttype{}\\
%
  \tau(\Theta)(\Gamma, c : T) & = & \ttcontext, \mbox{ if }
\Gamma(c) \mbox{ and } \Theta(c)  \mbox{ are undefined,}\\ & &
\tau(\Theta)(\Gamma) = \ttcontext, \mbox{and }\\
   & &  \tau(\Theta;\Gamma)(T) =
      \tttype{}\\
%
%
  \tau(\Theta)(\Gamma, x : \ttvar{}\ T) & = & \ttcontext, \mbox{ if }
\Gamma(x) \mbox{ and } \Theta(x) \mbox{ are undefined,} \\ & &
\tau(\Theta)(\Gamma) = \ttcontext, \mbox{and }\\
    & &  \tau(\Theta;\Gamma)(T)
    = \tttype{}
%
\end{eqnarray*}
\end{Defn}
\begin{Eg}{type rules for contexts}
\begin{eqnarray*}
\tau(\Omega)(\ttreal{} : \tttype{}, \mathtt{0} : \ttreal{},
\leq : \funtype{\tupletype{\ttreal{}, \ttreal{}}}{\ttbool}) & = & \ttcontext
\end{eqnarray*}
\end{Eg}

The following rule handles theory declarations.
\begin{Defn}{type rule for contexts with theory declarations}
\begin{eqnarray*}
  \tau(\Theta)(\Gamma, m : \tttheory{} = \Delta) & = & \ttcontext
     \mbox{ if } \Theta(m), \Gamma(m) \mbox{ are undefined} \\& &
     \Delta \mbox{ only has constant and type declarations,}\\ & &
     \tau(\Theta ; \Gamma)(\Delta)  = \ttcontext,\\& &
     \tau(\Theta)(\Gamma) = \ttcontext
\end{eqnarray*}
\end{Defn}
\begin{Eg}{contexts with theory declarations}
\begin{eqnarray*}
\lefteqn{\tau(\Omega)(
\mathtt{reals} : \tttheory{} = (\ttreal{} : \tttype{}, \mathtt{0} : \ttreal{},
\leq : \funtype{\tupletype{\ttreal{},
\ttreal{}}}{\ttbool})) }\\
& = & 
\ttcontext
\end{eqnarray*}

\end{Eg}


\begin{comment}
Perhaps each declaration should be required to introduce a new
symbol so that there is no confusion between $\Gamma$ and $\gamma$.
\end{comment}

\begin{comment}
Example:
bool: TYPE,
integer : THEORY = (integer: TYPE, + :
\funtype{\tupletype{integer, integer}}{ {\tt integer}}),
natural : THEORY = (nat : TYPE)
\end{comment}

Any reference to a type name or a constant $s$ declared in a theory $m$
outside of this theory must be prefixed by the theory name, as in $m.s$\@.
Note that references to a type name or constant that is declared in the
same theory 
should not be given a theory prefix.  Before we can give the type rules, we
must update the definition of the type expansion
operation $\delta$ to prefix symbols with their theory names.
Let $\Gamma(m)(s)$  abbreviate $ \itdef{}(\Gamma(m))(s)$, which is
the declaration of the symbol $s$ in the definition of the theory $m$\@.
Let $\eta(\Gamma, m)(a)$ be the result of prefixing every unprefixed type or
constant symbol in $a$ by $m$, where $a$ is either an individual or type
expression.  We omit the definition of $\eta$ since it is straightforward.
  
%The $\delta$ operation is modified so that its first argument is now a
%pair consisting of a context and the `current' theory name.  The theory
%name component can be empty, as in Definition~\ref{defn:expanded type}\@.
We modify the definition of $\delta$ in Definition~\ref{defn:expanded
type} with the following clauses.  
\begin{Defn}{expanded type for prefixed symbols}
\begin{eqnarray*}
  \delta(\Gamma)(m.s) & = & \delta(\Gamma)(
\eta(\Gamma, m)(\itdef{}(\Gamma(m)(s)))), 
                        \mbox{ if }\\& &  \itdef{}(\Gamma(m)(s)) \mbox{ is
nonempty.}\\
  \delta(\Gamma)(m.s) & = & m.s
                        \mbox{ if } \itdef{}(\Gamma(m)(s)) \mbox{ is
empty.}
%\\
% \delta(\Gamma, m)(s) & = & m.s \mbox{ if } 
%\itdef{}(\Gamma(m)(s)) \mbox{ is
%empty.  }
\end{eqnarray*}
\end{Defn}
\begin{Eg}{expanded type for prefixed symbols}
Let $\Omega''$ be the context
$$\Omega, 
\mathtt{reals} : \tttheory{} = \begin{array}[t]{l}
				   (\ttreal{} : \tttype{},\\\
                                    \mathtt{0} : \ttreal{},\\\ 
                        \leq : \funtype{\tupletype{\ttreal{},
              \ttreal{}}}{\ttbool}, \\\ 
        \mathtt{nonneg\_real} : \tttype{} = \{x : \ttreal{} \vbar
\leq(0, x)\},\\\  \mathtt{1} : \mathtt{nonneg\_real} )
				   \end{array}$$

$$\delta(\Omega'')( \mathtt{reals.nonneg\_real}) =
\{ {x} : \mathtt{reals.real} \vbar
\mathtt{reals}.\!\leq( \mathtt{reals.0}, {x})\}$$
\end{Eg}

The type rules for prefixed symbols are given below.
\begin{Defn}{type rules for prefixed symbols}
\begin{eqnarray*}
  \tauGamma{m.s} & = & \tttype{}, \mbox{ if } \itkind{}(\Gamma(m)) =
\tttheory{} \mbox{ and }\\ & &
\itkind{}(\Gamma(m)(s)) = \tttype{}\\
%
  \tauGamma{m.s} & = & \delta(\Gamma)(\eta(\Gamma, m)(\ittype(\Gamma(m)(s)))), \\ &
&
\mbox{ if } \itkind{}(\Gamma(m)) =
\tttheory{} \mbox{ and }\\ & &
\itkind{}(\Gamma(m)(s)) = \ttconstant{}
\end{eqnarray*}
\end{Defn}

\begin{Eg}{type rules for prefixed symbols}
\begin{eqnarray*}
\tau(\Omega'')(\mathtt{reals.nonneg\_real}) & = & \tttype{}\\
\tau(\Omega'')(\mathtt{reals.1}) & = &
\{ {x} : \mathtt{reals.real} \vbar
\mathtt{reals}.\!\leq( \mathtt{reals.0}, {x})\}
\end{eqnarray*}
\end{Eg}

\begin{comment}
\begin{Eg}{contexts with theory declarations}
\begin{eqnarray*}
\tau(\Omega)(
\begin{array}[t]{l}
\mathtt{reals} : \tttheory{} = (\ttreal{} : \tttype{}, \mathtt{0} : \ttreal{},
\leq : \funtype{\tupletype{\ttreal{},
\ttreal{}}}{\ttbool}), \\
\mathtt{nonneg\_reals}: \tttheory{}  = \\\ \
   ( \mathtt{nonneg\_real} :
\tttype{} = \{ {x} : \mathtt{reals.real} \vbar
\mathtt{reals.\leq}( \mathtt{reals.0}, {x})\},\\\ \  \mathtt{sqrt}: \funtype{
\mathtt{nonneg_real}}{ \mathtt{nonneg_real}}\} )) 
\end{array}
& = & 
\ttcontext
\end{eqnarray*}
%
\end{Eg}
\end{comment}




The operations  $\pi$, and $\mu$ remain unchanged.
  An assignment $\gamma$ now
maps a theory name $m$ to an assignment $\gamma(m)$.
\begin{Defn}{meaning function for prefixed symbols}
\begin{eqnarray*}
  \Mgamma{m.s} & = & \gamma(m)(s)
\end{eqnarray*}
\end{Defn}
\begin{Eg}{meaning function for prefixed symbols}
Let $\omega''$ be a satisfying assignment for $\Omega''$ of the form
$$\ldots \{ \mathtt{reals}\gets \{\ttreal{}\gets
\reals\}\{\mathtt{0}\gets \zerob\}\ldots \} \ldots.$$

\begin{eqnarray*}
\mathcal{M}(\Omega''\vbar\omega'')(\mathtt{reals.real}) & = & \reals\\
\mathcal{M}(\Omega''\vbar\omega'')(\mathtt{reals.0}) & = & \zerob
\end{eqnarray*}
\end{Eg}



\begin{Defn}{satisfaction for contexts with theories}
An assignment~$\gamma$ satisfies a context $\Gamma$ if in addition to the 
constraints stated in
Definition~\ref{defn:satisfaction with type definitions}, $\gamma$ maps 
every theory $m$  declared in $\Gamma$ to 
a satisfying assignment for the body of the theory given by $
\itdef{}(\Gamma(m))$, that is for each declared symbol $s$ in $m$:
\begin{enumerate}
\item If $ \itkind{}(\Gamma(m)(s)) = \tttype{}$, then $\gamma(m)(s)
\in U$.


\item If $\itkind{}(\Gamma(m)(s)) = \ttconstant{}$, then $\gamma(m)(s)
\in \mathcal{M}(\Gamma\vbar\gamma)(\tau(\Gamma)(m.s))$.

\item  If  $\itdef{}(\Gamma(m)(s))$ is nonempty,
then $$\gamma(m)(s) = \mathcal{M}(\Gamma| \gamma)(
\eta(\Gamma, m)(\itdef{}(\Gamma(m)(s)))).$$
\end{enumerate}
\end{Defn}


\section{Constant Definitions}  We first extend the subset of PVS
described so far to include 
constant definitions in a manner similar to type definitions.
This extension is used in formalizing the semantics of parametric theories.
The syntax
for a constant definition is $c : T = a$ where $\itdef{}(\Gamma(c))$ is
$a$\@.  These definitions are explicit, that is, not recursive.  With this
extension, the type rule for constant declarations in contexts changes
from that of Definition~\ref{defn:type rules with subtypes}.
\begin{Defn}{type rule with constant definitions}
\begin{eqnarray*}
\tau(\Theta)(\Gamma, c : T = a) & = & T, \mbox{ if } \Gamma(c) \mbox{ is
undefined}, \\& & \Theta(c) \mbox{ is undefined}, \\& &
\tau(\Theta)(\Gamma) = \ttcontext,\\& &
%\vdash_\Gamma (\exists(x : T) : \tttrue),\\& &
\tau(\Theta;\Gamma)(a) = T',\\& & 
(T\sim T')_\Gamma,\\& &
\vdash_\Gamma \pi(T)(a)
\end{eqnarray*}
\end{Defn}

The notion of satisfaction must be extended from that of
Definition~\ref{defn:satisfaction for contexts with theories} to ensure
that an 
assignment 
for a defined constant satisfies the definition.
\begin{Defn}{satisfaction with constant definitions}
An assignment $\gamma$ satisfies a context $\Gamma$ if
in addition to the conditions in
Definition~\ref{defn:satisfaction for contexts with theories}, whenever $
\textit{kind}(\Gamma(s)) = \ttconstant$ and 
$\textit{definition}(\Gamma(s))$ is nonempty, then
$\gamma(s) = \Mgamma{ \textit{definition}(\Gamma(s))}$.
\end{Defn}


\section{Parametric Theories}

The extension to parametric theories is obtained by permitting
theories to be declared as $ m[\Pi] : \tttheory{} = \Delta$,
where $\Pi$ is a context listing the parameters and $\Delta$ is
the body of the theory.  If the above declaration of $m$ occurs
in context $\Gamma$, then $\Pi$ is $ \emph{formals}(\Gamma(m))$, and
$\Delta$ is $ \itdef{}(\Gamma(m))$.
For nonparametric theories, $ \mathit{formals}(\Gamma(m))$
is empty.   
Types or constants declared in a parametric theory are
referenced outside the theory as $m[\sigma].s$, where $\sigma$
is a list of \emph{actual} parameters consisting of types and terms.  
The type rule from the nonparametric case
must be modified to check the parameters.
\begin{Defn}{type rule for contexts with parametric theories}
\begin{eqnarray*}
  \lefteqn{\tau(\Theta)(\Gamma, m[\Pi] : \tttheory{} = \Delta)}\\ & = & \ttcontext
     \mbox{ if } \Gamma(m), \Theta(m), \Pi(m) \mbox{ are undefined}\\& &
      \tau(\Theta)(\Gamma) = \ttcontext\\& &
      \tau(\Theta ; \Gamma)(\Pi)  = \ttcontext,  \\ & & 
      \Pi \mbox{ has only constant and}\\& & \mbox{type declarations
without definitions,}\\ & &
      \tau(\Theta ; \Gamma ; \Pi)(\Delta)  = \ttcontext \\& &
      \mbox{ $\Delta$ only has type and constant declarations}
\end{eqnarray*}
\end{Defn}

  The 
 type  rules for prefixed symbols are given below.
The notation $\Pi = \sigma$, where $\Pi$ is of the form
$s_1 : \alpha_1,\ldots, s_n : \alpha_n$, and $\sigma$ is of
the form $\sigma_1,\ldots, \sigma_n$, is short for
the context $s_1 : \alpha_1 = \sigma_1, \ldots, s_n : \alpha_n =
\sigma_n$\@.  The definition of $\eta$ is now extended to substitute
actual theory parameters for formals, so that $\eta(\Gamma, m[\sigma])(a)$
prefixes every unprefixed symbol $s$ in $a$ that is declared in
$\itdef(\Gamma(m))$ by $m[\sigma]$, and replaces any $s_i$ in $a$
that is declared in $\itformals(\Gamma(m))$ by the corresponding
$\sigma_i$ in $\sigma$\@.  

\begin{Defn}{type rules for prefixed names with actuals}
Let $\Pi$ be $\mathit{formals}(\Gamma(m))$\@.  
\begin{eqnarray*}
  \tauGamma{m[\sigma].s} & = & \tttype{}, \mbox{ if }\\& & 
\itkind{}(\Gamma(m)) =  \tttheory{} \\ & &
\itkind{}(\Gamma(m))(s) = \tttype{} \mbox{ and }\\& &
\tauGamma{\Pi = \sigma} = \ttcontext\\
%
  \tauGamma{m[\sigma].s} & = & \delta(\Gamma)((\eta(\Gamma, m[\sigma])(
\ittype(\Gamma(m)(s)) )), \\ & &
\mbox{ if } \itkind{}(\Gamma(m)) = \tttheory{} \\ & &
\itkind{}(\Gamma(m)(s)) = \ttconstant{} \mbox{ and }\\& &
\tauGamma{\Pi = \sigma} = \ttcontext
\end{eqnarray*}
\end{Defn}

\begin{comment}
The definition of $\delta$ has to be modified slightly to carry along the
parameters.  The notation $a[\Pi]$ represents the result of
replacing each occurrence of a symbol $s$ in $a$ such that $\Pi(s)$ is
defined, by $\itdef{}(\Pi(s))$ when it exists.
\end{comment}
\begin{Defn}{type expansion with parametric theories}
\begin{eqnarray*}
  \delta(\Gamma)(m[\sigma].s) & = & \delta(\Gamma)(
(\eta(\Gamma, m[\sigma])(\itdef{}(\Gamma(m)(s)))) ), 
                        \mbox{ if }\\ & &  \itdef{}(\Gamma(m)(s)) \mbox{ is
nonempty.}\\
  \delta(\Gamma)(m[\sigma].s) & = & m[\sigma].s, 
                        \mbox{ if } \itdef{}(\Gamma(m)(s)) \mbox{ is
empty.}
%\\
% \delta(\Gamma, m[\sigma])(s) & = & \delta(\Gamma)( \itdef{}((\Pi
%= \sigma)(s))) \mbox{ if } \\& &
% \Pi(s) \mbox{ is defined}\\
% \delta(\Gamma, m[\sigma])(s) & = & m[\sigma].s \mbox{ if } \\& &
%\itdef{}(\Gamma(m)(s)) \mbox{ is
%empty.  }
\end{eqnarray*}
\end{Defn}

The definition of an assignment for a context with parametric theories is
a bit complicated.  
In the nonparametric case, $\gamma(m)$ simply returns an
assignment of values for the types and constants declared in the theory
$m$\@.  For the case of parametric theories $m$, $\gamma(m)$  returns a
function that maps the meaning of the given actuals $\sigma$ to
an assignment $\gamma(m)(\Mgamma{\sigma})$ for the  types and
constants declared in the theory $m$\@.   
There is an important restriction that $\gamma(m)$ must be
\emph{rank-preserving}, that is,  if $\varpi$  
and $\varpi'$ are assignments for 
$\Pi$ so that for each $i$ where $\Pi_i$ is a type parameter, the
rank of $\varpi(\Pi_i)$ equals the rank of 
$\varpi'(\Pi_i)$,  then the ranks of $\gamma(m)(\varpi)(s)$
and $\gamma(m)(\varpi')(s)$ must be the same for each type symbol $s$
declared in $m$\@.  
%\memo{This needs to be stated more directly.}

It is also important to observe that the semantics of parametric theories
makes use of the axiom of choice since the assignment corresponding to a
theory $\mathtt{m}$ of the form  $\mathtt{m [t:TYPE]: THEORY =
\{c : t\}}$ is essentially a choice function.  

Let $\gamma\{\Pi \gets \varpi\}$ represent the assignment such that
$\gamma\{\Pi \gets \varpi\}(s) = \varpi(s)$ for $s$
in the domain of the context $\Pi$, and $\gamma(s)$, otherwise\@.  
The meaning of symbols of the form $m[\sigma].s$ can then be defined
as below.
\begin{Defn}{meaning function for prefixed symbols with actuals}
\begin{eqnarray*}
  \lefteqn{\Mgamma{m[\sigma].s}}\\ & = &
\mathcal{M}(\Gamma;\Pi;\Delta \vbar
\gamma\{\Pi\gets\varpi\}\{\Delta\gets\gamma(m)(\varpi)\})(s), \mbox{ where}\\
 & & \Pi = \mathit{formals}(\Gamma(m)) \\
& &      \Delta = \mathit{definition}(\Gamma(m)) \\
& &      \varpi(r) = \Mgamma{(\Pi = \sigma)(r)}, \mbox{ for } r\in \Pi
\end{eqnarray*}
\end{Defn}

The definition of a satisfying assignment given in
Definition~\ref{defn:satisfaction for contexts with theories} 
also must be strengthened.    
Let $\Pi$ be the
formal parameters to theory $m$ in context $\Gamma$; then, an assignment
$\varpi$ is 
said to be satisfying {\em parameter assignment} for $\Pi$ under the
assignment $\gamma$ to $\Gamma$ 
iff $\gamma\{\Pi\gets \varpi\}$ is a satisfying assignment for $\Pi$\@.  
\begin{Defn}{satisfaction for contexts with parametric theories}
An assignment
$\gamma$ satisfies a context $\Gamma$ if in addition to the 
constraints stated in
Definition~\ref{defn:satisfaction for contexts with theories}, $\gamma$ maps 
every parametric theory $m$  declared in $\Gamma$ with parameters $\Pi$
and definition $\Delta$, 
to a function that maps
any satisfying parameter assignment $\varpi$ for the theory parameters $\Pi$
(namely, 
$\itformals(\Gamma(m))$) to a satisfying assignment $\gamma\{\Pi\gets
\varpi\}\{\Delta \gets \gamma(m)(\varpi)\}$ for 
$\Delta$ (given by 
$\itdef{}(\Gamma(m))$).
\begin{comment}
That is, 
for each declared symbol $s$ in
$\Delta$: 
\begin{enumerate}
\item If $ \itkind{}(\Gamma(m)(s)) = \tttype{}$, then
$\gamma(m)(\Mgamma{\sigma})(s)
\in U$
%
\item if $\itkind{}(\Gamma(m)(s)) = \ttconstant{}$, then $\gamma(m)(\Mgamma{\sigma})(s)
\in \mathcal{M}(\Gamma| \gamma;\{\Pi\gets \Mgamma{\sigma}\};\gamma(m)(\Mgamma{\sigma})(\tau(\Gamma)(m[\sigma].s)))$
%
\item if $\itdef{}(\Gamma(m)(s))$ is defined,
then $\gamma(m)(\Mgamma{\sigma})(s) = \mathcal{M}(\Gamma| \gamma;
\{\Pi\gets \Mgamma{\sigma}\}; \gamma(m)(\Mgamma{\sigma}))(
\itdef{}(\Gamma(m)(s)))$.
\end{enumerate}
\end{comment}
\end{Defn}


\begin{comment}
This map preserves the
assignments given to the parameters so that if for $s : T$ in $\Pi$,
$m[\sigma]$ is such that the formal $s$ is matched with the actual $a$,
then 
$\gamma(m)(\{\Pi\gets \Mgamma{\sigma}\})(s)$ must be $\Mgamma{a}$\@.
\end{comment}
\begin{comment}
The delicate point here is that $\gamma(m)$ is not a set but a map (a
rule) that maps a given assignment (a set) of sets to symbols, to another
assignment of sets to symbols.  This means that $\gamma$ itself is not a
set.  There is no reason why $\gamma$ should be a set since all we require
is that it map a given type name to a set, and a constant to an element
of the set representing its type.  
\end{comment}
\begin{comment}
\begin{eqnarray*}
  \Mgamma{m[\sigma].s} & = & \gamma(m)(\{\Pi\gets \Mgamma(\sigma)\})(s)
\end{eqnarray*}
\end{comment}

\section{Summary}

Theories are used to package related declarations together.  Parametric
theories can be used to package together declarations that are generic in
type and individual parameters.  The type rules for contexts must be
extended to accommodate the theories.  The type rules for simple
(nonparametric) theories are straightforward given this extension.  The
operation of expanding a type using type definitions must be enhanced so
that symbols declared in a theory are prefixed with their theory name when
referenced outside the theory.  Assignments now have the same nested
structure as contexts, and the semantic definition is easily extended to
handle prefixed symbols.  Parametric theories are more complex.  The
theory prefixes now contain actual parameters that have to be typechecked
relative to the expected formal parameters.  The assignments corresponding
to parametric theories are functions that map given assignments for the
formals to assignments for the declarations within a theory.  Such a
mapping must be constrained to be rank-preserving.  Parametric theories
can have subtype parameters, and assumptions on the parameters.  The rules
for subtype parameters and assumptions are omitted for now but will be
included in an expanded version of this report.



\chapter{Conditional Expressions and Logical Connectives}\label{conditionals}

We have, so far, introduced the core of PVS containing types, type
definitions, constant and variable declarations, subtypes, dependent
types, and theories.  In extending the language with
both explicit and recursive constant definitions and formulas,  a crucial
difference  is that the logical context under which a type-correctness
condition is generated provides additional assumptions that can be used in
proving any proof obligations.  Examples of expressions where an
extended context is needed to establish type correctness by discharging
proof obligations include
\begin{enumerate}
\item $x \neq y \supset  (x + y)/(x - y) \leq 0$.  The type of the
division operator constrains the denominator to be nonzero,
that is, $\{x : \ttreal \vbar x \neq 0\}$\@.    
In the given expression,  the denominator
can be shown to be nonzero only in the context of the antecedent $x \neq
y$\@.

\item $\mathtt{IF}(i > 0, i, -i)$ has type $\ttnat$ given integer $i$
provided the {\em then\/} and {\em else\/} parts are typechecked with the
assumptions $i > 0$ and $\neg (i > 0)$, respectively.
\end{enumerate}

PVS has a polymorphic primitive equality predicate:
\begin{display}
equality[T : TYPE] : THEORY = \{ =: [[T, T] -> bool] \}
\end{display}
Note that an equality of the form $ \mathtt{equality}[T].\mathtt{=}(a, b)$ is
informally written as $a = b$\@.  When it is relevant to indicate the type
parameter, we write the equality as $a =_{T} b$\@.  It can be deduced from
the meaning of equality that if $S$ is a subtype of $T$, then for $a$ and
$b$ in $S$, it must be the case that $a =_{S} b$ iff $a =_{T} b$\@.  Thus,
we can assume that equality is always parameterized by a maximal type.  We
assume that any relevant context $\Gamma$ contains the above declaration
of the theory {\tt equality}.  Furthermore, any satisfying assignment
$\gamma$ for such a $\Gamma$ must satisfy
\begin{eqnarray*}
\gamma( \mathtt{ equality})(X)(=) &  = & \{\pair{x, x}\vbar x\in X\}.
\end{eqnarray*}

The negation operation can be defined in terms of equality as shown below.
We assume that the context contains a declaration of the form
\begin{alltt}
   \(\neg : \funtype{\ttbool}{\ttbool} = (\lambda (x : \ttbool): x = \ttfalse)\)
\end{alltt}

As is clear, a satisfying assignment $\gamma$ for a context $\Gamma$
containing the above declaration must be such that $\gamma(\neg)$ yields
the usual truth-table semantics, that is, 
$\{ \pair{\zerob, \oneb}, \pair{\oneb, \zerob} \}$\@.  

We can then introduce the polymorphic \texttt{IF-THEN-ELSE} operation
as follows: 
\begin{display}
if_def [T: TYPE]: THEORY = \{ IF:[bool,T,T -> T] \}
\end{display}

In typechecking conditional expressions, the notion of context
has to be extended to include formulas so that the typechecking
of the subterm $b$ in $\mathtt{IF}(a, b, c)$ is done in the context of $a$,
and the typechecking of $c$ is done in the context of $\neg a$\@.
There is one new typechecking rule for contexts with formulas.
\begin{eqnarray*}
\tau()(\Gamma, a) & = & \ttcontext, \mbox{ if}\\
& & \tau()(\Gamma) = \ttcontext, \mbox{ and} \\
& & (\tau(\Gamma)(a)   \sim \ttbool)_\Gamma
\end{eqnarray*}
Note that the type rule checks that the type of $a$ is compatible with
$\ttbool$ rather than equivalent to it since it is possible that the  type
of $a$ might be a subtype of $\ttbool$\@.  


\begin{Defn}{satisfaction for contexts with formulas}
An assignment $\gamma$ satisfies context $\Gamma$ when in 
addition to the conditions in
Definition~\ref{defn:satisfaction for contexts with parametric theories},  
for each prefix $\Gamma', a$ of $\Gamma$, $\mathcal{M}(\Gamma'\vbar \gamma)(a)
= \oneb$\@.  
\end{Defn}

The typechecking of conditional expressions is different from that of
other application expressions since the \emph{test} part of the
conditional expression is introduced into the context as a contextual
assumption.
\begin{Defn}{type rule for conditional expressions}
   \begin{eqnarray*}
     \tauGamma{ \mathtt{ if\_def[\mathit{T}].IF}(a, b, c)} & = & T, \mbox{ if }
(\tauGamma{a} \sim  \mathtt{bool})_\Gamma,\\ & &
   \tau(\Gamma, a)(b) = B,\\& &  (B \sim T)_{\Gamma, a}, \\& & \vdash_{\Gamma,
a}\pi(T)(b)\\& & 
    \tau(\Gamma, \neg a)(c) = C,\\& &  (C \sim T)_{\Gamma, \neg a},\\ & & 
\vdash_{\Gamma, \neg a} \pi(T)(c)
\end{eqnarray*}
\end{Defn}

The meaning of conditional expressions must be treated in a special way
since the \emph{else} part need not denote when the \emph{test} part is
true and, correspondingly, the \emph{then} part need not denote if
the \emph{test} part is false.  We assume that any relevant contexts $\Gamma$
contain the above declaration of the \texttt{if\_def} theory.
Conditional expressions can be regarded as a
new construct in the language rather than a form of application.
However, it is conservative to regard conditional expressions as
applications since the latter introduce the additional constraint that all
the arguments must already denote, that is, applications are
\emph{strict}.
\begin{Defn}{meaning function for conditional expressions}
\begin{eqnarray*}
\Mgamma{\mathtt{if\_def}[T].\ttif(a, b, c)} & = &
 \left\{\begin{array}{ll}
	\Mgamma{b}, & \mbox{ if } \Mgamma{a} = \oneb\\
        \Mgamma{c}, & \mbox{ otherwise}
	\end{array}\right.
\end{eqnarray*}
\end{Defn}

The semantics for conditional expressions raises an important issue.
The equality
$$\mathtt{if\_def}[\mathtt{bool}].\ttif(x, y, \ttfalse) =
\mathtt{if\_def}[\mathtt{bool}].\ttif(y, x, \ttfalse)$$
is semantically valid for variables $x$ and $y$ of type $\ttbool$\@.
An expression like $\mathtt{if\_def}[\ttbool].\ttif(i \neq 0, 1/i
> 0, \ttfalse)$ can be typechecked to have the type $\ttbool$ since
it generates a valid proof obligation $i\neq 0 \supset i\neq 0$, but
the seemingly equivalent expression
$\mathtt{if\_def}[\ttbool].\ttif( 1/i > 0,i \neq 0, \ttfalse)$
generates an unverifiable proof obligation $i\neq 0$\@.
This may seem contradictory since the equality suggests a transformation
of a type correct conditional expression to a type incorrect expression.  
The resolution here is that equality cannot be instantiated with
$i\neq 0$ for $x$ and $1/i > 0$ for $y$ since the expression
$1/i > 0$  typechecks as having type $\ttbool$ only when $i\neq 0$ is known
from the context.  The same applies in the case of the other
propositional connectives, thus ensuring that each expression is type
correct in the context in which it occurs.  

We can then define the propositional connectives in terms of  conditional
expressions. 
\begin{eqnarray*}
  \wedge : \funtype{\tupletype{\ttbool, \ttbool}}{\ttbool} & = & \lambda (x : \ttbool, y
: \ttbool) : \mathtt{if\_def[\ttbool].IF}(x, y, \ttfalse)\\
   \vee : \funtype{\tupletype{\ttbool, \ttbool}}{\ttbool} & = & \lambda (x : \ttbool, y
: \ttbool) : \mathtt{if\_def[\ttbool].IF}(x, \tttrue, y)\\
   \supset : \funtype{\tupletype{\ttbool, \ttbool}}{\ttbool} & = & \lambda (x : \ttbool, y
: \ttbool) : \mathtt{if\_def[\ttbool].IF}(x, y, \tttrue)\\
%  \neg : \funtype{\ttbool}{\ttbool} & = & \lambda (x : \ttbool) :
%\mathtt{if\_def[\ttbool].IF}(x, \ttfalse, \tttrue)
\end{eqnarray*}

In the typechecking of terms of the form $a \wedge b$,
we follow the corresponding rule for the definition so that 
the term $a$ is assumed in the context when typechecking term $b$\@.
Similarly, for $a\vee  b$, the formula $\neg a$ is assumed in
the context when typechecking $b$, and for $ a \supset b$,
the formula $ a$ is assumed in the context when typechecking $b$.
The Boolean equivalence operator {\tt IFF} has no special rules
for adding formulas to contexts during typechecking.   

\section{Summary}

The use of assumption formulas  enables expressions
to be typechecked within the narrow context of their use so that
the governing assumptions can be used in discharging any
proof obligations.  The type rules for conditional expressions and the
Boolean connectives $\wedge$, $\vee$, and $\supset$ make use of
contextual assumptions.  

\begin{comment}
We will assume that the context $\Gamma$ used below contain the
theories \mathtt{equality} and \mathtt{if\_def}\@.  
The typechecking of constant definitions is straightforward when
the definitions are not recursive.  As we have already seen, the
form of an explicit definitions is $ s : T = a$, where $s$ is a symbol,
$T$ is a type, and $a$ is a term.  
When a constant is defined, no nonemptiness TCCs are generated since
there is a definition present.
%
\begin{eqnarray*}
  \tauGamma{\Delta, s : T = a} & = & \mathtt{CONTEXT}, \mbox{ if }
    \tauGamma{\Delta} = \mathtt{CONTEXT} \\ & &
   \tau(\Gamma, \Delta)(T)  = \mathtt{TYPE}\\ & &
   \tau(\Gamma, \Delta)(a) = T', \mbox{ and}\\ & &
   (T\sim_\Gamma T')_{\Gamma, \Delta}\\& &
   \vdash_{\Gamma, \Delta} \pi(T)(a)
\end{eqnarray*}
%
A recursive constant definition has the syntactic form
$ s : RECURSIVE\ T = a\ MEASURE\ f RELATION p$\@.
%
\memo{This looks slightly complicated. Postpone for now.}
%
A formula declaration has the form
$s : \ \mathtt{FORMULA} a$.
%
\memo{Need to formally defined substitution, freevars, and closure
relative to a context.}  
\end{comment}

\begin{comment}
%
\memo{Need to highlight the shape of a context: theory declarations
followed by type/constant declarations followed by variable declarations.}
%
Introduce equality, if-then-else, connectives, updates.
%
\section{Abstract Datatypes}
\end{comment}


\chapter{Proof Theory of PVS}\label{proof theory}

The final step in the presentation of the semantics is the presentation of
the proof rules for the idealized subset of PVS described thus far.
As already indicated, the proof theory is an integral part of the
semantics since typechecking and proof checking are closely intertwined.
Fortunately, the proof rules turn out to be much less complicated than the
type rules.

\newcommand{\sequent}[3]{#1\vdash_{#2}#3}

The PVS proof theory is presented in terms of  a sequent calculus.  A
sequent is of the form 
$\sequent{\Sigma}{\Gamma}{\Lambda}$, where $\Gamma$ is the context,
$\Sigma$ is a {\em set\/} of {\em antecedent\/} formulas, and $\Lambda$ 
is a {\em set\/} of {\em consequent\/} formulas.  Such a sequent should
be read as stating that the conjunction of the formulas in $\Sigma$ implies the
disjunction of formulas in $\Lambda$\@.

% presumed in math mode #1=top, #2=bottom, #3=side condition
\renewcommand{\Infrule}[3]{
{{\displaystyle\strut #1}\over{\displaystyle\strut #2}}~\makebox[0pt][l]{{$\mathbf{ #3}$}}
}

Inference rules are presented in the form
\begin{center}
$\Infrule{\mathit{premise(s)}}{\mathit{conclusion}}{\mathit{name}}\hspace{2cm}
\mathit{side\ condition}$
\end{center}


\section{PVS Proof Rules} 

\subsection{Structural Rules}

The structural rules permit the sequent to be rearranged or weakened
via the introduction of new sequent formulas into the conclusion.  
All the structural rules can be expressed in terms of the single
powerful weakening rule  shown below.
It allows a weaker statement to
be derived from a stronger one by adding either antecedent formulas or
consequent formulas.  The relation $\Sigma_1\subseteq\Sigma_2$ holds
between two lists when all the formulas in $\Sigma_1$ occur in the list
$\Sigma_2$.  

\begin{center}
$\Infrule{\sequent{\Sigma_1}{\Gamma}{\Lambda_1}}
         {\sequent{\Sigma_2}{\Gamma}{\Lambda_2}} {W}$
\hspace{1cm}\mbox{\smaller\smaller if $\Sigma_1\subseteq\Sigma_2$ and $\Lambda_1\subseteq\Lambda_2$}
\end{center}

Both the Contraction and Exchange rules shown below are absorbed by the
above  weakening rule {\bf W}\@.     The  {\em Contraction\/} rules
$\mathbf{C}\vdash$ and $\vdash\mathbf{C}$
allow multiple occurrences of the same sequent formula
to be replaced by a single occurrence. 
\begin{center}
\begin{tabular}{ccc}
$\Infrule{\sequent{a,a, \Sigma}{\Gamma}{\Lambda}}%
         {\sequent{a, \Sigma}{\Gamma}{\Lambda}} {C\vdash}$
& \hspace{1in} &
$\Infrule{\sequent{\Sigma}{\Gamma}{ a, a, \Lambda}}%
         {\sequent{\Sigma}{\Gamma}{ a, \Lambda}} {\vdash C}$
\end{tabular}
\end{center}

The {\em Exchange\/} rule asserts that the order of the formulas in the
antecedent and the consequent parts of the sequent is immaterial.  It can
be stated as

\begin{center}
\begin{tabular}{c@{\hspace{1in}}c}
$\Infrule{\sequent{\Sigma_1, b, a,  \Sigma_2}{\Gamma}{\Lambda}}
         {\sequent{\Sigma_1, a, b, \Sigma_2}{\Gamma}{\Lambda}} {X\vdash}$
&
$\Infrule{\sequent{\Sigma}{\Gamma}{\Lambda_1, b, a, \Lambda_2}}
         {\sequent{\Sigma}{\Gamma}{\Lambda_1, a, b, \Lambda_2}} {\vdash X}$
\end{tabular}
\end{center}

As seen above, 
{\em inference rules\/} have the general form
$$\Infrule{\Sigma_1\vdash\Lambda_1\quad\cdots\quad\Sigma_n\vdash\Lambda_n}
{\Sigma\vdash\Lambda}{R}.$$ This says that if we are given a leaf of a
proof tree of the form $\Sigma\vdash\Lambda$, then by applying the rule
named {\bf R}, we may obtain a tree with $n$ new leaves.

\subsection{Cut Rule}

The cut rule {\bf Cut} can be used to introduce a case split on a formula $a$
into a proof of a sequent $\sequent{\Sigma}{\Gamma}{\Lambda}$ so as to yield the
subgoals 
$\sequent{\Sigma, a}{\Gamma}{\Lambda}$ and
$\sequent{\Sigma}{\Gamma}{a, \Lambda}$, which can be seen  
as assuming $a$ along one branch and $\neg a$ along the other.
\begin{center}
$\Infrule{(\tau(\Gamma)(a) \sim \ttbool)_\Gamma \qquad \sequent{\Sigma,  a}{\Gamma}{\Lambda}\qquad\sequent{\Sigma}{\Gamma}{ a, \Lambda}}
         {\sequent{\Sigma}{\Gamma}{\Lambda}}{Cut}$
\end{center}

\subsection{Propositional Axioms}

The axioms rule \textbf{Ax} simply asserts that $a$ follows from $a$\@.  
%\begin{center}%{(\tauGamma{a}\sim \ttbool)_\Gamma}
$$\Infrule{}{\sequent{\Sigma, a}{\Gamma}{a,\Lambda}}{Ax}$$  

The next two rules assert that any sequent with either
an antecedent occurrence of $\ttfalse$ or a consequent
occurrence of $\tttrue$ is an axiom.

$$
\begin{array}{c@{\hspace{1in}}c}
\Infrule{}{\sequent{\Sigma, \mathtt{FALSE}}{\Gamma}{\Lambda}}{
\ttfalse \vdash} &
\Infrule{}{\sequent{\Sigma}{\Gamma}{ \mathtt{TRUE}, \Lambda}}{
 \vdash \tttrue}
\end{array}
$$



\subsection{Context Rules}

Certain formulas hold in a context simply because they are already
asserted in the context either as a formula or a constant definition.
$$
\begin{array}{c}
\Infrule{}{\sequent{}{\Gamma}{a}}{Context Formula} \hspace{1.5in}\mbox{if
$a$ is a formula in $\Gamma$}\hspace{.8in}\\
\Infrule{}{\sequent{}{\Gamma}{s = a}}{Context Definition} \hspace{1.5in}\mbox{if
$s : T = a$ is a constant definition in $\Gamma$}
\end{array}
$$

The context $\Gamma$ can be extended with antecedent formulas or
negations of consequent formulas using the following two rules.
$$\begin{array}{c@{\hspace{1in}}c}
\Infrule{\sequent{\Sigma, a}{\Gamma,
a}{\Lambda}}{\sequent{\Sigma, a}{\Gamma}{\Lambda}}{Context \vdash} &
\Infrule{\sequent{\Sigma}{\Gamma,
\neg a}{a, \Lambda}}{\sequent{\Sigma}{\Gamma}{a, \Lambda}}{\vdash Context}
\end{array}
$$

The following context-weakening rule is useful since it shows that
provability is monotonic with respect to the context.
$$\begin{array}{c}
\Infrule{\sequent{\Sigma}{\Gamma}{\Lambda}}{\sequent{\Sigma}{\Gamma'}{\Lambda}}{Context
W}  \hspace{1in}\mbox{if $\Gamma$ is a prefix of $\Gamma'$}
\end{array} 
$$

\begin{comment}
\subsection{Negation Rules}
%
These rules are straightforward since negation is eliminated by moving
the body of a negated antecedent formula to the consequent, and
conversely, by moving the body of a negated consequent formula to
the antecedent.
$$
\begin{array}{c}
\Infrule{\sequent{\Sigma}{\Gamma}{ a, \Lambda}}
         {\sequent{\Sigma,\neg a }{\Gamma}{\Lambda}} {\neg\vdash}\\[.5cm]
\Infrule{\sequent{\Sigma, a }{\Gamma}{\Lambda}}
         {\sequent{\Sigma}{\Gamma}{ \neg a, \Lambda}} {\vdash\neg}
\end{array}
$$
\end{comment}

\subsection{Conditional Rules}

The rules governing the elimination of \texttt{IF-THEN-ELSE} in a proof
are unusual since they augment the context with the test part or its
negation, as in the corresponding type rules.
$$
\begin{array}{c}
\Infrule{\sequent{\Sigma, a, b}{\Gamma, a}{\Lambda}\hspace{1cm}
        \sequent{\Sigma,  c}{\Gamma, \neg a}{a, \Delta}}
         {\sequent{\Sigma, \mathtt{IF}(a, b, c)}{\Gamma}{ \Lambda}}
                  {\mathtt{IF} \vdash}\\[.5cm]
\Infrule{\sequent{\Sigma, a}{\Gamma,
a}{b,\Lambda}\hspace{1cm}\sequent{\Sigma}{\Gamma,\neg a}{a, c, \Lambda}}
         {\sequent{\Sigma}{\Gamma}{{\mathtt{IF}}(a, b, c), \Lambda}}
                      {\vdash {\mathtt {IF}}}
\end{array}
$$

\subsection{Equality Rules}

The rules for equality can be stated as below.  The rules of
transitivity and symmetry for equality can be derived from these rules.
The notation $a[e]$ is used to highlight one or more 
occurrences of $e$ in the formula $a$ such that there are no free variable
occurrences in $e$\@.\footnote{We enforce an invariant on a sequent that
it must not contain any free variables.  This invariant is preserved by
each of the proof rules.}   
The notation $\Lambda[e]$ similarly highlights
occurrences of $e$ in $\Lambda$. 

\begin{center}
\begin{tabular}{c@{\hspace{1in}}c}
$\Infrule{}{\sequent{\Sigma}{\Gamma}{ a=a,\Lambda}}{Refl}$
&
$\Infrule{\sequent{a=b, \Sigma[b]}{\Gamma}{\Lambda[b]}}
         {\sequent{a=b, \Sigma[a]}{\Gamma}{\Lambda[a]}} {Repl}$
\end{tabular}
\end{center}

\subsection{Boolean Equality Rules}

The rule $\mathbf{Repl~\tttrue}$ asserts that an antecedent formula
$a$
can be treated as an antecedent equality of the form $a = \tttrue$, and
correspondingly, 
a consequent formula $a$ can be treated as an antecedent
equality of the form $a = \ttfalse$\@.

$$
\begin{array}{c@{\hspace{1in}}c}
\Infrule{\sequent{\Sigma[\tttrue], a}{\Gamma}{\Lambda[\tttrue]}}{
\sequent{\Sigma[a], a}{\Gamma}{\Lambda[a]}}{Repl~\tttrue} &
\Infrule{\sequent{\Sigma[\ttfalse], a}{\Gamma}{\Lambda[\ttfalse]}}{
\sequent{\Sigma[a]}{\Gamma}{a, \Lambda[a]}}{Repl~\ttfalse}
\end{array}
$$

The rule $\tttrue\mbox{-}\ttfalse$ asserts that $\tttrue$ and $\ttfalse$ are
distinct Boolean constants.
$$
\begin{array}{c}
\Infrule{}{\sequent{\Sigma, \tttrue = \ttfalse}{\Gamma}{\Lambda}}{\tttrue\mbox{-}\ttfalse}
\end{array}
$$
\begin{comment}
Four rules exist for simplifying equalities of the form $a =
\tttrue$ and $a = \ttfalse$ in  a sequent.
%
$$
\begin{array}{c@{\hspace{1in}}c}
\Infrule{\sequent{\Sigma, a}{\Gamma}{\Lambda}}{\sequent{\Sigma, a =
\tttrue}{\Gamma}{\Lambda}}{\tttrue\vdash} &
\Infrule{\sequent{\Sigma}{\Gamma}{a, \Lambda}}{\sequent{\Sigma}{\Gamma}{a =
\tttrue, \Lambda}}{\vdash\tttrue} \\
%
\Infrule{\sequent{\Sigma}{\Gamma}{a, \Lambda}}{\sequent{\Sigma, a =
\ttfalse}{\Gamma}{\Lambda}}{\ttfalse\vdash} &
\Infrule{\sequent{a, \Sigma}{\Gamma}{\Lambda}}{\sequent{\Sigma}{\Gamma}{a =
\ttfalse, \Lambda}}{\vdash\ttfalse} 
\end{array}
$$
\end{comment}

\subsection{Reduction Rules}

The reduction rules are equality rules (axioms) that provide the obvious
simplifications for applications involving lambda abstractions and
product projections.

$$
\begin{array}{c}
\Infrule{}{\sequent{}{\Gamma}{(\lambda (x : T):~a)(b) =
a[b/x]}}{\beta}\\[.5cm]
%
\Infrule{}{\sequent{}{\Gamma}{\proj{i}(a_1, a_2) =
a_i}}{\pi}
\end{array}
$$

\subsection{Extensionality Rules}

The extensionality rules are also equality rules for establishing equality
between two expressions of function or product type.  The extensionality
rule for functions, \textbf{FunExt}, introduces a Skolem constant $s$
to determine that two functions $f$ and $g$ are equal when the results of
applying them to an arbitrary argument $s$ are equal.  
$$
\begin{array}{c}
\Infrule{\sequent{\Sigma}{\Gamma, s : A}{(f~s) =_{B[s/x]} (g~s), \Lambda}}
{\sequent{\Sigma}{\Gamma}{f =_{\funtype{x:A}{B}} g,
\Lambda}}{FunExt} \hspace{1in} \Gamma(s) \mbox{ undefined}\\[.5cm]
\end{array}
$$

The extensionality rule for products asserts that two products are equal if
their corresponding projections are equal.
\begin{comment}
Note that the type
$\tupletype{\listwo{T'}{n}}$ is just $a/\tupletype{\listwo{T}{n}}$\@.
\end{comment}
$$
\begin{array}{c}
\Infrule{\sequent{\Sigma}{\Gamma}{\proj{1}(a) =_{T_1} \proj{1}(b),\Lambda}\qquad\sequent{\Sigma}{\Gamma}{\proj{2}(a) =_{T_2[(\proj{1}~a)/x]} \proj{2}(b),\Lambda}
}{\sequent{\Sigma}{\Gamma}{a =_{\tupletype{x : T_1 T_2}} b, \Lambda}}{TupExt}
\end{array}
$$

Recall that the quantifiers can be defined in terms of lambda abstraction
and equality so that $(\forall (x: T): a)$ is just $(\lambda (x: T): a) =
(\lambda (x : T): \tttrue)$\@.   Existential quantification $(\exists (x:
T): a)$  can easily be defined as $\neg (\forall (x: T): \neg a)$\@.
The proof rules for quantifiers can then be derived from the
rules $\beta$, $\mathbf{TupExt}$, and the equality rules\@.

\subsection{Type Constraint Rule }

We need a rule to introduce the type constraint on a term as an antecedent
formula of the given goal sequent.

$$
\Infrule{\tau(\Gamma)(a) = A\qquad\sequent{\pi(A)(a),\Sigma}{\Gamma}{\Lambda}}
{\sequent{\Sigma}{\Gamma}{\Lambda}}{Typepred}
$$

\section{Soundness of the Proof Rules}
\begin{prop}\label{meaning-preservation}
If $\Gamma$ is a prefix of $\Gamma'$,
$\tau()(\Gamma) = \tau()(\Gamma') = \ttcontext$,
$\gamma'$ is a
satisfying assignment for $\Gamma'$, and $\gamma = \gamma'\restriction\Gamma$ then
for any $a$ such that 
$\tauGamma{a} = \tau(\Gamma')(a)$,  it is the case that $\Mgamma{a} =
\mathcal{M}(\Gamma'\vbar\gamma')(a)$\@.  
\end{prop}

\begin{theorem}[soundness]\label{proof-soundness}
If $\tau()(\Gamma) = \ttcontext$ such that
%\begin{enumerate}
%\item $\Gamma$ contains no variable declarations %bogus(1.18.96)
%\item
for every formula $a$ in $\Sigma;  \Lambda$, $(\tauGamma{a} \sim
\ttbool)_\Gamma$,
%\end{enumerate}
and $\sequent{\Sigma}{\Gamma}{\Lambda}$ is provable, then
for any satisfying assignment
$\gamma$ for $\Gamma$, 
either there is a formula $b$ in $\Sigma$, such that $\Mgamma{b} = \zerob$
or a formula $c$ in
$\Lambda$, such that $\Mgamma{c} = \oneb$\@.      
\end{theorem}
\begin{proof}
The proof is by induction on the structure of the proof of
$\sequent{\Sigma}{\Gamma}{\Lambda}$.  Recall that this proof is actually
part of a simultaneous induction that includes the soundness of the type
rules relative to the semantic function, that is,
Theorems~\ref{dependent-type-semantics}
and~\ref{dependent-term-semantics}\@.   Specific invocations of the
soundness theorem occur in the proofs of
Theorem~\ref{subtype-term-semantics}
and Proposition~\ref{substitution-equivalence}\@.  
\begin{enumerate}
\item {\em Structural Rules\/}:   Since the subset
of formulas in the premise and the conclusion of these rules are the same,
the conclusion follows easily from the induction hypothesis.

\item {\em Cut\/}: By the semantic soundness of the type rules,
we have $\Mgamma{a} \in \twob$\@.   If $\Mgamma{a} = \zerob$, then
by the induction hypothesis on the second subgoal of the proof rule,
there must be some $b$ in $\Sigma$ such that $\Mgamma{b} = \zerob$
or a $c$ in $\Lambda$ such that $\Mgamma{c} = \oneb$\@.  The case
when $\Mgamma{a} = \oneb$ is symmetrical.

\item {\em Propositional Axioms\/}:  Obvious.

\item \emph{Context Rules}:
\begin{description}
\item[ContextFormula:]  If $\gamma$ satisfies $\Gamma$ and $a\in \Gamma$,
then $\Mgamma{a} = \oneb$\@.

\item[ContextDefinition:] If $\gamma$ satisfies $\Gamma$ and $s: T = a$
is a declaration in $\Gamma$, then by the definition of satisfaction,
$\Mgamma{s} = \Mgamma{a}$\@.

\item [Context $\vdash$: ] The argument is trivial when $\Mgamma{a} =
\zerob$.  Otherwise, $\gamma$ satisfies the extended context $\Gamma, a$,
and the conclusion follows from the induction hypothesis.

\item [$\vdash$ Context: ] Similar to $\textbf{Context} \vdash$ above.

\item [ContextW: ]  If $\gamma$ satisfies $\Gamma'$, then it also
satisfies $\Gamma$, and hence the proof.
\end{description}

\item \emph{Conditional Rules}: We only consider $ \mathtt{IF} \vdash$
since the $\vdash \mathtt{IF}$ proof is similar.  If $\Mgamma{
\mathtt{IF}(a, b, c)} = \zerob$, the conclusion follows  trivially.
Otherwise, If $\gamma$ satisfies
$\Gamma$, then $\Mgamma{a}\in \twob$.  If $\Mgamma{a} = \oneb$, then
$\Mgamma{b} = \oneb$.  The induction hypothesis on the subgoal
$\sequent{\Sigma, a, b}{\Gamma, a}{\Lambda}$ yields the desired
conclusion.  Similarly, if $\Mgamma{a} = \zerob$, we have $\Mgamma{c} =
\oneb$
and the induction hypothesis on the second subgoal yields the
desired conclusion.

\item \emph{Equality Rules}: The \textbf{Refl} rule is obvious. 
For the \textbf{Repl} rule, if $\Mgamma{a = b} = \zerob$, the conclusion
follows trivially.  Otherwise, $\Mgamma{a} = \Mgamma{b}$.  Hence, $\gamma$
satisfies the extended context $\Gamma, a = b$.  Then for each
$c[a]$ in $\Sigma[a]$ or $\Lambda[a]$, $\Mgamma{c[a]} = \Mgamma{c[b]}$\@.
\begin{comment}
Note: no free variables in a, b.  Need to show that
replacement preserves meaning when there is no capture.
\end{comment}

\item \emph{Boolean Equality Rules}: The $\mathbf{Repl}~\tttrue$ and
$\mathbf{Repl}~\ttfalse$ rules follow easily since when $\Mgamma{a} =
\oneb$, we have $\Mgamma{c[a]} = \Mgamma{c[\tttrue]}$\@.
A similar argument applies to $\mathbf{Repl}~\ttfalse$\@.

The soundness of $\tttrue\mbox{-}\ttfalse$ is easy since
$\Mgamma{\tttrue = \ttfalse} = \zerob$\@.

\item \emph{Reduction Rules}:  The $\beta$-reduction rule follows because
$\Mgamma{(\lambda (x : T):~a)(b)}$ is
$\mathcal{M}(\Gamma, x : \ttvar~T\vbar \gamma\{x\gets\Mgamma{b}\})(a)$
which by the Substitution Lemma~\ref{substitution-lemma} is
equal to $\Mgamma{a[b/x]}$\@.

The soundness $\pi$-reduction rule is a direct consequent of
Definition~\ref{defn:meaning function}\@.

\item \emph{Extensionality Rules}:
\begin{description}
\item [FunExt:] 
First consider the case when the domain type $\Mgamma{A}$ is empty.  Then by
Definition~\ref{defn:meaning function with dependent types},
$\Mgamma{f} = \Mgamma{g} = \emptyset$\@.  Therefore $\Mgamma{f = g} =
\oneb$ and hence the conclusion.\footnote{Since
the  subgoal sequent $\sequent{\Sigma}{\Gamma, s:A}{(f~s) =
(g~s),\Lambda}$
is valid when $\Mgamma{A} = \emptyset$ for all assignments $\gamma$,
it is natural to ask how it is actually proved.  The only
way a type $A$ can be empty under any assignment $\gamma$ is if
$\Mgamma{\pi(A)(a) = \zerob}$.  The $\mathbf{Typepred}$ rule can
therefore be used on the Skolem constant $s$ to complete the proof.}

The case when $\Mgamma{A}$ is nonempty, we have for any $\gamma$
satisfying $\Gamma$ and $s\in \Mgamma{A}$, that $\gamma'$ given by 
$\gamma\{s\gets z\}$ is a satisfying assignment for $\Gamma, s: A$\@.
By the induction hypothesis, there is either an $a$ in $\Sigma$
such that $\mathcal{M}(\Gamma, s: A\vbar\gamma')(b) = \zerob$ or
a $c$ in  ${(f~s) = (g~s),\Lambda}$ such that $\mathcal{M}(\Gamma, s:
A\vbar\gamma')(c) = \oneb$\@.  If we have such a $b$ in $\Sigma$,
by Proposition~\ref{meaning-preservation}, we also have that
$\Mgamma{b} = \zerob$.  A similar argument can be used
if we have such a $c$ in $\Lambda$\@.  If $c$ is $(f~s) = (g~s)$, then
$\Mgamma{f}(z) = \Mgamma{g}(z)$ for every $z$ in $\Mgamma{A}$\@.
By set-theoretic extensionality, this means that $\Mgamma{f}$ and
$\Mgamma{g}$ are identical elements 
of $\Pi{}F$ where $F$ maps $z$ in $\Mgamma{A}$ to an element of
$\mathcal{M}(\Gamma, x : \ttvar~A\vbar \gamma\{x \gets z\})(B)$\@.
Therefore $\Mgamma{f = g} = \oneb$ as desired.

\item [TupExt:]  If there is some $d$ in $\Sigma$ such that by applying the
induction hypothesis to any of the subgoals $\Mgamma{d} = \zerob$,
then the same holds for the conclusion sequent.  Similarly,
if the induction hypothesis on some subgoal yields a $c$ in $\Lambda$
such that $\Mgamma{c} = \oneb$, then the same holds for the
conclusion sequent.  So the remaining case is when, by the
induction hypothesis, 
$\Mgamma{\proj{i}(a)} = \Mgamma{\proj{i}(b)}$ for each $i\in \{1,2\}$.
It is therefore easy to conclude by set-theoretic extensionality that
$\Mgamma{a}$ and $\Mgamma{b}$ are identical elements of
$\Mgamma{a/\tupletype{\listwo{T}{n}}}$\@.  We can then use
Proposition~\ref{substitution-equivalence} to conclude
that $\Mgamma{a}$ and $\Mgamma{b}$ are identical
elements of $\Mgamma{\tupletype{\listwo{T}{n}}}$\@. 

\end{description}

\item \emph{Type Constraint Rule}: Recall from
Proposition~\ref{maximal-subtype}
that when $\tauGamma{a} = A$, then
$\Mgamma{\pi(A)(a)} = \oneb$.  Given this and the induction hypothesis,
it must either be the case that we have a $b$ in $\Sigma$ such that
$\Mgamma{b} = \zerob$ or a $c$ in $\Lambda$ such that $\Mgamma{c} =
\oneb$\@. 
\end{enumerate}
\end{proof}

To tie the development so far into a single simultaneous
induction as promised, we state the key theorem whose subproofs have been
given by the theorems presented thus far, namely,
Theorems~\ref{dependent-type-semantics},~\ref{dependent-term-semantics},
and~\ref{proof-soundness}\@.  
\begin{theorem}\label{big-soundness}
If $\tau()(\Gamma) = \ttcontext$, then 
\begin{enumerate}
\item If $\Sigma, \Lambda$ is a list of preterms such that for every $a$
in $\Sigma; \Lambda$, $(\tauGamma{a} \sim 
\ttbool)_\Gamma$, and $\sequent{\Sigma}{\Gamma}{\Lambda}$ is provable, then
for any satisfying assignment
$\gamma$ for $\Gamma$, 
either there is a $b$ in $\Sigma$, such that $\Mgamma{b} = \zerob$ or a $c$ in
$\Lambda$, such that $\Mgamma{c} = \oneb$\@.

\item If $A$ is a pretype such that $\tauGamma{A} = \tttype$, then for any
assignment $\gamma$ 
satisfying $\Gamma$, $\Mgamma{A}\in U$\@.

\item If $a$ is a preterm such that $\tauGamma{a} = A$, then for any
assignment $\gamma$ satisfying $\Gamma$, $\Mgamma{a}\in \Mgamma{A}$\@.
\end{enumerate}
\end{theorem}


\section{Summary}

The logical inference rules for the PVS logic have been presented in a
sequent calculus format.  The formal semantics presented in the earlier
chapters is used to establish the soundness of these proof rules.


\chapter{Conclusion}\label{conclusion}

We have presented the syntax and semantics of idealized PVS in several
stages.  In the first stage we introduced the simply typed fragment,
which was then extended with type definitions.  The third such fragment
included subtyping;  the fourth fragment introduced dependent typing.
Finally, we introduced constant definitions and parametric and nonparametric
theories.

The semantic definition was given in a novel, functional style where a
canonical type was assigned to each type correct term.  The interplay between
types and proofs in PVS introduced subtleties and complexities
into the semantic definition.  We can now answer some of the questions
raised in Chapter~\ref{introduction}:
\begin{itemize}
\item {\em What is the semantic core of the language, and what is
just syntactic sugar?}

The semantic core of the language is a typed lambda calculus with
simple function and tuple types, predicate subtypes, dependent types,
parametric theories, and conditional expressions.  Many of the other
features of the PVS language such as records and update expressions
can be explained in terms of the core language.

\item {\em What are the rules for determining whether a given PVS
expression is well typed?  }

The typechecking rules have been presented in terms of the definition of
the $\tau$ operator in Chapters~\ref{simple},  \ref{subtypes},
\ref{dependent}, \ref{theories}, and~\ref{conditionals}\@. 

\item {\em How is subtyping handled, and in particular, how are
proof obligations corresponding to subtypes generated?}

Typechecking an expression $a$ with respect to predicate subtype
constraint $\subtype{x}{T}{p(x)}$
is done by generating the proof obligation $p(a)$ under the logical
context in which $a$ is being typechecked.  This is made precise
in Definitions~\ref{defn:type rules with subtypes} and~\ref{defn:type rule
for conditional expressions}\@.  
Proof obligations are generated when typechecking contexts (for
nonemptiness),  typechecking expressions with respect to expected subtypes, 
and  comparing two types containing subtype expressions for
compatibility.

\item  {\em What is the meaning, in set-theoretic terms, of a PVS expression or
assertion?}

The set-theoretic meaning of well-formed PVS types and expressions
is given by  a meaning function
$\mathcal{M}$ that assigns a set $\Mgamma{T}$ from the universe $U$ to each type
$T$,  and an element $\Mgamma{a}$ of $\Mgamma{T}$ to a given term $a$ of type
$T$\@.

\item {\em Are the type rules sound with respect to the semantics? }

The typechecking function $\tau$ is defined to check contexts,
preterms, and pretypes for type correctness.    The type rules are shown
to be sound with respect to the given semantics in
Theorem~\ref{big-soundness}\@. 

\item {\em Are the proof rules sound with respect to the semantics?}

The proof rules are given in Chapter~\ref{proof theory} in a sequent
calculus format and proved to be sound with respect to the semantics in
Theorem~\ref{big-soundness}\@.  

\item {\em What is the form of dependent typing used by PVS, and
what kinds of type dependencies are disallowed by the language?}

The semantic analysis of dependent typing in Chapter~\ref{dependent}
revealed that type dependencies were constrained to be rank-bounded.  This
is true because the dependencies in dependent typing only constrain the
predicate part of predicate subtypes. Thus, when there is a dependent type
$T(n)$ that depends on a parameter $n$, the meaning of $T(n)$ has a fixed
rank regardless of the meaning assigned to $n$\@.  The PVS language
features used to define dependent types all preserve the rank-boundedness.
Language extensions violating rank-boundedness such as a type dependency
of the form $\funtype{n : nat}{T^n}$ are disallowed.  One can extend the
language with such dependent types, but the semantics would then be
considerably more complicated.

\item {\em What is the meaning of theory-level parametricity, and
what, if any, are the semantic limits on such parameterization?}

The semantics of parametric theories is described in
Chapter~\ref{theories}\@.  In particular, the semantics for parametric
theories is given in terms of rank-preserving maps between the meanings of
the parameters and the meanings of the identifiers declared in the theory.
These maps must be such that the rank of an assignment to a type in a theory
depends only on the ranks of the (meanings of the) type parameters.

\item {\em What language extensions are incompatible with the reference
semantics given here?}

We have already indicated that any language extension, such as an $n$-tuple
type $T^n$, that violates rank-boundedness would be incompatible with the
semantics presented here.
\end{itemize}

This report presents only the core language of PVS.  A more complete
semantic treatment would include arithmetic, recursive constant
definitions, inductive definitions, recursive datatypes, assumptions on
theory parameters, and type judgements.

\paragraph{Acknowledgments. } The advice and encouragement of John Rushby,
Rick Butler, Paul Miner, Pat Lincoln, and Mandayam Srivas are greatly
appreciated, as are the useful expert comments of Peter Dybjer, Mike
Gordon, Doug Howe, and Paul Jackson.  Bruno Dutertre, Paul Miner, and
Harald Rue{\ss} suggested numerous improvements to earlier drafts.

\begin{comment}
What is a specification language?  A programming language is meant to
express an algorithm in such a way that it can be executed on a computing
machine.  For example, a program implementing a sorting algorithm such as
quicksort can be written in a programming language like C.  Such a program
is a recipe for computing the desired outcome but does not directly
describe this outcome.  In the case of a sorting program, the desired
result is that 
the output array must be an ordered permutation of the input array.
A specification language is the medium for expressing and establishing
such properties of programs.  A specification language could itself be a
programming language but non-executable features such as quantifiers and
temporal operators  add to the expressiveness of a specification language.
%
Specification languages need to capture mathematics
as well as programming.  The use of an expressive type system lends
considerable mathematical clarity to a specification.  
ProVeS is an environment for
specification  and verification based on a typed higher-order logic.
The type system of ProVeS includes
{\em predicate subtypes}, i.e.,
the type consisting of elements of a given type satisfying a specified
predicate, in addition to the usual constructs for
function, product, record, and array types.  Automated theorem proving
procedures are employed in the process of typechecking.
The design and rules
for the type system and logic of ProVeS are presented along
with arguments for their soundness.
\end{abstract}
%
\include{intro}
\include{rules}
\include{logic}
\include{semantics}
\include{soundness}
%
\section{Conclusion}
%
We have described a type system and a higher-order logic \ProL\ 
incorporating predicate subtypes and described its semantics.
The use of subtypes in general, and predicate subtypes in particular,
has many advantages from the point of view of specifications.  
The module system based on the \ProL\  logic will be discussed
in a forthcoming report.  
%
The primary philosophical difficulty in the design of \ProL\ 
was engendered by the choice between subsumption and coercion.
The crucial question here seems to be the treatment of equality and
whether the equality predicate on subtypes is inherited from the
supertype in the case of subsumption.  If equality is not inherited,
it is unclear whether the choice between subsumption and coercion
matters significantly.  If equality is inherited, then subsumption
entails a careful treatment of extensionality.  In an earlier
treatment using subsumption, we did not take extensionality
as axiomatic but instead defined the extensional subtype of each
type as a predicate subtype using the appropriate $\eta$-rule
as the predicate.  
%
Though we believe that the use of predicate subtypes is eliminable, we
have not yet arrived at a satisfactory proof that the \ProL\  logic
can be embedded in a simply typed higher-order logic.
A proof of completeness for the  \ProL\  logic as presented also
remains to be done.  A typechecker for a language similar to
the one described here has been implemented~\cite{EHDM:Tutorial}.  We do
need  substantially more experience writing specifications with
predicate subtypes before we can comment more fully on their utility.
%
\paragraph{Acknowledgments:}  Friedrich von Henke and John Rushby
sparked our interest in incorporating predicate subtypes into
a specification language.  Our work was funded internally by SRI
International.  
\end{comment}

\newpage
\bibliographystyle{modalpha} % or modplain
\addcontentsline{toc}{chapter}{Bibliography}
%\bibliography{/homes/rushby/nasa,/homes/rushby/jmr,/homes/shankar/tex}
\bibliography{../pvs}

\end{document}

\include{soundness}
%
\section{Conclusion}
%
We have described a type system and a higher-order logic \ProL\ 
incorporating predicate subtypes and described its semantics.
The use of subtypes in general, and predicate subtypes in particular,
has many advantages from the point of view of specifications.  
The module system based on the \ProL\  logic will be discussed
in a forthcoming report.  
%
The primary philosophical difficulty in the design of \ProL\ 
was engendered by the choice between subsumption and coercion.
The crucial question here seems to be the treatment of equality and
whether the equality predicate on subtypes is inherited from the
supertype in the case of subsumption.  If equality is not inherited,
it is unclear whether the choice between subsumption and coercion
matters significantly.  If equality is inherited, then subsumption
entails a careful treatment of extensionality.  In an earlier
treatment using subsumption, we did not take extensionality
as axiomatic but instead defined the extensional subtype of each
type as a predicate subtype using the appropriate $\eta$-rule
as the predicate.  
%
Though we believe that the use of predicate subtypes is eliminable, we
have not yet arrived at a satisfactory proof that the \ProL\  logic
can be embedded in a simply typed higher-order logic.
A proof of completeness for the  \ProL\  logic as presented also
remains to be done.  A typechecker for a language similar to
the one described here has been implemented~\cite{EHDM:Tutorial}.  We do
need  substantially more experience writing specifications with
predicate subtypes before we can comment more fully on their utility.
%
\paragraph{Acknowledgments:}  Friedrich von Henke and John Rushby
sparked our interest in incorporating predicate subtypes into
a specification language.  Our work was funded internally by SRI
International.  
\end{comment}

\newpage
\bibliographystyle{modalpha} % or modplain
\addcontentsline{toc}{chapter}{Bibliography}
%\bibliography{/homes/rushby/nasa,/homes/rushby/jmr,/homes/shankar/tex}
\bibliography{../pvs}

\end{document}

\include{soundness}
%
\section{Conclusion}
%
We have described a type system and a higher-order logic \ProL\ 
incorporating predicate subtypes and described its semantics.
The use of subtypes in general, and predicate subtypes in particular,
has many advantages from the point of view of specifications.  
The module system based on the \ProL\  logic will be discussed
in a forthcoming report.  
%
The primary philosophical difficulty in the design of \ProL\ 
was engendered by the choice between subsumption and coercion.
The crucial question here seems to be the treatment of equality and
whether the equality predicate on subtypes is inherited from the
supertype in the case of subsumption.  If equality is not inherited,
it is unclear whether the choice between subsumption and coercion
matters significantly.  If equality is inherited, then subsumption
entails a careful treatment of extensionality.  In an earlier
treatment using subsumption, we did not take extensionality
as axiomatic but instead defined the extensional subtype of each
type as a predicate subtype using the appropriate $\eta$-rule
as the predicate.  
%
Though we believe that the use of predicate subtypes is eliminable, we
have not yet arrived at a satisfactory proof that the \ProL\  logic
can be embedded in a simply typed higher-order logic.
A proof of completeness for the  \ProL\  logic as presented also
remains to be done.  A typechecker for a language similar to
the one described here has been implemented~\cite{EHDM:Tutorial}.  We do
need  substantially more experience writing specifications with
predicate subtypes before we can comment more fully on their utility.
%
\paragraph{Acknowledgments:}  Friedrich von Henke and John Rushby
sparked our interest in incorporating predicate subtypes into
a specification language.  Our work was funded internally by SRI
International.  
\end{comment}

\newpage
\bibliographystyle{modalpha} % or modplain
\addcontentsline{toc}{chapter}{Bibliography}
%\bibliography{/homes/rushby/nasa,/homes/rushby/jmr,/homes/shankar/tex}
\bibliography{../pvs}

\end{document}

\include{soundness}
%
\section{Conclusion}
%
We have described a type system and a higher-order logic \ProL\ 
incorporating predicate subtypes and described its semantics.
The use of subtypes in general, and predicate subtypes in particular,
has many advantages from the point of view of specifications.  
The module system based on the \ProL\  logic will be discussed
in a forthcoming report.  
%
The primary philosophical difficulty in the design of \ProL\ 
was engendered by the choice between subsumption and coercion.
The crucial question here seems to be the treatment of equality and
whether the equality predicate on subtypes is inherited from the
supertype in the case of subsumption.  If equality is not inherited,
it is unclear whether the choice between subsumption and coercion
matters significantly.  If equality is inherited, then subsumption
entails a careful treatment of extensionality.  In an earlier
treatment using subsumption, we did not take extensionality
as axiomatic but instead defined the extensional subtype of each
type as a predicate subtype using the appropriate $\eta$-rule
as the predicate.  
%
Though we believe that the use of predicate subtypes is eliminable, we
have not yet arrived at a satisfactory proof that the \ProL\  logic
can be embedded in a simply typed higher-order logic.
A proof of completeness for the  \ProL\  logic as presented also
remains to be done.  A typechecker for a language similar to
the one described here has been implemented~\cite{EHDM:Tutorial}.  We do
need  substantially more experience writing specifications with
predicate subtypes before we can comment more fully on their utility.
%
\paragraph{Acknowledgments:}  Friedrich von Henke and John Rushby
sparked our interest in incorporating predicate subtypes into
a specification language.  Our work was funded internally by SRI
International.  
\end{comment}

\newpage
\bibliographystyle{modalpha} % or modplain
\addcontentsline{toc}{chapter}{Bibliography}
%\bibliography{/homes/rushby/nasa,/homes/rushby/jmr,/homes/shankar/tex}
\bibliography{../pvs}

\end{document}
