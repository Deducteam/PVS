\chapter{Running PVS Standalone (without Emacs)}

Running PVS without Emacs is done with the \texttt{-raw} flag.  This
chapter describes how to make effective use of PVS when run in this way.

\section{Interaction}

When run in standalone mode, PVS acts as a uniz pipe, taking input on
\texttt{stdin} and writing output to \texttt{stdout}, with error messages
to \texttt{stderr}.  Normally PVS reads input, processes it, writes the
result, and prompts for more input.  If this is being done by a separate
program, it should recognize the prompt, so that it knows when PVS is
finished with the current task.

The prompts you should handle are given by the following Emacs regular
expression
\begin{alltt}
"^[ ]*\\(\\[[0-9]+i?c?\\] \\|\\[step\\] \\)?\\(\\(<[-A-Za-z]* ?[0-9]*> \\)\\3?\\|[-A-Za-z0-9]+([0-9]+): \\)\\|Rule\\? \\|(Y or N)\\|(Yes or No)\\|Please enter"
\end{alltt}

prompt ::= \rep{space} \opt{\lit{[} number \opt{\lit{i}} \opt{\lit{c}} \lit{]}

break

The basic prompt consists of the package name followed by a number in
parentheses, a colon, and a space.
\begin{alltt}
  pvs(27):
\end{alltt}

If in a break loop, this is preceded by a number in square brackets.
\begin{alltt}
  [1] pvs(28):
\end{alltt}
The number inside the square brackets indicates the break level.  This
means that lisp has suspended processing at this point, so that the cause
of the break may be investigated.  This usually indicates there is a
problem.  If you decide to keep going, send a \texttt{:reset} to unwind
the break to the top level.  Sometimes the break level is followed by a
`\texttt{c}', indicating that the error is continuable.  This simply means
that there is a lisp error handler that will take some default action if
the error is sent a continue command (normally \texttt{:cont 0}).

When in the prover, the prompt is \texttt{Rule?} followed by a space.
When in the ground evaluator, the prompt is \texttt{<GndEval>} followed by
a space.  Note that if an error is encountered while interacting with the
prover or the ground evaluator, the prompt reverts to the break prompt
described above.  In this case \texttt{:reset} will end the prover or ground
evaluator interaction, while \texttt{(restore)} returns control.
