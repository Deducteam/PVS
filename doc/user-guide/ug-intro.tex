% Document Type: LaTeX
% Master File: user-guide.tex
\chapter{Introduction}
\label{introduction}

The Prototype Verification System (PVS)\index{Prototype Verification
System (PVS)} provides an integrated environment for the development and
analysis of formal specifications, and supports a wide range of activities
involved in creating, analyzing, modifying, managing, and documenting
theories and proofs.  This manual describes the system, including the
system commands, the computing environment, how to get and install PVS,
customization, and a short tutorial on Emacs.  The main set of manuals
for the PVS system consists of this manual, the language
reference~\cite{PVS:language}, and the prover guide~\cite{PVS:prover}.
There are also several supporting technical reports: the formal semantics
of PVS~\cite{PVS:semantics}, an advanced
tutorial~\cite{Rushby&Stringer-Calvert95}, and a description of the
abstract datatypes mechanism~\cite{PVS-ADT:TR}.  All of these manuals (and
much more!) are available online at \url{http://pvs.csl.sri.com/}

The rest of this chapter provides a broad overview of PVS; the facilities
provided by the system are discussed in the order you are likely to
encounter them.

\section*{The PVS Environment}
\index{PVS environment@{PVS environment}}
\index{environment|see{PVS environment}}

PVS runs under 64-bit versions of Linux and MacOSX.  It can be run on
other 64-bit platforms using virtualbox or VMWare.  PVS is
implemented in Common Lisp\index{Common Lisp}, but it is not necessary to
know Lisp to effectively use the system.\footnote{The only exception to
  this is in writing complex prover strategies.}  PVS runs best using the
X window system, though it is not required.  The Emacs
(\gnuemacs\index{Gnu Emacs} or XEmacs\index{XEmacs}) editors provide one
interface to PVS; familiarity with Emacs and access to the GNU Emacs
manual~\cite{emacs20} (usually available as an info file) are desirable.
% A brief introduction to Emacs is provided in Appendix~\ref{emacs-intro} on
% page~\pageref{emacs-intro} of this manual.
The \LaTeX\index{Latex@{\LaTeX}} generating facilities require a good
understanding of the \LaTeX\ document preparation system~\cite{latex2e}.
If you have Tcl/Tk available, there are PVS interfaces provided that
display proof trees, theory hierarchies, and proof commands.  Instructions
for obtaining\index{obtaining PVS} and installing\index{installing PVS}
the PVS system as well as Emacs, X windows, \LaTeX, and Tcl/Tk may be
found at \url{http://pvs.csl.sri.com}.


\section*{The PVS Language}
\index{pvs language@{PVS language}}

The specification language of PVS is built on higher-order logic; \ie\
functions can take functions as arguments and return them as values, and
quantification can be applied to function variables.  There is a rich set
of built-in types and type-constructors, as well as a powerful notion of
subtype.  Specifications can be constructed using definitions or axioms,
or a mixture of the two.

Specifications are logically organized into parameterized
\emph{theories}\index{theories} and \emph{datatypes}\index{datatypes}.
Theories are linked by \emph{import} and \emph{export} lists.
Specifications for many foundational and standard theories are preloaded
into PVS as \emph{prelude}\index{prelude}\index{PVS!prelude} theories that
are always available and do not need to be explicitly imported.  Details
on the PVS language may be found in the PVS language
reference~\cite{PVS:language}.

\section*{Specification Files and the PVS Context}
\index{specifications}

PVS specifications are ordinary {\sc ascii} text files prepared and
modified using a text editor---usually the Emacs editor that acts as the
interface to PVS.  A PVS specification consists of any number of such
files, each of which contains one or more theories or datatypes.  PVS
specification files have the \texttt{.pvs} extension.

Each specification file has associated with it a proof file (with the
\texttt{.prf} extension) that saves the proof scripts generated during
proof attempts on formulas contained in the associated PVS specification
file.
In addition, the system generates binary representations of the
typechecked specification files (with the \texttt{.bin} extension) that
speed up retypechecking when a PVS session is resumed in the same
context.

The set of files and theories constituting a specification, together with
various items of status information, comprise a PVS
\emph{context}.\index{PVS context} The PVS context retains information
about the state of a specification and verification from one PVS session
to the next.  This information is primarily kept in the
\texttt{.pvscontext} file that is associated with each PVS context.  It
keeps track of which formulas have been proved, and which binary files are
valid by keeping track of the write dates associated with the various
files.

PVS contexts are closely related to directories, and the term
\emph{context} is used in this document to refer to either the PVS context
or its associated directory.  Note that the directory may contain files
other than those produced by or for PVS, but these are not considered to
be a part of the context.

During a PVS session, there is always a \emph{current
context}\index{current context} in which the activities of PVS take
place.  For example, typechecking of a specification file is allowed only
if that file is a part of the current context.  There are commands for
changing the current context during a PVS session, so that it is
unnecessary to exit PVS just to change contexts.  Because contexts are
associated with \unix\ directories there can be at most one PVS context
in a directory, so for most purposes a PVS context and its containing
directory can be treated synonymously.

\section*{PVS Libraries}
\index{libraries}\index{PVS!libraries}

PVS has a library facility that allows files and theories from one PVS
context to be used in another, thus allowing for general reuse, and making
it easier to standardize theories that are frequently used.  There are two
ways that the library facility can be used: by explicitly importing a
theory from a different PVS context within a specification, or by issuing
a command that effectively extends the prelude.


\section*{The PVS User Interface}
\index{user interface}

You interact with the PVS system through a customized Emacs.  It is expected, though
not required, that editing of specifications is performed with this editor.
Using other editors is quite painful, as they cannot directly interact
with the underlying Lisp image.

Instructions are issued to PVS by means of Emacs commands.  For example,
in order to perform a proof, the cursor is positioned at a formula
declaration in the Emacs buffer and the Emacs command \ecmd{prove} or
the key sequence \key{C-c p} is issued.  PVS returns information to you
through various display mechanisms provided by Emacs or Tcl/Tk.

The PVS interface allows a certain amount of parallel activity.  For
example, you can continue editing theories or perform any other activity
supported by Emacs while PVS is typechecking a series of theories or
performing a lengthy proof.  Also, you need not wait for one PVS activity
to finish before issuing another command; most commands are queued for
execution in the order they were issued, but certain status and other
short commands preempt any ongoing analyses, perform their function, and
then return the system to its previous activity.

\section*{Prettyprinting}
\index{prettyprinter}

The PVS prettyprinter rearranges the layout of PVS specification text into
a standard, regular format.  The commands allow the prettyprinting of
files, theories, regions, or individual declarations.  You can choose
whether to prettyprint specification text, but output from PVS itself is
always prettyprinted.

\section*{Parsing}
\index{parser}

The parser checks theories for syntactic consistency and builds an
internal representation that is used by other components of the system.
When errors are detected by the parser or other components of PVS, the
cursor is generally placed at the location where the error was detected
and an error message is displayed in a pop-up window.

\section*{Typechecking}
\index{typechecking}\label{typechecking}

The PVS typechecker analyzes theories for semantic consistency and adds
semantic information to the internal representation built by the parser.
The type system of PVS is not algorithmically decidable; theorem proving
may be required to establish the type-consistency of a PVS specification.
The theorems that need to be proved are called \emph{type-correctness
conditions}\index{type-correctness condition (\tcc)} (\tccs).  \tccs\ are
attached to the internal representation of the theory and displayed on
request.  There are commands available that attempt to prove the \tccs\
using built-in prover strategies.  You may choose when to prove the \tccs,
but until they are proved the theory that generated them is not considered
to be typechecked.

The PVS system automatically tracks the status of theories (whether they
have been changed, parsed, typechecked etc.) and also takes care of the
dependencies among theories.  For example, if the specification text of a
theory is changed and then a command is issued that requires semantic
information, PVS will parse and typecheck the theory automatically.
More subtly, if the text of a theory that is used by the current theory is
changed, both theories will need to be typechecked in order to guarantee
consistency.  This happens automatically as the need arises.

It is often necessary to make changes in theories on which long chains of
other theories depend, and frequent reparsing and retypechecking of such
theory chains can be very time-consuming.  Therefore PVS provides commands
which allow limited additions and modifications of declarations without
requiring that the associated theories be retypechecked
(Section~\ref{add-decl}, page~\pageref{add-decl}).

There is some incremental typechecking that goes on at the theory level.
When a typecheck command is issued on a PVS file that has been modified,
the file is first parsed, and the resulting abstract syntax is compared to
the previous abstract syntax.  If they are the same, the theory is not
retypechecked.  Otherwise it typechecks as usual.  Comments, added or
deleted whitespace, and certain kinds of expression transformations (such
as changing \texttt{a + 1} to \texttt{+(a,1)}) will thus not trigger
retypechecking.

\section*{Browsing}
\index{browsing}

Specifications can be quite large and involve many theories and files,
and it can become difficult to remember all the identifiers declared,
their locations, definitions, and uses.  PVS provides facilities for
displaying or visiting the declaration of an identifier indicated by
the cursor, for displaying all references to an identifier, and for
producing a cross-reference of all declared identifiers.

\section*{Proving}
\index{proving}

PVS provides a powerful interactive proof checker with the ability to
store and replay proofs.  PVS can be instructed to perform a single proof,
or to rerun all the proofs in a given theory, all the proofs of all the
lemmas used in a proof, or all the proofs in an entire specification.
This manual describes how to enter the prover and some of the commands for
obtaining and editing proof information.  Details on the proof checker
commands may be found in the prover guide.

\section*{Status and Proof Chain Analysis}
\index{status}\index{proof chain analysis}

The PVS system provides several commands for determining the status of
specification elements such as theories and formulas.  You can, for
example, inquire whether a theory has been typechecked or whether a specific
formula has been proved.

\emph{Proof chain analysis} is an important form of status report.  An
individual theorem is considered proved when it is the conclusion of a
successful proof, but this is a local notion; the result
is a true theorem only if all the lemmas appearing in its proof have
themselves been proved or stated as axioms or definitions, and all
\tccs\ have been discharged.  Proof chain analysis assures that all of
the aforementioned obligations are discharged.  In addition to recording
whether or not the proof chain is sound, the output of this analysis
also identifies the axiomatic foundation of the given theorem.


\section*{Generating Output}

When a formal specification and verification is complete, it is usually
desirable to present it to others in as readable a form as possible.  PVS
provides commands for generating \LaTeX\ versions of the specifications
and proofs that can be included in typeset documents.  The output produced
can be controlled by user-supplied tables so that mathematical notation,
including infix and mix-fix symbols and subscripts and superscripts, can
be created easily.  This customized prettyprinting facility makes it
possible to reproduce the notation standard to some branch of mathematics
or computer science, thereby assisting peer review of the formal
specification.  The typeset specifications are also of value during the
development of a formal specification and verification, as they allow
direct comparison with existing, informal presentations and analyses.


\section*{Display Commands}

There are a few commands available for displaying graphical information
using an interface to the Tcl/Tk system.  These include the display of
proof trees, theory hierarchies, and prover commands.  These displays are
interactive; for example the proof tree display is updated as a proof is
developed, and clicking on a theory in the theory hierarchy display pops
up an Emacs buffer containing that theory specification.

\section*{Other Commands}

There are other miscellaneous commands are not easily categorized, such as
commands for sending bug reports, interrupting PVS, getting help, and some
commands that help in editing PVS files.
