% Document Type: LaTeX
% Master File: user-guide.tex
\chapter{Customizing PVS}
\label{customization}
\index{customization}\index{PVS customization}

PVS is a complex system, and utilizes many subsystems, including Lisp,
Emacs, the X window system, and Tcl/Tk.  You can control aspects of these
subsystems by a combination of command-line arguments, environment
variables, and various files.  In this section we discuss some aspects of
the customization of these subsystems as they relate to PVS.

\section{Invoking PVS}\label{invoking-pvs}
\index{invoking PVS}\index{starting PVS}

PVS is invoked from a shell script named \texttt{pvs}\index{PVS shell
script} in the PVS directory---this is a text file, and may be examined or
copied and modified to suit your taste.  The script is a Bourne shell
script, and requires {\tt /bin/sh} to execute correctly.\footnote{On some
systems, \texttt{/bin/sh} is linked to the \texttt{bash} shell; this works
as well.}

PVS accepts a number of command-line arguments, as well as using
environment variables.  The command-line arguments specific to PVS are
\begin{description}

\item[\texttt{-h | -help | --help}]
  \index{-help@\texttt{-help}}\index{command-line arguments!\texttt{-help}}
  - Print a brief description of the command
  line options and exit.

\item[\texttt{-lisp} \emph{lispname}]
  \index{-lisp@\texttt{-lisp}}\index{command-line arguments!\texttt{-lisp}} -
  Specifies which lisp to use. The lisp image used for PVS is then
  \texttt{pvs-\emph{lispname}}, which should be located in a directory
  determined by the machine architecture.  See Section~\ref{pvsimage},
  page~\pageref{pvsimage} for details.

\item[\texttt{-runtime}]
  \index{-runtime@\texttt{-runtime}}\index{command-line arguments!\texttt{-runtime}}
  - This is only
  needed for those who have access to the development version of Allegro
  Common Lisp, where both the runtime and development versions of PVS are
  available, defaulting to the development version.  With this option the
  runtime image is used instead.

\item[\texttt{-emacs} \emph{emacsname}]
  \index{-emacs@\texttt{-emacs}}\index{command-line arguments!\texttt{-emacs}}-
  Specifies the Emacs to use; see below for details.

\item[\texttt{-decision-procedures} \emph{new}\texttt{|}\emph{old}]
  \index{-decision-procedures@\texttt{-decision-procedures}}
  \index{command-line arguments!\texttt{-decision-procedures}}
  - Sets the default decision procedures to be used in proofs.  See
  Section~\ref{decision-procedure-commands},
  page~\pageref{decision-procedure-commands} for details.

\item[\texttt{-force-decision-procedures} \emph{new}\texttt{|}\emph{old}]
  \index{-force-decision-procedures@\texttt{-force-decision-procedures}}
  \index{command-line arguments!\texttt{-force-decision-procedures}}
  - Forces the chosen decision procedure to be used regardless of the
  default decision procedure setting or which decision procedures were
  used in developing a proof.  Note that with this option there is no way
  to switch between the new and old decision procedures.

\item[\texttt{-nw}]\index{-nw@\texttt{-nw}}\index{command-line arguments!\texttt{-nw}}
  - Tells Emacs not to use its special interface to X.

\item[\texttt{-batch}]\index{-batch@\texttt{-batch}}\index{command-line arguments!\texttt{-batch}}
  - Run PVS in batch mode. See chapter~\ref{batchmode},
  page~\pageref{batchmode} for details.

\item[\texttt{-timeout}:]\index{-timeout@\texttt{-timeout}}
  \index{command-line arguments!\texttt{-timeout}}
  - In batch mode, this causes typechecking
  and individual proof attempts to be interrupted after the given number
  of seconds.

\item[\texttt{-nobg}:]\index{-nobg@\texttt{-nobg}}
  \index{command-line arguments!\texttt{-nobg}}
  - Normally PVS starts in
  the background (with the \& control operator).  This starts it in the
  foreground.

\item[\texttt{-raw}:]\index{-raw@\texttt{-raw}}
  \index{command-line arguments!\texttt{-raw}}
  - This runs PVS without
  Emacs.  This is only useful for front ends, which must do the same
  initialization as done by the Emacs interface.

\item[\texttt{-v} \emph{number}]\index{-v@\texttt{-v}}
  \index{command-line arguments!\texttt{-v}}
  - Specifies verbosity level for PVS batch mode. See Chapter~\ref{batchmode},
  page~\pageref{batchmode} for details.

\label{dash-q-option}
\item[\texttt{-q}]\index{-q@\texttt{-q}}\index{command-line arguments!\texttt{-q}}
  - A standard emacs option to
  inhibit loading of the users {\tt .emacs} file, but extended in PVS to
  inhibit loading of the users {\tt .pvsemacs}, {\tt .pvsxemacs-options}
  and \texttt{.pvs.lisp} files on startup.

\item[\texttt{-patchlevel} \emph{level}]
  \index{-patchlevel@\texttt{-patchlevel}}\index{command-line arguments!\texttt{-patchlevel}}
  - Specifies which
  patch files to load. Level \texttt{none} loads no patch files. Level
  \texttt{rel} loads the file \texttt{patch2} from your PVS directory,
  which usually contains the release versions of PVS patches. Other valid
  levels are \texttt{test} (loads the files \texttt{patch2} and
  \texttt{patch2-test}) and \texttt{exp} (loads the files \texttt{patch2},
  \texttt{patch2-test} and \texttt{patch2-exp}).  This option is mainly
  used for PVS development.


\end{description}
Any other command-line arguments are passed directly to the underlying
Emacs, including those for X windows---these are discussed
below.

In addition, the PVS script uses the environment
variables\index{PVS!environment variables}\index{environment variables}
\texttt{PVSLISP}\index{pvslisp@\texttt{PVSLISP}}\index{environment
variables!pvslisp@\texttt{PVSLISP}},
\texttt{PVSEMACS}\index{pvsemacs@\texttt{PVSEMACS}}\index{environment
variables!pvsemacs@\texttt{PVSEMACS}}, and
\texttt{PVSXINIT}\index{pvsxinit@\texttt{PVSXINIT}}\index{environment
variables!pvsxinit@\texttt{PVSXINIT}}, which may be set in your
\texttt{.cshrc} or \texttt{.login} file to specify the defaults you
prefer.  If both the environment variable and the corresponding
command-line argument are given, the command-line argument takes
precedence.  The \texttt{PVSXINIT} variable is described in
Section~\ref{windows}, page~\pageref{windows}.


\section{Emacs}
\index{Emacs|(}

The PVS system uses Emacs as its user interface, and provides a number of
files that extend Emacs for use with PVS.  For historical reasons, there
are currently a number of Emacs editors available.  Because we wanted PVS
to be freely available, we have chosen to concentrate on just \gnuemacs\
and XEmacs, which are also freely available.  To find out what version of
Emacs you are using, start up Emacs and type \iecmd{emacs-version}.  We
try to keep up with new releases of emacs and if necessary patch files
will be made available to support the new Emacs.

By default, the system uses \texttt{emacs}, which is assumed to be in your
path when you start up PVS.  You may specify a different Emacs as
specified above.  When you start PVS, is assumed (in order to supply X
resources in the correct format) that if the name of the emacs command
contains the character ``\texttt{x}'' then you are using XEmacs.

PVS loads your \texttt{\char'176/.emacs}\index{.emacs@\texttt{.emacs}}
file first (assuming you have not specified the {\tt -q} option as
described on page \pageref{dash-q-option}), followed by
\emph{PVSPATH}\texttt{/emacs/go-pvs.el}%
\index{go-pvs.el@\texttt{go-pvs.el}}, which determines which version of
emacs you are running and then loads the rest of the PVS emacs files,
including ILISP\index{ILISP}.  At this point you may receive an error from
PVS saying that your Emacs version is unknown.  PVS does not support Emacs
18 (or earlier), but we try to keep up with new Emacs versions as they are released.
Finally, the
\texttt{\char'176/.pvsemacs}\index{.pvsemacs@\texttt{.pvsemacs}} is
loaded.  If you are running XEmacs, the {\tt .pvsemacs} file will load
XEmacs options from the {\tt .pvsxemacs-options} file instead of the
standard {\tt .xemacs-options} file, as some are incompatible with
standard XEmacs.

In loading the files in this order, PVS functions and key
bindings will overwrite any conflicting ones defined in your
\texttt{.emacs} file.  \texttt{.pvsemacs} is the file to use to override
the key bindings and definitions given by PVS.  This approach was taken to
ensure that the behavior of PVS by default follows the user guide, but can
be readily modified to suit your taste.

One file that is worth noting is the
\emph{PVSPATH}\url{/emacs/emacs-src/pvs-abbreviations.el}%
\index{pvs-abbreviations.el@\texttt{pvs-abbreviations.el}} file, where the
abbreviations for many of the PVS commands are given.  You may define your
own abbreviations for commands you use a lot that don't currently have
abbreviations, by adding the appropriate lines in your \texttt{.pvsemacs}
file.  For example, adding
\begin{alltt}
  (pvs-abbreviate 'show-tccs 'st)
\end{alltt}
will make \ecmd{st} an abbreviation for \ecmd{show-tccs} in addition to
those already defined.  Note that you cannot redefine a name which is
already in use.

If you would like to byte-compile your Emacs customizations, create a
separate file, byte-compile it, and load it from your \texttt{.pvsemacs}.
Generally the kinds of forms provided in a \texttt{.pvsemacs} file are
simply variable settings and minor function definitions, and are not worth
byte-compiling.  It is only worthwhile if a function is being (re)defined
that will be invoked noninteractively and frequently, for example, if you
want to modify the way the process filter works.

\index{Emacs|)}


\section{The PVS Image}
\label{pvsimage}\index{PVS!lisp image|(}

PVS currently runs under Allegro Common Lisp on a number of different
platforms.  PVS is provided as a Common Lisp image, meaning that it
includes both the Lisp runtime system and the PVS programs, so you do not
need to have Allegro installed on your system.

There is usually just one PVS image available at a given site, and if the
system is properly installed, nothing further needs to be done.  If more
than one image is available, and the default one is not the desired one,
then it can be specified using either command-line arguments or
environment variables.  Invoking PVS with
{\smaller
\begin{alltt}
  pvs -lisp lucid -image pvs-lucid-sun4
\end{alltt}}
\noindent will use the \texttt{pvs-lucid-sun4} image.  Note that \texttt{-lisp
lucid} must be specified, so that the Emacs interface can be set up
properly.  For linux, also see the {\tt -redhat} option on page
\pageref{dash-redhat}.

Alternatively, the environment variables \texttt{PVSLISP} and
\texttt{PVSIMAGE} may be set to get the same effect.  Note that
command-line arguments take precedence.

After the PVS lisp image has started, it loads in the patch files as
specified by the {\tt -patchlevel} argument and then loads the file
{\tt .pvs.lisp} from your home directory.  This file can be used to
provide lisp customizations on a per user basis and for overriding
definitions in the patch file.

\index{PVS!lisp image|)}


\section{Window Systems}
\label{windows}

PVS was built primarily for the X window system\index{X windows}, though
it can be run from a terminal interface.
When run under X windows with the 
supported versions of Emacs, the resource name\index{resource
name}\index{PVS!resource name} will be set to \texttt{PVS}, and the
window\index{window name}\index{PVS!window name} and icon names\index{icon
name}\index{PVS!icon name} will be set to \texttt{PVS@}\textit{host\/},
where \textit{host\/} is the host name of the system on which PVS was
invoked.  These may be modified by adding command-line arguments or
setting the \texttt{PVSXINIT}\index{pvsxinit@\texttt{PVSXINIT}}%
\index{environment variables!pvsxinit@\texttt{PVSXINIT}} environment
variable.

You may customize the title and icon names by defining the function
\texttt{pvs-title-string} in your \texttt{.pvsemacs} file taking no
arguments and returning a string to be used as the title.  This function
is invoked at startup, and whenever the context is changed.  For example,
the following provides the name of the pvs path, the patch level
(\texttt{N} for none, \texttt{R} for released, \texttt{T} for test, and
\texttt{E} for experimental), the hostname, and the last two components of
the current context.

{\smaller
\begin{alltt}
    (defun pvs-title-string ()
      (format "%s%s%s:%s/"
          (trailing-components pvs-path 1)
        (cond ((stringp (cadddr *pvs-version-information*)) "E")
              ((stringp (caddr *pvs-version-information*)) "T")
              ((stringp (cadr *pvs-version-information*)) "R")
              (t "N"))
        (let ((host (car (string-split ?. (getenv "HOSTNAME")))))
          (format "@%s" host))
        (trailing-components *pvs-current-directory* 2)))
\end{alltt}}
\noindent For example, this might generate \url{pvs2.3N@photon:lib/finite_sets/}.

It is difficult to get a single setting for all of the Emacs versions; the
following table gives the arguments needed to set the resource, window,
and icon names for the various versions.

\begin{center}
\begin{tabular}{|l|l|l|l|} \hline
Emacs & Resource & Window & Icon \\ \hline\hline
emacs19 & \texttt{-rn} & \multicolumn{2}{|c|}{\texttt{-name}}\\ \hline
emacs19.29 (and later, & \multicolumn{3}{|c|}{\texttt{-name}}\\ 
including emacs20) & \multicolumn{3}{|c|}{ }\\ \hline
xemacs & \texttt{-name} & \texttt{-wn} & \texttt{-in} \\ \hline
\end{tabular}
\end{center}
\par\noindent Note: in emacs19, if \texttt{-rn} is not given, then
\texttt{-name} is used for the resource name as well.  Emacs19.29 and
later will give an error if the \texttt{-rn} argument is given.

The window name is the name used in the title bar of the PVS window, the
icon name is the name used in the icon, and the resource name is the name
referred to in the \texttt{.Xdefault} or \texttt{.Xresource} file that
controls the defaults for X clients.  An example entry for PVS in one of
these files might be
\begin{alltt}
!	PVS defaults
PVS.geometry: 80x63-0-0
PVS*pointerColor: Red
PVS*Font: *courier-medium-r-normal--12*
\end{alltt}
See the man pages for \texttt{X} and \texttt{emacs}, as well as the news
and info pages for the version of Emacs you are using for more details on
X resources.

The \texttt{PVSXINIT} environment variable may be set\footnote{Generally
environment variables are set in your shell startup file, \emph{e.g.},
\texttt{.profile} or \texttt{.cshrc}.} to a string
of command-line arguments that are then appended to the defaults described
above.  You can also change the default resource, window, and icon names,
simply by adding them to this variable (or by including them in the
command-line arguments).  Note that you should make certain that the
version of Emacs you are using matches the command-line arguments as shown
in the footnote.  You can tell that there is a mismatch when you start up
PVS and find buffers with names matching command-line arguments, \eg\
\texttt{-in} or \texttt{PVS@acorn}.
