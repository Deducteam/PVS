% Document Type: LaTeX
% Master File: user-guide.tex
\chapter{Errors}
\label{errors}
\index{errors}

This Appendix lists all the possible \pvs\ errors, along with a brief
explanation of their cause and possible solutions.  The errors are split
into \emacs\ errors, parser errors and typechecker errors.  For a
discussion of possible prover errors, see the PVS prover manual.

%\section{\emacs\ Errors}

\memo{Need to systematically check through the source code for
any new error messages, and to remember to update this document if
any new error messages are added in the future.}

\memo{Should the error messages occur in the index?}

\section{Parser Errors}

\begin{description}

\item[There is garbage at the end of your file or string:] - the parser
has parsed the file or string, and found more tokens after scanning a
complete nonterminal.

\item[End id {\em foo\/} does not match theory id {\em bar\/}] - indicates
that the identifier at the end of the theory is different from the
identifier at the beginning.

\item[End id {\em foo\/} does not match datatype id {\em bar\/}] -
indicates that the identifier at the end of the datatype is different from
the identifier at the beginning.

\item[Found \emph{foo} when expecting \emph{bar},\ldots] - this error
indicates that the wrong kind of token was found here, and gives a
somewhat arcane indication of what was expected.  If the error isn't
obvious, look at the grammar in the language manual, and keep in mind that
the position of the cursor may be beyond the actual cause of the error.

\item[\emph{foo} expected here] - this error is usually easy to diagnose
and fix, since the message indicates what is expected at the cursor
location.

\item[Inline datatypes may not have parameters] - this error indicates
that the parameters were provided to an inline datatype.  Note that if an
attempt is made to provide parameters in square brackets, an ``End
expected here'' error will result.

\item[Only one id allowed for datatypes] - it is illegal (and usually
undesirable) to give multiple datatypes in a single declaration.  If you
really want to do this, simply split the datatype apart---making one copy
for each identifier.

\item[Datatype identifier must be an id] - you may not use an operator
such as \texttt{+} as a datatype identifier.  It must be a letter followed
by any number of letters, digits, underscores, and question marks.

\item[May not have multiple ids with `\texttt{|}'] - expressions such as
\texttt{FORALL (x,y,z|p):\ foo} are illegal, and should either be written
as \texttt{FORALL x,y,(z|p):\ foo} or \texttt{FORALL x,y,z:\ p IMPLIES foo}.

\item[Enumeration types are only allowed at the top level] - this is
because enumeration types are translated into inline datatypes, which
generate a number of new top-level declarations.

\item[Function type must have a range] - the \texttt{FUNCTION} keyword was
used, but a range type (preceded by an arrow (\texttt{->}) was not given.

\item[Illegal character name] - this is an illegal name for a character.
The legal names are displayed in the table below.
\def\ct{\symbol{'136}}
\texttt{\begin{center}
\begin{tabular}{|cccccccc|}\hline
\ct @ & \ct A & \ct B & \ct C & \ct D & \ct E & \ct F & \ct G \\
\ct H & \ct I & \ct J & \ct K & \ct L & \ct M & \ct N & \ct O \\
\ct P & \ct Q & \ct R & \ct S & \ct T & \ct U & \ct V & \ct W \\
\ct X & \ct Y & \ct Z & \ct [ & \ct\symbol{'134} & \ct ] & \ct\ct & \ct\symbol{'137} \\
SP & ! & " & \# & \$ & \% & \& & ' \\
( & ) & * & + & , & - & . & / \\
@ & A & B & C & D & E & F & G \\
H & I & J & K & L & M & N & O \\
P & Q & R & S & T & U & V & W \\
X & Y & Z & [ & \symbol{'134} & ] & \ct & \symbol{'137} \\
` & a & b & c & d & e & f & g \\
h & i & j & k & l & m & n & o \\
p & q & r & s & t & u & v & w \\
x & y & z & \{ & | & \} & \symbol{'176} & \ct? \\ \hline
\end{tabular}\end{center}}

\item[Projection may not have actuals] - you may not include actuals with
a projection name, since projections do not belong to any theories.

\item[Projection may not have a theory id] - you may not include a theory
id with a projection name, since projections do not belong to any
theories.

\item[Projection may not have a library id] - you may not include a
library id with a projection name, since projections do not belong to any
libraries.

\item[Invalid id] - PVS identifiers must start with a letter, followed by
any number of letters, digits, underscores, or question marks.

\end{description}


\section{Typechecker Errors}

The typechecker errors are presented in subsections mostly based on
language constructs.

\subsection{Importing Errors}

\begin{description}

\item[Can't find file for theory \emph{foo}] - this error indicates that
there is an exporting of a theory that is not known in the context and a
file with the same name as the theory could not be found.  You must find
or create the specified theory and typecheck it before typechecking the
current theory.

\item[Wrong number of parameters] - a theory name was given with the wrong
number of actual parameters.  Compare the formal parameters of the theory
with the actual parameters given in the theory name; make sure that you
are not counting the \texttt{IMPORTING}s in the formal parameters.

\item[Theory {\em foo} is not in the IMPORTINGs] - this error results from
the use of a name of the form \texttt{\emph{foo}.\emph{bar}}, and the
theory \emph{foo} is neither a part of the prelude, nor has it been
previously imported to the theory.

\item[A theory may not import itself] - the current theory appears on an
importing chain.

\item[Theory {\em foo\/} not found] - this error occurs during the
typechecking of an importing clause, and indicates that the theory
\emph{foo} was not found in the current context.

\item[Wrong number of parameters] - in an importing clause, the indicated
theory has the wrong number of actual parameters given.

\item[Circularity found in importings of theories] - this error indicates
that there is a circularity in the closure of the importings clauses.  The
cursor is placed at the beginning of one of the offending theories, and a
list of theory names involved in the circularity is displayed.

\end{description}


\subsection{Exporting Errors}

\begin{description}

\item[May not export formal parameters] - an attempt was made to export a
formal parameter.  This is not allowed, as formal parameters are not
visible from outside the module.

\item[Name {\em foo\/} is not declared in this theory] - an attempt was
made to export an entity that was not declared in the current module.  For
further details on the use and restrictions of the exporting clause, see
the language manual.

\item[Name {\em foo\/} is not declared as {\em kind\/} in this theory] -
when names are overloaded, a kind (or type) may be included in the
exporting clause to specify which entity is actually exported.  This error
indicates that there is no entity of the kind (or type) specified.

\item[{\em foo\/} may not be exported] - this error indicates that an
attempt was made to export a variable or a field declaration.  Variable
declarations may not be exported, and field declarations are automatically
exported when the associated record type is, and may not be exported
otherwise.

\item[{\em foo\/} may not be exported unless the following are also: {\em
bar$_1$\/},\ldots] - entities may only be exported if all of the
declarations on which they depend are also exported.

\item[{\em foo\/} refers to {\em name\/}, which must be exported] -
entities may only be exported if all of the declarations on which they
depend are also exported.

\item[{\em foo\/} occurs in an EXPORTING WITH but is not in an IMPORTING
clause] - only theories that are imported may be exported.

\end{description}


\subsection{Declaration Errors}

\begin{description}

\item[Coercion must be a constant] - this error is associated with
coercion declarations, which only apply to constants, recursive
definitions, and inductive definitions.

\item[Coercion is not a unary function: \emph{foo}] - this error is given
when the constant to be used as a coercion has a domain with more than one
component.

\item[The domain and range of this coercion are compatible; the coercion
will never] \textbf{be used: \emph{foo}} - because coercions only take place when
there is otherwise a type error, if the domain and range are compatible
then the coercion would never be automatically applied.  Strictly
speaking, this does not have to be an error, but it is treated as such
because it is clearly not what is intended.  See the language manual for
more discussion.

\item[May not overload projection names] - an attempt has been made to
redeclare a projections name of the form \texttt{PROJ\_}$n$ (or some
variant of this with different cases, such as \texttt{Proj\_}$n$).  Because
of the special status of these names,\footnote{Projections are one of the
few entities of PVS that may not be declared in PVS, even using
parameterized theories} they may not be redeclared.

\item[{\em foo\/} is the name of a formal parameter and may not be
redeclared] - within a theory a formal parameter name may not be reused.
If you really want ot use the name, change the formal parameter name since
it will not be used outside the theory anyway.

\item[{\em foo\/} has previously been declared with type {\em type\/}] -
this error indicates that an entity has already been declared in this
theory.

\item[{\em foo\/} has previously been declared with type {\em type\/},
which may lead to ambiguity] - this error indicates that an entity has
been declared with a compatible type, which may lead to ambiguity.  The
difficulty here is that in disambiguating names, after the theory, actual
parameters, and library have all been pinned down, the type should
uniquely determine the resolution of a reference, and compatible types may
not allow the typechecker to make the distinction.

\item[Name {\em foo\/} already in use as a {\em kind\/}] - this error
applies to non-expression declarations such as formulas, theory
declarations, etc.  It indicates that the same name is being used twice
for the same kind of declaration in the same theory.

\end{description}

\subsubsection{Recursive Function Errors}

\begin{description}

\item[Recursive definition must be a function type] - the type of a
recursive declaration must be a function type, or a subtype of a function
type.

\item[Termination TCC could not be generated] - This error results from a
recursive call for which arguments have not been provided.

\item[Measure must be a function] - the provided measure must resolve to a
function.

\item[Wrong number of arguments in measure] - the argument pattern of the
measure function must match that of the recursive function being declared.

\item[Measure must have range a naturalnumber or an ordinal] - the range
type of the measure function usually is a natural number, but ordinal
numbers are also supported.  No other range types are allowed.

\item[Incompatible domain types] - the domain type of the measure is
incompatible with the corresponding domain type of the recursive function
being declared.

\end{description}


\subsubsection{Inductive Declaration Errors}

\begin{description}

\item[Inductive definition must be a function type] - the type of a
inductive declaration must be a function type, or a subtype of a function
type.

\item[Inductive definitions must have (eventual) range type boolean] - see
the language manual for a discussion of this restriction.

\item[Cannot determine parity of this occurrence of the inductive
definition] - the recursive occurrences of the inductive definition must
be positive; the system could not determine the parity of this occurrence.

\item[Negative occurrences of the inductive definition are not allowed] -
the recursive occurrences of the inductive definition must be positive,
this occurrence is negative.

\end{description}


\subsection{Type Expression Errors}

\begin{description}

\item[Enumtype must be declared at top level] - since enumeration types
are really just inline datatypes, they must be declared at the top level.

\item[No type found] - this error is displayed when a parenthesized
expression is used as a type and no type for the expression is found.

\item[Does not resolve to a predicate] - this error occurs when a
parenthesized expression is used as a type and the expression does not
resolve to a predicate, \ie, a function with boolean range type.

\item[Uninterpreted subtypes are only allowed at the top level] -
\memo{Can this error be reached?}

\item[Wrong number of variables] - this error comes from typechecking a
subtype expression of the form \texttt{\{x1,\ldots,xn:T | p\}}, where
\texttt{T} is a tuple type with number of component types different from
\texttt{n}.

\item[Duplicate field names not allowed] - the same field name was used
twice in the same record type.

\end{description}


\subsubsection{Judgement Declaration Errors}

\begin{description}

\item[No declaration of specified type could be found] - this error occurs
when typechecking a judgement declaration, and indicates that the entity
being declared as a judgement could not be found matching the specified
type.

\item[Only constant names or numbers may have HAS\_TYPE judgements] -
indicates that an expression other than a name or number was provided
here.

\end{description}


\subsubsection{Type Application Errors}

\begin{description}

\item[Uninterpreted types may not have parameters] - although type
declarations may now have parameters, they are not allowed for
uninterpreted type declarations.  See the language manual for a discussion
of this restriction.

\item[Type applications may not be curried] - the parameters to a type
declaration may not be curried, \eg\
\begin{alltt}
  t(x:int)(y:int): TYPE = {z:int | z < x + y}
\end{alltt}
is not allowed, for the simple reason that partial type applications are
not allowed.

\item[\emph{foo} may not be used in a type application] - only type names
declared to be applications (\ie, with declaration parameters) may be used
in type applications.

\item[Wrong number of parameters] - a type application was given with the
wrong number of parameters.

\end{description}


\subsection{Expression Errors}

\begin{description}

\item[Wrong arity] - this error occurs with tuple expressions, and
indicates that a different number of subexpressions was expected.

\item[{\em foo\/} expected here] - this error occurs when a lambda
expression is provided where a (non-functional) type \emph{foo} is
expected.

\item[Wrong number of bound variables - $n$ expected] - the given lambda
expression has a different number of bound variables than expected.

\item[Wrong number of assignments] - the given record expression has a
different number of assignments than expected.  Recall that each field of
a record expression must be given exactly once.

\item[No compatible types] - this error occurs for an update expression
when a function type is expected and the possible types of the update
expression are not compatible with the expected type.

\item[{\em foo\/} is not a record, function, or array type] - update
expressions are only allowed for these types.

\item[Field not found] - an assignment made in a record update references
a field that was not found in the expected type.

\item[TCC for this expression simplifies to false: \emph{foo}] - the
typechecker has generated a \tcc\ and found that it reduced to
\texttt{false} when it was simplified.  This is a type error, though the
expression \emph{foo} may need careful scrutiny to determine the cause of
the error.

\item[Recursive definition occurrence {\em foo\/} must have arguments] -
the termination of a recursive function is requires every recursive call
to have at least as many arguments as the function given in the measure.

\item[Formals are not allowed in {\em foo\/}s] - this error occurs when
formal declaration arguments are provided where they are not allowed, \eg\
in \texttt{foo(x:int): FORMULA x = y} it makes no sense to provide a
formal argument, as there is no way to reference the formula name with a
parameter.  Formal declaration arguments are only allowed for type,
constant, recursive, and inductive declarations.

\item[Library \emph{foo} does not exist, or has no .pvscontext file] -
this error indicates either that a specified library pathname could not be
found, or that its directory does not contain a .pvscontext file.  Either
modify the pathname, or create the library directory and files (you will
need to use the \unix\ \texttt{mkdir} command to create the directory,
and the \iecmd{change-context} command to move to that directory and
create the files).

\item[The argument to \texttt{PROJ\_}$n$ must be a tuple with length at
least $n$] - this error is displayed when a projection function is applied
to an expression that is ``too short,'' in that the type of the expression
is a tuple with fewer than $n$ components.

\item[There must be at least $n$ arguments for \texttt{PROJ\_}$n$] - this
error occurs with expressions such as \texttt{PROJ\_3(1,2)}.

\item[Record expression assignments must not have arguments] - the
left-hand side of an assignment in a record expression may not have
arguments, \eg\ the second assignment in the expression \texttt{(\# a := 2,
(b)(3) := 1 \#)} is illegal.

\item[Record expressions must have named fields] - the left-hand side of
an assignment in a record expression must be a simple identifier.

\item[Duplicate field assignments are not allowed] - the left-hand sides
of the assignments of a record expression must all be unique simple
identifiers.

\item[Expression type must be a datatype] - in a CASES expression of the
form \texttt{CASES \emph{foo} OF \ldots}, the type of \emph{foo} must be a
datatype.

\item[Selections must have a unique id] - the indicated selection of the
CASES expression was seen earlier.

\item[ELSE part will never be evaluated] - the ELSE part of the CASES
expression was provided, but it will never be evaluated since the
selections cover all possibilities.

\item[No matching constructor found] - the datatype constructor associated
with the specified selection of the CASES expression could not be found.

\item[Wrong number of arguments] - the constructor associated with the
given selection of the CASES expression has a different number of
arguments than the selection.

\item[Incompatible types in application] - this error indicates that no
type could be found that is consistent with the operator and the
arguments.  This error should rarely occur; more specific errors will
usually be given.

\item[Type mismatch in coercion \emph{foo} Possible expression types:
\emph{type}, \ldots] - this error indicates that the type given in the
coercion expression does not match the type found for the expression.  A
simple example of this is \texttt{3:bool}.

\item[Type mismatch in application \emph{foo} Operator types:
\emph{type},\ldots Arguments types:] \textbf{\emph{type},\ldots} - This indicates
that the type of the operator and the arguments are incompatible.

\item[Must resolve to a record, function, or array] - the specified
expression of the update expression could not be resolved to one of these
types.

\item[Duplicate assignments not allowed in recordtypes] - an update on an
expression whose type is a record type was found to have two assignments
with the same left-hand sides.

\item[Field name expected] - the left-hand side of an assignment of an
update expression whose type is a record type must be a field name of that
record type.

\item[May not have partial updates for dependent record types] - a partial
update is one of the form \texttt{r WITH [(a)(b) := x]}, \ie\ an update of
a functional at a specific point.  This is not allowed for dependent
record types, mostly due to the fact that the resulting forms (\eg\
\tccs)are very difficult to understand.

\item[Wrong number of assignment arguments, $n$ expected] - in an update
assignment such as the one in \texttt{f WITH [(a$_1$,\ldots,a$_n$) := y]},
the size of the domain of the type of {\tt f} was different from $n$.
This can also happen recursively, \eg\ the following is an error:
{\small
\begin{alltt}
  (LAMBDA (x:int): (LAMBDA (y,z:int): x + y * z)) WITH [(0)(1,2,3) := 0]
\end{alltt}}

\item[The expression type is inconsistent with this set of arguments] -
this error occurs in an update expression when there are more arguments in
the left-hand side of an assignment than are allowed by the type of the
expression being updated.  An example is
{\small
\begin{alltt}
  (LAMBDA (x:int): x + 1) WITH [(0)(1,2,3) := 0]
\end{alltt}}

\item[Variable \emph{foo} not previously declared] - this error indicates
that a binding was given without a type and there was no earlier variable
declaration with that name.  For example, \texttt{(FORALL x: p(x))}, where
\texttt{x} is not previously declared as a variable (in the current
theory).

\item[Variable \emph{foo} is ambiguous] - the type of the specified
variable was not given, and there is more than one variable declaration
with the same name.

\item[{\em foo\/} does not have a unique type - one of: {\em
type$_1$\/},\ldots] - this error pertains to expressions that are not
names that are still ambiguous.  The simplest way to disambiguate these
expressions is to add a colon followed by the type.  Parentheses may be
necessary as the colon binds tightly (\eg\ \texttt{x - 1:nat} is parsed as
\texttt{x - (1:nat)}).

\item[Incompatible types Found: {\em ftype$_1$},\ldots Expected: {\em
etype\/}] - this message is displayed when there is a type mismatch.  The
types are displayed in an expanded form, so that differences may be found.
However, they are not fully expanded, as this can lead to extremely large
displays and the information is just as useless as if it was unexpanded.
If the types look identical, then you will need to do a careful analysis
of the types of the entities involved to figure out where the discrepancy
lies.  In practice, this is rarely a problem.

\item[Bound variable outside of context] - \memo{Should this still be
possible?}

\end{description}


\subsection{Name Resolution Errors}

\begin{description}

\item[{\em foo\/} does not uniquely resolve - one of: {\em
fullname$_1$},\ldots] - this error indicates that there is an ambiguity in
the use of a name.  A name may be made unambiguous by adding one or more
of the following.
\begin{itemize}
\item theory identifier
\item actual parameters
\item type
\item library identifier
\end{itemize}
The complete name of an entity is of the form
\texttt{\emph{lib}@\emph{th}[\emph{act},\ldots].\emph{id}:\emph{type}}.
See the language manual for more information.

\item[Ambiguous theory instance, could be one of:
\emph{theoryname},\ldots] - the name expression is ambiguous in that
multiple instances are available, and the typechecker is not able to
decide between them.  This usually happens when multiple instances of the
same theory are imported, and the actual parameters were left off of a
name reference.  To continue, simply supply the correct actual parameters.

\item[Could not determine the full theory instance] - this error indicates
that the typechecker was only partially able to resolve the actual parameters
of the specified name.  This frequently happens when a theory is imported
generically, and some of the parameters (usually the type parameters)
could be uniquely determined, while others could not.  To continue, simply
supply the correct actual parameters.

\item[Expression provided where a type is expected] - this error indicates
that an actual parameter is an expression, but the corresponding formal
parameter expects a type expression.

\end{description}


\subsection{Datatype Errors}

\begin{description}

\item[Duplicate enumeration names are not allowed] - the names used in an
enumeration type declaration must all be unique.

\item[Duplicate constructor names are not allowed] - the constructor names
of a datatype must all be unique.

\item[Duplicate recognizer names are not allowed] - the recognizer names
of a datatype must all be unique.

\item[{\em foo} is used as both a constructor and recognizer name] - the
constructor names must be disjoint from the recognizer names.

\item[Datatype name may not be used as a constructor name] - the
identifier used for the datatype cannot be used as a constructor.

\item[Datatype name may not be used as a recognizer name] - the
identifier used for the datatype cannot be used as a recognizer.

\item[May not have duplicate accessor names for a single constructor] -
within a single constructor form, the accessor names must be unique.

\item[Datatype name may not be used as an accessor name] - the
identifier used for the datatype cannot be used as an accessor.

\item[There must be at least one non-recursive constructor] - at least one
of the constructor forms must have no reference to the datatype itself;
see the language manual for more discussion.

\item[Recursive uses of the datatype {\em foo\/} must be (a subtype of)
one of the forms] \textbf{\emph{foo}, sequence[\emph{foo}], or a
(positive) parameter to another datatype} - see the language manual for a
full discussion of this restriction.

\item[Assumings are not allowed for in-line datatypes] - since formal
parameters are not allowed for in-line datatypes, it makes no sense to
have an assuming section.

\end{description}


\subsection{Modify Declaration Errors}
\index{modify-declaration@\texttt{modify-declaration}!errors}

These errors arise during the use of the \ecmd{modify-declaration}
command, which has many restrictions on its use.  One way around many of
these restrictions is to use \iecmd{add-declaration} instead, and to
delete the declaration that you wanted to modify later (after finishing
with any proof in progress).

\begin{description}

\item[Modified declaration must have the same type - \emph{foo}] - the
type of the declaration edited using the \ecmd{modify-declaration}
command must not be changed.

\item[Only one declaration is allowed] - you may not add new declarations
using the \ecmd{modify-declaration} command.  Use \ecmd{add-declaration}
instead, after modifying the given declaration.

\item[Modified declaration must keep the same id - \emph{foo}] - an
attempt has been made to change the identifier associated with the
declaration being modified.  Change it back to what it was and try again.

\item[Modified declaration must be the same kind - \emph{foo}] - you may
not change the class of a declaration being modified, \eg\ from a formula
to a variable.  Change it back and try again.

\end{description}

\section{\LaTeX\ Errors}

There are two kinds of \LaTeX\ errors, those that arise in reading the
\LaTeX\ substitution files (\texttt{pvs-tex.sub}), and those that are
generated as a result of executing \LaTeX.  We will not go into the latter
here, see the \LaTeX\ manual~\cite{latex2e} for help in understanding these
errors.  \memo{Describe the presentation of the errors}

The substitution errors are described below, see
section~\ref{latex-output}, page~\pageref{latex-output} for a full
definition of the terms used.

\begin{description}

\item[Incomplete substitution defined for \emph{foo}] - indicates that the
entry for \emph{foo} is incomplete; there should be four separate fields.

\item[Illegal length \emph{arg} for \emph{foo}] - indicates that the width
field was illegal for the entry for \emph{foo}; it must either be a
positive integer or a hyphen.

\item[Illegal kind or arity \emph{foo} for \emph{id}: must be KEY, ID,
number, or list of numbers] - the kind field was not one of the legal
values.

\end{description}
