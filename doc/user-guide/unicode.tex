% Document Type: LaTeX
% Master File: user-guide.tex
\chapter{Unicode}
\label{unicode}

Unicode may be used in PVS specifications.  For example,
Figure~\ref{unicode-ex} is a valid PVS theory.

\pvstheory{unicode-ex.pvs}{Example Unicode Theory}{unicode-ex}

Most characters may be used in identifiers; a few are recognized as
operators and, like \texttt{+}, will always be separate tokens.  These are
discussed below.

The inclusion of Unicode has several aspects:
\begin{itemize}
\item PVS Grammar: Where are Unicode characters allowed?
\item Display: How to render Unicode characters
\item Input: How to insert Unicode characters
\item LaTeX: How to include PVS specs in LaTeX
\end{itemize}

\section{Unicode in PVS Syntax}

Most Unicode characters are treated the same as alpha-numeric characters.
The exceptions are listed below.

\section{Unicode Operators}

These are the operators specially recognized by PVS, as either special,
unary, binary infix, or bracketing operators.  All other Unicode
characters are treated as letters within identifiers.  Unary and binary
operators have precedence; most are grouped and are in the same precedence
as similar existing PVS operators.  Rather than repeat the precedence
information in the PVS Language manual, we simply indicate one of the
operators that share the precedence.  Of course, when in doubt, it's
easiest to add parentheses.

Note that most of these operators have no definition, they are provided
for making new definitions that are more readable or closer to standard
mathematical usage.  Those that have definitions in the PVS prelude are
indicated, but even those can be redefined, as PVS supports overloading.

\subsection{Alias Symbols}

\begin{center}
\smaller
\topcaption{Unicode Aliases} \label{unicode:aliases}
\tablehead{\hline\rmfamily\bfseries Uni & Input & Equiv & \larger{Unicode Name} \\ \hline}
\begin{xtabular}{|ll>{\ttfamily}l>{\smaller\ttfamily}l|}\hline
  $λ$ & \verb|\lambda| & LAMBDA & GREEK SMALL LETTER LAMDA \\
  $∀$ & \verb|\forall| & FORALL & FOR ALL \\
  $∃$ & \verb|\exists| & EXISTS & THERE EXISTS \\
  $⇔$ & \verb|\iff| & IFF & LEFT RIGHT DOUBLE ARROW \\
  $⇒$ & \verb|\implies| & IMPLIES & RIGHTWARDS DOUBLE ARROW \\
  $∨$ & \verb|\or| & OR & LOGICAL OR \\
  $∧$ & \verb|\and| & AND & LOGICAL AND \\
  $¬$ & \verb|\not| & NOT & NOT SIGN \\
  $≠$ & \verb|\neq| & /= & NOT EQUAL TO \\
  $∘$ & \verb|\circ| & o & RING OPERATOR \\
  $§$ & \verb|\section| & ; & SECTION SIGN \\ \hline
\end{xtabular}
\end{center}
  
Note that capital lambda Λ is not the same as λ, and is available for
identifiers.

§ (input \verb|\section|) is used as an alternative to semi-colon (;) as a
way of separating declarations.

\subsection{Unary operators}

Unary operators may be used without parentheses, e.g., ``□φ'' is a valid
term, and equivalent to ``□(φ)''.

\begin{center}
\smaller
\topcaption{Unicode Unary Operators} \label{unicode:unaryops}
\tablehead{\hline\textrm{\textbf{Uni}} & \textrm{\textbf{Input}} & %
  \textrm{\textbf{\larger{Unicode Name}}} \\ \hline}
\begin{tabular}{|ll>{\smaller\ttfamily}l|}\hline
  \multicolumn{3}{|l|}{\bfseries Precedence same as for \texttt{<>}} \\ \hline
  $□$ & \verb|\Box| & WHITE SQUARE \\
  $◇$ & \verb|\Diamond| & WHITE DIAMOND \\ \hline
  \multicolumn{3}{|l|}{\bfseries Precedence same as for \texttt{+}} \\ \hline
  $◯$ & \verb|\bigcirc| & LARGE CIRCLE \\
  $√$ & \verb|\surd| & SQUARE ROOT \\ \hline
\end{tabular}
\end{center}

Note that because of declaration parameters, the old '[]' operator is no
longer allowed.

¬ is equivalent to NOT, unless redeclared

◯ is both unary and binary — much like '+' and '-'.  It is unary because
it is useful as a possible 'next' operator.

√ is also both unary and binary and has the same precedence
as unary '+' and '-'.  It's an obvious candidate for the 'sqrt' operator.

\subsection{Binary (infix) operators}

The first ones have declarations in the prelude, and are intended as
symbolic equivalents for the given operator, and have the same precedence.

The rest are not declared in the prelude and are shown listed from lowest
to highest precedence.  Precedence is indicated with relative to an existing
operator, given in square brackets.

\begin{center}
\smaller
\topcaption{Unicode Binary Operators} \label{unicode:binops}
\tablehead{\hline\textrm{\textbf{Uni}} & \textrm{\textbf{Input}} & %
  \textrm{\textbf{\larger{Unicode Name}}} \\ \hline}
\tabletail{\hline \multicolumn{3}{|r|}{{Continued on next page}} \\ \hline}
\tablelasttail{\hline \hline}
  \begin{xtabular}{|ll>{\smaller\ttfamily}l|}\hline
  \multicolumn{3}{|l|}{\bfseries Precedence same as for \texttt{|-}} \\ \hline
  $⊢$ & \verb|\vdash| & RIGHT TACK \\
  $⊨$ & \verb|\vDash| & TRUE \\ \hline
  \multicolumn{3}{|l|}{\bfseries Precedence between \texttt{@@} and \texttt{+}} \\ \hline
  $≁$ & \verb|\nsim| & NOT TILDE \\
  $≃$ & \verb|\simeq| & ASYMPTOTICALLY EQUAL TO \\
  $≅$ & \verb|\cong| & APPROXIMATELY EQUAL TO \\
  $≇$ & \verb|\ncong|& NEITHER APPROXIMATELY NOR ACTUALLY EQUAL TO \\
  $≈$ & \verb|\approx| & ALMOST EQUAL TO \\
  $≉$ & \verb|\napprox| & NOT ALMOST EQUAL TO \\
  $≍$ & \verb|\asymp| & EQUIVALENT TO \\
  $≎$ & \verb|\Bumpeq| & GEOMETRICALLY EQUIVALENT TO \\
  $≏$ & \verb|\bumpeq| & DIFFERENCE BETWEEN \\
  $≐$ & \verb|\doteq| & APPROACHES THE LIMIT \\
  $≗$ & \verb|\circeq| & RING EQUAL TO \\
  $≙$ & \verb|\defs| & ESTIMATES \\
  $≡$ & \verb|\equiv| & IDENTICAL TO \\
  $⋈$ & \verb|\Join| & BOWTIE \\
  $≤$ & \verb|\leq| & LESS-THAN OR EQUAL TO \\
  $≥$ & \verb|\geq| & GREATER-THAN OR EQUAL TO \\
  $≦$ & \verb|\leqq| & LESS-THAN OVER EQUAL TO \\
  $≧$ & \verb|\geqq| & GREATER-THAN OVER EQUAL TO \\
  $≨$ & \verb|\lneq| & LESS-THAN BUT NOT EQUAL TO \\
  $≩$ & \verb|\gneq| & GREATER-THAN BUT NOT EQUAL TO \\
  $≪$ & \verb|\ll| & MUCH LESS-THAN \\
  $≫$ & \verb|\gg| & MUCH GREATER-THAN \\
  $≮$ & \verb|\nless| & NOT LESS-THAN \\
  $≯$ & \verb|\ngtr| & NOT GREATER-THAN \\
  $≰$ & \verb|\nleq| & NEITHER LESS-THAN NOR EQUAL TO \\
  $≱$ & \verb|\ngeq| & NEITHER GREATER-THAN NOR EQUAL TO \\
  $≺$ & \verb|\prec| & PRECEDES \\
  $≻$ & \verb|\succ| & SUCCEEDS \\
  $▷$ & \verb|\rhd| & WHITE RIGHT-POINTING TRIANGLE \\
  $◁$ & \verb|\lhd| & WHITE LEFT-POINTING TRIANGLE \\
  $∈$ & \verb|\in| & ELEMENT OF \\
  $∉$ & \verb|\notin| & NOT AN ELEMENT OF \\
  $∋$ & \verb|\ni| & CONTAINS AS MEMBER \\
  $⊂$ & \verb|\subset| & SUBSET OF \\
  $⊃$ & \verb|\supset| & SUPERSET OF \\
  $⊄$ & \verb|\nsubset| & NOT A SUBSET OF \\
  $⊅$ & \verb|\nsupset| & NOT A SUPERSET OF \\
  $⊆$ & \verb|\subseteq| & SUBSET OF OR EQUAL TO \\
  $⊇$ & \verb|\supseteq| & SUPERSET OF OR EQUAL TO \\
  $⊊$ & \verb|\subsetneq| & SUBSET OF WITH NOT EQUAL TO \\
  $⊋$ & \verb|\supsetneq| & SUPERSET OF WITH NOT EQUAL TO \\
  $⊏$ & \verb|\sqsubset| & SQUARE IMAGE OF \\
  $⊐$ & \verb|\sqsupset| & SQUARE ORIGINAL OF \\
  $•$ & \verb|\bullet| & BULLET \\
  $←$ & \verb|\leftarrow| & LEFTWARDS ARROW \\
  $↑$ & \verb|\uparrow| & UPWARDS ARROW \\
  $→$ & \verb|\rightarrow| & RIGHTWARDS ARROW \\
  $↓$ & \verb|\downarrow| & DOWNWARDS ARROW \\
  $↝$ & \verb|\leadsto| & RIGHTWARDS WAVE ARROW \\
  $↦$ & \verb|\mapsto| & RIGHTWARDS ARROW FROM BAR \\
  $⇐$ & \verb|\Leftarrow| & LEFTWARDS DOUBLE ARROW \\
  $⇑$ & \verb|\Uparrow| & UPWARDS DOUBLE ARROW \\
  $⇓$ & \verb|\Downarrow| & DOWNWARDS DOUBLE ARROW \\
  $∇$ & \verb|\nabla| & NABLA \\
  $⊣$ & \verb|\dashv| & LEFT TACK \\
  $⊥$ & \verb|\perp| & UP TACK \\
  $⊩$ & \verb|\Vdash| & FORCES \\
  $◯$ & \verb|\bigcirc| & LARGE CIRCLE \\
  $★$ & \verb|\bigstar| & BLACK STAR \\
  $✠$ & \verb|\maltese| & MALTESE CROSS \\ \hline
  \multicolumn{3}{|l|}{\bfseries Precedence same as for \texttt{+}} \\ \hline
  $⊕$ & \verb|\oplus| & CIRCLED PLUS \\
  $⊖$ & \verb|\ominus| & CIRCLED MINUS \\
  $⨁$ & \verb|\bigoplus| & N-ARY CIRCLED PLUS OPERATOR \\
  $±$ & \verb|\pm| & PLUS-MINUS SIGN \\
  $∓$ & \verb|\mp| & MINUS-OR-PLUS SIGN \\
  $∔$ & \verb|\dotplus| & DOT PLUS \\
  $⊞$ & \verb|\boxplus| & SQUARED PLUS \\
  $⊟$ & \verb|\boxminus| & SQUARED MINUS \\
  $⊎$ & \verb|\uplus| & MULTISET UNION \\
  $∪$ & \verb|\cup| & UNION \\
  $⊔$ & \verb|\sqcup| & SQUARE CUP \\
  $⋁$ & \verb|\bigvee| & N-ARY LOGICAL OR \\
  $⋃$ & \verb|\bigcup| & N-ARY UNION \\ \hline
  \multicolumn{3}{|l|}{\bfseries Precedence same as for \texttt{*}} \\ \hline
  $⊘$ & \verb|\oslash| & CIRCLED DIVISION SLASH \\
  $⊗$ & \verb|\otimes| & CIRCLED TIMES \\
  $⊙$ & \verb|\odot| & CIRCLED DOT OPERATOR \\
  $⊛$ & \verb|\circledast| & CIRCLED ASTERISK OPERATOR \\
  $⨂$ & \verb|\bigotimes| & N-ARY CIRCLED TIMES OPERATOR \\
  $⨀$ & \verb|\bigodot| & N-ARY CIRCLED DOT OPERATOR \\
  $×$ & \verb|\times| & MULTIPLICATION SIGN \\
  $÷$ & \verb|\div| & DIVISION SIGN \\
  $⊠$ & \verb|\boxtimes| & SQUARED TIMES \\
  $∩$ & \verb|\cap| & INTERSECTION \\
  $⊓$ & \verb|\sqcap| & SQUARE CAP \\
  $⋀$ & \verb|\bigwedge| & N-ARY LOGICAL AND \\
  $⋂$ & \verb|\bigcap| & N-ARY INTERSECTION \\
\end{xtabular}
\end{center}

\subsection{Bracketing Operators}

To use these, the left/right pair must be given a declaration (no space
between), then they can be used as brackets.  For example:
\begin{alltt}
  ⌊⌋(x: real): int = floor(x)
  floorex: formula ⌊5.3⌋ = 5
\end{alltt}

\begin{center}
\smaller
\topcaption{Unicode Bracketing Operators} \label{unicode:brackets}
\tablehead{\hline\textrm{\textbf{Uni}} & \textrm{\textbf{Input}} & %
  \textrm{\textbf{\larger{Unicode Name}}} \\ \hline}
\begin{tabular}{|ll>{\smaller\ttfamily}l|}\hline
$⟨ ⟩$ & \verb|\langle \rangle|) & MATHEMATICAL LEFT/RIGHT ANGLE BRACKET \\
$⟦ ⟧$ & \verb|\mlbracket \mrbracket| & MATHEMATICAL LEFT/RIGHT WHITE SQUARE BRACKET \\
$« »$ & \verb|\"< \">| & LEFT/RIGHT-POINTING DOUBLE ANGLE QUOTATION MARK\\
$⟪ ⟫$ & \verb|\mldata \mrdata| & MATHEMATICAL LEFT/RIGHT DOUBLE ANGLE BRACKET \\
$⌈ ⌉$ & \verb|\lceil \rceil| & LEFT/RIGHT CEILING \\
$⌊ ⌋$ & \verb|\lfloor \rfloor| & LEFT/RIGHT FLOOR \\
$⌜ ⌝$ & \verb|\ulcorner \urcorner| & TOP LEFT/RIGHT CORNER \\
$⌞ ⌟$ & \verb|\llcorner \lrcorner| & BOTTOM LEFT/RIGHT CORNER \\ \hline
\end{tabular}
\end{center}

\section{Emacs and Unicode}

In Emacs, non-ASCII characters are normally inserted using an input
method.  The \texttt{PVS-TeX} input method is modified from the
\texttt{TeX} input method, which uses backslash followed by the character
name, and uses TeX names.

See the last section of this for more on Emacs and Unicode.

Summary:
  M-x describe-char — describes the character under the cursor
  M-x set-input-method (C-x RET C-\\) — sets the input method
  M-x toggle-input-method (C-\\) — toggles between input methods
  M-x describe-input-method (C-h I) — shows the input sequences
         for all the characters handled by the method

The first thing to do is to get a font that supports Unicode.  If you're
reading this in Emacs, and the symbols are rendered correctly, then you're
set.  Otherwise, both Mac and Linux have a wide range of fonts available -
search on the internet for options.  Even if the symbols do render
correctly, another font may do a better job.

If the display is right, you can find out information about any character
by putting the cursor on that character and typing 'M-x describe-char',
which brings up a buffer containing information about the character,
including its Unicode name and input sequences.

To input a particular Unicode character, you can use 'C-x 8 RET' and type in
the name or hex value — the internet can be used to find the character
you're looking for.  However, it is usually easier to use an input
method.  Most input methods are for particular languages; the most useful
one for the mathematical fonts described above is TeX, which mostly
follows the TeX commands for the same character.  Use M-x set-input-method
to select an input method, and M-x toggle-input-method to switch between
it and the previous one.
