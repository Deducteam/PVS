% Document Type: LaTeX
% Master File: user-guide.tex
\chapter{Introduction to Emacs}
\label{emacs-intro}

The PVS system uses the GNU Emacs system as its user interface.  To make
effective use of PVS, you must become familiar with at least the basic
Emacs commands.  This section provides an introduction to Emacs that
should allow you to get started in PVS right away.  This Appendix
introduces enough of the basic ideas and commands of Emacs to use PVS, but
to become effective you really should consult the GNU Emacs
Manual~\cite{emacs20}.  It is also quite helpful to run through the online
tutorial.  To do this, start up PVS or Emacs, type \key{C-h t}, and follow
the instructions.

Emacs provides the primary interface to the PVS system.  We chose Emacs
as our interface for a number of reasons.  First, it is freely available,
and runs on a large number of different platforms.  It is also quite
popular; on Unix systems the \texttt{vi} editor is probably the only
editor that is used more than Emacs, but it is too limited to use as a
general-purpose interface.  In particular, it has no support for running
subprocesses.

Emacs is an extremely flexible editor, and includes a built in programming
language (Emacs Lisp) which makes it easy to increase its functionality.
There is a cost associated with this.  First, Emacs is rather large, and
takes longer to start up than smaller editors such as \texttt{vi}.  Emacs
is also quite complex, with a large number of commands and associated key
bindings that are not easy to learn.

Emacs is significantly different than other editors.  In most editors, you
start the editor, get a file, make some changes, save the file, and exit.
There is a tendency to think in terms of ``leaving'' the current file in
order to go to the next. To handle multiple files in a single session
usually requires extra care and some specialized commands.  For example,
\texttt{vi} can only focus on one file at a time, with one alternate.

In Emacs multiple buffers may be open at once, and as many can be made
visible as your screen allows.  Unlike other editors, buffers are not all
associated with files.  It is not unusual to have over a hundred buffers
associated with a single Emacs session.  It is also quite normal to have
the same Emacs session up for weeks at a time.\footnote{Some people have
even been known to use Emacs as their shell.}  When you are done editing
and saving a given file, you do not exit from that buffer, you simply go
on to the next one.

Unlike \texttt{vi}, there is no command mode.  By default an Emacs buffer
is in insert mode, and most keys on the keyboard simply insert themselves.
Emacs has a large number of interactive commands, any of which may be
bound to a key or key sequence.\footnote{It turns out that typing a letter
key actually invokes the command \texttt{self-insert-command}.}  Any
interactive command may be invoked by typing \texttt{M-x} followed by the
command.  Recall that \texttt{M-x} is gotten by holding down the
\texttt{Meta-} key and typing an \texttt{x}.  If you don't have a key
labeled \texttt{Meta-}, then look for and try the \texttt{Alt} or
$\Diamond$ keys.  If you really don't have a \texttt{Meta-} key, then the
\texttt{Esc} key will do, but in this case you must release the
\texttt{Esc} key before typing the \texttt{x}.

Commands may be bound to key sequences, in order to make typing easier.
For example, to page down repeatedly by typing \texttt{M-x
scroll-up} over and over would get quite tedious, so the key sequence
\texttt{C-v} was bound to this command.  This and most of the key bindings
of Emacs are not particularly mnemonic, but once learned they are
extremely effective.  With a little practice you will find that you don't
think about what key sequence is needed to get a particular effect---your
hands just do it automatically.

Each buffer in Emacs has an associated major mode, and any number of minor
modes.  The major mode indicates what kind of a buffer it is, and
generally defines the key bindings and functions associated with the
buffer.  This is usually determined from the file extension, so for
example the file \texttt{foo.pvs} is in \texttt{pvs-mode}, while a file
\texttt{foo.c} would normally be in \texttt{c-mode}.  Minor modes modify
the major mode.  Examples include \texttt{auto-fill-mode} and
\texttt{overwrite-mode}.

When you start up PVS, you will see the \texttt{PVS Welcome} buffer, which
takes up most of the window.  Toward the bottom of the window you will see
a line in inverse video; this is the mode line.  The last line of the
window is the minibuffer.  If you are running Emacs version 19 (or later) under X
windows, then you will see a menu line at the top of the window, and a
scroll bar to the right.  If you display more than one buffer in the
window, then the bottom of each buffer will have a mode line displaying
information for that buffer.  There will still be only one minibuffer,
however.\footnote{In Emacs 19 (and later versions), it is possible to have multiple windows,
called frames, associated with a single Emacs session.  In this case, each
frame by default has its own copy of the minibuffer.  See the Emacs manual
for more details.}

The mode line provides information relating to the buffer above it.  The
first five characters indicate whether the buffer is read-only, and
whether the buffer has changed relative to the file.  If you see
\texttt{---\%\%-}, then the file is read-only, and you will not be allowed
to modify it.  Sometimes this is set when you have copied a file from
somewhere else, and you think you should be able to make modifications.
In that case, use the \texttt{toggle-read-only} command, make your
changes, and save the file.  Emacs may still ask whether it should try to
save the file anyway, go ahead and answer yes in this case.

If the mode line is 5 dashes (\texttt{-----}), then the file can be
modified but has not yet been changed.  Once modified, the mode line
changes to \texttt{--**-}.  If you did not intend to modify the file, then
use the \texttt{undo} commands described below to undo your change.  If there are a
few changes, you may need to repeat the \texttt{undo} command until they
are all backed out.  If there are a lot of changes, then \texttt{M-x
revert-buffer} may be used to restore the buffer from the original file.
The other information in the mode line is the buffer name, possibly the
time, the mode of the buffer in parentheses, and the amount of the buffer
currently displayed.  Like everything else in Emacs, the mode line is
customizable; see the Emacs manual for details.

The minibuffer is used to display messages, echo longer commands as they
are typed in, and prompt for arguments.  Many of these arguments support
\emph{completion}, which means that you can type the start of an argument
and type a \texttt{TAB} to have it automatically filled in.  Emacs will
fill in as much as is unique, and then wait for more input.  If it is
ambiguous already, Emacs will pop up a buffer with the possible
completions in it.  You can force it to show all possible completions by
typing a \texttt{?}.  Not all arguments support completion, but it is
usually worthwhile to try typing a \texttt{TAB} after typing the start of
an argument to see if completion is supported; if it is then you will
either get a pop up buffer or a (partial) completion of what you typed.
Otherwise you will simply get a \texttt{TAB} inserted.

Each buffer has associated with it a current \emph{region}, which is
defined by two different locations within the buffer, called \emph{point}
and \emph{mark}.  Point is normally the cursor position, so any of the
cursor motion commands automatically move point.  Mark is not directly
displayed; it is set by command, and does not move until another mark
setting command is issued.  There are a large number of Emacs commands
that work on regions, though by far the most common usage is for cutting
and pasting operations.

%PVS uses \emacs\ as its basic interface, but does not attempt to ``take
%over'' the system---all the underlying capabilities of \emacs\ are still
%available.  Thus you can read your email, scan the news bulletins, edit
%non-PVS files, or execute commands in a shell buffer in the usual way.
%Many of the PVS commands allow you to do these while waiting for the
%command to finish.  This is especially useful for typechecking large
%specifications or while running proofs in the background.


\section{Leaving Emacs}
\begin{pvscmds}
\icmd{save-buffers-kill-emacs} & \key{C-x C-c} & Kill Emacs \\
\end{pvscmds}

This command exits Emacs, after first prompting whether to save each
modified file.


\section{Getting Help}
\begin{pvscmds}
\icmd{info} & \key{C-h i} & Read Emacs documentation\\
\icmd{help-with-tutorial} & \key{C-h t} & Display the Emacs tutorial \\
\icmd{command-apropos} & \key{C-h a} & Show commands matching a string \\
\icmd{describe-key} & \key{C-h k} & Display name and documentation a key
runs \\
\icmd{describe-function} & \key{C-h f} & Display documentation for
function \\
\icmd{describe-bindings} & \key{C-h b} & Display a table of key bindings
\\
\end{pvscmds}

These commands provide help.  When you type the \key{C-h} prefix key, you
are prompted for the next key, and can type \texttt{?} to find out all the
possibilities---only a few are described here.

The \cmd{info} command brings up a buffer containing the Emacs online
documentation.  Type \texttt{m} followed by a topic name to view the info
page for that topic, or click mouse button 2 over the highlighted name.

The \cmd{help-with-tutorial} command brings up an Emacs tutorial.  This
tutorial is interactive, inviting you to try out the commands as it
describes them.  If you are new to Emacs, you should try to go through
this at least once.

The \cmd{command-apropos} command displays a list of those commands whose
names contain a specified substring.  This is helpful if you know only
part of a command name, or suspect there is some command available for
performing some task, but do not know what it might be named.  For
example, you might do an \key{C-h a} on \texttt{mail} to find out what
mail commands are available.  If you know the beginning of a
command, it is usually better to simply start typing the command and use
the completion mechanism.

The \cmd{describe-key} and \cmd{describe-function} commands provide the
same information, but one prompts for a key and the other for a command
(with completion).  The key is not necessarily a single
keystroke, as some keystrokes are defined to be prefix keys.  In this case
the key typed so far will be displayed in the minibuffer, and the function
description will not be given until a completed key sequence has been
typed.

The \cmd{describe-bindings} command displays the key bindings in effect in
a separate buffer.  Many of these key bindings are specific to
the buffer mode, so issuing this command from different buffers will
generally lead to different results.


\section{Files}
\begin{pvscmds}
\icmd{find-file} & \key{C-x C-f} & Read a file into Emacs \\
\icmd{save-buffer} & \key{C-x C-s} & Save a file to disk \\
\end{pvscmds}

The file commands are needed to read a file into Emacs and save the
changes.  The \cmd{find-file} creates a new buffer with the same name as the
file and reads the file contents into it.  Completion is available on the
file name, including the directory.  If the file does not exist, then an
empty buffer is created.  Note that the buffer is not the same as the
file, and changes made to the buffer are not reflected in the file until
the file is saved.

The \cmd{save-buffer} command saves the current buffer to file.  If the
current buffer is not associated with a file, you are prompted to give a
file name.


\section{Buffers}
\begin{pvscmds}
\icmd{switch-to-buffer} & \key{C-x b} & Select another buffer \\
\icmd{list-buffers} & \key{C-x C-b} & List all buffers \\
\icmd{kill-buffer} & \key{C-x k} & Kill a buffer \\
\end{pvscmds}

The \texttt{switch-to-buffer} command is used to switch control from one
buffer to another.  When you type the command, you will be prompted for a
new buffer to switch to, and a default will be given.  If the default is
the right one, simply type the return key.  Otherwise type in the name of
the buffer you desire.  Completion is available.  If the buffer specified
does not already exist, then it is created.

The \texttt{kill-buffer} command is used to remove a buffer.  Completion
is available.  Note that some buffers have processes associated with them,
and killing that buffer also kills the associated process.  In particular,
the \texttt{*pvs*} buffer is associated with the PVS process.

The \texttt{list-buffers} command lists all the buffers currently
available, along with an indication of whether the buffer has changed,
its size, its major mode, and its associated file, if any.


\section{Cursor Motion commands}
\begin{pvscmds}
\icmd{forward-char} & \key{C-f} & Move forward one character \\
\icmd{backward-char} & \key{C-b} & Move backward one character \\
\icmd{forward-word} & \key{C-f} & Move forward one word \\
\icmd{backward-word} & \key{C-b} & Move backward one word \\
\icmd{next-line} & \key{C-n} & Move down one line vertically \\
\icmd{previous-line} & \key{C-p} & Move up one line vertically \\
\icmd{beginning-of-line} & \key{C-a} & Move to the beginning of the line \\
\icmd{end-of-line} & \key{C-e} & Move to the end of the line \\
\icmd{beginning-of-buffer} & \key{M-<} & Move to the beginning of the
buffer \\
\icmd{end-of-buffer} & \key{M-<} & Move to the end of the buffer \\
\end{pvscmds}

These are largely self explanatory; the best way to get used to what they
do is to simply try them out.  Note that, depending on your Emacs
environment, you may have appropriate key bindings for the arrow, Home,
PageUp, etc. keys.\footnote{As described above, you can find out what
these are bound to by typing \key{C-h k} followed by the key.}

\section{Error Recovery}
\begin{pvscmds}
\icmd{keyboard-quit} & \key{C-g} & Abort partially typed or executing
command \\
\icmd{undo} & \key{C-x u}, \key{C-\_} & Undo one batch of changes \\
\icmd{revert-buffer} & & Revert the buffer to the file contents \\
\icmd{recenter} & \key{C-l} & Redraw garbaged screen \\
\end{pvscmds}

\key{C-g} is used if you start to type a command and change your mind, or
a command is running and you want to abort it.  Sometimes it takes two or
three invocations before it has the desired effect.  For example if you
started an incremental search, the first \texttt{C-g} erases some of the
input and the second actually quits the incremental search.

The \cmd{undo} command is the normal way to undo changes made to the
current buffer.  If you undo twice in a row, then the last two changes are
undone.  In this manner you can eventually undo all the changes made to a
buffer.  Once you type something other than an undo, all the previous undo
commands are treated as changes that themselves can be undone.

If you made a large number of changes to a file buffer and simply want to
go back to the original file contents, use \ecmd{revert-buffer}.  Note that if you
have changed the file and saved it, then reverting will bring back the
saved version, not the earlier one.


\section{Search commands}
\begin{pvscmds}
\icmd{isearch-forward} & \key{C-s} & Incremental search forward \\
\icmd{isearch-backward} & \key{C-r} & Incremental search backward \\
\end{pvscmds}

These search through the text for a specified string.  The search is
incremental in that it starts searching as soon as a character is typed
in, finding the first occurrence of the string typed in so far.  If the
string can't be found, the minibuffer changes its prompt from
\texttt{I-search:} to \texttt{Failing I-search:}.  If it finds the string,
but you wish to go on to the next (previous) occurrence, type another
\key{C-s} (\key{C-r}).  To terminate the search, type the Enter key, or
any other Emacs command.  Consult the Emacs manual for other useful
options available for search.


\section{Killing and Deleting}
\begin{pvscmds}
\icmd{delete-next-character} & \key{C-d} & Delete next character \\
\icmd{delete-backward-char} & \key{DEL} & Delete previous character \\
\icmd{kill-word} & \key{M-d} & Kill word \\
\icmd{backward-kill-word} & \key{M-DEL} & Kill word backwards \\
\icmd{kill-line} & \key{C-k} & Kill rest of line \\
\icmd{kill-region} & \key{C-w} & Kill region \\
\icmd{copy-region-as-kill} & \key{M-w} & Save region a killed text
without killing \\
\end{pvscmds}

These commands delete or kill the specified entities.  The difference
between killing and deleting is that a killed entity is copied to the kill
ring, and can be \emph{yanked} later, while deleted entities are not.  The
kill ring is a stack of blocks of text that have been killed, with the
most recent kills at the top.  The kill ring is not associated with any
specific buffer, and in this respect acts much like a \emph{clipboard}
found in most window systems.

The \cmd{delete-backward-char} command is frequently mapped onto the
\key{Backspace} key instead; you may need to experiment with this. If
you want it mapped to the \key{Backspace} key, it is usually easier to map
it outside of Emacs, for example using the \texttt{xmodmap} command.  This
is because by default the \key{Backspace} key and the \key{C-h} key are
indistinguishable by Emacs, and the \key{C-h} key is used for accessing
various Emacs help functions.

The \cmd{kill-line} command kills from the current cursor location to the
end of the line, unless it is already at the end of the line, in which
case it kills the newline, thus merging the current line with the
following one.

The \cmd{copy-region-as-kill} command is similar to the \cmd{kill-region}
command, but does not actually kill any text.  This is useful when trying
to copy text from a file for which you do not have write access, since
Emacs will not let you modify such a buffer without first changing its
read-only status.

\section{Yanking}
\begin{pvscmds}
\icmd{yank} & \key{C-y} & Yank last killed test \\
\icmd{yank-pop} & \key{M-y} & Replace last yank with previous kill \\
\end{pvscmds}

The \cmd{yank} command puts the text of the most recent kill command into
the buffer at the current cursor position.  Note that the usual way to
move text from one place to another in Emacs is to kill it, move the
cursor to the new location, and yank it.  Because the kill ring is
globally used, this works across buffers as well.

The \cmd{yank-pop} command may only be used after the \cmd{yank} command,
and has the effect of replacing the yanked text with earlier killed text
from the kill ring.

\section{Marking}
\begin{pvscmds}
\icmd{set-mark-command} & \key{C-\char064}, \key{C-SPC} & Set mark here \\
\icmd{exchange-point-and-mark} & \key{C-x C-x} & Exchange point and mark
\\
\end{pvscmds}

The \cmd{set-mark} command sets the mark to the current cursor position.
Since point is also at the current cursor position, this defines an empty
region initially.  As the cursor is moved away from the mark position, the
region grows.  Note that the relative positions of mark and point do not
matter; the region is defined as the text between these two positions.

\key{C-x C-x} is used to exchange the point and mark positions, moving the
cursor to where mark was last set, and setting mark to the old cursor
position.  Doing this again puts mark and point back where they started.
This is useful for checking that the region is as desired, before doing
any destructive operations.
