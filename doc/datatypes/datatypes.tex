\documentclass[11pt,twoside]{book}
\usepackage{relsize,url,makeidx,alltt}
\usepackage{fancyhdr}
\usepackage{fancyvrb}
\usepackage[chapter]{tocbibind}
\makeatletter
\@ifundefined{pdfoutput}%
{\usepackage[bookmarks=true,hyperindex=true]{hyperref}}%
{\usepackage[pdftex,dvipsnames,usenames]{color}%
\usepackage[bookmarks=true,hyperindex=true,colorlinks=true,linkcolor=Brown,citecolor=blue,backref=page,pagebackref=true,plainpages=false,pdfpagelabels]{hyperref}%
}%
\makeatother
\makeindex
\sloppy
% full list of sections:
%\includeonly{title,intro,informal,pspace,undecide,conclu,ack,rules}
\setlength{\textwidth}{6in}
\setlength{\evensidemargin}{0in}
\setlength{\oddsidemargin}{.5in}
%\pagestyle{myheadings} % page number in upper right corner
%\markboth{Specification and Verification}{}
%\makeindex
%\setcounter{secnumdepth}{4} 
%\setcounter{tocdepth}{4}
\newenvironment{smalltt}{\begin{alltt}\small\tt}{\end{alltt}}
\newenvironment{stdisplay}{\begin{pagegroup}\begin{smalltt}}{\end{smalltt}\end{pagegroup}}
\newenvironment{tdisplay}{\hozline\begin{alltt}\footnotesize\tt\vspace{0.3\baselineskip}}{\vspace{0.3\baselineskip}\end{alltt}\hozline}
%\renewcommand{\baselinestretch}{2}
\newcommand{\tfuntype}[2]{[#1 -> #2]}
\newcommand{\mathfuntype}[2]{[#1 \aro #2]}
%\renewcommand{\baselinestretch}{2}
\newenvironment{display}{\begin{alltt}\small\tt\vspace{0.3\baselineskip}}{\vspace{0.3\baselineskip}\end{alltt}}
\newcommand{\choice}{[\!]}  \newcommand{\normtt}[1]{{\obeyspaces \texttt{#1
}}} \newenvironment{pagegroup}{}{}
%\newenvironment{smalltt}{\begin{alltt}\small\tt}{\end{alltt}}
%\newenvironment{tdisplay}{\begin{alltt}\footnotesize\tt\vspace{0.3\baselineskip}}{\vspace{0.3\baselineskip}\end{alltt}}
% % zero-width inference rule markers
 % (if they have width, they throw off centering)
\newcommand{\zwb}[1]{\makebox[0cm][l]{{\scriptsize $#1$}}}

\newcommand{\mall}{{\sc mall}}
\newcommand{\ol}{\overline}
\newcommand{\ul}{\underline}
\newcommand{\lan}{\langle}
\newcommand{\ran}{\rangle}

 % Linear Arrow
\newcommand{\la}{\mbox{$-\!\circ$}} 

 % Backwards Linear Arrow
\newcommand{\bla}{\mbox{$\circ\!-$}} 

 % Par
%\newcommand{\Par}{\,{\parfont P}\,}
\newcommand{\Par}{\mathrel{\wp}}

 % With (Linear And) (&)
\newcommand{\with}{\,{\&}\,}

% Old par
% \newcommand{\Par}{\mbox{\raisebox{.4ex}{$\wp$}}} 

 % PDL's prettier bounded reuse operator
\newcommand{\bang}[1]{\stackrel{\scriptstyle \dagger}{\scriptstyle #1}}
 
 % Implies in Relevant Implication
\newcommand{\ra}{\mbox{$\rightarrow_{R_{\rightarrow}}$}}

 % Relevant Implication (Logic)
\newcommand{\rara}{\mbox{${R_{\rightarrow}}$}}

 % Multiplicative Linear Logic  
\newcommand{\lala}{\mbox{$LL_{\la, \otimes, !}$}}

 % (Propositional) Circular Logic
\newcommand{\cl}{\mbox{ CL }}

 % (Propositional) Multiplicative Circular Logic
\newcommand{\mcl}{\mbox{ MCL }}

 % Definition
\newcommand{\define}{\mbox{${\stackrel{\Delta}{=}}$}}

 % Derived From
\newcommand{\derive}{\mbox{${\stackrel{\vdots}{\vdash}}$}}

 % Semi-Thue Reduction
\newcommand{\stred}[1]{\mbox{$\Longrightarrow$}}

\newcommand{\stredstar}[1]{\mbox{$\Longrightarrow^*$}}

\newcommand{\deltared}[1]{\mbox{$\, \stackrel{\delta_{#1}}
                                             {\longrightarrow} \, $}}

\newcommand{\deltaredp}[1]{\mbox{$\, \stackrel{\delta'_{#1}}
                                             {\longrightarrow} \, $}}


%% Derived from John Rushby's prelude.tex, modified for NFSS2
%
% define variants of the \LaTeX macro that avoid using \sc
% for use in headings
%

% Define fonts that work in math or text mode
\def\dwimrm#1{\ifmmode\mathrm{#1}\else\textrm{#1}\fi}
\def\dwimsf#1{\ifmmode\mathsf{#1}\else\textsf{#1}\fi}
\def\dwimtt#1{\ifmmode\mathtt{#1}\else\texttt{#1}\fi}
\def\dwimbf#1{\ifmmode\mathbf{#1}\else\textbf{#1}\fi}
\def\dwimit#1{\ifmmode\mathit{#1}\else\textit{#1}\fi}
\def\dwimnormal#1{\ifmmode\mathnormal{#1}\else\textnormal{#1}\fi}

\def\BigLaTeX{{\rm L\kern-.36em\raise.3ex\hbox{\small\small A}\kern-.15em
    T\kern-.1667em\lower.7ex\hbox{E}\kern-.125emX}}
\def\BoldLaTeX{{\bf L\kern-.36em\raise.3ex\hbox{\small\small\bf A}\kern-.15em
    T\kern-.1667em\lower.7ex\hbox{E}\kern-.125emX}}
%\def\labelitemi{$\bullet$}
\def\labelitemii{$\circ$}
\def\labelitemiii{$\star$}
\def\labelitemiv{$\diamond$}
\newcommand{\tcc}{{\small\small TCC}}
\newcommand{\tccs}{\tcc s}
\newcommand{\emacs}{{Emacs}}
\newcommand{\Emacs}{\emacs}
\newcommand{\ehdm}{{E{\small\small HDM}}}
\newcommand{\Ehdm}{\ehdm}
\newcommand{\tm}{$^{\mbox{\tiny TM}}$}
\newcommand{\hozline}{{\noindent\rule{\textwidth}{0.4mm}}}

\newcommand{\allclear}%
  {\mbox{\boldmath$\stackrel{\raisebox{-.2ex}[0pt][0pt]%
              {$\textstyle\oslash$}}{\displaystyle\bot}$}}

\newenvironment{private}{}{}

\newenvironment{smalltt}{\begin{alltt}\small}{\end{alltt}}

\newlength{\hsbw}

\newenvironment{session}%
  {\begin{flushleft}
   \setlength{\hsbw}{\linewidth}
   \addtolength{\hsbw}{-\arrayrulewidth}
   \addtolength{\hsbw}{-\tabcolsep}
   \begin{tabular}{@{}|c@{}|@{}}\hline 
   \begin{minipage}[b]{\hsbw}
   \begingroup\small\mbox{ }\\[-1.8\baselineskip]\begin{alltt}}
  {\end{alltt}\endgroup\end{minipage}\\ \hline 
   \end{tabular}
   \end{flushleft}}

\newenvironment{smallsession}%
  {\begin{flushleft}
   \setlength{\hsbw}{\linewidth}
   \addtolength{\hsbw}{-\arrayrulewidth}
   \addtolength{\hsbw}{-\tabcolsep}
   \begin{tabular}{@{}|c@{}|@{}}\hline 
   \begin{minipage}[b]{\hsbw}
   \begingroup\footnotesize\mbox{ }\\[-1.8\baselineskip]\begin{alltt}}%
  {\end{alltt}\endgroup\end{minipage}\\ \hline 
   \end{tabular}
   \end{flushleft}}

\newenvironment{spec}%
  {\begin{flushleft}
   \setlength{\hsbw}{\textwidth}
   \addtolength{\hsbw}{-\arrayrulewidth}
   \addtolength{\hsbw}{-\tabcolsep}
   \begin{tabular}{@{}|c@{}|@{}}\hline 
   \begin{minipage}[b]{\hsbw}
   \begingroup\small\mbox{ }\\[-0.2\baselineskip]}%
  {\endgroup\end{minipage}\\ \hline 
   \end{tabular}
   \end{flushleft}}

\newcommand{\memo}[1]%
  {\mbox{}\par\vspace{0.25in}%
   \setlength{\hsbw}{\linewidth}\addtolength{\hsbw}{-1.5ex}%
   \noindent\fbox{\parbox{\hsbw}{{\bf Memo: }#1}}\vspace{0.25in}}

\newcommand{\nb}[1]%
  {\mbox{}\par\vspace{0.25in}%
   \setlength{\hsbw}{\linewidth}\addtolength{\hsbw}{-1.5ex}%
   \noindent\fbox{\parbox{\hsbw}{{\bf Note: }#1}}\vspace{0.25in}}

\newcommand{\comment}[1]{}
\newcommand{\exfootnote}[1]{}
%\newcommand{\ifelse}[2]{#1}
\sloppy
\clubpenalty=100000
\widowpenalty=100000
%\displaywidowpenalty=100000
\setcounter{secnumdepth}{3} 
\setcounter{tocdepth}{3}
\setcounter{topnumber}{9}
\setcounter{bottomnumber}{9}
\setcounter{totalnumber}{9}
\renewcommand{\topfraction}{.99}
\renewcommand{\bottomfraction}{.99}
\renewcommand{\floatpagefraction}{.01}
\renewcommand{\textfraction}{.2}
\font\largett=cmtt10 scaled\magstep1
\font\Largett=cmtt10 scaled\magstep2
\font\hugett=cmtt10 scaled\magstep3

%% Derived from John Rushby's prelude.tex, modified for NFSS2
%
% define variants of the \LaTeX macro that avoid using \sc
% for use in headings
%

% Define fonts that work in math or text mode
\def\dwimrm#1{\ifmmode\mathrm{#1}\else\textrm{#1}\fi}
\def\dwimsf#1{\ifmmode\mathsf{#1}\else\textsf{#1}\fi}
\def\dwimtt#1{\ifmmode\mathtt{#1}\else\texttt{#1}\fi}
\def\dwimbf#1{\ifmmode\mathbf{#1}\else\textbf{#1}\fi}
\def\dwimit#1{\ifmmode\mathit{#1}\else\textit{#1}\fi}
\def\dwimnormal#1{\ifmmode\mathnormal{#1}\else\textnormal{#1}\fi}

\def\BigLaTeX{{\rm L\kern-.36em\raise.3ex\hbox{\small\small A}\kern-.15em
    T\kern-.1667em\lower.7ex\hbox{E}\kern-.125emX}}
\def\BoldLaTeX{{\bf L\kern-.36em\raise.3ex\hbox{\small\small\bf A}\kern-.15em
    T\kern-.1667em\lower.7ex\hbox{E}\kern-.125emX}}
%\def\labelitemi{$\bullet$}
\def\labelitemii{$\circ$}
\def\labelitemiii{$\star$}
\def\labelitemiv{$\diamond$}
\newcommand{\tcc}{{\small\small TCC}}
\newcommand{\tccs}{\tcc s}
\newcommand{\emacs}{{Emacs}}
\newcommand{\Emacs}{\emacs}
\newcommand{\ehdm}{{E{\small\small HDM}}}
\newcommand{\Ehdm}{\ehdm}
\newcommand{\tm}{$^{\mbox{\tiny TM}}$}
\newcommand{\hozline}{{\noindent\rule{\textwidth}{0.4mm}}}

\newcommand{\allclear}%
  {\mbox{\boldmath$\stackrel{\raisebox{-.2ex}[0pt][0pt]%
              {$\textstyle\oslash$}}{\displaystyle\bot}$}}

\newenvironment{private}{}{}

\newenvironment{smalltt}{\begin{alltt}\small}{\end{alltt}}

\newlength{\hsbw}

\newenvironment{session}%
  {\begin{flushleft}
   \setlength{\hsbw}{\linewidth}
   \addtolength{\hsbw}{-\arrayrulewidth}
   \addtolength{\hsbw}{-\tabcolsep}
   \begin{tabular}{@{}|c@{}|@{}}\hline 
   \begin{minipage}[b]{\hsbw}
   \begingroup\small\mbox{ }\\[-1.8\baselineskip]\begin{alltt}}
  {\end{alltt}\endgroup\end{minipage}\\ \hline 
   \end{tabular}
   \end{flushleft}}

\newenvironment{smallsession}%
  {\begin{flushleft}
   \setlength{\hsbw}{\linewidth}
   \addtolength{\hsbw}{-\arrayrulewidth}
   \addtolength{\hsbw}{-\tabcolsep}
   \begin{tabular}{@{}|c@{}|@{}}\hline 
   \begin{minipage}[b]{\hsbw}
   \begingroup\footnotesize\mbox{ }\\[-1.8\baselineskip]\begin{alltt}}%
  {\end{alltt}\endgroup\end{minipage}\\ \hline 
   \end{tabular}
   \end{flushleft}}

\newenvironment{spec}%
  {\begin{flushleft}
   \setlength{\hsbw}{\textwidth}
   \addtolength{\hsbw}{-\arrayrulewidth}
   \addtolength{\hsbw}{-\tabcolsep}
   \begin{tabular}{@{}|c@{}|@{}}\hline 
   \begin{minipage}[b]{\hsbw}
   \begingroup\small\mbox{ }\\[-0.2\baselineskip]}%
  {\endgroup\end{minipage}\\ \hline 
   \end{tabular}
   \end{flushleft}}

\newcommand{\memo}[1]%
  {\mbox{}\par\vspace{0.25in}%
   \setlength{\hsbw}{\linewidth}\addtolength{\hsbw}{-1.5ex}%
   \noindent\fbox{\parbox{\hsbw}{{\bf Memo: }#1}}\vspace{0.25in}}

\newcommand{\nb}[1]%
  {\mbox{}\par\vspace{0.25in}%
   \setlength{\hsbw}{\linewidth}\addtolength{\hsbw}{-1.5ex}%
   \noindent\fbox{\parbox{\hsbw}{{\bf Note: }#1}}\vspace{0.25in}}

\newcommand{\comment}[1]{}
\newcommand{\exfootnote}[1]{}
%\newcommand{\ifelse}[2]{#1}
\sloppy
\clubpenalty=100000
\widowpenalty=100000
%\displaywidowpenalty=100000
\setcounter{secnumdepth}{3} 
\setcounter{tocdepth}{3}
\setcounter{topnumber}{9}
\setcounter{bottomnumber}{9}
\setcounter{totalnumber}{9}
\renewcommand{\topfraction}{.99}
\renewcommand{\bottomfraction}{.99}
\renewcommand{\floatpagefraction}{.01}
\renewcommand{\textfraction}{.2}
\font\largett=cmtt10 scaled\magstep1
\font\Largett=cmtt10 scaled\magstep2
\font\hugett=cmtt10 scaled\magstep3

\def\labelitemii{$\circ$}
\def\labelitemiii{$\star$}
\def\labelitemiv{$\diamond$}
\newcommand{\tcc}{{\small\small TCC}}
\newcommand{\tccs}{\tcc s}

%\renewcommand{\memo}[1]{\mbox{}\par\vspace{0.25in}\noindent\fbox{\parbox{.95\linewidth}{{\bf Memo: }#1}}\vspace{0.25in}}

\newcommand{\eg}{{\em e.g.\/},}
\newcommand{\ie}{{\em i.e.\/},}

\newcommand{\pvs}{PVS}

\newcommand{\ch}{\choice}
\newcommand{\rsv}[1]{{\rm\tt #1}}

\newcommand{\lpvstheory}[3]{\figurehead{\hozline\smaller\smaller\begin{alltt}}%
                           \figuretail{\end{alltt}\vspace{-0in}\hozline}%
                           \figurelabel{#3}\figurecap{#2}%
                           \begin{longfigure}\input{#1}\end{longfigure}}

\newcommand{\bpvstheory}[3]
{\begin{figure}[b]\begin{boxedminipage}{\textwidth}%
      {\smaller\smaller\begin{alltt} \input{#1}\end{alltt}}\end{boxedminipage}%
    \caption{#2}\label{#3}\end{figure}}

\newcommand{\spvstheory}[1]
{\vspace{0.1in}\par\noindent\begin{boxedminipage}{\textwidth}%
    {\smaller\smaller\begin{alltt} \input{#1}\end{alltt}}\end{boxedminipage}\vspace{0.1in}%
}

\CustomVerbatimEnvironment{pvsex}{Verbatim}{commandchars=\\\{\},frame=single,fontsize=\relsize{-1}}

\CustomVerbatimCommand{\pvsinput}{VerbatimInput}{commandchars=\\\{\},frame=single,fontsize=\relsize{-1}}

\newcommand{\pvstheory}[3]
  {\begin{figure}[htb]%
      \pvsinput{#1}%
      \caption{#2}\label{#3}%
    \end{figure}}

% \newenvironment{pvsex}%
%   {\setlength{\topsep}{0in}\smaller\begin{alltt}}%
%   {\end{alltt}}

\newcommand{\pvsbnf}[2]
  {\begin{figure}[htb]\begin{boxedminipage}{\textwidth}%
   \input{#1}\end{boxedminipage}\caption{#2}\label{#1}\end{figure}}

\newcommand{\spvsbnf}[1]
  {\begin{boxedminipage}{\textwidth}\input{#1}\end{boxedminipage}}

\newcommand{\pidx}[1]{{\rm #1}} % primary index entry
\newcommand{\sidx}[1]{{\rm #1}} % secondary index entry
\newcommand{\cmdindex}[1]{\index{#1@\cmd{#1}}}
\newcommand{\icmd}[1]{\cmd{#1}\cmdindex{#1}}
\newcommand{\iecmd}[1]{\ecmd{#1}\cmdindex{#1}}
\newcommand{\buf}[1]{\texttt{#1}}
\newcommand{\ibuf}[1]{\buf{#1}\index{#1 buffer@\buf{#1} buffer}\index{buffers!\buf{#1}}}

\newenvironment{pvscmds}%
  {\par\noindent\smaller%
   \begin{tabular*}{\textwidth}{|l@{\extracolsep{\fill}}l@{\extracolsep{\fill}}l|}\hline%
     {\it Command} & {\it Aliases} & {\it Function}\\ \hline}%
  {\hline\end{tabular*}\vspace{0.1in}}

\newenvironment{pvscmdsna}%
  {\par\noindent\smaller%
   \begin{tabular*}{\textwidth}{|l@{\extracolsep{\fill}}l|}\hline%
     {\it Command} & {\it \,\,Function}\\ \hline}%
  {\hline\end{tabular*}\vspace{0.1in}}

\newcommand{\cmd}[1]{\texttt{#1}}
\newcommand{\ecmd}[1]{{\tt M-x #1}}

\newcommand{\latex}{\LaTeX}                  %  LaTeX
\newcommand{\sun}{{S{\smaller\smaller UN}}}                 %  Sun
\newcommand{\sparc}{{S{\smaller\smaller PARC}}}             %  Sparc
\newcommand{\sunos}{{S{\smaller\smaller UN}OS}}             %  SunOS
\newcommand{\solaris}{{\em Solaris\/}}        %  Solaris
\newcommand{\sunview}{{S{\smaller\smaller UN}V{\smaller\smaller IEW}}} %SunView
\newcommand{\unix}{{U{\smaller\smaller NIX}}}               %  Unix
\newcommand{\lisp} {{\sc Lisp}}              %  Lisp
\newcommand{\gnu}{{Gnu Emacs}}           %  Gnu Emacs
\newcommand{\gnuemacs}{{Gnu Emacs}}      %  Gnu Emacs
\newcommand{\emacsl}{{Emacs-Lisp}}       %  Emacs Lisp
\newcommand{\shell}{{\sc Csh}}               %  C-shell

\newcommand{\update}[3]{#1\{#2\leftarrow #3\}}
\newcommand{\interp}[3]{\cal{M}\dlb {\tt #1 : #2 }\drb #3}
\newcommand{\myforall}[2]{(\forall{#1 .}\ #2)}
\newcommand{\myexists}[2]{(\exists{#1 .}\ #2)}
\newcommand{\mth}[1]{$ #1 $}
\newcommand{\labst}[2]{(\lambda{#1}.\ #2)}
\newcommand{\app}[2]{(#1\ #2)}
\newcommand{\problem}[1]{{\bf Exercise: } {\em #1}}
\newcommand{\rectype}[1]{[\# 1 \#]}
\newcommand{\recttype}[1]{{\tt [\# 1 \#]}}
\newcommand{\dlb}{\lbrack\!\lbrack}
\newcommand{\drb}{\rbrack\!\rbrack}
\newcommand{\cross}{\times}
\newcommand{\key}[1]{{\tt #1}}
\newcommand{\keyindex}[1]{\index{#1@\key{#1}}}
\newcommand{\ikey}[1]{\key{#1}\keyindex{#1}}
\newcommand{\keyword}[1]{{\smaller\texttt{#1}}}

\newenvironment{keybindings}%
  {\begin{center}\begin{tabular}{|l|l|}\hline Key & Function\\ \hline}%
  {\hline\end{tabular}\end{center}}
\def\rmif{\mbox{\bf if\ }}
\def\rmiff{\mbox{\bf \ iff \ }}
\def\rmthen{\mbox{\bf \ then }}
\def\rmelse{\mbox{\bf \ else }}
\def\rmend{\mbox{\bf end}}
\def\rmendif{\mbox{\bf \ endif}}
\def\rmotherwise{\mbox{\bf otherwise}}
\def\rmwith{\mbox{\bf \ with\ }}
\def\mapb{\char"7B\char"7B}
\def\mape{\char"7D\char"7D}
\def\setb{\char"7B}
\def\sete{\char"7D}

% ---------------------------------------------------------------------
% Macros for little PVS sessions displayed in boxes.
%
% Usage: (1) \setcounter{sessioncount}{1} resets the session counter
%
%        (2) \begin{session*}\label{thissession}
%             .
%              < lines from PVS session >
%             .
%            \end{session*}
%
%            typesets the session in a numbered box in ALLTT mode.
%
%  session instead of session* produces unnumbered boxes
%
%  Author: John Rushby
% ---------------------------------------------------------------------
\newlength{\hsbw}
\newenvironment{session}{\begin{flushleft}
 \setlength{\hsbw}{\linewidth}
 \addtolength{\hsbw}{-\arrayrulewidth}
 \addtolength{\hsbw}{-\tabcolsep}
 \begin{tabular}{@{}|c@{}|@{}}\hline 
 \begin{minipage}[b]{\hsbw}
% \begingroup\small\mbox{ }\\[-1.8\baselineskip]\begin{alltt}}{\end{alltt}\endgroup\end{minipage}\\ \hline
 \begingroup\sessionsize\vspace*{1.2ex}\begin{alltt}}{\end{alltt}\endgroup\end{minipage}\\ \hline
 \end{tabular}
 \end{flushleft}}
\newcounter{sessioncount}
\setcounter{sessioncount}{0}
\newenvironment{session*}{\begin{flushleft}
 \refstepcounter{sessioncount}
 \setlength{\hsbw}{\linewidth}
 \addtolength{\hsbw}{-\arrayrulewidth}
 \addtolength{\hsbw}{-\tabcolsep}
 \begin{tabular}{@{}|c@{}|@{}}\hline 
 \begin{minipage}[b]{\hsbw}
 \vspace*{-.5pt}
 \begin{flushright}
 \rule{0.01in}{.15in}\rule{0.3in}{0.01in}\hspace{-0.35in}
 \raisebox{0.04in}{\makebox[0.3in][c]{\footnotesize \thesessioncount}}
 \end{flushright}
 \vspace*{-.57in}
 \begingroup\small\vspace*{1.0ex}\begin{alltt}}{\end{alltt}\endgroup\end{minipage}\\ \hline 
 \end{tabular}
 \end{flushleft}}
\def\sessionsize{\footnotesize}
\def\smallsessionsize{\small}
\newenvironment{smallsession}{\begin{flushleft}
 \setlength{\hsbw}{\linewidth}
 \addtolength{\hsbw}{-\arrayrulewidth}
 \addtolength{\hsbw}{-\tabcolsep}
 \begin{tabular}{@{}|c@{}|@{}}\hline 
 \begin{minipage}[b]{\hsbw}
 \begingroup\smallsessionsize\mbox{ }\\[-1.8\baselineskip]\begin{alltt}}{\end{alltt}\endgroup\end{minipage}\\ \hline 
 \end{tabular}
 \end{flushleft}}
\newenvironment{spec}{\begin{flushleft}
 \setlength{\hsbw}{\textwidth}
 \addtolength{\hsbw}{-\arrayrulewidth}
 \addtolength{\hsbw}{-\tabcolsep}
 \begin{tabular}{@{}|c@{}|@{}}\hline 
 \begin{minipage}[b]{\hsbw}
 \begingroup\small\mbox{
}\\[-0.2\baselineskip]}{\endgroup\end{minipage}\\ \hline 
 \end{tabular}
 \end{flushleft}}
\newcommand{\memo}[1]{\mbox{}\par\vspace{0.25in}%
\setlength{\hsbw}{\linewidth}%
\addtolength{\hsbw}{-2\fboxsep}%
\addtolength{\hsbw}{-2\fboxrule}%
\noindent\fbox{\parbox{\hsbw}{{\bf Memo: }#1}}\vspace{0.25in}}
\newcommand{\nb}[1]{\mbox{}\par\vspace{0.25in}\setlength{\hsbw}{\linewidth}\addtolength{\hsbw}{-1.5ex}\noindent\fbox{\parbox{\hsbw}{{\bf Note: }#1}}\vspace{0.25in}}

%%% Local Variables: 
%%% mode: latex
%%% TeX-master: t
%%% End: 

%%
%
%    LaTeX  Macro File  /usr2/jcm/tex/macros.tex
%
%
\tracingonline=0 % shorter error messages (on screen)

%\long\def\comment#1{} % mulit-line comments
%\newcommand{\note}[1]{\fbox{#1}}

% font changes  (as function calls, scribe style)

\renewcommand{\i}[1]{{\it #1\/}}       % italics with space correction
%\newcommand{\emex}[1]{\/{\em #1}}   % emphasis in example, exercise, theorem, etc.
\renewcommand{\c}[1]{{\sc #1}}       % small caps (eliminates \c as cedilla)
%\newcommand{\r}[1]{{\rm #1}}            % for roman font in math mode
%\newcommand{\s}[1]{{\scr #1}}          % script (for use with tatex) 
\newcommand{\s}[1]{{\cal #1}}
\renewcommand{\b}[1]{{\bf #1}}           % bold face
\newcommand{\calg}[1]{{\cal #1}}          % caligraphic 

%\newcommand{\q}[1]{``#1''}    % matching quotes for in-line quotation

% numbered environments

%\newcounter{partcounter}
%\setcounter{partcounter}{0}
%\renewcommand{\part}[1]{\newpage \addtocounter{partcounter}{1}
%\noindent{\Large \bf Part \Roman{partcounter}. #1 } \\[1ex]}

\newtheorem{thm}{Theorem}[section]
\newtheorem{theorem}[thm]{Theorem}
\newtheorem{lemma}[thm]{Lemma}
\newtheorem{cor}[thm]{Corollary}
\newtheorem{corollary}[thm]{Corollary}
\newtheorem{claim}[thm]{Claim}
\newtheorem{prop}[thm]{Proposition}
\newtheorem{conj}[thm]{Conjecture}
\newtheorem{definition}[thm]{Definition}
\newtheorem{exercise}[thm]{Exercise}
\newtheorem{example}[thm]{Example}
\newtheorem{remark}[thm]{Remark}
\newtheorem{open}[thm]{Open Problem}

%\newcommand{\proof}{\\{\bf Proof.}\ }
\newenvironment{proof}{{\bf Proof. }}{\thmbox}

% axiom and inference rule (centered, in math mode,  with name at left)

\newcommand{\axiom}[2]
{\[ \hbox to \columnwidth
    { \rlap{$#2$} \hfil {$ #1 $} \hfil }
\]}

\newcommand{\infrule}[4]
{\[ \hbox to \columnwidth %\textwidth
    { \rlap{$#4$} \hfil $
      \frac {\strut\displaystyle #1 } {\strut\displaystyle #2 } \; \rlap{$#3$} \hfil $
      \hfil }
\]}
\newcommand{\infruletw}[4]
{\[ \hbox to \textwidth
    { \rlap{$#4$} \hfil $
      \frac {\strut\displaystyle #1 } {\strut\displaystyle #2 } \; \rlap{$#3$} \hfil $
      \hfil }
\]}

%\newcommand{\infrule}[4]
%{\[ \hbox to \columnwidth
%    { \rlap{$#4$} \hfil $
%      {{\displaystyle\strut #1}\over{\displaystyle\strut #2}}\quad\makebox[0pt][l]{\it #3} $
%      \hfil }
%\]}

% axiom and inference rule macros for use in tables, etc.
% presumed in math mode #1=axiom, #2=side condition
\newcommand{\Axiom}[2]{
{\displaystyle\strut #1}\qquad\makebox[0pt][l]{\it #2}
}
% presumed in math mode #1=top, #2=bottom, #3=side condition
\newcommand{\Infrule}[3]{
{{\displaystyle\strut #1}\over{\displaystyle\strut #2}}\;\mbox{\scriptsize$\bf #3$}
}

% sequence of #1's, numbered up to #2  (e.g., seq{x}{n} for x1, ..., xn   )

\newcommand{\seq}[2]{#1_{1} \ldots #1_{#2}}  
%\newcommand{\ie}{{\it i.e.}}
%\newcommand{\eg}{{\it e.g.}}
%\newcommand{\cf}{{\it c.f.\,}}

% common symbols

\newcommand{\union}{\cup}
\newcommand{\intersect}{\cap}
\newcommand{\subs}{\subseteq}
\newcommand{\el}{\in}
\newcommand{\nel}{\not\in}
\newcommand{\ns}{\emptyset}
\newcommand{\compose}{\circ}
\newcommand{\set}[2]{ \{\, #1 \,\mid\, #2 \,\}  } % set macro
\newcommand{\infinity}{\infty}
\newcommand{\pair}[1]{\langle #1 \rangle}
\newcommand{\tuple}[1]{\langle #1 \rangle}


\newcommand{\fa}{\forall}
\newcommand{\te}{\exists}
\newcommand{\imp}{\supset}
\renewcommand{\implies}{\supset}
\newcommand{\ts}{\vdash}
\newcommand{\dts}{\models}

\newcommand{\aro}{\mathord\rightarrow} % see pages 154-155 of TeX manual
\newcommand{\paro}{\rightharpoonup} 
\newcommand{\karo}{\mathop\Rightarrow} % see pages 154-155 of TeX manual
%\newcommand{\cross}{\times}
%\newcommand{\dlb}{\lbrack\!\lbrack}
%\newcommand{\drb}{\rbrack\!\rbrack}
\newcommand{\mean}[1]{\lbrack\!\lbrack #1 \rbrack\!\rbrack}
\newcommand{\lam}{\lambda}
\newcommand{\subst}[2]{{}[#1/#2]}
\renewcommand{\dot}{\mathrel{\bullet}}
\newcommand{\Dinf}{D_{\infty}}
\newcommand{\bottom}{\perp}

\mathcode`:="603A  % treat : as punctuation instead of relation in math mode
\mathchardef\colon="303A	% relation colon

\newcommand{\eqdef}{\mathrel{:=}}
\newcommand{\Dom}{\mathop{\rm dom}}
\newcommand{\Pow}{\mathop{\rm Pow}}

\newcommand{\aequiv}{\equiv_\alpha}
\newcommand{\baro}{\buildrel \beta \over \rightarrow}
\newcommand{\earo}{\buildrel \eta \over \rightarrow}
\newcommand{\red}{\rightarrow\!\!\!\!\rightarrow}
\newcommand{\backred}{\leftarrow\!\!\!\!\leftarrow}
\newcommand{\bred}{\buildrel \beta \over \red}
\newcommand{\ered}{\buildrel \eta \over \red}
\newcommand{\conv}{\leftrightarrow}
\newcommand{\beconv}{\buildrel {\beta, \eta} \over\leftrightarrow}

% lambda calculus abbreviations
\newcommand{\letdec}[3]{\b{let\ } #1 = #2 \b{\ in\ } #3}
\newcommand{\letrec}[3]{\b{letrec\ } #1 = #2 \b{\ in\ } #3}

\newcommand{\thmbox}
   {{\ \hfill\hbox{%
      \vrule width1.0ex height1.0ex
   }\parfillskip 0pt }}
\newcommand{\qed}{\thmbox}


% make single spacing

\newcommand{\singlespace}{\renewcommand{\baselinestretch}{1}\@normalsize}
\newcommand{\etal}{{\em et al.}}

% bycase command for definition by cases (pg 49 of LaTeX)

\newcommand{\bycase}[1]
	{\left\{ \begin{array}{ll}  #1  \end{array} \right. }


% make \cite put blanks after the comma (use in alpha style)

%\def\@citex[#1]#2{\if@filesw\immediate\write\@auxout{\string\citation{#2}}\fi
%  \def\@citea{}\@cite{\@for\@citeb:=#2\do
%    {\@citea\def\@citea{,\penalty100\hskip2.5pt plus1.5pt minus.8pt}%
%       \@ifundefined{b@\@citeb}{{\bf ?}\@warning
%       {Citation `\@citeb' on page \thepage \space undefined}}
%\hbox{\csname b@\@citeb\endcsname}}}{#1}}

\newcommand{\aro}{\mathord\rightarrow} % see pages 154-155 of TeX manual
\newcommand{\pvsref}[1]{\fbox{\footnotesize\ref{#1}}} \def\id#1{\hbox{\textt{#1}}} %changing ids from roman to tt.
\begin{document}

\begin{titlepage}
\renewcommand{\thepage}{title}
\vspace*{1in}
\noindent
\rule[1pt]{\textwidth}{2pt}
\begin{center}
\newfont{\pvstitle}{cmss17 scaled \magstep4}
\textbf{\pvstitle Abstract Datatypes in PVS}
\end{center}
\begin{flushright}
{\Large Technical Report CSL-93-9R {\smaller$\bullet$} December 1993, Substantially Revised June 1997}
\end{flushright}
\rule[1in]{\textwidth}{2pt}
\vspace*{2in}
\begin{flushleft}
S.~Owre\\
N.~Shankar\\
{\smaller\url{{Owre,Shankar}@csl.sri.com}}\\
{\smaller\url{http://pvs.csl.sri.com/}}
\end{flushleft}
\vspace*{1in}
\vbox{\hbox to \textwidth{{\Large SRI International\hfill}}%
\hbox to \textwidth{{\small\sf%
Computer Science Laboratory $\bullet$ 333 Ravenswood Avenue $\bullet$ Menlo Park CA 94025\hfil}}}
\end{titlepage}

\pagestyle{fancy}
\renewcommand{\chaptermark}[1]{\markboth{{\em #1}}{}\markright{{\em #1}}}
\renewcommand{\sectionmark}[1]{\markright{\thesection \em \ #1}}
\lhead[\thepage]{\rightmark}
%\cfoot{\protect\small\bf \fbox{PVS 2.3 DRAFT}}
\cfoot{}
\rhead[\leftmark]{\thepage}
\thispagestyle{empty}

\newpage
\renewcommand{\thepage}{ack}
\vspace*{6in}\noindent
The development of the
initial version of PVS was funded by internal research funding from SRI
International.  More recent versions of PVS have been developed with
funding from ARPA, AFOSR, NASA, NSF, and NRL.  Support for the preparation
of this document came from the National Aeronautics and Space
Administration Langley Research Center under Contract NAS1-18969.  This
report is a revised and updated version of Technical Report
SRI-CSL-93-09.
\newpage
\pagenumbering{roman}
\setcounter{page}{1}

\raggedbottom
%\input{title}
\vspace{4in}
\pagenumbering{roman}
%\maketitle
%% \begin{abstract}
%% PVS (Prototype Verification System) is a general-purpose environment for
%% developing specifications and proofs.  This document deals primarily with
%% the abstract datatype mechanism in PVS which generates theories containing
%% axioms and definitions for a class of recursive datatypes.  The concepts
%% underlying the abstract datatype mechanism are illustrated using ordered
%% binary trees as an example.  Binary trees are described by a PVS abstract
%% datatype that is parametric in its value type.  The type of ordered binary
%% trees is then presented as a subtype of binary trees where the ordering
%% relation is also taken as a parameter.  We define the operations of
%% inserting an element into, and searching for an element in an ordered
%% binary tree; the bulk of the report is devoted to PVS proofs of some
%% useful properties of these operations.  These proofs illustrate various
%% approaches to proving properties of abstract datatype operations.  They
%% also describe the built-in capabilities of the PVS proof checker for
%% simplifying abstract datatype expressions.
%% \end{abstract}
\pagenumbering{roman}
\tableofcontents

\cleardoublepage
\pagenumbering{arabic}
\setcounter{page}{1}

\chapter{Introduction}

PVS is a specification and verification environment developed at SRI
International.\footnote{PVS is freely available and can be obtained via
FTP from \texttt{/pub/pvs/} through the Internet host
\url{ftp.csl.sri.com}.  The URL \texttt{http://www.csl.sri.com/pvs.html}
provides access to PVS-related information and documents.}  Several
documents describe the use of PVS~\cite{PVS:manuals}; this document
explains the PVS mechanisms for defining and using abstract
datatypes.\footnote{The PVS abstract datatype mechanism is still evolving.
Some of the contemplated changes could invalidate parts of the description
in this report.  This report itself updates SRI CSL Technical Report
CSL-93-9 so that it is accurate with respect to the alpha version of
PVS~2.1\@.  Future versions of the report will be similarly revised to
maintain accuracy.}  It describes a PVS specification for the data
structure of ordered binary trees, defines various operations on this
structure, and contains PVS proofs of some useful properties of these
operations.  It also describes various other data structures that can be
captured by the PVS abstract datatype mechanism, and documents the
built-in capabilities of the PVS proof checker for simplifying abstract
datatype expressions.  The exposition does assume some general familiarity
with formal methods but does not require any specific knowledge of PVS\@.



PVS provides a mechanism for defining abstract datatypes of a certain
class.  This class includes all of the ``tree-like'' recursive data
structures that are \emph{freely generated} by a number of
\emph{constructor} operations.\footnote{The abstract datatype mechanism of
PVS is partly inspired by the \emph{shell} principle used in the
Boyer-Moore theorem prover~\cite{Boyer-Moore79}.  Similar mechanisms exist
in a number of other specification and programming languages.}  For
example, the abstract datatype of lists is generated by the constructors
\texttt{null} and \texttt{cons}\@.  The abstract datatype of stacks is generated
by the constructors \texttt{empty} and \texttt{push}\@.  An unordered list or a
\emph{bag} is an example of a data structure that is not freely generated
since two different sequences of insertions of elements into a bag can
yield equivalent bags.  The queue datatype is freely generated but is not
considered recursive in PVS since the accessor \texttt{head} returning the
first element of the queue is not an inverse of the \texttt{enqueue}
constructor\@.  This means that the queue datatype must either be
explicitly axiomatized or implemented using some other datatype such as
the list or stack datatype.


At the semantic level, a recursive datatype introduces a new type
constructor that is a solution to a recursive type equation of the form $T
= \tau[T]$\@.  Typically, the recursive occurrences of the type name $T$
on the right-hand side must occur only \emph{positively} (as defined in
Section~\ref{positive}) in the type expression $\tau[T]$ and the datatype
is the least solution to the recursion equation\@.  For example, the
datatype of lists of element type $A$ is the least solution to the type
equation $T = \{ \texttt{null}\} + A\times T$, where $+$ is the disjoint
union operation and the $\times$ operation returns the Cartesian product.
The minimality of lists datatype yields a structural induction principle
asserting that any list predicate $P$, if $P$ is closed under the list
datatype operations, i.e., where $P( \texttt{null})$ and $\forall x, l: P(l)
\supset P( \texttt{cons}(x, l))$, then $P$ holds of all lists.  The induction
principle also yields a structural recursion theorem asserting that
a function that is defined by induction on the structure is total and
uniquely defined.  By the semantic definition of lists, the equality
relation on the lists datatype is also the least equality where the
constructor \texttt{cons} can be regarded as a congruence.  The minimality of
the equality relation asserts that the constructor \texttt{cons} is an
injective operation from $A \times \texttt{list}$ to \texttt{list}\@.
As a consequence of the minimality of equality on the datatype, one can
define accessor functions such as \texttt{car} and \texttt{cdr} on lists
constructed using \texttt{cons}, derive extensionality principles, and define
functions by case analysis on the constructor.   The PVS datatype
mechanism is used to generate theories introducing the datatype operations
for constructing, recognizing, and accessing datatype expressions,
defining structural recursion schemes over datatype expressions, 
and asserting axioms such as those for extensionality and induction.  

The datatype of lazy lists or streams is also generated by the same
recursion scheme using the constructors \texttt{null} and \texttt{cons} but it
is a co-recursive datatype (or a co-datatype) rather than a recursive
datatype in that it is the greatest solution to the same recursion
equation corresponding to lists.  PVS does not yet have a similar 
mechanism for introducing co-datatypes, and this would be a useful
extension to the language.  Such a theory of sequences has been
formalized in PVS by Hensel and  Jacobs~\cite{Hensel-Jacobs97}
(see also the URL: \url{http://www.cs.kun.nl/~bart/sequences.html})\@.
 
 PVS is a specification language with a set-theoretic semantics.  Types
are therefore interpreted as sets of elements and a function type \texttt{[A
-> B]} is interpreted as the set of all total maps from the set
corresponding to \texttt{A} to that for \texttt{B}.  The use of set-theoretic
semantics leads to some
important constraints on the form of recursive definitions that can be
used in PVS datatype declarations.  

In Section~\ref{lists}, we first present the declaration for the \texttt{list} datatype to convey the syntactic restrictions on such datatype
declarations.   The outcome of such datatype declarations in terms of
generated theories is explained in detail for the datatype of binary trees
in Section~\ref{binary-trees}.  In Section~\ref{ordered-binary-trees},
the binary tree data structure is used to define ordered binary trees.
Section~\ref{enumtypes} shows how enumerated datatypes can be
defined as simple forms of PVS datatypes.  Section~\ref{disjoint-union}
shows the definition for disjoint unions.  Mutually recursive datatypes
are described in Section~\ref{mutual}\@.  Subtyping on recursive
datatypes is described in Section~\ref{lifting}\@.  In
Section~\ref{section:ordinals},  datatypes are
used to construct effective representations for recursive ordinals
which are then used as lexicographic termination measures for
recursive functions.  Section~\ref{OBT-proofs} shows some proofs
about ordered binary trees which use some of the built-in simplifications
shown in~\ref{built-in} along with the proof strategies
described in Section~\ref{proof-strategies}\@.   Some limitations
of the PVS datatype mechanism are described in Section~\ref{limitations},
followed by a discussion of related work in Section~\ref{related}\@.

\chapter{Lists:  A Simple Abstract Datatype}\label{lists}

The PVS prelude contains the following declaration of 
the abstract datatype of lists of a given element type.  
\begin{session*}
list[t:TYPE] : DATATYPE 
 BEGIN
  null: null?
  cons (car: t, cdr :list) :cons?
 END list
\end{session*}
Here \texttt{list} is declared as a type that is parametric in the type
\texttt{t} with two \emph{constructors} \texttt{null} and \texttt{cons}\@.
The constructor \texttt{null} takes no arguments.  The predicate
\emph{recognizer} \texttt{null?}  holds for exactly those elements of the
\texttt{list} datatype that are identical to \texttt{null}.  The
constructor \texttt{cons} takes two arguments where the first is of the
type \texttt{t} and the second is a \texttt{list}.  The recognizer
predicate \texttt{cons?}\ holds for exactly those elements of the
\texttt{list} type that are constructed using \texttt{cons}, namely, those
that are not identical to \texttt{null}\@.  There are two \emph{accessors}
corresponding to the two arguments of \texttt{cons}\@.  The accessors
\texttt{car} and \texttt{cdr} can be applied only to lists satisfying the
\texttt{cons?}\ predicate so that \texttt{car(cons(x, l))} is \texttt{x}
and \texttt{cdr(cons(x, l))} is \texttt{l}\@.  Note that
\texttt{car(null)} is not a well typed expression in that it generates a
invalid proof obligation, a \emph{type correctness condition} (TCC), that
\texttt{cons?(null)} must hold.

The rules on datatype declarations as enforced by the PVS typechecker are:
\begin{enumerate}

\item The constructors must be pairwise distinct, i.e., there should be no
duplication among the constructors.

\item The recognizers must be pairwise distinct, and also distinct
from any of the constructors and the datatype name itself.

%\item The accessors within each constructor must be pairwise
%distinct and distinct from the datatype name itself.  

\item There must be at least one non-recursive constructor, that is, one
that has no recursive occurrences of the datatype in its accessor
types.\footnote{This is a needless restriction which will be removed
in future versions of PVS.  It was intended to ensure
that the recursive datatype had a base object.  However, it turns out
that the restriction does not always guarantee the existence of such a
base object such as when the base constructor has an accessor of an empty
type.   Also  datatypes violating this restriction can be well-formed
such as a datatype \texttt{okay} with one recursive constructor \texttt{mk\_okay}
that has one accessor \texttt{get} of type \texttt{list[okay]}\@.  The base
object in this case is \texttt{mk\_okay(null)}\@.  When there is no
base object, then the datatype is empty.  }

\item The recursive occurrences of the datatype name in its definition
must be positive as described in Section~\ref{positive}.
\end{enumerate}

When the \texttt{list} abstract datatype is typechecked, three theories are
generated in the file \texttt{list\_adt.pvs}\@.    The first theory,
\texttt{list\_adt},  contains
the basic declarations and axioms formalizing the datatype, including an
induction scheme and an extensionality axiom for each constructor.
The second theory, \texttt{list\_adt\_map}, defines a \texttt{map} operation
that lifts a function of
type \texttt{[s -> t]} to a function of type \texttt{[list[s] -> list[t]]}\@.   
The third theory, \texttt{list\_adt\_reduce}, formalizes a general-purpose
recursion operator 
over the abstract datatype.   These theories are  examined in more
detail below for the case of binary trees.  An important point to note
about the generated datatype axioms is that apart from the induction and
extensionality axioms, all the other axioms are automatically applied by
PVS proof commands such as \texttt{assert} and \texttt{beta} so that the
relevant axioms need never be explicitly invoked during a proof.

\section{Positive type occurrence. }\label{positive}
For each recursive datatype defined by means of the PVS \texttt{DATATYPE}
declaration, the typechecker generates theories, definitions, and axioms
similar to those shown above for the case of binary trees.  In general,
such a datatype can take individual and type parameters, and
is specified in terms of the constructors, and the corresponding
recognizers and accessors.  The type of the accessor fields can be
given recursively in terms of the datatype itself as long as this
recursive occurrence of the type is \emph{positive} in a certain
restricted sense\@.  A type occurrence
\texttt{T} is positive in a type expression $\tau$ iff either
\begin{enumerate}
\item $\tau\equiv \texttt{T}$.

\item \texttt{T} occurs positively in a supertype $\tau'$  of $\tau$.

\item $\tau\equiv \mathfuntype{\tau_1}{\tau_2}$ where
$ \texttt{T}$ occurs positively
in $\tau_2$\@.  For example,  $\texttt{T}$ occurs positively in 
\texttt{sequence[T]} where \texttt{sequence[T]} is defined in the
PVS prelude as the function type \texttt{[nat -> T]}\@.

\item $\tau \equiv [\tau_1,\ldots, \tau_n]$ where \texttt{T} occurs
positively in some $\tau_i$.


\item $\tau\equiv [\#\ l_1 : \tau_1, \ldots, l_n : \tau_n\ \#]$ where
\texttt{T} occurs positively in some $\tau_i$\@. 

\item $\tau\equiv \mbox{\emph{datatype}}[\tau_1,\ldots, \tau_n]$,
where \emph{datatype} is a previously defined datatype 
and  \texttt{T} occurs positively in $\tau_i$, where
$\tau_i$ is a \emph{positive parameter} of  \emph{datatype}\@. 
\end{enumerate}


The recursive occurrences of the datatype name in its definition must be
positive so that we can assign a set-theoretic interpretation to all
types.  It is easy to see that violating this condition in the recursion
leads to contradictions.  For example, a datatype \texttt{T} with an
accessor of type \texttt{\tfuntype{T}{bool}} would yield a contradiction
since the cardinality of \texttt{\tfuntype{T}{bool}} is that of the
power-set of \texttt{T} which by Cantor's theorem must be strictly greater
than the cardinality of \texttt{T}\@.  However, we have that distinct
accessor elements lead to distinct datatype elements as well, and hence a
contradiction.  Similarly, an accessor type of
\texttt{\tfuntype{\tfuntype{T}{bool}}{bool}} is also easily ruled out by
cardinality considerations even though the occurrence of \texttt{T} in it
is positive in terms of its polarity.

A \emph{positive type parameter} \texttt{T} in a datatype declaration 
is one that only occurs positively in the type of an accessor.
Positive type parameters in datatypes have a special role.
As an example of a nested recursive datatype with recursion on the positive
parameters,  a \emph{search tree} with leaf nodes bearing 
values of type \texttt{T} can be declared as in~\pvsref{leaftree}\@.
Note that the recursive occurrence of \texttt{leaftree} is as a (positive)
parameter to the \texttt{list} datatype.  
\begin{session*}\label{leaftree}
leaftree[T : TYPE] : DATATYPE
 BEGIN
  leaf(val : T) : leaf?
  node(subs : list[leaftree]): node?
 END leaftree
\end{session*}

 Positive datatype parameters are also used to generate the
combinators \texttt{every}, \texttt{some}, and \texttt{map} which
are described in detail for the datatype of binary trees in
Section~\ref{binary-trees}.  

\chapter{Binary Trees}\label{binary-trees}

A \emph{binary tree} is a recursive data structure that in the base case
is a \emph{leaf} node, and in the recursive case consists of a \emph{value}
component, and \emph{left} and \emph{right} subtrees that are themselves
binary trees.  Binary 
trees can be formalized in  several ways.  In most imperative
programming languages, they are defined as record structures
containing pointers to the subtrees.  They can also be encoded in
terms of more primitive recursive data structures such as the
s-expressions of Lisp.  In a declarative specification language, one
can formalize binary trees by enumerating the relevant axioms.  One
difficulty with this latter approach is the amount of effort involved
in correctly identifying all of the relevant axioms.  Another difficulty
is that it can be tedious to explicitly invoke these axioms during a
proof.  This is the motivation for providing a concise abstract
datatype mechanism in PVS that is integrated with the theorem prover.
With binary trees, the declaration of the 
datatype is similar to that for lists above.
\begin{session*}
binary_tree[T : TYPE] : DATATYPE
BEGIN
  leaf : leaf?
  node(val : T, left : binary_tree, right : binary_tree) : node?
END binary_tree
\end{session*}
The two constructors \texttt{leaf} and \texttt{node} have corresponding
recognizers \texttt{leaf?}\ and \texttt{node?}\@.  The \texttt{leaf}
constructor does not have any accessors.  The \texttt{node} constructor
has three arguments: the value at the node, the left subtree, and the
right subtree.  The accessor functions corresponding to these three
arguments are \texttt{val}, \texttt{left}, and \texttt{right},
respectively.  When the above datatype declaration is typechecked, the
theories \texttt{binary\_tree\_adt}, \texttt{binary\_tree\_adt\_map}, and
\texttt{binary\_tree\_adt\_reduce} are generated.  The first of these has
the form:
\begin{session*}
binary_tree_adt[T: TYPE]: THEORY
  BEGIN
  
  binary_tree: TYPE
  
  leaf?, node?: [binary_tree -> boolean]
  
  leaf: (leaf?)
  
  node: [[T, binary_tree, binary_tree] -> (node?)]
  
  val: [(node?) -> T]
  
  left: [(node?) -> binary_tree]
  
  right: [(node?) -> binary_tree]
  
  \emph{Various axioms and definitions omitted.}

  END binary_tree_adt
\end{session*}
Note that the theory is parametric in the value type \texttt{T}.
%% \comment{A hidden assumption on this parameter  is generated 
%% during typechecking as an \emph{assumption} TCC (type correctness
%% condition).\footnote{The PVS Emacs command \texttt{M-x ppe} displays the
%% entire theory, including the assumptions and type correctness
%% conditions.}  
%% It asserts that the type \texttt{T} must be nonempty.}

The first declaration above declares  \texttt{binary\_tree} as a type.
The two recognizer predicates on binary trees \texttt{leaf?}\ and \texttt{node?}\  are then declared.  The constructor \texttt{leaf} is declared
to have type \texttt{(leaf?)} which is the subtype of \texttt{binary\_tree}
constrained by the \texttt{leaf?} predicate.  The \texttt{node} constructor
is declared as a function with domain type \texttt{[T, binary\_tree,
binary\_tree]} and range type \texttt{(node?)} which is again the subtype of
\texttt{binary\_tree} constrained by the \texttt{node?} predicate.  
The three accessors on value (nonleaf)
nodes are then declared.  Each of these accessors takes as its domain
the subset of binary trees that are constructed by means of the \texttt{node} constructor.  Note that when \texttt{binary\_tree\_adt} is instantiated
with an empty actual parameter type, the subtype \texttt{(node?)}\ must be
empty since there is no value component corresponding to an element
of \texttt{(node?)}\@.


The remainder of this section presents the axioms and definitions that are
generated from the datatype declaration for binary trees.  These axioms
and definitions are not meant to be minimal and some of them are
in fact redundant. 

\subsubsection{Definition by cases. }  The primitive \texttt{CASES} construct is
used for definition by cases on the outermost constructor of a
a PVS datatype expression.  The syntax of the \texttt{CASES} construct
is
\begin{center}
\texttt{CASES~\textit{expression}~OF~\textit{selections}~ENDCASES}
\end{center}
where each \textit{selection} (typically one selection per constructor) is
of the form \textit{pattern : expression} 
and a \emph{pattern} is a constructor of arity $n$ applied to
$n$ distinct variables.  There
are no explicit axioms characterizing the behavior of \texttt{CASES}\@.  
In the case of
the binary tree datatype, when \texttt{w}, \texttt{x}, \texttt{y}, and \texttt{z}
range over binary trees, 
\texttt{a} and \texttt{b} range over the parameter type \texttt{T}, \texttt{u}  ranges over
the range type \texttt{range}, and \texttt{v} ranges over the type
\texttt{[T, binary\_tree, binary\_tree -> range]},  
we implicitly assume the two axioms:
\begin{eqnarray*}
\texttt{{CASES~leaf~OF~leaf~:~u,~node(a,~y,~z)~:~v(a,~y,~z)}} & = & \texttt{u}\\
\texttt{{CASES~node(b,~w,~x)~OF~leaf~:~u,~node(a,~y,~z)~:~v(a,~y,~z)}} & = &
\texttt{v(b,~w,~x)}
\end{eqnarray*}
Note that in the above axioms, the left-hand side occurrences of \texttt{a},
\texttt{y}, and \texttt{z} in \texttt{v(a,~y,~z)} are bound.

\subsubsection{The \texttt{ord} function. }
The function \texttt{ord} assigns a number to a
datatype construction, i.e., a datatype term given solely in terms of
the constructors,  according to its outermost constructor.
The \texttt{ord} function is mainly used to enumerate the
elements of an \emph{enumerated type} (see Section~\ref{enumtypes})\@.
 The \texttt{ord} function is defined using \texttt{CASES} in~\pvsref{ord}\@.  
\begin{session*}\label{ord}
  ord(x: binary_tree): upto(1) =
    CASES x OF leaf: 0, node(node1_var, node2_var, node3_var): 1 ENDCASES
\end{session*}
  Thus
\texttt{ord(leaf)} is \texttt{0}, whereas \texttt{ord(node(x, A, B))} is \texttt{1}\@.

\subsubsection{Extensionality axioms. }
An extensionality axiom is generated corresponding to each constructor.
The one for the \texttt{leaf} terms essentially asserts that \texttt{leaf} is
the unique term of type \texttt{(leaf?)}\@.
\begin{session*}
  binary_tree_leaf_extensionality: AXIOM
        (FORALL (leaf?_var: (leaf?), leaf?_var2: (leaf?)):
           leaf?_var = leaf?_var2);
\end{session*}
For the \texttt{node} constructor, the extensionality axiom is:
\begin{session*}
  binary_tree_node_extensionality: AXIOM
    (FORALL (node?_var: (node?)),
            (node?_var2: (node?)):
       val(node?_var) = val(node?_var2)
         AND left(node?_var) = left(node?_var2)
           AND right(node?_var) = right(node?_var2)
         IMPLIES node?_var = node?_var2)
\end{session*}
In other words, any two value nodes that agree on all the accessors
are equal.

\subsubsection{Accessor--constructor axioms. }  Each accessor--constructor pair
generates an axiom indicating the 
effect of applying the accessor to an expression constructed using
the constructor.  For example, the axiom corresponding to \texttt{val}
and \texttt{node} has the form:
\begin{session*}
  binary_tree_val_node: AXIOM
    (FORALL (node1_var: T), (node2_var: binary_tree),
            (node3_var: binary_tree):
       val(node(node1_var, node2_var, node3_var)) = node1_var)
\end{session*}
We do not need an explicit axiom asserting that the recognizers \texttt{leaf?}\ and \texttt{node?}\ hold of disjoint subsets of the type of binary trees.  This
property can be derived from the \texttt{ord} function and the semantics
of the \texttt{CASES} construct described above.  
%% \comment{
%% This axiom can be expensive to generate\footnote{If there are $n$
%% constructors, there will be $\left(\stackrel{\textstyle n}{2}\right)$ clauses to the axiom.}
%% so it is not explicitly stated in \texttt{binary\_tree\_adt} theory.  
%% }

\subsubsection{Eta axiom. } The \texttt{eta} rule is a useful corollary to the
extensionality axiom above and the accessor--constructor axioms shown
above.  It is introduced as an axiom in the \texttt{binary\_tree\_adt}
theory as shown below though it does follow as a lemma from
extensionality.\footnote{In future versions of PVS,  
it is intended that these will become lemmas with automatically generated
proofs.}     
\begin{session*}
  binary_tree_node_eta: AXIOM
    (FORALL (node?_var: (node?)):
       node(val(node?_var), left(node?_var), right(node?_var)) = node?_var)
\end{session*}



\subsubsection{Structural induction. } The theory \texttt{binary\_tree\_adt} also
contains a structural induction scheme 
and a few recursion schemes.  The induction scheme for binary trees is
stated as:
\begin{session*}
  binary_tree_induction: AXIOM
    (FORALL (p: [binary_tree -> boolean]):
       p(leaf)
         AND
         (FORALL (node1_var: T), (node2_var: binary_tree),
                 (node3_var: binary_tree): p(node2_var) AND p(node3_var)
              IMPLIES p(node(node1_var, node2_var, node3_var)))
         IMPLIES (FORALL (binary_tree_var: binary_tree): p(binary_tree_var)))
\end{session*}
In other words, to prove a property of all binary trees, it is
sufficient to prove in the base case that the property holds of the
binary tree \texttt{leaf}, and that in the induction case, the property holds
of a binary tree \texttt{node(v, A, B)} assuming (the induction
hypothesis) that it holds of the subtrees \texttt{A} and \texttt{B}.
One simple consequence of the induction axiom is the property that
all binary trees are either leaf nodes or value nodes.  This is
also introduced as an axiom in the theory \texttt{binary\_tree\_adt}\@.
\begin{session*}
  binary_tree_inclusive: AXIOM
    (FORALL (binary_tree_var: binary_tree):
       leaf?(binary_tree_var) OR node?(binary_tree_var))
\end{session*}

\subsubsection{Definition by recursion. } As another consequence of induction,
we can demonstrate the existence 
and uniqueness of functions defined by structural recursion over binary
trees.  It is, however, convenient to have a more direct means   
for  defining such recursive functions.  
PVS therefore provides various \emph{recursion combinators}\footnote{A combinator is a
lambda expression without any free variables, but the term can also be
applied to an operation that can be used as a building block for other
operations.} which can be
used to define recursive functions over datatype elements.  
One difficulty with
defining  a fully general recursion combinator is that it has to be
parametric in the 
range type of the function being defined.  Since PVS only provides such
type parametricity at the level of theories, 
the generic recursion combinators are defined in
a separate theory \texttt{binary\_tree\_adt\_reduce} which provides the
additional type parameter.  The recursion combinators for
the common cases of functions returning natural numbers and
sub-$\epsilon_0$ ordinals (see 
Section~\ref{section:ordinals}) are
defined in the theory \texttt{binary\_tree\_adt}
itself.
%% \comment{\footnote{$\epsilon_0$ is 
%% the smallest ordinal such that $\epsilon_0 = \omega^{\epsilon_0}$.
%% The sub-$\epsilon_0$ ordinals include all the usual lexicographic
%% orderings needed to establish the termination of recursive functions.
%% %
%% To admit arbitrary recursive definitions involving datatypes, we need to
%% establish the admissibility 
%% of such definitions by showing that there is a termination measure
%% on the arguments that decreases in a well-founded manner with each
%% recursive call.  In the 
%% case of abstract datatype arguments, such measures typically involve the
%% ``size'' of the datatype.  Such size functions are themselves recursive.
%% Indeed these recursion combinators on natural numbers and ordinals 
%% provide a way of generating  size functions on these data
%% structures that can be used as termination measures.}}

The recursion combinator used for defining
recursive functions over binary trees that return natural number
values, is shown below.  The idea is that we want to define a function
$f$ by the following recursion over binary trees:
\begin{eqnarray*}
  f(\texttt{leaf}) & = & a \\
  f(\texttt{node(v, A, B)}) & = & g(\texttt{v}, f(\texttt{A}), f(\texttt{B}))
\end{eqnarray*}
We define such an $f$ by taking $a$ and $g$ as arguments to
the function \texttt{reduce\_nat}\@.  Note the use of the \texttt{CASES} construct to define a pattern-matching case split over a
datatype value that in this case is a binary tree.  
\begin{session*}
  reduce_nat(leaf?_fun: nat, node?_fun: [[T, nat, nat] -> nat]):
      [binary_tree -> nat] =
    LAMBDA (binary_tree_adtvar: binary_tree):
      CASES binary_tree_adtvar OF
        leaf: leaf?_fun,
        node(node1_var, node2_var, node3_var):
            node?_fun(node1_var,
                      reduce_nat(leaf?_fun,
                                 node?_fun)
                        (node2_var),
                      reduce_nat(leaf?_fun,
                                 node?_fun)
                        (node3_var))
        ENDCASES;
\end{session*}

The \texttt{reduce\_nat} recursion combinator is useful for defining
a ``size'' function as shown in~\pvsref{size} but has the weakness that
\texttt{node?\_fun} 
only has access to the \texttt{val} field of the node.  The theory
\texttt{binary\_tree\_adt} also contains a variant \texttt{REDUCE\_nat}
where the \texttt{leaf?\_fun} is a function and the \texttt{node?\_fun} function
takes an additional argument.  The definition is omitted here since a more
generic version of this recursion combinator is described below.


A generic version of the structural recursion combinator on binary trees
is defined  in \texttt{binary\_tree\_adt\_reduce} where the type \texttt{nat}
in the definition of \texttt{reduce\_nat} has been generalized to an
arbitrary parameter type \texttt{range}\@.
\begin{session*}
binary_tree_adt_reduce[T: TYPE, range: TYPE]: THEORY
  BEGIN
  
  IMPORTING binary_tree_adt[T]
  
  reduce(leaf?_fun: range, node?_fun: [[T, range, range] -> range]):
      [binary_tree[T] -> range] =
    LAMBDA (binary_tree_var: binary_tree[T]):
      CASES binary_tree_var OF
        leaf: leaf?_fun,
        node(node1_var, node2_var, node3_var):
            node?_fun(node1_var,
                      reduce(leaf?_fun,
                             node?_fun)(node2_var),
                      reduce(leaf?_fun,
                             node?_fun)(node3_var))
        ENDCASES

   \pvsref{bigreduce}
  
  END binary_tree_adt_reduce
\end{session*}

The theory \texttt{binary\_tree\_adt\_reduce} also contains
the more flexible recursion combinator \texttt{REDUCE} where the \texttt{leaf?\_fun} and \texttt{node?\_fun} functions take \texttt{binary\_tree\_var} as
an argument. 
\begin{session*}\label{bigreduce}
  REDUCE(leaf?_fun: [binary_tree[T] -> range], node?_fun:
         [[T, range, range, binary_tree[T]] -> range]):
      [binary_tree[T] -> range] =
    LAMBDA (binary_tree_var: binary_tree[T]):
      CASES binary_tree_var OF
        leaf: leaf?_fun(binary_tree_var),
        node(node1_var, node2_var, node3_var):
            node?_fun(node1_var,
                      REDUCE(leaf?_fun,
                             node?_fun)(node2_var),
                      REDUCE(leaf?_fun,
                             node?_fun)(node3_var),
                      binary_tree_var)
        ENDCASES
\end{session*}

PVS~2 introduced certain extensions to the datatype mechanism
that were absent in PVS~1\@.  These include a
primitive subterm relation, the \texttt{some}, \texttt{every}, and
\texttt{map} combinators, and recursion through parameters of
previously defined datatypes.  

\subsubsection{Subterm relation. } The primitive subterm relation  is defined
on datatype objects and checks whether one object occurs as
a (not necessarily proper) subterm of another object.  This relation
is defined as \texttt{subterm}.
\begin{session*}
  subterm(x: binary_tree, y: binary_tree): boolean =
    x = y
      OR CASES y OF
           leaf: FALSE,
           node(node1_var, node2_var, node3_var):
               subterm(x, node2_var) OR subterm(x, node3_var)
           ENDCASES
 \end{session*}
The proper subterm relation is defined by \texttt{<<}.  The proper
subterm relation is useful as a well-founded termination relation
that can be given along with the measure for a recursively defined
function.  
\begin{session*}\label{propersubterm}
  <<(x: binary_tree, y: binary_tree): boolean =
    CASES y OF
      leaf: FALSE,
      node(node1_var, node2_var, node3_var):
          (x = node2_var OR x << node2_var)
            OR x = node3_var OR x << node3_var
      ENDCASES
\end{session*}

\subsubsection{Well-foundedness. } The next axiom asserts that
datatype objects are well-founded with respect to the proper subterm
relation.  The induction axiom \texttt{binary\_tree\_induction} can be
derived as a consequence of the axiom \texttt{binary\_tree\_well\_founded}
and the well-founded induction lemma \texttt{wf\_induction} in the prelude.
\begin{session*}
  binary_tree_well_founded: AXIOM well_founded?[binary_tree](<<);
\end{session*}

\subsubsection{The \texttt{every} combinator. }
The PVS typechecker generates the combinators \texttt{every} and \texttt{some}
corresponding to the positive parameters of a datatype.
For example, \texttt{every}  checks if all values of this parameter type in
an instance of the datatype satisfy 
a given predicate on the parameter type.  Furthermore, if all the type
parameters of a datatype are positive, then a \texttt{map} combinator is also
generated.  

The \texttt{every} combinator in the
theory  
\texttt{binary\_tree\_adt} takes a predicate  \texttt{p} on
the positive type parameter \texttt{T}, and checks that every occurrence of an
object of the type parameter in a binary tree satisfies the predicate.
The \texttt{binary\_tree\_adt} theory also contains a non-curried variant of
\texttt{every} that is 
written as \texttt{every(p, a)} instead of \texttt{every(p)(a)}\@.  
\begin{session*}
  every(p: PRED[T])(a: binary_tree): boolean =
    CASES a OF
      leaf: TRUE,
      node(node1_var, node2_var, node3_var):
          p(node1_var)
            AND every(p)(node2_var) AND every(p)(node3_var)
      ENDCASES
 \end{session*}

\subsubsection{The \texttt{some} combinator. } The \texttt{some} combinator is the
dual to \texttt{every} and checks that 
some occurrence of a value of type \texttt{T} in the binary tree
satisfies the given predicate.\footnote{For operations like \texttt{some} and
\texttt{every}, PVS allows a notational convenience where
\texttt{(some!\ x:\ p(x))} is shorthand for \texttt{some(lambda\ x:\ p(x))}\@. }
\begin{session*}
  some(p: PRED[T])(a: binary_tree): boolean =
    CASES a OF
      leaf: FALSE,
      node(node1_var, node2_var, node3_var):
          p(node1_var) OR some(p)(node2_var) OR some(p)(node3_var)
      ENDCASES
\end{session*}

\subsubsection{The \texttt{map} combinator. } Finally, when all the type
parameters of a datatype definition 
occur positively in the definition, as is the case with \texttt{binary\_tree}, a theory \texttt{binary\_tree\_adt\_map} is generated
that defines 
the curried and non-curried versions of the \texttt{map}
combinator.  In addition to the parameter \texttt{T},
\texttt{binary\_tree\_adt\_map} takes a range type parameter \texttt{T1}.
The \texttt{map} combinator applies a function \texttt{f} from \texttt{T} to \texttt{T1} to every value of type \texttt{T} in a given \texttt{binary\_tree[T]} to
return a result of type \texttt{binary\_tree[T1]}.  
We omit the definition of the non-curried variant of \texttt{map}\@. 
\begin{session*}
binary_tree_adt_map[T: TYPE, T1: TYPE]: THEORY
  BEGIN
  
  IMPORTING binary_tree_adt
  
  map(f: [T -> T1])(a: binary_tree[T]): binary_tree[T1] =
    CASES a OF
      leaf: leaf[T1],
      node(node1_var, node2_var, node3_var):
          node[T1](f(node1_var),
                   map(f)(node2_var), map(f)(node3_var))
      ENDCASES
  
  END binary_tree_adt_map
\end{session*}

In summary, the datatype mechanism accepts parametric recursive type
definitions in terms of constructors, accessors, and recognizers.  The
recursive occurrences of the datatype must be positive.  The typechecker
generates recognizer subtypes, accessor-constructor axioms, extensionality
axioms, a structural induction scheme, a subterm ordering relation, and
various recursion combinators.  With respect to positively occurring type
parameters, the typechecker generates the \texttt{some} and \texttt{every}
combinators.  When all type parameters are positive, the typechecker also
generates a \texttt{map} combinator.  We next examine the use of the
above theories formalizing binary trees 
in the definition of \emph{ordered} binary trees.

\chapter{Ordered Binary Trees}\label{ordered-binary-trees}

In ordered binary trees, the values in the nodes are ordered relative
to each other: the value at a node is no less than any of the values
in the left subtree, and no greater than any of the values in the
right subtree.  Such a data structure has many obvious uses since the
values are maintained in sorted form and the average time for looking
up a value or inserting a new value is logarithmic in the number of
nodes.

The  PVS specification of ordered binary trees is given in the theory
\texttt{obt} below.  It is worth noting the use of theory parameters in
this specification.   The body of the theory \texttt{obt} has been elided
from the specification displayed below.
\begin{session*}
obt [T : TYPE,  <= : (total_order?[T])] : THEORY
BEGIN
IMPORTING binary_tree[T]

A, B, C: VAR binary_tree
x, y, z: VAR T
pp: VAR pred[T]
i, j, k: VAR nat

\emph{\{definitions and lemmas shown below in~\pvsref{size} to~\pvsref{ordered?_insert}\}}

END obt
\end{session*}
The theory \texttt{obt} takes the type \texttt{T} of the values kept in the
binary tree as its first parameter.   Its  second parameter 
is the total ordering used to order the binary tree.  This parameter,
represented as \texttt{<=}, has the type \texttt{(total\_order?[T])} 
consisting of those binary relations on \texttt{T} that are total
orderings, that is, those that are reflexive, transitive, antisymmetric, and
linear.  Note that the type of the second parameter to this theory depends on
the first parameter \texttt{T}\@.  

%% \comment{
%% The next few declarations define various recursive functions.
%% These recursive definitions are given in several distinct styles
%% to illustrate the various options provided by PVS\@.
%% The first declaration defines a combinator \texttt{checkall} that
%% checks if all the values in a given binary tree \texttt{A} satisfy a given
%% predicate \texttt{pp}.  It uses the recursion combinator defined in \texttt{%% binary\_tree\_rec\_mod} so that there is no explicit recursion used in
%% the definition.  The value returned in the base case is \texttt{TRUE}
%% since there are no values in a leaf node.  In the recursive case,
%% \texttt{checkall} forms the conjunction of the values returned by the
%% recursive calls with the value of the predicate \texttt{pp} applied to
%% the value at the root node.   This is not the most convenient way of
%% defining \texttt{ checkall} since it involves an extra function \texttt{%% binary\_tree\_rec} in its definition that has to be unwound during a
%% proof.  The recursion combinators are mostly useful for defining
%% the measure functions that are used to demonstrate the termination of
%% functions defined by explicit recursion.  
%% \begin{session*}
%% checkall((pp : pred[T]), A): bool =
%%   binary_tree_rec(TRUE,
%%             (LAMBDA x, (a, b : bool):
%%                (a AND b AND pp(x))))(A)
%% \end{session*}
%% %
%% The first step is to define a function that measures the size of a binary
%% tree as the number of nodes in the tree.  The recursion combinator \texttt{%% reduce\_nat} comes in handy for this purpose.
%% \begin{session*}
%% \end{session*}
%% }

We can now use the \texttt{every} combinator to define when 
a binary tree is ordered relative to the theory parameter \texttt{<=}\@.
This notion is captured by the predicate \texttt{ordered?}\ on binary trees.
Since \texttt{ordered?}\ will be defined by a direct recursion, its definition
will need a measure that demonstrates the termination of the recursion.
In the definition of \texttt{size} below, the recursion combinator \texttt{reduce\_nat} is used to count the number of value nodes in a
given binary tree.  This function is defined to return \texttt{0} when
given a \texttt{leaf}, and to increment the sum of the sizes of the left
and right subtrees by \texttt{1} when given a \texttt{node}\@.  
\begin{session*}\label{size}
size(A) : nat =
  reduce_nat(0, (LAMBDA x, i, j: i + j + 1))(A)
\end{session*}

The recursive definition of \texttt{ordered?}\ shown below returns \texttt{TRUE} in the base case since a leaf node is clearly an ordered tree by
itself.  In the recursive case, the definition ensures that the left
and right subtrees of the given tree \texttt{A} are themselves ordered.  
It also uses \texttt{every} to check that all the values in the left
subtree are no greater than the value \texttt{val(A)} at \texttt{A}, and the
values in the right subtree are no less than the value at \texttt{A}\@.
%% \comment{
%% \footnote{The expression \texttt{every((LAMBDA\ y:\ y<=val(A)), left(A))}
%% could also be written as \texttt{every(LAMBDA\ y:\ y<=val(A))(left(A))}
%% or \texttt{(every!\ y:\ y<=val(A))(left(A))}\@.  }}

The measure \texttt{size} is used to demonstrate the termination of the
recursion displayed by \texttt{ordered?}\@.  The proper subterm relation
shown in~\pvsref{propersubterm} could also be used as a well-founded
relation in establishing the termination of \texttt{ordered?} by
writing \texttt{MEASURE A BY <<} (see~\pvsref{aritheval}) in place of \texttt{MEASURE size}\@.    
\begin{session*}\label{ordered?}
ordered?(A) : RECURSIVE bool =
  (IF node?(A) THEN (every((LAMBDA y: y<=val(A)), left(A)) AND
                     every((LAMBDA y: val(A)<=y), right(A)) AND
                     ordered?(left(A)) AND
                     ordered?(right(A)))
         ELSE TRUE ENDIF)
  MEASURE size
\end{session*}
When the above definition is typechecked, two proof obligations (TCCs)
are generated corresponding to the termination requirements for the
two recursive calls.  The first one requires that the size of the left
subtree of a binary tree \texttt{A} must be smaller than the size of 
\texttt{A}\@.   The second proof obligation requires that the size of the
right subtree of \texttt{A} must be smaller than the size of \texttt{A}\@.  
Note how the governing \texttt{IF-THEN-ELSE} condition and the preceding
conjuncts are  included as antecedents in the proof obligations below.
\begin{session*}
ordered?_TCC1: OBLIGATION
      (FORALL (A):
         node?(A)
             AND every((LAMBDA y: y <= val(A)), left(A))
               AND every((LAMBDA y: val(A) <= y), right(A))
             IMPLIES size(left(A)) < size(A));

ordered?_TCC2: OBLIGATION
      (FORALL (v: [binary_tree[T] -> bool], A):
         node?(A)
             AND every((LAMBDA y: y <= val(A)), left(A))
               AND every((LAMBDA y: val(A) <= y), right(A)) AND v(left(A))
             IMPLIES size(right(A)) < size(A));
\end{session*}
The PVS~Emacs command \texttt{M-x tc} typechecks a file in PVS\@.  The
PVS~Emacs command \texttt{M-x tcp} can be used to both typecheck the file
and attempt to prove the resulting TCCs using the existing proof (if
there is one) or a built-in strategy according to the source
of the TCC (subtype, termination, existence, assuming, etc.).  As it turns
out, the \texttt{termination-tcc} strategy automatically proves both
\texttt{ordered?\_TCC1} and  \texttt{ordered?\_TCC2}\@.


The next definition in the \texttt{obt} theory is that of the \texttt{insert} operation.  The term \texttt{insert(x, A)} returns that binary
tree obtained by inserting the value \texttt{x} at the
appropriate position in the binary tree \texttt{A}.  The \texttt{insert} operation is also defined by 
recursion but employs the \texttt{CASES} construct instead of the
\texttt{IF-THEN-ELSE} conditional.  In the base case, when the argument \texttt{A}
matches the term \texttt{leaf}, the binary tree containing the
single value \texttt{x} is returned as the result.  In the recursion case,
the argument \texttt{A} has the form \texttt{node(y, B, C)}, and
if \texttt{x} is at most \texttt{y} according to the given total ordering on
the type \texttt{T}, then we reconstruct the node with value \texttt{y},
left subtree \texttt{insert(x, B)}, and right subtree \texttt{C}\@.
Otherwise, we reconstruct the node with value \texttt{y}, left subtree
\texttt{B}, and right subtree \texttt{insert(x, C)}\@.
\begin{session*}
insert(x, A): RECURSIVE binary_tree[T] =
  (CASES A OF
    leaf: node(x, leaf, leaf),
    node(y, B, C): (IF x<=y THEN node(y, insert(x, B), C)
                            ELSE node(y, B, insert(x, C))
                    ENDIF)
   ENDCASES)
  MEASURE size(A)
\end{session*}

When the above definition is typechecked, two termination proof
obligations are generated corresponding to the two recursive
invocations of \texttt{insert}.  Both proof obligations \texttt{insert\_TCC1}
and \texttt{insert\_TCC2} are automatically discharged by the default
\texttt{termination-tcc} strategy.  
\begin{session*}
insert_TCC1: OBLIGATION
      (FORALL (B: binary_tree[T], C: binary_tree[T], y: T, A, x):
         A = node(y, B, C) AND x <= y IMPLIES size(B) < size(A));

insert_TCC2: OBLIGATION
      (FORALL (B: binary_tree[T], C: binary_tree[T], y: T, A, x):
         A = node(y, B, C) AND NOT x <= y IMPLIES size(C) < size(A))
\end{session*}

%% \comment{
%% Following the definition of \texttt{insert}, we have a series of lemmas.
%% The first two of lemmas state some trivial properties of \texttt{%% checkall}\@.   They are useful because \texttt{checkall} is
%% defined in terms of \texttt{binary\_tree\_rec}, and these lemmas save the
%% trouble of unwinding the definitions of \texttt{checkall} and  \texttt{%% binary\_tree\_rec}\@.  These lemmas would not be needed if \texttt{%% checkall} had been defined with an explicit recursion.  
%% Note that the above  definition of \texttt{checkall} is merely meant
%% to illustrate the use of \texttt{binary\_tree\_rec}\@.  
%% %
%% \begin{session*}
%% checkall_leaf: LEMMA
%%   checkall(pp, leaf)
%% %
%% checkall_step: LEMMA
%%   pp(x) IMPLIES
%%   checkall(pp, node(x, leaf, leaf))
%% \end{session*}
%% }

The following lemma states an interesting property of \texttt{insert}\@.
 Its proof requires the use of induction over binary trees. 
It asserts that if every value in the tree \texttt{A} has property \texttt{pp},
and the value \texttt{x} also has property \texttt{pp}, then every value in
the result of inserting \texttt{x} into \texttt{A} has property \texttt{pp}\@.
\begin{session*}\label{oistep}
ordered?_insert_step: LEMMA
   pp(x) AND every(pp, A) IMPLIES every(pp, insert(x, A))
\end{session*}

The  theorem \texttt{ordered?\_insert}  asserts the important
property of \texttt{insert} that it returns an ordered binary tree when
given an ordered binary tree.
\begin{session*}\label{ordered?_insert}
ordered?_insert: THEOREM
   ordered?(A) IMPLIES ordered?(insert(x, A))
\end{session*}

We examine some proofs of this theorem in  Section~\ref{OBT-proofs}.  

%% \comment{
%% and in fact the proof chain
%% for these proofs is also complete, implying that the proof does not
%% depend on any unproved facts.  The automatic strategy fails, however, on
%% the two termination TCCs generated by \texttt{insert}\@.   To see
%% why the strategy failed, we type \texttt{M-x show-tccs} to bring up a
%% buffer containing the TCCs.  We then attempt to explicitly prove the
%% first of the two unproved TCCs by invoking \texttt{M-x pr} with the
%% cursor on the TCC\@.  We then go through the following script:
%% \begin{session*}
%% insert_TCC1 :   
%% %
%%   |-------
%% \{1\}   (FORALL A, (B, C: binary_tree[T]), x,
%%               (y: T): A = node(y, B, C) AND x <= y IMPLIES size(B) < size(A))
%% %
%% Rule? (tcc)
%% size rewrites size(B) to binary_tree_rec(0, (LAMBDA x, i, j: i + j + 1))(B)
%% size rewrites size(A) to binary_tree_rec(0, (LAMBDA x, i, j: i + j + 1))(A)
%% binary_tree_rec rewrites binary_tree_rec(0, (LAMBDA x, i, j: i + j + 1))(A!1) to
%%  binary_tree_rec(0, (LAMBDA x, i, j: i + j + 1))(left(A!1))
%%   + binary_tree_rec(0, (LAMBDA x, i, j: i + j + 1))(right(A!1))
%%   + 1
%% %
%% Trying repeated skolemization, instantiation, and if-lifting,
%% this simplifies to: 
%% insert_TCC1 :   
%% %
%% \{-1\}   A!1 = node(y!1, B!1, C!1)
%% \{-2\}   x!1 <= y!1
%%   |-------
%% \{1\}   binary_tree_rec(0, (LAMBDA x, i, j: i + j + 1))(B!1)
%%         < binary_tree_rec(0, (LAMBDA x, i, j: i + j + 1))(left(A!1))
%%           + binary_tree_rec(0, (LAMBDA x, i, j: i + j + 1))(right(A!1))
%%           + 1
%% %
%% Rule? 
%% \end{session*}
%% The \texttt{tcc} strategy replaces the universal quantifiers with Skolem
%% constants, then applies propositional simplification and expands the
%% definitions in the given sequent to the extent that it can, and then
%% applies the decision procedures, but this is not enough to completely
%% prove the TCC\@.  The missing step is that the strategy is not able to
%% replace \texttt{A!1} with \texttt{node(y!1, B!1, C!1)} using the equality in
%% the antecedent.  We manually perform this replacement using the \texttt{%% replace} rule.
%% \begin{session*}
%% Rule? (replace -1)
%% Replacing using formula -1,
%% this simplifies to: 
%% insert_TCC1 :   

%% [-1]   A!1 = node(y!1, B!1, C!1)
%% [-2]   x!1 <= y!1
%%   |-------
%% \{1\}   binary_tree_rec(0, (LAMBDA x, i, j: i + j + 1))(B!1)
%%         <
%%         binary_tree_rec(0,
%%                         (LAMBDA x, i,
%%                                 j: i + j + 1))(left(node(y!1, B!1, C!1)))
%%           +
%%           binary_tree_rec(0,
%%                           (LAMBDA x, i,
%%                                   j: i + j
%%                                + 1))(right(node(y!1, B!1, C!1)))
%%           + 1

%% Rule?
%% \end{session*}
%% With the term \texttt{A!1} replaced in the consequent of the sequent, we
%% can now apply the decision procedures and simplifiers as shown below.
%% \begin{session*}
%% Rule? (assert)
%% Invoking decision procedures,
%% this simplifies to: 
%% insert_TCC1 :   

%% [-1]   A!1 = node(y!1, B!1, C!1)
%% [-2]   x!1 <= y!1
%%   |-------
%% \{1\}   binary_tree_rec(0, (LAMBDA x, i, j: i + j + 1))(B!1)
%%         < binary_tree_rec(0, (LAMBDA x, i, j: i + j + 1))(B!1)
%%           + binary_tree_rec(0, (LAMBDA x, i, j: i + j + 1))(C!1)
%%           + 1

%% Rule?
%% \end{session*}
%% While this invocation of \texttt{assert} did simplify the redexes
%% \texttt{left(node(y!1, B!1, C!1))} and \texttt{right(node(y!1, B!1, C!1))},
%% it did not prove the goal that at this point looks to be obviously
%% true.  The reason for this failure to complete the proof is that the
%% proof requires the information that the applications of \texttt{%% binary\_tree\_rec}  in the above sequent all return natural number
%% values.  This information is not available at the above invocation of
%% \texttt{assert} since the recording of this information is the last step
%% in the processing carried out by \texttt{assert}\@.  A second application
%% of \texttt{assert} now easily completes the proof. 
%% \begin{session*}
%% Rule? (assert)
%% Invoking decision procedures,
%% Q.E.D.


%% Run time  = 21.12 secs.
%% Real time = 34.63 secs.
%% \end{session*}

%% The proof of \texttt{insert\_TCC2} is identical to the one above.  
%%  To apply the proof of \texttt{insert\_TCC1} to \texttt{insert\_TCC2},
%% we can use \texttt{M-x edit-proof} to bring up the proof script for \texttt{%% insert\_TCC1} and attach it to \texttt{insert\_TCC2} by placing the
%% cursor on the latter declaration and typing \texttt{C-c C-i} at the \texttt{%% Proof} buffer.


%% We now explain the definition of the \texttt{tcc} rule shown below.
%% \begin{session*}
%% (defstep tcc ()
%%   '(then*
%%     (auto-rewrite-defs) 
%%     (ground)
%%     (repeat* (try (then* (repeat* (skolem!))
%%                          (repeat* (inst?))
%%                          (repeat* (lift-if)))
%%               (apply (then* (ground)(assert)))
%%               (skip))))
%%     "(tcc): The guts of the tcc-strategy.  Does auto-rewrite-explicit,
%% then applies skolem!, inst?, lift-if, and ground, until
%% nothing works."
%%     "~%Trying repeated skolemization, instantiation, and if-lifting")
%% \end{session*}
%% The first argument to \texttt{defstep} is the name of the rule being
%% defined.  The second argument is the list of arguments, which in this
%% case is empty.  The third argument is the body of the definition.  The
%% \texttt{tcc} rule is defined to first introduce the definitions of all
%% functions in the statement of the theorem being proved as rewrite
%% rules.  It then invokes the \texttt{ground} rule for applying
%% propositional simplification and decision procedures.  Following this
%% there is a \texttt{repeat*} loop where it first tries to repeatedly apply
%% each of \texttt{skolem!}, \texttt{inst?}\ and \texttt{lift-if}.  When each of
%% these inner \texttt{repeat*} loops has terminated, the \texttt{tcc} rule
%% applies \texttt{ground} followed by \texttt{assert} to any resulting
%% subgoals.  The second \texttt{assert} is useful in further simplifying
%% any subgoals returned by \texttt{ground}.  If the first argument to \texttt{%% try} has no effect, then the third argument, \texttt{(skip)}, is applied.
%% Since the \texttt{skip} rule has no effect on the current goal, the \texttt{%% repeat*} recursion is terminated.  The fourth argument to \texttt{%% defstep} is the help string, and the final argument is the commentary
%% printed when the \texttt{tcc} rule is successfully invoked in a proof.
%% }

\chapter{In-line and Enumeration Types}\label{enumtypes}

 The example of binary trees illustrated how abstract datatypes can be
declared as theories (that are automatically expanded) within PVS\@.
Abstract datatypes can be declared within other theories as long
as they do not employ any parameters.  Note that PVS  has
type parameterization only at the theory level and not at the declaration
level.   For example, the type of 
combinators constructed out of the $K$ and $S$ combinators is captured
by the following declaration that can occur at the declaration level
within a theory.   The axioms generated by the \texttt{DATATYPE} declaration
can be viewed using the PVS Emacs command \texttt{M-x
ppe}\@.  
\begin{session*}
combinators  : THEORY
  BEGIN
  combinators: DATATYPE
       BEGIN
         K: K?
         S: S?
         app(operator, operand: combinators): app?
       END combinators

  x, y, z: VAR combinators

  reduces_to: PRED[[combinators, combinators]]

  K: AXIOM reduces_to(app(app(K, x), y), x)
  S: AXIOM reduces_to(app(app(app(S, x), y), z), app(app(x, z), app(y, z)))

  END combinators
\end{session*}

The most frequently used such \emph{in-line} abstract datatypes are
enumeration types.  For example, the type of \texttt{colors} consisting of
\texttt{red}, \texttt{white}, and \texttt{blue} can given by the following
in-line datatype declaration.
\begin{session*}
  colors: DATATYPE
     BEGIN
       red: red?
       white: white?
       blue: blue?
     END colors
\end{session*}
  The above declaration is a rather verbose way
of defining the type of \texttt{colors}\@.  PVS provides an
abbreviation mechanism  that allows the above declaration to be
expressed more succinctly as shown below.
\begin{session*}
  colors: TYPE = \{red, white, blue\}
\end{session*}

All of the axiomatized properties of such enumeration types are built
into the PVS proof checker as shown  in the previous section so that
no axioms about enumeration types need ever be explicitly used.

\chapter{Disjoint Unions}\label{disjoint-union}

The type constructor for the disjoint union of two types is popular
enough to be included in several languages.  The disjoint union of two
sets $A$ and $B$ is a set  in which each element is tagged according
to whether it is from $A$ or from $B$\@. 
  It is easy to see that the type analogue of the
disjoint union operation can be defined using the \texttt{DATATYPE}
mechanism of PVS as shown below:
\begin{session*}
disj_union[A, B: TYPE] : DATATYPE
  BEGIN
    inl(left : A): inl?
    inr(right : B): inr?
  END disj_union
\end{session*}
The type \texttt{disj\_union[nat, bool]}  then includes
values such as \texttt{inl(1)} and \texttt{inr(TRUE)}\@.  

Rushby~\cite{Rushby95:movie} presents a toy compiler verification
exercise~\cite{Weber-Wulff93} in PVS and presents an extensive
discussion of the use of disjoint unions in PVS specifications and
proofs.

\chapter{Mutually Recursive Datatypes}\label{mutual}

Mutually recursive datatypes arise quite frequently in
programming and specification.  A common example is that of
a language definition where type expressions contain terms and vice-versa.  
Mutually recursive type definitions are not directly admissible using
the PVS datatype mechanism.  But most typical mutual recursive types can,
in fact, be  
defined as a single datatype in PVS with subtypes that
group together classes of constructors.  PVS~2 has been
extended to admit such datatypes with \emph{sub-datatypes}\@.
The example below describes the class of arithmetic expressions that
include numbers, sums, and conditional expressions classified
by the sub-datatype \texttt{term}, where the test
component of a conditional expression is a boolean expression
classified by the subdatatype \texttt{expr}.   Thus sub-datatypes
are a way of collecting together groups of constructors of a datatype
that form one part of a mutually recursive datatype definition.  
In the example below,  boolean expressions are defined as equalities
between arithmetic expressions, and conditional arithmetic expressions
contain boolean subexpressions, so that 
arithmetic and boolean expressions are mutually recursive.  
\begin{session*}
arith: DATATYPE WITH SUBTYPES expr, term
 BEGIN
  num(n:int): num?              :term
  sum(t1:term,t2:term):   sum?  :term
% ...
  eq(t1: term, t2: term): eq?   :expr
  ift(e: expr, t1: term, t2: term): ift? :term
% ...
 END arith
\end{session*}

The only restriction on the use of subdatatypes other than those listed in
Section~\ref{lists} is that the sub-datatypes should be pairwise distinct
and differ from the datatype itself\@.  In particular, sub-datatypes need
not actually be used in which case they are empty.  It is possible to
define mutual recursive types that lead to empty constructor subtypes
such as if the \texttt{eq} constructor in the \texttt{arith} datatype was
specified as \texttt{eq(t1:\ expr, t2:\ expr):\ eq?\ :\ expr}\@.  

An evaluator for such arithmetic/boolean expressions can be defined
as \texttt{eval} whose range type is a disjoint union of \texttt{bool} and \texttt{int} (according to whether the input expression is of type \texttt{expr} or
\texttt{term}.  The function \texttt{eval} is therefore dependently typed
to return values of type \texttt{(bool?)} on  inputs of type \texttt{expr}
and values of type \texttt{(int?)} on inputs of type \texttt{term}\@. 
\begin{session*}\label{aritheval}
arith_eval: THEORY
 BEGIN
  IMPORTING arith

  value: DATATYPE
   BEGIN
    bool(b:bool): bool?
    int(i:int): int?
   END value

  eval(a: arith): RECURSIVE
         \{v: value | IF expr(a) THEN bool?(v) ELSE int?(v) ENDIF\} =
   CASES a OF
    num(n):       int(n),
    sum(n1, n2):  int(i(eval(n1)) + i(eval(n2))),
    eq(n1, n2):   bool(i(eval(n1)) = i(eval(n2))),
    ift(e, n1, n2): IF b(eval(e)) THEN eval(n1) ELSE eval(n2) ENDIF    
   ENDCASES
   MEASURE a BY <<

 END arith_eval
\end{session*}

\chapter{Lifting Subtyping on Recursive Datatype Parameters}\label{lifting}

The datatype mechanism in PVS~2.0 had the limitation that though the type
of \texttt{nat} of natural numbers is a subtype of the type \texttt{int} of
integers, the type \texttt{list[nat]} of lists over the
natural numbers is not a subtype of the type \texttt{list[int]} of
lists over the integers.  The datatype mechanism in PVS~2.1 has
been modified to lift such subtyping over positive parameters to the
corresponding abstract datatypes.    In general, given a datatype \texttt{D} with
a positive type parameter, we have
\begin{alltt}
  D[\{x: T | p(x)\}] \(\equiv\) \{d: D[T] | every(p)(d)\}.
\end{alltt}

While \texttt{cons[nat]} is neither syntactically nor semantically identical
to \texttt{cons[int]}, constructor \emph{applications} involving \texttt{cons[int]} and \texttt{cons[nat]} such as \texttt{cons[nat](0, null[nat])} and \texttt{cons[int](0, null[int])} are syntactically identical.  Also, constructors
that are declared to have no accessors (e.g., \texttt{null}) are
syntactically equal, so \texttt{null[int] \(\equiv\) null[real]}, but \texttt{null[int]} and \texttt{null[bool]} belong to incompatible types. 

In general, when a constructor, accessor, or recognizer occurs as an
operator of an application, the actual parameter is only used for testing
compatibility.    Note that the actual parameter is not actually ignored.
For example, the expression \texttt{cons[nat](-1, null)} is not type correct
and generates the unprovable proof obligation \texttt{-1 > 0}\@. 

When multiple parameters are involved, only the positive ones satisfy the
subtyping equivalences given above.  Thus in the datatype declaration
\begin{alltt}
dt[t1, t2: TYPE, c: t1]: DATATYPE
 BEGIN
  b: b?
  c(a1:[t1 -> t2], a2: dt): c?
 END dt
\end{alltt}
only \texttt{t2} occurs positively, so \texttt{dt[int, nat, 3]} is a subtype of
\texttt{dt[int, int, 3]}, but bears no relation to \texttt{dt[nat, nat, 3]} or
to \texttt{dt[int, nat, 2]}.


\chapter{Representations of Recursive Ordinals}\label{section:ordinals}

Ordinals are needed to provide lexicographic termination measures for
recursive functions.  The Ackermann function provides a well known example
of a doubly recursive function that requires a lexicographic measure.
P\'eter's version~\cite{PETER} of the Ackermann function is defined in the
theory \texttt{ackermann} as \texttt{ack}\@.
\begin{session*}\label{ackermann}
ackermann: THEORY
BEGIN
i, j, k, m, n: VAR nat

ack(m,n): RECURSIVE nat =
  (IF m=0 THEN n+1
          ELSIF n=0 THEN ack(m-1,1)
                    ELSE ack(m-1, ack(m, n-1))
        ENDIF)
  MEASURE lex2(m, n)

  \vdots
END ackermann
\end{session*}
The lexicographic termination measure for \texttt{ack} is computed by
the function \texttt{lex2} (see~\pvsref{lex2})
which returns a representation for the
ordinal in the lexicographic ordering.  The ordinal $\epsilon_0$ is the
least ordinal $x$ such that $x = \omega^x$, and therefore includes
$0, 1, \ldots, \omega, \omega+1,\ldots \omega + \omega, \ldots, 3*\omega,
\ldots, \omega^2,\ldots, \omega^\omega, \ldots, \omega^{.^{.^\omega}}, \ldots.$
The sub-$\epsilon_0$ ordinals can be represented using the Cantor normal
form which asserts that to any non-zero ordinal $\alpha$,  there are $n$
ordinals $\alpha_1, \ldots, \alpha_n$ with $\alpha_1 \leq
\ldots \leq 
\alpha_n < \alpha$, such that $\alpha = \omega^{\alpha_1} + \omega^{\alpha_2}
+ \ldots 
+ \omega^{\alpha_n}$\@.  We can make this representation slightly more
compact by adding natural number coefficients so that to any $\alpha$,
there are ordinals $\alpha_1,\ldots, \alpha_n$ such that
$\alpha_1 \leq \ldots \leq \alpha_n < \alpha$,  and
natural numbers $c_1,\ldots, c_n$ 
such that $\alpha = c_1 * \omega^{\alpha_1} + c_2 * \omega^{\alpha_2} + \ldots
+ c_n * \omega^{\alpha_n}$\@.  It is easy to see that a lexicographic
measure can be given by $n * \omega^0 + m * \omega$ which is just $n + m* \omega$\@.  

We now explain how the sub-$\epsilon_0$ ordinals are defined in the PVS
prelude.   We start by defining an
\texttt{ordstruct} datatype that represents ordinal-like structures.
\begin{session*}
ordstruct: DATATYPE
 BEGIN
  zero: zero?
  add(coef: posnat, exp: ordstruct, rest: ordstruct): nonzero?
 END ordstruct
\end{session*}

In intuitive terms, the ordinal represented by \texttt{zero} is
\texttt{0}, and the ordinal represented by \texttt{add(c, alpha, beta)}
given by, say $\mbox{\textit{ordinal}}(\texttt{add(c, alpha, beta)})$ is $
\texttt{c} * \omega^{\mbox{\textit{ordinal}}(\texttt{alpha})} +
\mbox{\textit{ordinal}}(\texttt{beta})$\@.  We can then define an ordering
relation on \texttt{ordstruct} terms as given by \texttt{<}
in~\pvsref{ordlessp}\@.  It compares \texttt{add(i, u, v)} against
\texttt{add(j, z, w)} by either recursively ensuring \texttt{u < z}, or
checking that \texttt{u} is syntactically identical to \texttt{z} and
either \texttt{i < j} or \texttt{i = j} and recursively \texttt{v~<~w}\@.
\begin{session*}\label{ordlessp}
ordinals: THEORY
 BEGIN
  i, j, k: VAR posnat
  m, n, o: VAR nat
  u, v, w, x, y, z: VAR ordstruct
  size: [ordstruct->nat] = reduce[nat](0, (LAMBDA i, m, n: 1 + m+n));

  <(x, y): RECURSIVE bool =
     CASES x OF
        zero: NOT zero?(y),
        add(i, u, v): CASES y OF
                        zero: FALSE,
                        add(j, z, w): (u<z) OR
                                      (u=z) AND (i<j) OR
                                      (u=z) AND (i=j) AND (v<w)
                      ENDCASES
     ENDCASES
   MEASURE size(x);
\end{session*}
This is not quite the ordering relation we want since it will obviously
only work for normalized (and therefore, canonical) representations where
the exponent ordinals appear in sorted (decreasing) order.  In particular,
note that the use of syntactic identity on \texttt{ordstruct} terms will not
work unless the terms are in fact canonical representatives.  It is easy
to define a predicate which identifies an \texttt{ordstruct} term as being
in the required Cantor normal form by defining a recursive predicate
\texttt{ordinal?}\ as shown in~\pvsref{ordinal?}\@.  
\begin{session*}\label{ordinal?}
  >(x, y): bool = y < x;
  <=(x, y): bool = x < y OR x = y;
  >=(x, y): bool = y < x OR y = x

  ordinal?(x): RECURSIVE bool =
    CASES x OF
      zero: TRUE,
      add(i, u, v): (ordinal?(u) AND ordinal?(v) AND 
                      CASES v OF
                        zero: TRUE,
                        add(k, r, s): r < u
                      ENDCASES)
    ENDCASES
   MEASURE size

  ordinal: NONEMPTY_TYPE = (ordinal?)
\end{session*}
The definition of \texttt{ordinal?}\ checks \texttt{add(i, u, v)} to
recursively ensure that \texttt{u} and \texttt{v} are ordinals, and that
in \texttt{add(i, u, add(k, r, s))}, we have \texttt{r < u}\@.  This latter
use of the ordering relation is acceptable since we have already
checked that \texttt{r} and \texttt{u} are proper normal forms.
The definition of \texttt{lex2} is given in~\pvsref{lex2}.  Note that
\texttt{add(n, zero, zero)} represents $n$,
\texttt{add(m, add(1, zero, zero), zero)} represents $m * \omega$,
and \texttt{add(m, add(1,zero,zero), add(n,zero, zero))} represents
$n + m * \omega$\@.\footnote{The PVS \texttt{CONVERSION} mechanism can be
used to gracefully embed the natural numbers into the \texttt{ordinal} type by
converting \texttt{0} to \texttt{zero}, and a positive number \texttt{n} to
\texttt{add(n, zero, zero)}\@.  }
\begin{session*}\label{lex2}
  lex2(m, n): ordinal = 
    (IF m=0
       THEN IF n = 0
               THEN zero
               ELSE add(n, zero, zero)
            ENDIF
       ELSIF n = 0 THEN add(m, add(1,zero,zero),zero)
       ELSE add(m, add(1,zero,zero), add(n,zero, zero))
     ENDIF)

  lex2_lt: LEMMA
    lex2(i, j) < lex2(m, n) =
     (i < m OR (i = m AND j < n))
\end{session*}

Returning to the example of the Ackermann function in~\pvsref{ackermann},
the measure \texttt{lex2(m, n)} generates three termination TCCs
corresponding to the three recursive invocations of the function.
\begin{session*}
ack_TCC2: OBLIGATION
      (FORALL (m, n): NOT m = 0 AND n = 0
        IMPLIES lex2(m - 1, 1) < lex2(m, n));

ack_TCC5: OBLIGATION
      (FORALL (m, n):
         NOT m = 0 AND NOT n = 0
          IMPLIES lex2(m, n - 1) < lex2(m, n));

ack_TCC6: OBLIGATION
      (FORALL (v: [[nat, naturalnumber] -> nat], m, n):
         NOT m = 0 AND NOT n = 0
             IMPLIES lex2(m - 1, v(m, n - 1)) < lex2(m, n));
\end{session*}
All three TCCs are proved automatically by the default \texttt{termination-tcc} strategy.   

%% \comment{
%% \begin{session*}
%% ackr(n, (f: [nat->nat])): RECURSIVE nat =
%%   (IF n = 0 THEN f(1) ELSE f(ackr(n - 1, f)) ENDIF)
%%  MEASURE n

%% ackrmn(m)(n): RECURSIVE nat = 
%%  (IF m = 0 THEN n + 1 ELSE ackr(n, ackrmn(m - 1)) ENDIF)
%%  MEASURE m

%% ack_ackrmn: THEOREM
%%   ack(m, n) = ackrmn(m)(n)
%% \end{session*}
%% }


\chapter{Some Illustrative Proofs about Ordered Binary Trees}
\label{OBT-proofs}

We present two proofs of \texttt{ordered?\_insert} shown in~\pvsref{ordered?_insert}\@.  The second proof
exhibits a greater level of automation than the first proof.  The first
proof illustrates the various low-level datatype related proof commands
that are 
provided by PVS, and the second proof illustrates how these commands
can be combined to form more powerful and automatic proof strategies.
Strategies
are similar to the \textit{tactics} of the LCF~\cite{LCF} family of proof
checkers.

\section{A Low-level Proof}

When we invoke \texttt{M-x pr} on \texttt{ordered?\_insert}, the theorem to
be proved is displayed in the \texttt{*pvs*} buffer, and we are prompted
for an inference rule by the \texttt{Rule?} prompt.  Since the proof is by
induction, the first 
step in the proof is the command \texttt{(induct "A")}\@.  This indicates
that we wish to invoke the \texttt{induct} strategy with \texttt{A} as the
induction variable.  The induction strategy finds the induction axiom
corresponding to the datatype of \texttt{A}, instantiates it suitably, and
simplifies it to generate the base and induction cases.  We are then
presented the base case of the proof.  (The induction case can be
displayed with the PVS Emacs command \texttt{M-x~siblings}\@. 
\begin{session*}
 ordered?_insert :  
  |-------
\{1\}   (FORALL (A: binary_tree[T], x: T):
         ordered?(A) IMPLIES ordered?(insert(x, A)))
Rule? \fbox{(induct "A")}
Inducting on A,
this yields  2 subgoals: 
ordered?_insert.1 :  
  |-------
\{1\}   (FORALL (x: T): ordered?(leaf) IMPLIES ordered?(insert(x, leaf)))
\end{session*}
In the next  step, we replace the universally quantified variable
with a Skolem constant and flatten the sequent by simplifying all
top-level propositional connectives that are disjunctive (i.e.,
negations, positive implications and disjunctions, and negative
conjunctions).  
\begin{session*}
Rule? \fbox{(skosimp)}
Skolemizing and flattening,
this simplifies to: 
ordered?_insert.1 :  
\{-1\}   ordered?(leaf)
  |-------
\{1\}   ordered?(insert(x!1, leaf))
\end{session*}
The obvious step now is to open up the definitions of \texttt{insert} and
\texttt{ordered?}\@.  This is done by two invocations of the
\texttt{expand} rule.
\begin{session*}
Rule? \fbox{(expand "insert")}
Expanding the definition of insert,
this simplifies to: 
ordered?_insert.1 :  
[-1]   ordered?(leaf)
  |-------
\{1\}   ordered?(node(x!1, leaf, leaf))

Rule? \fbox{(expand "ordered?")}
Expanding the definition of ordered?,
this simplifies to: 
ordered?_insert.1 :  
  |-------
\{1\}   (every((LAMBDA (y: T): y <= x!1), leaf)
           AND every((LAMBDA (y: T): x!1 <= y), leaf)
             AND ordered?(leaf) AND ordered?(leaf))
\end{session*}
The problem now is that all the occurrences of \texttt{ordered?}\ are expanded
so that the antecedent formula \texttt{ordered?(leaf)}
reduces to \texttt{TRUE} and vanishes from the sequent.  This formula in
its unexpanded form is actually useful since it occurs in the
consequent part of the sequent.  We could press on and expand \texttt{ordered?}\ once again or, alternatively, we could undo this step of the
proof and expand \texttt{ordered?}\ more selectively using the command
\texttt{(expand "ordered?" +)}\@. 
\begin{session*}
Rule? \fbox{(undo)}
This will undo the proof to: 
ordered?_insert.1 :  
[-1]   ordered?(leaf)
  |-------
\{1\}   ordered?(node(x!1, leaf, leaf))
Sure? (Y or N): y
ordered?_insert.1 :  
[-1]   ordered?(leaf)
  |-------
\{1\}   ordered?(node(x!1, leaf, leaf))
Rule? \fbox{(expand "ordered?" +)}
Expanding the definition of ordered?,
this simplifies to: 
ordered?_insert.1 :  
[-1]   ordered?(leaf)
  |-------
\{1\}   (every((LAMBDA (y: T): y <= x!1), leaf)
           AND every((LAMBDA (y: T): x!1 <= y), leaf)
             AND ordered?(leaf) AND ordered?(leaf))
\end{session*}
Now an invocation of \texttt{assert} eliminates the occurrences of the
subformula \texttt{ordered?(leaf)} in the consequent since it appears in
the antecedent.  Expanding \texttt{every} then completes the base case of the proof without any further work.
\begin{session*}
Rule? \fbox{(assert)}
Simplifying, rewriting, and recording with decision procedures,
this simplifies to: 
ordered?_insert.1 :  
[-1]   ordered?(leaf)
  |-------
\{1\}   (every((LAMBDA (y: T): y <= x!1), leaf)
           AND every((LAMBDA (y: T): x!1 <= y), leaf))
Rule? \fbox{(expand "every")}
Expanding the definition of every,
this simplifies to: 
ordered?_insert.1 :  
[-1]   ordered?(leaf)
  |-------
\{1\}   TRUE
which is trivially true.
This completes the proof of ordered?_insert.1.
\end{session*}

Having completed the base case of the proof, we are left  with the
induction case.  Our first step here is to apply the rule \texttt{skosimp*}\@.  This is a strategy  
that  repeatedly performs a \texttt{skolem!} followed by a \texttt{flatten} until nothing changes, i.e., it is an iterated form of the
\texttt{skosimp} command.
%% \comment{The strategy \texttt{skosimp} illustrates the
%% difference between a rule and a strategy.  If \texttt{skosimp} were a
%% rule, it would have performed both  \texttt{skolem!} and \texttt{flatten}
%% in one atomic step, but since it is a strategy, these two steps
%% are both visible in the proof.  Note that \texttt{skolem!} is also a
%% strategy that is defined in terms of the \texttt{skolem}
%% rule.\footnote{The strategies \texttt{skolem!}, \texttt{skosimp}, and \texttt{%% skosimp*} will soon be made into defined rules rather than strategies.}}

\begin{session*}
ordered?_insert.2 :  
  |-------
\{1\}   (FORALL (node1_var: T, node2_var: binary_tree[T],
               node3_var: binary_tree[T]):
         (FORALL (x: T):
            ordered?(node2_var) IMPLIES ordered?(insert(x, node2_var)))
             AND
           (FORALL (x: T):
              ordered?(node3_var) IMPLIES ordered?(insert(x, node3_var)))
             IMPLIES
           (FORALL (x: T):
              ordered?(node(node1_var, node2_var, node3_var))
                  IMPLIES
                ordered?(insert(x, node(node1_var, node2_var, node3_var)))))
Rule? \fbox{(skosimp*)}
Repeatedly Skolemizing and flattening,
this simplifies to: 
ordered?_insert.2 :  
\{-1\}   (FORALL (x: T):
         ordered?(node2_var!1) IMPLIES ordered?(insert(x, node2_var!1)))
\{-2\}   (FORALL (x: T):
         ordered?(node3_var!1) IMPLIES ordered?(insert(x, node3_var!1)))
\{-3\}   ordered?(node(node1_var!1, node2_var!1, node3_var!1))
  |-------
\{1\}   ordered?(insert(x!1, node(node1_var!1, node2_var!1, node3_var!1)))
\end{session*}
Now we have a subgoal sequent in which the induction hypotheses are
the formulas number \texttt{-1} and \texttt{-2}, and the induction
conclusion formulas are numbered \texttt{-3} and \texttt{1}.  We
clearly need to expand the definitions of \texttt{insert} and \texttt{ordered?}\ in the induction conclusion.  We first expand \texttt{insert}
and then propositionally simplify the resulting \texttt{IF-THEN-ELSE}
expression as shown below.
\begin{session*}
Rule? \fbox{(expand "insert" +)}
Expanding the definition of insert,
this simplifies to: 
ordered?_insert.2 :  
[-1]   (FORALL (x: T):
         ordered?(node2_var!1) IMPLIES ordered?(insert(x, node2_var!1)))
[-2]   (FORALL (x: T):
         ordered?(node3_var!1) IMPLIES ordered?(insert(x, node3_var!1)))
[-3]   ordered?(node(node1_var!1, node2_var!1, node3_var!1))
  |-------
\{1\}   IF x!1 <= node1_var!1
        THEN ordered?(node(node1_var!1, insert(x!1, node2_var!1), node3_var!1))
      ELSE ordered?(node(node1_var!1, node2_var!1, insert(x!1, node3_var!1)))
      ENDIF

Rule? \fbox{(prop)}
Applying propositional simplification,
this yields  2 subgoals: 
ordered?_insert.2.1 :  
\{-1\}   x!1 <= node1_var!1
[-2]   (FORALL (x: T):
         ordered?(node2_var!1) IMPLIES ordered?(insert(x, node2_var!1)))
[-3]   (FORALL (x: T):
         ordered?(node3_var!1) IMPLIES ordered?(insert(x, node3_var!1)))
[-4]   ordered?(node(node1_var!1, node2_var!1, node3_var!1))
  |-------
\{1\}   ordered?(node(node1_var!1, insert(x!1, node2_var!1), node3_var!1))
\end{session*}
The propositional simplification step generates two subgoals according
to whether the recursive invocation of \texttt{insert} is on the left or
the right subtree.  We first consider the insertion into the left
subtree given by subgoal \texttt{ordered?\_insert.2.1}\@.
We can instantiate the induction hypothesis numbered \texttt{-2}
by applying the \texttt{inst?}\ command which uses syntactic
matching to find instantiating terms for the universally quantified
variable in \texttt{-2}.
\begin{session*}\label{ordered?_insert.2.1}
Rule? \fbox{(inst?)}
Found substitution:
x gets x!1,
Instantiating quantified variables,
this simplifies to: 
ordered?_insert.2.1 :  
[-1]   x!1 <= node1_var!1
\{-2\}   ordered?(node2_var!1) IMPLIES ordered?(insert(x!1, node2_var!1))
[-3]   (FORALL (x: T):
         ordered?(node3_var!1) IMPLIES ordered?(insert(x, node3_var!1)))
[-4]   ordered?(node(node1_var!1, node2_var!1, node3_var!1))
  |-------
[1]   ordered?(node(node1_var!1, insert(x!1, node2_var!1), node3_var!1))
\end{session*}
The next step is to expand the definition of \texttt{ordered?}
in the induction conclusion.  Note that the second argument to the \texttt{expand} proof command is a list of the formula numbers where the expansion
is to be performed.  It makes the proof considerably less robust if it
explicitly mentions such formula numbers, though this can be unavoidable
in some cases.\footnote{PVS is currently being enhanced to allow labels
to be introduced for sequent formulas so that formula selection in the PVS
proof commands can be done with labels as an alternative to formula
numbers.} 
\begin{session*}
Rule? \fbox{(expand "ordered?" (-4 1))}
Expanding the definition of ordered?,
this simplifies to: 
ordered?_insert.2.1 :  
[-1]   x!1 <= node1_var!1
[-2]   ordered?(node2_var!1) IMPLIES ordered?(insert(x!1, node2_var!1))
[-3]   (FORALL (x: T):
         ordered?(node3_var!1) IMPLIES ordered?(insert(x, node3_var!1)))
\{-4\}   (every((LAMBDA (y: T): y <= node1_var!1), node2_var!1)
           AND every((LAMBDA (y: T): node1_var!1 <= y), node3_var!1)
             AND ordered?(node2_var!1) AND ordered?(node3_var!1))
  |-------
\{1\}   (every((LAMBDA (y: T): y <= node1_var!1), insert(x!1, node2_var!1))
           AND every((LAMBDA (y: T): node1_var!1 <= y), node3_var!1)
             AND ordered?(insert(x!1, node2_var!1)) AND ordered?(node3_var!1))
\end{session*}
Applying propositional
simplification \texttt{prop} to the resulting subgoal generates two
further subgoals.  The first of these is easily proved by rewriting
using the lemma \texttt{ordered?\_insert\_step}\@.  Note that this is a
conditional rewrite rule and has the form $A\supset B$, where the
rewriting given by  $B$ can be applied  to a matching instance $\sigma(B)$
only when the 
corresponding $\sigma(A)$ (the condition) is provable.  The \texttt{rewrite} proof strategy attempts to discharge these conditions
automatically, and any undischarged conditions are 
generated as subgoals.
\begin{session*}
Rule? \fbox{(prop)}
Applying propositional simplification,
this simplifies to: 
ordered?_insert.2.1 :  
\{-1\}   ordered?(insert(x!1, node2_var!1))
[-2]   x!1 <= node1_var!1
[-3]   (FORALL (x: T):
         ordered?(node3_var!1) IMPLIES ordered?(insert(x, node3_var!1)))
\{-4\}   every((LAMBDA (y: T): y <= node1_var!1), node2_var!1)
\{-5\}   every((LAMBDA (y: T): node1_var!1 <= y), node3_var!1)
\{-6\}   ordered?(node2_var!1)
\{-7\}   ordered?(node3_var!1)
  |-------
\{1\}   every((LAMBDA (y: T): y <= node1_var!1), insert(x!1, node2_var!1))
Rule? \fbox{(rewrite "ordered?_insert_step")}
Found matching substitution:
A gets node2_var!1,
x gets x!1,
pp gets (LAMBDA (y: T): y <= node1_var!1),
Rewriting using ordered?_insert_step,
This completes the proof of ordered?_insert.2.1.
\end{session*}

We have now completed the part of the proof corresponding to the
insertion into the left subtree.   
Next, we proceed to the case when the \texttt{insert} operation is applied
to the right subtree.  This case is similar to the proof of
\texttt{ordered?\_insert.2.1}\@.  
\begin{session*}
ordered?_insert.2.2 :  
[-1]   (FORALL (x: T):
         ordered?(node2_var!1) IMPLIES ordered?(insert(x, node2_var!1)))
[-2]   (FORALL (x: T):
         ordered?(node3_var!1) IMPLIES ordered?(insert(x, node3_var!1)))
[-3]   ordered?(node(node1_var!1, node2_var!1, node3_var!1))
  |-------
\{1\}   x!1 <= node1_var!1
\{2\}   ordered?(node(node1_var!1, node2_var!1, insert(x!1, node3_var!1)))
\end{session*}
As in~\pvsref{ordered?_insert.2.1} earlier, we apply the step \texttt{inst?}\@.
\begin{session*}
Rule? \fbox{(inst?)}
Found substitution:
x gets x!1,
Instantiating quantified variables,
this simplifies to: 
ordered?_insert.2.2 :  
\{-1\}   ordered?(node2_var!1) IMPLIES ordered?(insert(x!1, node2_var!1))
[-2]   (FORALL (x: T):
         ordered?(node3_var!1) IMPLIES ordered?(insert(x, node3_var!1)))
[-3]   ordered?(node(node1_var!1, node2_var!1, node3_var!1))
  |-------
[1]   x!1 <= node1_var!1
[2]   ordered?(node(node1_var!1, node2_var!1, insert(x!1, node3_var!1)))
\end{session*}
It however  instantiates the formula \texttt{-1} which is not
the appropriate induction hypothesis for the right branch.  To
apply the \texttt{inst?}\ step with greater selectivity, we undo the
last step and supply
a further argument to \texttt{inst?}\ indicating the number of the quantified
formula to be instantiated.
\begin{session*}
Rule? \fbox{(inst? -2)}
Found substitution:
x gets x!1,
Instantiating quantified variables,
this simplifies to: 
ordered?_insert.2.2 :  
[-1]   (FORALL (x: T):
         ordered?(node2_var!1) IMPLIES ordered?(insert(x, node2_var!1)))
\{-2\}   ordered?(node3_var!1) IMPLIES ordered?(insert(x!1, node3_var!1))
[-3]   ordered?(node(node1_var!1, node2_var!1, node3_var!1))
  |-------
[1]   x!1 <= node1_var!1
[2]   ordered?(node(node1_var!1, node2_var!1, insert(x!1, node3_var!1)))
\end{session*}
Now, as before, we expand the definition of \texttt{ordered?}\ in the
induction conclusion formulas \texttt{-3} and \texttt{2}\@.
\begin{session*}
Rule? \fbox{(expand "ordered?" (-3 2))}
Expanding the definition of ordered?,
this simplifies to: 
ordered?_insert.2.2 :  
[-1]   (FORALL (x: T):
         ordered?(node2_var!1) IMPLIES ordered?(insert(x, node2_var!1)))
[-2]   ordered?(node3_var!1) IMPLIES ordered?(insert(x!1, node3_var!1))
\{-3\}   (every((LAMBDA (y: T): y <= node1_var!1), node2_var!1)
           AND every((LAMBDA (y: T): node1_var!1 <= y), node3_var!1)
             AND ordered?(node2_var!1) AND ordered?(node3_var!1))
  |-------
[1]   x!1 <= node1_var!1
\{2\}   (every((LAMBDA (y: T): y <= node1_var!1), node2_var!1)
           AND
         every((LAMBDA (y: T): node1_var!1 <= y), insert(x!1, node3_var!1))
             AND ordered?(node2_var!1) AND ordered?(insert(x!1, node3_var!1)))
\end{session*}
Propositional simplification yields a single goal sequent.
\begin{session*}\label{subgoal2.2}
Rule? \fbox{(prop)}
Applying propositional simplification,
this simplifies to: 
ordered?_insert.2.2 :  
\{-1\}   ordered?(insert(x!1, node3_var!1))
[-2]   (FORALL (x: T):
         ordered?(node2_var!1) IMPLIES ordered?(insert(x, node2_var!1)))
\{-3\}   every((LAMBDA (y: T): y <= node1_var!1), node2_var!1)
\{-4\}   every((LAMBDA (y: T): node1_var!1 <= y), node3_var!1)
\{-5\}   ordered?(node2_var!1)
\{-6\}   ordered?(node3_var!1)
  |-------
\{1\}   every((LAMBDA (y: T): node1_var!1 <= y), insert(x!1, node3_var!1))
[2]   x!1 <= node1_var!1
\end{session*}
As before, we attempt to rewrite the formula \texttt{1} using the
lemma \texttt{ordered?\_insert\_step}, but as shown in~\pvsref{2.2.proof.1},
this does not terminate 
the current branch of the proof.
\begin{session*}\label{2.2.proof.1}
Rule? \fbox{(rewrite "ordered?_insert_step")}
Found matching substitution:
A gets node3_var!1,
x gets x!1,
pp gets (LAMBDA (y: T): node1_var!1 <= y),
Rewriting using ordered?_insert_step,
this simplifies to: 
ordered?_insert.2.2 :  
[-1]   ordered?(insert(x!1, node3_var!1))
[-2]   (FORALL (x: T):
         ordered?(node2_var!1) IMPLIES ordered?(insert(x, node2_var!1)))
[-3]   every((LAMBDA (y: T): y <= node1_var!1), node2_var!1)
[-4]   every((LAMBDA (y: T): node1_var!1 <= y), node3_var!1)
[-5]   ordered?(node2_var!1)
[-6]   ordered?(node3_var!1)
  |-------
\{1\}   node1_var!1 <= x!1
[2]   every((LAMBDA (y: T): node1_var!1 <= y), insert(x!1, node3_var!1))
[3]   x!1 <= node1_var!1
\end{session*}
  We are left with having to discharge one of
the conditions of the rewrite rule, namely \texttt{node1\_var!1 <= x!1}.
This follows from the other consequent formula \texttt{x!1 <=
node1\_var!1} and the observation that \texttt{<=} here is a linear
ordering.
The proof  now requires that the type information of
\texttt{<=} be made explicit using the \texttt{typepred} command.
\begin{session*}
Rule? \fbox{(typepred "<=")}
<= does not uniquely resolve - one of:
  obt.<= : (total_order?[T]),
  reals.<= : [[real, real] -> bool],
  ordinals.<= : [[ordstruct, ordstruct] -> bool]
Restoring the state.
ordered?_insert.2.2 :  
[-1]   ordered?(insert(x!1, node3_var!1))
[-2]   (FORALL (x: T):
         ordered?(node2_var!1) IMPLIES ordered?(insert(x, node2_var!1)))
[-3]   every((LAMBDA (y: T): y <= node1_var!1), node2_var!1)
[-4]   every((LAMBDA (y: T): node1_var!1 <= y), node3_var!1)
[-5]   ordered?(node2_var!1)
[-6]   ordered?(node3_var!1)
  |-------
\{1\}   node1_var!1 <= x!1
[2]   every((LAMBDA (y: T): node1_var!1 <= y), insert(x!1, node3_var!1))
[3]   x!1 <= node1_var!1
\end{session*}
However, the command \texttt{(typepred "<=")} does
not succeed since the typechecker is unable to resolve among
the many possible references  for \texttt{<=}.  
The more explicit
command \texttt{(typepred "obt.<=")} does succeed.\footnote{Note
that in PVS~2.1, the typechecking of input expressions to proof commands
automatically resolves such ambiguities in favor of expressions occurring
in the goal sequent.  Thus, this ambiguity is no longer reported.}
\begin{session*}\label{2.2.proof.2}
Rule? \fbox{(typepred "obt.<=")}
Adding type constraints for  obt.<=,
this simplifies to: 
ordered?_insert.2.2 :  
\{-1\}   total_order?[T](obt.<=)
[-2]   ordered?(insert(x!1, node3_var!1))
[-3]   (FORALL (x: T):
         ordered?(node2_var!1) IMPLIES ordered?(insert(x, node2_var!1)))
[-4]   every((LAMBDA (y: T): y <= node1_var!1), node2_var!1)
[-5]   every((LAMBDA (y: T): node1_var!1 <= y), node3_var!1)
[-6]   ordered?(node2_var!1)
[-7]   ordered?(node3_var!1)
  |-------
[1]   node1_var!1 <= x!1
[2]   every((LAMBDA (y: T): node1_var!1 <= y), insert(x!1, node3_var!1))
[3]   x!1 <= node1_var!1
\end{session*}
We then expand  the definition of \texttt{total\_order?}.
\begin{session*}
Rule? \fbox{(expand "total_order?")}
Expanding the definition of total_order?,
this simplifies to: 
ordered?_insert.2.2 :  
\{-1\}   partial_order?(obt.<=) & dichotomous?(obt.<=)
[-2]   ordered?(insert(x!1, node3_var!1))
[-3]   (FORALL (x: T):
         ordered?(node2_var!1) IMPLIES ordered?(insert(x, node2_var!1)))
[-4]   every((LAMBDA (y: T): y <= node1_var!1), node2_var!1)
[-5]   every((LAMBDA (y: T): node1_var!1 <= y), node3_var!1)
[-6]   ordered?(node2_var!1)
[-7]   ordered?(node3_var!1)
  |-------
[1]   node1_var!1 <= x!1
[2]   every((LAMBDA (y: T): node1_var!1 <= y), insert(x!1, node3_var!1))
[3]   x!1 <= node1_var!1
\end{session*}
Applying \texttt{flatten} removes the conjunction in \texttt{-1}\@.
\begin{session*}
Rule? \fbox{(flatten)}
Applying disjunctive simplification to flatten sequent,
this simplifies to: 
ordered?_insert.2.2 :  
\{-1\}   partial_order?(obt.<=)
\{-2\}   dichotomous?(obt.<=)
[-3]   ordered?(insert(x!1, node3_var!1))
[-4]   (FORALL (x: T):
         ordered?(node2_var!1) IMPLIES ordered?(insert(x, node2_var!1)))
[-5]   every((LAMBDA (y: T): y <= node1_var!1), node2_var!1)
[-6]   every((LAMBDA (y: T): node1_var!1 <= y), node3_var!1)
[-7]   ordered?(node2_var!1)
[-8]   ordered?(node3_var!1)
  |-------
[1]   node1_var!1 <= x!1
[2]   every((LAMBDA (y: T): node1_var!1 <= y), insert(x!1, node3_var!1))
[3]   x!1 <= node1_var!1
\end{session*}
Expanding the definition of \texttt{dichotomous?}\ yields the needed
linearity property of the ordering relation.
\begin{session*}
Rule? \fbox{(expand "dichotomous?")}
Expanding the definition of dichotomous?,
this simplifies to: 
ordered?_insert.2.2 :  
[-1]   partial_order?(obt.<=)
\{-2\}   (FORALL (x: T), (y: T): (obt.<=(x, y) OR obt.<=(y, x)))
[-3]   ordered?(insert(x!1, node3_var!1))
[-4]   (FORALL (x: T):
         ordered?(node2_var!1) IMPLIES ordered?(insert(x, node2_var!1)))
[-5]   every((LAMBDA (y: T): y <= node1_var!1), node2_var!1)
[-6]   every((LAMBDA (y: T): node1_var!1 <= y), node3_var!1)
[-7]   ordered?(node2_var!1)
[-8]   ordered?(node3_var!1)
  |-------
[1]   node1_var!1 <= x!1
[2]   every((LAMBDA (y: T): node1_var!1 <= y), insert(x!1, node3_var!1))
[3]   x!1 <= node1_var!1
\end{session*}
When this linearity property is heuristically instantiated, we get a
tautologous subgoal that is polished off with \texttt{prop}, thus completing
the proof.
\begin{session*}
Rule? \fbox{(inst?)}
Found substitution:
y gets x!1,
x gets node1_var!1,
Instantiating quantified variables,
this simplifies to: 
ordered?_insert.2.2 :  
[-1]   partial_order?(obt.<=)
\{-2\}   (obt.<=(node1_var!1, x!1) OR obt.<=(x!1, node1_var!1))
[-3]   ordered?(insert(x!1, node3_var!1))
[-4]   (FORALL (x: T):
         ordered?(node2_var!1) IMPLIES ordered?(insert(x, node2_var!1)))
[-5]   every((LAMBDA (y: T): y <= node1_var!1), node2_var!1)
[-6]   every((LAMBDA (y: T): node1_var!1 <= y), node3_var!1)
[-7]   ordered?(node2_var!1)
[-8]   ordered?(node3_var!1)
  |-------
[1]   node1_var!1 <= x!1
[2]   every((LAMBDA (y: T): node1_var!1 <= y), insert(x!1, node3_var!1))
[3]   x!1 <= node1_var!1
Rule? \fbox{(prop)}
Applying propositional simplification,
This completes the proof of ordered?_insert.2.2.
This completes the proof of ordered?_insert.2.
Q.E.D.
Run time  = 12.32 secs.
Real time = 1916.88 secs.
\end{session*}

The above exercise illustrates several aspects of PVS proofs of
theorems involving abstract datatypes.  The \texttt{induct} strategy
automatically employs the datatype induction scheme.  Most of the
datatype axioms need never be explicitly invoked in a proof ---
the above proof does not mention any datatype axioms.
%% \comment{The
%% automatically generated recursion combinators such as \texttt{reduce\_nat}
%% can be annoying to use in place of explicitly 
%% recursive definitions since they require an extra definition expansion
%% and obscure the recursive structure of the concept being defined.
%% These combinators are useful for defining size functions.
%% The \texttt{CASES} and \texttt{IF-THEN-ELSE} forms are more convenient for
%% recursive definitions, and the PVS proof checker is quite powerful at
%% simplifying expressions constructed using these constructs.}

 More general lessons about PVS are also illustrated by
the above exercise.  Primary among these are the use of \texttt{undo} to
backtrack in a proof, the use of \texttt{expand} and \texttt{rewrite} to
rewrite using definitions and rewrite rules, \texttt{assert} to simplify
using the decision procedures and the assertions in the sequent, 
and \texttt{inst?}\ to heuristically
instantiate a suitably quantified variable using matching.   

We now examine a more automated proof of the same theorem.

\section{A Semi-automated Proof}

We can now retry the proof of the theorem \texttt{ordered?\_insert}
using a more high-level strategy defined in PVS\@.  This strategy is
called \texttt{induct-and-simplify}.  It applies the \texttt{induct} strategy
and then tries to complete the proof by repeatedly skolemizing and
instantiating quantifiers, and applying the decision procedures,
rewrite rules, and propositional simplification.  We employ 
as rewrite rules, the lemma \texttt{ordered?\_insert\_step} and any 
definitions used directly or indirectly in the statement of the theorem\@.
The script shown below has been automatically generated from the \texttt{induct-and-simplify} command up to the subgoal in~\pvsref{subgoal1}\@.  
The first segment of the proof shows the setting up of the rewrite
rules mentioned in the \texttt{induct-and-simplify} command.
\begin{session*}
 ordered?_insert :  
  |-------
\{1\}   (FORALL (A: binary_tree[T], x: T):
         ordered?(A) IMPLIES ordered?(insert(x, A)))
Rule? \fbox{(induct-and-simplify "A" :rewrites "ordered?_insert_step")}

\end{session*}
The internal steps of the strategy are not displayed but any applications
of rewrite rules are indicated in the proof commentary.  This rewriting
commentary 
can be turned off using the proof command \texttt{(rewrite-msg-off)} or
controlled using the PVS Emacs command \texttt{M-x set-rewrite-depth}\@.
The rewriting in the base case is shown below in~\pvsref{basecase}.
\begin{session*}\label{basecase}
ordered? rewrites ordered?(leaf)
  to TRUE
insert rewrites insert(x!1, leaf)
  to node(x!1, leaf, leaf)
every rewrites every((LAMBDA (y: T): y <= x!1), leaf)
  to TRUE
every rewrites every((LAMBDA (y: T): x!1 <= y), leaf)
  to TRUE
ordered? rewrites ordered?(node(x!1, leaf, leaf))
  to TRUE
\end{session*}
The rewriting steps occurring in the induction case are shown
in~\pvsref{indcase} 
\begin{session*}\label{indcase}
ordered? rewrites ordered?(node(node1_var!1, node2_var!1, node3_var!1))
  to (every((LAMBDA (y: T): y <= node1_var!1), node2_var!1)
           AND every((LAMBDA (y: T): node1_var!1 <= y), node3_var!1)
             AND ordered?(node2_var!1) AND ordered?(node3_var!1))
insert rewrites insert(x!1, node(node1_var!1, node2_var!1, node3_var!1))
  to (IF x!1 <= node1_var!1
         THEN node(node1_var!1, insert(x!1, node2_var!1), node3_var!1)
       ELSE node(node1_var!1, node2_var!1, insert(x!1, node3_var!1))
       ENDIF)
ordered? rewrites 
  ordered?(node(node1_var!1, insert(x!1, node2_var!1), node3_var!1))
  to (every((LAMBDA (y: T): y <= node1_var!1), insert(x!1, node2_var!1))
           AND every((LAMBDA (y: T): node1_var!1 <= y), node3_var!1)
             AND ordered?(insert(x!1, node2_var!1)) AND ordered?(node3_var!1))
ordered? rewrites 
  ordered?(node(node1_var!1, node2_var!1, insert(x!1, node3_var!1)))
  to (every((LAMBDA (y: T): y <= node1_var!1), node2_var!1)
           AND
         every((LAMBDA (y: T): node1_var!1 <= y), insert(x!1, node3_var!1))
             AND ordered?(node2_var!1) AND ordered?(insert(x!1, node3_var!1)))
ordered?_insert_step rewrites 
  every((LAMBDA (y: T): y <= node1_var!1), insert(x!1, node2_var!1))
  to TRUE
\end{session*}
One subgoal results from the \texttt{induct-and-simplify} command
as shown in~\pvsref{subgoal1}.  This subgoal is nearly the same as
subgoal \texttt{ordered?\_insert.2.2} in~\pvsref{subgoal2.2} from the
previous proof attempt.  This means that the \texttt{induct-and-simplify}
strategy completed the base case and most of the induction branch of the
proof automatically.  The subgoal in~\ref{subgoal1} corresponds to the
case of insertion into the right subtree.  The strategy failed to complete
this branch of the proof because it was unable to apply the rewrite rule
\texttt{ordered?\_insert\_step} automatically.  This application failed
because one of the conditions of the rewrite rule, \texttt{node1\_var!1 <=
x!1}, could not be discharged.  This condition follows from formula number
\texttt{1} in~\ref{subgoal1} and the linearity of the \texttt{<=} relation.
The latter constraint is, however, buried in the type constraint \texttt{(total\_order?)} of \texttt{<=}\@.  This information has to be made explicit
before the proof can be successfully completed.  
\begin{session*}\label{subgoal1}
By induction on A, and by repeatedly rewriting and simplifying,
this simplifies to: 
ordered?_insert :  
\{-1\}   ordered?(insert(node1_var!1, node2_var!1))
\{-2\}   ordered?(insert(x!1, node3_var!1))
\{-3\}   every((LAMBDA (y: T): y <= node1_var!1), node2_var!1)
\{-4\}   every((LAMBDA (y: T): node1_var!1 <= y), node3_var!1)
\{-5\}   ordered?(node2_var!1)
\{-6\}   ordered?(node3_var!1)
  |-------
\{1\}   x!1 <= node1_var!1
\{2\}   every((LAMBDA (y: T): node1_var!1 <= y), insert(x!1, node3_var!1))
\end{session*}

The rest of proof can be completed interactively as in the previous proof
attempt but we attempt 
a slightly different sequence of steps this time.  The first step is identical
to that in~\pvsref{2.2.proof.1} where the \texttt{ordered?\_insert\_step}
lemma is manually invoked as a rewrite rule using the \texttt{rewrite}
strategy.  
\begin{session*}
Rule? \fbox{(rewrite "ordered?_insert_step")}
Found matching substitution:
A gets node3_var!1,
x gets x!1,
pp gets (LAMBDA (y: T): node1_var!1 <= y),
Rewriting using ordered?_insert_step,
this simplifies to: 
ordered?_insert :  
[-1]   ordered?(insert(node1_var!1, node2_var!1))
[-2]   ordered?(insert(x!1, node3_var!1))
[-3]   every((LAMBDA (y: T): y <= node1_var!1), node2_var!1)
[-4]   every((LAMBDA (y: T): node1_var!1 <= y), node3_var!1)
[-5]   ordered?(node2_var!1)
[-6]   ordered?(node3_var!1)
  |-------
\{1\}   node1_var!1 <= x!1
[2]   x!1 <= node1_var!1
[3]   every((LAMBDA (y: T): node1_var!1 <= y), insert(x!1, node3_var!1))
\end{session*}
The next step is also identical to that of~\pvsref{2.2.proof.2}
where the type constraints for  the \texttt{<=} operator are explicitly
introduced into the sequent.
\begin{session*}
Rule? \fbox{(typepred "obt.<=")}
Adding type constraints for  obt.<=,
this simplifies to: 
ordered?_insert :  
\{-1\}   total_order?[T](obt.<=)
[-2]   ordered?(insert(node1_var!1, node2_var!1))
[-3]   ordered?(insert(x!1, node3_var!1))
[-4]   every((LAMBDA (y: T): y <= node1_var!1), node2_var!1)
[-5]   every((LAMBDA (y: T): node1_var!1 <= y), node3_var!1)
[-6]   ordered?(node2_var!1)
[-7]   ordered?(node3_var!1)
  |-------
[1]   node1_var!1 <= x!1
[2]   x!1 <= node1_var!1
[3]   every((LAMBDA (y: T): node1_var!1 <= y), insert(x!1, node3_var!1))
\end{session*}
The main difference from the previous proof attempt is that we now invoke
a somewhat powerful variant of the all-purpose \texttt{grind} strategy where
the \texttt{:if-match} flag is set to \texttt{all} indicating that all matching
instances of any quantified formulas are to be used.  If we do not supply
this option, the heuristic instantiator picks the wrong instance since
the type constraints for \texttt{ <=} themselves provide matching instances
for the one relevant type constraint, namely, \texttt{dichotomous?(obt.<=)}.
\begin{session*}
Rule? \fbox{(grind :if-match all)}
reflexive? rewrites reflexive?(obt.<=)
  to FORALL (x: T): obt.<=(x, x)
transitive? rewrites transitive?(obt.<=)
  to FORALL (x: T), (y: T), (z: T): obt.<=(x, y) & obt.<=(y, z) => obt.<=(x, z)
preorder? rewrites preorder?(obt.<=)
  to FORALL (x: T): obt.<=(x, x)
          & FORALL (x: T), (y: T), (z: T):
          obt.<=(x, y) & obt.<=(y, z) => obt.<=(x, z)
antisymmetric? rewrites antisymmetric?(obt.<=)
  to FORALL (x: T), (y: T): obt.<=(x, y) & obt.<=(y, x) => x = y
partial_order? rewrites partial_order?(obt.<=)
  to (FORALL (x: T): obt.<=(x, x)
           & FORALL (x: T), (y: T), (z: T):
           obt.<=(x, y) & obt.<=(y, z) => obt.<=(x, z))
          & FORALL (x: T), (y: T): obt.<=(x, y) & obt.<=(y, x) => x = y
dichotomous? rewrites dichotomous?(obt.<=)
  to (FORALL (x: T), (y: T): (obt.<=(x, y) OR obt.<=(y, x)))
total_order? rewrites total_order?[T](obt.<=)
  to ((FORALL (x: T): obt.<=(x, x)
            & FORALL (x: T), (y: T), (z: T):
            obt.<=(x, y) & obt.<=(y, z) => obt.<=(x, z))
           & FORALL (x: T), (y: T): obt.<=(x, y) & obt.<=(y, x) => x = y)
          & (FORALL (x: T), (y: T): (obt.<=(x, y) OR obt.<=(y, x)))
Trying repeated skolemization, instantiation, and if-lifting,
Q.E.D.
Run time  = 48.86 secs.
Real time = 230.49 secs.
\end{session*}

The above semi-automated proof attempt illustrates the power that is
gained from combining high-level strategies (e.g., \texttt{induct-and-simplify} and \texttt{grind}) to handle the easy portions of
a proof with low-level manual interaction to carry out the more delicate
steps.  Note that the inner workings of these strategies which are hidden
in the above proof can be observed by invoking them with a \texttt{\$} suffix
as in \texttt{induct-and-simplify\$} and \texttt{grind\$}\@.  

The proofs of the lemmas \texttt{ordered?\_insert\_step} in~\pvsref{oistep}
and \texttt{search\_insert} shown in~\pvsref{search} below  
can be completed automatically by the single command:
\begin{alltt}
  \fbox{(induct-and-simplify "A")}
\end{alltt}

\begin{session*}\label{search}
search(x, A): RECURSIVE bool =
  (CASES A OF
    leaf: FALSE,
    node(y, B, C): (IF x = y THEN TRUE
                         ELSIF x<=y THEN search(x, B)
                           ELSE search(x, C)
                    ENDIF)
   ENDCASES)
  MEASURE size(A)

search_insert: THEOREM search(y, insert(x, A)) = (x = y OR search(y, A))
\end{session*}

\section{Proof Status}

To conclude the development of binary trees and ordered binary trees,  we
can apply the PVS~Emacs command \texttt{M-x prt} to recheck all 
the proofs and print out the proof status.\footnote{The command
\texttt{M-x status-proof-theory} or \texttt{M-x spt} can be used to get the proof
status without rechecking the proofs.}  The output of this command is
shown below.  It indicates that not only have all the theorems been
proved but so have any non-axioms (lemmas, TCCs, etc.)  used
in any of these proofs.   
\begin{session*}
Proof summary for theory obt
    ordered?_TCC1..........................................proved - complete
    ordered?_TCC2..........................................proved - complete
    insert_TCC1............................................proved - complete
    insert_TCC2............................................proved - complete
    ordered?_insert_step...................................proved - complete
    ordered?_insert........................................proved - complete
    search_insert..........................................proved - complete
    Theory totals: 7 formulas, 7 attempted, 7 succeeded.
\end{session*}


\chapter{Built-in Datatype Simplifications}~\label{built-in}

As indicated at the outset, the primary advantage of using abstract
datatypes in PVS is that a lot of knowledge about such datatypes and
their operations is built into the system.  To illustrate
the sort of datatype simplifications that are built into PVS,
consider the theory \texttt{binary\_props} shown below.
\begin{session*}
binary_props[T : TYPE] : THEORY

  BEGIN
  IMPORTING binary_tree_adt[T]
  A, B, C, D: VAR binary_tree[T]
  x, y, z: VAR T

  leaf_leaf : LEMMA leaf?(leaf)
  node_node : LEMMA node?(node(x, B, C))

  leaf_leaf1: LEMMA A = leaf IMPLIES leaf?(A)

  node_node1: LEMMA A = node(x, B, C) IMPLIES node?(A)

  val_node: LEMMA val(node(x, B, C)) = x

  leaf_node: LEMMA NOT (leaf?(A) AND node?(A))

  node_leaf: LEMMA leaf?(A) OR node?(A)

  leaf_ext: LEMMA (FORALL (A, B: (leaf?)): A = B)
         
  node_ext:  LEMMA (FORALL (A : (node?)) : 
                    node(val(A), left(A), right(A)) = A)

  END binary_props
\end{session*}

All the lemmas excluding the  last one, \texttt{node\_ext},  are provable
by the command \texttt{(then (skosimp)(assert))}.  This means that the
\texttt{assert} rule builds in several simplifications.  In the case of
\texttt{leaf\_leaf} and \texttt{node\_node}, \texttt{assert} can reduce the
application of a recognizer to a constructor expression to either \texttt{TRUE} or \texttt{FALSE}.  In the case of \texttt{leaf\_leaf1} and \texttt{node\_node1}, it even can do this simplification across an equality.
The reason for this simplification is that subtype information
is asserted to the decision procedures so that when \texttt{A = node(x, B,
C)} is asserted to the decision procedures, so is \texttt{node?(node(x, B, C))},
and \texttt{node?(A)} is deduced by congruence closure in the decision
procedures.  
The simplifications in \texttt{leaf\_leaf} and \texttt{node\_node}, but not
\texttt{leaf\_leaf1} and \texttt{node\_node1}, can also be carried out by the
PVS beta-reduction rule \texttt{beta} since this rule does not make use of
equality information\@.

The lemma \texttt{val\_node} illustrates that the application of an
accessor to a constructor expression yields the appropriate field of
the constructor expression.  This simplification can also be done by
the \texttt{beta} rule.

The simplification implicit in \texttt{leaf\_node}
is more subtle and captures the exclusivity property of abstract
datatypes.  Here, from an antecedent formula \texttt{leaf?(A)},  \texttt{assert} is able to simplify the expression \texttt{node?(A)} to \texttt{FALSE}
since no datatype expression can satisfy two recognizers.
The simplification implicit in \texttt{node\_leaf} captures the
inclusivity property of abstract datatypes.  Here, \texttt{assert} is
able to determine that a recognizer  holds of an expression  by
demonstrating that all the other recognizers are false on the expression.
In general, when confronted with the application of a recognizer $r$ to a
datatype expression $e$, the simplifier evaluates the truth value of
each recognizer of that datatype when applied to the given expression
using the decision procedures.  If $r(e)$ is determined to be \texttt{TRUE}
by the decision procedures, then $r(e)$ is obviously simplified to \texttt{TRUE} by the simplifier\@.  If for some other recognizer $r'$,
$r'(e)$ is determined to be \texttt{TRUE} by the decision procedures, 
then $r(e)$ is simplified to \texttt{FALSE}.  If for all recognizers $r'$
distinct from $r$, $r'(e)$ is determined to be \texttt{FALSE}, then
$r(e)$ is simplified to \texttt{TRUE}\@.  

The lemma \texttt{leaf\_ext} essentially illustrates that for
constructors such as \texttt{leaf} that have no accessors, there is no
distinction between the forms \texttt{leaf?(A)} and \texttt{A = leaf}\@.
It also illustrates how the subtype information is used implicitly to
simplify \texttt{A = B} to \texttt{TRUE}\@.

The lemma \texttt{node\_ext} is the only one that cannot be proved by the
command \texttt{(then (skosimp)(assert))}\@.  Here, this command simplifies the
goal sequent to a single subgoal that is then proved by means of the \texttt{(apply-extensionality)} command.  This illustrates that the
extensionality axiom for datatypes is built into the primitive PVS
rule \texttt{extensionality} and is also employed by the strategies
\texttt{replace-extensionality} and \texttt{apply-extensionality}.




\chapter{Some Proof Strategies}\label{proof-strategies}
We briefly explain the definitions of the proof strategies
\texttt{induct-and-simplify} and \texttt{grind} that were used in 
Section~\ref{OBT-proofs}\@.  The PVS manuals~\cite{PVS:manuals} provides
more details.   These strategies are quite useful for proofs of
datatype-related theorems.  
The \texttt{induct-and-simplify} strategy takes an argument list of the form:
\begin{alltt}
  (var &optional (fnum 1) name (defs t) (if-match best)
                 theories rewrites exclude)
\end{alltt}
where
\begin{itemize}
\item \texttt{var} is the induction variable and is the only required
argument
\item \texttt{fnum} is the formula number of the induction formula where the
induction variable is universally quantified; it defaults to \texttt{1}
\item \texttt{name}  names the induction scheme to be employed
\item \texttt{defs} indicates which definitions of constants used in the
current goal are to be installed as rewrite rules; it defaults to \texttt{t}
indicating that all relevant definitions must be installed
\item \texttt{if-match} instructs the heuristic instantiator to use none,
one, all, or the best matching instantiation for a quantified formula;
it defaults to \texttt{best}
\item \texttt{theories} is the list of theories to be installed as rewrite rules
\item \texttt{rewrites} is the list of formulas or definitions that are to be
installed as rewrite rules, and
\item \texttt{exclude} is a list of formulas or definitions that should be
removed from the rewrite rule base
\end{itemize}

The body of the definition of the strategy has the form:
\begin{smalltt}
  (then
   (install-rewrites$ :defs defs :theories theories :rewrites rewrites
    :exclude exclude)
   (try (induct var fnum name)
    (then (skosimp*) (assert) (repeat (lift-if))
     (repeat*
      (then (assert) (bddsimp) (skosimp*)
       (if if-match (inst? :if-match if-match) (skip)) (lift-if))))
    (skip)))
\end{smalltt}

The \texttt{induct-and-simplify} strategy first installs the rewrites
using \texttt{install-rewrites\$} on \texttt{defs}, \texttt{theories}, \texttt{rewrites}, and \texttt{exclude}\@.  It then introduces the appropriate
instance of the induction scheme using \texttt{induct}\@.  Then the strategy
carries out one round of skolemization (introduction of new constants for
outermost universally bound variables) using \texttt{skosimp*}, rewriting
and simplification using \texttt{assert}, and repeated lifting of conditionals
to the top level using \texttt{lift-if}.  Following this, there are repeated
rounds of rewriting/simplification, propositional simplification,
skolemization, heuristic instantiation, and if-lifting until each
resulting subgoal has stabilized.

The \texttt{grind} strategy is similar.  It takes the following argument
list:
\begin{alltt}
  (&optional (defs !) theories rewrites exclude (if-match t) (updates? t))
\end{alltt}
where the only new argument from \texttt{induct-and-simplify} is \texttt{updates?}\ which when set to \texttt{NIL} blocks the if-lifting of
update applications of the form \texttt{(A~WITH~[(i)~:=~b])(j)} to
\texttt{(IF~i~=~j~THEN~b~ELSE~A(j)~ENDIF)}\@.

The body of the definition of \texttt{grind} is:
\begin{smalltt}
  (then
   (install-rewrites$ :defs defs :theories theories :rewrites rewrites
    :exclude exclude)
   (then (bddsimp) (assert)) (replace*)
   (reduce$ :if-match if-match :updates? updates?))
\end{smalltt}

Here the rewrite rules are installed using \texttt{install-rewrites\$},
and followed by propositional simplification, rewriting and
simplification,  equality replacement, followed by repeated
applications of these steps along with heuristic instantiation and
if-lifting. 

It should be clear from the above definition that it is fairly
straightforward to write powerful proof strategies using the
constructs provided by the PVS proof checker.  

%% \comment{
%% \chapter{A Further Exercise: Searching an Ordered Binary Tree}

%% Following the outline of the definition of \texttt{insert}, we can use
%% recursion to define an operation that searches for an element in a
%% given ordered binary tree.  This operation \texttt{search} returns a
%% Boolean value, and  it is defined as shown below.
%% \begin{session*}
%% search(x, A): RECURSIVE bool =
%%   (CASES A OF
%%     leaf: FALSE,
%%     node(y, B, C): (IF x = y THEN TRUE
%%                          ELSIF x<=y THEN search(x, B)
%%                            ELSE search(x, C)
%%                     ENDIF)
%%    ENDCASES)
%%   MEASURE (LAMBDA x, A: size(A))
%% \end{session*}
%% Typechecking this definition generates TCCs that are subsumed by those
%% generated by  \texttt{insert}\@.  We can  state the desired
%% relationship between \texttt{search} and \texttt{insert} as below.
%% \begin{session*}
%% search_insert: THEOREM search(y, insert(x, A)) = (x = y OR search(y, A))
%% \end{session*}

%% We can now try to prove this property based on what we have learned so
%% far.  As a first step, we apply the rule
%% \begin{alltt}
%%   (induct-and-rewrite "A" 1 "search" "insert")
%% \end{alltt}

%% This succeeds in proving a number of cases in the proof but eventually
%% stops with the two unproved cases:
%% \begin{session*}
%% search_insert.2.1.3 :   

%% \{-1\}   total_order?[T](obt.<=)
%% [-2]   x!1 <= node1_var!1
%% [-3]   search(y!1, insert(x!1, node2_var!1))
%%         = (x!1 = y!1 OR search(y!1, node2_var!1))
%% [-4]   search(y!1, insert(x!1, node3_var!1))
%%         = (x!1 = y!1 OR search(y!1, node3_var!1))
%%   |-------
%% [1]   y!1 <= node1_var!1
%% [2]   search(y!1, node3_var!1) = (x!1 = y!1 OR search(y!1, node3_var!1))
%% [3]   y!1 = node1_var!1
%% \end{session*}
%% and
%% \begin{session*}
%% search_insert.2.2.2 :   

%% [-1]   y!1 <= node1_var!1
%% [-2]   search(y!1, insert(x!1, node2_var!1))
%%         = (x!1 = y!1 OR search(y!1, node2_var!1))
%% [-3]   search(y!1, insert(x!1, node3_var!1))
%%         = (x!1 = y!1 OR search(y!1, node3_var!1))
%%   |-------
%% [1]   search(y!1, node2_var!1) = (x!1 = y!1 OR search(y!1, node2_var!1))
%% [2]   y!1 = node1_var!1
%% [3]   x!1 <= node1_var!1
%% \end{session*}

%% These two goals are quite similar to one another.  If we focus on the
%% first of these, the goal is to establish the formula \texttt{1} assuming
%% the formula \texttt{-1} and the negations of \texttt{2} and \texttt{3}, since
%% it does not look like the remnants of the induction hypotheses \texttt{%% -2} and \texttt{-3} are relevant here.  On first look, it might appear
%% that this sequent depends on the properties of the ordering \texttt{%% relation <=} for its proof.  However, if we focus on formula \texttt{1},
%% we see that it amounts to proving that \texttt{(x!1 = y!1) = FALSE}\@.
%% This is quite easily proved since if we assume \texttt{x!1 = y!1}, then
%% this contradicts \texttt{-1} and the negation of \texttt{3}\@.  This
%% reasoning is captured by the proof shown in the next two boxes.
%% \begin{session*}
%% Rule? \fbox{(case "x!1 = y!1")}
%% Case splitting on 
%%    x!1 = y!1, 
%% this yields  2 subgoals: 
%% search_insert.2.1.3.1 :   

%% \{-1\}   x!1 = y!1
%% [-2]   x!1 <= node1_var!1
%% [-3]   search(y!1, insert(x!1, node2_var!1))
%%         = (x!1 = y!1 OR search(y!1, node2_var!1))
%% [-4]   search(y!1, insert(x!1, node3_var!1))
%%         = (x!1 = y!1 OR search(y!1, node3_var!1))
%%   |-------
%% [1]   y!1 <= node1_var!1
%% [2]   search(y!1, node3_var!1) = (x!1 = y!1 OR search(y!1, node3_var!1))
%% [3]   y!1 = node1_var!1

%% Rule? \fbox{(assert)}
%% Invoking decision procedures,

%% This completes the proof of search_insert.2.1.3.1.
%% \end{session*}
%% \begin{session*}
%% search_insert.2.1.3.2 :   

%% [-1]   x!1 <= node1_var!1
%% [-2]   search(y!1, insert(x!1, node2_var!1))
%%         = (x!1 = y!1 OR search(y!1, node2_var!1))
%% [-3]   search(y!1, insert(x!1, node3_var!1))
%%         = (x!1 = y!1 OR search(y!1, node3_var!1))
%%   |-------
%% \{1\}   x!1 = y!1
%% [2]   y!1 <= node1_var!1
%% [3]   search(y!1, node3_var!1) = (x!1 = y!1 OR search(y!1, node3_var!1))
%% [4]   y!1 = node1_var!1

%% Rule? \fbox{(assert)}
%% Invoking decision procedures,
%% this simplifies to: 
%% search_insert.2.1.3.2 :   

%% [-1]   x!1 <= node1_var!1
%% \{-2\}   search(y!1, insert(x!1, node2_var!1)) = search(y!1, node2_var!1)
%% \{-3\}   search(y!1, insert(x!1, node3_var!1)) = search(y!1, node3_var!1)
%%   |-------
%% [1]   x!1 = y!1
%% [2]   y!1 <= node1_var!1
%% \{3\}   TRUE
%% [4]   y!1 = node1_var!1

%% which is trivially true.

%% This completes the proof of search_insert.2.1.3.2.


%% This completes the proof of search_insert.2.1.3.


%% This completes the proof of search_insert.2.1.
%% \end{session*}

%% The identical argument works on the sequent \texttt{search\_insert.2.2.2}
%% as well.  This example clearly illustrates the value of forcing case
%% splits using the \texttt{case} rule.  Such case splits can often be the
%% most crucial creative steps in a proof.  In this example, the case
%% split is not all that creative a step.  The relevant subgoal can also
%% be proved by  converting the Boolean equality on the formula
%% \begin{alltt}
%%     search(y!1, node3_var!1) = (x!1 = y!1 OR search(y!1, node3_var!1))
%% \end{alltt}
%% into an if-and-only-if form with the rule \texttt{(iff +)}\@.\footnote{The
%% \texttt{+} is there to leave the antecedent Boolean equalities untouched.}
%% The command \texttt{(ground)} then completes the proof.
%% }



\chapter{Limitations of the PVS Abstract Datatype
Mechanism}\label{limitations} 

The abstract datatype mechanism of PVS is intended to capture a fairly
large class of datatypes whose axioms can be easily and systematically
generated.  This class contains all the freely generated term algebras
over an order-sorted signature which includes the various stack and
tree-like data structures.  It excludes such important datatypes as
integers (which are built into PVS), bags, sets, and queues.  It also
excludes various lazy data structures such as lazy lists or streams.
These latter structures can be introduced by implementing
a similar mechanism for introducing co-datatypes as for datatypes.

 The \texttt{DATATYPE} mechanism is a primitive construct of PVS and
is not merely a definitional extension of PVS.  It therefore has
the disadvantage that it is not possible to prove general
theorems about all recursive datatypes in the way that one can
about all inductive definitions given as least fixed points.
For example, Bird's fusion theorem~\cite{Bird:MPC95} cannot be uniformly
proved for all recursive datatypes and has to be proved for each datatype
individually~\cite{Shankar:SCP96}. 


%% \comment{\begin{enumerate}
%%  \item    The recursive occurrences of the datatype can  occur
%% only as the supertype of a subtype, or within a list or sequence
%% construction within the type of an accessor.  This can be
%% significantly generalized by allowing these recursive occurrences
%% within any covariant position in a type constructor.   For example,
%% the abstract datatype corresponding to the abstract syntax of a
%% language may include the possibility that expressions of this language
%% contain binary trees of expressions.   
%%   \item
%% Another  limitation with the PVS \texttt{DATATYPE}
%% mechanism as described in the current document is that 
%% though the natural numbers are a subtype of the integers in
%% PVS,  the type of binary trees over
%% the natural numbers is not a subtype of the type of binary trees over the
%% integers.  The datatype mechanism in PVS~2.1 has been modified to 
%% lift such subtyping over positive parameters to the
%% corresponding abstract datatypes.  This is done by essentially viewing
%% the type \texttt{binary\_tree[nat]} as a shorthand for a subtype of
%% \texttt{binary\_tree[number]} with the constraint
%% \texttt{every(p)}, where \texttt{p} is the contraint on members of the
%% subtype \texttt{nat} relative to its maximal supertype \texttt{number}\@.
%% }

  

\chapter{Related Work}\label{related}

There are a number of algebraic specification languages such as
Larch~\cite{Larch93}, OBJ~\cite{popl85}, and ACT-ONE~\cite{Ehrig-Mahr}
that can be used to specify 
abstract datatypes but these specifications are manually axiomatized
and  not automatically generated from a succinct description as is the
case with the PVS \texttt{DATATYPE} construct.  The axioms are used as
rewrite rules so that there is no built-in automation of the simplification
of datatype expressions.   


The 
programming language ML~\cite{ML-report} has a similar recursive datatype
mechanism. 
Unlike the PVS mechanism, the ML construct allows arbitrary forms of
recursion.  As noted earlier, such recursive type definitions do not
always have a proper set-theoretic semantics.  Gunter~\cite{Gunter93}
explains how certain recursive datatypes that are admissible in ML
can lead to unsoundnesses if admitted into a higher-order logic.  

The HOL system has a mechanism for defining abstract
datatypes~\cite{Melham89} that is somewhat more restrictive than that of
PVS: there are more constraints on recursion and HOL lacks the useful
notion of subtyping that is available in PVS\@.  However, the HOL
construct is definitional in that a recursively specified datatype is
defined in terms of the primitive type constructors available in HOL.  In
particular, any newly defined recursive datatype is shown to be
interpretable as a subset of some existing datatype based on finitely
branching trees.  The axioms generated from the datatype declaration are
shown to be sound with respect to this interpretation.  Isabelle/ZF and
Isabelle/HOL both have a similar but more general facility for defining
datatypes and co-datatypes~\cite{Paulson97:JLC}.  The Isabelle datatype
mechanism also accomodates infinitely branching trees.  The Coq system has
a facility for defining recursive and co-recursive datatypes which,
like PVS and unlike HOL and Isabelle, is  
a primitive construct of the Coq logic~\cite{Coq:RecTypes}. 

The shell principle used in the Boyer-Moore theorem
prover~\cite{Boyer-Moore79,boyer-moore88} is quite similar to the PVS \texttt{DATATYPE} mechanism.  It permits recursive datatypes to be specified by
means of constructors, accessors, and recognizers.  Like PVS, the axioms
corresponding to a shell datatype are built into the inference mechanisms
of the theorem prover.  The shell principle, however, has many serious
limitations.  It is complicated by the lack of types or subtypes in the
Boyer-Moore logic.  The shell principle only allows one constructor and a
bottom object thus ruling out a great many useful datatypes where multiple
constructors are required.


\chapter{Conclusions}

We have described the \texttt{DATATYPE} mechanism of PVS and demonstrated
its use in proof construction.  This mechanism captures a large class
of commonly used type definitions within a succinct notation.  A
number of facts about these automatically generated abstract datatypes
are built into the inference mechanisms of PVS so that it is possible
to obtain a significant degree of automation when proving theorems
involving datatypes.  The high level of automation in the low-level
inference mechanisms in PVS makes it  easy to define
powerful and flexible high-level proof strategies.  


\subsubsection{Acknowledgements. } The design and implementation of PVS
was directed by John Rushby of the SRI
Computer Science Laboratory.    He, along with Rick Butler of NASA
and Mandayam Srivas, suggested several improvements to this document.
Donald Syme of Cambridge University carefully proofread the document
and gave numerous helpful suggestions.  We are also grateful to
Ulrich Hensel of TU Dresden for his as yet unheeded suggestion
to incorporate corecursive datatypes.  


\bibliographystyle{alpha}
\bibliography{../pvs}
\end{document}
