
\section{Auto-rewrite Declarations}
\label{auto-rewrite-decls}\index{auto-rewrites|(}

One of the problems with writing useful theories or libraries is that
there is no easy way to convey how the theory is to be used, other than in
comments or documentation.  In particular, the specifier of a theory
usually knows which lemmas should always be used as rewrites, and which
should never appear as rewrites.  Auto-rewrite declarations allow for both
forms of control.  Those that should always be used as rewrites are
declared with the \texttt{AUTO\_REWRITE+} or \texttt{AUTO\_REWRITE}
keyword, and those that should not are declared with
\texttt{AUTO\_REWRITE-}.  

When a proof is initiated for a given formula, all the
\texttt{AUTO\_REWRITE+} names in the current context that haven't
subsequently been removed by \texttt{AUTO\_REWRITE-} declarations are
collected and added to the initial proof state.  The
\texttt{AUTO\_REWRITE-} declaration, in addition to removing
\texttt{AUTO\_REWRITE+} names, also affects the
\texttt{auto-rewrite-theory}, \texttt{auto-rewrite-theories},
\texttt{auto-rewrite-theory-with-importings},
\texttt{simplify-with-rewrites}, \texttt{auto-rewrite-defs},
\texttt{install-rewrites}, \texttt{auto-rewrite-explicit}, \texttt{grind},
\texttt{induct-and-simplify}, \texttt{measure-induct-and-simplify}, and
\texttt{model-check} commands, described in the Prover manual.  These
commands collect all definitions and formulas except those that appear in
\texttt{AUTO\_REWRITE-} delcarations.  Thus suppose a theory \texttt{T}
contains the lemmas \texttt{lem1}, \texttt{lem2}, and \texttt{lem3} and
the declarations
\begin{alltt}
  AUTO_REWRITE+ lem1
  AUTO_REWRITE- lem3
\end{alltt}
Then in proving a formula of a theory that imports \texttt{T},
\texttt{lem1} is initially an auto-rewrite, and the command
\texttt{(auto-rewrite-theory "T")} will additionally install
\texttt{lem2}.  To auto-rewrite with \texttt{lem3}, simply use
\texttt{(auto-rewrite "lem3")}.  To exclude \texttt{lem1}, use
\texttt{(stop-auto-rewrite "lem1")} or \texttt{(auto-rewrite-theory "T"
:exclude "lem1")}.

The syntax for auto-rewrite declarations is as follows.
\begin{bnf}
\production{AutoRewriteDecl}
{AutoRewriteKeyword\ \ites{RewriteName}{,}}

\production{AutoRewriteKeyword}
{\lit{AUTO\_REWRITE}\ \choice\ \lit{AUTO\_REWRITE+}\ \choice\ \lit{AUTO\_REWRITE-}}

\production{RewriteName}
{Name \opt{\lit{!} \opt{\lit{!}}} \opt{\lit{:}\ \brc{TypeExpr \choice FormulaName}}}
\end{bnf}

The \texttt{autorewrites} theory shows a simple example.
\begin{alltt}
autorewrites: THEORY
BEGIN
 AUTO_REWRITE+ zero_times3
 a, b: real
 f1: FORMULA a * b = 0 AND a /= 0 IMPLIES b = 0
 AUTO_REWRITE- zero_times3
 f2: FORMULA a * b = 0 AND a /= 0 IMPLIES b = 0
END autorewrites
\end{alltt}
Here \texttt{f1} may be proved using only \texttt{assert}, but \texttt{f2}
requires more.

Rewrite names may have suffixes, for example, \texttt{foo!} or
\texttt{foo!!}.  Without the suffix, the rewrite is \emph{lazy}, meaning
that the rewrite will only take place if conditions and TCCs simplify to
true.  A condition in this case is a top-level \texttt{IF} or
\texttt{CASES} expression.  With a single exclamation point the
auto-rewrite is \emph{eager}, in which case the conditions are irrelevant,
though if it is a function definition it must have all arguments supplied.
With two exclamation points it is a \emph{macro} rewrite, and terms will
be rewritten even if not all arguments are provided.  See the prover guide
for more details; the notation is derived from the prover commands
\texttt{auto-rewrite}, \texttt{auto-rewrite!}, and
\texttt{auto-rewrite!!}.

In addition, a rewrite name may be disambiguated by stating that it is a
formula, or giving its type if it is a constant.  Without this any
definition or lemma in the context with the same name will be installed as
an auto-rewrite.

In order to be more uniform, these new forms of name are also available
for the \texttt{auto-rewrite} prover commands.  Thus the command
\begin{alltt}
  (auto-rewrite "A" ("B" "-2") "C" (("1" "D")))
\end{alltt}
may now be given instead as
\begin{alltt}
  (auto-rewrite "A" "B!" "-2!" "C" "1!!" "D!!")
\end{alltt}
The older form is still allowed, but is deprecated, and may not be mixed
with the new form.  Notice that in the auto-rewrite commands formula
numbers may also be used, and these may be followed by exclamation points,
but not by a formula keyword or type.

\index{auto-rewrites|)}