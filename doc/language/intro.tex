% Document Type: LaTeX
% Master File: intro.tex

\chapter{Introduction}

PVS is a \emph{P}rototype \emph{V}erification \emph{S}ystem for the
development and analysis of formal specifications.  The PVS system
primarily consists of a specification language, a parser, a typechecker, a
prover, specification libraries, and various browsing tools.  This
document describes the specification language and is meant to be used as a
reference manual.  The \emph{PVS System Guide}~\cite{PVS:userguide} is to
be consulted for information on how to use the system to develop
specifications and proofs.  The \emph{PVS Prover Guide}~\cite{PVS:prover}
is a reference manual for the commands used to construct proofs.  The web
site \url{http://pvs.csl.sri.com} provides many useful links, including
various tutorials and examples.

In this section, we provide a brief summary of the PVS specification
language, enumerate the key design principles behind the language, and
discuss a simple \texttt{stacks} example.

\section{Summary of the PVS Language}

A PVS specification consists of a collection of \emph{theories}.
Each theory consists of a \emph{signature} for the type names and
constants introduced in the theory, and the axioms, definitions, and
theorems associated with the signature.  For example, a typical
specification for a queue would introduce the \texttt{queue} type and the
operations of \texttt{enq}, \texttt{deq}, and \texttt{front} with their
associated types.  In such a theory, one can either define a
representation for the \texttt{queue} type and its associated operations in
terms of some more primitive types and operations, or merely axiomatize
their properties.  A theory can build on other theories: for example, a
theory for ordered binary trees could build on the theory for
binary trees.  A theory can be \emph{parametric} in certain specified
types and values: as examples, a theory of queues can be parametric in
the maximum queue length, and a theory of ordered binary trees can be
parametric in the element type as well as the ordering relation.  It is
possible to place constraints, called \emph{assumptions}, on the
parameters of a theory so that, for instance, the ordering relation
parameter of an ordered binary tree can be constrained to be a total
ordering.

The PVS specification language is based on simply typed higher-order
logic.  Within a theory, \emph{types} can be defined starting from
\emph{base} types (Booleans, numbers, etc.) using type constructors such
as function, record, and tuple types.  The \emph{terms} of the language
can be constructed using, for example, function application, lambda
abstraction, and record or tuple constructions.

There are a few significant enhancements to the simply typed language
above that lend considerable power and sophistication to PVS.  New
uninterpreted base types may be introduced.  One can define a
\emph{predicate subtype} of a given type as the subset of individuals in a
type satisfying a given predicate: the subtype of nonzero reals is
written as \texttt{$\{$x:real | x /= 0$\}$}.  One benefit of such
subtyping is that when an operation is not defined on all the elements of
a type, the signature can directly reflect this.  For example, the
division operation on reals is given a type where the denominator is
constrained to be nonzero.  Typechecking then ensures that
division is never applied to a zero denominator.  Since the predicate used
in defining a predicate subtype is arbitrary, typechecking is undecidable
and may lead to proof obligations called \emph{type correctness
conditions} (TCCs).  The user is expected to discharge these proof
obligations with the assistance of the PVS prover.  The PVS type system
also features dependent function, record, and tuple type constructions.
There is also a facility for defining a certain class of abstract datatype
(namely well-founded trees) theories automatically.

\section{PVS Language Design Principles}

There are several basic principles that have motivated the design of
PVS which are explicated in this section.

\paragraph{Specification vs. Programming Languages.}
A specification represents requirements or a design whereas a program
text represents an implementation of a design.  A program can be seen as
a specification, but a specification need not be a program.  Typically,
a specification expresses \emph{what} is being
computed whereas a program expresses \emph{how} it is computed.  A
specification can be incomplete and still be meaningful whereas an
incomplete program will typically not be executable.  A specification
need not be executable; it may use high-level constructs, quantifiers
and the like, that need have no computational meaning.  However, there
are a number of aspects of programming languages that a specification
language should include, such as:
\begin{itemize}

\item the usual basic types: booleans, integers, and rational numbers

\item the familiar datatypes of programming languages such as arrays,
records, lists, sequences, and abstract datatypes

\item the higher-order capabilities provided by modern functional
programming languages so that extremely general-purpose operations can
be defined

\item definition by recursion

\item support for dividing large specifications into parameterized
modules

\end{itemize}

It is clearly not enough to say that a specification language shares some
important features of a programming language but need not be executable.
Any useful formal language must have a clearly defined
semantics\footnote{The PVS semantics are presented in a technical
report~\cite{PVS:semantics}.} and must be capable of being manipulated in
ways that are meaningful relative to the semantics.  A programming
language for example can be given a denotational semantics so that the
execution of the program respects its denotational meaning.  The reason
one writes a specification in a formal language is typically to ensure
that it is sensible, to derive some useful consequences from it, and to
demonstrate that one specification implements another.  All of these
activities require the notion of a justification or a proof based on the
specification, a notion that can only be captured meaningfully within the
framework of logic.

\paragraph{Untyped set theory versus higher-order logic}
\index{set theory}\index{higher-order logic}
Which logic should be chosen?  There is a wide variety of choices:
simple propositional logics, which can be classical or intuitionistic,
equational logics, quantificational logics, modal and temporal logics,
set theory, higher-order logic, etc.  Some propositional and modal
logics are appropriate for dealing with finite state machines where one
is primarily interested in efficiently deciding certain finite state
machine properties.  For a general purpose specification language,
however, only a set theory or a higher-order logic would provide the
needed expressiveness.  Higher-order logic requires strict typing to
avoid inconsistencies whereas set theory restricts the rules for forming
sets.  Set theory is inherently untyped, and grafting a typechecker onto
a language based on set theory is likely to be too strict and arbitrary.
Typechecking, however, is an extremely important and easy way of
checking whether a specification makes semantic sense (although 
for an opposing view, the reader is referred to a report by Lamport
and Paulson~\cite{Lamport&Paulson97}).  Higher-order
logic does admit effective typechecking but at the expense of an
inflexible type system.  Recent advances in type theory have made it
possible to design more flexible type systems for higher-order logic
without losing the benefits of typechecking.  We have therefore chosen
to base PVS on higher-order logic.

\paragraph{Total versus partial functions}
\index{function!total}\index{function!partial} In the PVS higher-order
logic, an individual is either a function, a tuple, a record, or the
member of a base type.  Functions are extremely important in higher-order
logic.  They are \emph{first-class} individuals, i.e., variables can range
over functions.  In general, functions can represent either \emph{total}
or \emph{partial} maps.  A total map from domain $A$ to range $B$ maps
each element of $A$ to some element of $B$, whereas a partial map only
maps some of the elements of $A$ to elements of $B$.  Most traditional
logics build in the assumption that functions represent total maps.
Partial functions arise quite naturally in specifications.  For example,
the division operation is undefined on a zero denominator and the
operation of popping a stack is undefined on an empty stack.

Some recent logics, notably those of VDM~\cite{Jones:VDM},
LUTINS~\cite{Farmer:functions}, RAISE~\cite{RAISE-tutorial},
Beeson~\cite{Beeson:book} and Scott~\cite{Scott79}, admit partial
functions.  In these logics, some terms may be \emph{undefined} by not
denoting any individuals.  Some of these logics have mechanisms for
distinguishing defined and undefined terms, while others allow
``undefined'' to propagate from terms to expressions and therefore must
employ multiple truth values.  In all these cases, the ability to
formalize partially defined functions comes at the cost of complicating
the deductive apparatus, even when the specification does not involve any
partial functions.  Though logics that allow partial functions are
extremely interesting, we have chosen to avoid partial functions in PVS.
We have instead employed the notion of a \emph{predicate
subtype}\index{predicate subtype}, a type that consists of those elements
of a given type satisfying a given predicate.  Using predicate subtypes,
the type of the division operator, for example, can be constrained to
admit only nonzero denominators.  Division then becomes a total operation
on the domain consisting of arbitrary numerators and nonzero denominators.
The domain of a \emph{pop} operation on stacks can be similarly restricted
to nonempty stacks.  PVS thus admits partial functions within the
framework of a logic of total functions by enriching the type system to
include predicate subtypes.  We find this use of predicate subtypes to be
significantly in tune with conventional mathematical practice of being
explicit about the domain over which a function is defined.

\section{An Example: \texttt{stacks}}\label{stacks-example}
\index{stacks example@\texttt{stacks} example}

In this section we discuss a specific example, the theory of
\texttt{stacks}, in order to give a feel for the various aspects of the
PVS language before going into detail.  Apart from the basic notation for
defining a theory, this example illustrates the use of type parameters at
the theory level, the general format of declarations, the use of predicate
subtyping to define the type of nonempty stacks, and the generation of
typechecking obligations.

\pvstheory{stacks-alltt}{{Theory \texttt{stacks}}}{stacks-alltt}

Figure~\ref{stacks-alltt} illustrates a theory for stacks of an arbitrary
type with corresponding stack operations.  Note that this is not the
recommended approach to specifying stacks; a more convenient and complete
specification is provided in Section~\ref{stacks-adt},
page~\pageref{stacks-adt}.

The first line introduces a theory named \texttt{stacks} that is
parameterized by a type \texttt{t} (the \emph{formal parameter} of
\texttt{stacks}).  The keyword \texttt{TYPE+} indicates that \texttt{t} is
a \emph{non-empty} type.  The uninterpreted (nonempty) type \texttt{stack}
is declared, and the constant \texttt{empty} and variable \texttt{s} are
declared to be of type \texttt{stack}.  The defined predicate
\texttt{nonemptystack?}~is then declared on elements of type
\texttt{stack}; it is \texttt{true} for a given \texttt{stack} element
iff\footnote{If and only if.} that element is not equal to
\texttt{empty}.\footnote{The \texttt{bool} type and \texttt{/=} operator
are declared in the \emph{prelude}, which is a large body of theories that
are preloaded into PVS.  %This is described in Appendix~/ref{prelude}.
}
The functions \texttt{push}, \texttt{pop}, and \texttt{top} are then
declared.  Note that the predicate \texttt{nonemptystack?}~is being used
as a type in specifying the signatures of these functions; any predicate
may be used where a type is expected simply by putting parentheses around
it.

The variables \texttt{x} and \texttt{y} are then declared, followed by the
usual axioms for \texttt{push}, \texttt{pop}, and \texttt{top}, which make
\texttt{push} a stack constructor and \texttt{pop} and \texttt{top} stack
accessors.  Finally, there is the theorem \texttt{pop2push2}, that can
easily be proved by two applications of the \texttt{pop\_push} axiom.

This simple theorem has an additional facet that shows up during
typechecking.  Note that \texttt{pop} expects an element of type
\texttt{(nonemptystack?)} and returns a value of type \texttt{stack}.
This works fine for the inner \texttt{pop} because it is applied to
\texttt{push}, which returns an element of type \texttt{(nonemptystack?)};
but the outer occurrence of \texttt{pop} cannot be seen to be type correct
by such syntactic means.  In cases like these, where a subtype is expected
but not directly provided, the system generates a \emph{type-correctness
condition} (TCC).  In this case, the TCC is
\begin{pvsex}
  pop2push2_TCC1: OBLIGATION
    FORALL (s: stack, x, y: t): nonemptystack?(pop(push(x, push(y, s))));
\end{pvsex}
and is easily proved using the \texttt{pop\_push} axiom.  The system keeps
track of all such obligations and will flag the unproved ones during proof
chain analysis.

Parameterized theories such as \texttt{stacks} introduce theory schemas,
where the type \texttt{t} may be instantiated with any other nonempty
type.  To use the types, constants, and formulas of the \texttt{stacks}
theory from another theory, the \texttt{stacks} theory must be imported,
with \emph{actual parameters} provided for the corresponding theory
parameters.  This is done by means of an \texttt{IMPORTING} clause. For
example, consider the theory \texttt{ustacks}.
\begin{session}
  ustacks : THEORY
   BEGIN
    IMPORTING stacks[int], stacks[stack[int]]

    si : stack[int]
    sos : stack[stack[int]] = push(si, empty)
   END ustacks
\end{session}
It imports stacks of integers and stacks of stacks of integers.  The constant
\texttt{si} is then declared to be a stack of integers, and the constant
\texttt{sos} is a stack of stacks of integers whose top element is
\texttt{si}.  Note that the system is able to determine which instance of
\texttt{push} and \texttt{empty} is meant from the type of the first
argument.  In general, the typechecker infers the type of an expression
from its context.  

The following chapters provide more details on the various features of the
language.  The lexical aspects of the language are explained in
Chapter~\ref{lexical}.  Chapter~\ref{declarations} describes declarations,
Chapters~\ref{types} and~\ref{expressions} describe type expressions and
expressions, and Chapter~\ref{theories} explains theories, theory
parameters, and the importing and exporting of names.  Theory
interpretaions and mappings are described in
Chapter~\ref{interpretations}.  Chapter~\ref{names} describes names and
name resolution, and Chapter~\ref{adts} details the datatype facility of
PVS.  Finally, Appendix~\ref{grammar} provides the grammar of the
language.
%and Appendix~\ref{prelude} gives a brief overview of the theories
%of the PVS prelude.

%%% Local Variables: 
%%% mode: latex
%%% TeX-master: "language"
%%% End: 
