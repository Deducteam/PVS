% Document Type: LaTeX
% Master File: language.tex

\chapter{Types}\label{types}
\index{type|(pidx}

PVS specifications are strongly typed, meaning that every expression has
an associated type (although it need not be unique, more on this later).
The PVS type system is based on \emph{structural
equivalence}\index{structural equivalence} instead of \emph{name
equivalence}\index{name equivalence}, so types are closely related to
sets, in that two types are equal iff they have the same elements.
Section~\ref{type-declarations} describes the introduction of type names,
which are the simplest type expressions.  More complex type
expressions\index{type expressions} are built from these using \emph{type
constructors}\index{type constructors}.  There are type constructors for
\emph{subtypes}\index{subtypes}\index{type!subtype}, \emph{function
types}\index{function types}\index{type!function}, \emph{tuple
types}\index{tuple types}\index{type!tuple}, \emph{cotuple
types}\index{cotuple types}\index{type!cotuple}, and \emph{record
types}\index{record types}\index{type!record}.  Function, record, and
tuple types may also be \emph{dependent}\index{dependent
types}\index{type!dependent}.  A form of \emph{type
application}\index{type application}\index{type!application} is provided
that makes it more convenient to specify parameterized subtypes.  There
are also provisions for creating \emph{abstract datatypes}, described in
Chapter~\ref{datatypes}.

Type expressions occur throughout a specification; in particular, they may
appear in theory parameters, type declarations, variable declarations,
constant declarations, recursive and inductive definitions, conversions,
and judgements.  In addition, they may appear in certain expressions
(coercions and local bindings, see pages~\pageref{coercions}
and~\pageref{binding-expressions}, respectively), and as actual parameters
in names (page~\pageref{names}).  In the many examples which follow, type
expressions will be presented in the context of type declarations; but it
must be remembered that they can appear in any of the above places.

\pvsbnf{bnf-type-expr}{Type Expression Syntax}

\section{Subtypes}\label{subtypes}
\index{subtypes|(}\index{type!subtype|(}

Any collection of elements of a given type itself forms a type, called a
\emph{subtype}.  The type from which the elements are taken is called the
\emph{supertype}\index{supertype}.  The elements which form the subtype
are determined by a \emph{subtype predicate}\index{subtype predicate} on
the supertype.

Subtypes in PVS provide much of the expressive power of the language,
at the cost of making typechecking undecidable.  There are two forms of
subtypes.  The first is similar to the notation used to define a set:
\begin{pvsex}
  t: TYPE = \setb{}x: s | p(x)\sete
\end{pvsex}
%
where \texttt{p} is a predicate on the type \texttt{s}.\footnote{If \texttt{x}
has been previously declared as a variable of type \texttt{s}, then the
``\texttt{:~s}'' may be omitted.} This has the usual set-theoretical
meaning, since types in PVS are modeled as sets.  Subtypes may also
be presented in an abbreviated form, by giving a predicate surrounded by
parentheses:
\begin{pvsex}
  t: TYPE = (p)
\end{pvsex}
%
This is equivalent to the form above.

Note that if the predicate \texttt{p} is everywhere false, then the type
is empty.  PVS supports empty types\index{empty type}\index{type!empty},
and the term \emph{type} is used to refer to any type, including the empty
type.  This is discussed in Section~\ref{type-declarations} (page~\pageref{type-declarations}).

Subtypes tend to make specifications more succinct and easier to read.
For example, in a specification such as
\begin{pvsex}
  FORALL (i:int):
    (i >= 0 IMPLIES (EXISTS (j:int): j >= 0 AND j > i))
\end{pvsex}
it is much more difficult to see what is being stated than in the
equivalent
\begin{pvsex}
  FORALL (i:nat): (EXISTS (j:nat): j > i))
\end{pvsex}
%
where \texttt{nat} is defined in the prelude as
\begin{pvsex}
  naturalnumber: NONEMPTY\_TYPE = \setb{}i:integer | i >= 0\sete CONTAINING 0
  nat: NONEMPTY\_TYPE = naturalnumber
\end{pvsex}

Subtype constructors consist of a \emph{supertype}\index{supertype} and a
\emph{subtype predicate}\index{subtype predicate} on the supertype.  The
primary property of a subtype is that any element which belongs to the
subtype automatically belongs to the supertype.  In addition, functions
defined on a type automatically apply to the subtype.

\index{TCC|(}

There are two \emph{type-correctness conditions} (\tccs) associated with
subtypes.  The first concerns \emph{empty types}\index{empty
type}\index{type!empty} as described in section~\ref{emptytypes}.  The
second \tcc\ associated with subtypes is the \emph{subtype}
\tcc,\index{subtype TCC}\index{TCC!subtype}, which comes about from the
use of operations defined on subtypes that are applied to elements of the
supertype.  By this means partial functions may be handled directly,
without recourse to a partial term logic or some form of multi-valued
logic.  For instance, division in PVS is a total function, with signature
\texttt{[real, nonzero\_real -> real]}.  So given the formula
\begin{pvsex}
  div_form: FORMULA (FORALL (x, y: int):
                      x /= y IMPLIES (x - y)/(y - x) = -1)
\end{pvsex}
%
the denominator is of type integer, but the signature for \texttt{/}
demands a \texttt{nonzero\_real}.  The typechecker thus generates a
\emph{subtype} \tcc\ whose conclusion is \texttt{(y - x) /= 0}.  The
premises of the \tcc\ are obtained from the expressions
\emph{context}\index{context}---the conditions which lead to the
\texttt{/} operator---in this case \texttt{x /= y}.\footnote{As described
in the Formal Semantics~\cite{PVS:semantics}, the context containing
declarations is extended to allow boolean expressions.}  The \tcc\ is then
\begin{pvsex}
  div_form_TCC1: OBLIGATION
    (FORALL (x,y: int): x /= y IMPLIES (y - x) /= 0)
\end{pvsex}
which is easily discharged by the prover.  In general, the context of an
expression is obtained from expressions involving \texttt{IF-THEN-ELSE},
\texttt{AND}, \texttt{OR}, and \texttt{IMPLIES} by translating to the \texttt{IF-THEN-ELSE} form.  Specifically,
\begin{center}
\begin{tabular}{|lc|} \hline
Expression & Context for $e$ \\ \hline
\texttt{IF $a$ THEN $e$ ELSE $c$ ENDIF} & $a$ \\
\texttt{IF $a$ THEN $b$ ELSE $e$ ENDIF} & \texttt{NOT $a$} \\
\texttt{$a$ AND $e$} & $a$ \\
\texttt{$a$ OR $e$} & \texttt{NOT $a$} \\
\texttt{$a$ IMPLIES $e$} & $a$ \\ \hline
\end{tabular}
\end{center}
Note that only these operators are treated this way; if, for example,
\texttt{IMPLIES} is overloaded it will not include the left-hand side in
the context for typechecking the right-hand side.  The \tccs\ generated
from the context of expression involving a subtype are sufficient, but not
necessary conditions that ensure that the value of the expression does
not depend on the value of functions applied outside their domain.

Subtype \tccs\ may occur anywhere there is a mismatch between the type of
a term and the use of it, not just in function applications.  For example,
the following use of record types leads to an unprovable subtype \tcc.
\begin{pvsex}
  r: [# a, b: nzint #] = (# a := 0, b := 0 #)
\end{pvsex}

\index{TCC|)}

\index{type!subtype|)}\index{subtypes|)}

\section{Function Types}\label{function-types}
\index{function types|(}\index{type!function|(}

Function types have three equivalent forms:
\begin{itemize}
\item \texttt{[t\(_1\), \ldots, t\(_n\) -> t]}

\item \texttt{FUNCTION[t\(_1\), \ldots, t\(_n\) -> t]}

\item \texttt{ARRAY[t\(_1\), \ldots, t\(_n\) -> t]}
\end{itemize}
%
where each \texttt{t$_i$} is a type expression.  An element of this type is
simply a function whose domain is the sequence of types \texttt{t$_1$},
\ldots, \texttt{t$_n$}, and whose range is \texttt{t}.  A function type is empty
if the range is empty and the domain is not.  There is no difference in
meaning between these three forms; they are provided to support different
intensional uses of the type, and may suggest how to handle the given type
when an implementation is created for the specification.

The two forms \texttt{pred[t]}\index{pred@{\texttt{pred}}} and \texttt{setof[t]}\index{setof@{\texttt{setof}}} are both provided in the
prelude as shorthand for \texttt{[t ->
bool]}.  There is no difference in semantics, as sets in
PVS are represented as predicates.  The different keywords are
provided to support different intentions; \texttt{pred} focuses on
properties while \texttt{setof} tends to emphasize elements.

A function type \texttt{[t$_1$,\ldots,t$_n$ -> t]} is a subtype of
\texttt{[s$_1$,\ldots,s$_m$ -> s]} iff \texttt{s} is a subtype of
\texttt{t}, $n = m$, and \texttt{s$_i$} = \texttt{t$_i$} for $1 \leq i \leq n$.
This leads to subtype \tccs\ (called \emph{domain mismatch
\tccs})\index{domain mismatch TCC}\index{TCC!domain mismatch} that state
the equivalence of the domain types.  For example, given
\begin{pvsex}
  p, q: pred[int]
  f: [\setb{}x: int | p(x)\sete -> int]
  g: [\setb{}x: int | q(x)\sete -> int]
  h: [int -> int]
  eq1: FORMULA f = g
  eq2: FORMULA f = h
\end{pvsex}
%
The following \tccs\ are generated:
\begin{pvsex}
eq1_TCC1: OBLIGATION
  (FORALL (x1: \setb{}x : int | q(x)\sete, y1 : \setb{}x : int | p(x)\sete) :
     q(y1) AND p(x1))

eq2_TCC1: OBLIGATION
  (FORALL (x1: int, y1 : \setb{}x : int | p(x)\sete) :
     TRUE AND p(x1))
\end{pvsex}

Section~\ref{conversion-examples} on page~\pageref{conversion-examples}
explains how the \texttt{restrict} conversion may be automatically applied
in some cases to eliminate the production of these \tccs.

\index{type!function|)}\index{function types|)}


\section{Tuple Types}\label{tuple-types}
\index{tuple types|(}\index{type!tuple|(}

Tuple types (also called product types) have the form \texttt{[t$_1$,
\ldots, t$_n$]}, where the \texttt{t$_i$} are type expressions.  Note that
the 0-ary tuple type is not allowed.  Elements of this type are tuples
whose components are elements of the corresponding type.  For example,
\texttt{(1, TRUE, (LAMBDA (x:int):\ x + 1))} is an expression of type
\texttt{[int, bool, [int -> int]]}.  Order is important.  Associated with
every $n$-tuple type is a set of projection functions: \texttt{`1},
\texttt{`2}, \ldots, (or \texttt{proj\_1}, \texttt{proj\_2}, \ldots) where
the $i$th projection is of type \texttt{[[t$_1$, \ldots, t$_n$] ->
t$_i$]}.  A tuple type is empty if any of its component types is empty.
Function type domains and tuple types are closely related, as the types
\texttt{[t$_1$,\ldots, t$_n$ -> t]} and \texttt{[[t$_1$,\ldots, t$_n$] ->
t]} are equivalent; see Section~\ref{tuple-exprs} for more details.

\index{type!tuple|)}\index{tuple types|)}

\section{Cotuple Types}

Cotuple types (also known as coproduct or sum types) provide a way to
provide a disjoint union of types.  The syntax is similar to that for
tuple types, but with the comma replaced by a \texttt{+}.  For
example,

\begin{pvsex}
cT: TYPE = [int + bool + [int -> int]]
\end{pvsex}

Associated with a cotuple type are injections \texttt{IN\_}\textit{i},
predicates \texttt{IN?\_}\textit{i}, and extractions \texttt{OUT\_}\textit{i}
(none of these is case-sensitive).  For example, in this case we have
   
\begin{pvsex}
IN_1:  [int -> cT]
IN?_1: [cT -> bool]
OUT_1: [(IN?_1) -> int]
\end{pvsex}

Thus \texttt{IN\_2(true)} creates a \texttt{cT} element, and an arbitrary
\texttt{cT} element \texttt{c} is processed using \texttt{CASES}, e.g.,

\begin{pvsex}
CASES c OF
  IN_1(i): i + 1,
  IN_2(b): IF b THEN 1 ELSE 0 ENDIF,
  IN_3(f): f(0)
ENDCASES
\end{pvsex}

This is very similar to using the \texttt{union} datatype defined in the
prelude, but allows for any number of arguments, and doesn't generate
a datatype theory.

Typechecking expressions such as \texttt{IN\_1(3)} requires that the
context of its use be known.  This is similar to the problem of a
standalone \texttt{PROJ\_1}, and both are supported:
	 
\begin{pvsex}
F: [cT -> bool]
FF: FORMULA F(IN_1(3))
G: [[int -> [int, bool, [int -> int]]] -> bool]
GG: FORMULA G(PROJ_1)
\end{pvsex}

This means it is easy to write terms that are ambiguous:
	 
\begin{pvsex}
HH: FORMULA IN_1(3) = IN_1(4)
HH: FORMULA PROJ_1 = PROJ_1
\end{pvsex}

This can be disambiguated by providing the type explicitly:
	 
\begin{pvsex}
HH: FORMULA IN_1[cT](3) = IN_1(4)
HH: FORMULA PROJ_1 = PROJ_1[[int, int]]
\end{pvsex}

This uses the same syntax as for actual parameters, but doesn't mean
the same thing, as the projections, injections, etc., are builtin, and
not provided by any theories.  Note that coercions don't work in this
case, as \texttt{PROJ\_1::[[int, int] -> int]} is the same as

\begin{pvsex}
(LAMBDA (x: [[int, int] -> int]): x)(PROJ_1)
\end{pvsex}

and not

\begin{pvsex}
LAMBDA (x: [int, int]): PROJ_1(x)
\end{pvsex}

Note that the prover handles cotuple extensionality and reduction rules as
expected.


\section{Record Types}\label{record-types}
\index{record types|(}\index{type!record|(}

Record types are of the form \texttt{[\# a$_1$:t$_1$, \ldots, a$_n$:t$_n$
\#]}.  The \texttt{a$_i$} are called \emph{record accessors}\index{record
accessors} or fields and the \texttt{t$_i$} are types.  Record types are
similar to tuple types, except that the order is unimportant and accessors
are used instead of projections.  Record types are empty if any of the
component types is empty.

Note that the fields of a record type must be applied, they are not
understood as functions.  See Section~\ref{record-expressions}.

\index{type!record|)}\index{record types|)}

\section{Dependent types}\label{dependent-types}
\index{dependent types|(}\index{type!dependent|(}

Function, tuple, and record types may be dependent; in other words, some
of the type components may depend on earlier components.  Here are some
examples:
\begin{pvsex}
  rem: [nat, d: \setb{}n: nat | n /= 0\sete -> \setb{}r: nat | r < d\sete]
  pfn: [d:pred[dom], [(d) -> ran]]
  stack: [\# size: nat, elements: [\setb{}n:nat | n < size\sete -> t] \#]
\end{pvsex}
The declaration for \texttt{rem} indicates explicitly the range of the
remainder function, which depends on the second argument.  Function types
may also have dependencies within the domain types; \eg\ the second domain
type may depend on the first.  Note that for function and tuple dependent
types, local identifiers need to be given only for those types on which
later types depend.

The tuple type \texttt{pfn} encodes partial functions as pairs consisting
of a predicate on the domain type and a function from the subtype
defined by that predicate to the range \texttt{ran}.  If the second
component were given instead as a function of type \texttt{[dom -> ran]},
then equality no longer works as intended.  For example, the absolute
value function \texttt{abs} and the identity function \texttt{id} are the same
on the domain \texttt{nat}, so we would like to have
\begin{pvsex}
  ((LAMBDA (x:int):x >= 0),abs) = ((LAMBDA (x:int):x >= 0),id)
\end{pvsex}
%
but without the dependency this would be equivalent to \texttt{abs = id}.

\texttt{stack} encodes a stack as a pair consisting of a size and an array
mapping initial segments of the natural numbers to \texttt{t}.  This is
similar to the \texttt{pfn} example---in fact, if we were willing to use a
tuple instead of a record encoding, \texttt{stack} could be declared as an
instance of the type of \texttt{pfn}.

Another example, presented in~\cite{Cheng&Jones90} as a ``challenge'' to
specification languages without partial functions, is easily handled
with dependent types as shown below.
\begin{pvsex}
  subp(i:int,(j:int | i >= j)): RECURSIVE int =
       (IF (i=j) THEN 0 ELSE (subp(i, j+1)+1) ENDIF)
    MEASURE i - j
\end{pvsex}
However, some formulas that are valid with partial functions are not even
well-formed in PVS:
\begin{pvsex}
  subp_lemma: LEMMA subp(i, 0) = i OR subp(0, i) = i
\end{pvsex}
This generates unprovable \tccs.  In practice this is rarely a problem.

\index{type!dependent|)}\index{dependent types|)}
\section{Cotuple Types}\label{cotuple-types}
\index{cotuple types|(}\index{type!cotuple|(}
\index{coproduct types|see{cotuple type}}
\index{sum types|see{cotuple type}}

\emph{Cotuple types} (also called \emph{coproduct} or \emph{sum} types)
provide a way to form the disjoint union of types.  The syntax is similar
to that for tuple types, but with `\texttt{+}' in place of `\texttt{,}',
so have the form \texttt{[t$_1$ + \ldots + t$_n$]}.  Elements of this type
are essentially pairs consisting of an index and a value for the type
corresponding to the index.  In PVS the syntax for this is
\texttt{IN\_$i(e)$}, where $e$ is an expression of type \texttt{t$_i$}.
For example, \texttt{IN\_2(3)} is an expression of type \texttt{[bool + int
+ [int -> int]]}, or any other cotuple type whose second component type
contains \texttt{3}.  A cotuple type is empty iff all its component types
are empty.

\index{type!cotuple|)}\index{cotuple types|)}

\index{type|)}

%%% Local Variables: 
%%% mode: latex
%%% TeX-master: "language"
%%% End: 
