% Document Type: LaTeX
% Master File: language.tex

\chapter{Abstract Datatypes}\label{datatypes}\label{adts}

PVS provides a powerful mechanism for defining abstract datatypes.  This
mechanism is akin to, but more sophisticated than, the \emph{shell}
principle of the Boyer-Moore prover~\cite{Boyer-Moore79}).  A PVS datatype
is specified by providing a set of \emph{constructors} along with
associated \emph{accessors} and \emph{recognizers}.  When a datatype is
typechecked, a new theory is created that provides the axioms and
induction principles needed to ensure that the datatype is the initial
algebra defined by the constructors.

\pvsbnf{bnf-adts}{Datatype Syntax}

The syntax for PVS datatypes is given in Figure~\ref{bnf-adts}.  Datatypes
may appear at the \emph{top-level} as with theory declarations, or
\emph{in-line} as a declaration within a theory.\footnote{Enumeration
types are actually in-line datatypes---see Section~\ref{enum-types}.}
Typechecking a top-level datatype named \texttt{foo} causes the generation
of a new PVS file named \texttt{foo\_adt.pvs} containing up to three
theories as described below.  Typechecking an in-line datatype has the
effect of adding new declarations to the current theory, effectively
replacing the in-line datatype.  In-line datatypes are more restricted:
they may not have formal parameters or assuming parts, and they will not
generate the recursive combinators described below.  The declarations
generated for an in-line datatype may be viewed using the
\texttt{M-x~prettyprint-expanded} command (see the \emph{PVS System
Guide}~\cite{PVS:userguide}).

\section{A Datatype Example: \texttt{stack}}\label{stacks-adt}
An example of a datatype is \texttt{stack}:
\begin{session}
  stack[T: TYPE]: DATATYPE
   BEGIN
    empty: empty?
    push(top:T, pop:stack): nonempty?
   END stack
\end{session}
The \texttt{stack} datatype has two \emph{constructors}, \texttt{empty} and
\texttt{push}, that allow stack elements to be constructed.  For example,
the term \texttt{push(1, empty)} is an element of type \texttt{stack[int]}.
The \emph{recognizers} \texttt{empty?}\ and \texttt{nonempty?}\ are predicates
over the \texttt{stack} datatype that are true when their argument is
constructed using the corresponding constructor.  Given a \texttt{stack}
element that is known to be \texttt{nonempty?}, the \emph{accessors}
\texttt{top} and \texttt{pop} may be used to extract the first and second
arguments.

Typechecking the \texttt{stack} specification automatically creates a new
file \texttt{stack\_adt.pvs}, that contains the material found in
the next five figures.  This new file contains three theories:
\texttt{stack\_adt}, \texttt{stack\_adt\_map}, and
\texttt{stack\_adt\_reduce}.

\pvstheory{stack_adtA-alltt}{Theory \texttt{stack\_adt} (continues)}{stack_adtA-alltt}
\pvstheory{stack_adtB-alltt}{Theory \texttt{stack\_adt} (continues)}{stack_adtB-alltt}
\pvstheory{stack_adtC-alltt}{Theory \texttt{stack\_adt} (continues)}{stack_adtC-alltt}
\pvstheory{stack_adtD-alltt}{Theory \texttt{stack\_adt\_map}}{stack_adtD-alltt}
\pvstheory{stack_adtE-alltt}{Theory \texttt{stack\_adt\_reduce}}{stack_adtE-alltt}

The first theory \texttt{stack\_adt} is parametric in type \texttt{T}.
This is a specification of ``stacks of \texttt{T}'', where \texttt{T} may
be instantiated by any defined type when the stacks datatype is imported.
Thus ``stacks of integers'' as well as ``stacks of stacks of integers''
may be defined using this theory.  The first few lines of the theory
define the main type of stacks \texttt{stack}, the recognizers
\texttt{emptystack?} and \texttt{nonemptystack?}, the constructors
\texttt{empty} and \texttt{push}, and the accessors \texttt{top} and
\texttt{pop} are declared.

The \texttt{stack\_ord} function is defined, and an axiom provided for
it's definition.  This is provided instead of a disjointness axiom,
because the disjointness axiom becomes difficult to generate and use if
the number of constructors is large.  The disjointness comes from the fact
that the natural numbers are distinct.  The \texttt{ord} function is then
defined to return \texttt{0} on an empty stack and \texttt{1} on a
nonempty stack.  This is the same function as \texttt{stack\_ord}, but is
easier to use.

Then a series of axioms are given.  The
\texttt{stack\_empty\_extensionality} axiom states that there is only one
bottom element of the datatype: \texttt{empty}.
\texttt{stack\_push\_extensionality} states that any two stacks that have
the same \texttt{top} and \texttt{pop} (have the same components) are the
same.  The \texttt{stack\_push\_eta} axiom states that \texttt{pop}ping
and \texttt{push}ing the same element off and onto a stack results in a
stack identical to the original.  \texttt{stack\_top\_push} says that if
you \texttt{push} and element on a stack, you get that same element when
you \texttt{pop} it back off.  \texttt{stack\_pop\_push} says that pushing
something on a stack and then popping it back off results in the original
stack.

The \texttt{stack\_inclusive} axiom states that all stacks are either
\texttt{empty?} or \texttt{nonempty?}.  The PVS prover builds this axiom
in, so that it rarely needs be cited by a user.

\newpage
The next axiom, \texttt{stack\_induction}, introduces an induction formula
for stacks stating that any predicate $p$ of stacks that
\begin{enumerate}
\item holds for the empty stack (the base case), and
\item if $p$ holds for some stack then $p$ holds for the result of
\texttt{push}ing anything of the right type onto that stack (the induction
step),
\end{enumerate}
then $p$ holds for all stacks.

Then some useful functions are defined over stacks.  The stack predicate
\texttt{every} takes as arguments a predicate over \texttt{T} and a stack
and returns \texttt{TRUE} iff all elements on the stack satisfy the given
predicate.  \texttt{every} is introduced in both curried and uncurried
forms.  The stack predicate \texttt{some} is dual to \texttt{every},
returning \texttt{TRUE} iff there is some element on the stack that
satisfies the predicate.  The \texttt{subterm} predicate takes two stacks
and returns \texttt{TRUE} if and only if the first argument stack is a
subterm of the second.  That is, if the second stack consists of the first
stack with some (perhaps zero) elements pushed onto it.  The \texttt{<<}
predicate is the strict (irreflexive) \texttt{subterm} predicate.  Thus
for all stacks $s$, \texttt{subterm}$(s,s)$ holds, but for no stack $s$
does \texttt{<<}$(s,s)$ hold.  An alternative equivalent definition of
\texttt{<<} is as follows:
\begin{pvsex}
  <<(x: stack, y: stack): boolean = subterm(x,y) AND NOT x = y
\end{pvsex}
However, this definition is more awkward to use in a proof, as the
recursion is hidden in the definition of \texttt{subterm}.  For this
reason the definitions for \texttt{every}, \texttt{some},
\texttt{subterm}, and \texttt{<<}, are each defined as standalone
functions, though some of them could be defined in terms of the others.

The last four declarations of the theory \texttt{stack\_adt} are functions
which reduce a stack to a natural number or to an ordinal.  These
functions are useful for simplifying the proof of termination of
user-defined functions over stacks.  Recall that PVS requires recursive
functions to include a \emph{measure}, which is used to generate
termination conditions.  The primary use of the recursive combinator is to
allow measure functions to be specified.  The function
\texttt{reduce\_nat} takes a natural number and a function.  The natural
number is used for the empty stack, and then for each element on the
stack, the input function is applied to the element from the stack and the
current reduced natural number, returning a natural number.  The function
\texttt{reduce\_nat} returns the final natural number.  The function
\texttt{REDUCE\_nat} is analogous to \texttt{reduce\_nat}, except that the
reducing function is also given the entire contents of the stack.  This
version of reduction can be useful for complicated measures that involve,
for example, the number of repeated elements appearing on the stack.  The
simpler form of reduce is difficult to apply to such situations.  The
functions \texttt{reduce\_ordinal} and \texttt{REDUCE\_ordinal} are
analogous to \texttt{reduce\_nat} and \texttt{REDUCE\_nat} except that
they return ordinal numbers instead of natural numbers.  It is rare that a
termination argument requires the use of ordinals, so the simpler
\texttt{reduce\_nat} form is more often used.  This completes the
description of the \texttt{stack\_adt} theory.

The second theory in the file \texttt{stack\_adt.pvs} is
\texttt{stack\_adt\_map}.  This theory takes two types \texttt{T} and
\texttt{T1} as parameters, imports the \texttt{stack\_adt} theory, and
defines a mapping from \texttt{stacks[T]} to \texttt{stacks[T1]}.  The
higher-order \texttt{map} function takes a function \texttt{f} of type
\texttt{[T -> T1]}, and a stack of \texttt{T}, and returns a stack of
\texttt{T1} obtained by applying \texttt{f} to each element on the input
stack.  \texttt{map} is defined in both curried and uncurried forms.
\texttt{map} couldn't reside in the \texttt{stack\_adt} theory because
that theory has only one type parameter, while the \texttt{map} functions
require two: In order to construct and access stacks in two theories,
\texttt{map} must be parameterized in the two types.

Also in the \texttt{stack\_adt\_map} is a relational \texttt{every}
function.  It lifts a relation \texttt{R} between \texttt{T} and \texttt{T1},
to stacks of \texttt{T} and \texttt{T1}.  It is true if the stacks are the
same size, and corresponding elements satisfy \texttt{R}.

The third and final theory generated from \texttt{stack\_pvs} is
\texttt{stack\_adt\_reduce}.  This theory provides a generalized version
of \texttt{reduce\_nat} and \texttt{REDUCE\_nat}.  It takes as parameters
a type \texttt{T} and a range type \texttt{range}.  It defines a
generalized \texttt{reduce} which reduces stacks of \texttt{T} to elements
of \texttt{range}.  The functions \texttt{reduce\_nat},
\texttt{REDUCE\_nat}, \texttt{reduce\_ordinal}, and
\texttt{REDUCE\_ordinal} could have been defined using
\texttt{stack\_adt\_reduce}, but the direct definitions are provided for
additional user convenience.  The generalized \texttt{reduce} can be used
to provide evidence of termination of user-defined functions, but the
predefined versions such as \texttt{reduce\_nat} are easier to use in most
cases.

\section{Datatype Details}

In general, a datatype declaration has the form
\begin{pvsex}
  adt: DATATYPE WITH SUBTYPES S\(\sb{1}\), \ldots, S\(\sb{n}\)
    BEGIN
     cons\(\sb1\)(acc\(\sb{11}\): T\(\sb{11}\), \ldots, acc\(\sb{1{n\sb1}}\): T\(\sb{1{n\sb1}}\)): rec\(\sb1\) : S\(\sb{i\sb{1}}\)
     \vdots
     cons\(\sb{m}\)(acc\(\sb{m1}\): T\(\sb{m1}\), \ldots, acc\(\sb{1n\sb{m}}\): T\(\sb{1n\sb{m}}\)): rec\(\sb{m}\) : S\(\sb{i\sb{m}}\)
    END adt
\end{pvsex}
%
where the \texttt{cons$_i$} are the
\emph{constructors}\index{constructor}\index{datatype!constructor}, the
\texttt{acc$_{ij}$} are the
\emph{accessors}\index{accessor}\index{datatype!accessor}, the
\texttt{T$_{ij}$} are type expressions, and the \texttt{rec$_i$} are
\emph{recognizers}\index{recognizer}\index{datatype!recognizer}.  Each
line is referred to as a \emph{constructor
specification}\index{constructor specification}\index{datatype!constructor
specification}.  There are a number of restrictions enforced on
constructor specifications:
\begin{itemize}

\item The datatype identifier may not be used for a recognizer,
accessor, or subtype:\newline
($\texttt{adt} \not\equiv \texttt{rec}_i$ for all $i$, $\texttt{adt}
\not\equiv \texttt{acc}_{ij}$ for all $i$ and $j$, and $\texttt{adt}
\not\equiv \texttt{S}_i$ for all $i$).

\item The subtype names must be unique:
($i \neq j \Rightarrow \texttt{S}_i \not\equiv \texttt{S}_j$)

\item Each subtype name must be used at least once.

\item The constructor names must be unique:
($i \neq j \Rightarrow \texttt{cons}_i \not\equiv \texttt{cons}_j$).

\item The recognizer names must be unique:
($i \neq j \Rightarrow \texttt{rec}_i \not\equiv \texttt{rec}_j$).

\item No identifier may be used as both a constructor and a recognizer:\newline
($\texttt{cons}_i \not\equiv \texttt{rec}_j$ forall $i$ and $j$).

\item Duplicate accessor identifiers are not allowed within a single
constructor specification:
($j \neq k \Rightarrow \texttt{acc}_{ij} \not\equiv \texttt{acc}_{ik}$).

\end{itemize}

As seen in the \texttt{stack} example, datatypes may be recursive; this is
the case when the type of one or more of the accessors reference the
datatype.  In PVS, all such occurrences must be positive, where a type
occurrence \texttt{T} is positive in a type expression $\tau$ iff either
\begin{itemize}
\item $\tau\equiv \texttt{T}$.

\item $\tau\equiv \{x:\tau'|p(x)\}$ and the occurrence \texttt{T} is
positive in $\tau'$.

\item $\tau\equiv [{\tau_1} \rightarrow {\tau_2}]$ and the occurrence
\texttt{T} is positive in $\tau_2$\@.  For example, \texttt{T} occurs
positively in \texttt{sequence[T]} where \texttt{sequence[T]} is defined
in the PVS prelude as the function type \texttt{[nat -> T]}\@.

\item $\tau \equiv [\tau_1,\ldots, \tau_n]$ and the occurrence \texttt{T}
is positive in some $\tau_i$.

\item $\tau\equiv [\#\ l_1 : \tau_1, \ldots, l_n : \tau_n\ \#]$ and the occurrence \texttt{T} is positive in some $\tau_i$\@. 

\item $\tau\equiv \mbox{\emph{datatype}}[\tau_1,\ldots, \tau_n]$, where
\emph{datatype} is a previously defined datatype and the occurrence
\texttt{T} is positive in $\tau_i$, where $\tau_i$ is a \emph{positive
parameter} of \emph{datatype}\@.
\end{itemize}

When a top-level datatype is given with formal type parameters, they are
checked for whether their occurrences are all positive; this is used as
described above for any datatype that imports this one, as well as
determining some of the declarations described below.

When a datatype is typechecked, a number of new declarations are
generated:
\begin{itemize}

\item The datatype identifier is used to create an uninterpreted type
declaration.  In general, the term \emph{datatype} refers to this type.

\item Each recognizer is used to declare an uninterpreted subtype of the
datatype.

\item Each subtype identifier is used to declare an interpreted type that
is the disjunction of the types given by the recognizers that reference
the subtype identifier in the constructor specification.

\item Each constructor and accessor is used to generate a constant
declaration.

\item An \texttt{\emph{id}\_ord} uninterpreted function is created, and an
axiom \texttt{\emph{id}\_ord\_defaxiom} defines its values.  This is
provided instead of a disjointness axiom, because the disjointness axiom
becomes difficult to generate and use when the number of constructors is
large.

\item An \texttt{ord} function is generated that gives a zero-based number
to each constructor (e.g., \texttt{ord(null) = 0} and \texttt{cons(1,null)
= 1}).  This is mostly useful for enumeration types.

\item An extensionality axiom is generated for each constructor
specification.

\item An eta axiom is generated for each constructor specification
that has accessors.

\item For each accessor an axiom is created that says that the accessor
composed with the corresponding constructor returns the correct value; \eg\
\begin{pvsex}
  acc\(\sb{ij}\)(cons\(\sb{i}\)(e\(\sb{i1}\),\ldots, e\(\sb{i{m\sb{i}}})\) = e\(\sb{ij}\)
\end{pvsex}

\item An inclusive axiom is generated that says that every element of
the datatype belongs to at least one recognizer subtype.  This axiom is
not actually needed in practice as the prover checks for this directly.

\item Two induction schemes are provided for proving properties of the
datatype.

\item If there is at least one constructor with accessors,\footnote{Note
that enumeration types have no accessors.}  and there are positive type
parameters to the datatype, then \texttt{every} and \texttt{some}
functions are defined that provide a predicate on the datatype in terms of
the positive types.

\item The \texttt{subterm} and \texttt{<<} (irreflexive subterm) functions
are defined, and an axiom is generated that states that \texttt{<<} is
well-founded.  This allows it to be used as an ordering relation in
recursive function definitions.

\item If there is at least one constructor with
accessors,\addtocounter{footnote}{-1}\footnotemark{} the
\texttt{reduce\_nat}, \texttt{REDUCE\_nat}, \texttt{reduce\_ordinal}, and
\texttt{REDUCE\_ordinal} recursion combinators are defined.  These provide
a means for defining notions like the size or depth of a datatype term.

Note that accessor subtypes involving the datatype are
``lifted''.  The following example shows why.
\begin{pvsex}
  dt: DATATYPE
   BEGIN
    c0: c0?
    c1(a1: \setb{}x: list[dt] | length(x) > 0\sete): c1?
    c2(a2: \setb{}x: list[dt] | every(c0?)(x)\sete): c2?
   END dt
\end{pvsex}
Consider the \texttt{reduc\_nat} function.  The signature for the lifted
mapping function for \texttt{c1} and \texttt{c2} are the same:
\texttt{[list[nat] -> nat]}.  It's obvious the mapping function for
\texttt{c2} function could have the signature \texttt{[\setb{}x: list[nat]
| length(x) > 0\sete{} -> nat]}, but there is no obvious way to map
\texttt{c2} without lifting it.  Since it is not trivial to determine
which predicates map nicely, we lift them all.  In the future we may
provide heuristics that refine this.

\item If some type parameter is positive a \texttt{map} function is
generated in a separate theory.  Every positive type parameter in the
datatype is associated with a pair of \texttt{map} parameters, which form
the domain and range of a corresponding function argument.  Given a set of
such functions and a term of the datatype, \texttt{map} returns a term
that has the same structure, but with the ``leaf'' elements replaced by
the function values.

\item A separate theory is generated for the \texttt{reduce} and
\texttt{REDUCE} functions.  These generalize the \texttt{reduce} functions
above to an arbitrary range type.

\end{itemize}

Note that in the stack example, the \texttt{stack} type is nonempty, since
\texttt{empty} is an element of \texttt{stack} even if the parameter type
\texttt{T} is instantiated with an empty type.  However, there is no
requirement that a datatype be nonempty, though if it is imported and a
constant is declared to be of that type, a TCC will be generated as
described on page~\pageref{emptytypes} in section~\ref{emptytypes}.

The \texttt{stack\_adt} theory is parameterized in the type \texttt{T},
and introduces the uninterpreted type \texttt{stack}.  Under normal
circumstances, this would imply no relation between, for example,
\texttt{stack[nat]} and \texttt{stack[int]}.  However, since every
occurrence of \texttt{T} in the accessor types is positive, we can infer
that \texttt{stack[nat]} is a subtype of \texttt{stack[int]}.  In general,
given a type $T$ and a subtype $S \equiv \setb{}x:T | p(x)\sete$, then
\texttt{stack[$S$]} is treated the same as $\setb{}s:
\texttt{stack[}T\texttt{]} | \texttt{every}(p)(s)\sete$.  When a datatype
has a mix of positive and nonpositive type parameters, the subtype
relation only holds for the positive ones.  For example, in the datatype
\begin{session}
  dt[T1, T2: TYPE, c: T1]: DATATYPE
   BEGIN
    c(a1: T1, a2: [T2 -> T1]): c?
   END dt
\end{session}
\texttt{T1} is positive and \texttt{T2} is not, so \texttt{dt[nat, nat,
0]} is a subtype of \texttt{dt[int, nat, 0]}, but is not a subtype of
\texttt{dt[nat, int, 0]}, nor is it a subtype of \texttt{dt[nat, nat, 1]}.

More complex datatypes lead to correspondingly more complex declarations;
for example, in the following contrived datatype
\begin{session}
  adt1[t1,t2: TYPE, c:t1]: DATATYPE
   BEGIN
    bottom: bottom?
    c1(a11:t1, a12: [t2 -> int]): c1?
    c2(a21:adt1, a22:[nat -> adt1], a23: list[adt1]): c2?
    c3(a31:[list[int] -> adt1],
       a32:[# a: adt1, b: [int -> adt1] #],
       a33:[adt1, [set[int] -> adt1]]) : c3?
   END adt1
\end{session}
the curried \texttt{every} is generated as follows:
\begin{session}
  every(p: PRED[t1])(a1: adt1):  boolean =
      CASES a1
        OF bottom: TRUE,
           c1(c11_var, c12_var): p(c11_var),
           c2(c21_var, c22_var, c23_var):
             every(p)(c21_var) AND
              every(every(p))(c22_var) AND every[adt1](every(p))(c23_var),
           c3(c31_var, c32_var, c33_var):
                  (FORALL (x1: list[int]): every(p)(c31_var(x1)))
              AND every(p)(a(c32_var))
              AND FORALL (x: int): every(p)(b(c32_var)(x))
              AND every(p)(c33_var`1)
              AND FORALL (x: set[int]): every(p)(c33_var`2(x))
        ENDCASES;
\end{session}
Note that this is only defined for predicates over \texttt{t1}, since
the occurrence of \texttt{t2} in the constructor specification for
\texttt{c2} is not positive.

As with record types, constructor selectors may be dependent.  Here is a
simple example.
\begin{session}
  depdt: DATATYPE
   BEGIN
    b: b?
    c(x: int, y: \setb{}z: int | z < x\sete): c?
   END depdt
\end{session}

\section{Datatype Subtypes}

The \texttt{WITH SUBTYPES} keyword introduces a set of subtype names.
These are useful, for example, in defining the nonterminals of a language.
For example, we might try to describe a simple typed lambda calculus:
\begin{eqnarray*}
T & ::= & B \;|\; T \rightarrow T \\
E & ::= & x \;|\; \lambda x:T.E \;|\; E(E)
\end{eqnarray*}
This is difficult to express using datatypes without subtypes, but is
reasonably straightforward with them:\footnote{\texttt{TYPE},
\texttt{LAMBDA}, and \texttt{VAR} are PVS keywords, so variants are used
here.}
\begin{session}
tlc: DATATYPE WITH SUBTYPES typ, expr
 BEGIN
 base_type(n:nat): base_type? : typ
 fun_type(dom, ran: typ): fun_type? : typ
 expr_var(n:nat): expr_var? : expr
 lambda_expr(lvar:(expr_var?), ltype: typ, lexpr: expr)
                            : lambda_expr? : expr
 application(fun, arg: expr): application? : expr
 END tlc
\end{session}
In addition to the usual generated declarations, this generates
\begin{session}
  typ((x: tlc)): boolean = base_type?(x) OR fun_type?(x);
  typ: TYPE = \setb{}x: tlc | base_type?(x) OR fun_type?(x)\sete
  expr((x: tlc)): boolean =
     expr_var?(x) OR lambda_expr?(x) OR application?(x);
  expr: TYPE =
     \setb{}x: tlc | expr_var?(x) OR lambda_expr?(x) OR application?(x)\sete
\end{session}
immediately after the declarations generated for the recognizers, so they
may be referenced in the accessor types.  Note that only a single
induction scheme is generated.  To induct over a particular subtype,
extend the property of interest to the entire datatype so that it returns
true for everything else.


\section{\texttt{CASES} Expressions}\label{cases-expressions}
\index{cases expressions}

The \texttt{CASES} expression uses a simple form of pattern-matching on
abstract datatypes.  Patterns are of the form $c(x_1,\ldots, x_n)$ where
$c$ is an $n$-ary constructor and $x_1,\ldots, x_n$ is a list of distinct
variables.  Patterns here are simple so that certain logical properties of
the expression are easy to check.  Patterns are not defined in the grammar
but in the type rules, since the notion of a variable or a constructor is
only defined in the type rules.

For example, if \texttt{x} is of type \texttt{stack}, the cases expression
\begin{pvsex}
  CASES x OF
    empty : FALSE,
    push(y, z) : even?(y) AND empty?(z)
  ENDCASES
\end{pvsex}
is \texttt{TRUE} if \texttt{x} is a singleton even integer, and otherwise is
false.  This expression can be translated into
\begin{pvsex}
  IF empty?(x)
     THEN FALSE
     ELSE LET (y, z) = (car(x), cdr(x))
           IN even?(y) AND empty?(z)
  ENDIF
\end{pvsex}

The \texttt{CASES} expression also allows an \texttt{ELSE} clause, which
comes last and covers all constructors not previously mentioned in a
pattern.  If the \texttt{ELSE} clause is missing, and not all constructors
have been mentioned, then a \emph{cases TCC}\index{cases
TCC}\index{TCC!cases} is generated which states that the expression is not
any of the missing elements.  For example, if the \texttt{x} above is
declared to be a subtype of \texttt{stack} in which \texttt{empty} is
excluded, then the \texttt{empty} case can safely be left out, and a \tcc\
will be generated that obligates the user to prove that \texttt{x} is not
\texttt{empty}.  There is a trade-off here between simpler specifications
and simpler verifications; if the \texttt{empty} case is left in, then
there is no obligation to prove, but the extra case clutters up the
specification, and can mislead the reader into thinking that the
\texttt{empty} case is possible.  In general, we feel that the
specification should be as perspicuous as possible, even if it means a
little more work behind the scenes.

%%% Local Variables: 
%%% mode: latex
%%% TeX-master: "language"
%%% End: 
