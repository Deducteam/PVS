% Document Type: LaTeX
% Master File: language.tex
\chapter{The Lexical Structure}\label{lexical}

PVS specifications are text files, each composed of a sequence of lexical
elements which in turn are made up of characters.  The lexical elements of
PVS are the identifiers, reserved words, special symbols, numbers,
whitespace characters, and comments.

Identifiers\index{identifiers} are composed of letters, digits, and the
characters \texttt{\_} or \texttt{?}; they must begin with a letter.  They
may be arbitrarily long, constrained only by the limits imposed by the
underlying Common Lisp system.  Identifiers are case-sensitive;
\texttt{FOO}, \texttt{Foo}, and \texttt{foo} are different identifiers.
PVS strings contain any ASCII character: to include a \texttt{"} in the
string, use \texttt{\char'134 "} and to include a \texttt{\char'134} use
\texttt{\char'134\char'134}.

\pvsbnf{bnf-lexical}{Lexical Syntax}

The reserved words\index{reserved words} are shown in
Figure~\ref{reserved-words}.  Unlike identifiers, they are not
case-sensitive.  In this document, reserved words are always displayed in
upper case.  Note that identifiers may have reserved words embedded in
them, thus \texttt{ARRAYALL} is a valid identifier and will not be
confused with the two embedded reserved words.  The meaning of the
reserved words are given in the appropriate sections; they are collected
here for reference.

\begin{figure}[tb]
{\smaller\tt
\begin{tabular}{|*{5}{p{1.03in}}|}\hline
\input{keywords}
\hline
\end{tabular}}
\caption{\pvs\ Reserved Words}\label{reserved-words}
\end{figure}

The special symbols\index{special symbols} are listed in
Figure~\ref{special-symbols}.  All of these symbols are separators; they
separate identifiers, numbers, and reserved words.

\begin{figure}[tb]
\begin{center}
{\small\tt
\begin{tabular}{|*{6}{@{\hspace*{.2in}}c@{\extracolsep{.5in}}}@{\hspace*{.25in}}|}\hline
\input{operator-table}
\hline
\end{tabular}}
\end{center}
\caption{\pvs\ Special Symbols}\label{special-symbols}
\end{figure}

The whitespace characters are space, tab, newline, return, and newpage;
they are used to separate other lexical elements.  At least one whitespace
character must separate adjacent identifiers, numbers, and reserved words.

Comments\index{comments} may appear anywhere that a whitespace character
is allowed.  They consist of the `\texttt{\%}'\index{\%@\texttt{\%}} character
followed by any sequence of characters and terminated by a newline.

The \emph{definable} symbols are shown in table~\ref{definable-symbols}.
These keywords and symbols may be given declarations.  Some of them have
declarations given in the prelude.\footnote{In particular,
\texttt{\char38}, \texttt{*}, \texttt{+}, \texttt{-}, \texttt{/},
\texttt{/=}, \texttt{<}, \texttt{<<}, \texttt{<=}, \texttt{<=>},
\texttt{=}, \texttt{=>}, \texttt{>}, \texttt{>=}, \texttt{AND},
\texttt{IFF}, \texttt{IMPLIES}, \texttt{NOT}, \texttt{O}, \texttt{OR},
\texttt{WHEN}, \texttt{XOR}, \texttt{\char94}, and \texttt{\char126} are
declared there.  Note that many of these are overloaded, for example,
\texttt{\char94} has three different definitions.}  Any of these may be
(re)declared any number of times, though this may lead to ambiguities.
Such ambiguities may be resolved by including the theory name, actual
parameters,  and possibly the type as a coercion.

Symbols that are binary infix (\hyperlink{Binop}{\emph{Binop}}), for
example \texttt{AND} and \texttt{+}, may be declared with any number of
arguments.  If they are declared with two arguments then they may
subsequently be used in prefix or infix form.  Otherwise they may only be
used in prefix form.  Similarly for unary operators, and the \texttt{IF}
operator, which may be used in \texttt{IF-THEN-ELSE-ENDIF} form if
declared with three arguments.

Note that when typing the operators \texttt{/\\} or \texttt{\\/} outside
of a specification, the backslash may need to be doubled (or in rare
cases, quadrupled).  This is because it is commonly used as an ``escape''
character, and the character following may be interpreted specially.

The symbol pairs \lit{[|} and \lit{|]}, \lit{(|} and \lit{|)}, and
\lit{$\{$|} and \lit{|$\}$} are available as outfix operators.  They are
declared using \lit{[||]}, \lit{(||)}, and \lit{$\{$||$\}$}, respectively.
For example, with the declaration \texttt{[||]:\ [bool, int -> int]} the
outfix term \texttt{[| TRUE, 0 |]} is equivalent to the prefix form
\texttt{[||](TRUE, 0)}.

\begin{figure}[tb]
\begin{center}
{\small\tt
\begin{tabular}{|*{6}{@{\hspace*{.2in}}c@{\extracolsep{.4in minus .4in}}}@{\hspace*{.2in}}|}\hline
\char35\char35 &  \char47\char47 &  \char60\char124 &  AND &  ORELSE &  \char123\char124\char124\char125\\
\char38 &  \char47\char61 &  \char61 &  ANDTHEN &  TRUE &  \char124\char45\\
\char40\char124\char124\char41 &  \char47\char92\char92 &  \char61\char61 &  FALSE &  WHEN &  \char124\char61\\
\char42 &  \char60 &  \char61\char62 &  IF &  XOR &  \char124\char62\\
\char42\char42 &  \char60\char60 &  \char62 &  IFF &  \char91\char93 &  \char126\\
\char43 &  \char60\char60\char61 &  \char62\char61 &  IMPLIES &  \char91\char124\char124\char93 &  \\
\char43\char43 &  \char60\char61 &  \char62\char62 &  NOT &  \char92\char92\char47 &  \\
\char45 &  \char60\char61\char62 &  \char62\char62\char61 &  O &  \char94 &  \\
\char47 &  \char60\char62 &  \char64\char64 &  OR &  \char94\char94 &  \\


\hline
\end{tabular}}
\end{center}
\caption{\pvs\ Definable Symbols}\label{definable-symbols}
\end{figure}
