% Document Type: LaTeX
% Master File: language.tex

\chapter{Declarations}\label{declarations}
\index{declaration|(pidx}

Entities of PVS are introduced by means of \emph{declarations}, which are
the main constituents of PVS specifications.  Declarations are used to
introduce types, variables, constants, formulas, judgements, conversions,
and other entities.  Most declarations have an \emph{identifier} and
belong to a unique theory.  Declarations also have a body which indicates
the \emph{kind} of the declaration and may provide a signature or
definition for the entity.  \emph{Top-level}
declarations\index{declaration!top-level} occur in the formal parameters,
the assertion section and the body of a theory.  \emph{Local}
declarations\index{declaration!local} for variables may be given, in
association with constant and recursive declarations and \emph{binding
expressions} (\eg\ involving \texttt{FORALL} or \texttt{LAMBDA}).
Declarations are ordered within a theory; earlier declarations may not
reference later ones.\footnote{Thus mutual recursion is not directly
supported.  The effect can be achieved with a single recursive function
that has an argument that serves as a switch for selecting between two or
more subexpressions.}

\index{exporting|(}\index{importing|(}
Declarations introduced in one theory may be referenced in another by
means of the \texttt{IMPORTING} and \texttt{EXPORTING} clauses.  The
\texttt{EXPORTING} clause of a theory indicates those entities that may be
referenced from outside the theory.  There is only one such clause for a
given theory.  The \texttt{IMPORTING} clauses provide access to the
entities exported by another theory.  There can be many \texttt{IMPORTING}
clauses in a theory; in general they may appear anywhere a top-level
declaration is allowed.  See Section~\ref{importings} for more details.
\index{importing|)}\index{exporting|)}

PVS allows the overloading\index{overloading} of declaration identifiers.
Thus a theory named \texttt{foo} may declare a constant \texttt{foo} and a
formula \texttt{foo}.  To support this \emph{ad hoc} overloading,
declarations are classified according to kind\index{declaration!kind}; in
PVS the primary kinds are \emph{type}\index{declaration!kind!type},
\emph{prop}\index{declaration!kind!prop},
\emph{expr}\index{declaration!kind!expr}, and
\emph{theory}\index{declaration!kind!theory}.  Type declarations are of
kind \emph{type}, and may be referenced in type declarations, actual
parameters, signatures, and expressions.  Formula declarations are of kind
\emph{prop}, and may be referenced in auto-rewrite declarations
(Section~\ref{auto-rewrite-decls}) or proofs (see the PVS Prover
Guide~\cite{PVS:prover}).  Variable, constant, and recursive definition
declarations are of kind \emph{expr}; these may be referenced in
expressions and actual parameters.  Newly introduced names need only be
unique within a kind, as there is no way, for example, to use an
expression where a type is expected.\footnote{There are a few exceptions,
for example the actual parameters of theories, since theories may be
instantiated with types or expressions.}

\pvsbnf{bnf-decls}{Declarations Syntax}
\index{syntax!declarations}
\pvsbnf{bnf-decls-aux}{Declarations Syntax (cont.)}
\index{syntax!declarations}

Declarations generally consist of an
\emph{identifier}\index{declaration!identifier}, an optional list of
\emph{bindings}\index{declaration!binding}, and a
\emph{body}\index{declaration!body}.  The body determines the kind of the
declaration, and the bindings and the body together determine the
signature and definition of the declared entity.  Multiple
declarations\index{declaration!multiple} may be given in compressed form
in which a common body is specified for multiple identifiers; for example
%
\begin{pvsex}
  x, y, z: VAR int
\end{pvsex}
In every case this is treated the same as the expanded form, thus the
above is equivalent to:
\begin{pvsex}
  x: VAR int
  y: VAR int
  z: VAR int
\end{pvsex}

In the rest of this chapter we describe declarations for types, variables,
constants, recursive definitions, macros, inductive and coinductive
definitions, formulas, judgements, conversions, libraries, and
auto-rewrites.  The declarations for theory parameters, importings,
exportings, and theory abbreviations are given in Chapter~\ref{theories}.
Figure~\ref{bnf-decls} gives the syntax for declarations.

\section{Type Declarations}\label{type-declarations}
\index{type declarations|(}

Type declarations are used to introduce new type names to the context.
There are four kinds of type declaration:

\begin{itemize}

\item \emph{uninterpreted type declaration}: \texttt{T:\ TYPE}
\index{uninterpreted type}\index{type!uninterpreted}

\item \emph{uninterpreted subtype declaration}: \texttt{S:\ TYPE FROM T}
\index{uninterpreted subtype}\index{type!uninterpreted subtype}

\item \emph{interpreted type declaration}: \texttt{T:\ TYPE =
int}\index{interpreted type}\index{type!interpreted}

\item \emph{enumeration type declarations}: \texttt{T:\ TYPE = \setb r,
g, b\sete} \index{enumeration types}\index{type!enumeration}

\end{itemize}

These type declarations introduce \emph{type names}\index{type!name}
that may be referenced in type expressions (see Section~\ref{types}).
They are introduced using one of the keywords
\keyword{TYPE}\index{type@\texttt{TYPE}},
\keyword{NONEMPTY\_TYPE}\index{type@\texttt{NONEMPTY\_TYPE}}, or
\keyword{TYPE+}\index{type+@\texttt{TYPE+}}.

\subsection{Uninterpreted Type Declarations}
\index{type!uninterpreted|(}

Uninterpreted types support abstraction by providing a means of
introducing a type with a minimum of assumptions on the type.  An
uninterpreted type imposes almost no constraints on an implementation of
the specification.  The only assumption made on an uninterpreted type
\texttt{T} is that it is disjoint from all other types, except for
subtypes of \texttt{T}.  For example,
\begin{pvsex}
  T1, T2, T3: TYPE
\end{pvsex}
%
introduces three new pairwise disjoint types.  If desired, further
constraints may be put on these types by means of axioms or assumptions
(see Section~\ref{formula-declarations} on
page~\pageref{formula-declarations}).

It should be emphasized that uninterpreted types are important in
providing the right level of abstraction in a specification.  Specifying
the type body may have the undesired effect of restricting the possible
implementations, and cluttering the specification with needless detail.

\index{type!uninterpreted|)}\index{uninterpreted type|)}


\subsection{Uninterpreted Subtype Declarations}
\index{uninterpreted subtype|(}

Uninterpreted subtype declarations are of the form
\begin{pvsex}
  s: TYPE FROM t
\end{pvsex}
\index{FROM@\texttt{FROM}}
This introduces an uninterpreted
\emph{subtype}\index{subtypes}\index{type!subtype} \texttt{s} of
the \emph{supertype}\index{supertype}\index{type!supertype}
\texttt{t}.  This has the same meaning as
\begin{pvsex}
  s_pred: [t -> bool]
  s: TYPE = (s_pred)
\end{pvsex}
%
in which a new predicate is introduced in the first line and the type
\texttt{s} is declared as a \emph{predicate} subtype in the second
line\footnote{This is described in Section~\ref{subtypes}
(page~\pageref{subtypes}).}.  No assumptions are made about uninterpreted
subtypes; in particular, they may or may not be empty, and two different
uninterpreted subtypes of the same supertype may or may not be disjoint.
Of course, if the supertypes themselves are disjoint, then the
uninterpreted subtypes are as well.

\index{uninterpreted subtype|)}

\subsection{Structural Subtypes}

PVS has support for structural subtyping for record and tuple types.  A
record type \texttt{S} is a structural subtype of record type \texttt{R}
if every field of \texttt{R} occurs in \texttt{S}, and similarly, a tuple
type \texttt{T} is a structural subtype of a tuple type forming a prefix
of \texttt{T}.  Section \ref{type-extensions} gives
examples, as \texttt{colored\_point} is a structural subtype of
\texttt{point}, and \texttt{R5} is a structural subtype of \texttt{R3}.
Structural subtypes are akin to the class hierarchy of object-oriented
systems, where the fields of a record can be viewed as the slots of a
class instance.  The PVS equivalent of setting a slot value is the
override expression (sometimes called update), and this works with
structural subtypes, allowing the equivalent of generic methods to be
defined.  Here is an example:
\begin{pvsex}
points: THEORY
BEGIN
 point: TYPE+ = [# x, y: real #]
END points

genpoints[(IMPORTING points) gpoint: TYPE <: point]: THEORY
BEGIN
 move(p: gpoint)(dx, dy: real): gpoint =
  p WITH [`x := p`x + dx, `y := p`y + dy]
END genpoints

colored_points: THEORY
BEGIN
 IMPORTING points
 Color: TYPE = {red, green, blue}
 colored_point: TYPE = point WITH [# color: Color #]
 IMPORTING genpoints[colored_point]
 p: colored_point
 move0: LEMMA move(p)(0, 0) = p
END colored_points
\end{pvsex}

The declaration for \texttt{gpoint} uses the structural subtype operator
\texttt{<:}.  This is analogous to the \texttt{FROM} keyword, which
introduces a (predicate) subtype.  This example also serves to explain
why we chose to separate structural and predicate subtyping.  If they
were treated uniformly, then \texttt{gpoint} could be instantiated with
the unit disk; but in that case the \texttt{move} operator would not
necessarily return a \texttt{gpoint}.  The TCC could not be generated
for the \texttt{move} declaration, but would have to be generated when
the \texttt{move} was referenced.  This both complicates typechecking,
and makes TCCs and error messages more inscrutable.  If both are
desired, simply include a structural subtype followed by a predicate
subtype, for example:
\begin{pvsex}
genpoints[(IMPORTING points) gpoint: TYPE <: point,
          spoint: TYPE FROM gpoint]: THEORY
\end{pvsex}
Now \texttt{move} may be applied to \texttt{gpoint}s, but if applied to a
\texttt{spoint} an unprovable TCC will result.

Structural subtypes are a work in progress.  In particular, structural
subtyping could be extended to function and datatypes.  And to have
real object-oriented PVS, we must be able to support a form of method
invocation.


\subsection{Empty and Singleton Record and Tuple Types}

Empty and singleton record and tuple types are now allowed in PVS.
Thus the following are valid declarations:
\begin{pvsex}
Tup0: TYPE = [ ]
Tup1: TYPE = [int]
Rec0: TYPE = [# #]
\end{pvsex}
Note that the space is important in the empty tuple type, as otherwise
it is taken to be an operator (the box operator).


\subsection{Interpreted Type Declarations}
\index{interpreted type declarations|(}\index{type!interpreted|(}

Interpreted type declarations are primarily a means for providing names
for type expressions.  For example,
\begin{pvsex}
  intfun: TYPE = [int -> int]
\end{pvsex}
%
introduces the type name \texttt{intfun} as an abbreviation for the type
of functions with integer domain and range.  Because PVS uses
\emph{structural equivalence}\index{structural equivalence} instead of
\emph{name equivalence}\index{name equivalence}, any type expression
\texttt{T} involving \texttt{intfun} is equivalent to the type expression
obtained by substituting \texttt{[int -> int]} for \texttt{intfun} in
\texttt{T}.  The available type expressions are described in
Chapter~\ref{types} on page~\pageref{types}.

Interpreted type declarations may be given
parameters.\index{parameterized type names} For example, the type of
integer subranges may be given as
\begin{pvsex}
  subrange(m, n: int): TYPE = \setb{}i:int | m <= i AND i <= n\sete
\end{pvsex}
and \texttt{subrange} with two integer parameters may subsequently be used
wherever a type is expected.  Any use of a parameterized type must include
all of the parameters, so currying of the parameters is not allowed.  Note
that \texttt{subrange} may be overloaded to declare a different type in
the same theory without any ambiguity, as long as the number or type of
parameters is different.

\index{type!interpreted|)}\index{interpreted type declarations|)}


\subsection{Enumeration Type Declarations}\label{enum-types}
\index{enumeration types|(}\index{type!enumeration|(}

Enumeration type declarations are of the form
\begin{pvsex}
  enum: TYPE = \setb{}e_1,\ldots, e_n\sete
\end{pvsex}
%
where the \texttt{e\_i} are distinct identifiers which are taken to
completely enumerate the type.  This is actually a shorthand for the
datatype specification
\begin{pvsex}
  enum: DATATYPE
    e_1: e_1?
         \(\vdots\)
    e_n: e_n?
  END enum
\end{pvsex}
%
explained in Chapter~\ref{adts}.  Because of this, enumeration types may
only be given as top-level declarations, and are \emph{not} type
expressions.  The advantage of treating them as datatypes is that the
necessary axioms are automatically generated, and the prover has built-in
facilities for handling datatypes.

\index{type!enumeration|)}\index{enumeration types|)}

\index{type declarations|)}


\subsection{Empty versus Nonempty Types}
\label{emptytypes}
\index{nonempty type}
\index{empty type}
\index{type!nonempty|(}\index{type!empty|(}

As noted before, PVS allows empty types, and the term \emph{type} refers
to either empty or nonempty types.  Constants declared to be of a given
type provide elements of the type, so the type must be nonempty or there
is an inconsistency.  Thus whenever a constant is declared, the system
checks whether the type is nonempty, and if it cannot decide that it is
nonempty it generates an \emph{existence TCC}.\index{existence
TCC}\index{TCC!existence} This is the simple explanation, but it is made
somewhat complicated by the considerations of formal parameters,
uninterpreted versus interpreted type declarations, explicit declarations
of nonemptiness, and
\keyword{CONTAINING}\index{CONTAINING@\texttt{CONTAINING}} clauses on type
declarationss, as well as a desire to keep the number of TCCs generated to
a minimum, while guaranteeing soundness.  The details are provided below.

First note that having variables range over an empty type causes no
difficulties,\footnote{If the type \texttt{T} is empty, then the following
two equivalences hold:
\begin{alltt}
  (FORALL (x: T): p(x)) IFF TRUE \quad \mbox{\textrm{and}} \quad (EXISTS (x: T): p(x)) IFF FALSE
\end{alltt}
}
so variable declarations and variable bindings never trigger the
nonemptiness check.

During typechecking, type declarations may indicate that the type is
nonempty, and constant declarations of a given type require that the type
be nonempty.  When a type is determined to be nonempty, it is marked as
such so that future checks of constants do not trigger more TCCs.  Below
we describe how type declarations are handled first for declarations in the
body of a theory, and then for type declarations that appear in the formal
parameters, as they require special handling.

\paragraph{Theory Body Type Declarations}

\begin{itemize}

\item Uninterpreted type or subtype declarations introduced with the
keyword \keyword{TYPE} may be empty.  Declaring a constant of that type
will lead to a TCC that is unprovable without further axioms.

\item Uninterpreted type declarations introduced with the keyword
\keyword{NONEMPTY\_TYPE}\index{nonempty_type@\keyword{NONEMPTY\_TYPE}}
or \keyword{TYPE+}\index{type+@\texttt{TYPE+}} are assumed to be nonempty.
Thus the type is marked nonempty.

\item Uninterpreted subtype declarations introduced with the keyword
\keyword{NONEMPTY\_TYPE} or \keyword{TYPE+} are assumed to be nonempty, as long as the
supertype is nonempty.  Thus the supertype is checked, and an existence
TCC is generated if the supertype is not known to be nonempty.  Then the
subtype is marked nonempty.

\item The type of an interpreted constant is nonempty, as the definition
provides a witness.

\item Interpreted type declarations introduced with the keyword
\keyword{TYPE} may be non\-emp\-ty, depending on the type definition.

\item Any interpreted type declaration with a \keyword{CONTAINING} clause
is marked nonempty, and the \keyword{CONTAINING} expression is typechecked
against the specified type.  In this case no existence TCC is generated,
since the \keyword{CONTAINING} expression is a witness to the type.  Of
course, other TCCs may be generated as a result of typechecking the
\keyword{CONTAINING} expression.

\end{itemize}

\paragraph{Formal Type Declarations}

Only uninterpreted (sub)type declarations may appear in the formal
parameters list.

\begin{itemize}

\item Formal type declarations introduced with the \texttt{TYPE} keyword may
be empty.  This is handled according to the occurrences of constant
declarations involving the type.

\item If there is a constant declaration of that type in the formal
parameter list, then no TCCs are generated, since
any instance of the theory will need to provide both the type and a
witness.  The type is marked nonempty in this case.

\item If the type declaration is a formal parameter and a constant is
declared whose type involves the type, but is not the type itself (for
example, if the formal theory parameters are \texttt{[t:\ TYPE, f:\ [t ->
t]]}), then a TCC may be generated, and a comment is added to the TCC
indicating that an assuming clause may be needed in order to discharge the
TCC.  This TCC will be generated only if an earlier constant declaration
hasn't already forced the type to be marked nonempty.  Note that there are
circumstances in which the formal type may be empty but the type
expression involving that type is nonempty.  This is discussed further
below.

\end{itemize}

\subsection{Checking Nonemptiness}\label{nonemptiness-check}
\index{type!nonempty}
The typechecker knows a type to be nonempty under the
following circumstances:
\begin{itemize}

\item The type was declared to be nonempty, using either the
\keyword{NONEMPTY\_TYPE}\index{nonempty_type@\keyword{NONEMPTY\_TYPE}} or
the synonymous \keyword{TYPE+}\index{type+@\texttt{TYPE+}} keyword.  If the
type is uninterpreted, this amounts to an assumption that the type is
nonempty.  If the type has a definition, then an existence TCC is
generated unless the defining type expression is known to be nonempty.

\item The type was declared to have an element using a
\keyword{CONTAINING}\index{CONTAINING@\texttt{CONTAINING}} expression.

\item A constant was declared for the type.  In this case an existence TCC
is generated for the first such constant, after which the type is marked
as nonempty.

\item It was marked as nonempty from an earlier check.

\end{itemize}

Once an unmarked type is determined to be nonempty, it is marked by the
typechecker so that later checks will not generate existence TCCs.  In
addition, the type components are marked as nonempty.  Thus the types that
make up a tuple type, the field types of a record type, and the supertype
of a subtype are all marked.

It is possible for two equivalent types to be marked differently, for
example:
\begin{pvsex}
  t1: TYPE = \setb{}x: int | x > 2\sete
  t2: TYPE = \setb{}x: int | x > 2\sete
  c1: t1
\end{pvsex}
only marks the first type (\texttt{t1}).  Hence, it is best to name your types and
to use those names uniformly.

\index{type!empty|)}
\index{type!nonempty|)}

\section{Variable Declarations}
\index{variables|(}\index{declaration!variables|(}

Variable declarations introduce new variables and associate a type with
them.  These are \emph{logical} variables, not program variables; they
have nothing to do with state---they simply provide a name and associated
type so that binding expressions and formulas can be succinct.
Variables may not be exported.  Variable
declarations also appear in binding expressions such as \texttt{FORALL} and
\texttt{LAMBDA}.  Such local declarations ``shadow'' any earlier
declarations.  For example, in
\begin{pvsex}
  x: VAR bool
  f: FORMULA (FORALL (x: int): (EXISTS (x: nat): p(x)) AND q(x))
\end{pvsex}
%
The occurrence of \texttt{x} as an argument to \texttt{p} is of type
\texttt{nat}, shadowing the one of type \texttt{int}.  Similarly, the
occurrence of \texttt{x} as an argument to \texttt{q} is of type
\texttt{int}, shadowing the one of type \texttt{bool}.

\index{variables|)}\index{declaration!variables|)}

\section{Constant Declarations}\label{constants}
\index{constants|(}\index{declaration!constants|(}

Constant declarations introduce new constants, specifying their type and
optionally providing a value.  Since PVS is a higher order logic, the term
\emph{constant} refers to functions and relations, as well as the usual
(0-ary) constants.  As with types, there are both \emph{uninterpreted} and
\emph{interpreted} \index{constants!interpreted}%
\index{constants!uninterpreted} constants.  Uninterpreted constants make
no assumptions, although they require that the type be nonempty (see
Section~\ref{nonemptiness-check}, page~\pageref{nonemptiness-check}).
Here are some examples of constant declarations:
\begin{pvsex}
  n: int
  c: int = 3
  f: [int -> int] = (lambda (x: int): x + 1)
  g(x: int): int = x + 1
\end{pvsex}
%
The declaration for \texttt{n} simply introduces a new integer constant.
Nothing is known about this constant other than its type, unless further
properties are provided by \texttt{AXIOM}s.  The other three constants are
interpreted.  Each is equivalent to specifying two declarations: \eg\
the third line is equivalent to
\begin{pvsex}
  f: [int -> int]
  f: AXIOM  f = (LAMBDA (x: int): x + 1)
\end{pvsex}
%
except that the definition is guaranteed to form a \emph{conservative
extension}\index{conservative extension} of the theory.  Thus the
theory remains consistent after the declaration is given if it was
consistent before.

The declarations for \texttt{f} and \texttt{g} above are two different ways to
declare the same function.  This extends to more complex arguments, for
example
\begin{pvsex}
  f: [int -> [int, nat -> [int -> int]]] =
     (LAMBDA (x: int): (LAMBDA (y: int), (z: nat): (LAMBDA (w: int):
       x * (y + w) - z)))
\end{pvsex}
%
is equivalent to
\begin{pvsex}
  f(x: int)(y: int, z: nat)(w: int): int = x * (y + w) - z
\end{pvsex}
%
This can be shortened even further if the variables are declared first:
\begin{pvsex}
  x, y, w: VAR int
  z: VAR nat
  f(x)(y,z)(w): int = x * (y + w) - z
\end{pvsex}
%
Finally, a mix of predeclared and locally declared variables is possible:
\begin{pvsex}
  x, y: VAR int
  f(x)(y,(z: nat))(w: int): int = x * (y + w) - z
\end{pvsex}
%
Note the parentheses around \texttt{z:\ nat}; without these, \texttt{y} would
also be treated as if it were declared to be of type \texttt{nat}.

A construct that is frequently encountered when subtypes are involved is
shown by this example
\begin{pvsex}
  f(x: \setb{}x: int | p(x)\sete): int = x + 1
\end{pvsex}
%
There are two useful abbreviations for this expression.  In the first, we
use the fact that the type \texttt{\setb{}x:\ int | p(x)\sete} is equivalent to
the type expression \texttt{(p)} when \texttt{p} has type \texttt{[int ->
bool]}, and we can write
\begin{pvsex}
  f(x: (p)): int = x + 1
\end{pvsex}
%
The second form of abbreviation basically removes the set braces and the
redundant references to the variable, though extra parentheses are
required:
\begin{pvsex}
  f((x: int | p(x))): int = x + 1
\end{pvsex}
%
Which of these forms to use is mostly a matter of taste; in general,
choose the form that is clearest to read for a given declaration.

Note that functions with range type \texttt{bool} are generally referred
to as \emph{predicates}, and can also be regarded as relations or sets.
For example, the set of positive odd numbers can be characterized by a
predicate as follows:
\begin{pvsex}
  odd: [nat -> bool] = (LAMBDA (n: nat): EXISTS (m: nat): n = 2 * m + 1)
\end{pvsex}
%
PVS allows an alternate syntax for predicates that encourages a
set-theoretic interpretation:
\begin{pvsex}
  odd: [nat -> bool] = \setb{}n: nat | EXISTS (m: nat): n = 2 * m + 1\sete
\end{pvsex}

\index{constants|)}

\section{Recursive Definitions}\label{recursive-definitions}
\index{recursive definitions|(}

Recursive definitions are treated as constant declarations, except that
the defining expression is required, and a \emph{measure}\index{measure
function} must be provided, along with an optional well-founded order
relation.\index{well-founded order releation} The same syntax for
arguments is available as for constant declarations; see the preceding
section.

PVS allows a restricted form of recursive definition; mutual
recursion\index{recursion!mutual}\index{mutual recursion} is not allowed,
and the function must be \emph{total},\index{total function} so that the
function is defined for every value of its domain.  In order to ensure
this, recursive functions must be specified with a
\emph{measure}\index{measure}, which is a function whose signature matches
that of the recursive function, but with range type the domain of the
order relation, which defaults to \texttt{<} on \texttt{nat} or
\texttt{ordinal}\index{ordinal}\index{type!ordinal}.  If the order
relation is provided, then it must be a binary relation on the range type
of the measure, and it must be well-founded; a \emph{well-founded} \tcc\
\index{well-founded TCC}\index{TCC!well-founded} is generated if the order
is not declared to be well-founded.

Here is the classic example of the
\texttt{factorial}\index{factorial@\texttt{factorial}} function:
%
\begin{pvsex}
  factorial(x: nat): RECURSIVE nat =
    IF x = 0 THEN 1 ELSE x * factorial(x - 1) ENDIF
    MEASURE (LAMBDA (x: nat): x)
\end{pvsex}
%
The measure is the expression following the \texttt{MEASURE} keyword (the
optional order relation follows a \texttt{BY} keyword after the
measure).  This definition generates a \emph{termination
TCC};\index{TCC!termination}\index{termination TCC} a proof obligation
which must be discharged in order that the function be well-defined.  In
this case the obligation is
%
\begin{pvsex}
  factorial_TCC2: OBLIGATION
    FORALL (x: nat): NOT x = 0 IMPLIES x - 1 < x
\end{pvsex}

It is possible to abbreviate the given \texttt{MEASURE} function by
leaving out the \texttt{LAMBDA} binding.  For example, the measure
function of the factorial definition may be abbreviated to:
\begin{pvsex}
  MEASURE x
\end{pvsex}
The typechecker will automatically insert a lambda binding corresponding
to the arguments to the recursive function if the measure is not already
of the correct type, and will generate a typecheck error if this process
cannot determine an appropriate function from what has been specified.

A termination \tcc\ is generated for each recursive occurrence of the
defined entity within the body of the definition.\footnote{Some of these
may be subsumed by earlier TCCs, and hence will not be displayed with the
\texttt{M-x show-tccs} command.}  It is obtained in one of two ways.  If a
given recursive reference has at least as many arguments provided as
needed by the measure, then the \tcc\ is generated by applying the measure
to the arguments of the recursive call and comparing that to the measure
applied to the original arguments using the order relation.  The
\texttt{factorial} \tcc\ is of this form.  The context of the occurrence
is included in the \tcc; in this case the occurrence is within the
\texttt{ELSE} part of an \texttt{IF-THEN-ELSE} so the negated condition is
an antecedent to the proof obligation.

If the reference does not have enough arguments available, then the
reference is actually given a \emph{recursive signature}\index{recursive
signature} derived from the recursive function as described below.  This
type constrains the domain to satisfy the measure, and the termination
\tcc\ is generated as a \emph{termination-subtype}
\tcc.\index{termination-subtype TCC}\index{TCC!termination-subtype}
Termination-subtype \tccs\ are recognized as such by the occurrence of the
order in the goal of the \tcc.  For example, we could define a
substitution function for terms as follows.
\begin{session}
  term: DATATYPE
  BEGIN
   mk_var(index: nat): var?
   mk_const(index: nat): const?
   mk_apply(fun: term, args: list[term]): apply?
  END term

  subst(x: (var?), y: term)(s: term): RECURSIVE term =
    (CASES s OF
      mk_var(i): (IF index(x) = i THEN y ELSE s ENDIF),
      mk_const(i): s,
      mk_apply(t, ss): mk_apply(subst(x, y)(t), map(subst(x, y))(ss))
     ENDCASES)
  MEASURE s BY <<
\end{session}
Now the first recursive occurrence of \texttt{subst} has all arguments
provided, so the termination TCC is as expected.  The second occurrence
does not have enough arguments.  The recursive signature of that
occurrence is
\begin{pvsex}
  [[(var?), term] -> [\setb{}z1: term | z1 << s\sete -> term]]
\end{pvsex}
Hence the signature of \texttt{subst(x, y)} is \texttt{[\setb{}z1:\ term | z1 <<
s\sete -> term]}, and map is instantiated to \texttt{map[\setb{}z1:\ term | z1 <<
s\sete, term]}, which leads to the TCC
\begin{pvsex}
 subst_TCC2: OBLIGATION
   FORALL (ss: list[term], t: term, s: term, x: (var?)):
     s = mk_apply(t, ss) IMPLIES every[term](LAMBDA (z: term): z << s)(ss);
\end{pvsex}
Note that this \texttt{map} instance could be given directly, just don't
make the mistake of providing \texttt{map[term, term]}, as this leads to a
TCC that says every \texttt{term} is \texttt{<<} \texttt{s}.
For the same reason, if the uncurried form of this definition is given,
then a lambda expression will have to be provided and the type will have
to include the measure, for example,
\begin{session}
   subst(x: (var?), y, s: term): RECURSIVE term =
     (CASES s OF
       mk_var(i): (IF index(x) = i THEN y ELSE s ENDIF),
       mk_const(i): s,
       mk_apply(t, ss): mk_apply(subst(x, y, t),
                                 map(LAMBDA (s1: \setb{}z: term|z<<s\sete):
                                       subst(x, y, s1))(ss))
      ENDCASES)
   MEASURE s BY <<
\end{session}
\renewcommand{\textfraction}{.1}

The recursive signature is generated based on the type of the recursive
function and the measure.  For curried functions, it may be that the
measure does not have the entire domain of the recursive function, but
only the first few.  For example, consider the measure for the function
\texttt{f}.
\begin{pvsex}
  f(r: real)(x, y: nat)(b: boolean): RECURSIVE boolean
    = ...
   MEASURE LAMBDA (r: real): LAMBDA (x, y: nat): x
\end{pvsex}
The type of the measure function is \texttt{[real -> [nat, nat -> nat]]},
which is a prefix of the function type.  In deriving the recursive
signature, the last domain type of the measure is constrained (using a
subtype) in the corresponding position of the recursive function type.  In
this case the recursive signature is
\begin{pvsex}
  [real -> [\setb{}z: [nat, nat] | z`1 < x\sete -> [boolean -> boolean]]]
\end{pvsex}
Note that the recursive signature is a dependent type that depends on the
arguments of the recursive function (\texttt{x} in this case), and hence
only applies within the body of the recursive definition.

The formal argument that typechecking the body of a recursive function
using the recursive signature is sound will appear in a future version of
the semantics manual, for now note that simple attempts to subvert this
mechanism do not work, as the following example illustrates.
\begin{pvsex}
  fbad: RECURSIVE [nat -> nat] = fbad
   MEASURE lambda (n: nat): n
\end{pvsex}
This leads an unprovable TCC.
\begin{pvsex}
  fbad_TCC1: OBLIGATION FORALL (x1: nat, x: nat): x < x1;
\end{pvsex}
The TCC results from the comaprison of the expected type \texttt{[nat ->
nat]} to the derived type \texttt{[\setb{}z:\ nat | z < x1\sete -> nat]}.  Remember
that in PVS domains of function types must be equal in order for the
function types to satisfy the subtype relation, and this is exactly what
the TCC states.

\pvstheory{f91-alltt}{Theory \texttt{f91}}{f91-alltt}
\index{f91@{\texttt{f91}}}

When a doubly recursive call is found, the inner recursive calls are
replaced by variables in the termination \tccs\ generated for the outer
calls.  For example, the theory of Figure~\ref{f91-alltt} generates the
termination TCC of Figure~\ref{f91-tcc}

\begin{figure}[ht]
\begin{session}
f91_TCC5: OBLIGATION
  FORALL (i: nat,
          v: [i1:
               \setb{}z: nat |
                        (IF z > 101 THEN 0 ELSE 101 - z ENDIF) <
                         (IF i > 101 THEN 0 ELSE 101 - i ENDIF)\sete ->
               \setb{}j: nat | IF i1 > 100 THEN j = i1 - 10 ELSE j = 91 ENDIF\sete]):
    NOT i > 100 IMPLIES
     IF i > 100 THEN v(v(i + 11)) = i - 10 ELSE v(v(i + 11)) = 91 ENDIF;
\end{session}
\caption{Termination TCC for \texttt{f91}}\label{f91-tcc}
\end{figure}
where the inner calls to \texttt{f91} have been replaced by the
higher-order variable \texttt{v}, with the recursive signature as shown.
Since the obligation forces us to prove the termination condition for all
functions whose type is that of \texttt{f91}, it will also hold for
\texttt{f91}.  This example also illustrates the use of dependent types,
discussed in Section~\ref{dependent-types}.

\pvstheory{ackerman-alltt}{Theory \texttt{ackerman}}{ackerman-alltt}
\index{ackerman@{\texttt{ackerman}}}

\renewcommand{\textfraction}{.01}

In some cases the natural numbers are not a convenient measure; PVS
also provides the \texttt{ordinal}s, which allow recursion with measures up
to $\varepsilon_0$.  This is primarily useful in handling
lexicographical orderings.  For example, in the definition of the
Ackerman function in Figure~\ref{ackerman-alltt},\footnote{There are
ways of specifying \texttt{ackerman} using higher-order functionals, in
which case the measure is again on the natural numbers.} there are two
termination \tccs\ generated (along with a number of subtype \tccs).
The first termination \tcc\ is
\begin{pvsex}
  ack_TCC2:
    OBLIGATION
      (FORALL m, n:
        NOT m = 0 AND n = 0 IMPLIES ackmeas(m - 1, 1) < ackmeas(m, n))
\end{pvsex}
%
and corresponds to the first recursive call of \texttt{ack} in the body of
\texttt{ack}.  In this occurrence, it is known that \texttt{m $\neq$ 0}
and \texttt{n = 0}.  The remaining expression says that the measure
applied to the arguments of the recursive call to \texttt{ack} is less
than the measure applied to the initial arguments of \texttt{ack}.  Note
that the \texttt{<} in this expression is over the \texttt{ordinal}s, not
the \texttt{real}s.

\index{recursive definitions|)}


\section{Macros}\label{macro-declarations}
\index{macros|(}

There are some definitions that are convenient to use, but it's preferable
to have them expanded whenever they are referenced.  To some extent this
can be accomplished using auto-rewrites in the prover, but rewriting is
restricted.  In particular terms in types or actual parameters are not
rewritten; \texttt{typepred} and \texttt{same-name} must be used.  These
both require the terms to be given as arguments, making it difficult to
automate proofs.

The \texttt{MACRO} declaration is used to indicate definitions that are
expanded at typecheck time.  Macro declarations are normal constant
declarations, with the \texttt{MACRO} keyword preceding the
type.\footnote{This is similar to the \texttt{==} form of E\textsc{hdm}.}
For example, after the declaration
\begin{pvsex}
  N: MACRO nat = 100
\end{pvsex}
any reference to \texttt{N} is now automatically replaced by \texttt{100},
including such forms as \texttt{below[N]}.

Macros are not expanded until they have been typechecked.  This is because
the name overloading allowed by PVS precludes expanding during parsing.
TCCs are generated before the definition is expanded.
\index{macros|)}

% Master File: language.tex
\section{Inductive and Coinductive Definitions}
\label{inductive-definitions}
\index{inductive definition|(}

\emph{Inductive} definitions~\cite{Aczel:Handbook} are used frequently in
mathematics.  In general, some rules are given that generate elements of a
set, and the inductively defined set is the smallest set that contains
those elements.  The obvious example of an inductive definition is the
natural numbers, where the rules are given by Peano's axioms, with the
induction scheme ensuring that the natural numbers are the smallest set
containing $1$ and the successor of any natural number.  Language
definitions are another example.  Most logics have a notion of
\emph{formulas}, and these are usually defined inductively.

Paulson~\cite{paulson-fixedpoint} notes that this is simply a \emph{least
fixedpoint} with respect to a given domain of elements and a set of rules,
which is well-defined if the rules are \emph{monotonic}, by the
well known Knaster-Tarski theorem.  From this perspective, the greatest
fixedpoint also exists and corresponds to \emph{coinductive} definitions.
Inductive and coinductive definitions are similar to recursive
definitions, in that they have induction principles, and both must satisfy
additional constraints to guarantee that they are well defined.

We will describe inductive definitions first, as they are more familiar.
The even integers provide a simple example of an inductive
definition:\footnote{This is an alternative to the more traditional
definition of \texttt{even?} in the prelude.}
\begin{pvsex}
  even(n: int): INDUCTIVE bool = n = 0 OR even(n - 2) OR even(n + 2)
\end{pvsex}
With this definition, it is easy to prove, for example, that \texttt{0} or
\texttt{1000} are even, simply by expanding the definition enough
times.\footnote{In the latter case, \texttt{(apply (repeat (then (expand
"even") (flatten) (assert))))} is a good strategy to use, though it should
be used with care since it does not terminate on \texttt{even} applied to
anything other than an even numeral.}  More is needed, however, in proving
general facts, such as if $n$ is even, then $n+1$ is not even.  To deal
with these, we need a means of stating that an integer is even iff it is
so as a result of this definition.  In PVS, this is accomplished by the
automatic creation of two induction schemas, that may be viewed using the
\texttt{M-x~prettyprint-expanded} command:
\begin{session}
  even_weak_induction: AXIOM
    FORALL (P: [int -> boolean]):
      (FORALL (n: int): n = 0 OR P(n - 2) OR P(n + 2) IMPLIES P(n)) IMPLIES
       (FORALL (n: int): even(n) IMPLIES P(n));

  even_induction: AXIOM
    FORALL (P: [int -> boolean]):
      (FORALL (n: int):
         n = 0 OR even(n - 2) AND P(n - 2) OR even(n + 2) AND P(n + 2)
          IMPLIES P(n))
       IMPLIES (FORALL (n: int): even(n) IMPLIES P(n));
\end{session}
The weak induction axiom states that if \texttt{P} is another predicate
that satisfies the \texttt{even} form, then any \texttt{even} number
satisfies \texttt{P}.  Thus \texttt{even} is the smallest such \texttt{P}.
The second (strong) axiom allows the \texttt{even} predicate to be carried
along, which can make proofs easier.  These axioms are used by the
\texttt{rule-induct} strategy described in the Prover
Guide~\cite{PVS:prover}.

Inductive definitions are predicates, hence must be functions with
eventual range type \texttt{boolean}.  For example, in
\begin{session}
  f1(n,m:int) INDUCTIVE int = n
  f2(n,m:int)(x,y:int)(z:int): INDUCTIVE [int,int,int -> bool] =
      LAMBDA (a,b,c:int): n = m IMPLIES f2(n,m)(x,y)(z)(a,b,c)
\end{session}
\texttt{f1} is illegal, while \texttt{f2} returns a boolean value if
applied to enough arguments, hence is valid.

To be monotonic, every occurrence of the definition within the defining
body must be \emph{positive}.\index{positive occurrence} For this we need
to define the parity of an occurrence of a term in an expression $A$: If a
term occurs in $A$ with a given parity, then the occurrence retains its
parity in \texttt{$A$ AND $B$}, \texttt{$A$ OR $B$}, \texttt{$B$ IMPLIES
$A$}, \texttt{FORALL y:$A$}, \texttt{EXISTS y:$A$}, and reverses it in
\texttt{$A$ IMPLIES $B$} and \texttt{NOT $A$}.  Any other occurrence is of
unknown parity.

The parity of the inductive definition in the definition body is checked,
and if some occurrence of the definition is negative, a type error is
generated.  If some occurrence is of unknown parity, then a
\emph{monotonicity TCC}\index{TCC!monotonicity}\index{monotonicity TCC} is
generated.  For example, given the declarations
\begin{session}
  f: [nat, bool -> bool]
  G(n:nat): INDUCTIVE bool =
    n = 0 OR f(n, G(n-1))
\end{session}
the monotonicity TCC has the form
\begin{session}
  (FORALL (P1: [nat -> boolean], P2: [nat -> boolean]):
     (FORALL (x: nat): P1(x) IMPLIES P2(x))
         IMPLIES
       (FORALL (x: nat):
          x = 0 OR f(x, P1(x - 1)) IMPLIES x = 0 OR f(x, P2(x - 1))));
\end{session}

Inductive definitions act as constants for the most part, so they may be
expanded or used as rewrite rules in proofs.  However, they are not usable
as auto-rewrite rules, as there is no easy way to determine when to stop
rewriting.

To provide induction schemes in the most usable form, they are generated
as follows.  First, the variables in the definition are partitioned into
fixed\index{fixed inductive variable} and non-fixed variables.  For
example, in the transitive-reflexive closure
\begin{pvsex}
  TC(R)(x, y) : INDUCTIVE bool =
     R(x, y) OR (EXISTS z: TC(R)(x, z) AND TC(R)(z, y))
\end{pvsex}
\texttt{R} is fixed since every occurrence of \texttt{TC} has \texttt{R}
as an argument in exactly the same position, whereas \texttt{x} and
\texttt{y} are not fixed.  The induction is then over predicates $P$ that
take the non-fixed variables as arguments.  If the inductive definition is
defined for variable $V$ partitioned into fixed variables $F$, and
non-fixed variables $N$, the general form of the (weak) induction scheme
is
\begin{session}
  FORALL (\(F\), \(P\)):
   (FORALL (\(N\)):
     \emph{inductive_body}(\(N\))\([P/\emph{def}]\) IMPLIES \(P\)(\(N\)))
      IMPLIES
     (FORALL (\(N\)): \emph{def}(\(V\)) IMPLIES \(P\)(\(N\)))
\end{session}
In the case of \texttt{TC}, this becomes
\begin{session}
  TC_weak_induction: AXIOM
        (FORALL (R: relation, P: [[T, T] -> boolean]):
           (FORALL (x: T, y: T):
              R(x, y) OR (EXISTS z: (P(x, z) AND P(z, y))) IMPLIES P(x, y))
               IMPLIES (FORALL (x: T, y: T): TC(R)(x, y) IMPLIES P(x, y)));
\end{session}

\index{coinductive definitions(|}
Coinductive definitions have the same form as inductive definitions, but
are introduced with the keyword \texttt{COINDUCTIVE}, and generate the
greatest fix point, rather than the least fix point.  The monotonicity
conditions are the same, but the coinduction axioms reverse some of the
implications.  Thus the general form of the (weak) coinduction scheme is
\begin{session}
  FORALL (\(F\), \(P\)):
   (FORALL (\(N\)):
     \(P\)(\(N\)) IMPLIES \emph{coinductive_body}(\(N\))\([P/\emph{def}]\))
      IMPLIES
     (FORALL (\(N\)): \(P\)(\(N\)) IMPLIES \emph{def}(\(V\)))
\end{session}

As noted earlier, inductive and coinductive definitions are really
fixedpoint definitions.  For example, the theory in
Figure~\ref{inductive-fixpoints} shows that an
inductive definition is a least fixedpoint, a coinductive definition is a
greatest fixpoint, an inductively defined set is a subset of a
coindutively defined set, and, if the universe contains a non-wellfounded
element, then the coinductively defined set is strictly larger.  These
results all build on the definitions in  the \texttt{mucalculus} theory of
the prelude.

{\begin{figure}[htb]\begin{boxedminipage}{\textwidth}%
{\smaller\smaller\begin{alltt}
inductive_fixpoint: THEORY
 BEGIN
  N: TYPE+
  n, m: VAR N
  0: N
  S: [N -> N]
  Sax1: AXIOM 0 /= S(n)
  Sax2: AXIOM S(m) = S(n) => m = n
  % Assume a non-wellfounded element
  nwf_exists: AXIOM EXISTS n: n = S(n)

  Nind(n):     INDUCTIVE bool = n = 0 OR Nind(S(n))
  Ncoind(n): COINDUCTIVE bool = n = 0 OR Ncoind(S(n))

  % NN is the predicate transformer corresponding to the (co)inductive defs
  NN(p: pred[N])(n): bool = n = 0 OR p(S(n))

  % These use the lfp and gfp defs from the prelude mucalculus theory
  ind_lfp: FORMULA Nind = lfp(NN)
  coind_gfp: FORMULA Ncoind = gfp(NN)

  % Repeat Nind_weak_induction, which is proved from lfp_induction
  Nind_weak_induction_repeated: FORMULA 
    FORALL (P: [N -> boolean]):
      (FORALL n: n = 0 OR P(S(n)) IMPLIES P(n)) IMPLIES
       (FORALL n: Nind(n) IMPLIES P(n));

  % Inductive definitions are a subset of coinductive
  ind_sub_co: FORMULA Nind(n) => Ncoind(n)

  % Because there is a non-wellfounded element, we can show that
  % the coinductive set is larger.
  co_has_more: FORMULA EXISTS n: Ncoind(n) & NOT Nind(n)
 END inductive_fixpoint
\end{alltt}}\end{boxedminipage}%
\caption{Inductive definitions and fixpoints}\label{inductive-fixpoints}\end{figure}}

\index{coinductive definition|)}
\index{inductive definition|)}


\section{Formula Declarations}\label{formula-declarations}
\index{formula declarations|(}\index{declaration!formulas|(}

Formula declarations introduce \emph{axioms}\index{axioms},
\emph{assumptions}\index{assumptions}, \emph{theorems}\index{theorems},
and \emph{obligations}\index{obligations}.  The identifier associated with
the declaration may be referenced in auto-rewrite declarations (see
Section~\ref{auto-rewrite-decls} and in proofs (see the \texttt{lemma} command
in the PVS Prover Guide~\cite{PVS:prover}).  The expression that makes up
the body of the formula is a boolean expression.  Axioms, assumptions, and
obligations are introduced with the keywords \texttt{AXIOM},
\texttt{ASSUMPTION}, and \texttt{OBLIGA\-TION}, respectively.  Axioms may
also be introduced using the keyword \texttt{POSTULATE}\index{postulate}.
In the prelude postulates are used to indicate axioms that are provable by
the decision procedures, but not from other axioms.  Theorems may be
introduced with any of the keywords
\texttt{CHALLENGE}\index{claim@{\texttt{CHALLENGE}}},
\texttt{CLAIM}\index{claim@{\texttt{CLAIM}}},
\texttt{CONJECTURE}\index{conjecture@{\texttt{CONJECTURE}}},
\texttt{COROLLARY}\index{corollary@{\texttt{COROLLARY}}},
\texttt{FACT}\index{fact@{\texttt{FACT}}},
\texttt{FORMULA}\index{formula@{\texttt{FORMULA}}},
\texttt{LAW}\index{law@{\texttt{LAW}}},
\texttt{LEMMA}\index{lemma@{\texttt{LEMMA}}},
\texttt{PROPOSITION}\index{proposition@{\texttt{PROPOSITION}}},
\texttt{SUBLEMMA}\index{sublemma@{\texttt{SUBLEMMA}}}, or
\texttt{THEOREM}\index{theorem@{\texttt{THEOREM}}}.

Assumptions are only allowed in assuming clauses (see
Section~\ref{assuming}).  Obligations are generated by the system for
\tccs, and cannot be specified by the user.  Axioms are treated
specially when a proof is analyzed, in that they are not expected to
have an associated proof.  Otherwise they are treated exactly like
theorems.  All the keywords associated with theorems have the same
semantics, they are there simply to allow for greater diversity in
classifying formulas.

Formula declarations may contain free variables\index{free variables}, in
which case they are equivalent to the universal closure\index{universal
closure} of the formula.\footnote{The universal closure of a formula is
obtained by surrounding the formula with a \texttt{FORALL} binding
operator whose bindings are the free variables of the formula.  For
example, the universal closure of \texttt{p(x,y) => q(z)} is
\texttt{(FORALL x,y,z:\ p(x,y) => q(z))} (assuming \texttt{x}, \texttt{y}
and \texttt{z} resolve to variables).} In fact, the prover actually uses
the universal closure when it introduces a formula to a proof.  Formula
declarations are the only declarations in which free variables are
allowed.

\index{declaration!formulas|)}\index{formula declarations|)}

% Document Type: LaTeX
% Master File: language.tex
\section{Judgements}
\label{judgements}\index{judgements|(}

The facility for defining predicate subtypes is one of the most useful
features provided by PVS, but it can lead to a lot of redundant TCCs.
\emph{Judgements}\footnote{We prefer this spelling, though many spell
checkers do not.} provide a means for controlling this by allowing
properties of operators on subtypes to be made available to the
typechecker.  There are several kinds of judgements available in PVS\@. 
Most of them indicate that an expression belongs to a given type, but the
\emph{subtype judgement} indicates that two types are in the subtype
relation.
\begin{description}
\item[Number judgement] - 
\item[Name judgement]
\item[Application judgement]
\item[Recursive judgement]
\item[Expression judgement]
\item[Subtype judgement]
\end{description}


The \emph{constant judgement}\index{constant judgement} states that a
particular constant (or number) has a type more specific than its declared
type.  The \emph{subtype judgement}\index{subtype judgement} states that
one type is a subtype of another.

\subsection{Number and Name Judgements}
\index{number-name-judgement}

Number and name judgements 

There are two kinds of constant judgements.  The simpler kind 
states that a constant or number belongs to a type different than its
declared type.\footnote{Remember that all numbers are implicitly declared to
be of type \texttt{real}.}  For example, the constant judgement
declaration
\begin{pvsex}
  JUDGEMENT c, 17 HAS_TYPE (prime?)
\end{pvsex}
simply states that the constant \texttt{c} and the number \texttt{17} are
both prime numbers.  This declaration leads to the TCC formulas
\texttt{prime?(c)} and \texttt{prime?(17)}, but in any context in which
this declaration is visible, the use of \texttt{c} or \texttt{17} where a
prime is expected will not generate TCCs.  Thus no TCCs are generated for
the formula \texttt{F} in
\begin{pvsex}
  RP: [(prime?), (prime?) -> bool]
  F: FORMULA RP(c, 17) IMPLIES RP(17, c)
\end{pvsex}

The second kind of constant judgement is for functions; argument types are
provided and the judgement states that when the function is applied to
arguments of the given types, then the result has the type following the
\texttt{HAS\_TYPE} keyword.  Here is an example that illustrates the need
for this kind of judgement:
\begin{pvsex}
  x, y: VAR real
  f(x,y): real = x*x - y*y
  n: int = IF f(1,2) > 0 THEN f(4,3) ELSE f(3,2) ENDIF
\end{pvsex}
This leads to two TCCs:
\begin{pvsex}
  n_TCC1: OBLIGATION
    f(1, 2) > 0 IMPLIES
      rational_pred(f(4, 3)) AND integer_pred(f(4, 3))
  n_TCC2: OBLIGATION
    NOT f(1, 2) > 0 IMPLIES
      rational_pred(f(3, 2)) AND integer_pred(f(3, 2))
\end{pvsex}
The problem here is that although we know that \texttt{f} is closed under
the integers, the typechecker does not.  If \texttt{f} is heavily used,
dealing with these TCCs becomes cumbersome.  We can try the \emph{ad hoc}
solution of adding new overloaded declarations for \texttt{f}:
\begin{pvsex}
  i, j: VAR nat
  f(i, j): int = f(i, j)
\end{pvsex}
But now proofs require an extra definition expansion, and such overloading
leads to confusion.\footnote{This is one of the motivations for providing
the \texttt{M-x~show-expanded-sequent} command.}  A more elegant solution
is to use a judgement declaration:
\begin{pvsex}
  f_int_is_int: JUDGEMENT f(i, j: int) HAS\_TYPE int
\end{pvsex}
This generates the TCC
\begin{pvsex}
  f_int_is_int: FORALL (x:int, y:int):
                  rational_pred(f(x, y)) AND integer_pred(f(x, y))
\end{pvsex}
But now the declaration of \texttt{n} given above generates \emph{no}
TCCs, as the typechecker ``knows'' that \texttt{f} is closed on the
integers.  Note that this is different than the simple judgement
\begin{pvsex}
  f_int: JUDGEMENT f HAS\_TYPE [int, int -> int]
\end{pvsex}
In this case, the TCC generated is unprovable:
\begin{pvsex}
  f_int: OBLIGATION
    ((FORALL (x: real): rational_pred(x) AND integer_pred(x)) AND
      (FORALL (x: real): rational_pred(x) AND integer_pred(x)))
     AND
     (FORALL (x1: [real, real]):
        rational_pred(f(x1)) AND integer_pred(f(x1)));
\end{pvsex}
A warning is generated when simple constant judgements are declared to be
of a function type.\footnote{Earlier versions of PVS simply interpreted
this form as a closure condition, but this is less flexible.}  In
addition, this judgement will not help with the declaration \texttt{n}
above; it can only be used in higher-order functions, for example:
\begin{pvsex}
  F: [[int, int -> int] -> bool]
  FF: FORMULA F(f)
\end{pvsex}

The arguments for a function judgement follow the syntax for function
declarations; so a curried function may be given multiple judgements:
\begin{pvsex}
  f(x, y: real)(z: real): real
  f_closed: JUDGEMENT f(x, y: nat)(z: int) HAS\_TYPE int
  f2_closed: JUDGEMENT f(x, y: int) HAS\_TYPE [real -> int]
\end{pvsex}

If a constant judgement declaration specifies a name, it must refer to a
unique constant and its type must be compatible with the type expression
following the \texttt{HAS\_TYPE} keyword.  If it is a number, then its
type must be compatible with the \texttt{number} type.

Constant judgements generally lead to TCCs.  If no TCC is generated, then
the judgement is not actually needed, and a warning to this effect is
produced.  Simple (non-functional) constant judgements generate TCCs
indicating that the constant belongs to the specified type.  Constant
function judgements generate TCCs that reflect closure conditions.

The judgement facility cannot be used to remove all redundant TCCs; the
variables used for arguments must be unique, and full expressions may not
be included.  Hence the following are not legal:
\begin{pvsex}
  x: VAR real
  x_times_x_is_nonneg: JUDGEMENT *(x, x) HAS\_TYPE nonneg_real
  c: real
  x_times_c_is_even: JUDGEMENT *(x, c) HAS\_TYPE (even?)
\end{pvsex}


\subsection{Subtype Judgements}
\index{subtype judgement}

The subtype judgement is used to fill in edges of the subtype graph that
otherwise are unknown to the typechecker.  For example, consider the
following declarations:
\begin{pvsex}
  nonzero_real: NONEMPTY_TYPE = \setb{}r: real | r /= 0\sete CONTAINING 1
  rational: NONEMPTY_TYPE FROM real
  nonneg_rat: NONEMPTY_TYPE = \setb{}r: rational | r >= 0\sete CONTAINING 0
  posrat: NONEMPTY_TYPE = \setb{}r: nonneg_rat | r > 0\sete  CONTAINING 1
  /: [real, nonzero_real -> real]
\end{pvsex}
For \texttt{r} of type \texttt{real} and \texttt{q} of type
\texttt{posrat}, the expression \texttt{r/q} leads to the TCC \texttt{q
/= 0}.  One solution, if \texttt{q} is a constant, is to use a constant
judgement as described above.  But if there are many constants involving
the type \texttt{posrat}, this requires a lot of judgement declarations,
and does not help at all for variables or compound expressions.  The
subtype judgement solves this by stating that \texttt{posrat} is a subtype
of \texttt{nzrat}.  Another subtype judgement states that \texttt{nzrat}
is a subtype of \texttt{nzreal}:
\begin{pvsex}
  JUDGEMENT posrat SUBTYPE_OF nzrat
  JUDGEMENT nzrat SUBTYPE_OF nzreal
\end{pvsex}
With these judgements, TCCs will not be generated for any denominator that
is of type \texttt{posrat}.  With the (prelude) judgement declarations
\begin{pvsex}
  nnrat_plus_posrat_is_posrat:   JUDGEMENT +(nnx, py) HAS_TYPE posrat
  posrat_times_posrat_is_posrat: JUDGEMENT *(px, py)  HAS_TYPE posrat
\end{pvsex}
not only are there no TCCs generated for \texttt{r/q}, but none are
generated for \texttt{r/(q + 2)}, \texttt{r/((q + 2) * q)}, etc.

Given a subtype judgement declaration of the form
\begin{pvsex}
  JUDGEMENT S SUBTYPE_OF T
\end{pvsex}
it is an error if \texttt{S} is already known to be a subtype of
\texttt{T}, or if they are not compatible.  Otherwise, \texttt{T} must be
of the form \texttt{\setb{}x:\ ST | p(x)\sete}, where \texttt{ST} is the least
compatible type of \texttt{S} and \texttt{T}, and a TCC will be generated
of the form \texttt{FORALL (x:S): p(x)}.  Remember that subtyping on
functions only works on range types, so the subtype judgement
\begin{pvsex}
  JUDGEMENT [nat -> nat] SUBTYPE_OF [int -> int]
\end{pvsex}
leads to the unprovable TCC
\begin{pvsex}
FORALL (x1:nat, y1:int): y1 >= 0 AND TRUE
\end{pvsex}

\subsection{Judgement Processing}

When a judgement declaration is typechecked, TCCs are generated as
explained above and the judgement is added to the current context for use
in typechecking expressions.  The typechecker typechecks expressions in
two passes; in the first pass it simply collects possible types for
subexpressions, and in the second pass it recursively tries to determine a
unique type based on the expected type, and generates TCCs accordingly;
this is where judgements are used.  If the expression is a constant (name
or number), then all non-functional judgements are collected for that
constant and used to generate a minimal TCC relative to the expected type.

If it is an application whose operator is a name, then functional
judgements of the corresponding arity are collected for the operator, and
those judgements for which the application arguments are all known to be
of the corresponding judgement argument types are extracted, and a minimal
TCC is generated from these.

In addition to inhibiting the generation of TCCs during typechecking,
judgements are also important to the prover; they are used explicitly in
the \texttt{typepred} command, and implicitly in \texttt{assert}, where
the judgement type information is provided to the ground decision
procedures.

Subtype judgements are used in determining when one type is a subtype of
another, which is tested frequently during typechecking and proving,
including in the test on argument types described above.

\index{judgements|)}


% Document Type: LaTeX
% Master File: language.tex
\section{Conversions}
\label{coercion-decls}\index{conversions|(}

Conversions are functions that the typechecker can insert automatically
whenever there is a type mismatch.  They are similar to the implicit
coercions for converting integers to floating point used in many
programming languages.  PVS provides some builtin conversions in the
prelude, but conversions may also be provided by the user using
\emph{conversion declarations}.  A conversion declaration consists of the
keyword \texttt{CONVERSION}, optionally followed by `\texttt{+}' or
`\texttt{-}' and an expression.  \texttt{CONVERSION+} is equivalent to
\texttt{CONVERSION}.  The expression must be of type a (subtype of) a
function type, where the domain and range are not compatible.  This is
because conversions are only triggered when there would otherwise be a
type error, and compatible types may lead to unproveable TCCs, but not to
type errors.  Judgements are the proper way to control the generation of
TCCs, see Section~\ref{judgements} for details.

\subsection{Conversion Examples}
\label{conversion-examples}

Here is a simple example.
\begin{pvsex}
  c: [int -> bool]
  CONVERSION c
  two: FORMULA 2
\end{pvsex}
Here, since formulas must be of type boolean, the typechecker
automatically invokes the conversion and changes the formula to
\texttt{c(2)}.  This is done internally, and is only visible to the user
on explicit command\footnote{The \texttt{M-x~prettyprint-expanded} command.}
and in the proof checker.

A more complex conversion is illustrated in the following example.
\begin{pvsex}
  g: [int -> int]
  F: [[nat -> int] -> bool]
  F_app: FORMULA F(g)
\end{pvsex}
As this stands, \texttt{F\_app} is not type-correct, because a function of
type \texttt{[int -> int]} is supplied where one of type \texttt{[nat ->
int]} is required, and PVS requires equality on domain types for function
types to be compatible.  However it is clear that \texttt{g} naturally
induces a function from \texttt{nat} to \texttt{int} by simply restricting
its domain.  Such a domain restriction is achieved by the
\texttt{restrict} conversion\index{restrict
conversion}\index{conversion!restrict} that is defined in the PVS prelude
as follows:
\begin{session}
  restrict [T: TYPE, S: TYPE FROM T, R: TYPE]: THEORY
   BEGIN
    f: VAR [T -> R]
    s: VAR S
    restrict(f)(s): R = f(s)
    CONVERSION restrict
   END restrict
\end{session}
The construction \texttt{S: TYPE FROM T} specifies that the actual
parameter supplied for \texttt{S} must be a subtype of the one supplied
for \texttt{T}.  The specification states that \texttt{restrict(f)} is a
function from \texttt{S} to \texttt{R} whose values agree with \texttt{f}
(which is defined on the larger domain \texttt{T}).  Using this approach,
a type correct version of \texttt{F\_app} can be written as
\texttt{F(restrict[int,nat,int](g))}.  This provides the convenience of
contravariant subtyping, but without the inherent complexity (in
particular, with contravariant subtyping the type of equality must be
correct in substituting equals for equals, making proofs less
perspicuous).

It is not so obvious how to expand the domain of a function in the general
case, so this approach does not work automatically in the other direction.
It does, however, work well for the important special case of sets (or,
equivalently, predicates): a set on some type \texttt{S} can be extended
naturally to one on a supertype \texttt{T} by assuming that the members of
the type-extended set are just those of the original set.  Thus, if
\texttt{extend(s)} is the type-extended version of the original set
\texttt{s}, we have \texttt{extend(s)(x) = s(x)} if \texttt{x} is in the
subtype \texttt{S}, and \texttt{extend(s)(x) = false} otherwise.  We can
say that \texttt{false} is the ``default'' value for the type-extended
function.  Building on this idea, we arrive at the following specification
for a general type-extension function.
\begin{session}
  extend [T: TYPE, S: TYPE FROM T, R: TYPE, d: R]: THEORY
   BEGIN
    f: VAR [S -> R]
    t: VAR T
    extend(f)(t): R = IF S_pred(t) THEN f(t) ELSE d ENDIF
   END extend
\end{session}
The function \texttt{extend(f)} has type \texttt{[T -> R]} and is
constructed from the function \texttt{f} of type \texttt{ [S -> R]} (where
\texttt{S} is a subtype of \texttt{T}) by supplying the default value
\texttt{d} whenever its argument is not in \texttt{S} (\texttt{S\_pred} is
the {\em recognizer\/} predicate for \texttt{ S}).  Because of the need to
supply the default \texttt{d}, this construction cannot be applied
automatically as a conversion.  However, as noted above, \texttt{false} is
a natural default for functions with range type \texttt{bool} (i.e., sets
and predicates), and the following theory establishes the corresponding
conversion.\index{extend\_bool conversion}\index{conversion!extend\_bool}
\begin{session}
extend_bool [T: TYPE, S: TYPE FROM T]: THEORY
 BEGIN
  CONVERSION extend[T, S, bool, false]
 END extend_bool
\end{session}
In the presence of this conversion, the type-incorrect formula
\texttt{B\_app} in the following specification
\begin{pvsex}
  b: [nat -> bool]
  B: [[int -> bool] -> bool]
  B_app: FORMULA B(b)
\end{pvsex}
is automatically transformed to \texttt{B(extend[int,nat,bool,false](b))}.

\subsection{Lambda conversions}\label{lambda-conversion}
\index{lambda conversion}\index{conversion!lambda}

Conversions are also useful (for example, in semantic encodings of dynamic
or temporal logics) in ``lifting'' operations to apply pointwise to
sequences over their argument types.  Here is an example, where
\texttt{state} is an uninterpreted (nonempty) type, and a state variable
\texttt{v} of type real is represented as a constant of type
\texttt{[state -> real]}.
\begin{session}
  th: THEORY
   BEGIN
    CONVERSION+ K_conversion
    state: TYPE+
    l: [state -> list[int]]
    x: [state -> real]
    b: [state -> bool]
    bv: VAR [state -> bool]
    s: VAR state
    box(bv): bool = FORALL s: bv(s)
    F1: FORMULA box(x > 1)
    F2: FORMULA box(b IMPLIES length(l) + 3 > x)
   END th
\end{session}
In this example, the formulas \texttt{F1} and \texttt{F2} are not type
correct as they stand, but with a \emph{lambda conversion},\index{lambda
conversion}\index{conversion!lambda} triggered by the
\texttt{K\_conversion} in the PVS prelude, these formulas are converted to
the forms
\begin{session}
  F1: FORMULA box(LAMBDA (x1: state): x(x1) > 1)
  
  F2: FORMULA
    box(LAMBDA (x3: state):
          b(x3) IMPLIES
           (LAMBDA (x2: state):
              (LAMBDA (x1: state):
                 (LAMBDA (x: state): length(l(x)))(x1) + 3)
                (x2)
             > x(x2))(x3))
\end{session}

\subsection{Conversions on Type Constructors}\label{type-conversions}
\index{conversions!type constructor}\index{type constructor conversiona}
\index{componentwise conversions}

Conversions for record, tuple, and function types may be found
componentwise, without having to create the corresponding conversion
declaration.  Here is an example.
\begin{session}
  bi: [bool -> int]
  ib: [int -> bool]
  CONVERSION+ bi, ib
  t: [int, int, int] = (true, false, 3)
  r: [# a, b: int #] = (# a := true, b := false #)
  f: [int, int -> int] = AND
\end{session}
With conversions displayed, this becomes the following.
\begin{session}
  t: [int, int, int] = (b2n(TRUE), b2n(FALSE), 3)

  r: [# a: int, b: int #] =
      (LAMBDA (x: [# a: bool, b: bool #]): (# a := bi(x`a), b := bi(x`b) #))
          ((# a := TRUE, b := FALSE #))

  f: [int, int -> int] =
      (LAMBDA (f: [[bool, bool] -> bool]):
         LAMBDA (x: [int, int]): bi(f(ib(x`1), ib(x`2))))
          (AND)
\end{session}
Note that for \texttt{f}, both a tuple conversion and a function
conversion are used.


\subsection{Conversion Processing}

In general, conversions are applied by the typechecker whenever it would
otherwise emit a type error.  In the simplest case, if an expression
\texttt{e} of type \texttt{T$_1$} occurs where an incompatible type
\texttt{T$_2$} is expected, the most recent compatible conversion
\texttt{C} is found in the context and the occurrence of \texttt{e} is
replaced by \texttt{C(e)}.  \texttt{C} is compatible if its type is
\texttt{[D -> R]}, where \texttt{D} is compatible with \texttt{T$_1$} and
\texttt{R} is compatible with \texttt{T$_2$}.

Conversions are ordered in the context; if multiple compatible conversions
are available,  the most recently declared conversion is used.  Hence, in

\begin{pvsex}
  CONVERSION c1
  \(\cdots\)
  IMPORTING th1, th2
  \(\cdots\)
  CONVERSION c2
  \(\cdots\)
  F: FORMULA 2
\end{pvsex}

For formula \texttt{F}, \texttt{c2} is the most recent conversion,
followed by the conversions in theory \texttt{th2}, those in \texttt{th1},
and finally \texttt{c1}.  Note that the relative order of the constant
declarations (e.g., \texttt{c1} and \texttt{c2} above) doesn't matter,
only the \texttt{CONVERSION} declarations.

When conversions are available on either the argument(s) or the operator
of an application, the arguments get precedence.

For an application \texttt{e(x$_1$, \ldots, x$_n$)} the possible types of
the operator \texttt{e}, and the arguments \texttt{x$_i$} are determined,
and for each operator type \texttt{[D$_1$, \ldots, D$_n$ -> R]} and
argument type \texttt{T$_i$}, if \texttt{D$_i$} is not compatible with
\texttt{T$_i$}, conversions of type \texttt{[T$_i$ -> D$_i$]} are
collected.  If such conversions are found for every argument that doesn't
have a compatible type, then those conversions are applied.  Otherwise an
operator conversion is looked for.

Note that compositions of conversion are never searched for, as this would
slow down processing too much.  If you want to use a composition, include
a conversion declaration for it.  Here is an example:
\begin{session}
  T1, T2, T3: TYPE+
  f1: [T1 -> T2]
  f2: [T2 -> T3]
  x: T1
  g: [T3 -> bool]
  CONVERSION f1, f2
  F1: FORMULA g(x)
  CONVERSION f2 o f1
  F2: FORMULA g(x)
\end{session}
In this example, \texttt{F1} leads to a type error, but when we make the
composition a conversion, the same expression in \texttt{F2} applies the
conversion rather than give a type error.

\subsection{Conversion Control}

As stated above, conversions are only applied when typechecking otherwise
fails.  In some cases, a conversion can allow a specification to
typecheck, but the meaning is different than what was intended.  This is
most likely for the \texttt{K\_conversion}, which was introduced when the
\texttt{mucalculus} theory was added to the prelude in support of the
model checker.  When a conversion is applied that fact is noted as a
message, and may be viewed using the \texttt{show-theory-messages}
command.  However, these messages are easily overlooked, so instead PVS
allows finer control over conversions.

Thus in addition to the \texttt{CONVERSION} form, the \texttt{CONVERSION-}
form is available allowing conversions to be turned off.  For uniformity,
the \texttt{CONVERSION+} form is also available as an alias for
\texttt{CONVERSION}.  \texttt{CONVERSION-} disables conversions.

The following theory illustrates the idea:
\begin{session}
  t1: THEORY
  BEGIN
   c: [int -> bool]
   CONVERSION+ c
   f1: FORMULA 3
   CONVERSION- c
   f2: FORMULA 3
  END t1
\end{session}
Here \texttt{f2} leads to a type error.

Another example is provided by the definition of the CTL temporal
operators in the prelude theory \texttt{ctlops}, which are surrounded by
\texttt{CONVERSION+} and \texttt{CONVERSION-} declarations that first
enable the \texttt{K\_conversion} then disable it at the end of the
theory.  All other conversions declared in the prelude remain enabled.
They may be disabled within any theory by using the \texttt{CONVERSION-}
form.

When theories containing conversion declarations are imported, the
conversions are imported as well.  Thus if \texttt{t2} enables the
\texttt{c} declaration without subsequently disabling it, then
\texttt{IMPORTING t1, t2} would enable the conversion, but
\texttt{IMPORTING t2, t1} would leave it disabled.

Conversion declarations may be generic or instantiated.  This
allows, for example, enabling the generic form of a conversion while
disabling particular instances.

\index{conversions|)}


\section{Library Declarations}
\label{library-decls}
\index{library declaration|(}
\index{declaration!library|(}

Library declarations are used to introduce a new PVS context\index{PVS Context} into a
specification.  Thus a specification may be developed in one context, and
used in many other contexts.  This provides more flexibility, at the cost
of less portability.  Any PVS context other than the current one may be
considered a library.  An example of a library declaration
is\index{LIBRARY@{\tt LIBRARY}}
\begin{pvsex}
  lib: LIBRARY = "~/pvs/protocols"
\end{pvsex}
When encountered, the system verifies that the directory specified within
the quotation marks exists, and that it has a PVS context file
\index{.pvscontext@{\tt .pvscontext}}%
(\texttt{.pvscontext}).  The library declaration is made use of by
including the library id in an importing name:\index{IMPORTING@{\tt IMPORTING}}
\begin{pvsex}
  IMPORTING lib@sliding_window[n]
\end{pvsex}
This has the effect of bringing in the \texttt{sliding\_window} theory,
exactly as if the theory belonged to the current context.

There are several libraries distributed with PVS, in the directory {\tt lib}.
It is not necessary to give a library declaration for libraries in this
directory, as it will be automatically searched for library importings.
For example, to import the finite sets library over the natural numbers:
\begin{pvsex}
  IMPORTING finite\_sets@finite\_sets[nat]
\end{pvsex}
An alternative approach (described in the \emph{PVS User
Guide}\cite{PVS:userguide}) is to use the {\tt M-x load-prelude-library},
which augments the PVS prelude with the the theories from a given context.

\index{declaration!library|)}
\index{library declaration|)}



\section{Auto-rewrite Declarations}
\label{auto-rewrite-decls}\index{auto-rewrites|(}

One of the problems with writing useful theories or libraries is that
there is no easy way to convey how the theory is to be used, other than in
comments or documentation.  In particular, the specifier of a theory
usually knows which lemmas should always be used as rewrites, and which
should never appear as rewrites.  Auto-rewrite declarations allow for both
forms of control.  Those that should always be used as rewrites are
declared with the \texttt{AUTO\_REWRITE+} or \texttt{AUTO\_REWRITE}
keyword, and those that should not are declared with
\texttt{AUTO\_REWRITE-}.  These will be referred to as
\emph{auto-rewrites} and \emph{stop-auto-rewrites} below.

When a proof is initiated for a given formula, all of the auto-rewrite
names in the current context that haven't subsequently been removed by
stop-auto-rewrite declarations are collected and added to the initial proof
state.  The stop-auto-rewrite declaration, in addition to removing
auto-rewrite names, also affects the following commands described in the
Prover manual.
\begin{itemize}
\setlength{\itemsep}{-5pt}
\item \texttt{auto-rewrite-theory},
\item \texttt{auto-rewrite-theories},
\item \texttt{auto-rewrite-theory-with-importings},
\item \texttt{simplify-with-rewrites},
\item \texttt{auto\-rewrite-defs},
\item \texttt{install-rewrites},
\item \texttt{auto-rewrite-explicit},
\item \texttt{grind},
\item \texttt{induct\-and-simplify},
\item \texttt{measure-induct-and-simplify}, and
\item \texttt{model-check}
\end{itemize}
These commands collect all definitions and formulas except those that
appear in \texttt{AUTO\_\-REWRITE-} declarations.  Thus suppose a theory
\texttt{T} contains the lemmas \texttt{lem1}, \texttt{lem2}, and
\texttt{lem3} and the declarations
\begin{alltt}
  AUTO_REWRITE+ lem1
  AUTO_REWRITE- lem3
\end{alltt}
Then in proving a formula of a theory that imports \texttt{T},
\texttt{lem1} is initially an auto-rewrite, and the command
\texttt{(auto-rewrite-theory "T")} will additionally install
\texttt{lem2}.  To auto-rewrite with \texttt{lem3}, simply use
\texttt{(auto-rewrite "lem3")}.  To exclude \texttt{lem1}, use
\texttt{(stop-auto-rewrite "lem1")} or \texttt{(auto-rewrite-theory "T"
:exclude "lem1")}.

The \texttt{autorewrites} theory shows a simple example.
\begin{session}
autorewrites: THEORY
BEGIN
 AUTO_REWRITE+ zero_times3
 a, b: real
 f1: FORMULA a * b = 0 AND a /= 0 IMPLIES b = 0
 AUTO_REWRITE- zero_times3
 f2: FORMULA a * b = 0 AND a /= 0 IMPLIES b = 0
END autorewrites
\end{session}
Here \texttt{f1} may be proved using only \texttt{assert}, but \texttt{f2}
requires more.

Rewrite names may have suffixes, for example, \texttt{foo!} or
\texttt{foo!!}.  Without the suffix, the rewrite is \emph{lazy}, meaning
that the rewrite will only take place if conditions and TCCs simplify to
true.  A condition in this case is a top-level \texttt{IF} or
\texttt{CASES} expression.  With a single exclamation point the
auto-rewrite is \emph{eager}, in which case the conditions are irrelevant,
though if it is a function definition it must have all arguments supplied.
With two exclamation points it is a \emph{macro} rewrite, and terms will
be rewritten even if not all arguments are provided.  See the prover guide
for more details; the notation is derived from the prover commands
\texttt{auto-rewrite}, \texttt{auto-rewrite!}, and
\texttt{auto-rewrite!!}.

In addition, a rewrite name may be disambiguated by stating that it is a
formula, or giving its type if it is a constant.  Without this any
definition or lemma in the context with the same name will be installed as
an auto-rewrite.

In order to be more uniform, these new forms of name are also available
for the \texttt{auto-rewrite} prover commands.  Thus the command
\begin{alltt}
  (auto-rewrite "A" ("B" "-2") "C" (("1" "D")))
\end{alltt}
may now be given instead as
\begin{alltt}
  (auto-rewrite "A" "B!" "-2!" "C" "1!!" "D!!")
\end{alltt}
The older form is still allowed, but is deprecated, and may not be mixed
with the new form.  Notice that in the auto-rewrite commands formula
numbers may also be used, and these may be followed by exclamation points,
but not by a formula keyword or type.

\index{auto-rewrites|)}


\index{declaration|)}

%%% Local Variables: 
%%% mode: latex
%%% TeX-master: "language"
%%% End: 
