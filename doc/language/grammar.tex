\chapter{The Grammar}\label{grammar}

% Redefine \htarget so that only the earlier hypertargets are used.
\def\htarget #1{#1}

The complete \pvs\ grammar is presented in this Appendix, along with a
discussion of the notation used in presenting the grammar.

The conventions used in the presentation of the syntax are as follows.
\index{syntax!conventions}

\begin{itemize}

\item Names in {\it italics\/} indicate syntactic classes and
metavariables ranging over syntactic classes.

\item The reserved words of the language are
      printed in \lit{tt font, UPPERCASE}.

\item An optional part {\it A\/} of a clause is enclosed in square brackets:
\opt{{\it A\/}}.

\item Alternatives in a syntax production are separated by a bar
(``\choice''); a list of alternatives that is embedded in the right-hand
side of a syntax production is enclosed in brackets, as in

\begin{bnf}
\production{ExportingName}
{IdOp \opt{\lit{:} \brc{TypeExpr \choice \lit{TYPE} \choice \lit{FORMULA}}}}
\end{bnf}


\item Iteration of a clause {\it B\/} one or more times is indicated by
enclosing it in brackets followed by a plus sign: \ite{{\it B\/}};
repetition zero or more times is indicated by an asterisk instead of the
plus sign: \rep{{\it B\/}}.

\item A double plus or double asterisk indicates a clause separator; for
example, \reps{{\it B\/}}{,} indicates zero or more repetitions of the
clause {\it B} separated by commas.

\item Other items printed in tt font on the right hand side of
      productions are literals.  Be careful to distinguish where BNF
symbols occur as literals, \eg\ the BNF brackets \brc{} versus the
literal brackets \lit{\setb\sete}.

\end{itemize}

\subsubsection*{Specification}
\par\noindent
\spvsbnf{bnf-theory}

\subsubsection*{RecursiveTypes}
\par\noindent
\spvsbnf{bnf-adts}

\subsubsection*{Assumings}
\par\noindent
\spvsbnf{bnf-assuming}

\subsubsection*{Theory Part}
\par\noindent
\spvsbnf{bnf-theory-part}

\subsubsection*{Importings and Exportings}
\par\noindent
\spvsbnf{bnf-exporting}

\subsubsection*{Declarations}
\par\noindent
\spvsbnf{bnf-decls}
\par\noindent
\spvsbnf{bnf-decls-aux}

\subsubsection*{Type Expressions}
\par\noindent
\spvsbnf{bnf-type-expr}

\subsubsection*{Expressions}
\par\noindent
\spvsbnf{bnf-expr}

\subsubsection*{Expressions (continued)}
\par\noindent
\spvsbnf{bnf-expr-aux}

\subsubsection*{Names}
\par\noindent
\spvsbnf{bnf-names}

\subsubsection*{Identifiers}
\par\noindent
\spvsbnf{bnf-lexical}

%%% Local Variables: 
%%% mode: latex
%%% TeX-master: "language"
%%% End: 
