% Master File: language.tex
% Document Type: LaTeX

\chapter{Name Resolution}\label{names}\label{resolution}

Names in PVS are used to denote theories, variables, constants, and
formulas.  New names are introduced by declarations.  The syntax of names
is given in Figure~\ref{bnf-names}.

\pvsbnf{bnf-names}{Name Syntax}

The simplest form of a name is an \emph{idop}, \ie\ an identifier or
operator symbol.  This is generally all that is needed, unless names are
overloaded.

The overloading of names, both from different theories and within a single
theory, is allowed as long as there is some way for the system to
distinguish references to them.  Names from different theories may be
distinguished by prefixing them with the theory name.  Within a theory,
all names of the same kind must be unique, except for expression kinds;
which need only be unique up to the signature.  This is because the
signature is enough to distinguish these declarations.  For example, if
\texttt{<} is declared to have signature \texttt{[bool,int -> bool]}, the
system will recognize from the context that \texttt{TRUE < 3} contains a
reference to this declaration, whereas \texttt{2 < 3} does
not.\footnote{Of course, this assumes that \texttt{TRUE} has not itself
been overloaded.}  If the use of the name is not enough to distinguish,
coercion may be used to specify the signature directly (see
page~\pageref{coercions}).  Theory parameters must be unique across all
kinds.

There are three possible forms for names (two for theory names, which
appear in \texttt{IMPORTING}s, \texttt{EXPORTING} \texttt{WITH}s, and
theory declarations).  Given a theory named \emph{theoryid}, with formal
parameters $f_1,\ldots,f_n$, that contains a declaration named \emph{id},
the following three forms may be used to reference the declaration in a
theory that imports \emph{theoryid}:
\begin{itemize}
\item \emph{theoryid}\texttt{[$a_1,\ldots,a_n$]}.\emph{id}

\item \emph{id}\texttt{[$a_1,\ldots,a_n$]}

\item \emph{id}
\end{itemize}
where the $a_i$ are expressions or type expressions that are compatible
with the formal parameters as described in Section~\ref{parameters}.  Note
that any of these forms may have \emph{mappings} immediately after the
actual parameters.  As described in Section~\ref{mappings}, these can be
viewed as an extension of the actuals.  Note also that theory names allow
different kinds of mappings.  The forms above are listed in order of
increasing likelihood of ambiguity---that is, names that are given with
just an \emph{id} are far more likely to produce an ambiguity than those
further up.  Note that even the top form may be ambiguous, as \emph{id}
may be declared more than once in \emph{theoryid}.  If this is the case,
then either the context will disambiguate the name or a type will have to
be supplied in the form of a coercion expression, \eg\
\texttt{\emph{id}::~nat}.  This kind of ambiguity is allowed only for
constants (including functions and recursive functions) and variables.

Names are resolved based on the expected type and the number and types of
arguments to which the name is applied.  The expected type is generally
determined from the context of the name, for example in
\begin{pvsex}
  c1: int = c2
\end{pvsex}
\texttt{c2} has expected type \texttt{int}.  For most expressions, this is
straight-forward, but applications create special problems.  For example,
in
\begin{pvsex}
  f: FORMULA c1 = c2
\end{pvsex}
we know that the equality (which \emph{is} an application) has range type
\texttt{boolean} since it is a formula, but this gives no information
about the types of the arguments.  We will first describe the simpler
situation, and then explain how names used as operators of an application
are resolved.

In general, the typechecker works by first collecting possible types for
the expressions, and then chooses from among the possible types using the
expected type, which is determined from the context of the expression.
The expected type is used to resolve ambiguities, but otherwise does not
contribute to the type of an expression.  Thus if \texttt{2 + 3}
typechecks, and \texttt{+} has not been redeclared, then it has type
\texttt{number\_field} regardless of its context.  However, for the purpose
of checking for TCCs, it may be treated as having a different type
depending on the expected type and the available judgements.
