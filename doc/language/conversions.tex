% Document Type: LaTeX
% Master File: language.tex
\section{Conversions}
\label{coercion-decls}\index{conversions|(}

Conversions are functions that the typechecker can insert automatically
whenever there is a type mismatch.  They are similar to the implicit
coercions for converting integers to floating point used in many
programming languages.  PVS provides some builtin conversions in the
prelude, but conversions may also be provided by the user using
\emph{conversion declarations}.  A conversion declaration consists of the
keyword \texttt{CONVERSION}, optionally followed by `\texttt{+}' or
`\texttt{-}' and an expression.  \texttt{CONVERSION+} is equivalent to
\texttt{CONVERSION}.  The expression must be of type a (subtype of) a
function type, where the domain and range are not compatible.  This is
because conversions are only triggered when there would otherwise be a
type error, and compatible types may lead to unproveable TCCs, but not to
type errors.  Judgements are the proper way to control the generation of
TCCs, see Section~\ref{judgements} for details.

\subsection{Conversion Examples}
\label{conversion-examples}

Here is a simple example.
\begin{pvsex}
  c: [int -> bool]
  CONVERSION c
  two: FORMULA 2
\end{pvsex}
Here, since formulas must be of type boolean, the typechecker
automatically invokes the conversion and changes the formula to
\texttt{c(2)}.  This is done internally, and is only visible to the user
on explicit command\footnote{The \texttt{M-x~prettyprint-expanded} command.}
and in the proof checker.

A more complex conversion is illustrated in the following example.
\begin{pvsex}
  g: [int -> int]
  F: [[nat -> int] -> bool]
  F_app: FORMULA F(g)
\end{pvsex}
As this stands, \texttt{F\_app} is not type-correct, because a function of
type \texttt{[int -> int]} is supplied where one of type \texttt{[nat ->
int]} is required, and PVS requires equality on domain types for function
types to be compatible.  However it is clear that \texttt{g} naturally
induces a function from \texttt{nat} to \texttt{int} by simply restricting
its domain.  Such a domain restriction is achieved by the
\texttt{restrict} conversion\index{restrict
conversion}\index{conversion!restrict} that is defined in the PVS prelude
as follows:
\begin{session}
  restrict [T: TYPE, S: TYPE FROM T, R: TYPE]: THEORY
   BEGIN
    f: VAR [T -> R]
    s: VAR S
    restrict(f)(s): R = f(s)
    CONVERSION restrict
   END restrict
\end{session}
The construction \texttt{S: TYPE FROM T} specifies that the actual
parameter supplied for \texttt{S} must be a subtype of the one supplied
for \texttt{T}.  The specification states that \texttt{restrict(f)} is a
function from \texttt{S} to \texttt{R} whose values agree with \texttt{f}
(which is defined on the larger domain \texttt{T}).  Using this approach,
a type correct version of \texttt{F\_app} can be written as
\texttt{F(restrict[int,nat,int](g))}.  This provides the convenience of
contravariant subtyping, but without the inherent complexity (in
particular, with contravariant subtyping the type of equality must be
correct in substituting equals for equals, making proofs less
perspicuous).

It is not so obvious how to expand the domain of a function in the general
case, so this approach does not work automatically in the other direction.
It does, however, work well for the important special case of sets (or,
equivalently, predicates): a set on some type \texttt{S} can be extended
naturally to one on a supertype \texttt{T} by assuming that the members of
the type-extended set are just those of the original set.  Thus, if
\texttt{extend(s)} is the type-extended version of the original set
\texttt{s}, we have \texttt{extend(s)(x) = s(x)} if \texttt{x} is in the
subtype \texttt{S}, and \texttt{extend(s)(x) = false} otherwise.  We can
say that \texttt{false} is the ``default'' value for the type-extended
function.  Building on this idea, we arrive at the following specification
for a general type-extension function.
\begin{session}
  extend [T: TYPE, S: TYPE FROM T, R: TYPE, d: R]: THEORY
   BEGIN
    f: VAR [S -> R]
    t: VAR T
    extend(f)(t): R = IF S_pred(t) THEN f(t) ELSE d ENDIF
   END extend
\end{session}
The function \texttt{extend(f)} has type \texttt{[T -> R]} and is
constructed from the function \texttt{f} of type \texttt{ [S -> R]} (where
\texttt{S} is a subtype of \texttt{T}) by supplying the default value
\texttt{d} whenever its argument is not in \texttt{S} (\texttt{S\_pred} is
the {\em recognizer\/} predicate for \texttt{ S}).  Because of the need to
supply the default \texttt{d}, this construction cannot be applied
automatically as a conversion.  However, as noted above, \texttt{false} is
a natural default for functions with range type \texttt{bool} (i.e., sets
and predicates), and the following theory establishes the corresponding
conversion.\index{extend\_bool conversion}\index{conversion!extend\_bool}
\begin{session}
extend_bool [T: TYPE, S: TYPE FROM T]: THEORY
 BEGIN
  CONVERSION extend[T, S, bool, false]
 END extend_bool
\end{session}
In the presence of this conversion, the type-incorrect formula
\texttt{B\_app} in the following specification
\begin{pvsex}
  b: [nat -> bool]
  B: [[int -> bool] -> bool]
  B_app: FORMULA B(b)
\end{pvsex}
is automatically transformed to \texttt{B(extend[int,nat,bool,false](b))}.

\subsection{Lambda conversions}\label{lambda-conversion}
\index{lambda conversion}\index{conversion!lambda}

Conversions are also useful (for example, in semantic encodings of dynamic
or temporal logics) in ``lifting'' operations to apply pointwise to
sequences over their argument types.  Here is an example, where
\texttt{state} is an uninterpreted (nonempty) type, and a state variable
\texttt{v} of type real is represented as a constant of type
\texttt{[state -> real]}.
\begin{session}
  th: THEORY
   BEGIN
    CONVERSION+ K_conversion
    state: TYPE+
    l: [state -> list[int]]
    x: [state -> real]
    b: [state -> bool]
    bv: VAR [state -> bool]
    s: VAR state
    box(bv): bool = FORALL s: bv(s)
    F1: FORMULA box(x > 1)
    F2: FORMULA box(b IMPLIES length(l) + 3 > x)
   END th
\end{session}
In this example, the formulas \texttt{F1} and \texttt{F2} are not type
correct as they stand, but with a \emph{lambda conversion},\index{lambda
conversion}\index{conversion!lambda} triggered by the
\texttt{K\_conversion} in the PVS prelude, these formulas are converted to
the forms
\begin{session}
  F1: FORMULA box(LAMBDA (x1: state): x(x1) > 1)
  
  F2: FORMULA
    box(LAMBDA (x3: state):
          b(x3) IMPLIES
           (LAMBDA (x2: state):
              (LAMBDA (x1: state):
                 (LAMBDA (x: state): length(l(x)))(x1) + 3)
                (x2)
             > x(x2))(x3))
\end{session}

\subsection{Conversions on Type Constructors}\label{type-conversions}
\index{conversions!type constructor}\index{type constructor conversiona}
\index{componentwise conversions}

Conversions for record, tuple, and function types may be found
componentwise, without having to create the corresponding conversion
declaration.  Here is an example.
\begin{session}
  bi: [bool -> int]
  ib: [int -> bool]
  CONVERSION+ bi, ib
  t: [int, int, int] = (true, false, 3)
  r: [# a, b: int #] = (# a := true, b := false #)
  f: [int, int -> int] = AND
\end{session}
With conversions displayed, this becomes the following.
\begin{session}
  t: [int, int, int] = (b2n(TRUE), b2n(FALSE), 3)

  r: [# a: int, b: int #] =
      (LAMBDA (x: [# a: bool, b: bool #]): (# a := bi(x`a), b := bi(x`b) #))
          ((# a := TRUE, b := FALSE #))

  f: [int, int -> int] =
      (LAMBDA (f: [[bool, bool] -> bool]):
         LAMBDA (x: [int, int]): bi(f(ib(x`1), ib(x`2))))
          (AND)
\end{session}
Note that for \texttt{f}, both a tuple conversion and a function
conversion are used.


\subsection{Conversion Processing}

In general, conversions are applied by the typechecker whenever it would
otherwise emit a type error.  In the simplest case, if an expression
\texttt{e} of type \texttt{T$_1$} occurs where an incompatible type
\texttt{T$_2$} is expected, the most recent compatible conversion
\texttt{C} is found in the context and the occurrence of \texttt{e} is
replaced by \texttt{C(e)}.  \texttt{C} is compatible if its type is
\texttt{[D -> R]}, where \texttt{D} is compatible with \texttt{T$_1$} and
\texttt{R} is compatible with \texttt{T$_2$}.

Conversions are ordered in the context; if multiple compatible conversions
are available,  the most recently declared conversion is used.  Hence, in

\begin{pvsex}
  CONVERSION c1
  \(\cdots\)
  IMPORTING th1, th2
  \(\cdots\)
  CONVERSION c2
  \(\cdots\)
  F: FORMULA 2
\end{pvsex}

For formula \texttt{F}, \texttt{c2} is the most recent conversion,
followed by the conversions in theory \texttt{th2}, those in \texttt{th1},
and finally \texttt{c1}.  Note that the relative order of the constant
declarations (e.g., \texttt{c1} and \texttt{c2} above) doesn't matter,
only the \texttt{CONVERSION} declarations.

When conversions are available on either the argument(s) or the operator
of an application, the arguments get precedence.

For an application \texttt{e(x$_1$, \ldots, x$_n$)} the possible types of
the operator \texttt{e}, and the arguments \texttt{x$_i$} are determined,
and for each operator type \texttt{[D$_1$, \ldots, D$_n$ -> R]} and
argument type \texttt{T$_i$}, if \texttt{D$_i$} is not compatible with
\texttt{T$_i$}, conversions of type \texttt{[T$_i$ -> D$_i$]} are
collected.  If such conversions are found for every argument that doesn't
have a compatible type, then those conversions are applied.  Otherwise an
operator conversion is looked for.

Note that compositions of conversion are never searched for, as this would
slow down processing too much.  If you want to use a composition, include
a conversion declaration for it.  Here is an example:
\begin{session}
  T1, T2, T3: TYPE+
  f1: [T1 -> T2]
  f2: [T2 -> T3]
  x: T1
  g: [T3 -> bool]
  CONVERSION f1, f2
  F1: FORMULA g(x)
  CONVERSION f2 o f1
  F2: FORMULA g(x)
\end{session}
In this example, \texttt{F1} leads to a type error, but when we make the
composition a conversion, the same expression in \texttt{F2} applies the
conversion rather than give a type error.

\subsection{Conversion Control}

As stated above, conversions are only applied when typechecking otherwise
fails.  In some cases, a conversion can allow a specification to
typecheck, but the meaning is different than what was intended.  This is
most likely for the \texttt{K\_conversion}, which was introduced when the
\texttt{mucalculus} theory was added to the prelude in support of the
model checker.  When a conversion is applied that fact is noted as a
message, and may be viewed using the \texttt{show-theory-messages}
command.  However, these messages are easily overlooked, so instead PVS
allows finer control over conversions.

Thus in addition to the \texttt{CONVERSION} form, the \texttt{CONVERSION-}
form is available allowing conversions to be turned off.  For uniformity,
the \texttt{CONVERSION+} form is also available as an alias for
\texttt{CONVERSION}.  \texttt{CONVERSION-} disables conversions.

The following theory illustrates the idea:
\begin{session}
  t1: THEORY
  BEGIN
   c: [int -> bool]
   CONVERSION+ c
   f1: FORMULA 3
   CONVERSION- c
   f2: FORMULA 3
  END t1
\end{session}
Here \texttt{f2} leads to a type error.

Another example is provided by the definition of the CTL temporal
operators in the prelude theory \texttt{ctlops}, which are surrounded by
\texttt{CONVERSION+} and \texttt{CONVERSION-} declarations that first
enable the \texttt{K\_conversion} then disable it at the end of the
theory.  All other conversions declared in the prelude remain enabled.
They may be disabled within any theory by using the \texttt{CONVERSION-}
form.

When theories containing conversion declarations are imported, the
conversions are imported as well.  Thus if \texttt{t2} enables the
\texttt{c} declaration without subsequently disabling it, then
\texttt{IMPORTING t1, t2} would enable the conversion, but
\texttt{IMPORTING t2, t1} would leave it disabled.

Conversion declarations may be generic or instantiated.  This
allows, for example, enabling the generic form of a conversion while
disabling particular instances.

\index{conversions|)}
