% Master File: language.tex
\section{Inductive and Coinductive Definitions}
\label{inductive-definitions}
\index{inductive definition|(}

\emph{Inductive} definitions~\cite{Aczel:Handbook} are used frequently in
mathematics.  In general, some rules are given that generate elements of a
set, and the inductively defined set is the smallest set that contains
those elements.  The obvious example of an inductive definition is the
natural numbers, where the rules are given by Peano's axioms, with the
induction scheme ensuring that the natural numbers are the smallest set
containing $1$ and the successor of any natural number.  Language
definitions are another example.  Most logics have a notion of
\emph{formulas}, and these are usually defined inductively.

Paulson~\cite{paulson-fixedpoint} notes that this is simply a \emph{least
fixedpoint} with respect to a given domain of elements and a set of rules,
which is well-defined if the rules are \emph{monotonic}, by the
well known Knaster-Tarski theorem.  From this perspective, the greatest
fixedpoint also exists and corresponds to \emph{coinductive} definitions.
Inductive and coinductive definitions are similar to recursive
definitions, in that they have induction principles, and both must satisfy
additional constraints to guarantee that they are well defined.

We will describe inductive definitions first, as they are more familiar.
The even integers provide a simple example of an inductive
definition:\footnote{This is an alternative to the more traditional
definition of \texttt{even?} in the prelude.}
\begin{pvsex}
  even(n: int): INDUCTIVE bool = n = 0 OR even(n - 2) OR even(n + 2)
\end{pvsex}
With this definition, it is easy to prove, for example, that \texttt{0} or
\texttt{1000} are even, simply by expanding the definition enough
times.\footnote{In the latter case, \texttt{(apply (repeat (then (expand
"even") (flatten) (assert))))} is a good strategy to use, though it should
be used with care since it does not terminate on \texttt{even} applied to
anything other than an even numeral.}  More is needed, however, in proving
general facts, such as if $n$ is even, then $n+1$ is not even.  To deal
with these, we need a means of stating that an integer is even iff it is
so as a result of this definition.  In PVS, this is accomplished by the
automatic creation of two induction schemas, that may be viewed using the
\texttt{M-x~prettyprint-expanded} command:
\begin{session}
  even_weak_induction: AXIOM
    FORALL (P: [int -> boolean]):
      (FORALL (n: int): n = 0 OR P(n - 2) OR P(n + 2) IMPLIES P(n)) IMPLIES
       (FORALL (n: int): even(n) IMPLIES P(n));

  even_induction: AXIOM
    FORALL (P: [int -> boolean]):
      (FORALL (n: int):
         n = 0 OR even(n - 2) AND P(n - 2) OR even(n + 2) AND P(n + 2)
          IMPLIES P(n))
       IMPLIES (FORALL (n: int): even(n) IMPLIES P(n));
\end{session}
The weak induction axiom states that if \texttt{P} is another predicate
that satisfies the \texttt{even} form, then any \texttt{even} number
satisfies \texttt{P}.  Thus \texttt{even} is the smallest such \texttt{P}.
The second (strong) axiom allows the \texttt{even} predicate to be carried
along, which can make proofs easier.  These axioms are used by the
\texttt{rule-induct} strategy described in the Prover
Guide~\cite{PVS:prover}.

Inductive definitions are predicates, hence must be functions with
eventual range type \texttt{boolean}.  For example, in
\begin{session}
  f1(n,m:int) INDUCTIVE int = n
  f2(n,m:int)(x,y:int)(z:int): INDUCTIVE [int,int,int -> bool] =
      LAMBDA (a,b,c:int): n = m IMPLIES f2(n,m)(x,y)(z)(a,b,c)
\end{session}
\texttt{f1} is illegal, while \texttt{f2} returns a boolean value if
applied to enough arguments, hence is valid.

To be monotonic, every occurrence of the definition within the defining
body must be \emph{positive}.\index{positive occurrence} For this we need
to define the parity of an occurrence of a term in an expression $A$: If a
term occurs in $A$ with a given parity, then the occurrence retains its
parity in \texttt{$A$ AND $B$}, \texttt{$A$ OR $B$}, \texttt{$B$ IMPLIES
$A$}, \texttt{FORALL y:$A$}, \texttt{EXISTS y:$A$}, and reverses it in
\texttt{$A$ IMPLIES $B$} and \texttt{NOT $A$}.  Any other occurrence is of
unknown parity.

The parity of the inductive definition in the definition body is checked,
and if some occurrence of the definition is negative, a type error is
generated.  If some occurrence is of unknown parity, then a
\emph{monotonicity TCC}\index{TCC!monotonicity}\index{monotonicity TCC} is
generated.  For example, given the declarations
\begin{session}
  f: [nat, bool -> bool]
  G(n:nat): INDUCTIVE bool =
    n = 0 OR f(n, G(n-1))
\end{session}
the monotonicity TCC has the form
\begin{session}
  (FORALL (P1: [nat -> boolean], P2: [nat -> boolean]):
     (FORALL (x: nat): P1(x) IMPLIES P2(x))
         IMPLIES
       (FORALL (x: nat):
          x = 0 OR f(x, P1(x - 1)) IMPLIES x = 0 OR f(x, P2(x - 1))));
\end{session}

Inductive definitions act as constants for the most part, so they may be
expanded or used as rewrite rules in proofs.  However, they are not usable
as auto-rewrite rules, as there is no easy way to determine when to stop
rewriting.

To provide induction schemes in the most usable form, they are generated
as follows.  First, the variables in the definition are partitioned into
fixed\index{fixed inductive variable} and non-fixed variables.  For
example, in the transitive-reflexive closure
\begin{pvsex}
  TC(R)(x, y) : INDUCTIVE bool =
     R(x, y) OR (EXISTS z: TC(R)(x, z) AND TC(R)(z, y))
\end{pvsex}
\texttt{R} is fixed since every occurrence of \texttt{TC} has \texttt{R}
as an argument in exactly the same position, whereas \texttt{x} and
\texttt{y} are not fixed.  The induction is then over predicates $P$ that
take the non-fixed variables as arguments.  If the inductive definition is
defined for variable $V$ partitioned into fixed variables $F$, and
non-fixed variables $N$, the general form of the (weak) induction scheme
is
\begin{session}
  FORALL (\(F\), \(P\)):
   (FORALL (\(N\)):
     \emph{inductive_body}(\(N\))\([P/\emph{def}]\) IMPLIES \(P\)(\(N\)))
      IMPLIES
     (FORALL (\(N\)): \emph{def}(\(V\)) IMPLIES \(P\)(\(N\)))
\end{session}
In the case of \texttt{TC}, this becomes
\begin{session}
  TC_weak_induction: AXIOM
        (FORALL (R: relation, P: [[T, T] -> boolean]):
           (FORALL (x: T, y: T):
              R(x, y) OR (EXISTS z: (P(x, z) AND P(z, y))) IMPLIES P(x, y))
               IMPLIES (FORALL (x: T, y: T): TC(R)(x, y) IMPLIES P(x, y)));
\end{session}

\index{coinductive definitions(|}
Coinductive definitions have the same form as inductive definitions, but
are introduced with the keyword \texttt{COINDUCTIVE}, and generate the
greatest fix point, rather than the least fix point.  The monotonicity
conditions are the same, but the coinduction axioms reverse some of the
implications.  Thus the general form of the (weak) coinduction scheme is
\begin{session}
  FORALL (\(F\), \(P\)):
   (FORALL (\(N\)):
     \(P\)(\(N\)) IMPLIES \emph{coinductive_body}(\(N\))\([P/\emph{def}]\))
      IMPLIES
     (FORALL (\(N\)): \(P\)(\(N\)) IMPLIES \emph{def}(\(V\)))
\end{session}

As noted earlier, inductive and coinductive definitions are really
fixedpoint definitions.  For example, the theory in
Figure~\ref{inductive-fixpoints} shows that an
inductive definition is a least fixedpoint, a coinductive definition is a
greatest fixpoint, an inductively defined set is a subset of a
coindutively defined set, and, if the universe contains a non-wellfounded
element, then the coinductively defined set is strictly larger.  These
results all build on the definitions in  the \texttt{mucalculus} theory of
the prelude.

{\begin{figure}[htb]\begin{boxedminipage}{\textwidth}%
{\smaller\smaller\begin{alltt}
inductive_fixpoint: THEORY
 BEGIN
  N: TYPE+
  n, m: VAR N
  0: N
  S: [N -> N]
  Sax1: AXIOM 0 /= S(n)
  Sax2: AXIOM S(m) = S(n) => m = n
  % Assume a non-wellfounded element
  nwf_exists: AXIOM EXISTS n: n = S(n)

  Nind(n):     INDUCTIVE bool = n = 0 OR EXISTS m: n = S(m) & Nind(m)
  Ncoind(n): COINDUCTIVE bool = n = 0 OR EXISTS m: n = S(m) & Ncoind(m)

  % NN is the predicate transformer corresponding to the (co)inductive defs
  NN(p: pred[N])(n): bool = n = 0 OR EXISTS m: n = S(m) & p(m)

  % These use the lfp and gfp defs from the prelude mucalculus theory
  ind_lfp: FORMULA Nind = lfp(NN)
  coind_gfp: FORMULA Ncoind = gfp(NN)

  % Repeat Nind_weak_induction, which is proved from lfp_induction
  Nind_weak_induction_repeated: FORMULA
    FORALL (P: [N -> boolean]):
      (FORALL (n): (n = 0 OR (EXISTS m: n = S(m) & P(m))) IMPLIES P(n))
       IMPLIES (FORALL (n): Nind(n) IMPLIES P(n));

  % Inductive definitions are a subset of coinductive
  ind_sub_co: FORMULA Nind(n) => Ncoind(n)

  % Because there is a non-wellfounded element, we can show that
  % the coinductive set is larger.
  co_has_more: FORMULA EXISTS n: Ncoind(n) & NOT Nind(n)
 END inductive_fixpoint
\end{alltt}}\end{boxedminipage}%
\caption{Inductive definitions and fixpoints}\label{inductive-fixpoints}\end{figure}}

\index{coinductive definition|)}
\index{inductive definition|)}
