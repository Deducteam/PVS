% Document Type: LaTeX
% Master File: language.tex

\chapter{Expressions}\label{expressions}
\index{expressions|(}

The PVS language offers the usual panoply of expression constructs,
including logical and arithmetic operators, quantifiers, lambda
abstractions, function application, tuples, a polymorphic
\texttt{IF-THEN-ELSE}, and function and record overrides.  Expressions may
appear in the body of a formula or constant declaration, as the predicate
of a subtype, or as an actual parameter of a theory instance.  The syntax
for PVS expressions is shown in Figures~\ref{bnf-expr} and~\ref{bnf-expr-aux}.

\pvsbnf{bnf-expr}{Expression syntax}

\pvsbnf{bnf-expr-aux}{Expression syntax (continued)}

\index{precedence|(} The language has a number of predefined operators
(although not all of these have a predefined meaning).  These are given in
Figure~\ref{precedenceops} below, along with their relative precedence
from lowest to highest.  Most of these operators are described in the
following sections.  \texttt{IN} is a part of \texttt{ LET} expressions,
\texttt{WITH} goes with override expressions, and the double colon
(\texttt{::}) is for coercion expressions.  The \texttt{o} operator is
defined in the prelude as the function composition operator.  Note that
most operators may be overloaded, see Chapter~\ref{lexical}
(page~\pageref{lexical}) for details.

\begin{figure}[htb]
\begin{center}{\small\tt
\begin{tabular}{|l|l|} \hline
{\rm Operators} & {\rm Associativity} \\ \hline
FORALL, EXISTS, LAMBDA, IN & None \\
\verb/|/ & Left \\
\verb/|-/, \verb/|=/ & Right \\
IFF, <=> & Right \\
IMPLIES, =>, WHEN & Right \\
OR, \verb|\/|, XOR, ORELSE & Right \\
AND, \&, \&\&, \verb|/\|, ANDTHEN & Right \\
NOT, \verb|~| & None \\
=, /=, ==, <, <=, >, >=, <<, >>, <<=, >>=, <|, |> & Left \\
WITH & Left \\
WHERE & Left \\
@, \# & Left \\
@@, \#\#, || & Left \\
+, -, ++, ~ & Left \\
*, /, **, // & Left \\
- & None \\
o & Left \\
:, ::, HAS\_TYPE & Left \\
\verb|[]|, <> & None \\
\verb|^|, \verb|^^| & Left \\
` & Left \\ \hline
\end{tabular}}
\end{center}\caption{Precedence Table}\label{precedenceops}
\end{figure}
\index{precedence|)}

\index{operator symbols|(}

Many of the operators may be overloaded by the user and retain their
precedence and form (\eg\ infix).  All of the infix operators may also be
given in prefix form; \texttt{x + 1} and \texttt{+(x,1)} are semantically equivalent.  Care must be taken in redefining these operators---if the
preceding declaration ends in an expression there could be an ambiguity.
To handle this situation the language allows declarations to be terminated
with a '\texttt{;}'.  For example,
\begin{pvsex}
  AND: [state, state -> state] = (LAMBDA a,b: (LAMBDA t: a(t) AND b(t)));
  OR: [state, state -> state] = (LAMBDA a,b: (LAMBDA t: a(t) OR b(t)));
\end{pvsex}
%
without the semicolon the second declaration would be seen as an infix
\texttt{OR} and the result would be a parse error.

Another common mistake when overloading operators with predefined meanings
is the assumption that overloading, for example, {\tt IMPLIES} automatically
provides an overloading for {\tt =>}.  This is not the case---they are distinct
operators (which happen to have the same meaning by default) and not syntactic
sugar.

\index{operator symbols|)}

\section{Boolean Expressions}\label{bool-exprs}
\index{boolean expressions}

The Boolean expressions include the constants \texttt{TRUE}\index{true@{\texttt{TRUE}}} and
\texttt{FALSE}\index{false@{\texttt{FALSE}}},
the unary operator \texttt{NOT}\index{not@{\texttt{NOT}}}, and
the binary operators \texttt{AND}\index{and@{\texttt{AND}}} (also written
\texttt{ \&}\index{\&}), \texttt{OR}\index{or@{\texttt{OR}}}, \texttt{
IMPLIES}\index{implies@{\texttt{IMPLIES}}}
(\texttt{=>}\index{=>@\texttt{=>}}),
\texttt{WHEN}\index{when@{\texttt{WHEN}}}, and
\texttt{IFF}\index{iff@{\texttt{IFF}}}
(\texttt{<=>}\index{<=>@\texttt{<=>}}).  The declarations for these are in
the \texttt{booleans} prelude theory.  All of these have their standard
meaning, except for \texttt{WHEN}, which is the converse of
\texttt{IMPLIES} (\ie\ $A$ \texttt{WHEN} $B$ $\equiv$ $B$ \texttt{IMPLIES}
$A$).

Equality\index{equality} (\texttt{=}\index{=}) and
disequality\index{disequality} (\texttt{/=}\index{/=}) are declared in the
prelude theories \texttt{equalities} and \texttt{notequal}.  They are both
polymorphic, the type depending on the types of the left- and right-hand
sides.  If the types are compatible, meaning that there is a common
supertype, then the (dis)equality is of the greatest common supertype.  Otherwise it is a type
error.  For example,
\begin{pvsex}
  S,T: TYPE
  s: VAR S
  t: VAR T
  eq1: FORMULA s = t
  i: VAR \setb{}x: int | x < 10\sete
  j: VAR \setb{}x: int | x > 100\sete
  eq2: FORMULA i = j
\end{pvsex}
%
\texttt{eq1} will cause a type error---remember that \texttt{S} and \texttt{T}
are assumed to be disjoint.  \texttt{eq2} is perfectly typesafe because
they have a common supertype \texttt{int} even though the subtypes have no
elements in common; the equality simply has the value \texttt{FALSE}.

When the equality is between terms of type \texttt{bool}, the semantics
are the same as for \texttt{IFF}.  There is a pragmatic difference in the
way the PVS prover processes these operators.  Equalities may be
used for rewriting, which makes for efficient proofs but is incomplete,
\ie\ the prover may fail to find the proof of a true formula.  On the other
hand the \texttt{IFF} form is complete, but may lead to a large number of
cases.  When in doubt, use equality as the prover provides commands
that turn an equality into an \texttt{IFF}.

%The decision to disallow \texttt{eq1} is a pragmatic one; the
%utility of such a declaration is questionable, and most likely the user
%has made an error in the specification.


\section{\texttt{IF-THEN-ELSE} Expressions}
\index{if-then-else@{\texttt{IF-THEN-ELSE}}}

The \texttt{IF-THEN-ELSE} expression \texttt{IF} {\em cond\/} \texttt{THEN} {\em
expr1\/} \texttt{ELSE} {\em expr2\/} \texttt{ENDIF} is polymorphic; its type is the
common type of {\em expr1\/} and {\em expr2\/}.  The {\em cond\/} must
be of type \texttt{boolean}.  Note that the \texttt{ELSE} part is not
optional as this is an expression, not an operational statement.  The
declaration for \texttt{IF} is in the \texttt{if\_def} prelude theory.  \texttt{
IF-THEN-ELSE} may be redeclared by the user in the same way as \texttt{
AND}, \texttt{OR}, etc.  Note that only \texttt{IF} is explicitly redeclared,
the \texttt{THEN} and \texttt{ELSE} are implicit.

Any number of \texttt{ELSIF} clauses may be present; they are translated into nested
\texttt{IF-THEN-ELSE} expressions.  Thus the expression
\begin{pvsex}
  IF A THEN B
  ELSIF C THEN D
  ELSE E
  ENDIF
\end{pvsex}
%
translates to
\begin{pvsex}
  IF A THEN B
  ELSE (IF C THEN D
        ELSE E
        ENDIF)
  ENDIF
\end{pvsex}

\section{Numeric Expressions}
\index{numeric expressions}

The numeric expressions include the \emph{numerals}\index{numerals} (0, 1,
2, \ldots), the unary operator \texttt{-}\index{-}, and the binary infix
operators \texttt{\char94}\index{\^}, \texttt{+}\index{+},
\texttt{-}\index{-}, \texttt{*}\index{*}, and \texttt{/}\index{/}.  The
numerals are all of type \texttt{real}\index{real@\texttt{real}}.
The typechecker has implicit judgements on numbers; \texttt{0} is known to
be \texttt{real}, \texttt{rat}, \texttt{int} and \texttt{nat}; all others
are known to be non zero and greater than zero.  The relational operators
on numeric types are \texttt{<}\index{<@\texttt{<}}, \texttt{
<=}\index{<=@\texttt{<=}}, \texttt{>}\index{>@\texttt{>}}, and
\texttt{>=}\index{>=@\texttt{>=}}.  The numeric operators and axioms are
all defined in the prelude.  As with the boolean operators, all of these
operators may be defined on new types and retain their original
precedences.

The numerals may also be treated as names, and
overloaded.\index{overloading numberals}\index{numerals!overloading} This
is particularly useful for defining algebraic structures such as groups
and rings, where it is natural to overload `\texttt{0}' and `\texttt{1}'.
Note that such use may include actual parameters, just as for names.  Thus
\texttt{groups[int].0} or \texttt{0[int]} might refer to the group zero
instantiated with the integer carrier set.

\section{Applications}
\index{application expressions}

Function application is specified as in ordinary mathematics; thus the
application of function \texttt{f} to expression \texttt{x} is denoted \texttt{
f(x)}.  Those operator symbols that are binary functions, and their
applications, may be written in prefix or the usual infix notation.  For
example, \texttt{(3 + 5) = (2 * 4)} may be written as \texttt{=(+(3,5),
*(2,4))}.

PVS supports higher-order types, so that functions may yield functions
as values or be curried\index{curried applications}.  For example, given
\texttt{f} of type \texttt{[int -> [int, int -> int]]}, \texttt{f(0)(2,3)}
yields an \texttt{int}.

If the application involves a dependent function type then the result
type of the application is substituted for accordingly.  For example,
\begin{pvsex}
  f: [a:int, b:\setb{}x:int | a < x\sete -> \setb{}y:int | a < y & y <= b\sete]
\end{pvsex}
the application \texttt{f(2,3)} is of type \texttt{\setb{}y:int | 2 < y \& y <=
3\sete}.  This application will also lead to the subtype \tcc\ \texttt{2 < 3}.

Application and tuple expressions have a special relation, due to the
type equivalence of \texttt{[t$_1$,\ldots,t$_n$ -> t]} and \texttt{
[[t$_1$,\ldots,t$_n$] -> t]}, see Section~\ref{tuple-exprs} for details.

\section{Binding Expressions}\label{binding-expressions}
\index{binding expressions}

The binding expressions are those which create a local scope for
variables, including the quantified expressions and
$\lambda$-expressions.  Binding expressions consist of an operator, a
list of bindings, and an expression.  The operator is one of the
keywords \texttt{FORALL}\index{forall@\texttt{FORALL}}, \texttt{
EXISTS}\index{exists@\texttt{EXISTS}}, or \texttt{LAMBDA}\index{lambda@{\texttt{LAMBDA}}}.\footnote{Set
expressions are also binding expressions; see Section~\ref{set-exprs} (page~\pageref{set-exprs}).}
The bindings specify the variables bound by the operator; each variable
has an id and may also include a type or a constraint.  Here is a
contrived example:
\begin{pvsex}
  x,y,z,d,e: VAR real
  ex1: AXIOM FORALL x,y,z: (x + y) + z = x + (y + z)
  ex2: AXIOM FORALL (x,y,z: nat): x * (y + z) = (x * y) + (x * z)
  ex3: AXIOM FORALL (n: num | n /= 0): EXISTS (x | x /= 0): x = 1/n
\end{pvsex}
%
In \texttt{ex1}, variables \texttt{x}, \texttt{y}, and \texttt{z} are all of type
\texttt{real}.  In \texttt{ex2} these same variables are of type \texttt{nat},
shadowing the global declarations.  \texttt{ex3} illustrates
the use of constraints; this is equivalent to the declaration
\begin{pvsex}
  ex3: AXIOM FORALL (n: \setb{}n: num | n /= 0\sete):
               EXISTS (x: \setb{}x | x /= 0\sete): x = 1/n
\end{pvsex}

Quantified expressions\index{quantified expressions} are introduced with
the keywords \texttt{FORALL} and \texttt{EXISTS}.  These expressions are
of type \texttt{boolean}.

Lambda expressions\index{lambda expressions} denote unnamed functions.
For example, the function which adds \texttt{3} to an integer may be
written
\begin{pvsex}
  (LAMBDA (x: int): x + 3)
\end{pvsex}
%
The type of this expression is the function type \texttt{[int ->
numfield]}.\footnote{\texttt{numfield} sits between \texttt{number} and
\texttt{real}, and is where the field operators are introduced.  See
Section~{prelude-numbers}.}  In addition, when the range is \texttt{bool},
a lambda expression may be represented as a set expression; see
Section~\ref{set-exprs}.

All of the binding expressions may involve dependent
types\index{dependent types} in the bindings, \eg
\begin{pvsex}
  FORALL (x: int), (y: \setb{}z: int | x < z\sete): p(x,y)
\end{pvsex}
%
Note that in the instantiation of such an expression during a proof will
generally lead to a subtype \tcc.  For example, substituting \texttt{e$_1$} for
\texttt{x} and \texttt{e$_2$} for \texttt{y} will lead to the \tcc\ \texttt{e$_1$ <
e$_2$}.\footnote{Such \tccs\ may never be seen, as they tend to be
proved automatically during a proof; more complicated examples may be
given, for which the prover would need help from the user.  In addition,
a false \tcc\ can show up, \eg\ substituting \texttt{2} for \texttt{x} and
\texttt{1} for \texttt{y}.  This means that the corresponding expression is
not type correct.}

Constant names may be treated as binding expressions by using a
\texttt{!}  suffix.  For example,
\begin{pvsex}
foo! (x : int) : e
\end{pvsex}
is equivalent to
\begin{pvsex}
foo( LAMBDA (x : int) : e)
\end{pvsex}

\section{\texttt{LET} and \texttt{WHERE} Expressions}
\index{let expressions@{\texttt{LET} expressions}}
\index{where expressions@{\texttt{WHERE} expressions}}

\texttt{LET} and \texttt{WHERE} expressions are provided for convenience,
making some forms easier to read.  Both of these forms provide local
bindings for variables that may then be referenced in the body of the
expression, thus reducing redundancy and allowing names to be provided for common subterms.
Here are two examples:
\begin{pvsex}
  LET x:int = 2, y:int = x * x IN x + y
  x + y WHERE x:int = 2, y:int = x * x
\end{pvsex}
%
The value of each of these expressions is 6.

\texttt{LET} and \texttt{WHERE} expressions are internally translated to
applications of lambda expressions; in this case both expressions
translate to
\begin{pvsex}
  (LAMBDA (x:int) : (LAMBDA (y:int) : x + y)(x * x))(2)
\end{pvsex}
%
These translations should be kept in mind when the semantics of these
expressions is in question.

The type declaration is optional, so the above could be written as
\begin{pvsex}
  LET x = 2, y = x * x IN x + y
  x + y WHERE x = 2, y = x * x
\end{pvsex}
In this case the typechecking of these expressions depends on whether
\texttt{x} and/or \texttt{y} have been previously declared as variables.
If they have, then those delarations are used to determine the type.
Otherwise, the right-hand side of the \texttt{=} is typechecked, and if it
is unambiguous is used to determine the type of the variable.  This is 
one way in which these expressions differ from their translation.
It is usually better to either reference a variable or give the type, as
the typechecker uses the ``natural'' type of the expression as the type of
the variable, which can lead to extra \tccs.

The \texttt{LET} expression has a limited form of pattern matching over
tuples.  An example is
\begin{pvsex}
  p: VAR [int, int]
  +(p): int = LET (m, n) = p IN m + n
\end{pvsex}
which is shorter than the equivalent
\begin{pvsex}
  p: VAR [int, int]
  +(p): int = LET m = p`1, n = p`2 IN m + n
\end{pvsex}


\section{Set Expressions}\label{set-exprs}

In PVS, sets of elements of a type \texttt{t} are represented as
predicates, \ie\ functions from \texttt{t} to \texttt{bool}.  The type of a
set may be given as \texttt{[t -> bool]}, \texttt{pred[t]}, or \texttt{
setof[t]}, which are all type equivalent.\footnote{The prelude theory
\texttt{defined\_types} also defines \texttt{PRED}, \texttt{predicate}, \texttt{
PREDICATE}, and \texttt{SETOF} as alternate equivalents.}
The choice depends wholly on the intended use of the type.
Similarly, a set may be given in the form \texttt{(LAMBDA (x:\ t):\
p(x))} or \texttt{\setb{}x:\ t | p(x)\sete}; these are equivalent
expressions.\footnote{In fact, internally they are represented by the
same abstract syntax, they simply print differently.} Note that the
latter form may also represent a type---this usually causes no
confusion as the context generally makes it clear which is expected.
The usual functions and properties of sets are provided in the prelude
theory \texttt{sets}.


\section{Tuple Expressions}\label{tuple-exprs}
\index{tuple expressions}

A tuple expression of the type \texttt{[t$_1$,\ldots,t$_n$]} has the form
\texttt{(e$_1$,\ldots,e$_n$)}.  For example, \texttt{(2, TRUE, (LAMBDA x:\ x +
1))} is of type \texttt{[nat, bool, [nat -> nat]]}.  0-tuples are not
allowed, and 1-tuples are treated simply as parenthesized expressions.
The following relation holds between function types and tuple types:
\begin{pvsex}
  [[t\(\sb{1}\),\ldots,t\(\sb{n}\)] -> t] \(\equiv\) [t\(\sb{1}\),\ldots,t\(\sb{n}\) -> t]
\end{pvsex}
%
This equivalence is most important in theory parameters; it allows one
theory to take the place of many.  For example the \texttt{functions}
theory from the prelude may be instantiated by the reference
\texttt{injective?[[int,int,int],int]}.  Applications of an element \texttt{f} of
this type include \texttt{f(1,2,3)}, \texttt{f((1,2,3))}, and \texttt{f(e)},
where \texttt{e} is of type \texttt{[int,int,int]}.

\section{Projection Expressions}\label{projection-exprs}
\index{projection expressions}

The components of an expression whose type is a tuple can be accessed
using the projection operators \texttt{`1}, \texttt{`2}, \ldots or
\texttt{PROJ\_1}, \texttt{PROJ\_2}, \ldots.  The former are preferred.
Like reserved words, projection expressions are case insensitive and may
not be redeclared.  For the most part, projection expressions are
analogous to field accessors for record types.  For example,
\begin{pvsex}
  t: [int, bool, [int -> int]]
  ft: FORMULA t`2 AND t`1 > t`3(0)
  ft_deprecated: FORMULA PROJ_2(t) AND PROJ_1(t) > (PROJ_3(t))(0)
\end{pvsex}

Projection expressions may be used without an argument as long as the
context determines the tuple type involved.  For example, in the following
it is obvious what tuple type is involved.
\begin{pvsex}
  F: [[[int, bool, [int -> int]] -> bool] -> bool]
  FP: FORMULA F(PROJ_2)
\end{pvsex}
Note that the \texttt{PROJ} keyword must be used in such cases, as, e.g.,
\texttt{`2} is not an expression.  In the following example we see that
the context does not provide enough information.
\begin{pvsex}
  PP: FORMULA PROJ_2 = PROJ_2
\end{pvsex}
To deal with such situations, the syntax for projections has been extended
to allow the tuple type to be provided.
\begin{pvsex}
  PP: FORMULA PROJ_2[[int, bool, [int -> int]]] = PROJ_2
\end{pvsex}
In this case only one of the operators needs to be annotated.  This looks
like a use of actual parameters, but it is not, as the \texttt{PROJ} is
not a name, and does not belong to a theory.


\section{Record Expressions}\label{record-expressions}
\index{record expressions}

Record expressions are of the form \texttt{(\# a$_1$ := e$_1$, \ldots,
a$_n$ := e$_n$ \#)}, which has type \texttt{[\# a$_1$:\ t$_1$, \ldots,
a$_n$:\ t$_n$ \#]}, where each \texttt{e$_i$} is of type \texttt{t$_i$}.
Partial record expressions are not allowed; all fields must be given.  If
it is desired to give a partial record, declare an uninterpreted constant
or variable of the record type, and use override expressions to specify
the given record at the fields of interest.  For example,
\begin{pvsex}
  rc: [# a, b : int #]
  re: [# a, b : int #] = rc WITH [`a := 0]
\end{pvsex}

The type of a record expression is determined by the type of its
components.  Thus \texttt{(\# a := 3, b := 2 \#)} is of type \texttt{[\# a,
b: real \#]}.  This means that a record expression is never of a dependent
record type directly, though it may be used where a dependent record is
expected, and \tccs\ may be generated as a result.  For example,
\begin{pvsex}
  R: TYPE = [# a: int, b: \setb{}x: int | x < a\sete #]
  r: R = (# a := 3, b := 4 #)
\end{pvsex}
%
leads to the (unprovable) \tcc\ \texttt{4 < 3}.

Record expressions may be introduced without introducing the record type
first, and the type of a record expression is determined by its
components, independently of any previously declared record type.  For
this reason record types do not automatically generate associated accessor
functions.

\section{Record Accessors}

The components of an expression of a record type are accessed using the
corresponding field name.  There are two forms of access.  For example if
\texttt{r} is of type \texttt{[\# x, y: real \#]}, the x-component may be
accessed using either \texttt{r`x} or \texttt{x(r)}.  The first form is
preferred as there is less chance for ambiguity.

As noted above, accessors are not stand-alone functions.  However, you can
define your own functions to provide this capability, and even use the
same name.  For example:
\begin{pvsex}
  point: TYPE = [# x, y: real #]
  x(p:point): real = p`x
  y(p:point): real = p`y
\end{pvsex}
Now \texttt{x} and \texttt{y} may be provided wherever a function is
expected.  Note that this means that a subsequent expression of the form
\texttt{x(p)} could be ambiguous, but the record field accessor is always
preferred, so in practice such ambiguities don't arise.

\section{Cotuple Expressions}\label{cotuple-expressions}
\index{cotuple expression}

Elements of cotuple types \texttt{[t$_1$ + \ldots + t$_n$]} are constructed
with the \emph{injection} operators \texttt{IN\_$i$} of type
\texttt{[t$_i$ -> [t$_1$ + \ldots + t$_n$]]}.  Thus if $e$ is of type
\texttt{t$_i$}, \texttt{IN\_$i$($e$)} is of the cotuple type.  If $x$ is
an element of a cotuple type, \texttt{IN?\_$i$($x$)} is a boolean that
tests if $x$ belongs to the $i^{th}$ component, and if it does,
\texttt{OUT\_$i$($x$)} returns the associated value of type
\texttt{t$_i$}.  Note that this is similar to a datatype of the form
\begin{pvsex}
  cotup: DATATYPE
   BEGIN
    IN_1(OUT_1: t\(\sb{1}\)): IN?_1
    \(\cdots\)
    IN_\(n\)(OUT_\(n\): t\(\sb{n}\)): IN?_\(n\)
   END cotup
\end{pvsex}
The differences are that cotuples are not recursive, do not generate all
the functions and axioms associated with datatypes, and allow for any
number of component types---using datatypes a new one would have to be
given for each arity.

The analogy works also for the \texttt{CASES} expression described in
Section~\ref{cases-expressions}.  This allows access to the values of a
cotuple element.  It has the form
\begin{pvsex}
  CASES \(e\) OF
    IN_1(x1): f\(\sb{1}\)(x1),
    \vdots
    IN_\(n\)(x\(n\)): f\(\sb{n}\)(x\(n\))
  ENDCASES
\end{pvsex}
where each \texttt{f$_i$} is an expression of type \texttt{[t$_i$ ->
$T$]}, and the common return type $T$ is the type of the \texttt{CASES}
expression.  For example, if \texttt{x} is of type \texttt{[int + bool +
[int -> int]}, the following expression will return a boolean value.
\begin{pvsex}
  CASES x OF
    IN_1(i): i > 0,
    IN_2(b): NOT b,
    IN_3(f): FORALL (n: int): f(f(n)) = f(n)
  ENDCASES
\end{pvsex}
If there are any missing components in the \texttt{CASES} expression, a
\emph{cases \tcc}\index{cases TCC}\index{TCC!cases} will be generated
stating that the cotuple expression must be one of the given selections,
unless there is an \texttt{ELSE} selection.

Like the projection operators \texttt{PROJ\_$i$}, the \texttt{IN\_$i$},
\texttt{OUT\_$i$} and \texttt{IN?\_$i$} operators make be disambiguated by
adding the cotuple type reference to the operator, for example,
\texttt{IN\_2[int + int](3)} or \texttt{IN?\_1[coT]}.  Note that although
they have the form of actual parameters, they are not, as these operators
are built in and not associated with any theory.  Also, for brevity, only
the cotuple type is given, not the full type of the operator.  There are a
number of axioms associated with cotuples that are built in to the PVS
typechecker and prover.


\section{Override Expressions}
\index{override expression}
\index{update expression}
\index{with expression}

Functions, tuples, records, and datatype elements may be ``modified'' by
means of the override expression.  The result of an override expression is
a function, tuple, record, or datatype element that is exactly the same as
the original, except that at the specified arguments it takes the new
values.  For example,
\begin{pvsex}
  identity WITH [(0) := 1, (1) := 2]
\end{pvsex}
%
is the same function as the \texttt{identity} function (defined in the
prelude) except at argument values \texttt{0} and \texttt{1}.  This is exactly
the same expression as either of
\begin{pvsex}
  (identity WITH [(0) := 1]) WITH [(1) := 2] {\rm or}
  (LAMBDA x: IF x = 1 THEN 2 ELSIF x = 0 THEN 1 ELSE identity(x))
\end{pvsex}

This order of evaluation ensures that functions remain total, and allows
for the possibility of expressions such as
\begin{pvsex}
  identity WITH [(c) := 1, (d) := 2]
\end{pvsex}
where \texttt{c} and \texttt{d} may or may not be equal.  If they are
equal, then the value of the override expression at the common argument is
\texttt{2}.

More complex overrides can be made using nested arguments; for example,
\begin{pvsex}
  R: TYPE = [# a: int, b: [int -> [int, int]] #]
  r1: R
  r2: R = r1 WITH [`a := 0, `b(1)`2 := 4]
\end{pvsex}
{\tt r2} is equivalent to
\begin{pvsex}
  (# a := 0,
     b := LAMBDA (x: int):
           IF x = 1
           THEN (r1`b(x)`1, 4)
           ELSE r2`b(x)
           ENDIF #)
\end{pvsex}

Updating a datatype element amounts to updating the accessor(s) associated
with a constructor.  For example, if \texttt{lst} is of type
\texttt{(cons?[nat])}, then \texttt{lst WITH [`car := 3]} returns a list
that is the same as \texttt{lst}, but whose first element is \texttt{3}.
If \texttt{lst} is given type \texttt{list[nat]}, then the same override
expression generates a \tcc\ obligation to prove that \texttt{lst} is a
\texttt{cons?}.  Because accessors may be both dependent and overloaded,
\tccs\ may get complicated.  For example,
\begin{pvsex}
  dt: DATATYPE
  BEGIN
   c0: c0?
   c1(x: int, a: \{z: (even?) | z > x\}, b: int): c1?
   c2(x: int, a: \{n: nat | n > x\}, c: int): c2?
  END dt
\end{pvsex}
If \texttt{d} is of type \texttt{dt}, the update expression \texttt{d WITH
[a := y]} leads to the \tcc
\begin{pvsex}
  f1_TCC1: OBLIGATION
    (c1?(d) AND even?(y) AND y > x(d)) OR
     (c2?(d) AND y >= 0 AND y > x(d));
\end{pvsex}

Another form of override expression is the maplet, indicated using
\texttt{|->} in place of \texttt{:=}.  This is used to extend the domain
of the corresponding element; for example, if \texttt{f:[nat -> int]} is
given, then \texttt{f WITH [(-1) |-> 0]} is a function of type
\texttt{[\setb{}i:int | i >= 0 OR i = -1\sete -> int]}.  This is especially useful
with dependent types, see Section~\ref{dependent-types}.  Domain extension
is also possible for record and tuple types; for example, \texttt{r1 WITH
[`c |-> 3]} is of type \texttt{[\# a:\ int, b:\ [int -> [int,int]], c:\ int
\#]}, and if \texttt{t1} is of type \texttt{[int, bool]}, then \texttt{t1
WITH [`3 |-> 1]} is of type \texttt{[int, bool, int]}.  It is an error to
extend a tuple type such that gaps are left, so \texttt{t1 WITH [`4 |->
1]} is illegal, though \texttt{t1 WITH [`3 |-> 1, `4 |-> 1]} is allowed.
Gaps would also be left if nested arguments were given, so \texttt{r1 WITH
[`c(0) |-> 0]} is also illegal.  It would have to be given as \texttt{r1
WITH [`c := LAMBDA (x:\ int):\ IF x = 0 THEN 0 ELSE $\cdots$ ENDIF]}, where
the gap $\cdots$ now has to be filled in.  Domain extension is not
possible for datatype elements, as a new datatype theory would need to be
generated for each such extension.

In the past, the two forms of assignment (using \texttt{:=} and
\texttt{|->}) were merely alternative notation, and domains would be
extended automatically whenever the typechecker could not determine that
the argument belonged to the domain.  In most cases, extending the domain
unnecessarily is harmless.  However, when terms get large, the types can
get cumbersome, slowing down the system dramatically.  Even worse, when
domains are extended and matched against a rewrite rule with the original
type, the match can fail, and the automatic rewrite will not be triggered.
For this reason, it is always best to use the maplet on function types
only when actually extending the domain.

\section{Coercion Expressions}\label{coercions}

Coercion expressions are of the form \texttt{expr ::\ type-expr}, indicating
that the expression \texttt{expr} is expected to be of type \texttt{
type-expr}.  This serves two purposes.  First, although PVS allows a
liberal amount of overloading, it cannot always disambiguate things for
itself, and coercion may be needed.  For example, in
\begin{pvsex}
  foo: int
  foo: [int -> int]
  foo: LEMMA foo = foo::int
\end{pvsex}
%
the coercion of \texttt{foo} to \texttt{int} is needed, because otherwise the
typechecker cannot determine the type.  Note that only one of the sides
of the equation needs to be disambiguated.

The second purpose of coercion is as an aid to typechecking; by
providing the expected type in key places within complex expressions,
the resulting \tccs\ may be considerably simplified.

% Master File: language.tex

\section{Tables}
\index{tables|(}

Many expressions are easier to express and to read when presented in
tabular form, as described in~\cite{Heitmeyer94:SCR-checks,Parnas92}.
There are many types of tables, ten different interpretations are
described in~\cite{Parnas92} alone.  Rather than provide support for all
these tables, we chose to support a simple form of table initially,
providing extensions in later versions of PVS as the need arises.

PVS provides a form of table expressions that allows simple
tables\footnote{In Parnas' terms~\cite{Parnas92}, these tables are
\emph{normal function tables} of one or two dimensions.} to be presented,
and supports \emph{table consistency conditions}\index{table consistency}.
One of the consistency conditions (the \emph{Mutual Exclusion
Property}\index{mutual exclusion property} or
\emph{disjointness}\index{disjointness property}) requires the pairwise
conjunction of a set of formulas to be false; another (the \emph{Coverage
Property}\index{coverage property}) requires the disjunction of a set of
formulas to be true.

Tables are supported by means of the more generic \texttt{COND}
expression, which provides the semantic foundation.  In the following
sections, we first describe the \texttt{COND} expression, and then
\texttt{TABLE} expressions.

\subsection{\texttt{COND} Expressions}
\index{cond expressions@\texttt{COND} expressions|(}

The \texttt{COND} construct is a multi-way extension to the polymorphic
\texttt{IF-THEN-ELSE} construct of PVS\@.  Its form is
\begin{pvsex}
  COND
      be\_1 -> e\_1,
      be\_2 -> e\_2,
        \ldots
      be\_n -> e\_n
  ENDCOND
\end{pvsex}
where the \texttt{be\_}i's are boolean expressions, and the \texttt{e\_}i's are
expressions of some common supertype.  It is required that the \texttt{be\_}i's are pairwise disjoint and that their disjunction is a tautology:
these constraints are generated as \emph{disjointness}\index{disjointness
TCC}\index{TCC!disjointness} and \emph{coverage}\index{coverage
TCC}\index{TCC!coverage} TCCs that must be discharged before PVS will
consider a \texttt{COND} expression fully type-correct.
\begin{pvsex}
  foo_TCC1: OBLIGATION NOT (be\_1 AND be\_2) AND\ldots{}AND NOT (be\_n-1 AND be\_n)
  foo_TCC2: OBLIGATION be\_1 OR be\_2 OR\ldots{}OR be\_n
\end{pvsex}

Notice that a \texttt{COND} expression with $n$ clauses generates $O(n^2)$
clauses in its disjointness TCC\@.  

Assuming its associated TCCs are discharged, the schematic \texttt{COND}
shown above is equivalent to the following \texttt{IF-THEN-ELSE} form,
which is its semantic definition.

\begin{pvsex}
  IF be\_1 THEN e\_1
  ELSIF be\_2 THEN e\_2
          \ldots
  ELSIF be\_n-1 THEN e\_n-1
  ELSE e\_n
\end{pvsex}

The \texttt{COND} may include an \texttt{ELSE} clause:
\begin{pvsex}
  COND
      be\_1 -> e\_1,
      be\_2 -> e\_2,
        \ldots
      ELSE -> e\_n
  ENDCOND
\end{pvsex}
This form does not require the coverage TCC and is equivalent to the
\texttt{IF-THEN-ELSE} form shown above.

Using \texttt{COND}, we can translate the following tabular
specification of the \emph{sign} function
\[
\begin{array}{|c||c|c|c|}
\hline
 & x<0 & x=0 & x>0\\
\hline
\emph{sign}(x) & -1 & 0 & 1\\
\hline
\end{array}
\]
into
\begin{pvsex}
  sign(x): int = COND
                    x<0 -> -1, 
                    x=0 -> 0,
                    x>0 -> 1
                 ENDCOND
\end{pvsex}

Two dimensional tables can be generated by nested \texttt{COND}s.  For
example, the following table defining the value for \texttt{safety\_injection}
\begin{center}
\begin{tabular}{|c||c|c|}
\hline
modes & \multicolumn{2}{c|}{conditions} \\
\hline
\hline
normal & false & true\\
\hline
low & not overridden & overridden \\
\hline
voter\_failure & true & false\\
\hline
\hline
safety\_injection & on & off\\
\hline
\end{tabular}
\end{center}
can be represented as
\begin{pvsex}
  safety\_injection(mode, overridden): on\_off = 
    COND
      mode=normal -> off,
      mode=low -> (COND NOT overridden -> on, overridden -> off ENDCOND),
      mode=voter\_failure -> on
    ENDCOND
\end{pvsex}

Notice that \texttt{mode=low} provides the ``left context'' used in
generating the TCCs for the nested \texttt{COND}.  This causes some
redundancy in highly structured two dimensional tables as the
following example shows.
\begin{center}
\begin{tabular}{|c||c|c|}
\hline
 & \multicolumn{2}{c|}{input}\\
\cline{2-3}
state & x & y\\
\hline
\hline
a & a & b\\
\hline
b & b & b\\
\hline
\end{tabular}
\end{center}
This translates to
\begin{pvsex}
  COND
    state=a -> COND input=x -> a,input=y -> b ENDCOND,
    state=b -> COND input=x -> b,input=y -> b ENDCOND
  ENDCOND
\end{pvsex}
The coverage TCCs generated for the two inner \texttt{COND}s will have the form
\begin{pvsex}
   foo\_TCC2 : OBLIGATION state=a IMPLIES input=x OR input=y
   foo\_TCC3 : OBLIGATION state=b IMPLIES input=x OR input=y
\end{pvsex}
whereas, because of the disjointness and coverage of $\{$\texttt{a}, \texttt{b}$\}$, the correct TCC is the simpler form
\begin{pvsex}
   foo\_TCC: OBLIGATION input=x OR input=y
\end{pvsex}
The source of the error here is that our translation of the original
table is too simple-minded.  A better translation is the following.
\begin{pvsex}
  LET
    x1 = COND input=x -> a, input=y -> b ENDCOND,
    x2 = COND input=x -> b, input=y -> b ENDCOND
  IN
    COND state=a -> x1, state=b -> x2 ENDCOND
\end{pvsex}
And this generates the correct TCCs.

Note that if the \texttt{be\_i}'s are members of an enumerated type, then
the standard PVS \texttt{CASES} construct should be used instead of \texttt{COND}, since there is no need to generate TCCs in these cases.  For
example, if in the previous example $\{$ \texttt{a}, \texttt{b} $\}$ and
$\{$ \texttt{x}, \texttt{y} $\}$ had been enumerated types, then the table
could have been expressed as
\begin{pvsex}
  CASES state OF
    a: CASES input OF x: a, y: b ENDCASES,
    b: CASES input OF x: b, y: b ENDCASES
  ENDCASES
\end{pvsex}
and no TCCs would be generated.

If the \texttt{be\_i}'s are all equalities with the same left hand side,
whose right hand sides are ground arithmetic terms (involving only
numbers, \texttt{+}, \texttt{-}, \texttt{*}, \texttt{/}) then the typechecker
directly checks for coverage and disjointness so no \tccs\ are generated
in this case.

\index{cond expressions@\texttt{COND} expressions|)}


\subsection{Table Expressions}

The \texttt{COND} and \texttt{CASES} constructs (see datatypes on page~\pageref{datatypes}) provide the semantic
foundation for our treatment of tables in PVS; for convenience, we
also provide a \texttt{TABLE} construct that provides more attractive
syntax for the important special cases of regular one and
two-dimensional tables.  The example above can be written in the
alternative form.
\begin{pvsex}
   TABLE
%               ---------------------
               |[ input=x | input=y ]|
%     -------------------------------
      | state=a |    a    |    b    ||
%     -------------------------------
      | state=b |    b    |    b    ||
%     -------------------------------
   ENDTABLE
\end{pvsex}
This will translate internally into the \texttt{LET} and \texttt{COND} form
shown earlier.  Note that the horizontal lines are simply PVS
comments.\footnote{The \LaTeX\ generation translates these constructs into attractively
typeset tables.  See the PVS System Guide~\cite{PVS:userguide} for details.}

The row and column headers to a \texttt{TABLE} construct are arbitrary
boolean expressions.   In cases where the expressions are all of the
form \texttt{id=x}, the \texttt{id} can be factored out to produce simpler
tables of the following form.
\begin{pvsex}
   TABLE state,    input
%               ----------
               |[ x | y ]|
%        -----------------
         |  a   | a | b ||
%        -----------------
         |  b   | b | b ||
%        -----------------
   ENDTABLE
\end{pvsex}
In this form, as the headings are enumeration constructs this is
internally represented as a \texttt{CASES} construct, and so generates
no \tccs\ (the previous version generates 5 \tccs).

One-dimensional tables can be presented in both ``horizontal'' and
``vertical'' forms.  The \emph{sign} function example can be
presented as a ``vertical'' table as follows.
\begin{pvsex}
  sign(x): int = TABLE 
%                ------------
                |[ x<0 | -1 ]|
%                ------------
                 | x=0 |  0 ||
%                ------------
                 | x>0 |  1 ||
%                ------------
   ENDTABLE
\end{pvsex}

And as a horizontal one as follows.
\begin{pvsex}
  sign(x): int = TABLE 
%                --------------------
                 |[ x<0 | x=0 | x>0 ]|
%                -------------------
                 |   -1 |  0  |  1  ||
%                --------------------
   ENDTABLE
\end{pvsex}


A more complex two-dimensional example is provided by the mode
transition tables used in SCR\@.  These have the following form.
\[
\begin{array}{|c|c|c|}
\hline
\mbox{current mode} & \mbox{Event} & \mbox{New Mode} \\
\hline
m_{1} & e_{1,1} & m_{1,1} \\
    & e_{1,2} & m_{1,2} \\
    & \ldots & \ldots\\
    & e_{1,k_1} & m_{1,k_1} \\
\hline
m_{2} & e_{2,1} & m_{2,1} \\
    & e_{2,2} & m_{2,2} \\
    & \ldots & \ldots\\
    & e_{2,k_2} & m_{2,k_2} \\
\hline
\ldots  & \ldots & \ldots\\  
\hline
m_{p} & e_{p,1} & m_{p,1} \\
    & e_{p,2} & m_{p,2} \\
    & \ldots & \ldots\\
    & e_{p,k_p} & m_{p,k_p} \\
\hline
\end{array}
\]
And translate to the following form.
\smaller
\begin{center}
\begin{pvsex}
TABLE mode
%--------------------------------
     |  m\_1 | TABLE event
                  | e\_1,1 | m\_1,1 ||
                  | e\_1,2 | m\_1,2 ||
           \ldots
                  | e\_1,k1| m\_1,k1||
             ENDTABLE ||
%--------------------------------
     |  m\_2 | TABLE event
                  | e\_2,1 | m\_2,1 ||
                  | e\_2,2 | m\_2,2 ||
           \ldots
                  | e\_2,k2| m\_2,k2||
             ENDTABLE ||
%--------------------------------
        \ldots
%--------------------------------
     |  m\_p | TABLE event
                  | e\_p,1 | m\_p,1 ||
                  | e\_p,2 | m\_p,2 ||
           \ldots
                  | e\_p,kp| m\_p,kp||
             ENDTABLE ||
%--------------------------------
ENDTABLE
\end{pvsex}
\end{center}

The last row or column heading in a table may contain the \texttt{ELSE}
keyword, which has the same meaning as for the corresponding \texttt{COND}
or \texttt{CASES} expression.

The table may also have blank entries (except in the headings).  These
represent illegal values; in other words the entry may never be reached.
This is represented by generation of a TCC indicating that the
formulas corresponding to the row and column headings for that entry
cannot both be true.

Note that this is different than having ``don't care'' values.  If you
want to add don't care entries, make sure that you use an array; the table
\begin{pvsex}
DC: int
TABLE
         |[ x < 0 | x = 0 | x > 0 ]|
  | y < 0 |   1   |   0   |   DC  ||
  | y = 0 |  DC   |   2   |    3  ||
  | y > 0 |   -2  |  DC   |    0  ||
  ENDTABLE
\end{pvsex}
may seem like any integer may appear in place of \texttt{DC}, but it must
always be the same integer, which is probably not intended.  The right way
to do this is
\begin{pvsex}
DC(n:nat): int
TABLE
         |[ x < 0 | x = 0 | x > 0 ]|
  | y < 0 |   1   |   0   | DC(2) ||
  | y = 0 | DC(0) |   2   |    3  ||
  | y > 0 |   -2  | DC(1) |    0  ||
  ENDTABLE
\end{pvsex}

\index{tables|)}


\index{expression|)}
